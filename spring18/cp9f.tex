\documentclass[handout]{mcs}

\begin{document}

\inclassproblems{9, Fri.}

\begin{staffnotes}
Simple Graphs: Coloring, Ch. 12.6
\end{staffnotes}

%\pinput{CP_chromatic_number}

%\pinput{CP_latin_squares}

\pinput{CP_register_allocation}

\pinput{PS_coloring_no_triangles}         %interesting

\begin{staffnotes}
If it is hard for students to come up with a proof for why it is not 3 colorable, ask them why it is not two colorable so that they can think about the odd length outer cycle. From there they can get to why 3 colors is not enough.
\end{staffnotes}

\pinput{CP_3color_OR_gate}

%\examspace
\instatements{\newpage}
\pinput{FP_bogus_coloring_proof}

\begin{staffnotes}
Ask the students why the second statement does not work.
\end{staffnotes}

%\pinput{PS_graph_colorable}
%\pinput{CP_bipartite_coloring}  %save until after counting


%%%%%%%%%%%%%%%%%%%%%%%%%%%%%%%%%%%%%%%%%%%%%%%%%%%%%%%%%%%%%%%%%%%%%
% Problems end here
%%%%%%%%%%%%%%%%%%%%%%%%%%%%%%%%%%%%%%%%%%%%%%%%%%%%%%%%%%%%%%%%%%%%%


\end{document}
