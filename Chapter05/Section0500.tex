Graphs arise in different fields with all sorts of applications,
including scheduling, optimization, communications, and the design and
analysis of algorithms.  Two Stanford students even used graph theory to
become multibillionaires!

But we'll start with an application designed to get your attention: we are
going to make a professional inquiry into sexual behavior.  Namely, we'll
look at some data about who, on average, who has more opposite-gender
partners, men or women?

Sexual demographics have been the subject of many studies.  In one of the
largest, researchers from the University of Chicago interviewed a random
sample of 2500 people over several years to try to get an answer to this
question.  Their study, published in 1994, and entitled \emph{The Social
  Organization of Sexuality} found that on average men have 74\% more
opposite-gender partners than women.

Other studies have found that the disparity is even larger.  In
particular, ABC News claimed that the average man has 20 partners over his
lifetime, and the average woman has 6, for a percentage disparity of
233\%.  The ABC News study, aired on Primetime Live in 2004, purported to
be one of the most scientific ever done, with only a 2.5\% margin of
error.  It was called "American Sex Survey: A peak between the sheets,"
---which makes its science start to sound a little doubtful.  \iffalse The
promotion for the study is even better:
\begin{quote} 
A ground breaking ABC News ``Primetime Live'' survey finds a range of
eye-popping sexual activities, fantasies and attitudes in this country,
confirming some conventional wisdom, exploding some myths -- and venturing
where few scientific surveys have gone before.
\end{quote}
Probably that last part about going where few scientific surveys have gone
before is pretty accurate!
\fi
Yet again, in August, 2007, the N.Y. Times
\href{http://www.nytimes.com/2007/08/12/weekinreview/12kolata.html?_r=1&n=Top/Reference/Times%20Topics/People/K/Kolata,%20Gina&oref=slogin}{reported} on a study by the
  National Center for Health Statistics of the U.S. government showing
  that men had seven partners while women had four.

  Anyway, whose numbers do you think are more accurate, the University of
  Chicago, ABC News, or the National Center?  Don't answer: this is a setup
  question like ``When did you stop beating your wife?''  Using a little
  graph theory, we'll explain why none of these findings can be anywhere
  near the truth.
  
\endinput