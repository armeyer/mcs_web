\documentclass[12pt,twoside]{article}
\usepackage{../light}


\begin{document}
\lecture{Graph Theory III}{September 30, 2008}


\section{Build-Up Error}

Here is a false proof about connectivity.  It exposes a very common
flaw made on proofs by induction on graphs -- it even has a name --
it is known as ``build-up error''.


\begin{falseclm*}
If every vertex in a graph has degee at least 1, then the graph is
connected.
\end{falseclm*}


There are many counterexamples; here is one:

\mfigure{!}{0.75in}{false-connect-cx}

Since the claim is false, there must be at least one error in the
following ``proof''.  

\begin{proof}
We use induction.  Let $P(n)$ be the proposition that if every vertex
in an $n$-vertex graph has degree at least 1, then the graph is
connected.

\noindent \textit{Base case:} There is only one graph with a single
vertex and it has degree 0.  Therefore, $P(1)$ is vacuously true,
since the if-part is false.

\noindent \textit{Inductive step:} We must show that $P(n)$ implies
$P(n+1)$ for all $n \geq 1$.  Consider an $n$-vertex graph in which
every vertex has degree at least 1.  By the assumption $P(n)$, this
graph is connected; that is, there is a path between every pair of
vertices.  Now we add one more vertex $x$ to obtain an $(n+1)$-vertex
graph:

\mfigure{!}{1.75in}{false-connect-pic}

All that remains is to check that there is a path from $x$ to every
other vertex $z$.  Since $x$ has degree at least one, there is an edge
from $x$ to some other vertex; call it $y$.  Thus, we can obtain a
path from $x$ to $z$ by adjoining the edge $\edge{x}{y}$ to the path
from $y$ to $z$.  This proves $P(n+1)$.

By the principle of induction $P(n)$ is true for all $n \geq 1$, which
proves the theorem.
\end{proof}

That looks fine!  Where is the bug?  It turns out that
faulty assumption underlying this argument is that
\textit{every $(n+1)$-vertex graph with minimum degree 1 can be
obtained from an $n$-vertex graph with minimum degree 1 by adding 1
more vertex}.  Instead of starting by considering an arbitrary $(n+1)$- node
graph, this proof only considered an $(n+1)$-node graph that you
can make by starting with  an $n$-node graph with minimum degree 1.

The counterexample above shows that this assumption is false; there is
no way to build that 4-vertex graph from a 3-vertex graph with minimum
degree 1.  Thus the first error in the proof is the statement ``This
proves $P(n+1)$''.

More generally, this is an example of ``build-up error''.  Generally,
this arises from a faulty assumption that every size $n+1$ graph with
some property can be ``built up'' from a size $n$ graph with the same
property.  (This assumption is correct for some properties, but
incorrect for others--- such as the one in the argument above.)

One way to avoid an accidental build-up error is to use a ``shrink
down, grow back'' process in the inductive step: start with a size
$n+1$ graph, remove a vertex (or edge), apply the inductive hypothesis
$P(n)$ to the smaller graph, and then add back the vertex (or edge)
and argue that $P(n+1)$ holds.  Let's see what would have happened if
we'd tried to prove the claim above by this method:

\noindent \textit{Inductive step:} We must show that $P(n)$ implies
$P(n+1)$ for all $n \geq 1$.  Consider an $(n+1)$-vertex graph $G$ in
which every vertex has degree at least 1.  Remove an arbitrary vertex
$v$, leaving an $n$-vertex graph $G'$ in which every vertex has
degree... uh-oh!

The reduced graph $G'$ might contain a vertex of degree 0, making the
inductive hypothesis $P(n)$ inapplicable!  We are stuck--- and
properly so, since the claim is false!

\end{document}

