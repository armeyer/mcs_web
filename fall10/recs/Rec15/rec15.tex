\documentclass[12pt]{article}
\usepackage{../light}
\usepackage{verbatim}

%\hidesolutions
\showsolutions

\begin{document}

\recitation{15}{November 3, 2010}

%\newpage

\section{The Tao of BOOKKEEPER}

 In this problem, we seek enlightenment through contemplation
of the word $BOOKKEEPER$.

\begin{enumerate}

\item In how many ways can you arrange the letters in the word
$POKE$?

\solution[\vspace{0.5in}]{There are $4!$ arrangements corresponding to
the $4!$ permutations of the set $\set{P, O, K, E}$.}

\item In how many ways can you arrange the letters in the word
$BO_1O_2K$?  Observe that we have subscripted the O's to make them
distinct symbols.

\solution[\vspace{0.5in}]{There are $4!$ arrangements corresponding to
the $4!$ permutations of the set $\set{B, O_1, O_2, K}$.}

\item Suppose we map arrangements of the letters in $BO_1O_2K$ to
arrangements of the letters in $BOOK$ by erasing the subscripts.
Indicate with arrows how the arrangements on the left are mapped to
the arrangements on the right.

\[
\begin{array}{lcl}
\begin{array}{l}
O_2BO_1K \\
KO_2BO_1 \\
O_1BO_2K \\
KO_1BO_2 \\
BO_1O_2K \\
BO_2O_1K \\
\ldots
\end{array}
& \hspace{2in} &
\begin{array}{l}
BOOK \\
OBOK \\
KOBO \\
\ldots
\end{array}
\end{array}
\]

\item What kind of mapping is this, young grasshopper?

\solution[\vspace{0.5in}]{2-to-1}

\item In light of the Division Rule, how many arrangements are there
of $BOOK$?

\solution[\vspace{0.5in}]{$4! / 2$}

\item Very good, young master!  How many arrangements are there of
the letters in $KE_1E_2PE_3R$?

\solution[\vspace{0.5in}]{$6!$}

\item Suppose we map each arrangement of $KE_1E_2PE_3R$ to an
arrangement of $KEEPER$ by erasing subscripts.  List all the different
arrangements of $KE_1E_2PE_3R$ that are mapped to $REPEEK$ in this way.

\solution[\vspace{0.75in}]{$RE_1PE_2E_3K$, $RE_1PE_3E_2K$,
$RE_2PE_1E_3K$, $RE_2PE_3E_1K$, $RE_3PE_1E_2K$, $RE_3PE_2E_1K$}

\item What kind of mapping is this?

\solution[\vspace{0.75in}]{3!-to-1}

\item So how many arrangements are there of the letters in $KEEPER$?

\solution[\vspace{0.5in}]{$6! / 3!$}

\item {\em Now you are ready to face the BOOKKEEPER!}

How many arrangements of $BO_1O_2K_1K_2E_1E_2PE_3R$ are there?

\solution[\vspace{0.5in}]{$10!$}

\item How many arrangements of $BOOK_1K_2E_1E_2PE_3R$ are there?

\solution[\vspace{0.5in}]{$10! / 2!$}

\item How many arrangements of $BOOKKE_1E_2PE_3R$ are there?

\solution[\vspace{0.5in}]{$10! / (2! \cdot 2!)$}

\item How many arrangements of $BOOKKEEPER$ are there?

\solution[\vspace{0.5in}]{$10! / (2! \cdot 2! \cdot 3!)$}

\item How many arrangements of $VOODOODOLL$ are there?

\solution[\vspace{0.5in}]{$10! / (2! \cdot 2! \cdot 5!)$}

\item \textbf{(IMPORTANT) } How many $n$-bit sequences contain $k$
zeros and $(n-k)$ ones?

\solution[\vspace{0.75in}]{$n! / (k! \cdot (n-k)!)$}

This quantity is denoted $\binom{n}{k}$ and read ``$n$ choose $k$''.
You will see it almost every day in 6.042 from now until the end of
the term.

\end{enumerate}

\vspace{0.5in}

\begin{center}
\textit{Remember well what you have learned: subscripts on, subscripts off.}

\textit{This is the Tao of Bookkeeper.}
\end{center}

%%%%%%%%%%%%%%%%%%%%%%%%%%%%%%%%%%%%%%%%%%%%%%%%%%%%%%%%%%%%%%%%%%%%%%%%%%%%%%%

\instatements{\newpage}

\section{Pigeonhole Principle}

Solve the following problems using the pigeonhole principle.  For each
problem, try to identify the {\em pigeons}, the {\em pigeonholes}, and
a {\em rule} assigning each pigeon to a pigeonhole.

\begin{enumerate}
\item In a room of 500 people, there exist two who share a birthday.

\solution[\vspace{1in}]{The pigeons are the 500 people.  The
pigeonholes are 366 possible birthdays.  Map each person to his or her
own birthday.  Since there 500 people and 366 birthdays, some two
people must have the same birthday by the Pigeonhole Principle.}

\item 
%Every MIT ID number starts with a 9 (we think).  
Suppose that
each of the 115 students in 6.042 sums the nine digits of his or her ID
number.  Must two people arrive at the same sum?

\solution[\vspace{1in}]{Yes.  The students are the pigeons, the
possible sums are the pigeonholes, and we map each student to the sum
of the digits in his or her MIT ID number.  Every sum is in the range
from $0$ to $9 \cdot 9 = 81$, which means that
there are 82 pigeonholes.  Since there are more pigeons than
pigeonholes, there must be two pigeons in the same pigeonhole; in
other words, there must be two students with the same sum.}

\item In every set of 100 integers, there exist two whose difference
is a multiple of 37.

\solution[\vspace{1in}]{The pigeons are the 100 integers.  The pigeonholes are the
numbers 0 to 36.  Map integer $k$ to $k \rem 37$.  Since there are 100
pigeons and only 37 pigeonholes, two pigeons must go in the same
pigeonhole.  This means $k_1 \rem 37 = k_2 \rem 37$, which implies
that $k_1 - k_2$ is a multiple of 37.
}


\end{enumerate}


%%%%%%%%%%%%%%%%%%%%%%%%%%%%%%%%%%%%%%%%%%%%%%%%%%%%%%%%%%%%%%%%%%%%%%%%%%%%%%%

\newpage

\section{More Counting Problems}

Solve the following counting problems.  Define an appropriate mapping
(bijective or $k$-to-1) between a set whose size you know and the set
in question.

\begin{enumerate} 

\item \textbf{(IMPORTANT) } In how many ways can $k$ elements be
chosen from an $n$-element set $\set{x_1, x_2, \ldots, x_n}$?

\solution[\vspace{.75in}]{There is a bijection from $n$-bit sequences
with $k$ ones and $n-k$ zeros.  The sequence $(b_1, \ldots, b_n)$ maps to the subset
that contains $x_i$ if and only if $b_i = 1$.  Therefore, the number
of such subsets is $\binom{n}{k}$.}

\item How many different ways are there to select a dozen donuts if
  five varieties are available? (We discussed a bijection for this set
  in Recitation 15. Now use that bijection to give a count.)

  \solution[\vspace{.75in}]{There is a bijection from selections of a
    dozen donuts to 16-bit sequences with exactly 4 ones.  In
    particular, suppose that the varieties are glazed, chocolate,
    lemon, sugar, and Boston creme.  Then a selection of $g$ glazed,
    $c$ chocolate, $l$ lemon, $s$ sugar, and $b$ Boston creme maps to
    the sequence:
%
\[
(g\ 0's)\ 1\ (c\ 0's)\ 1\ (l\ 0's)\ 1\ (s\ 0's)\ 1\  (b\ 0's)
\]
%
Therefore, the number of selections is equal to the number of 16-bit
sequences with exactly 4 ones, which is:
%
\[
\frac{16!}{4!\ 12!} = \binom{16}{4}
\]
}

\item An independent living group is hosting eight pre-frosh,
affectionately known as $P_1, \ldots, P_8$ by the permanent residents.
Each pre-frosh is assigned a task: 2 must wash pots, 2 must
clean the kitchen, 1 must clean the bathrooms, 1 must clean the common
area, and 2 must serve dinner.  In how many ways can $P_1, \ldots,
P_8$ be put to productive use?

\solution[\vspace{.75in}]{There is a bijection from sequences containing two
$P$'s, two $K$'s, a $B$, a $C$, and two $D$'s.  In particular, the
sequence $(t_1, \ldots, t_8)$ corresponds to assigning $P_i$ to
washing pots if $t_i = P$, to cleaning the kitchen if $t_i = K$, to
cleaning the bathrooms if $t_i = B$, etc.  Therefore, the number of
possible assignments is:
%
\[
\frac{8!}{2!\ 2!\ 1!\ 1!\ 2!}
\]
}

\item Suppose that two identical 52-card decks of
are mixed together.  In how many ways can the cards in this
double-size deck be arranged?

\solution[\vspace{.75in}]{The number of sequences of the 104 cards containing 2 of each
card is $104! / (2!)^{52}$.}

\end{enumerate}


%%%%%%%%%%%%%%%%%%%%%%%%%%%%%%%%%%%%%%%%%%%%%%%%%%%%%%%%%


\newpage
\section{Fun with Phonology: Hawaiian}


The Hawaiian language is rich in vowels: it contains
 8 consonants and 25 vowels\footnote{Counting long vowels 
and diphthongs. For this problem, treat each of the 25 vowels as
 a unique single vowel.}! In addition, every word in Hawaiian
 must end in a vowel and must not contain two consonants
 in a row. Let's assume that all combinations of vowels
and consonants that satisfy these constraints are valid.

We'd like to know how many $n$-phoneme words there are in Hawaiian. 
(A \emph{phoneme} is either a single vowel or
a single consonant. Assume no phoneme can be both a vowel and a consonant.) For simplicity, let's assume $n$ is even.

\begin{enumerate}

\item Before tackling the general problem, work out how
 many different words there are with exactly $4$ phonemes.
(Which distributions of vowels and consonants are possible?)

\solution[\vspace{1in}]{Since a consonant cannot go at the end of a word
and no consonant can directly follow another (or equivalently, each consonant must be followed by a vowel), we have these possibilities for vowel/consonant distributions:

\begin{center}
 \begin{tabular}{ccccc}
\emph{VVVV} & \emph{VVCV} & \emph{VCVV} & \emph{CVVV} & \emph{CVCV} 
\end{tabular}
\end{center}

Since these are mutually exclusive, we can find the number of words for each of the five types and sum them together. Using the product rule for each type, we find that the total number of $n$-phoneme words is

\begin{align*}
25^4 + 25^2\cdot8\cdot25 + 25\cdot8\cdot25^2 + 8\cdot25^3 + 8\cdot25\cdot8\cdot25 &= 25^4 + 3\cdot25^3\cdot8 + 8^2\cdot25^2 \\
 &= 805625
\end{align*}
}

\item Now for the general case. Let $A$ be the set of all $n$-phoneme words, and let $A_k$ be the set of all $n$-phoneme words with exactly $k$ consonants. Express $|A|$ in terms of $|A_k|$ for all possible $k$.

\solution[\vspace{.5in}]{
$k$ can range from $0$ to $n/2$ since every consonant is followed by a vowel. Since the set of words with $k$ consonants and the set of words with $j$ consonants where $j \not= k$ are disjoint, we can use the sum rule to compute $|A|$:
\[|A| = \sum_{k=0}^{n/2} |A_k|\]
}

\item Now let's find $|A_k|$ for an arbitrary $k$. For simplicity's sake, assume Hawaiian has only one consonant and only one vowel. Find a bijection between $A_k$ and a set of arbitrary sequences of \texttt{0} and \texttt{1} of length $p$. What is $p$?

\solution[\vspace{1in}] {
Since every consonant must be followed by a vowel, we can group each consonant and the vowel after it into a cluster. If there are $k$ consonants, then there are $k$ clusters. Since there are no further constraints on the distribution of these clusters, we can map each cluster to \texttt{0} and each remaining vowel to \texttt{1}. Since the clustering removes $k$ vowels from consideration, and the number of consonants is equal to the number of clusters, the resulting sequences of \texttt{0} and \texttt{1} have length $n-k$.
}

\item Using this bijection, compute $|A_k|$.

\solution[\vspace{.5in}] {
The number of sequences of $k$ \texttt{0}'s and $n-2k$ \texttt{1}'s is
\[
 {{n-k}\choose{k}}
\]
}

\item How would you change your expression for $|A_k|$ to allow for 8 consonants and 25 vowels, not just one of each?

\solution[\vspace{1in}] {
Each word in $A_k$ is a sequence of $V$'s and $C$'s, where each $V$ can represent any vowel and each $C$ can represent any consonant. The total number of these sequences is ${{n-k}\choose{k}}$, as derived in the previous part.

Since each sequence has $k$ $C$'s and $n-k$ $V$'s, there are $8^k\cdot25^{n-k}$ distinct words that map to the same sequence of $V$'s and $C$'s. In other words, this mapping is $(8^k\cdot25^{n-k})$-to-1, so
\[|A_k| = {{n-k}\choose{k}}\cdot 8^k\cdot25^{n-k}\]
}

\item How many $n$-phoneme words are there in Hawaiian? (You don't have to find a closed form for your expression.)

\solution[\vspace{.5in}] {
Plugging this into the summation, we get
\begin{align*}
 |A| &= \sum_{k=0}^{n/2} |A_k| \\
     &= \sum_{k=0}^{n/2} {{n-k}\choose{k}}\cdot 8^k\cdot25^{n-k}
\end{align*}

}

%\item How many $n$-phoneme words are there in Hawaiian?

%\solution[\vspace{1in}]{Suppose there are $k$ consonants in an $n$-phoneme word. $k$ can range from $0$ to $\floor{n/2}$. Then since each consonant must be followed by a vowel, we can treat each consonant together with its succeeding vowel as a cluster, so that we are counting the number of ways $k$ clusters and $n-2k$ vowels are distributed in an $(n-k)$-phoneme word. This is ${ {n-k} \choose {k}}$. For each distribution, the total number of possible words is $(8\cdot25)^k \cdot 25^{n-2k}$, so that for any $k$, the total number is
%\[
%{{n-k}\choose{k}} (8\cdot25)^k \cdot 25^{n-2k} = {{n-k}\choose{k}} 8^k\cdot25^{n-k}
%\]

%We then sum this expression over all possible $k$:

%\[
%\sum_{k=0}^{\floor{n/2}} {{n-k}\choose{k}} 8^k\cdot25^{n-k}
%\]
%}

\end{enumerate}




%%%%%%%%%%%%%%%%%%%%%%%%%%%%%%%%%%%%%%%%%%%%%%%%%%%%%%%%%%%%%%%%%%%%%%%%%%%%%%%

\newpage
\section*{Appendix: Basic Counting Notions}

\begin{comment}
A \term{bijection} or \term{bijective function} is a function $f : X
\to Y$ such that every element of the codomain is related to exactly
one element of the domain.  Here is an example of a bijection:

\begin{center}
\unitlength = 2pt
\begin{picture}(50,60)(-10,-5)
\thinlines
\put(-30,30){\makebox(0,0){domain}}
\put(60,30){\makebox(0,0){codomain}}
\put(-5,50){\makebox(0,0){$X$}}
  \put(15,50){\makebox(0,0){$f$}}
  \put(35,50){\makebox(0,0){$Y$}}
\put(-5,40){\makebox(0,0){a}}
  \put(0,40){\vector(3,-1){28}}
  \put(35,40){\makebox(0,0){1}}
\put(-5,30){\makebox(0,0){b}}
  \put(0,30){\vector(3,-1){28}}
  \put(35,30){\makebox(0,0){2}}
\put(-5,20){\makebox(0,0){c}}
  \put(0,20){\vector(3,-1){28}}
  \put(35,20){\makebox(0,0){3}}
\put(-5,10){\makebox(0,0){d}}
  \put(0,10){\vector(1,1){28}}
  \put(35,10){\makebox(0,0){4}}
\put(-5,0){\makebox(0,0){e}}
  \put(0,0){\vector(1,0){28}}
  \put(35,0){\makebox(0,0){5}}
\end{picture}
\end{center}
\end{comment}

\begin{mathrule}[Bijection Rule]
If there exists a bijection $f : A \to B$, then $\size{A} = \size{B}$.
\end{mathrule}

\begin{mathrule}[Generalized Pigeonhole Principle]
If $\size{X} > k\cdot \size{Y}$, then for every function $f : X \to Y$ there
exist $k+1$ different elements of $X$ that are mapped to the same
element of $Y$.
\end{mathrule}

\begin{quotation}
\noindent ``If more than $n$ pigeons are assigned to $n$ holes, then
there must exist two pigeons assigned to the same hole.''
\end{quotation}

A \term{$k$-to-1 function} maps exactly $k$ elements of the domain to
every element of the range.  For example, the function mapping each
ear to its owner is 2-to-1:

\begin{center}
\unitlength = 2pt
\begin{picture}(70,56)(-15,-3)
% \put(-15,-3){\dashbox(70,56){}} % bounding box
\put(-2,50){\makebox(0,3)[rt]{ear 1}}
\put(-2,40){\makebox(0,3)[rt]{ear 2}}
\put(-2,30){\makebox(0,3)[rt]{ear 3}}
\put(-2,20){\makebox(0,3)[rt]{ear 4}}
\put(-2,10){\makebox(0,3)[rt]{ear 5}}
\put(-2, 0){\makebox(0,3)[rt]{ear 6}}
\put(30,50){\makebox(0,3)[lt]{person A}}
\put(30,30){\makebox(0,3)[lt]{person B}}
\put(30,10){\makebox(0,3)[lt]{person C}}
\put(0,50){\vector(1,0){28}}
\put(0,40){\vector(3,-1){28}}
\put(0,30){\vector(3,2){28}}
\put(0,20){\vector(3,-1){28}}
\put(0,10){\vector(1,0){28}}
\put(0, 0){\vector(1,1){28}}
\end{picture}
\end{center}

\begin{mathrule}[Division Rule]
If $f : A \to B$ is $k$-to-1, then $\size{A} = k \cdot \size{B}$.
\end{mathrule}

\begin{mathrule}[Product Rule]
If $P_1, P_2, \dots P_n$ are sets, then:
%
\begin{align*}
\size{P_1 \times P_2 \times \cdots \times P_n}
    & = \size{P_1} \cdot \size{P_2} \cdots \size{P_n}
\end{align*}
\end{mathrule}

\begin{mathrule}[Generalized Product Rule]
Let $S$ be a set of length-$k$ sequences.  If there are:
%
\begin{itemize}
\item $n_1$ possible first entries,
\item $n_2$ possible second entries for each first entry,
\item $n_3$ possible third entries for each combination of first and
second entries, etc.
\end{itemize}
%
then:
%
\[
\size{S} = n_1 \cdot n_2 \cdot n_3 \cdots n_k
\]
\end{mathrule}

\begin{mathrule}[Sum Rule]
If $A_1, \dots, A_n$ are disjoint sets, then:
%
\[
\size{A_1 \cup \cdots \cup A_n} = \sum_{k=1}^n \size{A_k}
\]
\end{mathrule}


\end{document}
