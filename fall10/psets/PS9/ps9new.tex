\documentclass[12pt,twoside]{article}   
\usepackage{light}

\newcommand{\hint}[1]{({\it Hint: #1})}
\newcommand{\card}[1]{\left|#1\right|}
\newcommand{\union}{\cup}
\newcommand{\lgunion}{\bigcup}
\newcommand{\intersect}{\cap}
\newcommand{\lgintersect}{\bigcap}
\newcommand{\cross}{\times}

%\hidesolutions
\showsolutions

\begin{document}
\problemset{9}{October 28, 2008}{Thursday, November 6}


%Pigeonhole
\begin{problem}{10}

\bparts
%S08_cp9f.3d
\ppart{5} Show that of any $n+1$ distinct numbers chosen from the
set $\{1,2,\ldots,2n\}$, at least 2 must be relatively prime.
\hint{$\gcd(k,k+1)=1$.}

\solution{Treat the $n+1$ numbers as the pigeons and the $n$ 
disjoint subsets of the form $\{2j-1,2j\}$ as the pigeonholes. The
pigeonhole principle implies that there must exist a pair of 
consecutive integers among the $n+1$ chosen which, as suggested
in the hint, must be relatively prime.}

\ppart{5} Show that any finite connected undirected graph with 
$n \geq 2$ vertices must have 2 vertices with the same degree. 

\solution{In a finite connected graph with $n \geq 2$ vertices,
the domain for the vertex degrees is the set $\{1,2,\ldots,n-1\}$ 
since each vertex can be adjacent to at most all of the remaining
$n-1$ vertices and the existence of a degree 0 vertex would violate
the assumption that the graph be connected.  Therefore, treating
the $n$ vertices as the pigeons and the $n-1$ possible degrees as 
the pigeonholes, the pigeonhole principle implies that there
must exist a pair of vertices with the same degree.}

\eparts
\end{problem}


%%%%%%%%%%%%%%%%%%%%%%%%%%%%%%%%%%%%%%%%%%%%%%%%%
% new problem
% Inclusion-Exclusion (with and without replacement)
\begin{problem}{15} {\bf Under Siege!}

Fearing retribution for the many long hours his students spent 
completing problem sets, Prof. Leighton decides to convert his 
office into a reinforced bunker. His only remaining task is to 
set the 10-digit numeric password on his door.  Knowing the 
students are a clever bunch, he is not going to pick any passwords 
containing the forbidden consecutive sequences "18062", "6042" 
or "35876" (his MIT extension).

%\bparts
%\ppart{10}\label{without-replacement}
How many 10-digit passwords can he pick that don't contain forbidden 
sequences if each number $0, 1, \ldots, 9$ can only be chosen once 
(i.e. without replacement)?

\solution{The number of passwords he can choose is the number of 
permutations of the 10 digits minus the number of passwords containing 
one or more of the forbidden words, which we will find using 
inclusion-exclusion.

There are 6 positions 18062 could appear and the remaining digits 
could be any permutation of the remaining 5 digits.  Therefore,
there are $6 \cdot 5!$ passwords containing 18062. Similarly, there 
are $7 \cdot 6!$ passwords containing 6042 and $6 \cdot 5!$ passwords 
containing 35876.

Each of the forbidden words contain the digit 6 and since he must 
choose each number exactly once, the only way two forbidden words 
can appear in the same password is if they overlap at 6.  The only 
case where this can happen is if the password contains 35876042 and
there are $3 \cdot 2!$ such passwords.

By inclusion-exclusion the total number of passwords not containing
any of the forbidden words is
\[
10! - (6 \cdot 5! + 7 \cdot 6! + 6 \cdot 5!) + 3 \cdot 2! = 3622326
\]
}

%\ppart{10} How many 10-digit passwords can he pick that don't contain
%forbidden sequences if each number $0, 1, \ldots, 9$ can be chosen any 
%number of times (i.e. with replacement)?
%
%\solution{The number of passwords he can choose is the number of 
%length 10 strings over the alphabet $0, 1, \ldots, 9$ minus the 
%number of passwords containing one or more of the forbidden words, 
%which we will again find using inclusion-exclusion.
%
%There are 6 positions 18062 could appear and the remaining digits 
%could be any 5-digit string.  Therefore, there are $6 \cdot 10^5$ 
%passwords containing 18062. Similarly, there are $7 \cdot 10^6$ 
%passwords containing 6042 and $6 \cdot 10^5$ passwords containing 
%35876.
%
%As in part \ref{without-replacement} the only forbidden words
%that can overlap are 6042 and 35876, however, it is now possible
%to have a password containing both 35876 and 18062, both 6042 and 
%18062, or even both 6042 and 35876 without overlap.  No 10-digit
%password can contain all 3 forbidden words.
%
%There are only 2 passwords that contain both 35876 and 18062:
%3587618062 and 1806235876.
%
%To count the passwords containing both 6042 and 18062, there are 2 
%possibilities: 6042 can either come before or after 18062.  For 
%each case there are 3 possible positions for the remaining digit: 
%the first digit, the last digit or between the two words.  There 
%are 10 values for the remaining digit.  Therefore, there are 
%$2 \cdot 3 \cdot 10$ passwords containing both 6042 and 18062.
%
%By similar reasoning there are $2 \cdot 3 \cdot 10$ passwords 
%containing both 6042 and 35876 {\it without overlap} and there 
%are $3 \cdot 10^2$ passwords containing 35876042.
%
%By inclusion-exclusion the total number of passwords not containing
%any of the forbidden words is
%\[
%10^10 - \left[2\cdot(6 \cdot 10^5) + 7 \cdot 10^6 \right] 
%      + \left[2 + 2\cdot(2 \cdot 3 \cdot 10) + 3 \cdot 10^2\right]
%= 10^10 - 8199578 = 9991800422.
%\]
%}
%
%\eparts
\end{problem}



\begin{problem}{50} Be sure to show your work to receive full credit. In this problem we assume a standard card deck of 52 cards.
\bparts

\ppart{5} How many 5-card hands have a single pair and 
no 3-of-a-kind or 4-of-a-kind?

\solution{
There is a bijection with sequence of the form:

\[
(\text{value of pair}, \text{suits of pair}, \text{value of other three cards}, \text{suits of other three cards})
\]

Thus, the number of hands with a single pair is:

\[
13 \cdot \binom{4}{2} \cdot \binom{12}{3} \cdot 4^{3} = 1,098,240
\]


Alternatively, there is also a 3!-to-1 mapping to the tuple:
\[
\begin{array}{l}
(\text{value of pair}, \text{suits of pair}, \\
\text{value 3rd card}, \text{suit 3rd card},
\text{value 4th card}, \text{suit 4th card},
\text{value 5th card}, \text{suit 5th card})
\end{array}
\]

Thus, the number of hands with a single pair is:

\[
\frac{13 \cdot \binom{4}{2} \cdot 12 \cdot 4 \cdot 11 \cdot 4 \cdot 10 \cdot 4}{3!} = 1,098,240
\]
}

\ppart{5} How many 5-card hands have two or more kings?

\solution{This is the set of all hands minus the hands
with either no kings or one king:

\[
\binom{52}{5} - \binom{48}{5} - 4\cdot \binom{48}{4} = 108,336
\]

Alternatively, this is also the set of all hands of two, three, or four kings:

\[
\binom{48}{3}\binom{4}{2} + \binom{48}{2}\binom{4}{3} + \binom{48}{1}\binom{4}{4} = 108,336
\]
}

\ppart{5} How many 5-card hands contain the ace of spades, the ace of
clubs, or both?

\solution{
There are $\binom{51}{4}$ hands containing the ace of spades, an equal
number containing the ace of clubs and $\binom{50}{3}$ containing
both.  By the inclusion-exclusion formula:

\[
\binom{51}{4} + \binom{51}{4} - \binom{50}{3}
\]

hands contain one or the other or both.}

\ppart{5} For fixed positive integers $n$ and $k$, how many nonnegative 
integer solutions $x_0,x_1,\ldots,x_k$ are there to the following equation?
\[
\sum_{i=0}^k x_i = n
\]

\solution{There is a bijection from the solutions of the equation
to the binary strings containing $n$ zeros and $k$ ones where
$x_0$ is the number of 0s preceding the first 1, $x_k$ is the 
number of 0s following the last 1 and $x_i$ is the number of 0s 
between the $i^{th}$ and $(i+1)^{th}$ 1 for $0 < i < k$.

\[
\binom{n+k}{k}
\]
}

\ppart{5} For fixed positive integers $n$ and $k$, how many nonnegative 
integer solutions $x_0,x_1,\ldots,x_k$ are there to the following equation? 
\[
\sum_{i=0}^k x_i \leq n
\]

\solution{There is a bijection from the solutions of
\begin{align*}
\sum_{i=0}^k x_i
 & \leq n & \\
 & = n - x_{k+1} & \text{(for some $x_{k+1} \geq 0$)}
\end{align*}
and the solutions of
\[
\sum_{i=0}^{k+1} x_i = n.
\] 

\[
\binom{n+k+1}{k+1}
\]
}

\ppart{5} In how many ways can $3n$ students be broken up into 
$n$ groups of 3?  

\solution{
\[
\frac{(3n)!}{(3!)^n n!}.
\]
}

\ppart{5} How many simple undirected graphs are there with $n$ vertices?

\solution{There are $\binom{n}{2}$ potential edges, each of which may or
may not appear in a given graph.  Therefore, the number of graphs is:
\[
2^{\binom{n}{2}}
\]
}

\ppart{5} How many directed graphs are there with $n$ vertices (self loops allowed)?

\solution{There are $n^2$ potential edges, each of which may or
may not appear in a given graph.  Therefore, the number of graphs is:
\[
2^{n^2}
\]
}

\newcommand{\beats}{\rightarrow}

\ppart{5} How many tournament graphs are there with $n$ vertices?

\solution{There are no self-loops in a tournament graph and 
for each of the $\binom{n}{2}$ pairs of distinct vertices $a$ and $b$,
either $a \beats b$ or $b \beats a$ but not both. Therefore, the 
number of tournament graphs is:
\[
2^{\binom{n}{2}}
\]
}

\ppart{5} How many acyclic tournament graphs are there with $n$ vertices?

\solution{For any path from $x$ to $y$ in a tournament graph, an edge
$y \beats x$ would create a cycle.  Therefore in any acyclic tournament
graph, the existence of a path between distinct vertices $x$ and $y$ 
would require the edge $x \beats y$ also be in the graph.  That is, the
"beats" relation for such a graph would be transitive.  Since each 
pair of distinct players are comparable (either $x \beats y$ or 
$y \beats x$) we can uniquely rank the players $x_1 < x_2 < \cdots < x_n$.
There are $n!$ such rankings.
}

\eparts
\end{problem}

\begin{problem}{10}
 Suppose we have a deck of cards that has 4 suits, each suit having 13 cards.  
The magician asks the audience to select an arbitrary set of 7 cards. His 
assistant selects $v$ cards out of the 7 cards and puts these $v$ cards on a 
table. Is it possible for the magician to figure out the identities of the 
$7-v$ remaining cards that are hidden from him (by only considering the $v$ 
cards that his assistant put on the table)?

\bparts
\ppart{5} Use a counting argument to show that for $v=5$ the magician and 
assistant can work together such that the magician is able find the 
identities of the hidden cards.

\solution{The assistant has ${7 \choose 5}$ choices to select $v=5$ cards 
and $5!$ choices to order these cards. This gives ${7 \choose 5}5!= 7!/2=2520$ 
ways to map a set of 7 cards to a sequence of 5 cards.

There are $7-v=2$ hidden cards. The complete deck has 52 cards and $5$ cards are 
visible. So, the two hidden cards are from a set of 47 cards. There are 
${47 \choose 2}=47\cdot 46/2=1081$ ways to select 2 out of 52 cards.

Since the number of possible mappings is at least the number of possible pairs 
of hidden cards, it is possible for the assistant and magician to agree on a 
mapping such that the magician is able to find the identities of the hidden cards.

{\bf UPDATED SOLUTION...}

The solution presented above is insufficient to prove that there exists a
perfect matching in a bipartite graph where $L$ consists of all ordered 
sequences of 5 cards chosen from a 52 card deck, $R$ consists of all 
7 element subsets of the deck and an edge exists between $l \in L$ and
$r \in R$ if and only if all the cards in $l$ are also in $r$.

To do this we would need to either provide a constructive proof that such 
a matching exists by describing an explicit encoding that is a surjective 
mapping from $L$ to $R$ or appeal to a sufficient condition for the 
existence of a perfect matching in a bipartite graph covering $R$ called 
Hall's condition, which is automatically satisfied if the degree of each 
vertex in $L$ is less than or equal to the degree of every vertex in $R$.

{\it Hall's theorem was not one of the topics covered this term and it is 
not something you should worry about when preparing for the final exam.}
}



\ppart{5}  Is it possible to make the card trick work for $v=4$? Explain your 
answer.

\solution{No. The assistant has ${7 \choose 4}$ choices to select $v=4$ cards 
and $4!$ choices to order these cards. This gives ${7 \choose 4}4!= 7!/3!=840$ 
ways to map a set of 7 cards to a sequence of 4 cards.

There are $7-v=3$ hidden cards. The complete deck has 52 cards and $4$ cards 
are visible. So, the two hidden cards are from a set of 48 cards. There are 
${48 \choose 3}=48\cdot 47\cdot 46/3!=17296$ ways to select 3 out of 52 cards.

Since the number of possible mappings is more than the number of possible 
triples of hidden cards, it is not possible for the assistant and magician to 
agree on a mapping such that the magician is able to find the identities of 
the hidden cards.
}
\eparts
\end{problem}



% S08 - cp11w
\begin{problem}{15}
Give a combinatorial proof of the following theorem:
\[
n 2^{n-1} = \sum_{k=1}^n k \binom{n}{k}
\]

\hint{Consider the set of all length-$n$ sequences of 0's, 1's and a
single *.}

\solution{ Let $P = \{0,\dots,n-1\} \times \{0,1\}^{n-1}$.  On
the one hand, there is a bijection from $P$ to $S$ by mapping $(k,x)$ to
the word obtained by inserting a * just after the $k$th bit in the
length-$n-1$ binary word, $x$.  So
\begin{equation}\label{cp11m.P}
\card{S} = \card{P}= n 2^{n-1}
\end{equation}
by the Product Rule.

On the other hand, every sequence in $S$ contains between 1 and $n$
nonzero entries since the $*$, at least, is nonzero.  The mapping from a 
sequence in $S$ with exactly $k$ nonzero entries to a pair consisting of
the set of positions of the nonzero entries and the position of the *
among these entries is a bijection, and the number of such pairs is
$\binom{n}{k}k$ by the Generalized Product Rule.
Thus, by the Sum Rule:
\[
\card{S} = \sum_{k=1}^n k \binom{n}{k}
\]
Equating this expression and the expression~\eqref{cp11m.P} for $\card{S}$
proves the theorem.}

\end{problem}

%%%%%%%%%%%%%%%%%%%%%%%%%%%%%%%%%%%%%%%%%%%%%
%Suggested new problems:

\begin{problem}{20}
At a congressional hearing, there are $2n$ members present. Exactly $n$ of them are Democrats and $n$ of them are Republicans. The members want to select a smaller subcommittee of size $n$ from within those present at the hearing. However, since the Democrats currently hold majority, they want there to be more Democrats then Republicans in the committee. In how many ways can you select such a committee?
\hint{Consider two cases: $n$ odd and $n$ even.}
\solution{First, look at the case when $n$ is odd. There are are
\[\binom{2n}{n}\]
ways to choose any subcommittee of size $n$. $n$ is odd, so the number of Democrats is always different than the number of Republicans in a committee. Each of these committees will therefore either have more Democrats or more Republicans. However, there is an equal number of Democrats and Republicans present at the hearing, so the number of committees with more Republicans in them should by symmetry equal to the number of committees with more Democrats in them. Consequently, the number of committees with more Democrats than Republicans is
\[\frac12\binom{2n}{n}.\]}
If, however, $n$ is even, then $\binom{n}{\frac{n}2}\binom{n}{\frac{n}2}$ committees will have an equal number of Democrats and Republicans (select $\frac{n}{2}$ out of $n$ Democrats and $\frac{n}{2}$ out of $n$ Republicans). The number of committees where one party has a majority is therefore $\binom{2n}{n}-\binom{n}{\frac{n}2}\binom{n}{\frac{n}2}$. Again by symmetry, there must be 
\[\frac12\Big(\binom{2n}{n}-\binom{n}{\frac{n}2}\binom{n}{\frac{n}2}\Big)\]
committees with more Democrats than Republicans.
\end{problem}

\begin{problem}{20}
Calculate the following. Make sure to explain your reasoning.
\bparts
\ppart{5}
How many n-digit PIN numbers are there where no $2$ consecutive digits are the same?
\solution{There are $10$ choices for the first digit, and $9$ choices for each of the remaining $n-1$ digits, since you can choose any digit that is not the same as the one right in front of it, so there are
\[10\cdot9^{n-1}\]
such PIN numbers.}
\ppart{5}
How many numbers are there that are in the range $[1..700]$ which are divisible by $2,5$ or $7$?
\solution{
Let $S$ be the set of all numbers in range $[1..700]$. Let $S_2$ be the numbers in this range divisible by $2$, $S_5$ be the numbers in this range divisible by $5$ and $S_7$ be the numbers in this range divisible by $7$. By inclusion-exclusion, the number of elements in $S$ divisible by $2$, $5$ or $7$ is
\begin{eqnarray*}
n &=& |S_2|+|S_5|+|S_7| - |S_2S_5|-|S_2S_7|-|S_5S_7|+|S_2S_5S_7|\\
&=& \frac{700}{2}+\frac{700}{5}+\frac{700}{7}-\frac{700}{2\cdot5}-\frac{700}{2\cdot7}-\frac{700}{5\cdot7}+\frac{700}{2\cdot5\cdot7}\\
&=& 350+140+100-70-50-20+10\\
&=& 460.
\end{eqnarray*}
}
\ppart{10}
How many $10$ digit numbers are there in which there are exactly $5$ occurances of the digit $9$ and the first two digits are not the same?
\solution{
Consider 2 cases
\begin{enumerate}
\item The first digit is a $9$.\\
Then the remaining nines have to be in the last $8$ slots and the remaining $5$ slots can be filled with any of the remaining $9$ digits. There are
\[\binom949^5\]
such numbers.
\item The first digit is not a $9$, but the second one is.\\
Then the remaining nines have to be in the last $8$ slots and the remaining $5$ slots can again be filled with any of the remaining $9$ digits. There are again
\[\binom949^5\]
such numbers.
\item The first two digits are not nines.
In this case, there are $9\cdot8$ ways to select the first two digits, the $5$ nines have to be in the last $8$ slots and the remaining $3$ slots can be filled with any digit other than $9$. This gives
\[9\cdot8 \binom859^3\]
such numbers.
\end{enumerate}
In the end, the final answer is
\[2\cdot\binom94\cdot9^5+8\cdot\binom85\cdot9^4.\]
}
\eparts
\end{problem}

\begin{problem}{10}
In the card game of bridge, you are dealt a hand of $13$ cards from the standard $52$-card deck.
\bparts
\ppart{5}
A balanced hand is one in which a player has roughly the same number of cards in each suit. How many different hands are there where the player has $4$ cards in one suit and $3$ cards in each of the other suits?
\solution{There are $4$ suits to pick from for the longest suit, $4$ cards out of $13$ to choose from in that suit, and $3$ cards out of $13$ to choose from in each of the remaining $3$ suits.
This gives
\[4\cdot\binom{13}4\cdot\binom{13}{3} \cdot\binom{13}{3}\cdot\binom{13}{3}\]
such hands.}
\ppart{5}
Not surprisingly, a non-balanced hand is one in which a player has more cards in some suits than others. Hands that are very disired are ones where over half the cards are in one suit. How many different hands are there where there is exactly $7$ cards in one suit?
\solution{
There are $4$ suits to choose from for the long suit, $7$ cards out of $13$ to choose in that suit, and $6$ cards out of the remaining $39$ in the other $3$ suits. This gives
\[4\cdot\binom{13}{7}\cdot\binom{39}6\]
such hands.
}
\eparts
\end{problem}

\begin{problem}{10}
Show that among any $6$ points within a circle of radius $r$, there exists a pair of points with distance at most $r$ between them.
\solution{Order the points clockwise according to their position and look at the radial distance. The total radial distance in $2\pi$. So there must be two consecutive points with radial distance at most $\frac{\pi}3$. The maximum possible distance between these two points is $r$.}
\end{problem}

\begin{problem}{50}
Let's count all poker hands that you might get assuming a standard $52$ card deck.
\bparts
\ppart{5}
To start off easily, how many total possible poker hands (of $5$ cards) are there?
\solution{\[\binom{52}5\]}
\ppart{5}
How many hands are there that have only one pair (no two pairs, no $3$ or $4$ of a kind)?
\solution{\[13 \cdot \binom42 \cdot 48 \cdot 44 \cdot 40\]} 
\ppart{5}
How many hands are there that have two pairs (and no $3$ of a kind)?
\solution{\[\binom{13}2 \cdot \binom 42 \cdot  \binom 42 \cdot 44\]}
\ppart{5}
How many hands are there that have a $3$ of a kind (but no full house)?
\solution{\[13 \cdot \binom42 \cdot 48 \cdot 44\]}
\ppart{5}
How many hands are there that have a $4$ of a kind?
\solution{\[13 \cdot 48\]}
\ppart{5}
How many hands are there that have a full house ($3$ of a kind plus a pair)?
\solution{\[13 \cdot 12 \cdot \binom43 \cdot \binom 42\]}
\ppart{5}
How many hands are there that have a run ($5$ consecutive cards in any suit)?
\solution{\[9 \cdot 4^5\]}
\ppart{5}
How many hands are there in which all cards are in one suit?
\solution{\[4 \cdot \binom{13}5\]}
\ppart{10}
How many hands are there which contain none of the things from parts b-g (these are really bad hands)? You can refer to your previous answer, so write $a$ to refer to the answer you got in part a.
\hint{Think about whether there are hands that can include more than one of the things and use inclusion-exclusion.}
\solution{The only two things you can get at the same time is g and h. There are $9\cdot4$ such hands, so the final answer is:
\[a-(b+c+d+e+f+g+h)+36.\]}
\eparts
\end{problem}

%Note: these are hard. Probably need more points per part then the previous problem.
\begin{problem}{30}
\bparts
\ppart{5}
How many ways are there to distibute $n$ dollars among $k$ people assuming each person has to get an integer amount of dollars?
\solution{\[\binom{n+k-1}{k-1}\]}
\ppart{5}
How about if everyone has to get at least one dollar?
\solution{\[\binom{n-1}{k-1}\]}
\ppart{10}
In how many ways can you arrange $n$ books on $k$ bookshelf (assuming the order of books on a shelf matters?)
\solution{\[n!\cdot \binom{n+k-1}{k-1}\]}
\ppart{10}
How about if there has to be at least $1$ book at each bookshelf?
\solution{\[k!\cdot \binom{n}{k}\cdot (n-k)! \cdot \binom{n-1}{k-1}\]}
\eparts
\end{problem}


\begin{problem}{10}
%%%%%%%%%%%%%%%%%%%%%%%%%%%%%%%%%%%%%%%%%%%%%%%%%%%%%%%%  from fa06, good example of subset split / multinomial coeffs
In preparation for a 6.042 study session, you want to calculate the
number of different ways to make sundaes for you and your friends.
You have 10 different toppings, and you want to make four sundaes
such that each sundae has between one and four (inclusive) toppings,
and you don't reuse any toppings. The sundaes are going to 4
different people, so their order matters! How many ways can this be
done?

\solution{We first enumerate the different ways to divide up 10
toppings to the 4 sundaes, where we don't distinguish between
toppings, and the order of the sundaes doesn't matter. One such way
would be to put 1 topping on one sundae, and 3 on each of the other
3, denoted by $[1, 3, 3, 3]$. Similarly, we have $[1, 1, 4,
4]$, $[1, 2, 3, 4]$, $[2, 2, 3, 3]$, and $[2, 2, 2, 4]$.\\

Now, let's consider the order of the sundaes. For $[1, 3, 3, 3]$, we
also have [3, 1, 3, 3], [3, 3, 1, 3], and [3, 3, 3, 1], so there are
4 ways to get 1 topping on 1 sundae and 3 on each of the other 3
(also, $4!/3!$, as we have 4 elements to order, 3 of which are the
same). Similarly, we have $4!/(2!2!) = 6$ ways of getting $[1, 1, 4,
4]$, $4! = 24$ ways for $[1, 2, 3, 4]$, $4!/(2!2!) = 6$ ways for
$[2, 2, 3, 3]$, and $4!/3! = 4$ ways for $[2, 2, 2, 4]$.\\

Finally, let's distinguish between the toppings. We assume that the
order of toppings on a given sundae doesn't matter (i.e. a sundae
with fudge and whipped cream is the same as a sundae with whipped
cream and fudge). We then have $10!/(1!3!3!3!)$ ways to distinguish
between toppings when we have the division $[1, 3, 3, 3]$, and so
on. Therefore, the final answer is:

$$
4\frac{10!}{1!3!3!3!} + 6\frac{10!}{1!1!4!4!} +
24\frac{10!}{1!2!3!4!} + 6\frac{10!}{2!2!3!3!} +
4\frac{10!}{2!2!2!4!}
$$
$$
= 10!(\frac{4}{216} + \frac{6}{576} + \frac{24}{288} + \frac{6}{144}
+ \frac{4}{192}) = 634200.
$$

(Note that this is a large proportion of the total $4^{10} =
1048576$ ways to distribute 10 toppings to 4 sundaes, so our
restrictions weren't that restrictive!)


}
\end{problem}



\end{document}







%%%%%%%%%%%%%%%% EXTRA CREDIT %%%%%%%%%%%%%%%%%%%%%%%%

% F07
\begin{problem}{8} {\bf EXTRA CREDIT}

  We will now use Inclusion-Exclusion to calculate Euler's function,
  $\phi(n)$.  By definition, if $n$ is a positive integer, $\phi(n)$
  is the number of nonnegative integers less than $n$ that are
  relatively prime to $n$.  But the set, $S$, of nonnegative integers
  less than $n$ that are \emph{not} relatively prime to $n$ will be
  easier to count.

  Suppose the prime factorization of $n$ is $p_1^{e_1}\cdots
  p_m^{e_m}$ for distinct primes $p_i$. For $i=1, \ldots, m$, let
  $D_i$ be the set of non-negative integers less than $n$ that are
  divisible by $p_i$.

\bparts

\ppart{2} Show that $S = \bigcup_{i=1}^m D_i$.

\solution{Suppose an integer $s$ belongs to $S$.  Then $s$ is not
  relatively prime to $n$.  Let $d>1$ be the gcd of $n$ and $s$.  Let
  $p$ be any prime dividing $d$. Since $p$ divides $n$, $p$ is one of
  the primes dividing $n$, say $p_i$.  Since $p_i$ also divides $s$,
  $s$ belongs to $D_i$.  On the other hand, if $s$ belongs to $D_i$,
  $s$ and $n$ are then both divisible by $p_i$ and so $s$ belongs to
  $S$.}

\ppart{1} If $r$ is a divisor of $n$, and $D(r)$ is the set of all
non-negative integers less than $n$ and divisible by $r$, what is the
size of $D(r)$?

\solution{$D(r) = \{0,r,2r,\ldots, (n/r-1)r\}$, so $|D(r)| = n/r$.}

\ppart{2} Write down the inclusion exclusion formula for $S$.

\solution{
\begin{align*}
\card{S}
  & = \card{\lgunion_{i=1}^m D_i}\\
  & = \sum_{i=1}^m \card{D_i} - \sum_{1\leq i < j \leq m} \card{D_i \intersect D_j}
       + \sum_{1\leq i < j < k \leq m} \card{D_i \intersect D_j \intersect D_k} -
       \cdots \pm \card{\lgintersect_{i=1}^m D_i}
     \end{align*}}

   \ppart{1} Use part b to replace the cardinalities in the
   inclusion-exclusion formula with expressions in terms of $n$ and
   the $p_i$.


   \solution{If $i<j$ then $D_i \cap D_j = D(p_i p_j)$. (An integer
     between $0$ and $n-1$ belongs to either set if and only if it is
     divisible by $p_i p_j$.)  Thus $|D_i \cap D_j| = n/(p_i p_j)$.
     Similarly, $D_i \cap D_j \cap D_k = D(p_i p_j p_k)$ if $i < j <k$
     and so $|D_i \cap D_k \cap D_k| = n/(p_i p_j p_k)$, etc.
     Plugging these formulas in to the inclusion-exclusion formula we
     get:
  \begin{align*}
\card{S}
  & = \sum_{i=1}^m \card{D_i} - \sum_{1\leq i < j \leq m} \card{D_i \intersect D_j}
       + \sum_{1\leq i < j < k \leq m} \card{D_i \intersect D_j \intersect D_k} -
       \cdots \pm \card{\lgintersect_{i=1}^m D_i}\\
  & = \sum_{i=1}^m \frac{n}{p_i} -
      \sum_{1\leq i < j \leq m} \dfrac{n}{p_ip_j}
       + \sum_{1\leq i < j < k \leq m} \dfrac{n}{p_ip_jp_k} - \cdots
       \pm \dfrac{n}{\prod_{i=1}^m p_i}\\
  & = n \paren{1 - \prod_{i=1}^m \paren{1 - \frac{1}{p_i}}}
\end{align*}
}

\ppart{1} Use the previous part to show that $\phi(n) = n \prod_{i=1}^m
\paren{1 - \frac{1}{p_i}}$.

\solution{$\phi(n)=n-\card{S}$ by definition so \[\phi(n) = n - n
  \paren{1 - \prod_{i=1}^m \paren{1 - \frac{1}{p_i}}} = n
  \prod_{i=1}^m \paren{1 - \frac{1}{p_i}}\]}

\ppart{2} Use the last part to prove that, as claimed in the Notes on
Number Theory,
\[
\phi(ab) = \phi(a)\phi(b)
\]
for relatively prime integers $a,b>1$.

\solution{Suppose the list of all primes dividing $a$ is $p_1, \ldots,
  p_k$ and the list of all primes dividing $b$ is $q_1, \ldots ,q_l$.
  Since $a$ and $b$ are relatively prime, both lists are disjoint and
  the list of primes dividing $a b$ is $p_1, \ldots, p_k,q_1, \ldots
  q_l$.  But this means that \begin{align*}\phi(ab) = &
    ab(1-1/p_1)\cdots(1-1/p_k)(1 -
    1/q_1)\cdots(1-1/c_l)\\
    =& (a(1-1/p_1)\cdots(1-1/p_k))(b(1 - 1/q_1)\cdots(1-1/c_l))\\
    = & \phi(a) \phi(b)\end{align*}} \eparts

\end{problem}


\begin{problem}{$\sqrt{-1}$} {\bf JUST FOR FUN (very tough)}

Recall from Problem Set 6 that the binary de Bruijn strings are 
the length $2^k+k-1$-bit strings containing all possible $k$-bit
strings as consecutive subsequences.  How many de Bruijn strings
are there?

Finding the closed form is not incredibly difficult but finding 
a clear derivation may be very tricky.

\end{problem}

%%%%%%%%%%%%%%%% DISCARDED %%%%%%%%%%%%%%%%%%%%%%%%%%%

\begin{problem}{10}
From what we have learned, we may be able to do an even more magical card trick! Suppose we have a deck of cards that has 4 suits, each suit having 57 cards. This makes $4\cdot 57 = 228$ cards in total. The magician asks the audience to select an arbitrary set of 9 cards. His assistant selects $v$ cards out of the 9 cards and puts these $v$ cards on a table. Is it possible for the magician to figure out the identities of the $9-v$ remaining card that are hidden from him (by only considering the $v$ cards that his assistant put on the table)?

\bparts
\ppart{0} Use a counting argument to show that for $v=7$ the magician and assistant can work together such that the magician is able find the identities of the hidden cards.

\solution{The assistant has ${9 \choose 7}$ choices to select $v=7$ cards and $7!$ choices to order these card. This gives ${9 \choose 7}7!= 9!/2=181440$ ways to map a set of 9 cards to a sequence of 7 cards.

There are $9-v=2$ hidden cards. The complete deck has 228 card and $7$ cards are visible. So, the two hidden cards are from a set of 221 cards. There are ${221 \choose 2}=221\cdot 220/2=24310$ ways to select 2 out of 221 cards.

Since the number of possible mappings is at least the number of possible pairs of hidden cards, it is possible for the assistant and magician to agree on a mapping such that the magician is able to find the identities of the hidden cards.
}

\ppart{0} Prove that there exists at least 3 cards of the same suit among an arbitrary set of 9 cards.

\solution{Use the pigeon hole principle: there are 9 cards (pigeons) and 4 suits (holes). Since $9>2\cdot 4$, there is at least one suit (hole) that matches 3 cards (pigeons).
}

\ppart{0} Consider 3 cards that have the same suit. Order the 57 cards of the suit in a circle. Label the 3 cards A, B, and C, according to their clockwise position on the circle. That is, B follows A, C follows B, and A follows C in clockwise direction. Prove that the number of cards between A and B is at least 18, or the number of cards between B and C is at least 18, or the number of cards between C and A is at least 18.

\solution{Prove by contradiction. Suppose that the number of cards between A and B, B and C, and C and A are each $<18$. Then, there are less than $1+18+1+18+1+18=57$ cards on the circle. This contradict the fact that a suit has exactly 57 cards.
}

\ppart{0} Suppose that the number of cards between A and B is at least 18. Let card A be the first card of the sequence of $v=7$ cards shown by the assistant. Let B and C be the cards that are hidden from the magician. Show that the assistant is able to order the remaining 6 cards such that the magician can compute the positions of cards B and C on the circle. 

\solution{Card A shows which $1+18=19$ positions on the circel do not contain B or C. The part of the circle that contains B and C has 57-19=38 cards. There are ${38 \choose 2}=38\cdot 37/2=703$ ways to choose B and C among these 38 cards. The remaining 6 cards on the table can be ordered in $6!=720$ ways. Since $720>703$, the 6 cards can be used to indicate the positions of B and C.
}

\ppart{0} The previous parts show how a card trick for 
$v=7$ may work (altough the magician and assistant need to memorize an ugly large mapping). Is it possible to make the card trick work for $v=6$? Explain your answer.

\solution{No. The assistant has ${9 \choose 6}$ choices to select $v=6$ cards and $6!$ choices to order these card. This gives ${9 \choose 6}6!= 9!/3!=60480$ ways to map a set of 9 cards to a sequence of 6 cards.

There are $9-v=3$ hidden cards. The complete deck has 228 card and $6$ cards are visible. So, the two hidden cards are from a set of 222 cards. There are ${222 \choose 3}=222\cdot 221\cdot 220/3!=1798940$ ways to select 3 out of 222 cards.

Since the number of possible mappings is more than the number of possible triples of hidden cards, it is not possible for the assistant and magician to agree on a mapping such that the magician is able to find the identities of the hidden cards.
}
\eparts
\end{problem}


\begin{problem} Let $T=\{X_1,\ldots, X_t\}$ be a set whose elements $X_i$ are themselves sets such that each $X_i$ has size 3 and is $\subseteq \{1,2,\ldots, n\}$. We call the elements of $T$ triangles. Suppose that for all $P\subseteq \{1,2,\ldots, n\}$ with $|P|=2$ there are exactly $\lambda$ triangles $X\in T$ with $P\subseteq X$. Prove 
$$ \lambda n(n-1) = 6t$$
by counting the set
$$C= \{ (P,X) : X\in T, P\subseteq X, |P|=2\}$$
in two different ways.
\end{problem}

\solution{There are ${n \choose 2}$ sets $P\subseteq \{1,\ldots, n\}$ of size $|P|=2$. For each such $P$ there are exactly $\lambda$ triangles $X\in T$ with $P\subseteq X$. So, $|C|=\lambda {n \choose 2}$.

There are $t$ triangles. Each triangle has exactly ${3 \choose 2}=3$ subsets $P$ of size 2. So, $|C|=3t$. 
}



\begin{problem}[10 points]
%%%%%%%%%%%%%%%%%%%%%%%%%%%%%%%%%%%%%%%%%%%%%%%%%%%%%%%%  from fa06, good example of subset split / multinomial coeffs
In preparation for a 6.042 study session, you want to calculate the
number of different ways to make sundaes for you and your friends.
You have 10 different toppings, and you want to make four sundaes
such that each sundae has between one and four (inclusive) toppings,
and you don't reuse any toppings. The sundaes are going to 4
different people, so their order matters! How many ways can this be
done?

\solution{We first enumerate the different ways to divide up 10
toppings to the 4 sundaes, where we don't distinguish between
toppings, and the order of the sundaes doesn't matter. One such way
would be to put 1 topping on one sundae, and 3 on each of the other
3, denoted by $[1, 3, 3, 3]$. Similarly, we have $[1, 1, 4,
4]$, $[1, 2, 3, 4]$, $[2, 2, 3, 3]$, and $[2, 2, 2, 4]$.\\

Now, let's consider the order of the sundaes. For $[1, 3, 3, 3]$, we
also have [3, 1, 3, 3], [3, 3, 1, 3], and [3, 3, 3, 1], so there are
4 ways to get 1 topping on 1 sundae and 3 on each of the other 3
(also, $4!/3!$, as we have 4 elements to order, 3 of which are the
same). Similarly, we have $4!/(2!2!) = 6$ ways of getting $[1, 1, 4,
4]$, $4! = 24$ ways for $[1, 2, 3, 4]$, $4!/(2!2!) = 6$ ways for
$[2, 2, 3, 3]$, and $4!/3! = 4$ ways for $[2, 2, 2, 4]$.\\

Finally, let's distinguish between the toppings. We assume that the
order of toppings on a given sundae doesn't matter (i.e. a sundae
with fudge and whipped cream is the same as a sundae with whipped
cream and fudge). We then have $10!/(1!3!3!3!)$ ways to distinguish
between toppings when we have the division $[1, 3, 3, 3]$, and so
on. Therefore, the final answer is:

$$
4\frac{10!}{1!3!3!3!} + 6\frac{10!}{1!1!4!4!} +
24\frac{10!}{1!2!3!4!} + 6\frac{10!}{2!2!3!3!} +
4\frac{10!}{2!2!2!4!}
$$
$$
= 10!(\frac{4}{216} + \frac{6}{576} + \frac{24}{288} + \frac{6}{144}
+ \frac{4}{192}) = 634200.
$$

(Note that this is a large proportion of the total $4^{10} =
1048576$ ways to distribute 10 toppings to 4 sundaes, so our
restrictions weren't that restrictive!)


}
\end{problem}
