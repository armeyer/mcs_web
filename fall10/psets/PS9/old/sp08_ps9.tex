\documentclass[11pt,twoside]{article}   
\usepackage{latex-macros/course}% !! Use the macros

%#well.ordering
\renewcommand{\reading}{Notes for
\href{http://courses.csail.mit.edu/6.042/spring08/ln10.pdf}{Week 10}.}


%\showsolutions

\begin{document}
\problemset{9}                 % !! Start the problem set with its number

%%%%%%%%%%%%%%%%%%%%%%%%%%%%%%%%%%%%%%%%%%%%%%%%%
% From F07, ps9, problem 1

\begin{problem}
Suppose you have seven dice--- each a different color of the rainbow;
otherwise the dice are standard, with six faces numbered 1 to 6.  A
\emph{roll} is a sequence specifying a value for each die in rainbow
(ROYGBIV) order.  For example, one roll is $(3,1,6,1,4,5,2)$ indicating
that the red die showed a 3, the orange die showed 1, the yellow 6, the
green 1, the blue 4, the indigo 5, and the violet 2.

For the problems below, describe a bijection between the specified set of
rolls and another set that is easily counted using the Product,
Generalized Product, and similar rules.  Then write a simple numerical
expression for the size of the set of rolls.  You do not need to prove
that the correspondence between sets you describe is a bijection, and you
do not need to simplify the expression you come up with.

For example, let $A$ be the set of rolls where 4 dice come up showing the
same number, and the other 3 dice also come up the same, but with a
different number.  Let $R$ be the set of seven rainbow colors and $S$
be the set $\set{1,\dots,6}$ of dice values.

Define $B \eqdef S_2 \cross \set{3,4} \cross R_3$, where $S_2$ is the set
of size 2 subsets of $S$, and $R_3$ is the set of size 3 subsets of $R$.
Then define a bijection from $A$ to $B$ by mapping a roll in $A$ to the
sequence in $B$ whose first element is the set of two numbers that came
up, whose second element is the number of times the smaller of the two
numbers came up in the roll, and whose third element is the set of colors
of the three matching dice.

For example, the roll
\[
(4,4,2,2,4,2,4) \in A
\]
maps to the triple
\[
(\set{2,4},3,\set{\text{yellow,green,indigo}}) \in B.
\]

Now by the Bijection rule $\card{A} = \card{B}$, and by the Product rule,
\[
\card{B} = \binom{6}{2} \cdot 2 \cdot \binom{7}{3}.
\]

\bparts

\ppart For how many rolls is the value on every die different?

\solution{None, by the Pigeonhole Principle.}

\ppart\label{66} For how many rolls do two dice have the value 6 and the
remaining five dice all have different values?

Example: $(6, 2, 6, 1, 3, 4, 5)$ is a roll of this type, but $(1, 1, 2, 6,
3, 4, 5)$ and $(6, 6, 1, 2, 4, 3, 4)$ are not.

\solution{As in the example, map a roll into an element of $B \eqdef R_2
\cross P_5$ where $P_5$ is the set of permutations of $\set{1,\dots,5}$.  A roll
maps to the pair whose first element is the set of colors of the two dice
with value 6, and whose second element is the sequence of values of the
remaining dice (in rainbow order).  So $(6, 2, 6, 1, 3, 4, 5)$ above maps
to $(\set{\text{red,yellow}}, (2,1,3,4,5))$.  By the Product rule,
\[
\card{B} = \binom{7}{2}\cdot 5!.
\]}

\ppart For how many rolls do two dice have the same value and the
remaining five dice all have different values?

Example: $(4, 2, 4, 1, 3, 6, 5)$ is a roll of this type, but $(1, 1, 2, 6,
1, 4, 5)$ and $(6, 6, 1, 2, 4, 3, 4)$ are not.  

\solution{Map a roll into a triple whose first element is in $S$,
indicating the value of the pair of matching dice, whose second element is the
set of colors of the two matching dice, and whose third element is the
sequence of the remaining five dice values (in rainbow order).

So $(4, 2, 4, 1, 3, 6, 5)$ above maps to $(4, \set{\text{red,yellow}},
(2,1,3,6,5))$.  Notice that the number of choices for the third element of
a triple is the number of permutations of the remaining five values,
namely $5!$.  This mapping is a bijection, so the number of such rolls
equals the number of such triples.
By the Generalized Product rule, the number of such triples is
\[
6 \cdot \binom{7}{2} \cdot 5!.
\]

Alternatively, we can define a map from rolls in this part to the
rolls in part~\eqref{66}, by replacing the value of the duplicated values
with 6's and replacing any 6 in the remaining values by the value of the
duplicated pair.  So the roll $(4, 2, 4, 1, 3, 6, 5)$ would map to the
roll $(6, 2, 6, 1, 3, 4, 5)$.  Now a type~\ref{66} roll, $r$, is mapped to
by exactly the rolls obtainable from $r$ by exchanging occurrences of 6's
and $i$'s, for $i = 1,\dots,6$.  So this map is 6-to-1, and by the
Division rule, the number of rolls here is 6 times the number of rolls in
part~\eqref{66}.

%%%%%%%
\iffalse
An useful variation of this idea is to map a roll into a triple in the set
$B \eqdef S cross R_2 \cross P_5$.  Again, the first element of a triple
in $B$ is the value of the matching dice, the second element is the set of
colors of the two matching dice, and the third element is the sequence of
\emph{rankings}\footnote{Given a set of $n$ numbers, the \emph{rank} of a
number in the set is its position when the numbers are listed in
increasing order.  So the rank of the smallest number is 1, the rank of
the second smallest is 2, \dots, and the rank of the largest is $n$.} of
the remaining five dice values (in rainbow order).

So now $(4, 2, 4, 1, 3, 6, 5)$ above maps to $(4, \set{\text{red,yellow}},
(2,1,3,5,4))$.  By the Product rule, $\card{B} = \binom{7}{2}\cdot 5!$.

The use of ranking let us define $B$ without having the possible values of
third elements in triples depend on the first elements.  This let us give
a simpler definition of the set, $B$, of triples corresponding to rolls.
With the simpler definition of $B$, we could calculate its size using the
Product rule instead of the Generalized Product rule.
\fi
%%%%%%%
}

\ppart For how many rolls do two dice have one value, two different dice
have a second value, and the remaining three dice a third value?

Example: $(6, 1, 2, 1, 2, 6, 6)$ is a roll of this type, but $(4, 4, 4, 4,
1,3,5)$ and $(5, 5, 5, 6, 6,1,2)$ are not.

\solution{Map a roll of this kind into a 4-tuple whose first element is
the set of two numbers of the two pairs of matching dice, whose second
element is the set of two colors of the pair of matching dice with the
smaller number, whose third element is the set of two colors of the larger
of the matching pairs, and whose fourth element is the value of the
remaining three dice.  For example, the roll $(6, 1, 2, 1, 2, 6, 6)$ maps
to the triple
$(\set{1,2},\set{\text{orange,green}},\set{\text{yellow,blue}}, 6)$.

There are $\binom{6}{2}$ possible first elements of a triple,
$\binom{7}{2}$ second elements, $\binom{5}{2}$ third elements since the
second set of two colors must be different from the first two, and 4 ways
to choose the value of the three dice since their value must differ from
the values of the two pairs.  So by the Generalized Product rule, there
are
\[
\binom{6}{2} \cdot \binom{7}{2} \cdot \binom{5}{2} \cdot 4
\]
possible rolls of this kind.}

\eparts

\end{problem}


%%%%%%%%%%%%%%%%%%%%%%%%%%%%%%%%%%%%%%%%%%%%%%%%%
% From F07, ps9, problem 2; S06, ps7, problem 7

\begin{problem}           

  Answer the following questions with a number or a simple formula
  involving factorials and binomial coefficients (that is, $n!$ and
  $\binom{n}{k}$).  Briefly explain your answers.

\bparts

\ppart How many ways are there to order the 26 letters of the
alphabet so that no two of the vowels {\tt a}, {\tt e}, {\tt i},
{\tt o}, {\tt u} appear consecutively and the last letter in the
ordering is not a vowel?

\hint  Every vowel appears to the left of a consonant.

\solution{The constraint on where vowels can appear is equivalent to the
requirement that every vowel appears to the left of a consonant.  So given
a sequence of the 21 consonants, there are $\binom{21}{5}$ positions where
the 5 vowels can be placed.  After determining such a placement, we can
reorder the consonants and vowels in any order.  Thus, the number is:
\[
\binom{21}{5}\cdot 21! \cdot 5!.
\]
}

%%%%%%%
\iffalse
\ppart How many ways are there to order the 26 letters of the alphabet
so that there are {\em at least two} consonants immediately following
each vowel?

\solution{The pattern of consonants and vowels in any permutation of the
26 letters of the alphabet can be indicated by a binary string with 5
ones indicating where the vowels occur and 21 zeros where the consonants
occur.  Patterns where every vowel has at least two consonants to its
right can be constructed by taking a sequence of 16 zeros and inserting
``10'' to the left of 5 of the 16 zeros.  There are $\binom{16}{5}$ ways to
do this.  For any such pattern, there are $5!$ ways to place the vowels
in the positions where ones occur and $21!$ ways to place the consonants
where the ones occur.  Thus, the final answer is:
\[
\binom{16}{5}\cdot 5! \cdot 21!.
\]
}
\fi
%%%%%%%

\ppart In how many different ways can the letters in the name of the
popular 1980's band \texttt{BANANARAMA} be arranged?

\solution{There are 5 $A$'s, 2 $N$'s, 1 $B$, 1 $R$, and 1 $M$.
Therefore, by the Bookkeeper Rule, the number of arrangements is:
\[
\frac{10!}{5!\ 2!\ 1!\ 1!\ 1!}
\]
}

\ppart In how many different ways can $2n$ students be paired
up?

\solution{Pair up students by the following procedure.  Line up the
students and pair the first and second, the third and fourth, the
fifth and sixth, etc.  The students can be lined up in $(2n)!$ ways.
However, this overcounts by a factor of $2^n$, because we would get
the same pairing if the first and second students were swapped, the
third and fourth were swapped, etc.  Furthermore, we are still
overcounting by a factor of $n!$, because we would get the same
pairing even if pairs of students were permuted, e.g. the first and
second were swapped with the ninth and tenth.  Therefore, the number
of pairings is:
\[
\frac{(2n)!}{2^n \cdot n!}
\]
}

\ppart How many simple graphs are there with $n$ vertices numbered $1,
\dots n$?

\solution{There are $\binom{n}{2}$ potential edges, each of which may or
may not appear in a given graph.  Therefore, the number of graphs is:
\[
2^{\binom{n}{2}}
\]
}

  \ppart Consider the set of $n$-digit sequences of digits 0,1,\dots,9.
  Two sequences are said to be of the \emph{same type} if the digits of
  one are a permutation of the digits of the other.  How many types of
  $n$-digit integers are there?
  
\solution{The type of a string is determined by the numbers of occurrences
   of the 9 different digits in the string.  So there is a bijection
   between types of strings and strings with $n$ 0's and nine 1's: the
   length of the block of 0's before the $i$th 1 equals the number of
   occurrences of the digit $i$ (and the length of the final block of 0's
   equals the number of occurrences of the digit 9). Therefore, the number
   of different types is $\binom{n+9}{9}$}

%%%%%%%

\eparts

\end{problem}


%%%%%%%%%%%%%%%%%%%%%%%%%%%%%%%%%%%%%%%%%%%%%%%%%
% From S06, ps7, problem 5

\begin{problem}
  
  A certain company wants to have security for their computer systems.
  So they have given everyone a name and password.  The passwords are
  exactly 10 letters long.
\bparts

  \ppart If all the passwords must satisfy the following rules, how many
  possible passwords are there :
  \begin{enumerate}
  \item[1.] The passwords must be permutations of the letters (a, d,
    f, i, l, o, p, r, s, t)
  \item[2.] No password can contain the sequence "fail"
  \item[3.] No password can contain the sequence "drop"
  \end{enumerate}
  
  \solution{ The number of possible passwords equals the number of
    permutations of the letters minus the ``illegal" permutations
    (containing ``fail" or ``drop").  The total number of permutations of
    the letters is $10!$.  According to the inclusion-exclusion principle,
    the number of permutations containing either ``fail" or ``drop" is:
    
    \# permutations containing "fail" plus \# permutations containing "drop" -
    permutations containing "fail" and "drop".
    
    The number of permutations that contain the sequence "fail" can be
    seen as finding the number of permutations of the following 7 elements
    : \{d,o, p, r, s, t, \fbox{fail}\} which is $7!$.  A similar argument
    holds for the number of permutations with "drop", so these are also
    $7!$.  The number of permutations with both "fail" and "drop" is the
    same as finding the number of permutations of the following 4 elements
    \{ \fbox{drop}, \fbox{fail}, s, t\} = $4!$.

    Thus the number of possible passwords is $10! - 7! - 7! + 4!$.
  }
  
  \ppart Bob is trying to break into the system.  He has obtained the
  list of all the passwords and the list of all the names, yet he
  doesnt know which passwords go with which names.  If the password
  file contains 100 passwords and there are 100 names, how many
  possible ways can those passwords be associated with the names?
  Approximate your answer using Stirling's approximation.

  \solution{ There are 100 possible names for the first password, 99 for
    the second, \dots and 1 possible for the last.  Multiplying these
    possiblities together gives you $100!$.  By Stirling's approximation
    this is approximately $9.32\times 10^{157}$ ---an unimaginably large
    number.
    
    Notice that the problem could also be stated as: ``find the number
    of \emph{bijections} from the set of names to the set of
    passwords.''}
  
  \ppart When trying to break into the system, Bob doesn't have to
  submit an entire matching of the 100 user names with the 100
  passwords at the same time. Instead, he can try individual pairs of
  names and passwords, that is, he can try to log in as any of the 100
  users with any of the 100 passwords and see if the login attempt is
  successful.
  
  Describe a strategy that Bob can use to completely determine the pairing
  of names with passwords using fewer than 5000 login attempts.
  
  \solution{ Bob uses the following strategy: First he figures out the
    password for the first name on the list. In the \emph{worst} case,
    he'll have to try 99 passwords (if he fails the first 99 times, he
    knows that the last password is correct so he doesn't need to try
    it). Then, for the second name, there are 99 possible passwords so
    he needs to try 98 of them in the worst case, and so on. For the
    99th password there are 2 possible choices, so he needs to try 1
    in the worst case, and he always gets the 100th password for free.
    
    So, in the worst case, he needs $99 + 98 + 97 + \cdots + 1 =
    4,950$ trials.}

\eparts
\end{problem}

%%%%%%%%%%%%%%%%%%%%%%%%%%%%%%%%%%%%%%%%%%%%%%%%%
% From F07, ps9, problem 4

\begin{problem}
  This problem is a continuation of Class Problems 9F,
  \href{http://courses.csail.mit.edu/6.042/spring08/cp9f.pdf#numbered.trees}{problem
    4} on the bijection between trees with vertices labelled $1,2,\dots,n$
  and length $n-2$ words over the alphabet $1,2,\dots,n$.

\bparts

\ppart How is the degree of a vertex related to the number of times it
appears in the code.?

\solution{During the course of creating the code, every vertex is
  reduced from it's orginal degree, to degree $1$.  (The vertex may be
  subsequently be deleted, or may be one of the two vertices that are
  not deleted.)  Every time the degree is reduced, it is because the
  vertex is the father of some current leaf, which is then deleted.
  But this means the vertex appears in the code.  After it has vertex
  of degree $1$, it can no longer appear in the code.  Thus every
  vertex $v$ appears $\deg(v)-1$ times in the code.}

\ppart The degree sequence of a graph, is a list of $\deg(v)$ for
every vertex $v$, that has been sorted into non-increasing order.  How
many trees on $\{1, \ldots, 9\}$ are there with degree sequence
$4,3,2,2,1,1,1,1,1$?

\solution{${9 \choose 5,2,1,1}{7 \choose 3,2,1,1}$

By the product rule, the number of trees is (the number of different ways to assign which value has which degree) $\times$ (the number of different ways to attach the vertices, given their assigned degrees).

To help define this value, we first note: If there are $n_1$ letters of
type $1$, ..., $n_k$ letters of type $k$, the ``bookeeper rule'' gives us
the number of ways of rearranging these letters into a word.  The formal
name for this number is the {\em multinomial coefficient}: ${n_1 + n_2 +
  \cdots + n_k \choose n_1,n_2, \ldots, n_k} := (n_1 + n_2 + \cdots +
n_k)!/(n_1!n_2!\cdots n_k!)$.

Now, the (number of different ways to assign which value has which degree)
corresponds to the number of permutations on the sequence
$4,3,2,2,1,1,1,1,1$, which is ${9 \choose 5,2,1,1}$.

The (number of different ways to attach the vertices, given their assigned
degrees) can be defined by the previous section as a bijection to the
number of different code sequences, given the degree sequence. For degree
sequence $4,3,2,2,1,1,1,1,1$, the number of times the corresponding
vertices appear in the code is $3,2,1,1,0,0,0,0,0$, and a code will
therefore be a permutation of the sequence $a,a,a,b,b,c,d$. The number of
permutations is ${7 \choose 3,2,1,1}$.  }

\eparts
\end{problem} 


\iffalse
%%%%%%%%%%%%%%%%%%%%%%%%%%%%%%%%%%%%%%%%%%%%%%%%%
% From S07, ps9, problem 5

\begin{problem}
How many of the numbers $2, \dots, n$ are prime?  One way to answer this
question is to test each number up to $n$ for primality and keep a count.
A somewhat more efficient method is to use the ``Sieve of Erastosthenes''
procedure which you may have learned about in 6.001 (but, don't worry, you
needn't know about this).  In this problem, we will use the
Inclusion-Exclusion Principle to get the count; this approach turns out to
be much more efficient when $n$ is large.

Actually, we will use Inclusion-Exclusion to count the number of
\emph{composite} (nonprime) integers from 2 to $n$.  Subtracting this
from $n-1$ gives the number of primes.

Let $C_n$ be the set of composites from $2$ to $n$, and let $A_m$ be the
set of numbers in the range $m+1,\dots,n$ that are divisible by $m$.
Notice that by definition, $A_m = \emptyset$ for $m \geq n$.
So
\[
C_n = \lgunion_{i=2}^{n-1} A_i.
\]

\bparts

\ppart Write $C_n$ in terms of a union of $A_p$'s, where $p$ is prime.
Explain why $\sqrt{n}$ is an upper bound on the largest $p$ needed.

\solution{ If $p \divides k$, then $A_p \supseteq A_k$, so there is no
need to include $A_k$ as long as $A_p$ is included in the union.  Also,
any composite $\leq n$ must be divisible by a prime $\leq \sqrt{n}$
(because it is a product of at least two primes, and they can't both be
bigger than $\sqrt{n}$).  So we have
\[
C_n = \lgunion_{p \leq \sqrt{n}} A_p.
\]
}

\ppart What is the cardinality of $A_p$?

\solution{$\card{A_p} = \floor{n/p} - 1$.}

\ppart Let $P$ be a set of primes.  Give a simple formula for
\[
\card{\lgintersect_{p \in P} A_p}.
\]

\solution{
If $\card{P} = 1$, just look at the previous part.

Otherwise, if $\card{P} > 1$,
let $m \eqdef \prod_{p \in P} p$.  Then
\[
\card{\lgintersect_{p \in P} A_p} = \floor{n/m}.
\]
}

\ppart Use the Inclusion-Exclusion principle to obtain a formula for
$\card{C_{150}}$ in terms of nonempty intersections among the sets
$A_2,A_3,A_5,A_7, A_{11}$.

\solution{
\begin{eqnarray*}
|C| & = & |A_2|+|A_3|+|A_5|+|A_7|+|A_{11}|\\
& & - |A_2 \cap A_3| - |A_2 \cap A_5| - |A_2 \cap A_7| - |A_2 \cap A_{11}|\\
& & - |A_3 \cap A_5| - |A_3 \cap A_7| - |A_3 \cap A_{11}|\\
& & - |A_5 \cap A_7| - |A_5 \cap A_{11}|\\
& & - |A_7 \cap A_{11}|\\
& & + |A_2 \cap A_3 \cap A_5| + |A_2 \cap A_3 \cap A_7| + |A_2 \cap A_3 \cap A_{11}|\\
& & + |A_2 \cap A_5 \cap A_7| + |A_2 \cap A_5 \cap A_{11}|\\
& & +|A_3 \cap A_5 \cap A_7|
\end{eqnarray*}
}

\ppart Use this formula to find the number of primes up to 150.

\solution{We have:
\begin{eqnarray*}
|C_{150}| & = & 74+49+29+20+12\\
& & - 25-15-10-6\\
& & - 10-7-4\\
& & - 4 -2\\
& & - 1\\
& & + 5+3+2\\
& & + 2+ 1\\
& & + 1\\
& = & 114
\end{eqnarray*}
The number of primes from 2 to 150 is $(150-1) - C_{150} = 149 -114= 35$.
}

\eparts
\end{problem}
\fi


%%%%%%%%%%%%%%%%%%%%%%%%%%%%%%%%%%%%%%%%%%%%%%%%%
% From F05, ps7, problem 5


\begin{problem}
A \textit{derangement} is a permutation $(x_1, x_2, \dots, x_n)$ of
the set $\set{1, 2, \dots, n}$ such that $x_i \neq i$ for all $i$.
For example, $(2, 3, 4, 5, 1)$ is a derangement, but $(2, 1, 3, 5, 4)$
is not because 3 appears in the third position.  The objective of this
problem is to count derangements.

It turns out to be easier to start by counting the permutations that are
\textit{not} derangements.  Let $S_i$ be the set of all permutations
$(x_1, x_2, \dots, x_n)$ that are not derangements because $x_i = i$.
So the set of non-derangements is
\[
\lgunion_{i=1}^n S_i.
\]

\bparts

\ppart What is $\card{S_i}$?

\solution{There is a bijection between permutations of $\set{1, 2,
\dots, n}$ with $i$ in the $i$-th position and unrestricted
permutations of $\set{1, 2, \dots, n} - i$.  Therefore, $\card{S_i} =
(n-1)!$.}

\ppart What is $\card{S_i \cap S_j}$ where $i \neq j$?

\solution{The set $S_i \cap S_j$ consists of all permutations with $i$
in the $i$-th position and $j$ in the $j$-th position.  Thus, there is
a bijection with permutations of $\set{1, 2, \dots, n} - \set{i, j}$,
and so $\card{S_i \cap S_j} = (n-2)!$.}

\ppart What is $\card{S_{i_1} \cap S_{i_2} \cap \cdots \cap S_{i_k}}$
where $i_1, i_2, \dots, i_k$ are all distinct?

\solution{By the same argument, $(n - k)!$.}

\ppart\label{ie} Use the inclusion-exclusion formula to express the number of
non-derangements in terms of sizes of possible intersections of the sets
$S_1, \dots, S_n$.

\solution{
\[
\sum_i \card{S_i}
- \sum_{i,j} \card{S_i \cap S_j}
+ \sum_{i,j,k} \card{S_i \cap S_j \cap S_k}
- \cdots
\pm \card{S_1 \cap S_2 \cap \cdots \cap S_n}
\]
%
In each summation, the subscripts are distinct elements of $\set{1,
\dots, n}$.}

\ppart How many terms in the expression in part~\eqref{ie} have the form
$\card{S_{i_1} \cap S_{i_2} \cap \cdots \cap S_{i_k}}$?

\solution{There is one such term for each $k$-element subset of the
$n$-element set $\set{1, 2, \dots, n}$.  Therefore, there are
$\binom{n}{k}$ such terms.}

\ppart Combine your answers to the preceding parts to prove the number of
non-derangements is:
\[
n! \paren{\frac{1}{1!} - \frac{1}{2!} + \frac{1}{3!} - \cdots \pm \frac{1}{n!}}.
\]
Conclude that the number of derangements is
\[
n! \paren{1 - \frac{1}{1!} + \frac{1}{2!} - \frac{1}{3!} + \cdots \pm \frac{1}{n!}}.
\]

\solution{
By Inclusion-Exclusion, the number of non-derangements is
\begin{align}
\lefteqn{\sum_i \card{S_i}
- \sum_{i,j} \card{S_i \cap S_j}
+ \sum_{i,j,k} \card{S_i \cap S_j \cap S_k}
- \cdots
\pm \card{S_1 \cap S_2 \cap \cdots \cap S_n}} \notag \\
    & = \binom{n}{1} \cdot (n-1)!
      - \binom{n}{2} \cdot (n-2)!
      + \binom{n}{3} \cdot (n-3)!
      - \cdots
      \pm \binom{n}{n} \cdot 0! \notag\\
    & = n! \paren{\frac{1}{1!} - \frac{1}{2!} + \frac{1}{3!} - \cdots \pm \frac{1}{n!}}\label{nd}
\end{align}

Since there are $n!$ permutation, the number of derangements is $n!$ minus expression~\eqref{nd}.
}

\ppart As $n$ goes to infinity, the number of derangements approaches a constant
fraction of all permutations.  What is that constant?  \hint
%
\[
e^x = 1 + x + \frac{x^2}{2!} + \frac{x^3}{3!} + \cdots
\]

\solution{$1/e$}

\eparts

\end{problem}


%%%%%%%%%%%%%%%%%%%%%%%%%%%%%%%%%%%%%%%%%%%%%%%%%


\begin{problem}

\bparts

\ppart How many 5-card hands have a single pair and no 3-of-a-kind or
4-of-a-kind?

\solution{
There is a bijection with sequence of the form:

\[
(\text{value of pair}, \text{suits of pair}, \text{value of other three cards}, \text{suits of other three cards})
\]

Thus, the number of hands with a single pair is:

\[
13 \cdot \binom{4}{2} \cdot \binom{12}{3} \cdot 4^{3} = 1,098,240
\]


Alternatively, there is also a 3!-to-1 mapping to the tuple:
\[
\begin{array}{l}
(\text{value of pair}, \text{suits of pair}, \\
\text{value 3rd card}, \text{suit 3rd card},
\text{value 4th card}, \text{suit 4th card},
\text{value 5th card}, \text{suit 5th card})
\end{array}
\]

Thus, the number of hands with a single pair is:

\[
\frac{13 \cdot \binom{4}{2} \cdot 12 \cdot 4 \cdot 11 \cdot 4 \cdot 10 \cdot 4}{3!} = 1,098,240
\]
}

\ppart How many 5-card hands have two or more kings?

\solution{This is the set of all hands minus the hands
with either no kings or one king:

\[
\binom{52}{5} - \binom{48}{5} - 4\cdot \binom{48}{4} = 108,336
\]

Alternatively, this is also the set of all hands of two, three, or four kings:

\[
\binom{48}{3}\binom{4}{2} + \binom{48}{2}\binom{4}{3} + \binom{48}{1}\binom{4}{4} = 108,336
\]
}

\ppart How many 5-card hands contain the ace of spades, the ace of
clubs, or both?

\solution{
There are $\binom{51}{4}$ hands containing the ace of spades, an equal
number containing the ace of clubs and $\binom{50}{3}$ containing
both.  By the inclusion-exclusion formula:

\[
\binom{51}{4} + \binom{51}{4} - \binom{50}{3}
\]

hands contain one or the other or both.}

\iffalse

\ppart A committee has 15 Republicans and 10 Democrats.  How many
different 5-person subcommittees with a majority of Republicans are
possible?

\solution{We can choose a committee with $k$ Republicans
and $j$ Democrats in $\binom{15}{k} \cdot \binom{15}{j}$ different
ways.  Therefore, the number of 5-person subcommittees with a majority
of Republicans is:

\[
\binom{15}{5}\binom{10}{0} + 
\binom{15}{4}\binom{10}{1} + 
\binom{15}{3}\binom{10}{2}
\]
}
\fi

\ppart In how many different ways can Blockbuster arrange 64 copies of
{\em Cat in the Hat}, 96 copies of {\em Matrix Revolutions}, and 1
copy of {\em Amelie} on 5 shelves?

\solution{This is the number of ways to arrange 64 $C$'s
(Cat in the Hat), 96 $M$'s (Matrix), 1 $A$'s (Amelie), and 4 $X$'s
(dividers between shelves).  This is equal to:

\[
\frac{(64 + 96 + 1 + 4)!}{64!\ 96!\ 1!\ 4!}
\]
}

\iffalse
\ppart I want two large pizzas.  Dominos Pizza offers 11 different
toppings.  At first, I thought this gave me $(2^{11})^2$ options.
What's wrong with this answer, and what is the right answer?

\solution{There are $2^{11}$ pizzas.  The number of
sequences consisting of two pizzas is indeed $(2^{11})^2$.  However,
this double-counts certain orders; for example, the two sequences:

\[
\begin{array}{c}
(\text{mushroom and pepperoni}, \text{pineapple}) \\
(\text{pineapple}, \text{mushroom and pepperoni})
\end{array}
\]

correspond to the same order.

There are $\binom{2^{11}}{2}$ ways to select two {\em different} kinds
of pizza, and $2^{11}$ ways to select two pizzas of the same kind.
Therefore, the number of possibilities is:

\[
\binom{2^{11}}{2} + 2^{11}
\]

Alternatively, we can analyze this in the same way that we considered
orders of donuts.  In this case, there is a bijection between pizza
orders and the number of bit-sequences with $2^{11} - 1$ ones and 2
zeros, which is:

\[
\binom{2^{11} + 1}{2}
\]
}

\ppart In No Limit Texas Hold 'Em poker, each player's hand consists
of five face-up cards and two face-down cards.  Suppose that these
five cards are dealt face up:

\[
\fbox{$A \spadesuit$} \quad
\fbox{$7 \diamondsuit$} \quad
\fbox{$Q \heartsuit$} \quad
\fbox{$7 \heartsuit$} \quad
\fbox{$2 \clubsuit$}
\]

In how many ways can the two face-down cards be chosen so as to form a
3-of-a-kind, but not a 4-of-a-kind?

\solution{There are several cases to consider.  We can get three aces
  in $\binom{3}{2} = 3$ ways, since the face-down cards could be any 2
  of the 3 remaining aces.  Similary, we can get three queens in 3
  ways and three 2's in 3 ways.  Otherwise, one of the cards must be a
  7 and the other must not.  We can chose the 7 in 2 ways and the
  remaining card in 45 ways.  All of these possibilities are disjoint,
  so this gives $3 + 3 + 3 + 2 \cdot 45$ pairs.}


\ppart Suppose that we leave two jokers--- a red one and a black one--- in
a deck of cards and treat them as a {\em wildcards}, which can serve as
any ordinary card.  How many different hands have a 4-of-a-kind, but no
5-of-a-kind?

\solution{There are three possibilities.

\begin{itemize}

\item We get a 4-of-a-kind with no wildcard.  This can happen in $13
\cdot 48$ ways.

\item We get a 3-of-a-kind and one wildcard.  This can happen in $13
\cdot \binom{13}{3} \cdot 48 \cdot 2$ ways.

\item We get a pair and both wildcards.  This can happen in $13 \cdot
\binom{13}{2} \cdot 48$ ways.

\end{itemize}

Thus, the total number of hand with a 4-of-a-kind is:

\[
13 \cdot 48 +
13 \cdot \binom{13}{3} \cdot 48 \cdot 2 +
13 \cdot \binom{13}{2} \cdot 48
\]
}
\fi

\ppart How many nonnegative integers less than 1,000,000 have exactly
one digit equal to 9 and have a sum of digits equal to 17?

\solution{There is a bijection with pairs:

\[
(\text{position of the 9}, \text{values of other 5 digits})
\]

The sum of the other 5 digits is equal to 8, so the number of ways to
choose their values is equal to the number of solutions over the
nonnegative integers to:

\begin{eqnarray*}
x_1 + x_2 + x_3 + x_4 + x_5 & = & 8
\end{eqnarray*}

This is equal to the number of binary sequences with 8 zeros and 4
ones, which is $\binom{12}{4}$.  Therefore, by the general product
rule, there are:

\[
6 \cdot \binom{12}{4}
\]

such integers.
}

\eparts

\end{problem}


\end{document}
