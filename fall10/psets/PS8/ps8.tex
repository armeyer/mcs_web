\documentclass[12pt,twoside]{article}   
\usepackage{light}

% \hidesolutions
\showsolutions

\begin{document}
\problemset{8}{October 28, 2010}{Tuesday, November 2 @ 7pm}



%%%%%%%%%%%%%%%%%%%%%%%%%%%%%%%%%%%%%%%%%%%%%%%%%%%%%%%%%%%%%%%%%%%%%%%%%%%%%%%
% Problem written Fall 2008

\begin{problem}{25}
Find $\Theta$ bounds for the following divide-and-conquer recurrences.
Assume $T(1) = 1$ in all cases.  Show your work.

\begin{problemparts}

\ppart{5} $T(n) = 8T(\floor{n/2}) + n$

\ppart{5} $T(n) = 2T(\floor{n/8} + 1/n)+n$

\ppart{5} $T(n) = 7T(\floor{n/20}) + 2T(\floor{n/8}) + n$

\ppart{5} $T(n) = 2T(\floor{n/4}+1)+n^{1/2}$

\ppart{5} $T(n) = 3T(\floor{n/9+n^{1/9}}) + 1$

\end{problemparts}

\end{problem}

\solution{
We use the method of Akra-Bazzi for these problems.

\begin{enumerate}[(a)]

\item
We see that $a = 8$, $b = 1/2$, $h = \floor{n/2} - n/2$ so $p=3$ gives $ab^p = 1$.  
\[
T(n) = \Theta(n^{3}(1+ \int_1^n \frac{u}{u^{4}} du)) = \Theta(n^{3}(1+ \int_1^n u^{-3} du)) 
= \Theta(n^3).
\]

\item
$a_1 = 2$, $b_1 = 1/8$, $h_1(n) = \floor{n/8} - n/8 +  1/n$, 
$g(n) = n$, $p = 1/3$,
\begin{align*}
T(n) &= \Theta\left(n^{p}\left(1+ \int_1^n \frac{g(u)}{u^{p+1}} du \right)\right)\\
&= \Theta\left(n^{1/3}\left(1+\int_1^n \frac{u}{u^{4/3}}\, du \right)\right)\\
&= \Theta\left(n^{1/3} + n^{1/3} \int_1^n u^{-1/3} \, du\right)\\
&= \Theta(n^{1/3} + n^{1/3} \frac{3}{2}(n^{2/3}-1))\\
&= \Theta(n).
\end{align*}

\item
$a_1 = 7$, $b_1 = 1/20$,$a_2 = 2$, $b_2 = 1/8$, $h_1(n) = \floor{n/20} - n/20$, $h_2(n) = \floor{n/8} - n/8$, and $g(n) = n$.  Finally, note that although we do not know what $p$ is, we are guaranteed
that $p<1$.
\begin{align*}
T(n) = \Theta(n^p(1+ \int_1^n \frac{u}{u^{p+1}} du)) &= \Theta(n^p(1+ \int_1^n u^{-p} du))\\
&= \Theta(n^p + n^p \frac{1}{1-p}(n^{1-p}-1))\\
&= \Theta(n).
\end{align*}

\item
$a_1 = 2$, $b_1 = 1/4$, $h_1(n) = \floor{n/4}-n/4 + 1$,
 $g(n) = n^{1/2}$, $p = 1/2$,
\[
T(n) = \Theta(n^{1/2}(1+ \int_1^n \frac{u^{1/2}}{u^{3/2}} du)) =
\Theta(n^{1/2} \log n).
\]

\item
$a_1 = 3$, $b_1 = 1/9$, $h_1(n) = \floor{n/9+ n^{1/9}} - n/9$, 
$g(n) = 1$, $p = 1/2$,
\[
T(n) = \Theta(n^{1/2}(1+ \int_1^n \frac{1}{u^{3/2}} du)) =
\Theta(n^{1/2}).
\]

\end{enumerate}
}

\begin{problem}{30}
It is easy to misuse induction when working with asymptotic notation.

\textbf{False Claim}  If

$$T(1) = 1 \textrm{ and}$$
$$T(n) = 4T(n/2) + n$$

Then T(n) = O(n).

\textbf{False Proof}
We show this by induction.  Let $P(n)$ be the proposition that $T(n) = O(n)$.  

\textbf{Base Case}:
$P(1)$ is true because $T(1) = 1 = O(1)$. 

\textbf{Inductive Case}: 
For $n \geq 1$, assume that $P(n-1), \ldots, P(1)$ are true.  We then have that
$$T(n) = 4T(n/2) + n = 4O(n/2) + n = O(n)$$

And we are done.

\bparts
	\ppart{5} Identify the flaw in the above proof.  
	
	\ppart{10} A simple attempt to prove $T(n) \neq O(n)$ via induction ultimately fails. We assume for sake of contradiction that $T(n) = O(n)$. Then there exists positive integer $n_0$ and positive real number $c$ such that for all $n \geq n_0$, $T(n) \leq cn$. We then define $P(n)$ as the proposition that $T(n) \leq cn$.

	We then proceed with strong induction.

	\textbf{Base Case, $n=n_0$}: $P(n_0)$ is true, by assumption.

	\textbf{Inductive Step}: We assume $P(n_0)$, $P(n_0 + 1)$, \ldots, $P(n-1)$ true.

	Fill in the rest of this proof attempt, and explain why it doesn't work.

	\textit{Note: As this problem was updated so late, the graders will be instructed to be exceedingly lenient when grading this.}
	
	\ppart{5} Using Akra-Bazzi theorem, find the correct asymptotic behavior of this recurrence.
	
	\ppart{10} We have now seen several recurrences of the form $T(n) = aT(\floor{n/b}) + n$.  Some of them give a runtime that is $O(n)$, and some don't.  Find the relationship
	between $a$ and $b$ that yields $T(n) = O(n)$, and prove that this is sufficient.
	
\eparts
\end{problem}
\solution{
\begin{enumerate}[(a)]
	\item 
		The flaw is that $P(n)$ is a predicate on $n$, whereas $O(n)$ is a statement not on $n$, but on the limit of $n$ as $n$ approaches infinity.  $T(n) = O(n)$ does not depend on the value of $n$ - it is either true or false.
	\item
         We first take some $n \geq 2n_0$. Then,
		$$T(n) = 4T(n/2) +n$$
		From the inductive hypothesis, $n/2 \geq n_0$, so $T(n/2) \leq cn/2$.  So this means that
		$$T(n) \leq 4cn/2 + n = 2cn + n = n(2c+1)$$
		Which is not less than $cn$.  So the induction is simply not powerful enough.
	\item
		We have that $p=2$, so $T = \Theta( n^2 ( 1 + \int_{1}^{n}(u/u^3)du)) = \Theta(n^2)$.
	\item
		From analyzing the integral we can see that any case where $p < 1$ will give a linear solution, so having the condition $a < b$ is sufficient.
	
\end{enumerate}

}
\begin{problem}{15}
Define the sequence of numbers $A_i$ by

$A_0=2$

$A_{n+1}=A_n/2 + 1/A_n$ (for $n \geq 1$)

Prove that $A_n\leq \sqrt{2}+1/2^n$ for all $n\geq 0$.

\solution{
\begin{proof}
The proof is by induction on $n$.  Let $P(n)$ be the proposition that $A_n \leq \sqrt{2} + 1 / 2^n$.

{\bf Base case:}  
$A_0=2\leq \sqrt{2}+1/2^0$ is true. 

{\bf Inductive step:}
Let $n\geq 0$ and assume the inductive hypothesis $A_n\leq \sqrt{2}+1/2^n$.
We need the following lemma.

\begin{lemma*} For real numbers $x>0$, $x/2+1/x\geq \sqrt{2}$.
\end{lemma*}

\begin{proof}
For real numbers $x>0$,
\begin{eqnarray*}
&& x/2+1/x\geq \sqrt{2} \\
&\Leftrightarrow & x^2+2\geq 2\sqrt{2}\cdot x \\
&\Leftrightarrow & x^2-2\sqrt{2} \cdot x+2 \geq 0 \\
&\Leftrightarrow & (x-\sqrt{2})^2 \geq 0,
\end{eqnarray*}
which is true.
\end{proof}

By using induction it is straightforward to prove that
$A_n>0$ for $n\geq 0$ (base case: $A_0=2>0$; inductive step: if $A_n>0$, then $A_{n+1}=A_n/2 + 1/A_n>0$). By the lemma,
$A_n\geq \sqrt{2}$ for $n\geq 0$.
Together with the inductive hypothesis
this can be used in the following derivation:
\begin{eqnarray*}
A_{n+1} &=& A_n/2 + 1/A_n \\
&\leq & (\sqrt{2}+1/2^n)/2+1/\sqrt{2}\\
&=& \sqrt{2}+1/2^{n+1}.
\end{eqnarray*}

This completes the proof.
\end{proof}
}

\end{problem}

%%%%%%%%%%%%%%%%%%%%%%%%%%%%%%%%%%%%%%%%%%%%%%%%%%%%%%%%%%%%%%%%%%%%%%%%%%%%%%%
% Based on problem from Fall 2006, pset 8 (I changed some of the numbers)

\begin{problem}{30}
Find closed-form solutions to the following linear recurrences.

\begin{problemparts}

\ppart{15} 
$x_{n} = 4x_{n-1} - x_{n-2} - 6x_{n-3} \quad
	(x_0 = 3, x_1 = 4, x_2 = 14)$

\solution{The characteristic equation is $r^3 - 4r^2 + r + 6 = 0$.

Generally, solving a cubic equation is a difficult problem.  However, we can find from inspection that the
roots are:
\begin{align*}
r_1 & = -1 \\
r_2 & = 2 \\
r_3 & = 3
\end{align*}

Therefore a general form for a solution is
\[
x_n = A (-1)^n + B (2)^n + C (3)^n.
\]

Substituting the initial conditions into this general form gives a
system of linear equations.
\begin{eqnarray*}
3 & = & A + B + C \\
4 & = & -A + 2B + 3C \\
14 & = & A + 4B + 9C
\end{eqnarray*}

The solution to this linear system is $A = 1$, $B = 1$, and
$C = 1$.  The complete solution to the recurrence is therefore
\[
x_n  =  (-1)^{n} + 2^{n} + 3^{n}.
\]
}

\ppart{15} $x_{n} = -x_{n-1} + 2 x_{n-2} + n \quad (x_0 = 5, x_1 = -4/9)$

\solution{
First, we find the general solution to the homogenous recurrence.  The
characteristic equation is $r^2 + r - 2 = 0$.  The roots of this
equation are $r_1 = 1$ and $r_2 = -2$.  Therefore, the general solution
to the homogenous recurrence is

\begin{eqnarray*}
x_n & = & A(-1)^n + B 2^n.
\end{eqnarray*}

Now we must find a particular solution to the recurrence, ignoring the
boundary conditions.  Since the inhomogenous term is linear, we guess
there is a linear solution, that is, a solution of the form $an + b$.  If
the solution is of this form, we must have
\[
an+b = -a(n-1) - b + 2a(n-2) + 2b + n
\]
Gathering up like terms, this simplifies to
\[
n(a + a - 2a - 1) + (b + a + b + 4a - 2b) = 0
\]
which implies that
\[
n = -5a
\]

But $a$ is a constant, so this cannot be so.  
So we make another guess, this time that there is a quadratic solution of
the form $an^2 + bn +c$.  If the solution is of this form, we must have
\[
an^2 + bn + c = -[a(n-1)^2 + b(n-1) + c] + 2[a(n-2)^2 + b(n-2) + c] + n
\]
which simplifies to
\[
n^2(a + a - 2a) + n(b + b - 2a + 8a - 2b - 1) + (c + a - b + c - 8a + 4b - 2c) = 0
\]

This simplifies to

\[
	n(6a - 1) + (-7a + 3b) = 0
\]
This last equation is satisfied only if the coefficient of $n$ and the
constant term are both zero, which implies $a = 1/6$ and $b = 7/18$.
Apparently, any value of $c$ gives a valid particular solution.  For
simplicity, we choose $c = 0$ and obtain the particular solution:
\[
x_n = \frac{1}{6}n^2 - \frac{7}{18} n.
\]

The complete solution to the recurrence is the homogenous solution
plus the particular solution:
\[
x_n = A(-1)^n + B 2^n + \frac{1}{6}n^2 - \frac{7}{18} n
\]
Substituting the initial conditions gives a system of linear equations:
\begin{eqnarray*}
5	& = &	A + B \\
-4/9	& = &	-A + 2 B - + 1/6 + 7/18
\end{eqnarray*}

The solution to this linear system is $A = 3$ and $B = 2$.
Therefore, the complete solution to the recurrence is
\[
x_n =3+  2(-2)^{n} + \frac{1}{6}n^2 + \frac{7}{18} n 
\]
}

\end{problemparts}

\end{problem}

\end{document}


