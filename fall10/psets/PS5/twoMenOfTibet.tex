\documentclass{article}
\usepackage{light}

\showsolutions

\begin{document}
\problemset{5}{October 5, 2010}{Monday, October 11}
Readings: Section 5.4 to 5.7 and 6.1-6.2.

\begin{problem}{0}
Two holy men live on opposite sides of a very irregular mountain, and they would like to meet.  There are no plateaus in these mountains. However tradition dictates that both men being holy, they must always keep at the same height.  So it is not possible for one of them to wait, and have the other one cross the mountains. 

 In this problem, we use graph theory to model this situation and to reach the surprising conclusion that, given some restrictions on the mountain, it is always possible.

\ppart{1}
Show that if there is a depression in the mountains that dips below ground level, they may not be able to meet.

\solution{ Show drawing of  an 'N'  shaped mountain, The two monks would never be able to get started.}


For the rest of this problem we will assume that once in the mountain, the height never drops below the ground level.

\ppart{1}
Prove that if it is possible for them to meet, and there is a single highest peak in the mountains, then they must initially meet at the peak.

\solution{
Assume by way of contradiction that they can meet anywhere at point Q, not P, for the first time. Assume, without loss of generality, that the Q lies to the left of P. the the monk starting at the B must have crossed P while the monk starting at A was in lower ground.
}

\ppart{1}
Consider an M shaped mountain. Describe how the two monks can meet. What happens when we keep the M shape but make the sides have different inclinations, or a single side changes inclination along the way.

\solution{
Let them walk at the same pace up and down the mountains. (editing note: this part is mostly so that they work through this case, it won't be worth any points)
}

\ppart{6}
The previous suggests that whether they can meet or not only depends on a few critical points of the shape: the local maxima and minima.  We will model the problem in the following way.

For every local maximum and minimum, we will consider all the points in the mountain at that level (on the same horizontal). As the picture shows (todo: insert picture). These are the only points that matter (can you see why?). Now let \( S \) be the set of all those points. Consider the set \( S \times S \). Intuitively this will describe the positions of monk A and monk B respectively. This set has many illegal states, so we will use a more constrained subset of it:

\(GS =  \{ (x,y) \mbox{ such that } (x,y) \in S \times S, \mbox{ and } x \mbox{ and } y \mbox{ are at the same level in the mountains } \} \)

We will think of these states a vertices int a graph. \( V = GS  \).  
And we will make our set of edges encode whether a position is reachable directly from another. State \( s_1 = \left(a_1, b_1 \right) \) is connected to state \(s_2 = \left(a_2, b_2 \right) \) if and only if $a_1$ is adjacent (in the mountain) to  $b_1$ are adjacent, and $a_2$ is also adjacent to $b_2$.

Show that vertices in this graph can only have degree 0, 1, 2 and 4, and describe which vertices have degree 1.

\solution{
Reason by cases. A particular point in the graph corrsponds to two positions on the original mountain at the same level. These are the visual cases
 
$\mid$ $\mid$: 2 edges out (both move  up or both move  down)

V  $\mid$ or $\mid$ V: 2 edges out (the V one can move to left or right, the $\mid$ only up).

V V:  4 edges out (each picks left or right)

$\wedge$  $\mid$: 2 edges out.

$\wedge$ $\wedge$: 4 edges out.

$\wedge$  V: 0 edges out;

Lastly, the only degree one cases involve  the starting points. There are no other cases. (Remember we allow no plateaus). The states involved are (A,B), (B,A), (A,A), and (B,B). We start in (A,B).

 }

\ppart{4}
Finally, show that there is a path from the inital node $s_0 = \left(A, B\right)$, to some other node of degree 1, and using this prove that the two holy men can always meet.

\solution{By the Handshake lemma, a node of degree one cannot be alone in the same connected component. This implies that (A,B) is connected to (A,A), (B,B) or (B,A). If (A,A) or (B,B) is reachable, then they have clearly met (they are at the same node). Otherwise, If the two can switch places, they must have met somewhere in the middle.}

\end{problem}
\end{document}
