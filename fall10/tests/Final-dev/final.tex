\documentclass[12pt,oneside]{article}
\usepackage{light}
\usepackage{multicol}
\usepackage{pifont} % for the star
\usepackage{palatino}
\usepackage{mathpazo}
\usepackage{verbatim}

\newcommand{\mfigure}[3]{\bigskip\centerline{\resizebox{#1}{#2}{\includegraphics{#3}}}\bigskip}
\newcommand{\hint}[1]{({\it Hint: #1})}
\newcommand{\brule}[1]{\underline{\hspace{#1}}}
\newcommand{\ang}[1]{\left< #1 \right>}
\newcommand{\beats}{\rightarrow}


\newenvironment{falseproof}
{\begin{proof}[False proof]}
{\end{proof}}

%\showsolutions
\hidesolutions

\begin{document}
\generic{Final}{December 14, 2010}

\instatements{
\vspace{12pt}
\textbf{Name:} \rule{5in}{0.5pt}

\textbf{Circle the name of your recitation instructor}:

\begin{center}
\begin{tabular}{llllll}
David & Darren & Martyna & Nick & Oscar & Stav 
\end{tabular}
\end{center}

\begin{itemize}

\item This quiz is \textbf{closed book}, but you may have one $8.5
\times 11$'' sheet with notes in your own handwriting on both sides.

\item Calculators are not allowed.

\item You may assume all of the results presented in class.

\item Please show your work.  Partial credit cannot be given for a wrong
answer if your work isn't shown.

\item Write your solutions in the space provided.  If you
need more space, write on the back of the sheet containing the
problem.  Please keep your entire answer to a problem on that
problem's page.

\item Be neat and write legibly.  You will be graded not only on the
correctness of your answers, but also on the clarity with which you
express them.

\item If you get stuck on a problem, move on to others. The problems 
are not arranged in order of difficulty.

\item The exam ends at 4:30 PM.

\end{itemize}

\vspace{0.25in}

\begin{center}
{\large
\begin{tabular}{|c|c|c|c|}
\hline
Problem & Points & Grade & Grader \\ \hline \hline
1 & 10 & & \\ \hline
2 & 10 & & \\ \hline
3 & 20 & & \\ \hline
4 & 15 & & \\ \hline
5 & 20 & & \\ \hline
6 & 25 & & \\ \hline
7 & 10 & & \\ \hline
8 & 10 & & \\ \hline
Total & 120 & & \\ \hline
\end{tabular}
}
\end{center}
}
\instatements{\newpage}


\begin{problem}{10}
Outline of problem:

Induction: 

part 1: show, as in recitation, that for fibonacci sequence $f_n$ and $f_{n+1}$ are rel prime, using the fact that two numbers a,b rel prime iff sa+tb =1 for some s and t

part 2: prove that $f_n$ and $f_{n+2}$ are rel prime

part 3: prove that $f_n$ and $f_{n+3 }$ are not relatively prime 
\solution{
solution
}
\end{problem}

\newpage

\begin{problem}{10}
Monty hall variation, obtained from:

 http://answers.yahoo.com/question/index?qid=20080402012832AARhVQM

If you have selected the door that hides the prize, the host will always open one of the remaining doors (and he chooses among these doors at random, i.e. he is equally likely to open either door) and then will always offer you the possibility to switch doors. Otherwise, with probability 0.5 the host will open immediately the door that you have selected (revealing that you have lost) and with probability 0.5 he will open the other door that does not hide the prize. In the latter case he will always give you the opportunity to switch doors. After your decision on whether to switch, the door that you have selected in the end is opened to see whether the prize is hidden behind it, in which case you will win the prize.
What is the probability of winning if:
(b) Initially you select door 3 and you always switch if you are offered the opportunity to switch.
(c) Initially you select door 3 and if the host opens door 2 you switch otherwise you stick with door 3.

\end{problem}

\newpage
\begin{problem}{10}
Monty hall variation, obtained from:

http://scienceforums.com/topic/9802-the-four-card-trick-a-variation-on-monty-hall/

There are four cards face down - two are Jacks, two are Queens. There is a host present who knows what each card is. 

1.You choose one of the cards.

2a.You instruct the host to eliminate either a Jack or a Queen. It is your decision. You can flip a coin if you wish. He will follow your instructions and turn over one of the other three cards accordingly.

2b. Alternatively, the host is not involved. You simply turn over one of the cards yourself. 

3. Whichever way it happened, let us say a Jack was eliminated.

4. At this point, you make a new selection. You can stick to your original card or choose one of the other two.

5. Finally, another card is eliminated. This time it must be the host who turns it over. He will show the opposite of the last so as not to spoil the game. In this case, he turns over a Queen.

There is now one Jack and and one Queen remaining. Your card is one of them.

Questions: Which is more likely to be the Queen? Are the odds 50/50? Does it make a difference whether you stuck or swapped? Does it make a difference whether you used the host for the first elimination?

\end{problem}

\newpage
\begin{problem}{10}

From: http://www.cat4mba.com/forum/math-o-mania/tricky-probability-questions

A natural number a is chosen at random from the first one hundred natural numbers. What is the probability that $a + 100/a > 50$?
\end{problem}

\newpage
\begin{problem}{10}
\\Invariant problems: http://www.cs.nott.ac.uk/~jff/G51APS/exercises/InvariantExercises.pdf

Induction problems: http://www.cs.nott.ac.uk/~jff/G51APS/exercises/InductionExercises.pdf
\end{problem}

\newpage
\begin{problem}{20}
\bparts
\ppart{5}
A chessboard is covered with tetris blocks (each covers 4 fields). Prove that there is an even number of black fields on that chessboard.
\solution{Number of fields divisible by $4$. Two cases: both sides even length or one side with length divisible by $4$.}
\solution{By contradiction. If both sides odd, then total number fields odd, but it must be divisible by 4, so contradiction.}
\ppart{15}
Now consider a specific case when all the tetris blocks used are of the shape in the figure below. Prove that the number of fields on the chessboard is divisible by $8$.
\solution{(Sketch) Each block covers $3$ black and $1$ white or $1$ black and $3$ white fields. Show by induction that an odd number of blocks covers an odd number of black fields. Number of black fields is even, so need an even number of blocks. Each of them is of size $4$, so they cover a multiple of $8$ fields.}
\eparts
\end{problem}

\newpage
\begin{problem}{20}
In the far off land of Spain two teams: Barcelona and Madrid have been battling each other for centuries to see who is more awesome. They play two games a year: one in the Spring and one in the Fall. The Madrid team has a tendency to fire their coaches very often to try to improve their results. The outcomes of the games are as follows:
\begin{enumerate}
\item
If Madrid has not just fired their coach:
\begin{itemize}
\item
Barcelona wins with probability $\frac25$
\item
Madrid wins with probability $\frac25$
\item
They tie with probability $\frac15$
\end{itemize}
\item
If Madrid has just fired their coach (they haven't yet realized that it's a bad idea):
\begin{itemize}
\item
Barcelona wins with probability $\frac35$
\item
Madrid wins with probability $\frac15$
\item
They tie with probability $\frac15$
\end{itemize}
\end{enumerate}
Now, Madrid does not fire their coach if they win or tie, but if Barcelona wins, they will fire their coach with $90\%$ probability.
For the following two questions, assume that Madrid did not fire their coach before the Spring 2010 game.

\bparts
\ppart{10}
Given that Madrid lost the Fall 2010 game, what is the probability that they fired their coach between the Spring and Fall games this year.
\ppart{10}
What is the probability that Madrid fired their coach between the Spring and Fall games this year given that they lost BOTH of these games?
\eparts
\end{problem}

\newpage
\begin{problem}{25}
\bparts
%\ppart{5}
%You repeatedly roll a fair die. What is the probability that you will roll a $5$ before you roll a $6$?
\ppart{5}
You roll two fair dice and look at the sum. What is the expected value of that sum? 
\ppart{5}
What is the variance?
\ppart{5}
You repeatedly roll two fair dice and look at the sum. What is the probability that you will roll a sum of $4$ before you roll a sum of $7$?
\ppart{5}
What is the expected number of rolls before you get a sum of $4$ or a sum of $7$?
\ppart{5}
What is the probability that you will roll a sum of $4$ and a sum of $10$ before you roll a sum of $7$?
\ppart{5}
What is the expected number of rolls before either you get a sum of $7$ or both a sum of $4$ and a sum of $10$ appear?
\eparts
\end{problem}

\newpage
\begin{problem}{20}
In class, we have shown that the congestion of a butterfly network is $\sqrt{N}$. In this problem, we will analyze the expected number of packages going through a specific switch in the network. Recall that if you look at the switch in the $i-th$ layer of the network and represent it's position in that layer as a binary number $(y_1y_2\ldots y_ix_{i+1}x_{i+1}\ldots x_{n})$ where $N=2^n$, then the packets going through that switch are the ones where the output starts with $y_1y_2\ldots y_i$ in binary AND the input ends with $x_{i+1}x_{i+1}\ldots x_{n}$ in binary.
Assume you are given a random assignment of outputs to inputs. How many packets do you expect to go through a switch in the $i-th$ layer of the network?
\solution{
Let $X_j$ be the indicator random variable that the packet from input $j$ goes through our switch and let $E_j$ be the indicator variable that $j$ ends with  $x_{i+1}x_{i+1}\ldots x_{n}$ in binary representation. We have two cases.
\begin{enumerate}
\item
$Pr(X_j=0|E_j=0)=1$.
\item
$Pr(X_j=1|E_j=1)=\frac{1}{2^x}$ and $Pr(X_j=0|E_j=1) = 1-frac{1}{2^x}.
\end{itemize}
Therefore $Ex(X_j)=\frac{1}{2^x} \cdot E_j.
Consequently,
\begin{eqnarray*}
Ex(number of packets through switch) &=& \sum_{j=0}^{N-1}Ex(X_j)\\
& = &\sum_{j=0}^{N-1}\frac{1}{2^x} \cdot E_j\\
& = &\frac{1}{2^x}\sum_{j=0}^{N-1}E_j\\
& = &\frac{1}{2^x} \cdot (number of inputs that start with $x_{i+1}x_{i+1}\ldots x_{n}$ in binary representation)\\
& = &\frac{1}{2^x} \cdot 2^x\\
& = & 1.
\end{eqnarray*}
}
\end{problem}

\newpage
\begin{problem}{20}
In the card game of bridge, you are dealt a hand of $13$ cards from the standard $52$-card deck.
\bparts
\ppart{5}
A balanced hand is one in which a player has roughly the same number of cards in each suit. How many different hands are there where the player has $4$ cards in one suit and $3$ cards in each of the other suits?
\solution{There are $4$ suits to pick from for the longest suit, $4$ cards out of $13$ to choose from in that suit, and $3$ cards out of $13$ to choose from in each of the remaining $3$ suits.
This gives
\[4\cdot\binom{13}4\cdot\binom{13}{3} \cdot\binom{13}{3}\cdot\binom{13}{3}\]
such hands.}
\ppart{5}
Not surprisingly, a non-balanced hand is one in which a player has more cards in some suits than others. Hands that are very disired are ones where over half the cards are in one suit. How many different hands are there where there is exactly $7$ cards in one suit?
\solution{
There are $4$ suits to choose from for the long suit, $7$ cards out of $13$ to choose in that suit, and $6$ cards out of the remaining $39$ in the other $3$ suits. This gives
\[4\cdot\binom{13}{7}\cdot\binom{39}6\]
such hands.
}
\ppart{5} In how many ways can $3n$ students be broken up into 
$n$ groups of 3?  

\solution{
\[
\frac{(3n)!}{(3!)^n n!}.
\]
}
\ppart{5}
How many n-digit PIN numbers are there where no $2$ consecutive digits are the same?
\solution{There are $10$ choices for the first digit, and $9$ choices for each of the remaining $n-1$ digits, since you can choose any digit that is not the same as the one right in front of it, so there are
\[10\cdot9^{n-1}\]
such PIN numbers.}
\eparts
\end{problem}

\newpage
\begin{problem}{10}
Show that
\[\sum_{i=0}^n\sum_{j=0}^{n-i}\frac{n!}{i!j!(n-i-j)!} = 3^n\]
\solution{
Number of sequences of length $n$ composed of digits $0$,$1$, and $2$.
}
\end{problem}

\newpage
\begin{problem}{15}
Show that for any $m,n,k$ such that $k<m<n-k$, the following identity holds:
\[ \binom{n}{k} = \sum_{i=0}^{k} {\binom{m}{i}} \cdot {\binom{n-m}{k-i}}\]
\solution{
Select $k$ people from a group of $n$ by splitting the people into two groups of $m$ and $n-m$, each with at least $k$ people, and then select $i$ from first group and $k-i$ from second group.
}
\end{problem}


\newpage
\begin{problem}{10}
\bparts
\ppart{5}

A man and his five sons bought a 20-piece bucket of chicken from Olive Oyl's.  

A man and his five sons are fencing off an area into a six-sided region.  There is exactly enough material for $20$ meters of fence, and the material comes in 1m segments.  
If each builder claims a side of the fence (and each builder lays at least 1m of fence), how many ways can the border be built?

\ppart{5}

Now, suppose the father says that he must use at least $5$ meters of fencing.  Now, how many ways can things be built?

\ppart{5}

Across town, the six symmetry brothers are doing the same thing with $24$ meters of fence.  They pair up in a way where the brothers opposie each other use the same amount
of fencing.  How many ways are there to build this enclosure?

\eparts
\end{problem}

\newpage
\begin{problem}{15}
We are going to play a series of fun coin-flipping games!  The core idea is to keep a finite sequence of coinflips in your head, and flip until you get your sequence!

\bparts

\ppart{5}
Let's say your sequence is just H.  What is the expected number of flips until the game ends?

\ppart{5}
Now say that your sequence is HH.  What is the expected number of flips until the game ends?

\ppart{5}
Say you are playing a game against Professor Leighton.  You choose sequence HHH, and Professor Leighton chooses sequence HTT.  You flip the same coin until one of your sequences
appear, and the player whose sequence it is wins.  Who is expected to win in this game?

\eparts

\end{problem}

\newpage

\begin{problem}{10}

Find a closed-form solution to the following recurrence: 

$x_n = 11x_{n-1} - 30x_{n-2}$ where $(x_0 =4, x_1 = 23)$
\end{problem}

\newpage
\begin{problem}{10}
Let G be a bipartite graph with $n$ nodes and $k$ components. You independently color each of the nodes of $G$ red or black with equal probabilities. What is the probability that your coloring is a valid $2$-coloring of $G$?
\solution
{There are $2$ ways to correctly color each component, and there are $k$ components, so there is $2^k$ valid colorings. The number of colorings you might produce is $2^n$ and each is equallly likely. The probability that you get a valid coloring is 
\[\frac{2^k}{2^n} = \frac{2^{k-n}}\]
}
\end{problem}

\newpage
\begin{problem}{10}
Graph G is built as follows: first, a complete graph $G_n$ with $n$ vertices is created, then a complete graph $G_m$ is created. After this, a random node $u$ from $G_n$ and a random node $v$ from $G_m$ are selected and connected by an edge. This is where you come in. You select $k$ random nodes from $G$ ($k<n$ and $k<m$) and connect them by all the edges between them that are in $G$. What is the probability that the graph you created has only one connected component?
\solution{
The ways to select a connected graph are:
\begin{enumerate}
\item 
Select nodes $u$ and $v$ and then any $k-2$ of the remaining $m+n-2$ nodes. There are \[\choose{m+n-2}{k-2}\] ways of doing this.
\item
Do not select both $u$ and $v$. Then the only way to get a connected graph is to have all nodes either from $G_m$ or $G_n$. That gives us
\[\binom{m}{k}+\binom{n}{k}\]
ways.
\end{enumerate}
The total number of ways to select $k$ nodes is \[\binom{m+n}{k}.\]
This means the probability of getting a connected graph is
\[\frac{\binom{m+n-2}{k-2}+\binom{m}{k}+\binom{n}{k}}{\binom{m+n}{k}}.\]
}
\end{problem}
\newpage

\begin{problem} %%oscar, taken from 06 final. need to modify/ reformulate
{\bf [10 points]}
Two identical complete decks of cards, each with 52 cards, have been mixed together. A hand of 5 cards is picked uniformly at random from amongst all subsets of exactly 5 cards.
\bparts
\ppart {\bf [5 points]} What is the probability that the hand has no identical cards (i.e., cards with the same suit and value. For example, the hand {$\left<Q\heartsuit, 5\spadesuit, 6\spadesuit, 8\clubsuit, Q\heartsuit\right>$} {\em has} identical cards.)?
\vspace{.15in}
We can calculate this probability by computing
%
\[
\frac{\text{hands with no identical cards}}{\text{total possible hands}}
\]
%
There are 104 cards.  There are 5 cards in a hand.  Order does not matter.  The total number of possible hands is:

$\binom{104}{5}$

There are 52 possible card faces, and we can choose 5 of them if no identical cards are allowed.  Additionally, each card can be from either deck 1 or deck 2.  Therefore the number of hands with no identical cards, chosen from 2 decks is:
%
\[
\binom{52}{5} \cdot 2^5
\]
%

Therefore the probability of drawing a hand with no identical cards is:

%
\[
\frac{\binom{52}{5} \cdot 2^5}{\binom{104}{5}}
\]
%

\ppart {\bf [5 points]} What is the probability that the hand has exactly one pair of identical cards?
\vspace{.15in}
This can be solved by a similar approach.  A hand of this type can be distinguished by the face (suit and value) of the repeated card, and by the faces of the 3 non-repeated cards.  There are 52 possible values for the face of the repeated card.  There are $\binom{51}{3}$ possible faces for the non-repeated cards, since none of these can be repeated.  Each of these could come from either the 1st deck or the 2nd deck.  There are  $\binom{104}{5}$ possible hands, as before.  So the probability of getting a hand with exactly one pair of identical cards is:

%
\[
\frac{52 \cdot \binom{51}{3} \cdot 2^3}{\binom{104}{5}}
\]
%

\eparts
\end{problem}

\newpage

\newpage
\begin{center}
(Notes)
\end{center}

\newpage

\mbox{}

\newpage

\mbox{}

\end{document}
