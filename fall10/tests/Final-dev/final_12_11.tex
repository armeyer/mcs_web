\documentclass[12pt,oneside]{article}
\usepackage{light}
\usepackage{multicol}
\usepackage{pifont} % for the star
\usepackage{palatino}
\usepackage{mathpazo}
\usepackage{verbatim}

\newcommand{\mfigure}[3]{\bigskip\centerline{\resizebox{#1}{#2}{\includegraphics{#3}}}\bigskip}
\newcommand{\hint}[1]{({\it Hint: #1})}
\newcommand{\brule}[1]{\underline{\hspace{#1}}}
\newcommand{\ang}[1]{\left< #1 \right>}
\newcommand{\beats}{\rightarrow}


\newenvironment{falseproof}
{\begin{proof}[False proof]}
{\end{proof}}

%\showsolutions
\hidesolutions

\begin{document}
\generic{Final}{December 14, 2010}

\instatements{
\vspace{8pt}
\textbf{Name:} \rule{5in}{0.5pt}

\textbf{Circle the name of your recitation instructor}:

\begin{center}
\begin{tabular}{llllll}
David & Darren & Martyna & Nick & Oscar & Stav 
\end{tabular}
\end{center}

\begin{itemize}

\item This quiz is \textbf{closed book}, but you may have two $8.5\text{''}
\times 11$'' sheets with notes in your own handwriting on both sides.

\item Calculators are not allowed.

\item You may assume all of the results presented in class.

\item Please show your work.  Partial credit cannot be given for a wrong
answer if your work isn't shown.

\item Write your solutions in the space provided.  If you
need more space, write on the back of the sheet containing the
problem.  Please keep your entire answer to a problem on that
problem's page.

\item Be neat and write legibly.  You will be graded not only on the
correctness of your answers, but also on the clarity with which you
express them.

\item If you get stuck on a problem, move on to others. The problems 
are not arranged in order of difficulty.

\item The exam ends at 4:30 PM.

\end{itemize}

\vspace{0.10in}

\begin{center}
{\large
\begin{tabular}{|c|c|c|c|c|c|c|c|}
\hline
Problem & Points & Grade & Grader & Problem & Points & Grade & Grader \\ \hline \hline
1 & 20 & & & 8 & 20 & & \\ \hline
2 & 12 & & & 9 & 11 & & \\ \hline
3 & 10 & & & 10 & 10 & & \\ \hline
4 & 10 & & & 11 & 10 & & \\ \hline
5 & 15 & & & 12 & 32 & & \\ \hline
6 & 10 & & & 13 & 10 & & \\ \hline
7 & 10 & & & total & 180 & & \\ \hline
%7 & 20 & & \\ \hline
%8 & 11 & & \\ \hline
%9 & 20 & & \\ \hline
%10 & 10 & & \\ \hline
%11 & 32 & & \\ \hline
%12 & 10 & & \\ \hline
\end{tabular}
}
\end{center}
}
\instatements{\newpage}

%%%%%%%%%%%%%%%%%%%%%%%%%%%%%%%%%%%%%%%%%%%%%%%%%%%%%%%%%%%%%%%%%%%%%%%%%%%%%%%%
%Problem 1
\begin{problem}{20}

We define the following recurrence for $n \geq 0$: 

$$T_{n+2} = T_{n+1} + 2T_{n}$$

where $T_0 = T_1 = 1$.

\bparts
	\ppart{8} Prove by induction that $T_n$ is odd for $n \geq 0$.  You do not need to solve the recurrence for this.
	\solution{We use induction on $n$.  Let $P(n)$ be the proposition that $T_n$ is odd. 

\noindent \textit{Base case:} $P(0)$ is true because $T_0 =1$, an odd number. 

\noindent \textit{Inductive step:} Assum that $ P(n)$ is true where $n \geq 0$; that is, $T_n$ is odd.  We must show that $T_{n+1}$ is odd. But we know that $T_{n+1} = T_{n} + 2T_{n-1}$. And by our induction hypothesis $P(n)$ is true, so $T_{n}$ is odd. But we now have that $T_{n+1}$ is a sum of an odd number, $T_{n}$ and an even number $2T_{n-1}$, and thus must be odd as well.

The theorem follows by induction.}

	\newpage
	
	\ppart{12} Prove by induction that $\gcd(T_{n+1}, T_{n}) = 1$ for $n \geq 0$.  You may assume that $T_n$ is odd for all $n$.  You do not need to solve the recurrence for this.
	\solution{We use induction on $n$.  Let $P(n)$ be the proposition that $\gcd(T_{n+1}, T_{n}) = 1$.

\noindent \textit{Base case:} $P(0)$ is true because $T_0 = 1$ and $T_1 = 1$
are relatively prime.

\noindent \textit{Inductive step:} Assume that $P(n)$ is true where $n
\geq 0$; that is, $\gcd(T_{n+1}, T_{n})=1 $.  We must
show that $\gcd(T_{n+2}, T_{n+1})=1 $. Assume not, then $T_{n+2}$ and $T_{n+1}$  have a common divisor $d > 1$. That is $d|T_{n+2}$ and $d|T_{n+1}$. Since all $T_n$ are odd, we know that $d$ cannot be even. But then $d$ must also divide the linear combination $\frac{T_{n+2} -T_{n+1}}{2} = T_n$, showing that $gcd(T_{n+1},T_{n})=d$, contradicting the assumption that $T_n$ and $T_{n+1}$ are relatively prime.  So $\gcd(T_{n+2}, T_{n+1})=1$, as desired.

The theorem follows by induction.}

\eparts

\end{problem}
\newpage

%%%%%%%%%%%%%%%%%%%%%%%%%%%%%%%%%%%%%%%%%%%%%%%%%%%%%%%%%%%%%%%%%%%%%%%%%%%%%%%%
%Problem 2
\begin{problem}{12}

Find a closed-form solution to the following recurrence: 
\begin{eqnarray*}
x_0 &=& 4  \\
x_1 &=& 23  \\
x_n &=& 11x_{n-1} - 30x_{n-2} \text{ for } n \geq 2.
\end{eqnarray*}
\solution{The characteristic equation is $r^2 - 11r -30 = 0$.

Factoring this out we get $(r-5)(r-6)$, so our roots are
\begin{align*}
r_1 & = 5 \\
r_2 & = 6 \\
\end{align*}
Therefore a general form for a solution is
\[
x_n = A (5)^n + B (6)^n
\]
Substituting the initial conditions into this general form gives a
system of linear equations.
\begin{eqnarray*}
4 & = & A + B \\
23 & = & 5A + 6B
\end{eqnarray*}

The solution to this linear system is $A = 1$ and $B = 3$.  The complete solution to the recurrence is therefore
\[
x_n  =  5^n+3*6^n
\]
}
\end{problem}

\newpage

%%%%%%%%%%%%%%%%%%%%%%%%%%%%%%%%%%%%%%%%%%%%%%%%%%%%%%%%%%%%%%%%%%%%%%%%%%%%%%%%
%Problem 3

\begin{problem}{10}
Note: in this question, you may use ``choose'' notation or factorials in your answers for both (a) and (b).  In the card game of bridge, you are dealt a hand of $13$ cards from the standard $52$-card deck.
\bparts

\ppart{5}
A balanced hand is one in which a player has roughly the same number of cards in each suit. How many different hands are there where the player has $4$ cards in one suit and $3$ cards in each of the other suits?
\solution{
There are $4$ suits to pick from for the longest suit, $4$ cards out of $13$ to choose from in that suit, and $3$ cards out of $13$ to choose from in each of the remaining $3$ suits.
This gives
\[4\cdot\binom{13}{4}\cdot\binom{13}{3} \cdot\binom{13}{3}\cdot\binom{13}{3}\]
such hands.}
\vspace{4 in}

\ppart{5}
Not surprisingly, a non-balanced hand is one in which a player has more cards in some suits than others. Hands that are very desired are ones where over half the cards are in one suit. How many different hands are there where there are exactly $7$ cards in one suit?
\solution{
There are $4$ suits to choose from for the long suit, $7$ cards out of $13$ to choose in that suit, and $6$ cards out of the remaining $39$ in the other $3$ suits. This gives
\[4\cdot\binom{13}{7}\cdot\binom{39}{6}\]
such hands.
}
\eparts
\end{problem}

\newpage
%%%%%%%%%%%%%%%%%%%%%%%%%%%%%%%%%%%%%%%%%%%%%%%%%%%%%%%%%%%%%%%%%%%%%%%%%%%%%%%%
%Problem 4
\begin{problem}{10}
Three pairs of twins are sharing a bucket of 24 pieces of chicken from Olive Oyl's.  Through the magic of twinship, each person always eats exactly as many pieces of chicken as his or her twin.  If each person eats at least one piece of chicken, how many ways can the chicken pieces be distributed amongst this family?  For the purposes of this problem, chicken pieces are indistinguishable, and every piece is eaten.  
\end{problem}
\newpage
%%%%%%%%%%%%%%%%%%%%%%%%%%%%%%%%%%%%%%%%%%%%%%%%%%%%%%%%%%%%%%%%%%%%%%%%%%%%%%%%
%Problem 5

\begin{problem}{15}
A family of seven folk eat a meal at Kansas Flightless Chicken that comes with unlimited free biscuits (the restaurant owner has deep pockets).  Being a caring family, they \textbf{only} eat biscuits by sharing them with other family members.  However, the family obeys the following two rules:

\begin{itemize}
	\item A biscuit must be shared between \textbf{exactly two} people
	\item If two people share a biscuit, they cannot share another biscuit with each other.  
\end{itemize}

Prove that two members of this family must have shared the same number of biscuits over the course of dinner.

\end{problem}

\newpage

%%%%%%%%%%%%%%%%%%%%%%%%%%%%%%%%%%%%%%%%%%%%%%%%%%%%%%%%%%%%%%%%%%%%%%%%%%%%%%%%
%Problem 6

\begin{problem}{10}
Prove that for any $m,n,k$ such that $0 \leq k < m<n-k$, the following identity holds:
\[ \binom{n}{k} = \sum_{i=0}^{k} {\binom{m}{i}} \cdot {\binom{n-m}{k-i}}\]
\solution{
Select $k$ people from a group of $n$ by splitting the people into two groups of $m$ and $n-m$, each with at least $k$ people, and then select $i$ from first group and $k-i$ from second group.
}
\end{problem}

\newpage
%%%%%%%%%%%%%%%%%%%%%%%%%%%%%%%%%%%%%%%%%%%%%%%%%%%%%%%%%%%%%%%%%%%%%%%%%%%%%%%%
%Problem 7

\begin{problem}{10}
Let G be a bipartite graph with $n$ nodes and $k$ connected components. You (mutually) independently color each of the nodes of $G$ red or black with equal probabilities. What is the probability that your coloring is a valid $2$-coloring of $G$? (Hint: the answer does not depend on the number of edges.)
\solution{ There are $2$ ways to correctly color each component, and there are $k$ components, so there is $2^k$ valid colorings. The number of colorings you might produce is $2^n$ and each is equallly likely. The probability that you get a valid coloring is 
\[\frac{2^k}{2^n} = \frac{2^{k-n}}{}\]
}
\end{problem}
\newpage

%%%%%%%%%%%%%%%%%%%%%%%%%%%%%%%%%%%%%%%%%%%%%%%%%%%%%%%%%%%%%%%%%%%%%%%%%%%%%%%%
%Problem 8

\begin{problem}{20}
In the far-off land of Spain, two soccer teams, Barcelona and Madrid have been battling each other for centuries to see who is more awesome. They play two games a year: one in the Spring and one in the Fall. The Madrid team has a tendency to fire their coaches very often to try to improve their results. The outcomes of the games are as follows:
\begin{enumerate}
\item
If Madrid has not fired their coach since the previous game:
\begin{itemize}
\item
Barcelona wins with probability $\frac25$
\item
Madrid wins with probability $\frac25$
\item
They tie with probability $\frac15$
\end{itemize}
\item
If Madrid has fired their coach since the previous game (they haven't yet realized that it's a bad idea):
\begin{itemize}
\item
Barcelona wins with probability $\frac35$
\item
Madrid wins with probability $\frac15$
\item
They tie with probability $\frac15$
\end{itemize}
\end{enumerate}
Now, Madrid does not fire their coach if they win or tie, but if they lose, they will fire their coach with $90\%$ probability following the loss.
For the following two questions, assume that Madrid did not fire their coach before the Spring 2010 game.  Here, everything is mutually independent unless otherwise specified. You can use the following timeline to help you visualize the situation.

\begin{center}
\includegraphics{timeline.jpg}
\end{center}

\bparts
\ppart{10}
Given that Madrid lost the Fall 2010 game, what is the probability that they fired their coach between the Spring and Fall games of 2010?

\newpage
\ppart{10}
What is the probability that Madrid fired their coach between the Spring and Fall games in $2010$ given that they lost BOTH of the games in $2010$?
\eparts
\end{problem}

\newpage

%%%%%%%%%%%%%%%%%%%%%%%%%%%%%%%%%%%%%%%%%%%%%%%%%%%%%%%%%%%%%%%%%%%%%%%%%%%%%%%%
%Problem 9
\begin{problem}{11}
Harvard's loan officer is evaluating her portfolio of student loans.  There are 4 especially suspicious loans.  Loan A is given to an average Harvard student who does average Harvard stuff.  It has a probability of 1/10 of being paid back.  Loan B was given to Bill Gates and has a $1/5$ probability of being paid back.  Loan C was given to David the evil TA and has a probability of 1/3 of being paid back.  Loan D was given to a 6.042 student and has 1/2 probability of being paid back.  In each of these parts, express any numbers in your answer as ratios of integers.

\bparts
	\ppart{3} If the probabilities of being paid back are mutually independent, what is the probability that all four loans are paid back? 
	\vspace{3 in}
	\ppart{4} If the probabilities are pairwise independent, what is the most that you can say about the probability that all four loans get paid back?  

\newpage
	\ppart{4} If you cannot assume any independence, what is the most you can say about the probability that all four loans are paid back?

\eparts
\end{problem}

\newpage
%%%%%%%%%%%%%%%%%%%%%%%%%%%%%%%%%%%%%%%%%%%%%%%%%%%%%%%%%%%%%%%%%%%%%%%%%%%%%%%%
%Problem 10
\begin{problem}{10}

Consider a length $n$ vector of integers, $x = (x_1, x_2, \ldots, x_n)$ where the entries
of the vector are integers in the set $\{1, 2, \ldots, n \}$, and the $x_i$ are selected uniformly at random.  Note that this means numbers can repeat; $x$ could, for example, be a vector of all $1$'s.  Let $A$ be the number of entries $x_i$ in the vector for which $x_i \leq i$. 	Calculate $\textrm{Ex}[A]$ and give it in closed form.


\end{problem}

\newpage
%%%%%%%%%%%%%%%%%%%%%%%%%%%%%%%%%%%%%%%%%%%%%%%%%%%%%%%%%%%%%%%%%%%%%%%%%%%%%%%%
%Problem 11

\begin{problem}{10}
Let $R$ be a positive random variable with $\textrm{Ex}[R^3] = k$. Prove that $Pr(R \geq x) \leq k/x^3$ for any $x > 0$.
\end{problem}
\newpage
%%%%%%%%%%%%%%%%%%%%%%%%%%%%%%%%%%%%%%%%%%%%%%%%%%%%%%%%%%%%%%%%%%%%%%%%%%%%%%%%
%Problem 12


\begin{problem}{32}

In all parts of this problem, assume that we are using fair, regular dice (six-sided with values $1, 2, 3, 4, 5, 6$ appearing equally likely).
Furthermore, assume that all dice rolls are mutually independent events.

\bparts
%\ppart{5}
%You repeatedly roll a fair die. What is the probability that you will roll a $5$ before you roll a $6$?
\ppart{4}
You roll two dice and look at the sum of the faces that come up.  What is the expected value of this sum?  Express your answer as a real number.
\vspace{4 in}

\ppart{7}
Assuming that the two dice are independent, calculate the variance of their sum.  Express your answer as a real number.
\vspace{4 in}


\ppart{7}
You repeatedly roll two fair dice and look at the sum. What is the probability that you will roll a sum of $4$ before you roll a sum of $7$? Express your answer as a real number.
\vspace{4 in}

\ppart{7}
What is the expected number of rolls until you get a sum of $4$ or a sum of $7$?  (For example, if you get $7$ on the first roll, the number of rolls is $1$.) Express your answer as a real number.

\vspace{4 in}

\ppart{7}
You roll 10 dice.  Using the Chernoff Bound, give an upper bound for the probability that 8 or more of them rolled a 1 or a 2?  You don't need to calculate the value with a 
calculator (since you do not have one), but please write it in simplest terms.



\eparts
\end{problem}

\newpage

%%%%%%%%%%%%%%%%%%%%%%%%%%%%%%%%%%%%%%%%%%%%%%%%%%%%%%%%%%%%%%%%%%%%%%%%%%%%%%%%
%Problem 13
\begin{problem}{10}

An MIT student is walking along Mass Ave and is torn between making a trip to the pub and going to her dorm to finish her 6.042 homework.  The student is $n$ steps from the pub and $T-n$ steps from the dorm where $T >=3$ is the distance between the pub and the dorm, both located on Mass Ave (which we will assume to be a straight line).  The student flips a coin at each step to decide which way to go.  With probability $p$, the student takes $2$ steps toward the pub and with probability $1-p$, she takes 1 step toward her dorm.  If the student is one or two steps away from the pub and takes two steps toward the pub, then she reaches the pub.  If she is one step away from her dorm and takes a step toward her dorm, then she reaches the dorm..  Let $X_n$ be the probability that the student reaches the pub before she reaches the dorm.  Assume that all coin tosses are mutually independent.  Find a recurrence for $X_n$. (You do not need to solve the recurrence, but your answer should contain sufficient information such that a 6.042 could solve it given enough time.)
\end{problem}
 


\newpage

\newpage
\begin{center}
(This space left blank internationally)
\end{center}

\newpage
\begin{center}
(This space left blank internationally)
\end{center}

\mbox{}

\newpage
\begin{center}
(This space left blank internationally)
\end{center}

\mbox{}

\end{document}
