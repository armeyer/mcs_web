\documentclass[handout]{mcs}

\begin{document}

\renewcommand{\reading}{
  Chapter~\bref{recursive_data_chap}.\ \emph{Recursive Data Types};
  Chapter~\bref{infinite_chap}.\ \emph{Infinite Sets};
  Chapter~\bref{number_theory_chap}.\ \emph{Number Theory}\ through~\bref{fundamental_theorem_sec}.\ \emph{The Fundamental Theorem of Arithmetic}}

\problemset{4}

\begin{staffnotes}
Lectures covered: Infinite Cardinality; The Halting Problem, Number Theory: GCD's, Number Theory: Modular Arithmetic
\end{staffnotes}

%%%%%%%%%%%%%%%%%%%%%%%%%%%%%%%%%%%%%%%%%%%%%%%%%%%%%%%%%%%%%%%%%%%%%
% Problems start here
%%%%%%%%%%%%%%%%%%%%%%%%%%%%%%%%%%%%%%%%%%%%%%%%%%%%%%%%%%%%%%%%%%%%%

% from spring10/ps6
\pinput{PS_super_symmetric_strings}

% from fall11/cp5m
\pinput{PS_koch_snowflake}

% from fall11/ps3
\pinput{PS_unit_interval}

% from fall09/ps7
\pinput{PS_linear_combination_game}

%%%%%%%%%%%%%%%%%%%%%%%%%%%%%%%%%%%%%%%%%%%%%%%%%%%%%%%%%%%%%%%%%%%%%
% Problems end here
%%%%%%%%%%%%%%%%%%%%%%%%%%%%%%%%%%%%%%%%%%%%%%%%%%%%%%%%%%%%%%%%%%%%%
\end{document}
