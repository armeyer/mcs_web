\documentclass[handout]{mcs}

\begin{document}

\inclassproblems{7, Fri.}

\begin{staffnotes}
Partial Orders \& Equivalence Relations, Ch.9.6-9.10
\end{staffnotes}

%%%%%%%%%%%%%%%%%%%%%%%%%%%%%%%%%%%%%%%%%%%%%%%%%%%%%%%%%%%%%%%%%%%%%
% Problems start here
%%%%%%%%%%%%%%%%%%%%%%%%%%%%%%%%%%%%%%%%%%%%%%%%%%%%%%%%%%%%%%%%%%%%%

\pinput{CP_partially_ordered_by_divisibility}
%\pinput{CP_partial_order_on_power_set}
%\pinput{CP_inverse_partial_order}

\pinput{TP_relation_properties_expressions} Note: A mistaken version
of part~(e) with the expression $R \subseteq R \compose R$ was used in
class.  This mistaken expression does not imply anything particularly
interesting about $R$.

\pinput{PS_subsequences_partial_order_Dilworth_Lemma}

\pinput{TP_basic_relations}

%\pinput{CP_equivalence_same_property}
%\pinput{PS_strict_partial_order_isomorphic_to_subset}

\begin{center}
\textbf{Supplemental Problem}\footnote{There is no need to study supplemental
  problems when preparing for quizzes or exams.}
\end{center}

%\pinput{CP_binary_relations_on_01}
\pinput{CP_equiv_partition_proof}

%\pinput{CP_product_relation_properties}
%\pinput{CP_weak_partial_order_isomorphic_to_subset}

%%%%%%%%%%%%%%%%%%%%%%%%%%%%%%%%%%%%%%%%%%%%%%%%%%%%%%%%%%%%%%%%%%%%%
% Problems end here
%%%%%%%%%%%%%%%%%%%%%%%%%%%%%%%%%%%%%%%%%%%%%%%%%%%%%%%%%%%%%%%%%%%%%
\end{document}
