TP\_mean\_time\_variance\_given

Problem 6 from Spring '98 final exam (suggested by Michaela)
  Prove that there are 16,800 permutations of the letters
  {a,b,c,d,e,f,g,h} such that neither the letters {a,b,c} nor the
  letters {d,e} occur in order. For example, fbgadceh is not allowed
  because d occurs before e, and fagbecdh is not allowed because a
  occurs before b and b occurs before c, but fbgaedch is OK.

PS\_neighborhood\_census

MQ9:
MQ\_voldemort\_returns,
MQ\_conditional\_prob\_inequality
CP\_max\_ranvar\_n
MQ\_expectHHH

Keshav P:
FP\_structural\_induction\_arithmetic\_expressions.tex
FP\_paths\_inclusion\_exclusion.tex
Find the last digit of $7^7^7^7$.

Ishaan: Let $h(k) \eqdef \rem{k}{m}$ where $m = 2^p - 1$ and $k$ is a
character string interpreted in radix $2^p$.  Show that if we derive
string $x$ from string $y$ by permuting its characters, then $h(x) = h(y)$.

Boggs:
P_counting_given_answers
FP_logic_of_leq

Mariam: see email

Giuliano:
\begin{problem}
For the following statements circle T for True or F for False.

(a) For integers $a$ and $b$ there are integers $x$ and $y$ such that:

$ax + by = 1$

(b) $gcd(mb+r, b) = gcd(r,b)$ for all integers $m,r$ and $b$.

(c) For every prime $p$ and every integer $k$, $k^{p-1} \equiv 1 (mod p)$

(d) For primes $p \not= q$, $\phi(pq) = (p-1)(q-1)$

(e) Let $a,b c and d$ be integers and $gcd(a,b) = c$. If $a \equiv b (mod d)$ then $ a/c \equiv b/c (mod d)$.

\begin{solution}

(a) FALSE. $a$ and $b$ must be relative primes.

(b) TRUE.

(c) FALSE. $k$ must be relative prime with $p$.

(d) TRUE.

(e) FALSE. Let $a=4, b=6$ and $d=2$, then $gcd(a,b) = gcd(4,6)
=2$. Thus  $a/2 = 2 \not\equiv b/2 = 3 (mod 2)$.

\end{solution}
\end{problem}

\begin{problem}[Induction]

A Lucas number is defined as follows:
\begin{equation}
  L_n :=
  \begin{cases}
    2               & \text{if } n = 0; \\
    1               & \text{if } n = 1; \\
    L_{n-1}+L_{n-2} & \text{if } n > 1. \\
   \end{cases}
\end{equation}

Prove by induction that for all $n \geq 2$,
\[
L_n L_{n-2} - L_{n-1}^2 = 5 (-1)^n.
\]
\end{problem}


Shu Zheng:

Question 1 - Number Theory (from question 4 of Fall 2006 pset 3:
http://courses.csail.mit.edu/6.042/fall06/ps3-sol.pdf)

Let S = 1^k + 2^k + . . . + (p − 1)^k, where p is an odd prime and k
is a positive multiple of (p − 1). Use Fermat’s theorem to prove that
S ≡ -1 (mod p).

Question 2 - Counting (from question 1 of Fall 2008 pset 9:
http://courses.csail.mit.edu/6.042/fall08/ps9-sol.pdf)

2a. Show that of any n + 1 distinct numbers chosen from the set {1, 2,
  . . . , 2n}, at least 2 must be relatively prime. (Hint: gcd(k, k +
1) = 1.)

2b. Show that any finite connected undirected graph with n ≥ 2
vertices must have 2 vertices with the same degree.


Jayson:

1) Prove that the set of quadratic polynomials with integer
coefficients is countably infinite.


Di Liu:

Week 8 Monday, Colorability of graphs with triangles: Problem 6, Spring 07
http://courses.csail.mit.edu/6.042/spring07/solutions/finalsol.pdf
Week 6 Wednesday, Task scheduling with DAGs: Problem 5, Spring 08
http://courses.csail.mit.edu/6.042/spring08/staff/finalsol.pdf
