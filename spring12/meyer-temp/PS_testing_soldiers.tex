 \documentclass[problem]{mcs}

\begin{pcomments}
  \pcomment{PS_testing_soldiers}
  \pcomment{F95.ps11}
  \pcomment{from Feller, exercises on expectation}
  \pcomment{edited by ARM 5/9/12}
\end{pcomments}

\pkeywords{
 probability
 random_variable
 expectationa
}

%%%%%%%%%%%%%%%%%%%%%%%%%%%%%%%%%%%%%%%%%%%%%%%%%%%%%%%%%%%%%%%%%%%%%
% Problem starts here
%%%%%%%%%%%%%%%%%%%%%%%%%%%%%%%%%%%%%%%%%%%%%%%%%%%%%%%%%%%%%%%%%%%%%

\begin{problem} (A true story from world war two).

Suppose the army needs to test its $n$ soldiers for a disease that
occurs independently at random with probability $p$ for each
soldier. The test is by means of a blood test.  One approach (i) is to
test each soldier individually: this requires $n$ tests.  An
alternative approach (ii) is the following.  Group the soldiers into
groups of $k$.  Blend the blood samples of each group and apply the
test once to each blended sample.  If the group-blend is free of the
disease, we are done after one test.  If the group-blend fails the
test, then someone has the disease and we have to test all $k$ people
for a total of $k+1$ tests.

\bparts

\ppart What is the expected number of tests in method (ii) as a
function of the number of soldiers $n$, the disease probability $p$,
and the group size $k$?

\begin{solution}
There are $n/k$ groups of size $k$ each. Let $X_i$ be a random
variable that denotes the number of tests performed in group
$i$.  $X_i$ takes value $1$ with probability $(1-p)^k$ and value $k+1$
with probability $1-(1-p)^k$. Hence the expected number of tests is
$E(\sum_{i=1}^{\frac{n}{k}} X_i) = \sum_{i=1}^{\frac{n}{k}} E(X_i) =
(\frac{n}{k})((1-p)^k + (k+1)(1 - (1-p)^k)) = n(1 - (1-p)^k +
\frac{1}{k})$.
\end{solution}

\ppart How should $k$ be chosen to minimize the expected number of
test performed, and what is the resulting expectation?

\begin{solution}
The $k$ must be chosen so that the derivative of the answer from part
a) w.r.t. $k$ is $0$, i.e. $(1-p)^k\ln(1-p) + \frac{1}{k^2} =
0$. Assuming that $p$ is very small, so that we can approximate
$(1-p)^k$ by $1$ and $\ln(1-p)$ by $-p$ we get that $k \sim
\sqrt{\frac{1}{p}}$. And the resulting expectation is $n\sqrt{p}$.

\end{solution}

\ppart What fraction of the work does method (ii) expect to save over
method (i) in a million-strong army with disease incidence $1\%$?

\begin{solution}
If $p = 0.01$ then the fraction of work saved is $1 - \sqrt{0.01} =
0.9$, i.e. $90\%$ of the work is saved in such a situation.

\end{solution}

\ppart Can you come up with a better scheme by using multiple levels
of grouping (ie, groups of groups)?

\begin{solution}
There are many possible improvements. Here is one. Consider a scheme
where we blend all the soldiers' blood and perform the test. If it
comes up negative we stop, else we split the soldiers into two groups
and repeat on each group. If $E_n$ is the expected number of test with
$n$ soldiers we see that it satisfies the recurrence
\[
E_n = (1-p)^n + (1- (1-p)^n)2E_{\frac{n}{2}}.
\]
It is easily proven by induction that $E_n = \Theta(pn)$. Since, in an
expected sense, there will $pn$ soldiers with the disease, and the
number of soldiers with the disease is a lower bound on the number of
tests, hence this scheme is optimal up to constant factors.
\end{solution}

\eparts

\end{problem}


%%%%%%%%%%%%%%%%%%%%%%%%%%%%%%%%%%%%%%%%%%%%%%%%%%%%%%%%%%%%%%%%%%%%%
% Problem ends here
%%%%%%%%%%%%%%%%%%%%%%%%%%%%%%%%%%%%%%%%%%%%%%%%%%%%%%%%%%%%%%%%%%%%%

\endinput
