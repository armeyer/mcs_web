\documentclass[handout]{mcs}

\begin{document}

\inclassproblems{2, Mon.}

%%%%%%%%%%%%%%%%%%%%%%%%%%%%%%%%%%%%%%%%%%%%%%%%%%%%%%%%%%%%%%%%%%%%%
% Problems start here
%%%%%%%%%%%%%%%%%%%%%%%%%%%%%%%%%%%%%%%%%%%%%%%%%%%%%%%%%%%%%%%%%%%%%


\pinput{FP_6_and_15_cent_stamps_by_WOP}

\begin{staffnotes}
For the following problem, encourage your team to use the WOP proof
template and get a complete solution written on their whiteboard.  If
someone brings up induction, confirm that induction would be a
standard approach, but we're focussing on WOP today and don't want to
be distracted by other approaches.
\end{staffnotes}

%\pinput{MQ_10_and_15_cent_stamps_by_WOP}

\pinput{CP_sum_of_squares}

\pinput{PS_Lehmans_equation}

\pinput{FP_AND_circuit}

\pinput{CP_m_envelopes_WOP}

%\pinput{MQ_prove_by_wop_odds}

%\pinput{MQ_wop_proof_sumofcubes}

%\pinput{TP_lowest_terms}


\iffalse

\inhandout{\newpage}

And if you have time\dots

\begin{staffnotes}
The next two problems illustrate some challenges in proving abstract
properties of well ordering.  
\end{staffnotes}

\begin{center}
\textbf{Supplemental Problems}\footnote{There is no need to study supplemental
  problems when preparing for quizzes or exams.}
\end{center}

\pinput{TP_well_order_examples}
\pinput{CP_well_order_decreasing}
\fi

%%%%%%%%%%%%%%%%%%%%%%%%%%%%%%%%%%%%%%%%%%%%%%%%%%%%%%%%%%%%%%%%%%%%%
% Problems end here
%%%%%%%%%%%%%%%%%%%%%%%%%%%%%%%%%%%%%%%%%%%%%%%%%%%%%%%%%%%%%%%%%%%%%
\end{document}
