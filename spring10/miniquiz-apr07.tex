\documentclass[quiz]{mcs}

\begin{document}

\miniquiz{Apr. 7}

%\section*{MORNING}

%%%%%%%%%%%%%%%%%%%%%%%%%%%%%%%%%%%%%%%%%%%%%%%%%%%%%%%%%%%%%%%%%%%%%
% Problems start here
%%%%%%%%%%%%%%%%%%%%%%%%%%%%%%%%%%%%%%%%%%%%%%%%%%%%%%%%%%%%%%%%%%%%%

\pinput[points = 6]{MQ_pulverizer_linear_combination}
\instatements{\newpage}
\pinput[points = 6]{MQ_isomorphic_but_not_planar}
\instatements{\newpage}
\pinput[points = 8]{MQ_ambiguous_recursive-def.tex}

%%%%%%%%%%%%%%%%%%%%%%%%%%%%%%%%%%%%%%%%%%%%%%%%%%%%%%%%%%%%%%%%%%%%%
% Problems end here
%%%%%%%%%%%%%%%%%%%%%%%%%%%%%%%%%%%%%%%%%%%%%%%%%%%%%%%%%%%%%%%%%%%%%

\instatements{\newpage

\section*{Appendix}

\subsection*{Matched Brackets}

  Recursively define the set, $\text{MB}$, of strings of ``matching'' brackets
  as follows:

\begin{itemize}

\item \textbf{Base case:} $\emptystring \in \text{MB}$.

\item \textbf{Constructor case:} If $s,t \in \text{MB}$, then
  $\mtt{[}s\mtt{]}t \in \text{MB}$.

\end{itemize}



}

\end{document}
