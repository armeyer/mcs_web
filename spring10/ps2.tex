\documentclass[handout]{mcs}

\begin{document}

\renewcommand{\reading}{Chapter~\bref{sets_chap}, \emph{Mathematical
    Data Types}; Chapter~\bref{logic_chap}, \emph{First-Order Logic}.
  (Assigned readings do not include the Problem sections.)

\emph{Reminder}: Comments on the reading using the \emph{NB online
  annotation system} are due at times indicated in the online tutor
problem set TP.3.  Reading Comments count for 5\% of the final grade.}

\problemset{2}

\begin{staffnotes}
TOPICS: Sets \& Relations, Mapping Lemma \& Finite Cardinality,
Predicates \& Quantifiers
\end{staffnotes}

%%%%%%%%%%%%%%%%%%%%%%%%%%%%%%%%%%%%%%%%%%%%%%%%%%%%%%%%%%%%%%%%%%%%%
% Problems start here
%%%%%%%%%%%%%%%%%%%%%%%%%%%%%%%%%%%%%%%%%%%%%%%%%%%%%%%%%%%%%%%%%%%%%

<<<<<<< .mine
<<<<<<< .mine
%needs def of relation matrix DONE
=======
%OK, but isolated and abstract; can we connect it to something conceptual?
=======
>>>>>>> .r834
\pinput{PS_composition_to_bijection.tex}

\begin{staffnotes}
\newpage
Rewritten after pset was submitted, reversing $R$ and $S$:
\end{staffnotes}

>>>>>>> .r719
\pinput{PS_relation_matrices.tex}

<<<<<<< .mine
%more class prob level than pset; maybe escalate to express some ZF axiom? ADDED
=======
>>>>>>> .r834
\pinput{PS_logical_set_theory.tex}

\pinput{PS_disjoint_cartesian_products.tex}

%%%%%%%%%%%%%%%%%%%%%%%%%%%%%%%%%%%%%%%%%%%%%%%%%%%%%%%%%%%%%%%%%%%%%
% Problems end here
%%%%%%%%%%%%%%%%%%%%%%%%%%%%%%%%%%%%%%%%%%%%%%%%%%%%%%%%%%%%%%%%%%%%%
\end{document}
