\documentclass[handout]{mcs}

\begin{document}

\inclassproblems{5, Fri.}

%%%%%%%%%%%%%%%%%%%%%%%%%%%%%%%%%%%%%%%%%%%%%%%%%%%%%%%%%%%%%%%%%%%%%
% Problems start here
%%%%%%%%%%%%%%%%%%%%%%%%%%%%%%%%%%%%%%%%%%%%%%%%%%%%%%%%%%%%%%%%%%%%%

\large\textbf{F05PS5P3}
\begin{problem}
Suppose that $a \equiv b \pmod{n}$ and $n > 0$.  Prove or disprove the
following assertions:

\bparts

\ppart $a^c \equiv b^c \pmod{n}$ where $c \geq 0$

\solution{The proof is by induction on $c$ with the hypothesis that
$a^c \equiv b^c \pmod{n}$.  If $c = 0$, then the claim holds, because
$1 \equiv 1 \pmod{n}$.  Now suppose that:

\begin{eqnarray*}
a^c & \equiv & b^c \pmod{n}
\end{eqnarray*}

Multiplying both sides by $a$ gives:

\begin{eqnarray*}
a^{c+1} & \equiv & a b^c \pmod{n}
\end{eqnarray*}

Since $a \equiv b \pmod{n}$, we can replaced the $a$ on the right side
by $b$:

\begin{eqnarray*}
a^{c+1} & \equiv & b^{c+1} \pmod{n}
\end{eqnarray*}

Therefore, the claim holds by induction.
}



\ppart $c^a \equiv c^b \pmod{n}$ where $a, b, \geq 0$

\solution{The claim is false.  For example:

\begin{eqnarray}
2^0 \not\equiv 2^3 \pmod{3}
\end{eqnarray}
}

\iffalse

\ppart $a \equiv b \pmod{n}$ implies $k^a \equiv k^b \pmod{n}$ for all
$k \geq 0$.

\solution{This is false.  For example, $0 \equiv 3 \pmod{3}$, but $2^0
\not\equiv 2^3 \pmod{3}$.}
\fi

\eparts

\end{problem}

\large\textbf{Repo: CP\_13th\_roots}
\pinput{CP_13th_roots}

\large\textbf{Repo: PS\_calculating\_inverses}
\pinput{PS_calculating_inverses}
\large\textbf{CP\_multiples\_of\_9\_and\_11}
\pinput{CP_multiples_of_9_and_11}
\large\textbf{CP\_proving\_basic\_congruence\_properties}
\pinput{CP_proving_basic_congruence_properties}

% Problems end here
%%%%%%%%%%%%%%%%%%%%%%%%%%%%%%%%%%%%%%%%%%%%%%%%%%%%%%%%%%%%%%%%%%%%%

\end{document}

\endinput
