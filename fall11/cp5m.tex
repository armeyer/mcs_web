\documentclass[handout]{mcs}

\begin{document}

\inclassproblems{5, Mon.}

%%%%%%%%%%%%%%%%%%%%%%%%%%%%%%%%%%%%%%%%%%%%%%%%%%%%%%%%%%%%%%%%%%%%%
% Problems start here
%%%%%%%%%%%%%%%%%%%%%%%%%%%%%%%%%%%%%%%%%%%%%%%%%%%%%%%%%%%%%%%%%%%%%

\large\textbf{Repo: PS\_koch\_snowflake -- better for PSet?}
\pinput{PS_koch_snowflake}

\large\textbf{Repo: CP\_binary\_trees}
\pinput{CP_binary_trees}

\large\textbf{F07Rec5TP1}

\begin{problem}
The Elementary 18.01 Functions ($\EF$'s) are the set of functions of
one real variable defined recursively as follows:

\textbf{Base cases:}
\begin{itemize}


\item The identity function, $\ide(x) \eqdef x$ is an $\EF$,

\item any constant function is an $\EF$,

\item the sine function is an $\EF$,

\end{itemize}

\textbf{Constructor cases:}

If $f,g$ are $\EF$'s, then so are
\begin{enumerate}

\item $f + g$, $fg$, $e^g$ (the constant $e$),\label{+-}

\item the inverse function $f^{-1}$,\label{inversefunc}

\item the composition $f \compose g$.\label{cmp}
\end{enumerate}

Prove, by Structural Induction on this definition, that the Elementary
18.01 Functions are \emph{closed under taking derivatives}.  That is, 
show
that if $f(x)$ is an $\EF$, then so is $df/dx$.

\solution{
\begin{proof}
By Structural Induction on def of $f \in \EF$.  The induction hypothesis
is the above statement to be shown.


\item[Base Cases:] We want to show that the derivatives of all the
  base case functions are in $\EF$.
  This is easy: for example, $\deriv \ide(x) = 1$ is a constant 
fucntion, and so is
   is in $\EF$. Similarly, $\deriv \sin(x) =
  \cos(x)$ which is also in $\EF$ since $\cos(x) = \sin(x+\pi/2) \in
  \EF$ by rules for constant functions, the identity function, sum, and
  composition with sine.

This proves that the induction hypothesis holds in the Base cases.

\item[Constructor Cases:] ($f^{-1}$).  Assume $f, df/dx \in \EF$ to 
prove
$\deriv f^{-1}(x) \in \EF$.

Letting $y = f(x)$, so $x=f^{-1}(y)$, we know from Leibniz's rule in
calculus that
\begin{equation}\label{Leib}
df^{-1}(y)/dy = dx/dy = \frac{1}{dy/dx}.
\end{equation}
For example,
\[
d \sin^{-1}(y)/dy = 1/(d \sin(x)/dx) = 1/\cos(x) = 1/\cos(\sin^{-1}(y)).
\]
Stated as in ~(\ref{Leib}), this rule is easy to remember, but can 
easily
be misleading because of the variable switching between $x$ and $y$.
It's more clearly stated using variable-free notation:
\begin{equation}\label{Leib2}
(f^{-1})' = (1/f') \compose f^{-1}.
\end{equation}
Now, since $f' \in \EF$ (by assumption), so is $1/f' = f'^{(-1)}$ 
(by~\ref{+-}.) and
$f^{-1}$ (by~\ref{inversefunc}.), and therefore so is their composition
(by~\ref{cmp}).  Hence the righthand side of equation~(\ref{Leib2})
defines a function in $\EF$.

\item[Constructor Case:] ($f \compose g$).  Assume $f, g, df/dx, dg/dx 
\in \EF$
to prove $d(f \compose g)(x)/dx \in \EF$.

The Chain Rule states that
\[
\frac{d(f(g(x)))}{dx} = \frac{d f(g)}{dg}\cdot\frac{dg}{dx}.
\]
Stated more clearly in variable-free notation, this is
\[
(f \compose g)' = (f' \compose g) \cdot g'.
\]
The righthand side of this equation defines a function in $\EF$ by
rules~\ref{cmp}.\ and~\ref{+-}.

The other Constructor cases are similar, so we conclude that the 
induction
hypothesis holds in all Constructor cases.

This completes the proof by structural induction that the statement 
holds
for all $f \in \EF$.
\end{proof}
}

\end{problem}

\large\textbf{F07Rec5TP2}

\begin{problem}
Provide simple \emph{recursive} definitions of the following sets:

\begin{problemparts}

\item The set $S \eqdef \set{ 2^k 3^m 5^n \suchthat k,m,n \in
\naturals}$.

\solution{We can define the set $S$ recursively as follows:

\begin{enumerate}
\item $1 \in S$
\item If $n \in S$, then $2n$, $3n$, and $5n$ are in $S$.
\end{enumerate}
}

\item  The set $T \eqdef \set{2^k 3^{2k+m} 5^{m+n} \suchthat k,m,n \in
\naturals}$.

\solution{We can define the set $T$ recursively as follows:

\begin{enumerate}
\item $1 \in T$
\item If $n \in S$, then $18n$, $15n$, and $5n$ are in $T$.
\end{enumerate}
}

\item The set $L \eqdef \set{ (a, b) \in \integers^2 \suchthat 3
  \divides (a-b)}$.

\solution{We can define a set $L' = L$ recursively as follows:

\begin{enumerate}
\item $(0, 0), (1,1), (2,2) \in L'$
\item If $(a, b) \in L'$, then $(a + 3, b)$, $(a - 3, b)$, $(a, b +3)$, 
and
$(a, b -3)$ are in $L'$.
\end{enumerate}

Lots of other definitions are also possible.}

\ppart Prove that your recursive definition of $L$ is correct.  That is,
let $L'$ be your recursively defined set and prove that $L$ and $L'$ 
have
exactly the same elements.

\solution{
For the $L'$ defined above, a straightforward structural induction shows
that if $(c,d) \in L'$, then $(c,d) \in L$.  Namely, each of the base
cases in the definition of $L'$ are in $L$ since $3 \divides 0$.  For 
the
constructor cases, we may assume $(a,b) \in L$, that is $3 \divides
(a-b)$, and must prove that $(a\pm 3,b) \in L$ and $(a,b \pm 3) \in L$.
In the first the case, we must show that $3 \divides ((a\pm 3)-b)$.  But
this follows immediately because $((a\pm 3)-b) = (a-b)\pm 3$ and 3 
divides
both $(a-b)$ and 3.  The other constructor case $(a,b\pm 3)$ follows in
exactly the same way.  So we conclude by structural induction on the
definition of $L'$ that $L' \subseteq L$.

Conversely, we must show that $L \subseteq L'$.  So suppose $(c,d) \in 
L$,
that is, $3 \divides (c-d)$.  This means that $c = r+3k$ and $d=r +3j$ 
for
some $r \in \set{0,1,2}$ and $j,k \in \integers$.  Then starting from 
base
case $(r,r) \in L'$, we can apply the $(a \pm 3,b)$ constructor rule
$\abs{k}$ times to conclude that $(c,r) \in L'$, and then apply the $(a 
,b
\pm 3)$ rule $\abs{j}$ times to conclude that $(c,d) \in L'$.  This
implies that $L \subseteq L'$, which completes the proof that $L = L'$.
}

\ppart (Optional) Give an \emph{unambiguous} recursive definition of 
$L$.

\solution{This is tricky.  Here is an attempt:

  \textbf{base cases}: $(0, 0), (1,1), (2,2), (-1,-1), (-2,-2), (-3,-3)
  (1, -2), (2, -1), (-1, 2), (-2, 1) \in L$

Now the idea is to constrain the constructors so the two coordinates 
have
absolute values that increase differing by at most 1, then one 
coordinate only
can continue to grow in absolute value.  Let
\[
sg(x) \eqdef \begin{cases}
              1 \text{ if } x \geq 0,\\
             -1 \text{ if } x < 0.
              \end{cases}
\]

\textbf{constructors}: if $(a,b) \in L'$, then

\begin{itemize}

\item if $\abs{\abs{a} - \abs{b}} \leq 1 $, then $(a+3sg(a),b+3sg(b)),
  (a+3sg(a),b), (a,b+3sg(b)) \in L'$,

\item if $\abs{a} > \abs{b}+1$, then $(a+3sg(a),b) \in L'$,

\item if $\abs{b} > \abs{a} + 1$, then $(a,b+3sg(b)) \in L'$.

\end{itemize}
}


\end{problemparts}
\end{problem}

\large\textbf{F07Rec5TP3}

\begin{problem}

$\baexp$'s are defined in the Appendix.

\bparts

\ppart The value of $\text{flatten}(e)$ for $e \in \baexp$ is the 
sequence
of integers in $e$ obtained by ``erasing'' everything but the integers
that appear within tagged variables and tagged \texttt{int}'s.  For
example,
\begin{align*}
e & \eqdef \ang{\texttt{sum}, \ang{\texttt{var},
3}, \ang{\texttt{sum}, \ang{\texttt{var}, 2},\ang{\texttt{int}, 2}}}\\
f & \eqdef \ang{\texttt{prod}, \ang{\texttt{var},4},\ang{\texttt{var}, 
5}}\\
g & \eqdef \ang{\texttt{prod}, e,  \ang{\texttt{sum}, 
\ang{\texttt{var},7}, f}}\\
\text{flatten}(g) & = \ang{3,2,2,7,4,5}.
\end{align*}
Give a recursive definition of flatten.  (You may use the operation of
\emph{concatenation} (append) of two sequences.)

\solution{
We define flatten recursively on the definition of $\baexp$.

\begin{itemize}

\item \textbf{Base cases:}
\begin{enumerate}
\item $\text{flatten}(\ang{\texttt{int}, n}) \eqdef \ang{n}$
\item $\text{flatten}(\ang{\texttt{var}, n}) \eqdef \ang{n}$
\end{enumerate}

\item \textbf{Constructor cases:}
\begin{enumerate}

\item $\text{flatten}(\ang{\texttt{sum}, e_1, e_2}) \eqdef
\text{flatten}(e_1)\text{flatten}(e_2)$

\item $\text{flatten}(\ang{\texttt{prod}, e_1, e_2}) \eqdef
\text{flatten}(e_1)\text{flatten}(e_2)$

\end{enumerate}
\end{itemize}
}

\ppart Prove by structural induction on the definition of $\baexp$
that for all $e \in \baexp$,
\[
2\cdot \text{length}(\text{flatten}(e)) = \card{e}+1
\]

\solution{
The proof is by structural induction on the definition of $e \in 
\baexp$.
The induction hypothesis is the equation above.

\begin{itemize}
\item \textbf{Base cases:}
\begin{enumerate}
\item $2 \cdot \text{length}(\text{flatten}(\ang{\texttt{int}, n})) = 
2\cdot 1 =
  2 = 1 + 1 = \card{\ang{\texttt{int}, n}} + 1.$
\item \texttt{var} case the same.
\end{enumerate}
So the equation holds in the base cases.

\item \textbf{Constructor cases:}
\begin{enumerate}

\item Say $e = \ang{\texttt{sum}, e_1, e_2}$ where we assume the 
Structural
Induction hypothesis that $e_1$ and $e_2$ satisfy the equation given
above.  Now,
\begin{align*}
\lefteqn{2\cdot \text{length}(\text{flatten}(e))}\\
  & = 2 \cdot \text{length}(\text{flatten}(\ang{\texttt{sum}, e_1, 
e_2}))\\
  & = 2\cdot \text{length}(\text{flatten}(e_1)\text{flatten}(e_2))
        & \text{(def of flatten)}\\
  & = 2\cdot \text{length}(\text{flatten}(e_1))+
     2 \cdot \text{length}(\text{flatten}(e_2))
        & \text{(length of a string)}\\
  & = (\card{e_1}+1) + \card{e_2}+1 & \text{(structural induction 
hyp.)}\\
  & = \card{\ang{\texttt{sum}, e_1, e_2}} + 1
           & \text{(def. of $\card{\ang{\text{sum},\ }}$)}\\
  & = \card{e} +1.
\end{align*}
So the equation holds for $e$.
\item \texttt{prod} the same.
\end{enumerate}
This completes the proof for the Constructor cases.
\end{itemize}

We conclude by Structural Induction that the equation holds for all $e 
\in
\baexp$.}

\eparts
\end{problem}

\large\textbf{Repo: CP\_F18\_functions}
\pinput{CP_F18_functions}
\large\textbf{CP\_erasable\_strings}
\pinput{CP_erasable_strings}
\large\textbf{CP\_recursively\_defined\_sets}
\pinput{CP_recursively_defined_sets}


%%%%%%%%%%%%%%%%%%%%%%%%%%%%%%%%%%%%%%%%%%%%%%%%%%%%%%%%%%%%%%%%%%%%%
% Problems end here
%%%%%%%%%%%%%%%%%%%%%%%%%%%%%%%%%%%%%%%%%%%%%%%%%%%%%%%%%%%%%%%%%%%%%

\end{document}
