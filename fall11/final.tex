\documentclass[quiz]{mcs}

\renewcommand{\exampreamble}{   % !! renew \exampreamble
\iffalse
  \textbf{Circle the name of your TA}:
  \begin{center}
  \renewcommand{\arraystretch}{2.5}
  \begin{tabular}{l}
     \courseassistants
  \end{tabular}
  \end{center}\fi

  \begin{itemize}

  \item This exam is \textbf{closed book} except for a two page, 2-sided
    crib sheet.  Total time is 3 hours.

  \item Write your solutions in the space provided with your name on every
    page.  If you need more space, write on the back of the sheet
    containing the problem.  Please keep your entire answer to a problem
    on that problem's page.

  \item
   GOOD LUCK!
  \end{itemize}}

\begin{document}
\final

%%%%%%%%%%%%%%%%%%%%%%%%%%%%%%%%%%%%%%%%%%%%%%%%%%%%%%%%%%%%%%%%%%%%%
% Problems start here
%%%%%%%%%%%%%%%%%%%%%%%%%%%%%%%%%%%%%%%%%%%%%%%%%%%%%%%%%%%%%%%%%%%%%

\pinput[points = 6, title = \textbf{FP\_binary\_relations\_on\_01}]{FP_binary_relations_on_01}

%\examspace
\pinput[points = 6, title = \textbf{FP\_chebyshev\_hat\_check}]{FP_chebyshev_hat_check}

%\examspace
\pinput[points = 6, title = \textbf{MQ\_countable\_unions}]{MQ_countable_union}
   %{FP_countable_sets}

%\examspace
\pinput[points = 6, title =
  \textbf{FP\_infinite\_binary\_sequences}]{FP_infinite_binary_sequences}
\hint You may assume the result in the previous problem.

%\examspace
%\pinput[points = 6, title =
%  \textbf{FP\_rational\_structural\_induction}]{FP_rational_structural_induction}

%\examspace
\pinput[points = 6, title = \textbf{FP\_divisibility\_tests}]{FP_divisibility_tests}

%\examspace
\pinput[points = 6, title = \textbf{FP\_graph\_degree}]{FP_graph_degree}

%\examspace
\pinput[points = 6, title = \textbf{FP\_big\_o}]{FP_big_o}

%\examspace
\pinput[points = 6, title = \textbf{FP\_neighborhood\_census}]{FP_neighborhood_census}

\iffalse
%\examspace
\pinput[points = 6, title =
  \textbf{FP\_uncountable\_infinite\_sequences}]{FP_uncountable_infinite_sequences}

%\examspace
\pinput[points = 6, title =
  \textbf{FP\_product\_rule\_and\_independence}]{FP_product_rule_and_independence}
\fi

%\examspace
\pinput[points=6, title =
  \textbf{FP\_4\_and\_7\_cent\_stamps\_by\_induction}]{FP_4_and_7_cent_stamps_by_induction}

%\examspace
\pinput[points=6, title =
  \textbf{FP\_bogus\_coloring\_proof}]{FP_bogus_coloring_proof}

%\examspace
%\pinput[points=6, title =
%  \textbf{FP\_graphs\_short\_answer}]{FP_graphs_short_answer}

%%\examspace
%\pinput[points=6, title =
%  \textbf{FP\_planar\_structural\_induction}]{FP_planar_structural_induction}


%\examspace
\pinput[points = 6, title =
  \textbf{FP\_boat\_trip}]{FP_boat_trip}

%\examspace
\pinput[points = 6, title =
  \textbf{FP\_marvels\_trip}]{FP_marvels}


%%\examspace
%\pinput[points = 6, title =
%  \textbf{FP\_towers\_of\_Sheboygan}]{FP_towers_of_Sheboygan}

%\examspace
\pinput[points=6, title =
  \textbf{FP\_college\_probability}]{FP_college_probability}

%%\examspace
%\pinput[points = 6, title =
%  \textbf{FP\_sampling\_wafers}]{FP_sampling_wafers}

%\examspace
\pinput[points = 6, title =
  \textbf{FP\_check\_factor\_by\_digits}]{FP_check_factor_by_digits}

%\examspace
\pinput[points = 6, title =
  \textbf{FP\_counting\_poker\_high\_cards}]{FP_counting_poker_high_cards}

\begin{center}
{\large F09? Problems}
\end{center}

%\pinput[points = 8, title = \textbf{Graph Coloring
%  Induction}]{FP_bogus_coloring_proof}

%\examspace
\textbf{NOT USED:}

  \pinput[points = 10, title= \textbf{Logic formulas, Counting }]{FP_lining_up}

%\examspace
\pinput[points = 7, title=
  \textbf{FP\_structural\_induction\_arithmetic\_composition}]
       {FP_structural_induction_arithmetic_composition}

%\examspace
  \pinput[points = 5, title = \textbf{Bipartite Average}]{FP_bipartite_matching_sex}

%\examspace
  \pinput[points = 8, title = \textbf{Graphs, Congruences}]{FP_multiple_choice}

%\examspace
  \pinput[points = 7, title= \textbf{State Machine}]{FP_santa_state_machine}

\textbf{NOT USED:}
%\examspace
  \pinput[points = 10, title= \textbf{Stable Matching}]{FP_marriage_modify}

%\examspace
  \pinput[points = 5, title= \textbf{Asymptotics}]{FP_asymptotics_define_functions}

%\examspace
  \pinput[points = 7, title = \textbf{Conditional Probability}]{FP_conditional_prob_inequality}

%%\examspace
%  \pinput[points = 7, title = \textbf{Probable Satisfiability}]{CP_probable_satisfiability_nk}

%%\examspace
%  \pinput[points = 8, title= \textbf{Probable Congestion}]{FP_network_probability}

%\examspace
%  \pinput[points = 7, title= \textbf{Convolution}]{FP_boat_trip} 

%\examspace
  \pinput[points = 5, title= \textbf{Counting Strings}]{FP_string_counting}

%\examspace
  \pinput[points = 5, title= \textbf{Euler's Probability}]{FP_modular_exponential}

%%\examspace
%  \pinput[points = 8, title= \textbf{Counting}]{FP_counting_given_answers}  

%\examspace
  \pinput[points = 8, title= \textbf{Counting paths; Combinatorial Proof}]{FP_more_counting}  

%\examspace
  \pinput[points = 7, title= \textbf{Markov, Chebyshev Bounds}]{FP_gambling_man}

%\examspace
  \pinput[points = 6, title= \textbf{Sampling \& Confidence}]{FP_random_sampling}

\begin{center}
{\large FINAL S10}
\end{center}

%\examspace
  \pinput[points = 10, title = \textbf{Asymptotic Bounds and Partial
      Orders}]{FP_asymptotic_partial_order}


%\examspace
  \pinput[points = 10, title = \textbf{Euler's Function}]{FP_modular_powerful}


%\examspace
  \pinput[points = 10, title = \textbf{Magic Trick
      Redux}]{FP_magic_trick_27_cards}

%\examspace
  \pinput[points = 10, title = \textbf{Combinatorial
      Proof}]{FP_combinatorial_binomial}

NOT USED:

%\examspace
  \pinput[points = 10, title = \textbf{Rich \#4}]{FP_asymptotics}


NOT USED:

%\examspace
  \pinput[points = 10, title = \textbf{Rich \#5}]{FP_bijection_counting}

\begin{center}
{\large FINAL S11}
\end{center}


\pinput[points = 15, title =
  \textbf{Number Theory}]{FP_numbers_short_answer}

%\examspace
\pinput[points = 15, title =
  \textbf{Graphs}]{FP_graphs_short_answer}

%\examspace
\pinput[points = 10, title =
  \textbf{Partial orders}]{FP_partial_order_short_answer}

NOT USED:

%\examspace
\pinput[points = 30, title =
  \textbf{FP\_multiple\_choice\_unhidden}]{FP_multiple_choice_unhidden}

\begin{center}
{\large FINAL S06}
\end{center}

%\examspace
\pinput[points = 6, title = \textbf{FP\_counting\_finesse}]{FP_counting_finesse}

%\examspace
\pinput[points = 6, title = \textbf{MQ\_graph\_state\_machine}]{MQ_graph_state_machine}

%\examspace
\pinput[points = 6, title = \textbf{FP\_dangerous\_generating\_func}]
       {FP_dangerous_generating_func}

%\examspace
\pinput[points = 6, title = \textbf{MQ\_coloring}]{MQ_coloring}

%% graph theory, independence, expected value, variance (adapted from F03 final)

\begin{problem}
Let $K_n$ be the complete graph with $n$ vertices.  Each of the edges of
the graph will be randomly assigned one of the colors red, green, or blue.
The assignments of colors to edges are mutually independent, and the
probabilty of an edge being assigned red is $r$, blue is $b$, and green is
$g$ (so $r+b+g=1$).

A set of three vertices in the graph is called a \emph{triangle}.  A
triangle is \emph{monochromatic} if the three edges connecting the vertices
are all the same color.

\bparts 

\ppart[3] Let $m$ be the probability that any given triangle, $T$, is
monochromatic.  Write a simple formula for $m$ in terms of $r,b,$ and $g$.

\exambox{2.0in}{0.5in}{-0.4in}
\examspace[0.5in]

\begin{solution}
$m = r^3+b^3 +g^3$
\end{solution}

\ppart[4] Let $I_T$ be the indicator variable for whether $T$ is
monochromatic.  Write simple formulas in terms of $m,r,b,$ and $g$ for
$\expect{I_T}$ and $\variance{I_T}$.

\exambox{2.0in}{0.5in}{-0.8in}
\examspace[1.0in]

\begin{solution}
\begin{align*}
\expect{I_T} & = m\\
\variance{I_T} & = m(1-m).
\end{align*}
\end{solution}

\eparts

\examspace

\begin{center}
{\large Now assume $r=b=g = 1/3$.}
\end{center}

\bparts

\ppart[7] Let $T$ and $U$ be distinct triangles.  Show that $I_T$ and $I_U$
are independent random variables.

\examspace[6.0in]
\begin{solution}

Since $I_T$ and $I_U$ are indicators for events, it suffices to verify that
\[
\pr{I_T = 1}\cdot \pr{I_U = 1} = \pr{I_T\cdot I_U =1}.
\]
There are two cases depending on whether $T$ and $U$ share an edge.  In each
case, $\pr{I_T = 1}\cdot \pr{I_U = 1} = \pr{I_T\cdot I_U =1} = 1/3^4$.

\end{solution}

\ppart[5] Let $M$ be the number of monochromatic triangles.  Write simple
formulas in terms of $n, m,r,b,$ and $g$ for $\expect{M}$ and
$\variance{M}$.

\exambox{2.0in}{0.5in}{-0.8in}
\examspace[1.0in]


\begin{solution}
\begin{align*}
\expect{M} & =  m \cdot \text{(\# triangles)}\\
           & = m\binom{n}{3}.\\
\variance{M} & = \variance{I_T} \cdot \text{(\# triangles)}\\
           & = m(1-m)\binom{n}{3}.
\end{align*}
\end{solution}


\examspace

\ppart[8]  Let $\mu \eqdef \expect{M}$.  Prove that
\[
\pr{\abs{M - \mu} > \sqrt{\mu \log \mu}} = O \paren{\frac{1}{\log n}}
\]

\examspace{5in}

\begin{solution}
By Chebyshev,
\begin{equation}\label{C}
\pr{\abs{M - \mu} > c \sigma} \leq \frac{1}{c^2}
\end{equation}
But
\[
\sigma = \sqrt{(1-m)\mu} < \sqrt{\mu}
\]
so
\begin{align*}
\pr{\abs{M - \mu} > \sqrt{\mu \log \mu}}
 & = \pr{\abs{M - \mu} > \paren{\sqrt{\log \mu}} \sqrt{\mu}}\\
 & \leq \pr{\abs{M - \mu} > \paren{\sqrt{\log \mu}} \sigma}\\
 & \leq \frac{1}{\log \mu}  & \text{(by~\eqref{C})}\\
 & = O \paren{\frac{1}{\log n}}.
\end{align*}
The last step follows from the fact that
\[
\log \mu = \log \paren{m\binom{n}{3}} = \log m + \log n + \log (n-1) +
\log (n-2) - \log 6 = \Theta(\log n).
\]
\end{solution}

\ppart[3] Conclude that
\[
\lim_{n\to \infty} \pr{\abs{M - \mu} > \sqrt{\mu \log \mu}} = 0
\]

\begin{solution}

Since $1/(\log n) \to 0$ as $n \to \infty$, the $O()$ bound on the
probability goes to 0 which means an upper bound on the limit is 0.  Since
the probability is nonnegative, the limit must be exactly 0.

\end{solution}

\eparts
\end{problem}

\end{document}
