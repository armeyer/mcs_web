\documentclass[handout]{mcs}

\begin{document}

\renewcommand{\reading}{ 
\begin{itemize}
\item Chapter~\bref{induction_chap}.\bref{state_machine_sec}.\ \emph{State Machines}
\item Chapter~\bref{infinite_chap}.\ \emph{Infinite Sets: The
    Halting Problem}.
\end{itemize}
}

\problemset{4}

\begin{staffnotes}
Lectures covered: Structural Induction, Recursive Data Structures, Infinite Sets
\end{staffnotes}

%%%%%%%%%%%%%%%%%%%%%%%%%%%%%%%%%%%%%%%%%%%%%%%%%%%%%%%%%%%%%%%%%%%%%
% Problems start here
%%%%%%%%%%%%%%%%%%%%%%%%%%%%%%%%%%%%%%%%%%%%%%%%%%%%%%%%%%%%%%%%%%%%%

%Structural Induction
\pinput{FP_structural_induction_rational_composition}
\pinput{PS_koch_snowflake}
\pinput{PS_linear_combination_game}
\pinput{PS_bracket_good_count} %spring13 
\pinput{PS_linear_combination_by_structural_induction} %spring13
\pinput{PS_value_games} %spring13
\pinput{CP_binary_trees}

%State Machines
\pinput{PS_robot_on_2D_grid} %spring14
\pinput{PS_card_shuffle_state_machine} % appeared in spring14
\pinput{PS_2logb_mults}
\pinput{CP_state_machine_multiply}


%Recursive Data Structures

%Infinite sets
\pinput{CP_finite_strings_of_nonneg}
%\pinput{CP_recognizable_sets} % used as a class problem in cp5f
\pinput{FP_countable_quadratics}
\pinput{FP_countable_sets}
\pinput{FP_infinite_binary_sequences}
\pinput{FP_uncountable_infinite_sequences}
\pinput{FP_uncountable_ones}

\end{document}
