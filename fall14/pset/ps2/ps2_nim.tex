\documentclass[twoside,12pt]{article}
\newcommand{\tab}{\hspace*{2em}}
\usepackage{light}
\usepackage{subfigure}
\usepackage{graphicx}
\usepackage{amsmath}
\usepackage{verbatim}

\usepackage{amsfonts}

\newcommand{\lr}{l_{right}}
\renewcommand{\ll}{l_{left}}

\newcommand{\hint}[1]{({\it Hint: #1})}
\newcommand{\card}[1]{\left|#1\right|}
\newcommand{\union}{\cup}
\newcommand{\lgunion}{\bigcup}
\newcommand{\intersect}{\cap}
\newcommand{\lgintersect}{\bigcap}
\newcommand{\cross}{\times}


%\hidesolutions
\showsolutions

\newlength{\strutheight}
\newcommand{\prob}[1]{\mathop{\textup{Pr}} \nolimits \left( #1 \right)}
\newcommand{\prsub}[2]{\mathop{\textup{Pr}_{#1}}\nolimits\left(#2\right)}
\newcommand{\prcond}[2]{%
  \ifinner \settoheight{\strutheight}{$#1 #2$}
  \else    \settoheight{\strutheight}{$\displaystyle#1 #2$} \fi %
  \mathop{\textup{Pr}}\nolimits\left(
    #1\,\left|\protect\rule{0cm}{\strutheight}\right.\,#2
  \right)}
\newcommand{\cE}{\mathcal{E}}
\renewcommand{\setminus}{-}
\renewcommand{\complement}[1]{\overline{#1}}

\providecommand{\abs}[1]{\lvert#1\rvert}

\begin{document}
\problemset{2}{September 9, 2014}{Monday, September 15}


%%%%%%%%%%%%%%%%%%%%%%%%%%%%%%%%%%%%%%%%%%%%%%%%%%%%

\begin{problem}{30}
Nim is a game played between two players with three piles of stones. Players
alternate removing stones. A player picks a pile and removes any positive number
of stones. The goal is to be the last player to take a stone.
\bparts

\ppart{5} The winning strategy in Nim requires computing a Nim sum. A Nim sum
is defined to be the binary xor of the number of stones in each pile. Prove that
if the Nim sum is zero, then any move will result in Nim sum that is not zero.

\solution{
When a player removes stones from a pile, the binary representation must change
in at least one digit, otherwise the number of stones would be the same. At that
digit, the xor can no longer be zero since the bit has flipped. The Nim sum is
then non-zero.
}

\ppart{5} Prove that if the Nim sum is not zero that it is always possible to
make the Nim sum zero with one move.

\solution{
Let the Nim sum be $t$.  Let $d$ be the position of the most significant bit of
the Nim sum. One of the piles must have this bit non-zero, pick one of them. It
has $s$ stones remaining. Remove stones so that it has $s \otimes t$ stones
remaining. This is always possible because bit $d$ will be zero in $s \otimes
t$, so it will be less than $s$. This results in a Nim sum of zero because the
Nim sum of the other two piles stays the same, $s \otimes t$ and the third pile
has $s \otimes t$ stones left, so the Nim sum is zero.
}

\ppart{10} Using parts a and b, prove that if the game begins with a non-zero
Nim sum, then the first player has a winning strategy.

\solution{
The first player can remove stones such that the Nim sum is zero. Either this
results in no stones left or the second player must make a move that results
in a non-zero Nim sum. There must be at least one stone left because the Nim
sum is non-zero, so the second player has not won. This is the same situation
as the start except there are less stones. The result follows from induction
on the number of stones.
}

\eparts
\end{problem}


\end{document}
