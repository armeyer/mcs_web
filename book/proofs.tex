%Intro to Part I

\iffalse

\chapter{What is a Proof?}\label{proofs_chap}

\newtheorem{method}{Method}

\section{Introduction}

Proofs are the part of Math that Computer Science students

This is a Math text aimed at students of Computer Science, and the
topics we cover were selected because of their connection to Computer
Science.  
 for students to Computer Scientists need who mostly
chose not to be Math majors, even though they did just fine in High
School Math and introductory calculus.
\fi

\section*{Mathematical Proofs}

This text is all about methods for constructing and understanding
proofs.  In fact, we could have titled the book \emph{Proofs, Proofs,
and More Proofs}.  We will begin in Part~I with a description of basic
proof techniques.  We then apply these techniques in
chapter~\ref{number_theory_chap} to establish some very important
facts about numbers, facts that form the underpinning of the world's
most widely used cryptosystem.

Simply put, a proof is a method of establishing truth.  Like beauty,
``truth'' sometimes depends on the eye of the beholder, however, and
it should not be surprising that what constitutes a proof differs
among fields.  For example, in the judicial system, \emph{legal} truth
is decided by a jury based on the allowable evidence presented at
trial.  In the business world, \emph{authoritative} truth is specified
by a trusted person or organization, or maybe just your boss.  In
fields such as physics and biology, \emph{scientific}
truth\footnote{Actually, only scientific
\emph{falsehood} can really be demonstrated by an experiment---when
the experiment fails to behave as predicted.  But no amount of
experiment can confirm that the \emph{next} experiment won't fail.
For this reason, scientists rarely speak of truth, but rather
of \emph{theories} that accurately predict past, and anticipated
future, experiments.} is confirmed by experiment.  In
statistics, \emph{probable} truth is established by statistical
analysis of sample data.

\emph{Philosophical} proof involves careful exposition and
persuasion typically based on a series of small, plausible arguments.
The best example begins with ``Cogito ergo sum,'' a Latin sentence
that translates as ``I think, therefore I am.''  It comes from the
beginning of a 17th century essay by the mathematician/philosopher,
Ren\'e Descartes, and it is one of the most famous quotes in the
world: do a web search on the phrase and you will be flooded with
hits.

Deducing your existence from the fact that you're thinking about your
existence is a pretty cool and persuasive-sounding idea.
However, with just a few more lines of argument in this vein, Descartes
\href{http://www.btinternet.com/~glynhughes/squashed/descartes.htm}{goes
  on} to conclude that there is an infinitely beneficent God.  Whether
or not you believe in a beneficent God, you'll probably agree that any
very short proof of God's existence is bound to be far-fetched.  So even
in masterful hands, this approach is not reliable.

Mathematics has its own specific notion of ``proof.''

\begin{definition*}
A \term{formal proof} of a \term{proposition} is a chain of \term{logical
deductions} leading to the proposition from a base set of \term{axioms}.
\end{definition*}

The three key ideas in this definition are highlighted: proposition,
logical deduction, and axiom.  These three ideas are explained in the
following chapters, beginning with propositions in
chapter~\ref{prop_chap}.  We will then provide \emph{lots} of examples
of proofs and even some examples of ``false proofs'' (\ie arguments
that look like a proof but that contain mis-steps, or deductions that
aren't so logical when examined closely).

\endinput
