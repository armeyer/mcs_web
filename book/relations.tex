\chapter{Relations and Partial Orders}\label{chap:partial_orders}

A relation is a mathematical tool for describing associations between
elements of sets.  Relations are widely used in computer science,
especially in databases and scheduling applications.  A relation can
be defined across many items in many sets, but in this text, we will
focus on \emph{binary} relations, which represent an association
between two items in one or two sets.

\section{Binary Relations}

\subsection{Definitions and Examples}

\begin{definition}\label{def:binary_relation}
Given sets $A$ and~$B$, a \term{binary relation} $R: A \to B$
\emph{from}\footnote{We also say that the relationship is
  \emph{between} $A$ and $B$, or \emph{on}~$A$ if $B = A$.} $A$ to~$B$
is a subset of $A \cross B$.  The sets $A$ and~$B$ are called the
\term{domain} and \term{codomain} of~$R$, respectively.  We commonly
use the notation $aRb$ or $a \sim_R b$ to denote that $(a, b) \in R$.
\end{definition}

A relation is similar to a function.  In fact, every function $f: A
\to B$ is a relation.  In general, the difference between a function
and a relation is that a relation might associate multiple elements
of$B$ with a single element of$A$, whereas a function can only
associate at most one element of~$B$ (namely, $f(a)$) with each
element $a \in A$.

We have already encountered examples of relations in earlier chapters.
For example, in Section~\ref{sexam}, we talked about a relation
between the set of men and the set of women where $mRw$ if man~$m$
likes woman~$w$.  In Section~\ref{sec:coloring}, we talked about a
relation on the set of MIT courses where $c_1 R c_2$ if the exams for
$c_1$ and~$c_2$ cannot be given at the same time.  In
Section~\ref{sec:comm_nets}, we talked about a relation on the set of
switches in a network where $s_1 R s_2$ if $s_1$ and~$s_2$ are
directly connected by a wire.  We did not use the formal definition of
a relation in any of these cases, but they are all examples of
relations.

As another example, we can define an ``in charge of'' relation~$T$ for
Spring'10 from the set of MIT faculty~$F$ to the set of subject
numbers in the 2010 catalogue.  This relation contains pairs of the
form
\[
    (\ang{\text{instructor-name}}, \ang{\text{subject-num}})
\]
where the faculty member named $\ang{\text{instructor-name}}$ is in
charge of the subject with number $\ang{\text{subject-num}}$ in Spring
'10.  So $T$ contains pairs like thos shown in Figure~\ref{fig:FA}.

\begin{figure}

\begin{tabular}{ll}
(Meyer,    & 6.042),\\
(Meyer,    & 18.062),\\
(Meyer,    & 6.844),\\
(Leighton, & 6.042),\\
(Leighton, & 18.062),\\
(Freeman,  & 6.011),\\
(Freeman,  & 6.881)\\
(Freeman,  & 6.882)\\
(Freeman,  & 6.UAT)\\
(Eng,      & 6.UAT)\\
(Guttag,   & 6.00)
\end{tabular}

\caption{Some items in the ``in charge of'' relation~$T$ between
  faculty and subject numbers.}

\label{fig:FA}

\end{figure}

This is a surprisingly complicated relation: Meyer is in charge of
subjects with three numbers.  Leighton is also in charge of subjects
with two of these three numbers---because the same subject,
Mathematics for Computer Science, has two numbers: 6.042 and 18.062,
and Meyer and Leighton are co-in-charge of the subject.  Freeman is
in-charge of even more subjects numbers (around 20), since as
Department Education Officer, he is in charge of whole blocks of
special subject numbers.  Some subjects, like 6.844 and 6.00 have only
one person in-charge.  Some faculty, like Guttag, are in charge of
only one subject number, and no one else is co-in-charge of his
subject, 6.00.

Some subjects in the codomain, $N$, do not appear in the list ---that
is, they are not an element of any of the pairs in the graph of $T$;
these are the Fall term only subjects.  Similarly, there are faculty
in the domain, $F$, who do not appear in the list because all their
in-charge subjects are Fall term only.

\subsection{Representation as a Bipartite Graph}



\endinput
