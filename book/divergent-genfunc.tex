Let $F(x)$ be the generating function for $n!$, that is,
\[
F(x) \eqdef 1 + 1x + 2x^2 + \cdots + n! x^n + \cdots.
\]
Because $x^n = o(n!)$ for all $x \neq 0$, this generating function
converges only at $x=0$.  Nevertheless, purely by manipulating terms,
we can conclude that the generating function for $(n+1)!$ is simply
$(F(x)-1)/x$.

Now a little further formal reasoning about generating functions will
allow us to deduce the following unexpected identity for $n!$:
\begin{equation}\label{n!sum}
n! = 1 + \sum_{i=1}^{n} (i-1) \cdot (i-1)!
\end{equation}

To prove~\eqref{n!sum}, let $H(x)$ be the generating function for $n
\cdot n!$, that is, 
\[
H(x) \eqdef 0 + 1x + 4x^2 + \cdots + n\cdot n! x^n + \cdots.
\]
Noting that 
\[
[x^n]H(x) \eqdef n \cdot n! = (n+1)! - n!,
\]
we conclude that
\begin{equation}\label{HF-1}
H(x) = \frac{F(x) - 1}{x} - F(x).
\end{equation}

Solving~\eqref{HF-1} for $F(x)$, we get
\begin{equation}\label{FxH1}
F(x) = \frac{xH(x)+1}{1-x}.
\end{equation}
But $[x^n](xH(x)+1)$ is $(n-1) \cdot (n-1)!$ for $n \geq 1$ and is 1
for $n=0$, so by the convolution formula,
\[
[x^n]\frac{xH(x)+1}{1-x} = 1 + \sum_{i=1}^{n} (i-1) \cdot (i-1)! \, .
\]
The identity~\eqref{n!sum} now follows immediately from~\eqref{FxH1}.
