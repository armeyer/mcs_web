\chapter{Generating Functions}\label{generating_function_chap}

\idx{Generating Functions} are one of the most surprising and useful
inventions in Discrete Mathematics.  Roughly speaking, generating
functions transform problems about \textit{sequences} into problems
about \textit{functions}.  This is great because we've got piles of
mathematical machinery for manipulating functions.  Thanks to
generating functions, we can apply all that machinery to problems
about sequences.  In this way, we can use generating functions to
solve all sorts of counting problems.  

Several flavors of generating functions ---\emph{ordinary},
\emph{exponential}, and \emph{Dirichlet} ---regularly come up in
combinatorial mathematics.  In addition, \emph{Z-transforms}, which
are closely related to ordinary generating functions, are important in
control theory and signal processing.  But ordinary generating
functions are enough to illustrate the power of the idea, so we'll
stick to them.  So from now on \emph{generating function} will mean
the ordinary kind, and we will offer a taste of this large subject by
showing how generating fucntions can be used to solve some certain
kinds of counting problems and how they can be used to find simple
formulas for \emph{linear-recursive} functions.

\section{Infinite Series}
Informally, a generating function, $F(x)$, is an infinite series
\begin{equation}\label{def:Fxf_0}
F(x) = f_0 + f_1 x + f_2 x^2 + f_3 x^3 + \cdots.
\end{equation}
For example, the infinite \idx{geometric series}
\begin{equation}\label{def:geomseries}
G(x) \eqdef 1+x+x^2+\cdots+x^n+\cdots.
\end{equation}
is a familiar generating function, and we can illustrate typical
reasoning about generating functions by deriving a simple formula for
$G(x)$.  The approach is actually a simpler version of the
\idx{perturbation method} of Section~\ref{sec:perturbation}.  Namely,
\[
\begin{series}
       G(x) & = & 1 & + & x & + & x^2 & + & x^3 & + &\cdots & + & x^{n} & + &\cdots \cr
     -xG(x) & = &   & - & x & - & x^2 & - & x^3 & - &\cdots & - & x^{n} & - & \cdots \cr
     \hline\cr
G(x) -xG(x) & = & 1.
\end{series}
\]
Solving for~$G(x)$ gives 
\begin{equation}\label{eq:G1/1-x}
    \sum_{n=0}^\infty x^n = G(x) = \frac{1}{1 - x}.
\end{equation}

Continuing with this approach yields a nice formula for
\begin{equation}\label{def:nonnegintseries}
N(x) \eqdef 1+2x+3x^2+\cdots+(n+1)x^n+\cdots.
\end{equation}
Namely
\[
\begin{series}
              N(x) & = & 1 & + & 2x & + & 3x^2 & + & 4x^3 & + &\cdots & + & (n+1)x^{n} & + &\cdots \cr
            -xN(x) & = &   & - & x & - & 2x^2 & - & 3x^3 & - &\cdots & - & nx^{n} & - & \cdots\cr
            \hline\cr
       N(x) -xN(x) & = & 1 & + & x & + & x^2 & + & x^3 & + &\cdots & + & x^{n} & + &\cdots\cr
                   & = & G(x).
\end{series}
\]
Solving for~$N(x)$ gives 
\begin{equation}\label{eq:N1/1-x2}
    \sum_{n=0}^\infty (n+1)x^n = N(x) = \frac{G(x)}{1 - x} = \frac{1}{(1-x)^2}.
\end{equation}

We use the notation $[x^n]F(x)$ for the coefficient of $x^n$ in the
generating function $F(x)$.  That is, $[x^n]F(x) \eqdef f_n$ for
$F(x)$ given by equation~\eqref{def:Fxf_0}.  For example, we now have
\begin{align*}
[x^n]\paren{\frac{1}{1 - x}} & = 1\\
[x^n]\paren{\frac{1}{(1-x)^2}} & = n+1.
\end{align*}

\section{Counting with Generating Functions}\label{sec:gf_counting}
Generating functions are particularly useful for representing and
counting the number of ways to select $n$ things.  For example, if
there are two flavors of donuts ---chocolate and vanilla ---let $d_n$
be the number of ways to select $n$ chocolate or vanilla flavored
donuts.  So $d_n =n+1$ because there are $n+1$ such donut selctions,
namely, all chocolate, 1 vanilla and $n-1$ chocolate, 2 vanilla and
$n-2$ chocolate,\dots, all vanilla.  We define a generating function,
$D(x)$, for counting these donut selections by letting the
coefficient of $x^n$ be $d_n$.  So by equation~\eqref{eq:N1/1-x2}
\begin{equation}\label{2donutgen}
D(x) = \frac{1}{(1-x)^2}.
\end{equation}

More generally, suppose we have two kinds of things ---say apples and
bananas ---and some constraints on how many of each may be selected.
Say there are $a_n$ ways to select $n$ apples and $b_n$ ways to select
$n$ bananas.  So the generating function for counting apples would be
\[
A(x) \eqdef \sum_{n=0}^\infty a_nx^n,
\]
and for bananas would be
\[
B(x) \eqdef \sum_{n=0}^\infty b_nx^n.
\]

Now suppose apples come in baskets of 6, so there no way to select $1$
to $5$ apples, one way to select 6 apples, no way to select 7, \etc.
In other words,
\[
a_n = \begin{cases}
      1 & \text{if $n$ is a multiple of 6},\\
      0 & \text{otherwise}.
\end{cases}
\]
In this case we would have
\begin{align*}
A(x)
& = 1 + x^6 + x^{12} + \cdots + x^{6n} + \cdots\\
& = 1 + x^6 + \paren{x^6}^2 + \cdots +\paren{x^6}^n + \cdots\\
& = \frac{1}{1 - x^6}.
\end{align*}
Let's also suppose there are two kinds of bananas ---red and yellow.
Now $b_n = n+1$ by the same reasoning used to count selections of $n$
chocolate and vanilla donuts, so we would have
\[
B(x) = \frac{1}{(1-x)^2}.
\]

So how many ways are there to select a mix of $n$ apples and bananas?
We could select one apple in $a_1$ ways and then $n-1$ bananas in
$b_{n-1}$ ways, for a total of $a_1b_{n-1}$ ways to select $n$ apples
and bananas using only one apple.  More generally, we could select $k$
apples in $a_k$ ways and then $n-k$ bananas in $b_{n-k}$ ways, for a
total of $a_kb_{n-k}$ ways to select select $n$ apples and bananas
including exactly $k$ apples.  So the total number of ways to select a
mix of $n$ apples and bananas is
\begin{equation}\label{a0bnconvolve}
a_0b_n + a_1b_{n-1} + a_2b_{n-1} + \cdots + a_nb_0.
\end{equation}

Now here's the cool connection between counting and generating
functions: expression~\eqref{a0bnconvolve} is equal to the coefficient
of $x^n$ in the product $A(x)B(x)$.

\subsection{Products of Generating Functions}

In other words, we're claiming that
\begin{equation}\label{xnAxBx}
[x^n](A(x)\cdot B(x)) = a_0b_n + a_1b_{n-1} + a_2b_{n-1} + \cdots + a_nb_0.
\end{equation}
To explain this formula, we can think about evaluating the product
$A(x) \cdot B(x)$ by using a table to identify all the cross-terms
from the product of the sums:
%
\[
\begin{array}{c|@{\quad}c@{\qquad}c@{\qquad}c@{\qquad}c@{\qquad}c}
      & b_0 x^0 & b_1 x^1 & b_2 x^2 & b_3 x^3 & \dots \\
\hline
\\
a_0 x^0 & a_0 b_0 x^0 & a_0 b_1 x^1 & a_0 b_2 x^2 & a_0 b_3 x^3 & \dots \\
\\
a_1 x^1 & a_1 b_0 x^1 & a_1 b_1 x^2 & a_1 b_2 x^3 & \dots \\
\\
a_2 x^2 & a_2 b_0 x^2 & a_2 b_1 x^3 & \dots \\
\\
a_3 x^3 & a_3 b_0 x^3 & \dots \\
\\
\vdots & \dots\\
\end{array}
\]
Notice that all terms involving the same power of $x$ lie on a
45-degree sloped diagonal.  So the diagonal index -$n$ contains all
the $x^n$-terms, and the coefficient of $x^n$ in the product
$A(x)\cdot B(x)$ is the sum of all the coefficients of the terms on
this diagonal, namely,~\eqref{a0bnconvolve}.  The sequence of
coefficients of the product $A(x)\cdot B(x))$ is called the
\term{convolution} of the sequences $(a_0, a_1, a_2, \dots)$ and
$(b_0, b_1, b_2, \dots)$.  In addition to their algebraic role,
convolutions of sequences play a prominent role in signal processing
and control theory.

\subsection{The Convolution Rule}

We can summarize the discussion above with the
\begin{mathrule*}[\idx{Convolution}]\label{convolution_rule}
Let $A(x)$ be the generating function for selecting items from a set
${\cal A}$, and let $B(x)$ be the generating function for selecting
items from a set ${\cal B}$ disjoint from ${\cal A}$.  The generating
function for selecting items from the union ${\cal A} \cup {\cal B}$
is the product $A(x) \cdot B(x)$.
\end{mathrule*}

The Rule depends on a precise definition of what ``selecting selecting
items from the union ${\cal A} \cup {\cal B}$'' means.  Informally,
the idea is that the restrictions on the selection of items from sets
${\cal A}$ and ${\cal B}$ carry over to selecting items from ${\cal A}
\cup {\cal B}$.  Formally, the Convolution Rule applies when there is
a bijection between $n$-element selections from ${\cal A} \cup {\cal
  B}$ and ordered pairs of selections from the sets ${\cal A}$ and
${\cal B}$ containing a total of $n$ elements.  We think the informal
statement is clear enough.

\subsection{Applying the Convolution Rule}

We can use the Convolution Rule to derive in another way the
generating function $D(x)$ for the number of ways to select chocolate
and vanilla donuts given in~\eqref{2donutgen}.  Namely, there is only
one way to select exactly $n$ chocolate donuts.  That means every
coefficient of the generating function for selecting $n$ chocolate
donuts equals one.  So the generating function for chocolate donut
selections is $1/(1-x)$; likewise for the generating function for
selecting only vanilla donuts.  Now by the Convolution Rule, the
generating function for the number of ways to select $n$ donuts when
both chocolate and vanilla flavors are available is
\[
D(x) = \frac{1}{1-x} \cdot \frac{1}{1-x} = \frac{1}{(1-x)^2}.
\]
So we have derived~\eqref{2donutgen} without appeal
to~\eqref{eq:N1/1-x2}.

The first general counting problem we considered was the number of
ways to select a $n$ doughnuts when $k$ flavors were available.  Our
application of the Convolution Rule for two flavors carries right over
to this general case, and we conclude that the generating function for
selections of donuts when $k$ flavors are available is $1/(1=x)^k$.
So we have
\begin{equation}\label{eq:donut_coefficient}
[x^n]\paren{\frac{1}{(1-x)^k}} = \binom{n+(k-1)}{n}
\end{equation}
by Corollary~\ref{cor:donut_binom}.

\subsection{Extracting Coefficients from Maclauren's Theorem}

We've used a donut-counting argument to derive the coefficients of
$1/(1-x)^k$, but it's instructive to derive this coefficient
algebraically, which we can do using Maclauren's Theorem:
\begin{theorem}[Maclauren's Theorem]
\[
f(x) = f(0) + f'(0) x + \frac{f''(0)}{2!} x^2 + \frac{f'''(0)}{3!} x^3 + \cdots
+ \frac{f^{(n)}(0)}{n!} x^n + \cdots.
\]
\end{theorem}

This theorem says that the $n$th coefficient of $1 / (1 - x)^k$ is
equal to its $n$th derivative evaluated at 0 and divided by $n!$.
Computing the $n$th derivative turns out not to be very difficult
\[
\frac{d}{dx}^n \frac{1}{(1-x)^k} = k (k+1) \cdots (k + n - 1)(1-x)^{-(k+n)}
\]
(see Problem~\bref{CP_nth_derivative_of_A}), so
\begin{align*}
[x^n]\paren{\frac{1}{(1-x)^k}}
  & = \paren{\frac{d^n}{d^n x} \frac{1}{(1-x)^k}}(0)\frac{1}{n!}\\
  & = \frac{k (k+1) \cdots (k + n - 1)(1-0)^{-(k+n)}}{n!}\\
  & = \binom{n+ (k-1)}{n}.
\end{align*}
So instead of using the donut-counting
formula~\eqref{eq:donut_coefficient} to find the coefficients of
$x^n$, we could have used this algebraic argument and the Convolution
Rule to derive the donut-counting formula.

\begin{editingnotes}
Let
%
\[
G(x) \eqdef \frac{1}{(1-x)^k} = (1-x)^{-k}.
\]
%
Then we have:
%
\begin{align*}
G'(x) & = k (1-x)^{-(k+1)} \\
G''(x) & = k (k+1) (1-x)^{-(k+2)} \\
G'''(x) & = k (k+1) (k+2) (1-x)^{-(k+3)} \\
G^{(n)}(x) & = k (k+1) \cdots (k + n - 1)(1-x)^{-(k+n)}
\end{align*}
%
Thus, the coefficient of $x^n$ in the generating function is:
%
\begin{align*}
G^{(n)}(0) / n! & = \frac{k (k+1) \cdots (k + n - 1)}{n!} \\
                & = \frac{(k + n - 1)!}{(k - 1)! \ n!} \\
                & = \binom{n + k - 1}{n}.
\end{align*}

\end{editingnotes}

\begin{problems}
\practiceproblems
\pinput{MQ_bouquet}
\pinput{TP_Generating_function_of_a_recurrence}
\pinput{TP_Generating_Functions_and_Sequences}

\classproblems
\pinput{CP_nth_derivative_of_A}
\pinput{CP_bag_of_donuts}
\pinput{CP_gen_func_sum_of_squares}

\homeworkproblems
\pinput{PS_gen_fcns_pennies_nickels_etc}

\examproblems
\pinput{MQ_binomial_coef}

\end{problems}

\section{An ``Impossible'' Counting Problem}
\label{sec:impossible_counting}

\begin{editingnotes}

Not so impossible.  From Rebecca Freund, F09:

Note that the fruits can be divided into two groups, the apples-and-pears
and the bananas-and-oranges. Once you know how many are apples-and-pears,
there's only one way to distribute them: Use a pear if the number is odd,
otherwise don't. Make the rest apples. Similarly, once you've decided on
the number of bananas-and-oranges, you have to throw in
the-greatest-multiple-of-five-less-than-or-equal-to-that bananas and add
oranges as needed. So the number of apples-and-pears exactly determines
the arrangement. You can have 0-n apples-and-pears, so there are n+1
possibilities.

\end{editingnotes}

So far everything we've done with generating functions we could have
done another way.  But here is an absurd counting problem ---really
over the top!  In how many ways can we fill a bag with $n$ fruits
subject to the following constraints?

\begin{itemize}
\item The number of apples must be even.
\item The number of bananas must be a multiple of 5.
\item There can be at most four oranges.
\item There can be at most one pear.
\end{itemize}

For example, there are 7 ways to form a bag with 6 fruits:
%
\[
\begin{array}{c|ccccccc}
\text{Apples}  & 6 & 4 & 4 & 2 & 2 & 0 & 0 \\
\text{Bananas} & 0 & 0 & 0 & 0 & 0 & 5 & 5 \\
\text{Oranges} & 0 & 2 & 1 & 4 & 3 & 1 & 0 \\
\text{Pears}   & 0 & 0 & 1 & 0 & 1 & 0 & 1
\end{array}
\]
% These constraints are so complicated that getting a nice answer may
seem impossible.  But let's see what generating functions reveal.

Let's first construct a generating function for choosing apples.  We
can choose a set of 0 apples in one way, a set of 1 apple in zero
ways (since the number of apples must be even), a set of 2 apples in
one way, a set of 3 apples in zero ways, and so forth.  So we have:
%
\[
A(x) = 1 + x^2 + x^4 + x^6 + \cdots = \frac{1}{1 - x^2}
\]
%
Similarly, the generating function for choosing bananas is:
%
\[
B(x) = 1 + x^5 + x^{10} + x^{15} + \cdots = \frac{1}{1 - x^5}
\]
Now, we can choose a set of 0 oranges in one way, a set of 1 orange in
one way, and so on.  However, we can not choose more than four
oranges, so we have the generating function:
%
\[
O(x) = 1 + x + x^2 + x^3 + x^4 = \frac{1-x^5}{1-x}
\]
Here we're using the formula~\eqref{geometric-sum-n} for a finite
geometric sum.  Finally, we can choose only zero or one pear, so we
have:
%
\[
P(x) = 1 + x
\]

The Convolution Rule says that the generating function for choosing
from among all four kinds of fruit is:
%
\begin{align*}
A(x) B(x) O(x) P(x)
    & = \frac{1}{1-x^2} \frac{1}{1-x^5} \frac{1-x^5}{1-x} (1 + x) \\
    & = \frac{1}{(1-x)^2} \\
    & = 1 + 2x + 3x^2 + 4 x^3 + \cdots
\end{align*}
%
Almost everything cancels!  We're left with $1 / (1-x)^2$, which we
found a power series for earlier: the coefficient of $x^n$ is simply
$n+1$.  Thus, the number of ways to form a bag of $n$ fruits is just
$n+1$.  This is consistent with the example we worked out, since there
were 7 different fruit bags containing 6 fruits.  \textit{Amazing!}

\begin{problems}

\homeworkproblems
\pinput{PS_crazy_pet_lady}

% generating functions, counting
%\pinput{PS_gen_fcns_with_donuts}

% generating functions
%\pinput{PS_Catalan_numbers_meyer_version}

\examproblems
\pinput{FP_boat_trip}

\end{problems}

\endinput
