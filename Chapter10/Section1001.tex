\section{Monty Hall}

In the September 9, 1990 issue of \textit{Parade} magazine, the
columnist Marilyn vos Savant responded to this letter:

%  future:
%  Include the other problem that Marilyn addressed in the same issue.

\begin{quotation}
\noindent \textit{Suppose you're on a game show, and you're given the
choice of three doors.  Behind one door is a car, behind the others,
goats.  You pick a door, say number 1, and the host, who knows what's
behind the doors, opens another door, say number 3, which has a goat.
He says to you, "Do you want to pick door number 2?"  Is it to your
advantage to switch your choice of doors?}

\vspace{1ex}

\hspace{3in} Craig. F. Whitaker

\hspace{3in} Columbia, MD
\end{quotation}

The letter describes a situation like one faced by contestants on the
1970's game show \textit{Let's Make a Deal}, hosted by Monty Hall and
Carol Merrill.  Marilyn replied that the contestant should indeed switch.
She explained that if the car was behind either of the two unpicked doors
---which is twice as likely as the the car being behind the picked door
---the contestant wins by switching.  But she soon received a torrent of
letters, many from mathematicians, telling her that she was wrong.  The
problem generated thousands of hours of heated debate.

This incident highlights a fact about probability: the subject uncovers
lots of examples where ordinary intuition leads to completely wrong
conclusions.  So until until you've studied probabilities enough to have
refined your intuition, a way to avoid errors is to fall back on a
rigorous, systematic approach such as the Four Step Method.

\subsection{The Four Step Method}

Every probability problem involves some sort of randomized experiment,
process, or game.  And each such problem involves two distinct
challenges:
%
\begin{enumerate}
\item How do we model the situation mathematically?
\item How do we solve the resulting mathematical problem?
\end{enumerate}
%
In this section, we introduce a four step approach to questions of the
form, ``What is the probability that ----- ?''  In this approach, we build
a probabilistic model step-by-step, formalizing the original question in
terms of that model.  Remarkably, the structured thinking that this
approach imposes provides simple solutions to many famously-confusing
problems.  For example, as you'll see, the four step method cuts through
the confusion surrounding the Monty Hall problem like a Ginsu knife.
However, more complex probability questions may spin off challenging
counting, summing, and approximation problems--- which, fortunately,
you've already spent weeks learning how to solve.

\subsection{Clarifying the Problem}

Craig's original letter to Marilyn vos Savant is a bit vague, so we
must make some assumptions in order to have any hope of modeling the
game formally:
%
\begin{enumerate}

\item The car is equally likely to be hidden behind each of the three
doors.

\item The player is equally likely to pick each of the three doors,
regardless of the car's location.

\item After the player picks a door, the host {\em must} open a
different door with a goat behind it and offer the player the choice
of staying with the original door or switching.

\item If the host has a choice of which door to open, then he is
equally likely to select each of them.

\end{enumerate}
%
In making these assumptions, we're reading a lot into Craig Whitaker's
letter.  Other interpretations are at least as defensible, and some
actually lead to different answers.  But let's accept these
assumptions for now and address the question, ``What is the
probability that a player who switches wins the car?''

\subsection{Step 1:  Find the Sample Space}

Our first objective is to identify all the possible outcomes of the
experiment.  A typical experiment involves several randomly-determined
quantities.  For example, the Monty Hall game involves three such
quantities:
%
\begin{enumerate}
\item The door concealing the car.
\item The door initially chosen by the player.
\item The door that the host opens to reveal a goat.
\end{enumerate}
%
Every possible combination of these randomly-determined quantities is
called an \term{outcome}.  The set of all possible outcomes is called
the \term{sample space} for the experiment.

A \term{tree diagram} is a graphical tool that can help us work
through the four step approach when the number of outcomes is not too
large or the problem is nicely structured.  In particular, we can use
a tree diagram to help understand the sample space of an experiment.
The first randomly-determined quantity in our experiment is the door
concealing the prize.  We represent this as a tree with three
branches:
\begin{center}
\mfigure{!}{2.5in}{figures/monty1}
\end{center}
In this diagram, the doors are called $A$, $B$, and $C$ instead of 1, 2,
and 3 because we'll be adding a lot of other numbers to the picture later.

Now, for each possible location of the prize, the player could initially
choose any of the three doors.  We represent this in a second layer added to
the tree.  Then a third layer represents the possibilities of the final step when
the host opens a door to reveal a goat:
\iffalse
\begin{center}
\mfigure{!}{3.3in}{monty2}
\end{center}
\fi 

\begin{center}
\mfigure{!}{4.0in}{figures/monty3}
\end{center}

Notice that the third layer reflects the fact that the host has either one
choice or two, depending on the position of the car and the door initially
selected by the player.  For example, if the prize is behind door A and
the player picks door B, then the host must open door C.  However, if the
prize is behind door A and the player picks door A, then the host could
open either door B or door C.

Now let's relate this picture to the terms we introduced earlier: the
leaves of the tree represent \textit{outcomes} of the experiment, and
the set of all leaves represents the \textit{sample space}.  Thus, for
this experiment, the sample space consists of 12 outcomes.  For
reference, we've labeled each outcome with a triple of doors
indicating:
%
\[
(\text{door concealing prize}, \ \text{door initially chosen}, \ \text{door opened to reveal a goat})
\]
%
In these terms, the sample space is the set:
%
\[
\set{
\begin{array}{cccccc}
(A, A, B), & (A, A, C), & (A, B, C), & (A, C, B), & (B, A, C), & (B, B, A), \\
(B, B, C), & (B, C, A), & (C, A, B), & (C, B, A), & (C, C, A), & (C, C, B)
\end{array}}
\]
%
The tree diagram has a broader interpretation as well: we can regard the
whole experiment as following a path from the root to a leaf, where the
branch taken at each stage is ``randomly'' determined.  Keep this
interpretation in mind; we'll use it again later.

\subsection{Step 2: Define Events of Interest}

Our objective is to answer questions of the form ``What is the probability
that \dots ?'', where the missing phrase might be ``the player wins by
switching'', ``the player initially picked the door concealing the
prize'', or ``the prize is behind door C'', for example.  Each of these
phrases characterizes a set of outcomes.  For example, the outcomes
specified by ``the prize is behind door $C$' is:
%
\[
\set{(C, A, B), (C, B, A), (C, C, A), (C, C, B)}
\]
%
A set of outcomes is called an \term{event}.  So the event that the player
initially picked the door concealing the prize is the set:
%
\[
\set{(A, A, B), (A, A, C), (B, B, A), (B, B, C), (C, C, A), (C, C, B)}
\]
%
And what we're really after, the event that the player wins by
switching, is the set of outcomes:
%
\[
\set{(A, B, C), (A, C, B), (B, A, C), (B, C, A), (C, A, B), (C, B, A)}
\]
%
%These are the outcomes labelled with ``X'' in the diagram above.
%\iffalse
Let's annotate our tree diagram to indicate the outcomes in this
event.
%
\begin{center}
\mfigure{!}{3.5in}{figures/monty4}
\end{center}
%\fi
Notice that exactly half of the outcomes are marked, meaning that the
player wins by switching in half of all outcomes.  You might be
tempted to conclude that a player who switches wins with probability
$1/2$.  \textit{This is wrong.}  The reason is that these
outcomes are not all equally likely, as we'll see shortly.

\subsection{Step 3: Determine Outcome Probabilities}

So far we've enumerated all the possible outcomes of the experiment.  Now
we must start assessing the likelihood of those outcomes.  In particular,
the goal of this step is to assign each outcome a probability, indicating
the fraction of the time this outcome is expected to occur.  The sum of
all outcome probabilities must be one, reflecting the fact that there
always is an outcome.

Ultimately, outcome probabilities are determined by the phenomenon
we're modeling and thus are not quantities that we can derive
mathematically.  However, mathematics can help us compute the
probability of every outcome \textit{based on fewer and more
elementary modeling decisions.}  In particular, we'll break the task
of determining outcome probabilities into two stages.

\subsubsection{Step 3a: Assign Edge Probabilities}

First, we record a probability on each \textit{edge} of the tree
diagram.  These edge-probabilities are determined by the assumptions
we made at the outset: that the prize is equally likely to be behind
each door, that the player is equally likely to pick each door, and
that the host is equally likely to reveal each goat, if he has a
choice.  Notice that when the host has no choice regarding which door
to open, the single branch is assigned probability 1.
%
\begin{center}
\mfigure{!}{3.5in}{figures/monty5}
\end{center}

\subsubsection{Step 3b: Compute Outcome Probabilities}

Our next job is to convert edge probabilities into outcome
probabilities.  This is a purely mechanical process: \textit{the
probability of an outcome is equal to the product of the
edge-probabilities on the path from the root to that outcome}.  For
example, the probability of the topmost outcome, $(A, A, B)$ is
\[
\frac{1}{3} \cdot \frac{1}{3} \cdot \frac{1}{2} = \frac{1}{18}.
\]

There's an easy, intuitive justification for this rule.  As the steps in
an experiment progress randomly along a path from the root of the tree to
a leaf, the probabilities on the edges indicate how likely the walk is to
proceed along each branch.  For example, a path starting at the root in
our example is equally likely to go down each of the three top-level
branches.

Now, how likely is such a walk to arrive at the topmost outcome, $(A,
A, B)$?  Well, there is a 1-in-3 chance that a walk would follow the
$A$-branch at the top level, a 1-in-3 chance it would continue along
the $A$-branch at the second level, and 1-in-2 chance it would follow
the $B$-branch at the third level.  Thus, it seems that about 1 walk
in 18 should arrive at the $(A, A, B)$ leaf, which is precisely the
probability we assign it.

Anyway, let's record all the outcome probabilities in our tree
diagram.
%
\begin{center}

\mfigure{!}{3.5in}{figures/monty6}
\end{center}

Specifying the probability of each outcome amounts to defining a
function that maps each outcome to a probability.  This function is
usually called \textbf{Pr}.  In these terms, we've just determined
that:
%
\begin{align*}
\pr{(A, A, B)} & = \frac{1}{18} \\
\pr{(A, A, C)} & = \frac{1}{18} \\
\pr{(A, B, C)} & = \frac{1}{9} \\
               & \text{etc.}
\end{align*}

\subsection{Step 4: Compute Event Probabilities}

We now have a probability for each \textit{outcome}, but we want to
determine the probability of an \textit{event} which will be the sum of
the probabilities of the outcomes in it.  The probability of an event,
$E$, is written $\pr{E}$.  For example, the probability of the event that
the player wins by switching is:
%
\begin{align*}
\pr{\text{switching wins}}
    & = \pr{(A, B, C)} + \pr{(A, C, B)} + \pr{(B, A, C)} + \\
    & \qquad \pr{(B, C, A)} + \pr{(C, A, B)} + \pr{(C, B, A)} \\
    & = \frac{1}{9} + \frac{1}{9} + \frac{1}{9} + 
        \frac{1}{9} + \frac{1}{9} + \frac{1}{9} \\
    & = \frac{2}{3}
\end{align*}
%
It seems Marilyn's answer is correct; a player who switches doors wins
the car with probability $2/3$!  In contrast, a player who stays with
his or her original door wins with probability $1/3$, since staying
wins if and only if switching loses.

We're done with the problem!  We didn't need any appeals to intuition
or ingenious analogies.  In fact, no mathematics more difficult than
adding and multiplying fractions was required.  The only hard part was
resisting the temptation to leap to an ``intuitively obvious'' answer.

\subsection{An Alternative Interpretation of the Monty Hall Problem}

Was Marilyn really right?  Our analysis suggests she was.  But a more
accurate conclusion is that her answer is correct \textit{provided we
accept her interpretation of the question}.  There is an equally
plausible interpretation in which Marilyn's answer is wrong.  Notice
that Craig Whitaker's original letter does not say that the host is
\textit{required} to reveal a goat and offer the player the option to
switch, merely that he \textit{did} these things.  In fact, on the
\textit{Let's Make a Deal} show, Monty Hall sometimes simply opened
the door that the contestant picked initially.  Therefore, if he
wanted to, Monty could give the option of switching only to
contestants who picked the correct door initially.  In this case,
switching never works!

\endinput