\documentclass[handout]{mcs}

\begin{document}

\inclassproblems{5, Fri.}

\begin{staffnotes}
Recursive Data, Ch. 6
\end{staffnotes}

%%%%%%%%%%%%%%%%%%%%%%%%%%%%%%%%%%%%%%%%%%%%%%%%%%%%%%%%%%%%%%%%%%%%%
% Problems start here
%%%%%%%%%%%%%%%%%%%%%%%%%%%%%%%%%%%%%%%%%%%%%%%%%%%%%%%%%%%%%%%%%%%%%

\pinput{CP_recursive_prop_form_eval}

\begin{quote}
\textbf{ The problem on two-person games of perfect information that
  was originally Class Problem 2 has been moved to be
  \href{https://courses.csail.mit.edu/6.042/spring16/ps4.pdf}{Pset 4},
  Problem 1.  Time was needed for the peer-grading discussions at the
  start of class, and no teams later had time to work in this
  informative, challenging problem.}
\end{quote}

%\pinput{CP_VG}

\pinput{CP_recursive_prenex}

\begin{center}
\textbf{Supplemental problem: (carried over from class Wed., Mar 2, 2016)}
\end{center}

\pinput{CP_recursively_defined_sets}

%\pinput{FP_structural_induction_arithmetic_expressions}


%%%%%%%%%%%%%%%%%%%%%%%%%%%%%%%%%%%%%%%%%%%%%%%%%%%%%%%%%%%%%%%%%%%%%
% Problems end here
%%%%%%%%%%%%%%%%%%%%%%%%%%%%%%%%%%%%%%%%%%%%%%%%%%%%%%%%%%%%%%%%%%%%%

\end{document}
