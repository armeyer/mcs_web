\documentclass[handout]{mcs}

\begin{document}

\inclassproblems{6, Wed.}

%%%%%%%%%%%%%%%%%%%%%%%%%%%%%%%%%%%%%%%%%%%%%%%%%%%%%%%%%%%%%%%%%%%%%
% Problems start here
%%%%%%%%%%%%%%%%%%%%%%%%%%%%%%%%%%%%%%%%%%%%%%%%%%%%%%%%%%%%%%%%%%%%%
\begin{staffnotes}
	Halting Problem;  Ch. 8.2
\end{staffnotes}

\pinput{CP_computable_reducibility}

\pinput{CP_Schroeder_Bernstein_theorem}

%busy beaver

%program size?

%%%%%%%%%%%%%%%%%%%%%%%%%%%%%%%%%%%%%%%%%%%%%%%%%%%%%%%%%%%%%%%%%%%%%
% Problems end here
%%%%%%%%%%%%%%%%%%%%%%%%%%%%%%%%%%%%%%%%%%%%%%%%%%%%%%%%%%%%%%%%%%%%%

\end{document}


\iffalse

\begin{problem}
Briefly Explain why the validity problem for pure predicate logic is
undecidable as a consequence of the material presented in the logic of
the counter machine lecture at
\href{https://courses.csail.mit.edu/6.042/spring16/predicate-logic7.pdf}{Reducing
  Counter Machine Halting to Validity}.

\begin{solution}

\end{solution}
\end{problem}
\fi
