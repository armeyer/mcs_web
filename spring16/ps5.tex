\documentclass[handout]{mcs}

\begin{document}

\renewcommand{\reading}{
\begin{itemize}
\item Chapter~\bref{number_theory_chap}.\ \emph{Number Theory} through
Section~\bref{sec:inverse}.
\item Chapter~\bref{digraphs_chap}.\ \emph{Digraphs}, (Section~\bref{poset-as-sets_sec} optional).
\end{itemize}}

\problemset{5}

%%%%%%%%%%%%%%%%%%%%%%%%%%%%%%%%%%%%%%%%%%%%%%%%%%%%%%%%%%%%%%%%%%%%%
% Problems start here
%%%%%%%%%%%%%%%%%%%%%%%%%%%%%%%%%%%%%%%%%%%%%%%%%%%%%%%%%%%%%%%%%%%%%
\begin{staffnotes}
\begin{verbatim}
Set Theory                           Ch.8.4-8.5
Number Theory-GCD's                  Ch.9-9.4
Number Theory-Congruences            Ch.9.5-9.9
Digraphs & Scheduling                Ch.10.5
Directed Walks & Paths               Ch.10-10.4
Partial orders & equivalence         Ch.10.6, 10.8-10.11 (optional 10.7)

    Sets - halting problem, Set Theory

I honestly do not think that we should add a problem about sets. A lot
of class time has been/is being spent on this, and there is so much
other material to be covered on this problem set.

    Number Theory - GCD’s

        S14, PS4, Problem 2

Comment: I think this problem is short and clear,
reinforcing the key knowledge of the number theory GCD
material.

    Number Theory - Congruences

        S13, PS5, Problem 2

Comment: I think this problem provides some insight, but
not in the same succinct way as problem 3.

DO NOT USE     S13, PS5, Problem 3 (favorite)

ARM: It's a beautiful problem, but in the repo it's called
PS_Euler_theorem_not_rel_prime, and it does in fact depend on Euler's
function phi.  We are not covering Euler's thm this term.

Comment: The solution to this problem is general and concise, and I
generally like problems that build up to a conclusion from
intermediate guided steps.

     Digraphs and Scheduling

        S15, PS6, Problem 2

Comment: This problem is good because it asks for a concrete example,
then reinforces the concept of WOP which students did not seem to
understand well on the midterm.

  (DO NOT USE - in 8m)   S15, PS6, Problem 3 (favorite) 

Comment: This is the chicken pecking problem. I think this problem is
concrete and accessible because of the first two prompts for actual
graphs.  The solution to this problem is also concise and
understandable.

    Directed Walks and Paths

       F13, PS6 Problem 1 - shorter/easier -> prob this one

Comment: This problem is relatively short, asks for concrete examples,
then generalizes to a useful conclusion. The solution is also fairly
clear.

(DO NOT USE - in 8m)    F13, PS6 Problem 3 - longer/harder 

Comment: This problem has some of the same characteristics of the
previous, but is more difficult and lengthy. One should be picked over
the other based on how long and different the remainder of the problem
set is

    Partial Orders & equivalence

DO NOT USE - in 8w)      F13, PS7 Problem 1 - equivalence

Comment: Quicker problem, but might make good use of the peer grading
because there are multiple ways to prove. Very definition based, so a
good problem to make sure they’ve looked at the definition without
adding too much to the problem set.

DO NOT USE              F13, PS7 Problem 4 - partial order

ARM This is PS_strict_partial_order_isomorphic_to_subset depends on OPTIONAL
    section 10.7
Comment: A bit more complicated. More satisfying problem to solve,
though I think. Requires more than just definitions - be able to
understand and manipulate them.
\end{verbatim}
\end{staffnotes}

\pinput{PS_gcd_three_integers_hint}                    %S14.ps4.2

\pinput{CP_n5_last_digit}                              %S13.ps5.2

\pinput{MQ_cycle_from_closed_walk}                     %S15.ps6.2

\pinput{PS_odd_length_walk}                            %F13.ps6.1, S13.ps6.2
\end{document}
