\section{Propositions}

\begin{definition*}
A {\em proposition} is a statement that is either true or false.
\end{definition*}

This definition sounds very general, but it does exclude sentences
such as, ``Wherefore art thou Romeo?'' and ``Give me an A!''.

But not all propositions are mathematical.  For example, ``Albert's wife's
name is `Irene' '' happens to be true, and could be proved with legal
documents and testimony of their children, but it's not a mathematical
statement.

Mathematically meaningful propositions must be about well-defined
mathematical objects like numbers, sets, functions, relations, \etc, and
they must be stated using mathematically precise language.  We can
illustrate this with a few examples.

\begin{proposition}
2 + 3 = 5.
\end{proposition}

This proposition is true.

A {\em prime} is an integer greater than one that is not divisible by any
integer greater than 1 besides itself, for example, 2, 3, 5, 7, 11, \dots.
\begin{proposition}\label{41}
For every nonnegative integer, $n$, the value of $n^2 + n + 41$ is prime.
\end{proposition}

Let's try some numerical experimentation to check this proposition.
Let $p(n) \eqdef  n^2 + n + 41$.\hyperdef{proofs}{eqdef}{
\footnote{The symbol $\eqdef$ means
 ``equal by definition.''  It's always ok to simply write ``='' instead of
 $\eqdef$, but reminding the reader that an equality holds by definition
 can be helpful.}}
We begin with $p(0) = 41$ which is prime.  $p(1) = 43$ which is prime.  $p(2) = 47$
which is prime.  $p(3)=53$ which is prime. \dots $p(20) = 461$ which is
prime.  Hmmm, starts to look like a plausible claim.  In fact we can keep
checking through $n=39$ and confirm that $p(39)=1601$ is prime.

But $p(40) = 40^2 + 40 + 41 = 41 \cdot 41$, which is not prime.  So it's
not true that the expression is prime {\em for all} nonnegative integers.
The point is that in general you can't check a claim about an infinite set
by checking a finite set of its elements, no matter how large the finite
set.

By the way, propositions like this about \emph{all} numbers or other
things are so common that there is a special notation for it.  With this notation,
Proposition~\ref{41} would be
\begin{equation}\label{pn}
\forall n \in \naturals.\; p(n) \text{ is prime}.
\end{equation}
Here the symbol $\forall$ is read ``for all''.  The symbol $\naturals$
stands for the set of {\em nonnegative integers}, namely, 0, 1, 2, 3,
\dots (ask your TA for the complete list).  The symbol ``$\in$'' is read
as ``is a member of'' or simply as ``is in''.  The period after the
$\naturals$ is just a separator between phrases.

\begin{notesproblem}
Show that no nonconstant polynomial can map all nonnegative integers into
prime numbers.  (This can be proved using elementary algebra, but it's a
little tricky.  It will be easier to show after we study modular
arithmetic later in the term.)
\end{notesproblem}


Here are two even more extreme examples:
\begin{proposition}\label{a4}
$a^4 + b^4 + c^4 = d^4$ has no solution when $a, b, c, d$ are positive
integers.
\end{proposition}
Euler (pronounced ``oiler'') conjectured this in 1769.  But the proposition
was proven false 218 years later by Noam Elkies at a liberal arts school
up Mass Ave.  The solution he found was $a = 95800, b = 217519, c = 414560, d
= 422481$.

In logical notation, Proposition~\ref{a4} could be written,
\[
\forall a \in \posints\, \forall b \in \posints\, \forall c \in \posints\, \forall
d \in \posints.\; a^4 + b^4 + c^4 \neq d^4.
\]
Here, $\posints$ is a symbol for the positive integers.
Strings of $\forall$'s like this are usually abbreviated for easier reading:
\[
\forall a, b, c, d \in \posints.\; a^4 + b^4 + c^4 \neq d^4.
\]


\begin{proposition}
$313 (x^3 + y^3) = z^3$ has no solution when $x, y, z\in\posints$.
\end{proposition}

This proposition is also false, but the smallest counterexample has
more than 1000 digits!

\begin{proposition}
\hyperdef{map}{color}{Every map can be colored with 4 colors} so that
adjacent\footnote{Two regions are adjacent only when they share a boundary
segment of positive length.  They are not considered to be adjacent if
their boundaries meet only at a few points.} regions have different
colors.
\end{proposition}

This proposition is true and is known as the ``Four-Color Theorem''.
However, there have been many incorrect proofs, including one that stood
for 10 years in the late 19th century before the mistake was found.  An
extremely laborious proof was finally found in 1976 by mathematicians
Appel and Haken, who used a complex computer program to categorize the
four-colorable maps; the program left a couple of thousand maps
uncategorized, and these were checked by hand by Haken and his
assistants---including his 15-year-old daughter.  There was a lot of
debate about whether this was a legitimate proof: the proof was too big to
be checked without a computer, and no one could guarantee that the
computer calculated correctly, nor did anyone have the energy to recheck
the four-colorings of thousands of maps that were done by hand.  Finally,
about five years ago, a mostly intelligible proof of the Four-Color
Theorem was found, though a computer is still needed to check colorability
of several hundred special maps (see

\href{http://www.math.gatech.edu/~thomas/FC/fourcolor.html}
{\texttt{http://www.math.gatech.edu/\~{}thomas/FC/fourcolor.html}}).
\footnote{The story of the Four-Color Proof is told in a well-reviewed
  popular (non-technical) book: ``Four Colors Suffice.  How the Map
  Problem was Solved.'' \emph{Robin Wilson}.  Princeton Univ. Press, 2003,
  276pp. ISBN 0-691-11533-8.}

\begin{proposition}[Goldbach]
Every even integer greater than 2 is the sum of two primes.
\end{proposition}

No one knows whether this proposition is true or false.  This is the
``Goldbach Conjecture,'' which dates back to 1742.

\iffalse

For a Computer Scientist, some of the most important questions are about
program and system ``correctness'' -- whether a program or system does what
it's supposed to.  Programs are notoriously buggy, and there's a growing
community of researchers and practitioners trying to find ways to prove
program correctness.  These efforts have been successful enough in the case
of CPU chips that they are now routinely used by leading chip manufacturers
to prove chip correctness and avoid mistakes like the notorious Intel
division bug in the 1990's.

Developing mathematical methods to verify programs and systems remains an
active research area.  We'll consider some of these methods later in the
course.
\fi

\endinput
