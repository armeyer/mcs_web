\documentclass[handout]{mcs}

\begin{document}

\inclassproblems{2, Fri.}

%%%%%%%%%%%%%%%%%%%%%%%%%%%%%%%%%%%%%%%%%%%%%%%%%%%%%%%%%%%%%%%%%%%%%
% Problems start here
%%%%%%%%%%%%%%%%%%%%%%%%%%%%%%%%%%%%%%%%%%%%%%%%%%%%%%%%%%%%%%%%%%%%%


\begin{problem}
\emph{Set Formulas and Propositional Formulas.}
\bparts

\ppart\label{verprop}

Verify that the propositional formula $(P\ \QAND\ \,
\QNOT(Q))\ \QOR\ (P\ \QAND\ Q)$ is equivalent to $P$.

\solution{
There is a simple verification by truth table with 4 rows which we omit.

There is also a simple cases argument: if $Q$ is \true, then the formula
simplifies to $(P\ \QAND\ \false)\ \QOR\ (P\ \QAND\ \true)$ which further
simplifies to $\false\ \QOR\ P$ which is equivalent to $P$.

Otherwise, if $Q$ is \textcolor{red}{F}, then the formula simplifies to
$(P\ \QAND\ \true)\ \QOR\ (P\ \QAND\ \false)$ which
is likewise equivalent to $P$.}

\ppart Use part~\eqref{verprop} to prove that
\[
A = (A-B) \union (A \intersect B)
\]
for any sets, $A,B$, where
\[
A-B \eqdef \set{a \in A \suchthat a \notin B}.
\]

\solution{
We need only show that the two sets have the same elements, that is $x$ is
in one set iff $x$ is in the other set, for any $x$.

Let $P$ be $x \in A$ and $Q$ be $x \in B$.  Then
\begin{align*}
\lefteqn{x \in (A-B) \union (A \intersect B)}\\
 & \qiff x \in (A-B)\ \QOR\ x \in (A
\intersect B) & \text{(by def of $\union$)}\\
& \qiff (x \in A\ \QAND\  \QNOT(x \in B))\ \QOR\ (x \in A\ \QAND\  x \in B) 
  & \text{(by def of $\intersect$ and $-$)}\\
& \qiff (P\ \QAND\  \QNOT(Q))\ \QOR\ (P\ \QAND\ Q) & \text{(by def of
  $P$ and $Q$)}\\
&\qiff P & \text{(by part~\eqref{verprop})}\\
& \qiff x \in A & \text{(by def of $P$)}.
\end{align*}
}

\eparts
\end{problem}


\iffalse
%from fall95 pset2

Explain where the following proof goes wrong.

{\em Theorem:} Every positive integer can be described in fewer than
fourteen words.

\begin{proof}
  Let $S$ be the set of all numbers that cannot be so described. It
  contains a least element.  This element is ``the smallest number that
  cannot be described in less than fourteen English words.''  But this is
  a thirteen-word description, a contradiction.  Thus $S$ must be empty,
  implying the theorem.
\end{proof}


%%%%%%% Problem %%%%%%%%
% source: spring99 ps8
% topic: infinity, incomputable functions

\problem In this problem we will prove the remarkable fact that
there exist mathematical functions that computers, no matter how powerful,
simply cannot compute.  We will do this through countability arguments.
\bparts
\ppart 

Show that the set of finite length binary strings is countable.

\solution{
Let $B_n$ denote the set of binary strings of length $n$.
This set has exactly $2^n$ elements.  Therefore the set of finite
length binary strings $B=\bigcup_{n\in\naturals}B_n$ is a union of
countable number of finite sets.  We know from Lecture 17 that such
union is countable.

Namely, one can easily list all elements of $B$ by first listing the
elements of $B_1$, then the elements of $B_2$, then the elements of
$B_3$, etc,.  We can do this since each set $B_i$ has a finite number
of elements.


{\bf Alternative Solution:} 
Some students noticed that finite length binary strings can be thought
of as natural numbers represented base $2$.  Therefore, an injective mapping
from finite length binary strings to natural numbers can be constructed as
follows:
\[
f(b_0b_1b_2...b_n)= 2^{n+1}+(b_0+b_12+b_22^2+...+b_n2^n)
\]
i.e. we map a string of bits to a natural number that it represents
in binary, except that we add $2^{n+1}$ term to make sure that bit
strings of different length get mapped to a different natural number
(Otherwise strings $0010$, $010$, $10$ would all map to the same number).
You can check that $f$ is an injection from $B$ to $\naturals$ and hence
$B$ is countable.  

You can also easily modify $f$ to be a bijection if instead of adding
$2^{n+1}$ term to the right side, you add $2^{n+1}-1$.  }


\ppart 

From your answer to the previous part, 
what can you conclude about the countability of the set of all
computer programs?  

\solution{
Since every computer program can be represented as a finite string of
bits (e.g. machine language code), it follows that the set of all
computer programs is countable.

Namely, the mapping which represents computer programs as finite
strings of bits is an injection.  Hence there is a surjection from the
set of finite strings of bits to the set of computer programs.  From
(a) we know that there is a surjection from the set of natural numbers
to the set of finite strings of bits.  By transitivity of surjective
mapping, there is a surjection between natural numbers and computer
programs.  Hence the set of all computer programs is countable.  }


\ppart 
Show that the set of {\em infinite} length binary strings is uncountable.

\solution{
We construct a bijection from the set $B'$ of infinite length binary
strings to the power set of the natural numbers $P(\naturals)$.  Since
$P(\naturals)$ is uncountable, it will then follow that $B'$ is
uncountable.

We need to associate a subset of the naturals with each infinite
length binary string.  Let $b=b_0 b_1 b_2 b_3\ldots \in{B'}$ be an
infinite length binary string (here each $b_i \in \{0,1\}$).  We
associate with this string the set $f(b) = S = \{ i \ | \ b_i = 1 \}$.
That is, we include $i$ in $S=f(b)$ if and only if the $i$-th bit
position (from the left) of $b$ is 1.  Such mapping
$f:B'\rightarrow{P(\naturals)}$ is a bijection: It is an injection
because if two infinite length binary strings are different, then
clearly they map to different subsets.  Furthermore, it is a
surjection because for every subset $S\in P(\naturals)$, there is a
binary string $b$ s.t. $f(b)=S$, namely $b=b_0b_1b_2b_3\ldots$ where
$b_i$ is defined as $0$ if $i\not\in{S}$ and $1$ if $i\in{S}$.

{\bf Alternative Solution:} 
One can also show that the set $B'$ is uncountable directly by a
diagonalization argument.

Assume $B'$ is countable and let $b^{(1)},b^{(2)},b^{(3)},b^{(4)},...$
be a list of all elements of $B'$.  By diagonalization, we will
construct an element $b=b_0b_1b_2b_3...\in{B'}$ s.t. $b\neq b^{(i)}$
for all $i\in\naturals$ as follows: $j$-th bit of $b$ is defined as
the opposite of the $j$-th bit of string $b^{(j)}$.  Obviously, so
defined $b$ is not equal to any $b^{(i)}$ from the list because for
every $i$, the $i$-th bit of $b^{(i)}$ is different than the $i$-th
bit of $b$.  Therefore, the list $b^{(1)},b^{(2)},b^{(3)},b^{(4)},...$
is {\it not} a list of all elements in $B'$, and thus the assumption
that $B'$ is countable leads to a contradiction.  

}

\ppart 
A function is a {\em decision function} if it maps finite length bit
strings into the range $\{0,1\}$.  
% In other words, the function assigns a single bit 
% to every finite length bit string.  
Let $F$ be the set consisting of {\em all possible} decision
functions.  Show that set $F$ is uncountable.

\solution{
From part (a), we know that there is a bijection from the set of
all finite length bit strings to the set of natural numbers (because
the set of all finite length bit strings was infinite and countable),
hence we can think of decision functions as mapping natural numbers to
the range $\{0,1\}$, i.e. to single bits.

It is easy to see a bijection between a set of such mappings and a
power set of the natural numbers.  Namely, we associate with a
decision function $f$ a set $S=\{i\in\naturals\mbox{ s.t. }f(i)=1\}$.
Clearly, this is an injection and a surjection.  Therefore, since the
power set of the set of natural numbers is uncountable then so is the
set of decision functions.

{\bf Alternative Solution:} 
We can give a bijection between the set of all decision functions and
the set of all infinite length bit strings, which by part (c), implies
that the set of decision functions is uncountable.  Let $f$ be a
decision function.  We map $f$ to an infinite length string as
follows.  We set bit $i$ of the string to be $f(i)$.  That is, our
infinite length bit string will be $f(1) f(2) f(3) \ldots $.  Now,
observe that this mapping is a bijection.  In particular, suppose we
have two different decision functions $f$ and $g$.  Since they are
different, they must give a different output on some particular input.
Suppose that this particular input is the number $j$.  That is,
$f(j)\neq g(j)$.  Then, the corresponding infinite length binary
strings will differ in the $j-th$ position.  This shows that the
function is injective.  Now, if we have an infinite length bit string
$b_0 b_1 b_2
\ldots$, then a decision function $f$ defined as
$f(0) = b_0, f(1) = b_1, f(2) = b_2, \ldots$, will be mapped to that
string.  Therefore, the mapping function is surjective.  Since it's
both injective and surjective, it is also bijective.

{\bf Alternative Solution 2:} 
Some students gave a diagonalization argument for why $F$ is
uncountable: Assume otherwise and let $\{f_1,f_2,f_3,...\}$ be the
list of all elements of $F$.  From part (a) we know that $B$, the set
of all finite length binary strings is countable, so there must exist
a bijection $g:B\rightarrow\naturals$.  Now, define a decision
function $f:B\rightarrow\{0,1\}$ as follows:
\[
f(s)=\mbox{ the opposite of }f_{g(s)}(s)
\]
Then $f$ is different from all $f_i$'s because for all
$i\in\naturals$, $h$ differs from $f_i$ on string $s=g^{-1}(i)$.


{\bf Alternative Solution 3:} 
One can also argue, as some students did, that $F$ is uncountable by
showing the bijection between $F$ and the set, $P(B)$, of all subsets of
$B$.  The bijection associates with $f\in{F}$ a set of strings $b \in B$
s.t. $f(b)=1$.  Then, since by part (a) there is a bijection between $B$
and $\naturals$, there must be a bijection between $P(B)$ and
$P(\naturals)$.  By transitivity of bijective relation between sets, there
is a bijection then between $F$ and $P(\naturals)$.  Hence $F$ is
uncountable.

}


\ppart 

A function is computable if there is a computer program that computes
it.  From your answers to the previous parts, prove that there is a
decision function that is not computable.
% STAS: repetitions...
% That is, there is no computer program that computes it.  

(Hint: compare your answer from the second part to the the answer you
just got from the previous part)

\solution{
Since there are only countably many computer programs,
but uncountably many decision functions, there can be no surjection
from the set of computer programs to the set of decision functions (In
fact, this means that there is {\it more} decision functions than
computer programs).  In particular, if we associate a computer program
to a decision function which this program is computing\footnote{If a
program is not computing any decision function, associate it with a
trivial decision function $f(b)=0$, for all finite bit strings $b$.},
this association is not a surjection either.  Therefore there must be
some decision function which is not computed by any computer program.}

\eparts                          

\end{problems}\fi


\begin{problem}
Subset take-away\footnote{From Christenson \& Tilford, \emph{David Gale's
Subset Takeaway Game, American Mathematical Monthly, Oct. 1997}} is a two
player game involving a fixed finite set, $A$.  Players alternately choose
nonempty subsets of $A$ with the conditions that a player may not choose
\begin{itemize}
\item the whole set $A$, or
\item any set containing a set that was named earlier.
\end{itemize}
The first player who is unable to move loses the game.

For example, if $A$ is $\set{1}$, then there are no legal moves and the
cond player wins.  If $A$ is $\set{1,2}$, then the only legal moves are
$\set{1}$ and $\set{2}$.  Each is a good reply to the other, and so once
again the second player wins.

The first interesting case is when $A$ has three elements.  This time, if
the first player picks a subset with one element, the second player picks
the subset with the other two elements.  If the first player picks a
subset with two elements, the second player picks the subset whose sole
member is the third element.  \iffalse In short, in response to any first
move, the second player may choose the complementary set.\fi Both cases
produce positions equivalent to the starting position when $A$ has two
elements, and thus leads to a win for the second player.

Verify that when $A$ has four elements, the second player still has a
winning strategy.\footnote{David Gale worked out some of the properties of
this game and conjectured that the second player wins the game for
any set $A$.  This remains an open problem.}

\solution{
Solution postponed.
}

\solution{There are way too many cases to work out by hand if we tried to
list all possible games.  But the elements of $A$ all behave the same, so
we can cut to a small number of cases using the fact that permuting around
the elements of $A$ in any game yields another possible game.  We can do
this by not mentioning specific elements of $A$, but instead using the
\emph{variables} $a,b,c,d$ whose values will be the four elements of $A$.

We consider two cases for the move of the Player 1 when the game starts:

\begin{enumerate}
\item Player 1 chooses a one element or a three element subset.  Then
Player 2 should choose the complement of Player one's choice.  The game
then becomes the same as playing the $n=3$ game on the three element set
chosen in this first round, where we know Player 2 has a winnging
strategy.

\item Player 1 chooses a subset of 2 elements.  Let $a,b$ be these
elements, that is, the first move is $\set{a,b}$.  Player 2 should choose
the complement, $\set{c,d}$, of Player 1's choice.  We then have the
following subcases:
\begin{enumerate}

\item Player 1's second move is a one element subset, $\set{a}$.  Player 2
should choose $\set{b}$.  The game is then reduced to the two element game
on $\set{c,d}$ where Player 2 has a winnging strategy.

\item
Player 1's second move is a two element subset, $\set{a,c}$.  Player 2
should choose its complement, $\set{b,d}$.  This leads to two subsubcases:
\begin{enumerate}

\item Player 1's third move is one the remaining sets of size two,
$\set{a,d}$.  Player 2 should choose its complement, $\set{b,c}$.  The
remaining possible moves are the four sets of size 1, where the Player 2
clearly wins after two more rounds.

\item Player 1's third move is a one element set, $\set{a}$.  Player 2
should choose $\set{b}$.  The game is then reduced to the case two element
game on $\set{c,d}$ where Player 2 has a winnging strategy.
\end{enumerate}
\end{enumerate}
\end{enumerate}
So in all cases, Player 2 has a winning strategy in the Gale game for
$n=4$.}



\end{problem}

%%%%%%%%%%%%%%%%%%%%%%%%%%%%%%%%%%%%%%%%%%%%%%%%%%%%%%%%%%%%%%%%%%%%%
% Problems end here
%%%%%%%%%%%%%%%%%%%%%%%%%%%%%%%%%%%%%%%%%%%%%%%%%%%%%%%%%%%%%%%%%%%%%
% End document if it's a stand alone and decrease documentdepth otherwise
\ifnum\value{documentdepth}>0
  \setcounter{documentdepth}{\value{documentdepth}-1}
  \endinput
\else
  \end{document}
\fi

