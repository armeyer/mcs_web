%ps8

\documentclass[handout]{mcs}

\begin{document}

\renewcommand{\reading}{Notes
  Ch.~\bref{Turing_sec}--\bref{arithmetic_modn_sec};
  Ch.~\bref{annuity_sec}--\bref{Stirling_sec}}

\iffalse
TOPICS:

Modular arithmetic, Fermat, Euler, RSA

Harmonic sums, integral method, Stirling's approx}
\fi

\problemset{8}

%%%%%%%%%%%%%%%%%%%%%%%%%%%%%%%%%%%%%%%%%%%%%%%%%%%%%%%%%%%%%%%%%%%%%
% Problems start here
%%%%%%%%%%%%%%%%%%%%%%%%%%%%%%%%%%%%%%%%%%%%%%%%%%%%%%%%%%%%%%%%%%%%%

%\pinput{PS_number_theory_finger_exercises}

\pinput{PS_congruent_modulo_1000}

\pinput{PS_Euler_theorem_calculation}

%\pinput{PS_self-inverse_mod_p}

%\pinput{CP_any_prime} was PS_any_prime.  Proved in the Notes.

\pinput{PS_Euler_function_multiplicativity}

%\pinput{CP_Sk_equiv_-1_mod_p}

%\pinput{PS_RSA_correctness}

%\pinput{PS_RSA_key_implies_factoring}

%\pinput{PS_MIT_Harvard_degree_value}

\pinput{PS_bug_on_rug_harmonic_number}

\pinput{PS_Stirlings_and_log_n_factorial}

%\pinput{PS_credit_union}


%%%%%%%%%%%%%%%%%%%%%%%%%%%%%%%%%%%%%%%%%%%%%%%%%%%%%%%%%%%%%%%%%%%%%
% Problems end here
%%%%%%%%%%%%%%%%%%%%%%%%%%%%%%%%%%%%%%%%%%%%%%%%%%%%%%%%%%%%%%%%%%%%%


\end{document}
