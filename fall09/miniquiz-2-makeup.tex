;; This buffer is for notes you don't want to save, and for Lisp evaluation.
;; If you want to create a file, visit that file with C-x C-f,
;; then enter the text in that file's own buffer.

\documentclass[quiz]{mcs}

\begin{document}

\miniquiz{Miniquiz 2 Makeup}

%\section*{MORNING / AFTERNOON}                                                                    

%%%%%%%%%%%%%%%%%%%%%%%%%%%%%%%%%%%%%%%%%%%%%%%%%%%%%%%%%%%%%%%%%%%%%                  
% Problems start here                                                                  
%%%%%%%%%%%%%%%%%%%%%%%%%%%%%%%%%%%%%%%%%%%%%%%%%%%%%%%%%%%%%%%%%%%%%                  


%\pinput[points = 5]{MQ_irrational_raised_to_sqrt3}
%\instatements{\newpage}
%\pinput[points = 5]{MQ_10_and_15_cent_stamps_by_WOP}
%\instatements{\newpage}                                                               
%\pinput[points = 10]{MQ_implies_relation_on_propositonal_formulas}

% new problem - partial orders
% -------------------------------------------------------------------
\begin{problem}[7]

  \newcommand{\FLBS}{$\mathtt{0,1}^*$}
  \newcommand{\stringlnth}[1]{\mopt{length}(#1)}
 
  Define a strict partial order, $\prec$, on the set of positive length 
  binary strings based on their length.  Namely, for binary strings $s$ 
  and $t$,
  \[
  [s \prec t]\quad \eqdef\quad [\stringlnth{s} < \stringlnth{t}].
  \]
  You may assume without proof that $\prec$ is a strict partial order on
  this set of strings.

  \bparts

  \ppart[1] Describe the minimal element(s) of this partial order.  Is there
  a minimum element?

  \solution[\inhandout{\vspace{0.5in}}]{The length one strings, \texttt{0} 
  and \texttt{1}, are the minimal elements.  Since there is more than one 
  minimal element, there can't be any minimum element.}

  \ppart[1] Give an example of 6 strings that form an antichain in this
  partial order.

  \solution[\inhandout{\vspace{0.5in}}]{Any 6 strings of the same length, 
  for example,
  \[
  \mtt{000, 001, 010, 011, 100, 101}.
  \]}

  \ppart[3] Prove that $\prec$ is a well-founded partial order.

  \hint Apply the Well Ordering Principle to lengths.

  \solution[\inhandout{\vspace{3in}}]{
  Let $A$ be a nonempty set of positive length binary strings.  We need 
  only show that $A$ has at least one minimal element.  But any minimum 
  length string in $A$ will be $\prec$-minimal.

  The above concise argument is a full credit answer to this part, but 
  since we're still practicing with Well Ordering proofs, let's spell 
  this argument out more fully.

  Let $L \subseteq \naturals^+$ be the set of lengths of strings in $A$.
  Now $A$ is not empty, so $L$ is nonempty, and therefore by the Well
  Ordering Principle, there is a minimum element $m \in L$.  By definition
  of $L$, there is a string $s \in A$ with $\stringlnth{s} = m$.  Then $s$
  is a $\prec$-minimal element of $A$.  This follows by contradiction: if
  there was another string $t \in A$ and $t \prec s$, then $\stringlnth{t}$
  would be an element of $L$ that was less than $\stringlnth{s}$.}

  \ppart[2] Let $\preceq$ be a similarly defined reflexive relation on
  binary strings, namely,
  \[
  [s \preceq t]\quad \eqdef\quad [\stringlnth{s} \leq \stringlnth{t}].
  \]
  Explain why $\preceq$ is not a weak partial order on binary strings.
 
  \solution{It is not antisymmetric (and therefore also not asymmetric)
  since two different binary strings can both be the same length.}

  \eparts
\end{problem}
\inhandout{\instatements{\newpage}}


%%%%%%%%%%%%%%%%%%%%%%


\begin{problem}[2]
For each of the binary relations below, indicate whether it is
\textbf{I}njective, \textbf{S}urjective, \textbf{B}ijective, a Total relation
\textbf{TRel}, or \textbf{N}one of the above.  (More than one property may
hold for some relations.)

\begin{enumerate}

\item The relation, $R$, on $\mathbb{N}$,
the set non-negative integers, such that

$m\, R\, n\ \qiff n=m^3$. \hfill \brule{1.5in}

\solution{ \textbf{I,TRel} }

\item The relation, $Q$, on the set of all {\it living people}
such that

$s\, Q\, t  \qiff s \text{ is the child of } t$.
\hfill\brule{1.5in}

\solution{
\textbf{N}  (not \textbf{TRel} since not all parents are living),
}

\end{enumerate}

\end{problem}


\iffalse
%[Hide Quoted Text] 3.
%[Induction]
%Class problem #1 for Wed.
\fi

\begin{problem}[2]  Let $H$ be the set of integers greater than 5 that are
multiples of 3.  Define a bijection, $f:\naturals \to H$.

$f(n) \eqdef$ \brule{2in}

\solution{$f(n) \eqdef 3n+6$}

\end{problem}


\begin{problem}[3]
For each of the binary relations below, indicate whether it is a
\textbf{F}unction, \textbf{A}symmetric, \textbf{TRAN}sitive, a total order
\textbf{TOrd}, or \textbf{N}one of the above.

\iffalse
 \textbf{R}eflexive,
\textbf{A}ntisymmetric, \textbf{TRAN}sitive, \textbf{TOT}al, or
\textbf{N}one of the above.
\fi

\iffalse

\begin{tabular}{l}
A \textbf{F}unction,\\
\textbf{A}symmetric,\\
\textbf{TRAN}sitive,\\
%a total relation \textbf{TRel},\\
%a strict \textbf{P}artial Order,\\
a total order \textbf{TOrd} or\\
\textbf{N}one of the above.
\end{tabular}
\fi


(More than one property may hold for some relations.)
\begin{enumerate}

\item
The relation on the set $\set{1,2,3,4}$ whose graph is

$\set{(1,1),(1,3),(3,1)}$ \hfill \brule{1.5in}

\item The relation on the set $\set{1,2,3,4}$ whose graph is

$\set{(1,2), (1,3), (1,4), (3,2), (2,4), (3,4)}$ \hfill \brule{1.5in}

\item The relation ``has the same name'' on people.   \hfill \brule{1.5in}

(People are considered to have the same name as themselves.)
\end{enumerate}


\solution{ \textbf{N} for the first:

% Not reflexive since (2,2), (3,3), and (4,4) are not in the relation.

Not function since both (1,1) and (1,2) are in the relation. Not
asymmetric since (1,3) and (3,1) are both in the relation. Not
transitive since (3,1) and (1,3) are in the relation, but (3,3)
is not. Not total since 2 and 4 do not relate to anything.

\textbf{A, TRAN, TOrd} for the second:

%Not reflexive since $(1,1),(2,2), (3,3)$, and $(4,4)$ are not in the relation.

Asymmetric since for each pair, the reverse pair is not in the
relation; for example, $(1,2)$ is in the relation, but $(2,1)$ is
not. Transitive since if $(a,b)$ and $(b,c)$ are in this
relation, then $(a,c)$ is also in this relation.  Hence it is a
strict partial order, and in fact a total order, since every two
elements are comparable.

It is not a function since $(1,2)$ and $(1,3)$ are in the
relation.

\textbf{TRAN} for the third:

Not asymmetric since if $A$ has the same name as $B$, then $B$
has the same name as $A$ (so in fact it is \emph{symmetric}).
Transitive since if $A$ and $B$ have the same name, and so do $B$
and $C$, then clearly so do $A$ and $C$. }
\end{problem}


%[Hide Quoted Text]

%[Relations: minimal, total, chain, antichain]
%Based off of TP.3.4 (smaller version of the set) and class problems 3T,

\begin{problem}[2]
Let the set $\set{2,6,9,12,18,27,36,96}$ be \iffalse
(weakly)\fi partially ordered by
the divides relation, $m$ is considered ``less that or equal to'' $n$ iff
$m$ divides $n$.  For this partial order

\begin{itemize}

\item what are the minimal elements? \hfill\brule{1.5in}

\solution{2,9}

\item what are the maximal elements? \hfill\brule{1.5in}

\solution{27, 36, 96}

\item give an example of a maximum-size chain. \hfill\brule{1.5in}

\solution{There are three possible solutions:
$\set{2,6,12,36},\set{2,6,12,96},\set{2,6,18,36}$.}

\item give an example of a maximum-size antichain. \hfill\brule{1.5in}

\solution{There are three possible solutions: $\set{12,18,27},
\set{18,27,96}, \set{27,36,96}$.}

\end{itemize}

\end{problem}

%%%%%%%%%%%%%%%%%%%%%%%%%%%%%%%%%%%%%

\begin{problem}[4]

\bparts

\ppart[1] Express the sum of the first $n$ odd numbers, $1+3+5+\dots$,
without the dots by filling in the two missing parts in the following
$\sum$-expression:
\[
\underbrace{1+3+5+\dots}_{n \text{ terms}} = \sum_{i=0}^{(\qquad)} (\quad\qquad)
\]
\solution{\[
\sum_{i=0}^{(n-1)} (2i+1) 
\]}

\ppart[3] Prove by induction that this sum is $n^2$.

\solution{We will prove this by induction on $n$ with 
{\bf Induction hypothesis:}
\[
P(n) ::= \sum_{i=0}^{n-1} (2i+1) = n^2.
\]

{\bf Base case $n=0$:} The left hand sum is 0 in this case, since it has
no terms in it.  Likewise, the right hand side is a sum over an empty set
($\set{i \suchthat i \ge 0 \text{ and } i \le -1}$), and so is also 0 by convention.

{\bf Induction step:}
Assume that $P(n)$ is true for some $n \geq 0$, which means
\[
1+3+ \cdots +(2(n-1)+1) = n^2.
\]
Now,
\begin{align*}
\sum_{i=0}^{(n+1)-1} (2i+1) & = \paren{\sum_{i=0}^{n-1} (2i+1)} +
           2((n+1)-1)+1 & \text{(def. of $\Sigma$)}\\
        & = n^2 + 2((n+1)-1)+1 & \text{(ind. hyp)}\\
        & = n^2 + 2n + 1 = (n+1)^2,
\end{align*}
which proves $P(n+1)$.

Therefore, by the principle of induction, $P(n)$ holds for all $n \geq 0$.
}
\eparts
\end{problem}



%%%%%%%%%%%%%%%%%%%%%%%%%%%%%%%%%%%%%%%%%%%%%%%%%%%%%%%%%%%%%%%%%%%%%                  
% Problems end here                                                                    
%%%%%%%%%%%%%%%%%%%%%%%%%%%%%%%%%%%%%%%%%%%%%%%%%%%%%%%%%%%%%%%%%%%%%                  
\end{document}

