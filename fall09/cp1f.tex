\documentclass[handout]{mcs}

\begin{document}

\inclassproblems{1, Fri.}

%%%%%%%%%%%%%%%%%%%%%%%%%%%%%%%%%%%%%%%%%%%%%%%%%%%%%%%%%%%%%%%%%%%%%
% Problems start here
%%%%%%%%%%%%%%%%%%%%%%%%%%%%%%%%%%%%%%%%%%%%%%%%%%%%%%%%%%%%%%%%%%%%%

\begin{problem}\label{generprob}
  Generalize the proof from lecture (reproduced below) that $\sqrt{2}$ is
  irrational, for example, how about $\sqrt[3]{2}$?  Remember that an
  irrational number is a number that cannot be expressed as a ratio of two
  integers.

\begin{quote}

\begin{theorem*}
$\sqrt{2}$ is an irrational number.
\end{theorem*}

\begin{proof}
  The proof is by contradiction: assume that $\sqrt{2}$ is rational, that
  is,
  \begin{equation}\label{2nd}
    \sqrt{2} = \frac{n}{d},
  \end{equation}
  where $n$ and $d$ are integers.  Now consider the smallest such positive
  integer denominator, $d$.  We will prove in a moment that the numerator,
  $n$, and the denominator, $d$, are both even.  This implies that
         \[
        \frac{n/2}{d/2}
        \]
  is a fraction equal to $\sqrt{2}$ with a smaller positive integer
  denominator, a contradiction.

\begin{quote}
\emph{Since the assumption that $\sqrt{2}$ is rational leads to this
  contradiction, the assumption must be false.  That is, $\sqrt{2}$ is
  indeed irrational.}  This italicized comment on the implication of the
contradiction normally goes without saying, but since this is the first
6.042 exercise about proof by contradiction, we've said it.
\end{quote}

  To prove that $n$ and $d$ have 2 as a common factor, we start by
  squaring both sides of~\eqref{2nd} and get $2 = n^2 / d^2$, so
\begin{equation}\label{2d2}
2 d^2 = n^2.
\end{equation}
So 2 is a factor of $n^2$, which is only possible if 2 is in fact a
factor of $n$.

This means that $n=2k$ for some integer, $k$, so
\begin{equation}\label{n24}
  n^2 = (2k)^2 = 4 k^2.
\end{equation}
Combining~\eqref{2d2} and~\eqref{n24} gives $2 d^2 = 4 k^2$, so
\begin{equation}\label{n22}
d^2 = 2k^2.
\end{equation}
So 2 is a factor of $d^2$, which again is only possible if 2 is in fact
also a factor of $d$, as claimed.
\end{proof}
\end{quote}

\begin{solution}

\begin{proof}
We prove that for any $n>1$, $\sqrt[n]{2}$ is irrational by
contradiction.

Assume that $\sqrt[n]{2}$ is rational.  Under this assumption, there exist
integers $a$ and $b$ with $\sqrt[n]{2} = a/b$, where $b$ is the smallest
such positive integer denominator.  Now we prove that $a$ and $b$ are both
even, so that
         \[
        \frac{a/2}{b/2}
        \]
  is a fraction equal to $\sqrt[n]{2}$ with a smaller positive integer
  denominator, a contradiction.

\begin{eqnarray*}
\sqrt[n]{2} & = & \frac{a}{b}\\
2 & = & \frac{a^{n}}{b^{n}}\\
2b^{n} & = & a^{n}.
\end{eqnarray*}
The lefthand side of the last equation is even, so $a^{n}$ is even.  This
implies that $a$ is even as well (see below for justification).

In particular, $a = 2c$ for some integer $c$.  Thus, 
\begin{eqnarray*}
2b^{n} & = & (2c)^n = 2^{n}c^{n},\\
b^{n} & = & 2^{n-1}c^{n}.
\end{eqnarray*}
Since $n-1>0$, the righthand side of the last equation is an even number,
so $b^{n}$ is even.  But this implies that $b$ must be even as well,
contradicting the fact that $a/b$ is in lowest terms.
\end{proof}

Now we justify the claim that if $a^n$ is even, so is $a$.

There is a simple proof by contradiction: suppose to the contrary that $a$
is odd.  It's a familiar (and easily verified\footnote{Two odd integers
  can be written as $2x+1$ and $2y+1$ for some integers $x$ and $y$.  Then
  their product is also odd because it equals $2z +1$ where $z=
  2(2xy+x+y)+1$.}) fact that the product of two odd numbers is odd, from
which it follows that the product of \emph{any} finite number of odd
numbers is odd, so $a^n$ would also be odd, contradicting the fact that
$a^n$ is even.

More generally for \emph{any} integers $m,k >0$, if $m^k$ is divisible
by a prime number, $p$, then $m$ must be divisible by $p$.  This follows
from the factorization of an integer into primes (which we'll discuss
further in a coming lecture): the primes in the factorization of $m^k$ are
precisely the primes in the factorization of $m$ repeated $k$ times, so if
there is a $p$ in the factorization of $m^k$ it must be one of $k$ copies
of a $p$ in the factorization of $m$.

\end{solution}
\end{problem}


\begin{problem}
  Here is a generalization of Problem~\ref{generprob} that you may not have
  thought of:

\begin{lemma}\label{poly}
Let the coefficients of the polynomial
$a_0+a_1x+a_2x^2+\cdots + a_{n-1}x^{n-1} + x^n$ be
integers.  Then any real root of the polynomial is either integral or
irrational.
\end{lemma}

\bparts

\ppart Explain why Lemma~\eqref{poly} immediately implies that
$\sqrt[n]{k}$ is irrational whenever $k$ is not an $n$th power of some integer.

\begin{solution}
Saying that an integer, $k$, is not the $n$th power of an
  integer, is equivalent to saying that the equation $x^n = k$ has no
  integer solutions.  Another way to say this is that the polynomial $x^n
  - k$ has no integer root.  Lemma~\eqref{poly} therefore implies that any
  root of $x^n-k$ is irrational.  But $\sqrt[n]{k}$ is, by definition, a
  root of this polynomial, so it is irrational.
\end{solution}

\ppart Collaborate with your tablemates to write a clear, textbook quality
proof of Lemma~\ref{poly} on your whiteboard.  (Besides clarity and
correctness, textbook quality requires good English with proper
punctuation.  When a real textbook writer does this, it usually takes
multiple revisions; if you're satisfied with your first draft, you're
probably misjudging.)  You may find it helpful to appeal to the following:
\begin{lemma}\label{ppow}
If a prime, $p$, divides some power of an integer, then it divides that
integer.
\end{lemma}
You may assume Lemma~\ref{ppow} without writing down its proof, but see if
you can explain why it is true.

\begin{solution}
%From S08 class problem 2M(?) and S99 Tutorial 1 Notes:

\begin{proof}

Let $r$ be a real root of the polynomial, so that
\[
a_0+a_1r+a_2r^2+\cdots+  a_{n-1}r^{n-1}+ r^n = 0.
\]
There are three cases: either $r$ is an integer, or $r$ is irrational, or
$r = s/t$ for integers $s$ and $t$ which have no common divisors and such
that $t >1$.  We want to eliminate the last case, so assume for the sake
of contradiction that it held for some $r$.

Substituting $s/t$ for $r$ and multiplying both sides of the above
equation by $t^n$ yields:
\begin{eqnarray}
a_0t^n+a_1st^{n-1}+a_2s^2t^{n-2}+ \cdots + a_{n-1}s^{n-1}t+s^n & = & 0,\\
a_0t^n+a_1st^{n-1}+a_2s^2t^{n-2}+ \cdots + s^{n-1}t & = & -s^n.\label{e2}
\end{eqnarray}

Now since $t>1$, it must have a prime divisor, $p$.  The prime, $p$,
therefore divides each term of the lefthand side of equation~\eqref{e2},
so $p$ also divides the righthand side, $-s^n$.  This means that $p$
divides $s^n$, so by Lemma~\ref{ppow}, $p$ is also a divisor of
$s$.  So $p$ is a common factor of $s$ and $t$, contradicting the
fact that $s$ and $t$ have no common factors.
\end{proof}

Lemma~\ref{ppow} is a simple consequence of the \emph{Fundamental Theorem
  of Arithmetic} which says that every integer $> 1$ factors into a
product of primes that is \emph{unique} except for the order in which the
primes are multiplied.

For example, here are some ways to express 140 as a product of primes:
\[
140 = 2\cdot 2 \cdot 5 \cdot 7 = 2\cdot 5 \cdot 7 \cdot 2 = 7 \cdot 5 \cdot
2\cdot 2 = \cdots.
\]
By the Fundamental Theorem, every such product will have exactly two
occurrences of 2 and one each of 5 and 7.  Next, we can obviously get a
product of primes equal to, say, the third power of 140 by taking a
product that equals $140$ and repeating it three times.  For example,
\[
(140)^3 = 2\cdot 2 \cdot 5 \cdot 7 \ \ \cdot \ \  2\cdot 2 \cdot 7 \cdot 5
\ \ \cdot \ \ 2\cdot 2 \cdot 7 \cdot 5.
\]
The Fundamental Theorem now says that \emph{every} prime product equal to
the third power of 60 must have the same primes as this repeated product,
namely, six occurrences of 2 and three occurrences each of 5 and 7.  In
particular, the \emph{only} primes that are factors of $(140)^3$ are the
primes 2, 5 and 7 that are factors of 140.  This reasoning applies equally
well with any other integer greater than 1 in place of 140 and any power
greater than 0 in place of 3, proving that if $p$ is a prime factor of
$s^n$, then $p$ must have been a factor of $s$.

The Fundamental Theorem of Arithmetic is also know as the \emph{Unique
  Prime Factorization Theorem}.  It one of those familiar Mathematical
facts that is not exactly obvious.  We'll work out a proof of the
Fundamental Theorem later in the term.
\end{solution}

\eparts

\end{problem}


\begin{problem}
If we raise an irrational number to an irrational power, can the result be
rational?  Show that it can by considering $\sqrt{2}^{\sqrt{2}}$ and
arguing by cases.

\begin{solution}
We want to find irrational numbers $a,b$ such that $a^b$ is rational.
We argue by cases.

Case 1: [$\sqrt{2}^{\sqrt{2}}$ is rational].  Let $a = b = \sqrt{2}$.  $a$
and $b$ are irrational since $\sqrt{2}$ is irrational as we know.  Also,
$a^b$ is rational by case hypothesis.  So we have found the required $a$
and $b$ in this case.

Case 2: [$\sqrt{2}^{\sqrt{2}}$ is irrational].  Let $a =\sqrt{2}^{\sqrt{2}}$
and $b = \sqrt{2}$.  Then $a$ is irrational by case hypothesis, we know
$b$ is irrational, and
\[
a^b = \paren{\sqrt{2}^{\sqrt{2}}}^{\sqrt{2}} =
      \sqrt{2}^{\sqrt{2} \cdot \sqrt{2}} = \sqrt{2}^2 = 2,
\]
which is rational.  So we have found the required $a$
and $b$ in this case also.

So in any case, there will be irrational $a,b$ such that $a^b$ is
rational.  Note that we have no clue about which case is true, but that
didn't matter.
\end{solution}
\end{problem}

\begin{problem}
Here is a different proof that $\sqrt{2}$ is irrational, taken from the
American Mathematical Monthly, v.116, \#1, Jan. 2009, p.69:

\begin{proof}
  Suppose for the sake of contradiction that $\sqrt{2}$ is rational, and
  choose the least integer, $q>0$, such that $\paren{\sqrt{2}-1}q$ is a
  nonnegative integer.  Let $q' \eqdef \paren{\sqrt{2}-1}q$.  Clearly $0<
  q' < q$.  But an easy computation shows that $\paren{\sqrt{2}-1}q'$ is a
  nonnegative integer, contradicting the minimality of $q$.
\end{proof}

\bparts

\ppart This proof was written for an audience of college teachers, and is
a little more concise than desirable at this point in 6.042.  Write out a
more complete version which includes an explanation of each step.

\begin{solution}
The points that need justification are:

\begin{enumerate}
\item Why is there a postive integer, $q$, such that $\paren{\sqrt{2}-1}q$
  is a nonnegative integer?  \emph{Answer:} Since $\sqrt{2}$ is rational,
  so is $\sqrt{2}-1$.  So $\sqrt{2}-1$ can be expressed as an integer
  quotient with positive denominator; now just let $q$ be that
  denominator.

\item Why is there such a \emph{least} positive integer, $q$?
  \emph{Answer:} As long as there is one such positive integer, there has
  to be a least one.  This obvious fact is known as the
  \emph{Well-Ordering Principle}.

\item Why is $0< q' < q$?  \emph{Answer:} We know that $1 < \sqrt{2} < 2$,
  so $0 < \sqrt{2}-1 < 1$.  Therefore, $0 < \paren{\sqrt{2}-1}r < r$ for any
  real number $r>0$.

\item Why is $(\sqrt{2}-1)q'$ a nonnegative integer?  \emph{Answer:} It's
  actually positive, because it is a product of positive numbers.  It's
  integer because
  \[
  \paren{\sqrt{2}-1}q' = \paren{\sqrt{2}-1}^2q = 2q - 2q\sqrt{2} + q =
  q-2\cdot\brac{\paren{\sqrt{2}-1}q}
  \]
  and the last term is of the form
  $\ang{\text{integer}-2 \cdot \brac{\text{integer}}}$.

\end{enumerate}
\end{solution}

\ppart Now that you have justified the steps in this proof, do you have a
preference for one of these proofs over the other?  Why?  Discuss these
questions with your teammates for a few minutes and summarize your team's
answers on your whiteboard.

\begin{solution}
Both proofs seem about equally easy to understand.  The previous
  problems shows that the first proof generalizes pretty directly from
  square roots to $k$th roots, which doesn't seems as clear for the this
  second proof.  On the other hand, the first proof requires appeal to
  Unique Prime Factorization, while the second just uses simple algebra.
\end{solution}

\eparts
\end{problem}

%%%%%%%%%%%%%%%%%%%%%%%%%%%%%%%%%%%%%%%%%%%%%%%%%%%%%%%%%%%%%%%%%%%%%
% Problems end here
%%%%%%%%%%%%%%%%%%%%%%%%%%%%%%%%%%%%%%%%%%%%%%%%%%%%%%%%%%%%%%%%%%%%%

\end{document}
