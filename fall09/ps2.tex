%minor revision by ARM 9/8/09

\newcommand{\C}{C}

% Define the course staff here.
ppp
\newcommand{\Eric}{\text{Justin }}
\newcommand{\Tom}{\text{Megumi }}
\newcommand{\Albert}{\text{Tom }}
\newcommand{\Claire}{\text{Steven }}
\newcommand{\Edmond}{\text{Richard }}
\newcommand{\Florent}{\text{Jodyann }}
\newcommand{\Nick}{\text{Rajeev }}

\begin{problem}
  A cabal consisting of some of the 6.042 TA's is plotting to make the
  final exam \textit{ridiculously hard}.  (``Problem 1.  Prove the
  Goldbach Conjecture starting from the axioms of ZFC.  Express your
  answer in khipu -- the knot language of the Incas.'')  The only way to
  stop their evil plan is to determine exactly who is in the cabal.  The
  course TA's consists of seven people:
%
\[
\set{\Eric, \Tom, \Albert, \Claire, \Edmond, \Florent, \Nick }
\]
%
The cabal is a subset of these seven.
A membership roster has been found and appears below, but it is deviously
encrypted in logic notation.  The predicate $\C$ indicates who is in the
cabal; that is, $\C(x)$ is true if and only if $x$ is a member.  Translate
each statement below into English and deduce who is in the cabal.

\begin{enumerate}%[\upshape (i)]

\item\label{eee} $\exists x \ \exists y \ \exists z \
    (x \neq y \wedge
     x \neq z \wedge
     y \neq z \wedge
     \C(x) \wedge \C(y) \wedge \C(z))$

\solution{A direct English paraphrase would be ``There
exist people we'll call $x,y$, and $z$, who are all different, such that
$x,y$ and $z$ are each in the cabal.''  A better version would use the
fact that there's no need in this case to give names to the people.
Namely, a better paraphrase is ``There are 3 different people in the
cabal.''  Perhaps a simpler way to say this is: ``The cabal is of size at
least 3.''}

\item\label{nNC} $\neg (\C(\text{\Nick}) \wedge \C(\text{\Claire}))$

\solution{\Nick and \Claire are not both in the cabal.
Equivalently: at least one of \Nick and \Claire is not in the cabal.}

\item\label{Fall} $\C(\text{\Florent}) \rightarrow \forall x \ \C(x)$

\solution{If \Florent is in the cabal, then everyone is.}

\item\label{CN} $\C(\text{\Claire}) \rightarrow \C(\text{\Nick})$

\solution{If \Claire is in the cabal, then \Nick is also.}

\item\label{EAnT}
$(\C(\text{\Edmond}) \vee \C(\text{\Albert})) \rightarrow \neg \C(\text{\Tom})$

\solution{If either of \Edmond or \Albert is in the cabal,
then \Tom is not.  Equivalently, if \Tom \emph{is} in the cabal, the neither
\Albert nor \Edmond is.}

\item\label{ENnE}
$(\C(\text{\Edmond}) \vee \C(\text{\Nick})) \rightarrow \neg \C(\text{\Eric})$

\solution{If either of \Edmond or \Nick is in the cabal,
then \Eric is not.  Equivalently, if \Eric \emph{is} in the cabal, the
neither \Edmond nor \Nick is.  }
\end{enumerate}

\insolutions{So much for the translations.  We now argue that the only
cabal satisfying all six propositions above is one whose members are
exactly \Nick, \Edmond, and \Albert.

We first observe that by~\eqref{nNC}, there must be someone -- either \Nick
or \Claire -- who is not in the cabal.  But if Flo were in the cabal, then
by~\eqref{Fall}, everyone would be.  So we conclude by contradiction
that:

\begin{equation}\label{nF}
\text{\Florent is not in the cabal.}
\end{equation}

Next observe that if \Claire was in the cabal, then by~\eqref{CN}, \Nick would
be too, contradicting~\eqref{nNC}.  So by again contradiction, we conclude:
\begin{equation}\label{nC}
\text{\Claire is not in the cabal.}
\end{equation}

Now suppose \Tom is in the cabal.  Then by~\eqref{EAnT}, \Edmond and \Albert
are not, and we already know \Florent and \Claire are not, so only three remain
who could be in the cabal, namely, \Tom, \Nick, and \Eric.  But
by~\eqref{eee} the cabal must have at least three members, so it follows
that the cabal must consist of exactly these three.  This proves:
\begin{lemma}\label{TNE}
\text{If \Tom is in the cabal, then \Nick and \Eric are in the cabal.}
\end{lemma}

But by~\eqref{ENnE}, if \Nick is the cabal, then \Eric is not.  That is, 
\begin{lemma}\label{NnE}
\text{\Nick and \Eric cannot both be in the cabal.}
\end{lemma}
Now from Lemma~\ref{NnE} we conclude that the conclusion of
Lemma~\ref{TNE} is false.  So by contrapositive, the hypothesis of
Lemma~\ref{TNE} must also be false, namely,
\begin{equation}\label{nT}
\text{\Tom is not in the cabal.}
\end{equation}

Finally, suppose \Eric is in the cabal.  Then by~\eqref{ENnE}, \Edmond
and \Nick are not, and we already know \Florent, \Claire and \Tom are not. So
the cabel must consist of at most two people (\Albert and \Eric). This
contradicts~\eqref{eee}, and we conclude by contradiction that
\begin{equation}\label{nE}
\text{\Eric is not in the cabal.}
\end{equation}
So the only remaining people who could be in the cabal are \Albert, \Edmond,
and \Nick.  Since the cabal must have at least three members, we conclude
that
\begin{lemma}
The only possible cabal consists of \Albert, \Edmond, and \Nick.
\end{lemma}

But we're not done yet: we haven't shown that a cabal consisting of
\Albert, \Edmond, and \Nick actually does satisfy all six conditions.  So let
$\mathcal{A} =\set{\text{\Albert}, \text{\Edmond}, \text{\Nick}}$, and let's quickly
check that $\mathcal{A}$ satisfies~\eqref{eee}--\eqref{ENnE}:

\begin{itemize}

\item $\size{A} = 3$, so $A$ satisfies~\eqref{eee}.
\item \Claire is not in $A$, so $A$ satisfies~\eqref{nNC} and~\eqref{CN}.
\item \Florent is not in $A$, so the hypothesis of~\eqref{Fall} is false, which
means that $A$ satisfies~\eqref{Fall}.
\item Finally, \Tom and \Eric are not in $A$, so the conclusions of
both~\eqref{EAnT} and~\eqref{ENnE} are true, and so $A$
satisfies~\eqref{EAnT} and ~\eqref{ENnE}.

\end{itemize}

So now we have proved
\begin{proposition*}
$\set{\text{\Albert}, \text{\Edmond}, \text{\Nick}}$ is the \emph{unique} cabal
satisfying conditions~\eqref{eee}--\eqref{ENnE}.
\end{proposition*}}
\end{problem}


\begin{problem} %verbatim from Fall 05, 

\bparts
\ppart Give an example where the following result fails:

\begin{falsethm*}
For sets $A$, $B$, $C$, and $D$, let
\begin{align*}
L \eqdef (A \union C) \times (B \union D),\\
R \eqdef (A \times B) \union (C \times D).
\end{align*}
Then $L=R$.
\end{falsethm*}

\solution{
If $A=D=\emptyset$ and $B$ and $C$ are both nonempty, then $L = C \times B
\neq \emptyset$, but $R = \emptyset$.
}

\ppart Identify the mistake in the following proof of the False Theorem.

\begin{bogusproof}
Since $L$ and $R$ are both sets of pairs, it's sufficient to prove that
$(x,y) \in L \iff (x,y) \in R$ for all $x,y$.

The proof will be a chain of iff implications:
\begin{center}
\begin{tabular}{ll}
    & $(x, y) \in L$  \\
iff & $x \in A \union C$ and $y \in B \union D$ \\
iff & either $x \in A$ or $x \in C$, and either $y \in B$ or $y \in D$ \\
iff & ($x \in A$ and $y \in B$) or else ($x \in C$ and $y \in D$)  \\
iff & $(x,y) \in A \times B$, or $(x,y) \in C \times D$ \\
iff & $(x,y) \in (A \times B) \union (C \times D) $ \\
iff & $(x,y)\in R$.
\end{tabular}
\end{center}

\end{bogusproof}

\solution{The mistake is in the third ``iff.''  If [$x \in A$ or $x \in
C$, and either $y \in B$ or $y \in D$], it does not necessarily follow
that $(x,y) \in (A \times B) \union (C \times D)$.  It might be that
$(x,y)$ is in $A \times D$ instead.  This happens, for example, if $A =
\set{1}, B = \set{2}, C = \set{3}, D = \set{4}$, and $(x,y) = (1, 4)$.}

\ppart Fix the proof to show that $R \subseteq L$.

\solution{
Replacing the third ``iff'' by ``which will be true when,'' yields a
correct proof that $(x,y) \in L$ will be true when $(x,y) \in R$, which
implies that $R \subseteq L$.
}

\eparts

\end{problem}
