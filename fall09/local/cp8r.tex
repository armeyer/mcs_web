% Start document if it's a stand alone, otherwise,
% increase the documentdepth counter by one
\ifnum\value{page}=1
  \documentclass[11pt,twoside]{article}
  \newcounter{documentdepth}
  \usepackage{latex-macros/book}
  \handouttrue
  \begin{document}
\else
  \setcounter{documentdepth}{\value{documentdepth}+1}
\fi

\inhandout{
\inclassproblems{8, Thurs.}
\usesimpleproblems
}
%%%%%%%%%%%%%%%%%%%%%%%%%%%%%%%%%%%%%%%%%%%%%%%%%%%%%%%%%%%%%%%%%%%%%
% Problems start here
%%%%%%%%%%%%%%%%%%%%%%%%%%%%%%%%%%%%%%%%%%%%%%%%%%%%%%%%%%%%%%%%%%%%%

%same as S06, cp7w


\begin{problem}
Let's try out RSA!  There is a complete description of the algorithm at
the bottom of the page.  You'll probably need extra paper.  \textbf{Check
your work carefully!}

\bparts

\ppart As a team, go through the \textbf{beforehand} steps.

\begin{itemize}

\item Choose primes $p$ and $q$ to be relatively small, say in the
range 10-40.  In practice, $p$ and $q$ might contain several hundred
digits, but small numbers are easier to handle with pencil and paper.

\item Try $e = 3, 5, 7, \ldots$ until you find something that works.
Use Euclid's algorithm to compute the gcd.

\item Find $d$ (using the Pulverizer ---see appendix for a reminder on how
the Pulverizer works ---or Euler's Theorem).

\end{itemize}

When you're done, put your public key on the board.  This lets another
team send you a message.

\ppart Now send an encrypted message to another team using their
public key.  Select your message $m$ from the codebook below:

\begin{itemize}

\item 2 = Greetings and salutations!

\item 3 = Yo, wassup?

\item 4 = You guys are slow!

\item 5 = All your base are belong to us.

\item 6 = Someone on {\em our} team thinks someone on {\em your} team
is kinda cute.

\item 7 = You {\em are} the weakest link.  Goodbye.

\end{itemize}

\ppart Decrypt the message sent to you and verify that you received
what the other team sent!


\eparts


\begin{center}
RSA Public Key Encryption
\fbox{
\begin{minipage}[t]{6in}
\vspace{0.1cm}
\begin{description}

\item[Beforehand] The receiver creates a public key and a secret key
as follows.

\begin{enumerate}

\item Generate two distinct primes, $p$ and $q$.

\item Let $n = pq$.

\item Select an integer $e$ such that $\gcd(e, (p-1)(q-1)) = 1$.\\ The
{\em public key} is the pair $(e, n)$.  This should be distributed
widely.

\item Compute $d$ such that $de \equiv 1 \pmod{(p-1)(q-1)}$.\\ The
{\em secret key} is the pair $(d, n)$.  This should be kept hidden!

\end{enumerate}

\item[Encoding] The sender encrypts message $m$, where $0 \leq m < n$, to
  produce $m^\prime$ using the public key:
\[
m' = \rem{m^e}{n}.
\]

\item[Decoding] The receiver decrypts message $m'$ back to message $m$
using the secret key:
\[
m = \rem{(m')^d}{n}.
\]

\end{description}

\vspace{0.1cm}
\end{minipage}
}
\end{center}


\end{problem}

%%%%%%%%%%%%%%%%%%%%%%%%%%%%%%%%%%%%%%%%%%%%%%%%%%%%%%%%%%%%%%%%%%%%%%%%%%%%%%%

\instatements{\newpage}
\begin{problem}
  A critical question is whether decrypting an encrypted message always
  gives back the original message!  That is, whether $\rem{(m^{d})^e}{pq}
  =m$.  This will follow from something slightly more general:
\begin{lemma}\label{npdp}
Let $n$ be a product of distinct primes and $a \equiv 1 \pmod {\phi(n)}$
for some nonnegative integer, $a$.  Then
\begin{equation}\label{ma}
m^a \equiv m \pmod n.
\end{equation}
\end{lemma}

\bparts


\ppart Explain why Lemma~\ref{npdp} implies that $k$ and $k^5$ have the
same last digit.  For example:
%
\[
\underline{2}^5 = 3\underline{2}
\hspace{1in}
7\underline{9}^5 = 307705639\underline{9}
\]
\hint What is $\phi(10)$?

\solution{Two nonnegative integers have the same last digit iff they are
$\equiv \pmod {10}$.  Now $\phi(10) = \phi(2)\phi(5) = 4$ and $5 \equiv 1 \pmod 4$,
so by Lemma~\ref{npdp},
\[
k^5 \equiv k \pmod {10}.
\]}

\ppart Explain why Lemma~\ref{npdp} implies that the original message,
$m$, equals $\rem{(m^e)^d}{pq}$.

\solution{ To apply Lemma~\ref{npdp} to RSA, note that the first condition
  of the Lemma is that $n$ be a product of primes.  In RSA, $n=pq$ so this
  condition holds.

  For $n=pq$, the Euler function equations (see the
  Appendix) imply that $\phi(n) = (p-1)(q-1)$.  So when $d$ and $e$ are
  chosen according to RSA, $de \equiv 1 \pmod {\phi(n)}$.  So $a \eqdef
  de$ satisfies the second condition of the Lemma.

  Now, from equation~\eqref{ma} with $n=pq$ and $a=de$, we have
\[
(m^e)^d = m^{de} \equiv m \pmod {pq}.
\]
Hence,
\[
\rem{(m^e)^d}{pq} = \rem{m}{pq},
\]
but $\rem{m}{pq} = m$, since $0\leq m<pq$.}

\eparts

\bparts

\ppart\label{pma} Prove that if $p$ is prime, then
\begin{equation}\label{mp}
m^a \equiv m \pmod{p}
\end{equation}
for all nonnegative integers $a \equiv 1 \pmod {p-1}$.

\solution{
If $p \divides m$, then equation~\eqref{mp} holds since both sides of the
congruence are $\equiv 0 \pmod p$.

So assume $p$ does not divide $m$.  Now if $a \equiv 1 \pmod {p-1}$,
then $a = 1 + (p-1)k$ for some $k$, so
\begin{align*}
m^a & = m^{1+ (p-1)k}\\
    & = m\cdot \paren{m^{p-1}}^k\\
    & \equiv m\cdot \paren{1}^k \pmod p
            & \text{(by Fermat's Little Thm.)}\\
    & \equiv m \pmod p.
\end{align*}}

\iffalse

\ppart\label{abk} Show that for any positive integers $j,k$,
if $a \equiv b \pmod k$ and $j \divides k$, then $a \equiv b \pmod j$

\solution{
$a \equiv b \pmod k$ iff $k \divides (a-b)$.  But if $k \divides (a-b)$
and $j \divides k$, then also $j \divides (a-b)$, which implies $a \equiv
b \pmod j$.
}
\fi

\ppart\label{abp}
Prove that if $n$ is a product of distinct primes, and $a \equiv b
\pmod p$ for all prime factors, $p$, of $n$, then $a \equiv b
\pmod n$.

\solution{ By definition of congruence, $a \equiv b \pmod k$ iff $k
\divides (a-b)$.  So if $a \equiv b \pmod p$ for each prime factor, $p$,
of $n$, then $p \divides (a-b)$ for each prime factor, $p$, and hence, so
does their product (by the Unique Factorization Theorem).  That is, $n
\divides (a-b)$, which means $a \equiv b \pmod n$.}

\iffalse

\ppart\label{phip} Verify that for any $n>1$ and any prime divisor, $p$,
of $n$,
\[
\phi(p) \divides \phi(n).
\]

\solution{Let $p$ be a prime factor of $n$ and factor $n$ as $m\cdot p^k$
where $p$ does not divide $m$.  By the Euler function equations,
$\phi(n)=\phi(m)\phi(p^k)$, so $\phi(p^k) \divides \phi(n)$.  But
$\phi(p) = (p-1)$ which divides $(p-1)p^{k-1} = \phi(p^k)$.}
\fi


\ppart Combine the previous parts to complete the proof of
Lemma~\ref{npdp}.  \solution{Suppose $n$ is a product of distinct primes,
  $p_1p_2\cdots p_k$.  Then from the formulas for the Euler function,
  $\phi$, we have
\[
\phi(n) = (p_1 -1)(p_2 -1)\cdots (p_k-1).
\]

Now suppose $a \equiv 1 \pmod {\phi(n)}$, that is, $a$ is 1 plus a
multiple of $\phi(n)$, so it is also 1 plus a multiple of $p_i-1$.  That
is,
\[
a \equiv 1 \pmod {p_i-1}.
\]
Hence, by part~\eqref{pma},
\[
m^a \equiv m \pmod{p_i}
\]
for all $m$.  Since this holds for all factors, $p_i$, of $n$, we conclude
from part~\eqref{abp} that
\[
m^a \equiv m \pmod{n},
\]
which proves Lemma~\ref{npdp}.}

\eparts

\end{problem}

\instatements{\newpage}

\section*{Appendix}

\subsection*{Inverses, Fermat, Euler}

\begin{lemma*}[Inverses mod $n$]
If $k$ and $n$ are relatively prime, then there is integer
$k'$ called the \emph{modulo $n$ inverse} of $k$, such that
\[
k \cdot k' \equiv 1 \pmod n.
\]
\end{lemma*}

\textbf{Remark:} If $\gcd(k,n) =1$, then $sk+tn = 1$ for some $s,t$, so we
can choose $k' \eqdef s$ in the previous Lemma.  So given $k$ and $n$, an
inverse $k'$ can be found efficiently using the Pulverizer.

\begin{theorem*}[Fermat's (Little) Theorem]
If $p$ is prime and $k$ is not a multiple of $p$, then
\[
k^{p-1} \equiv 1 \pmod{p}
\]
\end{theorem*}

\begin{definition*}
The value of \emph{Euler's totient function}, $\phi(n)$, is defined to be
the number of positive integers less than $n$ that are relatively prime to
$n$.
\end{definition*}

\begin{lemma*}[Euler Totient Function Equations]
\begin{align*}
\phi(p^k)& = p^k - p^{k-1} & \text{for prime, $p$, and $k>0$},\\
\phi(mn) & = \phi(m)\cdot \phi(n) &\text{ when } \gcd(m,n)=1.
\end{align*}
\end{lemma*}

\begin{theorem*}[Euler's Theorem]
If $k$ and $n$ are relatively prime, then
\[
k^{\phi(n)} \equiv 1 \pmod{n}
\]
\end{theorem*}

\begin{corollary*}
If $k$ and $n$ are relatively prime, then $k^{\phi(n)-1}$ is an inverse
modulo $n$ of $k$.
\end{corollary*}

\textbf{Remark:} Using fast exponentiation to compute $k^{\phi(n)-1}$ is
another efficient way to compute an inverse modulo $n$ of $k$.

\subsection*{The Pulverizer}

Euclid's algorithm for finding the GCD of two numbers relies on
repeated application of the equation: 
\[
\gcd(a, b) = \gcd(b, \rem{a}{b})
\]
For example, we can compute the GCD of 259 and 70 as follows:
\[
\begin{array}{rclcl}
\gcd(259, 70)
    & = & \gcd(70, 49) & \quad & \text{since $\rem{259}{70} = 49$}\\
    & = & \gcd(49, 21) && \text{since $\rem{70}{49} = 21$} \\
    & = & \gcd(21, 7) && \text{since $\rem{49}{21} = 7$} \\
    & = & \gcd(7, 0) && \text{since $\rem{21}{7} = 0$} \\
    & = & 7.
\end{array}
\]
The Pulverizer goes through the same steps, but requires some extra
bookkeeping along the way: as we compute $\gcd(a, b)$, we keep track
of how to write each of the remainders (49, 21, and 7, in the example)
as a linear combination of $a$ and $b$ (this is worthwhile, because
our objective is to write the last nonzero remainder, which is the
GCD, as such a linear combination).  For our example, here is this
extra bookkeeping:
\[
\begin{array}{ccccrcl}
x & \quad & y & \quad & \rem{x}{y} & = & x - q \cdot y \\ \hline
259 && 70 && 49 & = &   259 - 3 \cdot 70 \\
70 && 49 && 21  & = &   70 - 1 \cdot 49 \\
&&&&            & = &   70 - 1 \cdot (259 - 3 \cdot 70) \\
&&&&            & = &   -1 \cdot 259 + 4 \cdot 70 \\
49 && 21 && 7   & = &   49 - 2 \cdot 21 \\
&&&&            & = &   (259 - 3 \cdot 70) -
                                2 \cdot (-1 \cdot 259 + 4 \cdot 70) \\
&&&&            & = &   \fbox{$3 \cdot 259 - 11 \cdot 70$} \\
21 && 7 && 0
\end{array}
\]
We began by initializing two variables, $x = a$ and $y = b$.  In the
first two columns above, we carried out Euclid's algorithm.  At each
step, we computed $\rem{x}{y}$, which can be written in the form $x - q
\cdot y$.  (Remember that the Division Algorithm says $x = q \cdot y +
r$, where $r$ is the remainder.  We get $r = x - q \cdot y$ by
rearranging terms.)  Then we replaced $x$ and $y$ in this equation
with equivalent linear combinations of $a$ and $b$, which we already
had computed.  After simplifying, we were left with a linear
combination of $a$ and $b$ that was equal to the remainder as desired.
The final solution is boxed.

%%%%%%%%%%%%%%%%%%%%%%%%%%%%%%%%%%%%%%%%%%%%%%%%%%%%%%%%%%%%%%%%%%%%%
% Problems end here
%%%%%%%%%%%%%%%%%%%%%%%%%%%%%%%%%%%%%%%%%%%%%%%%%%%%%%%%%%%%%%%%%%%%%
% End document if it's a stand alone and decrease documentdepth otherwise
\ifnum\value{documentdepth}>0
  \setcounter{documentdepth}{\value{documentdepth}-1}
  \endinput
\else
  \end{document}
\fi

