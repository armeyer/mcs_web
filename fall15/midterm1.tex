\documentclass[quiz]{mcs}

\renewcommand{\exampreamble}{   % !! renew \exampreamble
    \textbf{Indicate your}\ \teaminfo

  \begin{itemize}

  \item
   This exam is \textbf{closed book} except for a 2-sided cribsheet.
   Total time is 60 minutes. 

  \item
   Write your solutions in the space provided.  If you need more
   space, write on the back of the sheet containing the problem.

%   Please keep your entire answer to a problem on that problem's page.
   
   \item In answering the following questions, you may use without
     proof any of the results from class or text.
\iffalse (unless explicitly instructed otherwise)\fi

\iffalse
  \item
   GOOD LUCK!
\fi

  \end{itemize}}

\begin{document}

\midterm{September 24}

%%%%%%%%%%%%%%%%%%%%%%%%%%%%%%%%%%%%%%%%%%%%%%%%%%%%%%%%%%%%%%%%%%%%%
% Problems start here
%%%%%%%%%%%%%%%%%%%%%%%%%%%%%%%%%%%%%%%%%%%%%%%%%%%%%%%%%%%%%%%%%%%%

\examspace

\begin{center}
{\Large DRAFT}
\end{center}

\begin{center}
{\large Proofs}
\end{center}

UNCOVERED TOPIC (OK to skip): simple buggy proof?

\begin{problem}[points = 6, title= \textbf{Surprise paradox}]
Why is the ``surprise'' paradox a philosophical but not a mathematical one?

\begin{solution}
The concept of ``surprise,'' like similar concepts such as ``knowing''
or ``believing'' have no satisfactory mathematical definitions.
\end{solution}

\end{problem}

\examspace

\begin{center}
{\large Contradiction}
\end{center}

\pinput[points = 6, title= \textbf{Irrational
    logarithm}]{MQ_log12_of_18_irrational}

\pinput[points = 6, title= \textbf{Irrational
    logarithm}]{MQ_seventh_root_35_irrational}

%MQ_log4_of_6_irrational on ps1
\examspace

\begin{center}
{\large Well Ordering}
\end{center}

\pinput[points = 3, title= \textbf{Minimum elements}]{TP_minimum_elements}

\pinput[points = 10, title= \textbf{Well ordering}]{FP_wop_nn1}

\pinput[points = 10, title= \textbf{Well ordering}]{MQ_wop_proof_sumofcubes}

\pinput[points = 8, title= \textbf{Well ordering}]{CP_like_Lehmans_equation}

\pinput[points = 10, title= \textbf{Well ordering}]{FP_6_and_15_cent_stamps_by_WOP}

\examspace

\begin{center}
{\large Prop Logic}
\end{center}

UNCOVERED TOPIC (good to have, but OK to skip): binary adder circuits

\pinput[points = 4, title= \textbf{Satisfiability}]{TP_satisfiable_vs_valid}

\pinput[points = 8, title= \textbf{Truth tables, Cases}]
{MQ_truth_table_case_reasoning}

Conflict version of above is: MQ\_truth\_table\_case\_reasoning\_afternoon

\examspace

\begin{center}
{\large Predicate Logic}
\end{center}

UNCOVERED TOPICS (OK to skip): quantifer swap, counter model

\pinput[points = 6, title= \textbf{Predicate Formulas}]{MQ_equality_logic}

\pinput[points = 6, title= \textbf{Predicate Formulas}]{FP_logic_of_leq}

\pinput[points = 6, title= \textbf{Predicate Formulas}]{PS_emailed_exactly_2_others}

\begin{center}
{\large Sets and Sequences}
\end{center}

TOPICS: Find variants of TP\_basic\_set\_formulas,
CP\_proving\_basic\_set\_identity

UNCOVERED TOPIC (OK to skip): CP\_set\_pairing

%%%%%%%%%%%%%%%%%%%%%%%%%%%%%%%%%%%%%%%%%%%%%%%%%%%%%%%%%%%%%%%%%%%%%
% Problems end here
%%%%%%%%%%%%%%%%%%%%%%%%%%%%%%%%%%%%%%%%%%%%%%%%%%%%%%%%%%%%%%%%%%%%%
\end{document}
