\documentclass[quiz]{mcs}

\renewcommand{\exampreamble}{   % !! renew \exampreamble
    \textbf{Indicate your}\ \teaminfo

  \begin{itemize}

  \item
   This exam is \textbf{closed book} except for a 2-sided cribsheet.
   Total time is 60 minutes. 

  \item
   Write your solutions in the space provided.  If you need more
   space, write on the back of the sheet containing the problem.

%   Please keep your entire answer to a problem on that problem's page.
   
   \item In answering the following questions, you may use without
     proof any of the results from class or text.
\iffalse (unless explicitly instructed otherwise)\fi

\iffalse
  \item
   GOOD LUCK!
\fi

  \end{itemize}}

\begin{document}

\midterm{September 24}

%%%%%%%%%%%%%%%%%%%%%%%%%%%%%%%%%%%%%%%%%%%%%%%%%%%%%%%%%%%%%%%%%%%%%
% Problems start here
%%%%%%%%%%%%%%%%%%%%%%%%%%%%%%%%%%%%%%%%%%%%%%%%%%%%%%%%%%%%%%%%%%%%

\examspace

\begin{center}
{\Large DRAFT}
\end{center}

\begin{center}
{\large Proofs}
\end{center}

UNCOVERED TOPIC: simple buggy proof?

\begin{problem}[points = 6, title= \textbf{Surprise paradox}]
Why is the ``surprise'' paradox a philosophical but not a mathematical one?

\begin{solution}
The concept of ``surprise,'' like similar concepts such as ``knowing''
or ``believing'' have no satisfactory mathematical definitions.
\end{solution}

\end{problem}

\examspace

\begin{center}
{\large Contradiction}
\end{center}

\pinput[points = 6, title= \textbf{Irrational logarithm}]{MQ_log4_of_6_irrational}

\examspace

\begin{center}
{\large Well Ordering}
\end{center}

\begin{problem}[points = 6, title= \textbf{Minimum element sets}]
For each  of the following sets, describe a subset with no minimum element:
\bparts
\ppart $\integers$.
\examspace[0.5in]

\ppart $\rationals^{\geq 0}$,
\examspace[0.5in]

\ppart irrational real numbers $\geq \sqrt{2}$.
\examspace[0.5in]
\eparts
\end{problem}

\pinput[points = 6, title= \textbf{Well ordering}]{FP_wop_nn1}

\pinput[points = 6, title= \textbf{Well ordering}]{MQ_wop_proof_sumofcubes}

\pinput[points = 6, title= \textbf{Well ordering}]{CP_like_Lehmans_equation}

\pinput[points = 6, title= \textbf{Well ordering}]{FP_6_and_15_cent_stamps_by_WOP}

\examspace

\begin{center}
{\large Prop Logic}
\end{center}

UNCOVERED TOPIC: binary adder circuits

\begin{problem}[points = 6, title= \textbf{Satisfiability}]

\begin{pcomments}
  \pcomment{TP_satisfiable_vs_valid}
  \pcomment{trivial variant of CP_valid_vs_satisfiable}
  \pcomment{ARM 9/20/15}
\end{pcomments}

Explain why a propositional formula $P$ is satisfiable iff its
negation $\QNOT(P)$ is \emph{not} valid.

\begin{solution}
Essentially the same as the solution to~\bref{CP_valid_vs_satisfiable}

To prove the iff, we prove that the left hand statement implies the
right hand one and vice-versa.

\textbf{(left-to-right case)}: If $P$ is satisfiable, then $\QNOT(P)$ is \emph{not}
valid.

\begin{proof}
$P$ is true in an environment iff $\QNOT(P)$ is false in that
environment.  Since $P$ is satisfiable, it is true in some environment, which
means that $\QNOT(P)$ is false in some environment.  So not all environments make
$\QNOT(P)$ true, which means that $\QNOT(P)$ is \emph{not} valid.
\end{proof}

\textbf{(right-to-left case)}: If $\QNOT(P)$ is \emph{not}
valid, the $P$ is satisfiable.

\begin{proof}
If $\QNOT(P)$ is not valid, some environment makes it false, and
therefore this environment make $P$ true.  This means that $P$ that
$P$ is satisfiable by defintion..
\end{proof}

\end{solution}

\end{problem}

\pinput[points = 6, title= \textbf{Truth tables, Cases}]
{MQ_truth_table_case_reasoning}

Conflict version of above: MQ\_truth\_table\_case\_reasoning\_afternoon

\examspace

\begin{center}
{\large Predicate Logic}
\end{center}

UNCOVERED TOPICS: quantifer swap, counter model

\begin{problem}[points = 6, title= \textbf{Predicate Formulas}]
Predicate Formulas whose only predicate symbol is equality are called
``pure equality'' formulas.  For example,
\begin{equation}
\forall x\, \forall y.\, x = y\tag{1-element}
\end{equation}
is a pure equality formula.  Its meaning is that there is exactly one
element in the domain of discourse.\footnote{Reminder, a domain of
  discourse is not allowed to be empty.}  Another such formula
equivalent to~(1-element) is
\begin{equation}
\exists a\, \exists b\, \forall x.\, x=a \QOR x = b. \tag{$\leq 2$-elements}
\end{equation}
Its meaning is that there are at most two elements in the domain of
discourse.

\bparts

\ppart Write a pure equality formula that means that there are
\emph{exactly} two elements in the domain of discourse.

\examspace[2.0in]

\begin{solution}
A simple answer using the above formulas is:
\[
\text{formula~($\leq 2$-elements)} \QAND \QNOT(\text{formula~(1-element)})
\]

An alternative is to use a variant of~($\leq 2$-elements):
\[
\exists a,b.\, a \neq b \QAND \forall x.\, x=a \QOR x = b.
\]
\end{solution}

\ppart Write a pure equality formula that means that there are
\emph{exactly} three elements in the domain of discourse.

\examspace[2.0in]

\begin{solution}
We can say there are at most three elements using
\begin{equation}
\exists a,b,c\, \forall x.\, x=a \QOR x = b \QOR x =c.\tag{$\leq 3$-elements}
\end{equation}
Then  exactly three elements can be expressed with
\[
\text{formula~($\leq 3$-elements)} \QAND 
\QNOT(\text{formula~($\leq 2$-elements)}).
\]

An alternative is
\[
\exists a,b,c.\, a \neq b \QAND b \neq c \QAND a \neq c \QAND \forall
x.\, x=a \QOR x = b \QOR x = c.
\]
\end{solution}
\eparts

\end{problem}

\pinput[points = 6, title= \textbf{Predicate Formulas}]{FP_logic_of_leq}

\pinput[points = 6, title= \textbf{Predicate Formulas}]{PS_emailed_exactly_2_others}

\begin{center}
{\large Sets and Sequences}
\end{center}

Find variants of TP\_basic\_set\_formulas,
CP\_proving\_basic\_set\_identity

UNCOVERED TOPIC (OK to skip): CP\_set\_pairing

%%%%%%%%%%%%%%%%%%%%%%%%%%%%%%%%%%%%%%%%%%%%%%%%%%%%%%%%%%%%%%%%%%%%%
% Problems end here
%%%%%%%%%%%%%%%%%%%%%%%%%%%%%%%%%%%%%%%%%%%%%%%%%%%%%%%%%%%%%%%%%%%%%
\end{document}
