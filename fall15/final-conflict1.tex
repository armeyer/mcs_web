%final-conflict

\documentclass[quiz]{mcs}

%\renewcommand{\examspace}[1][]{}

\renewcommand{\exampreamble}{   % !! renew \exampreamble

\begin{center}
{\large   \textbf{Circle your}\qquad   \teaminfo}
\end{center}

  \begin{itemize}

  \item
   This exam is \textbf{closed book} except for two 2-sided cribsheets.
   Total time is 180 minutes.

  \item

   Write your solutions in the space provided.  If you need more
   space, \textbf{write on the back} of the sheet containing the
   problem.

%   Please keep your entire answer to a problem on that problem's page.   
   \item In answering the following questions, you may use without
     proof any of the results from class or text.

   \item Incorrect short answers are eligible for part credit when
     there is an explanation.

\iffalse
  \item
   GOOD LUCK!
\fi

  \end{itemize}}

\begin{document}

\conflictfinalone

%%%%%%%%%%%%%%%%%%%%%%%%%%%%%%%%%%%%%%%%%%%%%%%%%%%%%%%%%%%%%%%%%%%%%
% Problems start here
%%%%%%%%%%%%%%%%%%%%%%%%%%%%%%%%%%%%%%%%%%%%%%%%%%%%%%%%%%%%%%%%%%%%

\begin{staffnotes}
\begin{center}
{\Large DRAFT}
\end{center}

\end{staffnotes}

\pinput[points = 16, title = \textbf{Structural Induction}]
{FP_arith_trig_functions}
\examspace 

\pinput[points = 8, title = \textbf{Exponential mod $n$}]
{FP_Euler_theorem_calculation}
\examspace

\iffalse
\pinput[points = 14, title = \textbf{Combinatorial Proof}]
{FP_combinatorial_binomial}
\examspace
\fi

\pinput[points = 14, title = \textbf{Magic Trick Redux}]
{FP_magic_trick_27_cards}
\examspace

\pinput[points = 16, title= \textbf{Tree coloring}]
{FP_tree_kcolor}
\examspace

\pinput[points = 16, title= \textbf{Counting Graphs}]
{PS_counting_graphs}
\examspace

\pinput[points = 14, title= \textbf{Stable Matching}]
{FP_stable_matching_unlucky}
\examspace

\pinput[points = 18, title= \textbf{Generating functions}]
{FP_boat_trip_fall11}
\examspace

\pinput[points = 8, title= \textbf{Probability, Propositional Logic}]
{FP_satisfy_implies_probability}
\examspace

\pinput[points = 18, title = \textbf{Probability}]
{FP_red_and_blue_goats_fall11}
\examspace

\pinput[points = 12, title = \textbf{Conditional Probability}]
{FP_neighborhood_census}
\examspace

\pinput[points = 20, title = \textbf{Counting, Conditional Probability}]
{FP_random_grid_walk}
\examspace

\iffalse
\pinput[points = 10, title= \textbf{Expectation}]
{FP_expectation_dice}

\pinput[points = 16, title= \textbf{Probability, Logic, Expectation}]
{CP_probable_satisfiability_nk}
\examspace
\fi

\pinput[points = 20, title = \textbf{Expectation}]
{CP_expected_number_of_keys}
\examspace

%%%%%%%%%%%%%%%%%%%%%%%%%%%%%%%%%%%%%%%%%%%%%%%%%%%%%%%%%%%%%%%%%%%%%
% Problems end here
%%%%%%%%%%%%%%%%%%%%%%%%%%%%%%%%%%%%%%%%%%%%%%%%%%%%%%%%%%%%%%%%%%%%%

\begin{staffnotes}
\begin{verbatim}
\pinput[points = 1, title= \textbf{Graphs, Logic, Probability}]
{CP_graph_logic_probability}

\pinput[points = 1, title= \textbf{Counting}]
{FP_more_counting_F15}

\pinput[points = 10, title = \textbf{Magic Trick Redux}]
{FP_magic_trick_27_cards}

\pinput[points = 12, title = \textbf{Expectation}]
{MQ_infinite_repeat}

\end{verbatim}

Would need revision since was already used as CP;
later parts depend heavily on earlier:

\begin{verbatim}
\pinput[points = 10, title = \textbf{Variance \& Deviation}]
{CP_chebyshev_hat_check}
\end{verbatim}

\begin{center}
{\large Week 11 -- Tomas}
\end{center}

\begin{verbatim}
\pinput[points = 15, title = \textbf{Graphs}]
{FP_graphs_short_answer}

\pinput[points = 10, title = \textbf{Partial orders}]
{FP_partial_order_short_answer}

\pinput[points = 10, title = \textbf{Big Oh}]
{MQ_big_oh_def}

\pinput[points = 10, title = \textbf{Counting passwords}]
{CP_inclusion-exclusion_passwords}

\pinput[points = 6, title = \textbf{Matching}]
{FP_bipartite_matching_sex}
\end{verbatim}

Fall 11 final

\begin{verbatim}
\pinput[points = 9, title = \textbf{numbers short answer}]
{FP_numbers_short_answer_fall11}

\pinput[points = 7, title = \textbf{asymptotics}]
{FP_asymptotics_define_functions}

\pinput[points = 6, title = \textbf{lining up fall11}]
{FP_lining_up_F15}

\pinput[points = 6, title = \textbf{probability}]
{FP_red_and_blue_goats_fall11}
\end{verbatim}

Spring 12

\begin{verbatim}
\pinput[points = 7, title = \textbf{Relations Short Answer}]
{FP_partial_order_short_answer_S12}

\pinput[points = 6, title = \textbf{Super 7}]
{CP_7777}
\end{verbatim}

Spring 13

\begin{verbatim}
\pinput[points = 8, title = \textbf{Graphs, Congruences}]
{FP_multiple_choice}
\end{verbatim}

Fall 13

\begin{verbatim}
\pinput[points = 9, title = \textbf{Counting Graphs \& Relations}]
{FP_counting_graphs_f13}
\end{verbatim}

\begin{center}
{\large Week 3 -- Tomas \& Mike}
\end{center}
\begin{verbatim}
FP_relation_properties_expressions_final_f13
\end{verbatim}
Nice problem, but I don't think we've had a prob like this before.

\begin{center}
{\large Mid 1 topics -- ARM }
\end{center}

\begin{verbatim}
\pinput[points = 6, title= \textbf{Well Ordering Principle}]
{FP_wop_nn1}
\end{verbatim}

\begin{center}
{\large Week 4 -- Tanya and Elizabeth S.}
\end{center}

\begin{verbatim}
\pinput[points = 6, title= \textbf{Structural Induction}]
{FP_structural_induction_arithmetic_expressions}
\end{verbatim}

Use one of these two stable marriage problems for conflict.  The first
is somehwhat harder.

\begin{verbatim}
FP_santa_state_machine
\end{verbatim}

\begin{center}
{\large Week 6 -- Tasha and Lisa}
\end{center}

\begin{verbatim}
\pinput[points = 6, title= \textbf{Euler's and RSA}]
{FP_RSA_TF_f15}

\pinput[points = 6, title= \textbf{Scheduling}]
{FP_chains_scheduling}

\pinput[points = 6, title= \textbf{DAGs}]
{MQ_minimum_DAG_positive_walk_relation} % (might want to combine with another short problem)

(aka the question we almost kept for midterm 3)
(or one of its variants - if I remember correctly, it has a few very similar versions)

\end{verbatim}

\begin{center}
{\large Midterm 2 -- Isabella and Xavier}
\end{center}

\begin{verbatim}
\pinput[points = 6, title= \textbf{Stable Marriage}]
{FP_stable_matching_unequal_invariants}

\end{verbatim}

\begin{center}
{\large Week 7 -- Julian and Harlin}
\end{center}
Here's a few problems that will likely need shortening/modification.

\begin{verbatim}
\pinput[points = 6, title= \textbf{Partial Order}]
{CP_minimal_maximal_elements}

\pinput[points = 6, title= \textbf{Digraphs}]
{CP_covering_edges}

CP_weak_partial_order_isomorphic_to_subset
\pinput[points = 6, title= \textbf{Partial Order}]
{CP_weak_partial_order_isomorphic_to_subset}
\end{verbatim}
isomorphic\_to\_subset too long, pedantic

\begin{center}
{\large Week 8 -- Annie and Jodie}
\end{center}
\begin{verbatim}
\pinput[points = 6, title= \textbf{Trees \& Paths}]
{TP_average_degree_of_tree_and_simple_path}

\pinput[points = 6, title= \textbf{Tree}]
{MQ_tree_plus_edge}

\pinput[points = 6, title= \textbf{Tree coloring}]
{FP_tree_kcolor}

\pinput[points = 6, title= \textbf{Harmonic Sum}]
{PS_bug_on_rug_harmonic_number}

\pinput[points = 6, title= \textbf{Connectedness}]
{CP_remove_connected}

\pinput[points = 6, title= \textbf{Coloring, induction}]
{PS_coloring_induction}

\pinput[points = 6, title= \textbf{Trees, Induction}]
{PS_tree_degree_sequence}
\end{verbatim}

\begin{verbatim}
TP_Summation
MQ_Summation_with_hint
\pinput[points = 6, title= \textbf{Coloring}]
{PS_graph_colorable}
\end{verbatim}

\begin{center}
{\large Week 9 -- Maria and Tanya}
\end{center}

For following with explanations required, so we would need to give
more than 6 points, maybe 10.
\begin{verbatim}
\pinput[points = 1, title= \textbf{Asymptotics}]
{MQ_asymptotic_true_false_makeup} 

\pinput[points = 1, title= \textbf{Asymptotics}]
{MQ_asymptotic_incomparable}

\pinput[points = 1, title= \textbf{Counting with Bijection}]
{FP_bijection_counting}

\pinput[points = 1, title= \textbf{Counting}]
{FP_toy_button_counting}

\end{verbatim}

\begin{center}
{\large Week 10 -- Parker}
\end{center}

\begin{verbatim}
\pinput[points = 1, title= \textbf{Generating Functions}]
{FP_dangerous_dan_gen_func_S14 }
\end{verbatim}

too tricky:
\begin{verbatim}
\pinput[points = 1, title= \textbf{Combinatorics}]
{PS_3_friends}
\end{verbatim}

\begin{center}
{\large Week 12 -- Jodie and Harlin}
\end{center}

\begin{verbatim}
\pinput[points = 1, title= \textbf{Independence}]
{TP_mutual_independent_pairs.tex}

\pinput[points = 1, title= \textbf{Graphs, Logic, Probability}]
{CP_graph_logic_probability}

\end{verbatim}

\begin{center}
{\large Week 13 -- Julian and Elizabeth S.}
\end{center}

\begin{verbatim}
\pinput[points = 1, title= \textbf{Expectation}]
{FP_expectation_dice}

\pinput[points = 1, title= \textbf{Expectation}]
{FP_expected_adjacent}

\end{verbatim}

too much story for final:
\begin{verbatim}
\pinput[points = 1, title= \textbf{Probability}]
{PS_testing_soldiers}
\end{verbatim}

From Mike:

\begin{verbatim}
\pinput[points = 1, title= \textbf{induction, with counting}]
{FP_hockey_stick_formula}

\pinput[points = 1, title= \textbf{Logic, Relations}]
{FP_logic_relations}

\pinput[points = 1, title= \textbf{relations, counting, graphs}]
{PS_counting_graphs}

\pinput[points = 1, title= \textbf{Probability, Propositional Logic}]
{FP_satisfy_implies_probability}
\end{verbatim}

Good problem that would require modification:
\begin{verbatim}
\pinput[points = 1, title= \textbf{Probability}]
{PS_fair_ruin_probability}
\end{verbatim}

Spring '10 final

\begin{verbatim}
\pinput[points = 10, title = \textbf{Asymptotic Bounds and Partial Orders}]
{FP_asymptotic_partial_order}

\pinput[points = 10, title = \textbf{Combinatorial Proof}]
{FP_combinatorial_binomial}

\pinput[points = 1, title= \textbf{Variance}]
{FP_variance_dice_sum}

\pinput[points = 6, title = \textbf{Structural Induction}]
{PS_red_black_tree_induction}
\end{verbatim}

\end{staffnotes}

\end{document}
