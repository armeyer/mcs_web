\documentclass[handout]{mcs}

\begin{document}

\renewcommand{\reading}{
\begin{itemize}
\item Chapter~\bref{recursive_data_chap}.\ \emph{Recursive Data}
\item Chapter~\bref{set_theory_chap}.\ \emph{Infinite Sets}
\item Chapter~\bref{number_theory_chap}.\ \emph{Number Theory} through~\bref{fundamental_theorem_sec}.\ \emph{Fundamental Theorem of
    Arithmetic}
\end{itemize}
}

\problemset{4}

%%%%%%%%%%%%%%%%%%%%%%%%%%%%%%%%%%%%%%%%%%%%%%%%%%%%%%%%%%%%%%%%%%%%%
% Problems start here
%%%%%%%%%%%%%%%%%%%%%%%%%%%%%%%%%%%%%%%%%%%%%%%%%%%%%%%%%%%%%%%%%%%%%

%\pinput{PS_uncountable_infinite_sequences}

\pinput{CP_finite_strings_of_nonneg}

\pinput{CP_N_to_N_diagonal_argument}

%\pinput{PS_unit_interval}

%\pinput{PS_eval_cong_aexp}

\pinput{PS_check_factor_by_digits}

\pinput{PS_non_unique_factoring}

%\pinput{PS_pulverizer_machine}

%\pinput{FP_countable_sets}

%\pinput{PS_chinese_remainder_general}

%\pinput{PS_calculating_inverses}

%\pinput{CP_proving_basic_congruence_properties}

%%%%%%%%%%%%%%%%%%%%%%%%%%%%%%%%%%%%%%%%%%%%%%%%%%%%%%%%%%%%%%%%%%%%%
% Problems end here
%%%%%%%%%%%%%%%%%%%%%%%%%%%%%%%%%%%%%%%%%%%%%%%%%%%%%%%%%%%%%%%%%%%%%
\end{document}
