\documentclass[quiz]{mcs}

\renewcommand{\exampreamble}{   % !! renew \exampreamble
     \textbf{Indicate your}\ \teaminfo

  \begin{itemize}

  \item
   This exam is \textbf{closed book} except for a 2-sided cribsheet.
   Total time is 90 minutes. 

  \item
   Write your solutions in the space provided.  If you need more
   space, write on the back of the sheet containing the problem.

%   Please keep your entire answer to a problem on that problem's page.
   
   \item In answering the following questions, you may use without
     proof any of the results from class or text.
     \iffalse (unless explicitly instructed otherwise).\fi

     \iffalse
   \item
     GOOD LUCK!
     \fi

\end{itemize}}

\begin{document}

\midterm{October 15}

%%%%%%%%%%%%%%%%%%%%%%%%%%%%%%%%%%%%%%%%%%%%%%%%%%%%%%%%%%%%%%%%%%%%%
% Problems start here
%%%%%%%%%%%%%%%%%%%%%%%%%%%%%%%%%%%%%%%%%%%%%%%%%%%%%%%%%%%%%%%%%%%%

\examspace

\begin{center}
{\Large DRAFT}
\end{center}

\begin{center}
{\large Binary Relations, Mapping Lemma}
\end{center}
%1
\pinput[points = 7, title= \textbf{Binary Relations}]{PS_binary_relations_on_a_set}
%2
\pinput[points = 7, title= \textbf{Binary Relations}]{TP_composition_of_jections}
%3
\pinput[points = 7, title= \textbf{Binary Relations}]{TP_Inverse_Relations}
%4
\pinput[points = 7, title= \textbf{Binary Relations}]{TP_total_inj_not_bij}


\begin{center}
{\large Induction}
\end{center}
%5
\pinput[points = 7, title= \textbf{Induction}]{TP_a_bogus_fibonacci_induction}
%6
\pinput[points = 7, title= \textbf{Induction}]{TP_divide_product_induction}
%7
\pinput[points = 7, title= \textbf{Induction}]{CP_bogus_induction_nth_power}
%8
\pinput[points = 7, title= \textbf{Induction}]{FP_prove_strong_induction}


\begin{center}
{\large State machines}
\end{center}

%9
\pinput[points = 7, title= \textbf{State Machines}]{CP_10_heads_and_100_tails}
%10
\pinput[points = 7, title= \textbf{State Machines}]{MQ_state_machine_invariant_afternoon}
%11
\pinput[points = 7, title= \textbf{State Machines}]{MQ_state_machine_invariant_morning}
%12
\pinput[points = 7, title= \textbf{State Machines}]{CP_98_heads_and_4_tails}

\begin{center}
{\large Stable Marriage}
\end{center}

%13
\pinput[points = 7, title= \textbf{Stable Marriage}]{MQ_stable_matching_unique_morning}
%14
\pinput[points = 7, title= \textbf{Stable Marriage}]{TP_Stable_Marriage_Invariants}
%15
\pinput[points = 7, title= \textbf{Stable Marriage}]{TP_stable_match_example}
%16
\pinput[points = 7, title= \textbf{Stable Marriage}]{FP_stable_matching_unequal_invariants}

\begin{center}
{\large Recursive Data}
\end{center}

%17
\pinput[points = 7, title= \textbf{Recursive Data}]{MQ_ambiguous_recursive_def}
%18
\pinput[points = 7, title= \textbf{Recursive Data}]{MQ_ambiguous_recursive-def_morning}
%19
\pinput[points = 7, title= \textbf{Recursive Data}]{FP_structural_induction_rational_composition_S13}
%20
\pinput[points = 7, title= \textbf{Recursive Data}]{FP_rational_structural_induction}
%21
\pinput[points = 7, title= \textbf{Recursive Data}]{FP_structural_ind_polynomials}

\begin{center}
{\large Sets}
\end{center}

%22
\pinput[points = 7, title= \textbf{Sets}]{TP_cardinality_class}
%\pinput[points = 7, title= \textbf{Sets}]{CP_N_to_N_diagonal_argument}
%23
\pinput[points = 7, title= \textbf{Sets}]{FP_countable_quadratics}
%24
\pinput[points = 7, title= \textbf{Sets}]{TP_uncountable_example}

\begin{center}
{\large Number Theory: GCDs and Congruence}
\end{center}

%25
\pinput[points = 7, title= \textbf{NT}]{CP_GCD_algebra}
%26
\pinput[points = 7, title= \textbf{NT}]{CP_relative_primality_under_remainder}
%27
\pinput[points = 7, title= \textbf{NT}]{FP_GCD_algebra}
%28
\pinput[points = 7, title= \textbf{NT}]{TP_GCDs_I}
%29
\pinput[points = 7, title= \textbf{NT}]{PS_gcd_termination}
%30
\pinput[points = 7, title= \textbf{NT}]{TP_GCDs_II}


\begin{center}
{\large Euler's theorem, NOT RSA}
\end{center}
%31
\pinput[points = 7, title= \textbf{Euler}]{TP_Eulers_Theorem}
%32
\pinput[points = 7, title= \textbf{Euler}]{TP_Fermats_Little_Theorem_F13}
%33
\pinput[points = 7, title= \textbf{Euler}]{TP_relative_primality_3780}
%34
\pinput[points = 7, title= \textbf{Euler}]{FP_modular_powerful}
%35
\pinput[points = 7, title= \textbf{Euler}]{FP_bogus_Fermat_theorem.spring12}
%36
\pinput[points = 7, title= \textbf{Euler}]{TP_inverse_by_Fermat}



%%%%%%%%%%%%%%%%%%%%%%%%%%%%%%%%%%%%%%%%%%%%%%%%%%%%%%%%%%%%%%%%%%%%%
% Problems end here
%%%%%%%%%%%%%%%%%%%%%%%%%%%%%%%%%%%%%%%%%%%%%%%%%%%%%%%%%%%%%%%%%%%%%
\end{document}
