\documentclass[quiz]{mcs}

\renewcommand{\exampreamble}{   % !! renew \exampreamble
     \textbf{Indicate your}\ \teaminfo

  \begin{itemize}

  \item
   This exam is \textbf{closed book} except for a 2-sided cribsheet.
   Total time is 90 minutes. 

  \item
   Write your solutions in the space provided.  If you need more
   space, write on the back of the sheet containing the problem.

%   Please keep your entire answer to a problem on that problem's page.
   
   \item In answering the following questions, you may use without
     proof any of the results from class or text.
     \iffalse (unless explicitly instructed otherwise).\fi

     \iffalse
   \item
     GOOD LUCK!
     \fi

\end{itemize}}

\begin{document}

\midterm{October 15}

%%%%%%%%%%%%%%%%%%%%%%%%%%%%%%%%%%%%%%%%%%%%%%%%%%%%%%%%%%%%%%%%%%%%%
% Problems start here
%%%%%%%%%%%%%%%%%%%%%%%%%%%%%%%%%%%%%%%%%%%%%%%%%%%%%%%%%%%%%%%%%%%%

\examspace

\begin{center}
{\Large DRAFT}
\end{center}

\begin{center}
{\large Binary Relations, Mapping Lemma}
\end{center}
%1
%\pinput[points = 7, title= \textbf{Binary Relations}]{PS_binary_relations_on_a_set}

\pinput[points = 7, title= \textbf{Token Replacing Game}]{TP_sub_one_with_two_of_opposite_color}

\begin{staffnotes}
CUT PS\_binary\_relations\_on\_a\_set.  uninformative, tedious, and unlike any prior problem.
\end{staffnotes}

%2
\pinput[points = 7, title= \textbf{Binary Relations}]{TP_composition_of_jections}
\begin{staffnotes}
TP\_composition\_of\_jections. Proof takes too much explanation for a
midterm.  Maybe convert to true/false of short/answer ``if $f$ is
[in-out] and $g$ is [in-out], then $f \compose g$ is [in-out]
\end{staffnotes}

%3
%\pinput[points = 7, title= \textbf{Binary Relations}]{TP_Inverse_Relations}

\begin{staffnotes}
CUT TP\_Inverse\_Relations: trivial if jection words replaced by
[in-out], which they should be.
\end{staffnotes}

%4
\pinput[points = 7, title= \textbf{Binary Relations}]{TP_total_inj_not_bij}
\begin{staffnotes}
TP\_total\_inj\_not\_bij.  Rewrite to use [arrows in-out].
\end{staffnotes}

\begin{center}
{\large Induction}
\end{center}
%5
\pinput[points = 7, title= \textbf{Induction}]{TP_a_bogus_fibonacci_induction}
\begin{staffnotes}
TP\_a\_bogus\_fibonacci\_induction
\end{staffnotes}

%6
\pinput[points = 7, title= \textbf{Induction}]{TP_divide_product_induction}
\begin{staffnotes}
TP\_divide\_product\_induction
\end{staffnotes}

%7
\pinput[points = 7, title= \textbf{Induction}]{CP_bogus_induction_nth_power}
\begin{staffnotes}
CP\_bogus\_induction\_nth\_power
\end{staffnotes}

%8
%\pinput[points = 7, title= \textbf{Induction}]{FP_prove_strong_induction}

\begin{staffnotes}
CUT FP\_prove\_strong\_induction.  Unfamiliar and will mind-bend
unsophisticated students.  Should be renamed ``CP\_'' or ``PS\_''
\end{staffnotes}


\begin{center}
{\large State machines}
\end{center}

%9
\pinput[points = 7, title= \textbf{State Machines}]{CP_10_heads_and_100_tails}

%10
\pinput[points = 7, title= \textbf{State Machines}]{MQ_state_machine_invariant_afternoon}

%11
\pinput[points = 7, title= \textbf{State Machines}]{MQ_state_machine_invariant_morning}

%12
\pinput[points = 7, title= \textbf{State Machines}]{CP_98_heads_and_4_tails}

\begin{center}
{\large Stable Marriage}
\end{center}

%13
\pinput[points = 7, title= \textbf{Stable Marriage}]{MQ_stable_matching_unique_morning}

%14
\pinput[points = 7, title= \textbf{Stable Marriage}]{TP_Stable_Marriage_Invariants}

%15
%\pinput[points = 7, title= \textbf{Stable Marriage}]{TP_stable_match_example}

\begin{staffnotes}
CUT TP\_stable\_match\_example: finger exercise to perform mating
ritual.  not good for an exam.
\end{staffnotes}

%16
\pinput[points = 7, title= \textbf{Stable Marriage}]{FP_stable_matching_unequal_invariants}
\begin{staffnotes}
FP\_stable\_matching\_unequal\_invariants good variation of probs on cp4w.
\end{staffnotes}

\begin{center}
{\large Recursive Data}
\end{center}

%17
\pinput[points = 7, title= \textbf{Recursive Data}]{MQ_ambiguous_recursive_def}

%18
\pinput[points = 7, title= \textbf{Recursive Data}]{MQ_ambiguous_recursive-def_morning}

%19
\pinput[points = 7, title= \textbf{Recursive Data}]{FP_structural_induction_rational_composition_S13}

%20
\pinput[points = 7, title= \textbf{Recursive Data}]{FP_rational_structural_induction}

%21
\pinput[points = 7, title= \textbf{Recursive Data}]{FP_structural_ind_polynomials}

\begin{center}
{\large Sets}
\end{center}

%22
\pinput[points = 7, title= \textbf{Sets}]{CP_axiom_of_choice_formula}
\begin{staffnotes}
CUT CP\_axiom\_of\_choice\_formula: Axiom of Choice is long winded
motivation for midterm 1 topic of set formulas.
\end{staffnotes}

%23
%\pinput[points = 7, title= \textbf{Sets}]{PS_size_n_set_formula}

\begin{staffnotes}
CUT PS\_size\_n\_set\_formula -- appeared on conflict midterm 1
\end{staffnotes}

%22
\pinput[points = 7, title= \textbf{Sets}]{TP_cardinality_class}

%\pinput[points = 7, title= \textbf{Sets}]{CP_N_to_N_diagonal_argument}
\begin{staffnotes}
CUT CP\_N\_to\_N\_diagonal\_argument.  On ps4.  Would need variant for
midterm.
\end{staffnotes}

\pinput[points = 7, title= \textbf{Sets}]{FP_countable_quadratics}

%24
\pinput[points = 7, title= \textbf{Sets}]{TP_uncountable_example}

\begin{staffnotes}
TP\_uncountable\_example: good problem for midterm 2
\end{staffnotes}


\begin{center}
{\large Number Theory: GCDs and Congruence}
\end{center}

%25
\pinput[points = 7, title= \textbf{NT}]{CP_GCD_algebra}

%26
\pinput[points = 7, title= \textbf{NT}]{CP_relative_primality_under_remainder}

%27
\pinput[points = 7, title= \textbf{NT}]{FP_GCD_algebra}

%\pinput[points = 7, title= \textbf{NT}]{TP_GCDs_I}
\begin{staffnotes}
CUT TP\_GCDs\_I.  Very lightweight; maybe use for a couple of points.
\end{staffnotes}

%28
%\pinput[points = 7, title= \textbf{NT}]{PS_gcd_termination}
\begin{staffnotes}
CUT PS\_gcd\_termination.  Has no solution.  Too hard for midterm.
\end{staffnotes}

%\pinput[points = 7, title= \textbf{NT}]{TP_GCDs_I}

%29
%\pinput[points = 7, title= \textbf{NT}]{PS_gcd_termination}

%30
\pinput[points = 7, title= \textbf{NT}]{TP_GCDs_II}
\begin{staffnotes}
TP\_GCDs\_II.  Lightweight variant of CP\_gcd\_lcm on cp5w.
\end{staffnotes}

\begin{center}
{\large Euler's theorem, NOT RSA}
\end{center}

%31
\pinput[points = 7, title= \textbf{Euler}]{TP_Eulers_Theorem}

%32
\pinput[points = 7, title= \textbf{Euler}]{TP_Fermats_Little_Theorem_F13}

%33
\pinput[points = 7, title= \textbf{Euler}]{TP_relative_primality_3780}

%34
\pinput[points = 7, title= \textbf{Euler}]{FP_modular_powerful}
\begin{staffnotes}
FP\_modular\_powerful Needs revision: change probability to ``fraction''
\end{staffnotes}

%35
\pinput[points = 7, title= \textbf{Euler}]{FP_bogus_Fermat_theorem.spring12}

%36
\pinput[points = 7, title= \textbf{Euler}]{TP_inverse_by_Fermat}

\begin{staffnotes}
TP\_inverse\_by\_Fermat: would need substantial rewrite for midterm.
We didn't teach this, and it's not as good a way to find inverses as
the Pulverizer.
\end{staffnotes}

%%%%%%%%%%%%%%%%%%%%%%%%%%%%%%%%%%%%%%%%%%%%%%%%%%%%%%%%%%%%%%%%%%%%%
% Problems end here
%%%%%%%%%%%%%%%%%%%%%%%%%%%%%%%%%%%%%%%%%%%%%%%%%%%%%%%%%%%%%%%%%%%%%
\end{document}
