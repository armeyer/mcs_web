\documentclass[handout]{mcs}

\begin{document}

\renewcommand{\reading}
{Chapter~\bref{sec:coloring}{--\bref{spantree_subsec}{, Coloring,
      Connectedness, \& Trees}}; Chapter~\bref{planar_graphs_chap}{,
    Planar Graphs}; Chapter~\bref{chap:asymptotics}{, Sums and
    Asymptotics}.

\noindent\textbf{Skip} the following sections which will not be
covered this term: Chapter~\bref{MST_subsec}{, Minimum Weight Spanning
  Trees}, Chapter~\bref{chap:state-machines}{, State Machines},
Chapter~\bref{doublesum_sec}{, Double Sums}, \&
Chapter~\bref{omega_subsec}{, Omega notation}.}

\problemset{7}


%%%%%%%%%%%%%%%%%%%%%%%%%%%%%%%%%%%%%%%%%%%%%%%%%%%%%%%%%%%%%%%%%%%%%
% Problems start here
%%%%%%%%%%%%%%%%%%%%%%%%%%%%%%%%%%%%%%%%%%%%%%%%%%%%%%%%%%%%%%%%%%%%%

%\pinput{CP_tree_characterizations}
\pinput{PS_shortest_undirected_closed_walk}
%\pinput{CP_n_dim_hypercube}
\pinput{PS_Euler_circuits}
\pinput{FP_bogus_coloring_proof}
\pinput{PS_tight_bounds_with_integral_method}
\pinput{CP_theta_examples}


%%%%%%%%%%%%%%%%%%%%%%%%%%%%%%%%%%%%%%%%%%%%%%%%%%%%%%%%%%%%%%%%%%%%%
% Problems end here
%%%%%%%%%%%%%%%%%%%%%%%%%%%%%%%%%%%%%%%%%%%%%%%%%%%%%%%%%%%%%%%%%%%%%

\end{document}
