\documentclass[12pt,twoside]{article}   
\usepackage{light}

\newcommand{\card}[1]{\left|#1\right|}
\newcommand{\union}{\cup}
\newcommand{\lgunion}{\bigcup}
\newcommand{\intersect}{\cap}
\newcommand{\lgintersect}{\bigcap}
\newcommand{\cross}{\times}
\newcommand{\naturals}{\mathbb{N}}
\newcommand{\mfigure}[3]{\bigskip\centerline{\resizebox{#1}{#2}{\includegraphics{#3}}}\bigskip}
\newcommand{\eqdef}{\mathbin{::=}}
\newcommand{\heads}{H}
\newcommand{\tails}{T}

\hidesolutions
\showsolutions

\newlength{\strutheight}
\newcommand{\prob}[1]{\mathop{\textup{Pr}} \nolimits \left( #1 \right)}
\newcommand{\prsub}[2]{\mathop{\textup{Pr}_{#1}}\nolimits\left(#2\right)}
\newcommand{\prcond}[2]{%
  \ifinner \settoheight{\strutheight}{$#1 #2$}
  \else    \settoheight{\strutheight}{$\displaystyle#1 #2$} \fi%
  \mathop{\textup{Pr}}\nolimits\left(
    #1\,\left|\protect\rule{0cm}{\strutheight}\right.\,#2
  \right)}
\newcommand{\comment}[1]{}
\newcommand{\cE}{\mathcal{E}}
\renewcommand{\setminus}{-}
\renewcommand{\complement}[1]{\overline{#1}}


\begin{document}
\problemset{10}{November 15, 2016}{Monday, November 21, 7:30pm}
\newcommand{\proofrubric}[3][Any correct proof.]
  {
  	\begin{center}
	\fbox{\begin{minipage}{35em}
	\textbf{Rubric}
	\par
	[#3pts] #1
		\begin{center}
		\textbf{or}
		\end{center}
	#2
	\end{minipage}}
	\end{center}
  }
  
  \newcommand{\proofrubriceach}[3][Any correct proof.]
  {
  	\begin{center}
	\fbox{\begin{minipage}{35em}
	\textbf{Rubric For Each}
	\par
	[#3pts] #1
		\begin{center}
		\textbf{or}
		\end{center}
	#2
	\end{minipage}}
	\end{center}
  }
  
  \newcommand{\rubric}[1]
  {
  	\begin{center}
	\fbox{\begin{minipage}{35em}
	\textbf{Rubric}
	\par
	#1
	\end{minipage}}
	\end{center}
  }
\noindent \textbf{Reading Assignment:}   Sections  12.1-12.6, 14.1-14.5


%%%%%%%%%%%%%%%%%%%%%%%%%%%%%%%%%%%%%%%%%%%%%%%%%%%%%%%%%%%%%%%%%%%%%%%%%%%%%
%%% Fall07 ps10 problem 1

\begin{problem}{30}
Generating functions are very useful for turning difficult combinatorial problems into simple algebra.  Solve the following combinatorial problems by using generating functions.
(\textit{Hint}: the product rule will be helpful in solving these problems).
\bparts
\ppart{4}  Consider rolling a pair of normal six-sided dice.  How many ways are there for the dice to sum to 7?
\rubric{
[2pts] correct polynomial \par
[2pts] 6
}
\solution{
We use the polynomial $(x + x^2 + x^3 + x^4 + x^5 + x^6)$ to represent the possible outcomes when rolling one die (the exponents represent the values, and the coefficients represent that there is one way to get this outcome).  By using the product rule, the distribution across the sum of the dice can be modeled as $(x + x^2 + x^3 + x^4 + x^5 + x^6)^2$.  Expanding out this product, we see that the coefficient of $x^7$ will be $6$.  
}

\ppart{6}  Bob has a basket with 4 apples and 5 bananas.  Suppose the apples are all distinguishable, and the bananas are all distinguishable.  How many ways are there for Bob to select 6 pieces of fruit such that he picks an even number of apples and at least two bananas?
\rubric{
[2pts] correct A polynomial \par
[2pts] correct B polynomial \par
[2pts] 40
}
\solution{
We represent the choices of an even number of apples by the following polynomial: 
\begin{equation*}
A(x) = \binom{4}{0} + \binom{4}{2}x^2 + \binom{4}{4}x^4
\end{equation*}
We represent the choices of at least two bananas by the following polynomial:
\begin{equation*}
B(x) = \binom{5}{2}x^2 + \binom{5}{3}x^3 + \binom{5}{4}x^4+ \binom{5}{5}x^5
\end{equation*}
Then we determine the solution to the problem by selecting the coefficient of $x^6$ in the expansion of $A(x) \cdot B(x)$.  This value is $\binom{4}{2} \cdot  \binom{5}{4} + \binom{4}{4} \cdot \binom{5}{2} = 40$.
}

\ppart{4}   Bob has a basket with 4 apples and 5 bananas.  Suppose the apples are now indistinguishable, and the bananas are indistinguishable.  How many ways are there for Bob to select 6 pieces of fruit such that he picks an even number of apples and at least two bananas?
\rubric{
[1pt] correct A polynomial \par
[1pt] correct B polynomial \par
[2pts] 2
}
\solution{
We represent the choices of an even number of apples by the following polynomial: 
\begin{equation*}
A(x) = 1 + x^2 + x^4
\end{equation*}
We represent the choices of at least two bananas by the following polynomial:
\begin{equation*}
B(x) = x^2 + x^3 + x^4+ x^5
\end{equation*}
Then we determine the solution to the problem by selecting the coefficient of $x^6$ in the expansion of $A(x) \cdot B(x)$.  This value is $1 \cdot 1 + 1 \cdot 1 = 2$.
}



\ppart{6}  Find the number of ways to collect 15 dollars from 20 people if each of the first 19 people can give a dollar or nothing, and the twentieth person can give either 1 dollar, 5 dollars, or nothing.
\rubric{
[2pts] correct A polynomial \par
[2pts] correct B polynomial \par
[2pts] 107882 or $ \binom{19}{15} + \binom{19}{14} + \binom{19}{10}$
}
\solution{
We represent the choices of the first 19 people by the following polynomial: 
\begin{equation*}
A(x) = (1 + x)^{19}
\end{equation*}
We represent the choices of the twentieth person by the following polynomial:
\begin{equation*}
B(x) = 1 + x + x^5
\end{equation*}
Then we determine the solution to the problem by selecting the coefficient of $x^15$ in the expansion of $A(x) \cdot B(x)$.  This value is $ \binom{19}{15} + \binom{19}{14} + \binom{19}{10} = 107882$.
}

\ppart{6}  We have three pennies, four nickels, and two quarters. Find the generating function of the number of ways we can make change for $n$ cents.  Assume the coins are indistinguishable.
\rubric{
[2pts] correct A polynomial \par
[2pts] correct B polynomial \par
[2pts] correct C polynomial
}
\solution{
We represent the choices of how many pennies to use by the following polynomial: 
\begin{equation*}
A(x) = 1 + x + x^2 + x^3
\end{equation*}
We represent the choices of how many nickles to use by the following polynomial:
\begin{equation*}
B(x) = 1 + x^5 + x^{10} + x^{15} + x^{20}
\end{equation*}
We represent the choices of how many quarters to use by the following polynomial:
\begin{equation*}
C(x) = 1 + x^{25} + x^{50}
\end{equation*}
Then we determine the solution to the problem by selecting the coefficient of $x^n$ in the expansion of $A(x) \cdot B(x) \cdot C(x)$.
}


\ppart{4} We have three pennies, four nickels, and two quarters. Find the generating function of the number of ways we can make change for $n$ cents.  Assume the coins are distinguishable.
\rubric{
[2pts] correct A polynomial \par
[2pts] correct B polynomial \par
[2pts] correct C polynomial
}
\solution{
We represent the choices of how many pennies to use by the following polynomial: 
\begin{equation*}
A(x) = 1 + \binom{3}{1}x + \binom{3}{2}x^2 + \binom{3}{3}x^3
\end{equation*}
We represent the choices of how many nickles to use by the following polynomial:
\begin{equation*}
B(x) = 1 + \binom{4}{1}x^5 + \binom{4}{2}x^{10} + \binom{4}{3}x^{15} + \binom{4}{4}x^{20}
\end{equation*}
We represent the choices of how many quarters to use by the following polynomial:
\begin{equation*}
C(x) = 1 + \binom{2}{1}x^{25} + \binom{2}{2}x^{50}
\end{equation*}
Then we determine the solution to the problem by selecting the coefficient of $x^n$ in the expansion of $A(x) \cdot B(x) \cdot C(x)$.
}

\eparts

\end{problem}

\begin{problem}{10}
Let $\mathcal{C}$ be the set of sequences formed by $\{a, b, c, d, 1, 2, 3\}$ such that the letters $\{a, b, c, d\}$ appear before the numbers $\{1, 2, 3\}$.  As an example, $abba12$ and $cdab321$ are sequences in  $\mathcal{C}$ but $a3b2c1$ is not a sequence in  $\mathcal{C}$.  Let $c_n$ be the number of sequences in  $\mathcal{C}$ of length $n$.  Let $C(x) = \displaystyle\sum\limits_{n=0}^\infty c_n x^n$. 

\bparts
\ppart{6} Determine an expression for $C(x)$.  
\rubric{
[2pts] correct A polynomial \par
[2pts] correct B polynomial \par
[2pts] $\frac{1}{(1-4x)(1-3x)}$
}
\solution{
We have the following Taylor polynomial for the letters:
\begin{equation*}
A(x) = 1 + 4x + (4x)^2 + (4x)^3 + (4x)^4 + \ldots = \displaystyle\sum\limits_{i=0}^{\infty} (4x)^i = \frac{1}{1-4x}
\end{equation*}
We have the following Taylor polynomial for the numbers:
\begin{equation*}
B(x) = 1 + 3x + (3x)^2 + (3x)^3 + (3x)^4 + \ldots = \displaystyle\sum\limits_{i=0}^{\infty} (3x)^i = \frac{1}{1-3x}
\end{equation*}
Hence by the product rule $C(x) = A(x) \cdot B(x) = \frac{1}{(1-4x)(1-3x)}$

}

\ppart{4} Determine an explicit expression for $c_n$ (\textit{Hint}: use partial fraction decomposition on the generating function you find in part a).
\rubric{
[2pts] partial decomposition done correctly \par
[2pts] $4^{n+1} - 3^{n+1}$
}  
\solution{
By using partial fraction decomposition, we find that 
\begin{equation*}
C(x) = \frac{1}{(1-4x)(1-3x)} = \frac{4}{1-4x} - \frac{3}{1-3x} = \displaystyle\sum\limits_{i=0}^{\infty} 4 \cdot(4x)^i - 3 \cdot (3x)^i
\end{equation*}
Hence the coefficient of $x^n$ in $C(x)$, which is $c_n$, is just $4^{n+1} - 3^{n+1}$.
}
\eparts

\end{problem}

\begin{problem}{10}
Let $a_n = a_{n-1} + 2a_{n-2}$ for $n \in \mathbb{N}$ with $a_0 = 1, a_1 = 1$.  Use generating functions to find an explicit expression for $a_n$.  
\end{problem}
\rubric{
[4pts] Attempt to find a way to manipulate polynomials to get the form $a_n = a_{n-1} + 2a_{n-2}$ \par
[6pts] $a_n = \frac{2^{n+1} + (-1)^n}{3}$
}
\solution{
Let $A(x) = \displaystyle\sum\limits_{i=0}^{\infty} a_nx^n$ be the generating function for the sequence $a_n$.w
We now consider the following equations:
\begin{alignat*}{3}
A(x) &= a_0 + a_1x + a_2x^2 + a_3x^3 + \ldots \\
xA(x) &= \qquad a_0x + a_1x^2 + a_2x^3 + \ldots \\ 
2x^2A(x) &= \qquad \qquad 2a_0x^2 + 2a_1x^3 + \ldots \\
\end{alignat*}

Now after subtracting appropriately, we get $A(x) - xA(x) - 2x^2A(x) = a_0 + (a_1 - a_0)x $ or equivalently,
\begin{equation*}
A(x) = \frac{1}{1 - x - 2x^2} = \frac{1}{(1-2x)(1+x)} = \frac{2}{3} \cdot \frac{1}{1-2x} +\frac{1}{3} \cdot \frac{1}{1+x} = \displaystyle\sum\limits_{i=0}^{\infty} \frac{2}{3}(2x)^i + \frac{1}{3}(-x)^i
\end{equation*}
Hence $a_n$ is the coefficient $n^{th}$ term in $A(x)$, and so $a_n = \frac{2^{n+1} + (-1)^n}{3}$.
}

\begin{problem}{20}
For a given $n$, let $p_n$ be the number of ways of writing $n$ as a sum of 3 positive integers, where the order matters.  For example, we can write 5 as:
\begin{equation*}
1 + 1 + 3 \qquad 1 + 3 + 1 \qquad 3 + 1 + 1 \qquad 1 + 2 + 2 \qquad 2 + 1 + 2 \qquad 2 + 2 + 1
\end{equation*}
\bparts
\ppart{8} What is a formula for the generating function $P(x) =  \displaystyle\sum\limits_{n=0}^\infty p_n x^n$?
\rubric{
[4pts] Notice that the Taylor polynomial for each integer is the same\par
[4pts] $P(x) = A(x)^3 = \frac{x^3}{(1-x)^3}$
}
\solution{
We have the following Taylor polynomial for each of the three positive integers:
\begin{equation*}
A(x) = x + x^2 + x^3 + x^4 + \ldots = \displaystyle\sum\limits_{i=1}^{\infty} x^i = \frac{x}{1-x}
\end{equation*}
By the product rule, the generating function $P(x) = A(x)^3 = \frac{x^3}{(1-x)^3}$

}

Now again, for a given $n$, consider the number of ways of writing $n$ as a sum of 3 positive integers as describe above.  Let $f_n$ be the sum of the product of each of the triples of numbers.  For example, for $n = 5$ again, we get 
\begin{equation*}
f_5 = 1 \cdot 1 \cdot 3 + 1 \cdot 3 \cdot 1+ 3 \cdot 1 \cdot 1 + 1 \cdot 2 \cdot 2 + 2 \cdot 1 \cdot 2 + 2 \cdot 2 \cdot 1
\end{equation*} 
%
Let $f_0 = 0$.  
\ppart{12} What is the formula for the generating function $F(x) =  \displaystyle\sum\limits_{n=0}^\infty f_n x^n$? (Your answer should be a closed formula, not a Taylor Series) 
\rubric{
[4pts] Notice that the Taylor polynomial for each integer is the same\par
[4pts] Encode products in the coefficients of the terms \par
[4pts] $P(x) = A(x)^3 = \frac{x^3}{(1-x)^6}$
}
\solution{
We use a similar Taylor polynomial as in part a, except we use the coefficients to encode the products that we want.
Hence we get the following Taylor polynomial for each of the terms:
\begin{equation*}
A(x) = x + 2x^2 + 3x^3 + 4x^4 + \ldots = \displaystyle\sum\limits_{i=1}^{\infty} ix^i = x \cdot \frac{d}{dx}  \left( \displaystyle\sum\limits_{i=0}^{\infty} x^i \right) = \frac{x}{(1-x)^2}
\end{equation*}
By the product rule, the generating function $F(x) = A(x)^3 = \frac{x^3}{(1-x)^6}$.

}

\eparts
\end{problem}


%%%%%%%%%%%%%%%%%%%%%%%%%%%%%%%%%%%%%%%%%%%%%%%%%%%%%%%%%%%%%%%%%%%%%%%%


\begin{problem}{20}
Recall the strange dice from lecture:
\begin{center}
%\mfigure{!}{3.5in}{lec2-dice}
\mfigure{!}{2in}{dice}
\end{center}
In lecture we proved that if we roll each die once, then die $A$ beats $B$ more often, die $B$ beats die $C$ more often, and die $C$ beats die $A$ more often. Thus, contrary to our intuition, the ``beats'' relation $>$ is not transitive. That is, we have $A > B > C > A$.

We then looked at what happens if we roll each die twice, and add the result. In lecture, we showed that rolling die $B$ twice is more likely to win, i.e., have a larger sum, than rolling die $A$ twice, which is the opposite of what happened if we were to just roll each die once! In fact, we will show that the ``beats'' relation reverses in this game, that is, $A < B < C < A$, which is very counterintuitive!
\rubric{
\textbf{For each part:} \par
[2pts] Work shown \par
[3pts] Calculations that show rolling die twice follows $A < B < C < A$ asked in the part.
}
\bparts
\ppart{5}
Show that rolling die $C$ twice is more likely to win than rolling die $B$ twice.

\solution{We draw the sample space. In the figure, it should be understood that the tree corresponding to $B$ is connected to each leaf of the tree corresponding to $C$. 
\begin{center}
\mfigure{!}{3in}{diceBC}
\end{center}
As in lecture, there are $81$ leaves and the space is uniform, i.e., each outcome occurs with probability $(1/3)^4 = 1/81$. Let's work out the chances of winning. The sum of the two rolls of the $B$ die is equally likely to be any element of the following multiset:
$$S_B = \{2, 6, 6, 10, 10, 10, 14, 14, 18\}.$$
The sum of the two rolls of the $C$ die is equally likely to be any element of the following multiset:
$$S_C = \{6, 7, 7, 8, 11, 11, 12, 12, 16\}.$$
We can treat each outcome as a pair $(x,y) \in S_B \times S_C$, where $C$ wins iff $y > x$. If $y = 6$, there is $1$ value of $x$, namely $x = 2$, for which $y > x$. Continuing the count in this way, the number of pairs for which $y > x$ is $$1 + 3 + 3 + 3 + 6 + 6 + 6 + 6 + 8 = 42,$$
while there are $2$ ties and $37$ cases where $B$ wins. Thus, rolling die $C$ twice is more likely to win than rolling die $B$ twice.
}

\ppart{5} 
Show that rolling die $A$ twice is more likely to win that rolling die $C$ twice.

\solution{We draw the sample space. In the figure, it should be understood that the tree corresponding to $C$ is connected to each leaf of the tree corresponding to $A$. 
\begin{center}
\mfigure{!}{3in}{diceCA}
\end{center}
As in lecture, there are $81$ leaves and the space is uniform, i.e., each outcome occurs with probability $(1/3)^4 = 1/81$. Let's work out the chances of winning. The sum of the two rolls of the $C$ die is equally likely to be any element of the following multiset:
$$S_C = \{6, 7, 7, 8, 11, 11, 12, 12, 16\}.$$
The sum of the two rolls of the $A$ die is equally likely to be any element of the following multiset:
$$S_A = \{4, 8, 8, 9, 9, 12, 13, 13, 14\}.$$
We can treat each outcome as a pair $(x,y) \in S_C \times S_A$, where $A$ wins iff $y > x$. If $y = 4$, there is no $x$ for which $y > x$. If $y = 8$, there are $3$ values of $x$, namely $x = 6, 7, 7$, for which $y > x$. Continuing the count in this way, the number of pairs for which $y > x$ is $$0 + 3 + 3 + 4 + 4 + 6 + 8 + 8 + 8 = 44,$$
while a similar count shows that there are only $33$ pairs for which $x > y$, and there are $4$ ties. Thus, rolling die $A$ twice is more likely to win than rolling die $C$ twice.
}

\ppart{5}
Show that rolling die $B$ twice is more likely to win that rolling die $A$ twice.

\solution{We draw the sample space. In the figure, it should be understood that the tree corresponding to $C$ is connected to each leaf of the tree corresponding to $A$. 
\begin{center}
\mfigure{!}{3in}{diceAB}
\end{center}
As in lecture, there are $81$ leaves and the space is uniform, i.e., each outcome occurs with probability $(1/3)^4 = 1/81$. Let's work out the chances of winning. The sum of the two rolls of the $A$ die is equally likely to be any element of the following multiset:
$$S_A = \{4, 8, 8, 9, 9, 12, 13, 13, 14\}.$$
The sum of the two rolls of the $A$ die is equally likely to be any element of the following multiset:
$$S_B = \{2, 6, 6, 10, 10, 10, 14, 14, 18\}.$$
We can treat each outcome as a pair $(x,y) \in S_A \times S_B$, where $B$ wins iff $y > x$. If $y = 2$, there is no $x$ for which $y > x$. If $y = 6$, there is $1$ value of $x$, namely $x = 4$, for which $y > x$. Continuing the count in this way, the number of pairs for which $y > x$ is $$0 + 1 + 1 + 5 + 5 + 5 + 8 + 8 + 9 = 42,$$
while a similar count shows that there are only $37$ pairs for which $x > y$, and there are $4$ ties. Thus, rolling die $A$ twice is more likely to win than rolling die $C$ twice.
}

\eparts 

\end{problem}

%%%%%%%%%%%%%%%%%%%%%%%%%
%% fall 08, pset11

\instatements{\vspace{0.5in}}
\begin{problem}{10}
We're covering probability in 6.042 lecture one day, and you volunteer for one of Professor Leighton's demonstrations. He shows you a coin and says he'll bet you \$1 that the coin will come up heads. Now, you've been to lecture before and therefore suspect the coin is biased, such that the probability of a flip coming up heads, $\pr{H}$, is $p$ for $1/2 < p \leq 1$.

You call him out on this, and Professor Leighton offers you a deal. He'll allow you to come up with an algorithm using the biased coin to \textit{simulate} a fair coin, such that the probability you win and he loses, $\pr{W}$, is equal to the probability that he wins and you lose, $\pr{L}$. You come up with the following algorithm:

\begin{enumerate}
\item Flip the coin twice.
\item Based on the results:
	\begin{itemize}
	\item $TH \implies$ you win [$W$], and the game terminates.
	\item $HT \implies$ Professor Leighton wins [$L$], and the game terminates.
	\item $(HH \lor TT) \implies$ discard the result and flip again.
	\end{itemize}
\item If at the end of $N$ rounds nobody has won, declare a tie.
\end{enumerate}
As an example, for $N=3$, an outcome of $HT$ would mean the game ends early and you lose, $HHTH$ would mean the game ends early and you win, and $HHTTTT$ would mean you play the full $N$ rounds and result in a tie.

\bparts

\ppart{5}
Assume the flips are mutually independent. Show that $\pr{W} = \pr{L}$.
\proofrubric{
[2pts] Correct formula for $\pr{W}$ \par
[3pts] Substitute $\pr{TH} = \pr{HT}$
}{5}
\solution{
The probability of you winning is equal to the probability that you win in the first round, plus the probability that nobody won in the first round times the probability that you win in the second round, plus the probability that nobody won in the first two round times the probability that you win in the third round, etc. The same goes for Professor Leighton. Hence:
\begin{align*}
\pr{W} &= \pr{TH} + \pr{HH \lor TT}\pr{TH} + \pr{HH \lor TT}^2\pr{TH} + \ldots \\
& = \pr{TH} \cdot \sum_{i=0}^N \pr{HH \lor TT}^i \\
& = \pr{HT} \cdot \sum_{i=0}^N \pr{HH \lor TT}^i\\
& = \pr{L}
\end{align*}
The middle step is possible because $\pr{TH} = (1-p)p = p(1-p) = \pr{HT}$.
}

\ppart{5}
Show that, if $p<1$, the probability of a tie goes to 0 as $N$ goes to infinity.
\rubric{
[3pts] Correct probability formula for tie \par
[2pts] Show formula = 0 for $0 < p < 1$
}
\solution{
The probability of a tie is just the probability that nobody won all $N$ rounds, namely:
\[
\pr{tie} = (\pr{HH \lor TT})^N = (\pr{HH} + \pr{TT})^N = (p^2 + (1-p)^2)^N
\]
So the limit as $N$ goes to infinity is 0, given that $p$ and therefore $p^2 + (1-p)^2$ are $< 1$.
}

\eparts

\end{problem}

%%%%%%%%%%%%%%%%%%%%%%%%%%%%%%%%%%%%%%%%%%%%%%%%%%%
\end{document}
