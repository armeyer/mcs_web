% TikZ chains with labeled edges
% Author: Stefan Kottwitz , http://texblog.net
\documentclass[a4paper,10pt]{article}
\usepackage{tikz}
%%%<
\usepackage{verbatim}
\usepackage[active,tightpage]{preview}




\usepackage{amsmath, amssymb, amsthm, amsfonts}
\usepackage{hyperref,doi}
\usepackage{epsf}
\usepackage{fullpage}
\usepackage{enumerate}
\usepackage{graphicx}
\usepackage{array}
\setlength{\extrarowheight}{.12cm}

\newcommand{\ZZ}{\ensuremath{\mathbb{Z}}}

\DeclareMathOperator{\init}{in}






\PreviewEnvironment{tikzpicture}
\setlength\PreviewBorder{5pt}%
%%%>
\begin{comment}
:Title: Chains with labeled edges
:Tags: Matrices, Chains,Styles
:Author: Stefan Kottwitz
:Slug: labeled-chain
http://texblog.net/latex-archive/maths/pgf-tikz-commutative-diagram/

The chains library is very useful for writing exact sequences. Yet there's no feature
for labeling the edges of a chain. Besides arrows, we might need to write symbols for
maps over, under, or just next to it.

This example shows a way by modifying the join method of the chains library.
Its original syntax is join=with<node> by <options>, here the syntax is changed
to join={node[options] {label}}.

Furthermore, we define a style for the labels, so that all are in math mode
and typeset in scriptstyle.
\end{comment}
\usetikzlibrary{arrows,chains,matrix,positioning,scopes}
%
\makeatletter
\tikzset{join/.code=\tikzset{after node path={%
\ifx\tikzchainprevious\pgfutil@empty\else(\tikzchainprevious)%
edge[every join]#1(\tikzchaincurrent)\fi}}}
\makeatother
%
\tikzset{>=stealth',every on chain/.append style={join},
         every join/.style={->}}
\tikzstyle{labeled}=[execute at begin node=$\scriptstyle,
   execute at end node=$]
%
\begin{document}
%\tikzstyle{every node}=[circle, draw, inner sep=5pt, minimum width=4pt]
\begin{tikzpicture}[thick,scale=0.8]
  
  \node [circle, draw, fill=red!100, inner sep=2pt, minimum width=2pt] (1) at 	(0,0) {};
  \node [circle, draw, inner sep=2pt, minimum width=2pt] (2) at 	(0,1) {};
  \node [circle, draw, inner sep=2pt, minimum width=2pt] (3) at 	(0,2) {};
  \node [circle, draw, inner sep=2pt, minimum width=2pt] (4) at 	(0,3) {};
  \node [circle, draw, inner sep=2pt, minimum width=2pt] (5) at 	(1,0) {};
  \node [circle, draw, inner sep=2pt, minimum width=2pt] (6) at 	(1,1) {};
  \node [circle, draw, inner sep=2pt, minimum width=2pt] (7) at 	(1,2) {};
  \node [circle, draw, inner sep=2pt, minimum width=2pt] (8) at 	(1,3) {};
  \node [circle, draw, inner sep=2pt, minimum width=2pt] (9) at 	(2,0) {};
  \node [circle, draw, inner sep=2pt, minimum width=2pt] (10) at 	(2,1) {};
  \node [circle, draw, inner sep=2pt, minimum width=2pt] (11) at 	(2,2) {};
  \node [circle, draw, inner sep=2pt, minimum width=2pt] (12) at 	(2,3) {};
  \node [circle, draw, inner sep=2pt, minimum width=2pt] (13) at 	(3,0) {};
  \node [circle, draw, inner sep=2pt, minimum width=2pt] (14) at 	(3,1) {};
  \node [circle, draw, inner sep=2pt, minimum width=2pt] (15) at 	(3,2) {};
  \node [circle, draw, inner sep=2pt, minimum width=2pt] (16) at 	(3,3) {};

  
  \draw   (1) -- (2) ;
  \draw   (2) -- (3) ;
  \draw   (3) -- (4) ;
  \draw   (1) -- (5) ;
  \draw   (2) -- (6) ;
  \draw   (3) -- (7) ;
  \draw   (4) -- (8) ;
  \draw   (5) -- (6) ;
  \draw   (6) -- (7) ;  
  \draw   (7) -- (8) ;
  \draw   (5) -- (9) ;
  \draw   (6) -- (10) ;
  \draw   (7) -- (11) ;
  \draw   (8) -- (12) ;
  \draw   (9) -- (10) ;
  \draw   (10) -- (11) ;
  \draw   (11) -- (12) ;  
  \draw   (9) -- (13) ;
  \draw   (10) -- (14) ;
  \draw   (11) -- (15) ;  
  \draw   (12) -- (16) ;
  \draw   (13) -- (14) ;
  \draw   (14) -- (15) ;
  \draw   (15) -- (16) ;  
  \draw[bend left]  (2) edge (3);
  \draw[bend right]  (5) edge (9);
  \draw[bend left]  (8) edge (12);
  \draw[bend right]  (14) edge (15);

  
 
%== LABELS ====
\end{tikzpicture}

\end{document}