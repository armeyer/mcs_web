\documentclass[twoside,12pt]{article}
\newcommand{\tab}{\hspace*{2em}}
\usepackage{light}
\usepackage{subfigure}
\usepackage{graphicx}
\usepackage{amsmath}
\usepackage{verbatim}

\usepackage{amsfonts}

\newcommand{\lr}{l_{right}}
\renewcommand{\ll}{l_{left}}

\newcommand{\card}[1]{\left|#1\right|}

\newcommand{\hint}[1]{({\it Hint: #1})}
%\newcommand{\card}[1]{\left|#1\right|}
\newcommand{\union}{\cup}
\newcommand{\lgunion}{\bigcup}
\newcommand{\intersect}{\cap}
\newcommand{\lgintersect}{\bigcap}
\newcommand{\cross}{\times}


\hidesolutions
\showsolutions

\newlength{\strutheight}
\newcommand{\prob}[1]{\mathop{\textup{Pr}} \nolimits \left( #1 \right)}
\newcommand{\prsub}[2]{\mathop{\textup{Pr}_{#1}}\nolimits\left(#2\right)}
\newcommand{\prcond}[2]{%
  \ifinner \settoheight{\strutheight}{$#1 #2$}
  \else    \settoheight{\strutheight}{$\displaystyle#1 #2$} \fi %
  \mathop{\textup{Pr}}\nolimits\left(
    #1\,\left|\protect\rule{0cm}{\strutheight}\right.\,#2
  \right)}
%\newcommand{\comment}[1]{}
\newcommand{\cE}{\mathcal{E}}
\renewcommand{\setminus}{-}
\renewcommand{\complement}[1]{\overline{#1}}

\providecommand{\abs}[1]{\lvert#1\rvert}
%\newcommand{\variance}[1]{(\Delta{#1})^2}

\begin{document}
\problemset{1}{September 8, 2016}{Monday, September 12}

%%%%%%%%%%%%%%%%%%%%%%%%%%%%%%%%%%%%%%%%%%%%%%%%%%%%%%%%%%%%%%%%%%%%%%%%%%%%%%%
\begin{problem}{20}
\bparts
\ppart{12}
Use a truth table to verify that $(P \oplus (P \oplus Q))$ is
equivalent to $Q$.
\solution{
\[                                                                                                                                                                                
\begin{array}{c|c|ccc}                                                                                                                                                   
P         &\oplus        & \text{(}P  & \oplus       & Q\text{)}\\ \hline
\true     &\true       & \true      & \false     & \true    \\
\true     &\false      & \true      & \true      & \false   \\                                                                                                         
\false    &\true       & \false     & \true      & \true    \\
\false    &\false      & \false     & \false     & \false   \\
\end{array}
\]

Compare the $(P \textsc{ XOR } (P \textsc{ XOR } Q))$ column with the $Q$ column to see
that they are equivalent.}
\ppart{8}
Find a predicate $P(x)$ that is a counterexample to the following:
\[
\exists x . P(x) \footnotesize{\textsc{ IMPLIES }} \forall x . P(x)
\]

\solution{
Define $P(x) ::= ``x=1"$ and let the domain of discourse be the
two element set $\set{1,2}$.  Then the left hand side of the
implication is true since there is an $x$, namely $x=1$, for which
$P(x)$ is true.  The right hand side is false because it is not true
that every element in $\set{1,2}$ is equal to 1.  Therefore the
implication does not hold and the statement is not a validity.
}
\eparts
\end{problem}


%%%%%%%%%%%%%%%%%%%%%%%%%%%%%%%%%%%%%%%%%%%%%%%%%%%%%%%%%%%%%%%%%%%%%%%%%%%


\begin{problem}{20} A student is trying to prove that propositions $p$, $q$, and $r$ are
all true.  She proceeds as follows.  First, she proves three facts: $p
\rightarrow q$, $q \rightarrow r$, and $r
\rightarrow p$.  Then she concludes, ``thus obviously $p$, $q$, and
$r$ are all true.''  Let's first formalize her deduction as a logical
statement and then evaluate whether or not it is correct.


\bparts
\ppart{6} Using logic notation and the symbols $p$, $q$, and $r$,
write down the logical implication that she uses in her final step.

\solution{
\[
((p \rightarrow q) \wedge (q \rightarrow r) \wedge (r \rightarrow p))
\rightarrow (p \wedge q \wedge r)
\]
}

\ppart{8} Use a truth table to determine whether this logical
implication is a tautology (i.e., a universal truth in logic).
\solution{

\[
\begin{array}{ccc|c|c|c}
p & q & r &
((p \rightarrow q) \wedge (q \rightarrow r) \wedge (r \rightarrow p)) &
(p \wedge q \wedge r) &
\begin{array}{c}
\mbox{complete} \\
\mbox{expression}
\end{array} \\ \hline
T & T & T & T & T & T \\
T & T & F & F & F & T \\
T & F & T & F & F & T \\
T & F & F & F & F & T \\
F & T & T & F & F & T \\
F & T & F & F & F & T \\
F & F & T & F & F & T \\
F & F & F & T & F & \rightarrow F \leftarrow
\end{array}
\]

The truth table indicates that the implication she uses is {\em
not} a tautology.

}
\ppart{6} Is her proof that propositions $p$, $q$, and $r$ are all
true correct?  Briefly explain.

\solution{Her proof is incorrect; she makes use of a
false proposition.  (However, the first part of her argument is
sufficient to show that either all three statements are true {\em or}
all three statements are false.)
}

\eparts
\end{problem}


%%%%%%%%%%%%%%%%%%%%%%%%%%%%%%%%%%%%%%


\begin{problem}{24}
Translate the following statements into predicate logic.  For each,
specify the domain.  In addition to logic symbols, you
may build predicates using arithmetic, relational symbols, and
constants.  For example, the statement ``$n$ is an odd number'' could
be translated into $\exists m. (2m+1 = n)$, where the domain is $\mathbb{Z}$, 
the set of integers.  Another example, ``$p$ is a prime number,'' could be 
translated to 
\[
%(p > 1) \QAND\ \QNOT \paren{\exists m. \exists n. (m > 1 \QAND\ n > 1 \QAND\ mn = p)}
(p > 1) \footnotesize{\textsc{  AND } \textsc{NOT}} \paren{\exists m. \exists n. (m > 1 \footnotesize{\textsc{   AND   }} n > 1 \footnotesize{\textsc{   AND   }} mn = p)}
\]
Let $\text{prime}(p)$ be an abbreviation that you could use to denote the above formula 
in this problem.

\bparts

\ppart{4} (Lagrange's Four-Square Theorem) Every nonnegative integer is
expressible as the sum of four perfect squares.

\solution{The domain is $\mathbb{N}$, the natural numbers.

\[
\forall n. \exists w \exists x \exists y \exists z. (n = w^2 + x^2 + y^2 + z^2)
\]

}

\ppart{4} (Goldbach Conjecture) Every even integer greater than two is
the sum of two primes.

\solution{
The domain is $\mathbb{N}$, the natural numbers.  The statement could be
translated to
\[
\forall n\,
%\paren{((n > 2) \QAND\ \exists m (n = 2m)) \QIMP\
%      \exists p \exists q (\text{prime}(p) \QAND\ \text{prime}(q) \QAND\ (n = p + q))}
\paren{((n > 2) \footnotesize{\textsc{   AND   }} \exists m (n = 2m)) \footnotesize{\textsc{   IMPLIES   }}
      \exists p \exists q (\textsc{Prime}(p) \footnotesize{\textsc{   AND   }} \textsc{Prime}(q) \footnotesize{\textsc{   AND   }} (n = p + q))}
\]
}

\ppart{4} Every finite integer set has a maximum element.
\solution{
The statement could be translated to
\[
\forall S \subset \mathbb{Z} ~(\exists n \in \mathbb{N} ~~ (|S|=n)) \footnotesize{\textsc{  IMPLIES  }}  \exists x \in S. \text{ }\forall y \in S ~~ (y \leq x)
\]
}

%\ppart{4} The function $f : \mathbb{R} \mapsto \mathbb{R}
%$ is continuous.

%\solution{
%The domain is the real numbers, $\mathbb{R}$.
%\[
%\forall a \forall x. \exists b. \forall y.
%\paren{(a > 0 \QAND\ b > 0 \QAND\ \abs{x-y} < b) \QIMP\ \abs{f(x) - f(y)} < a}
%\paren{(a > 0 \text{   AND   } b > 0 \text{   AND   } \abs{x-y} < b) \text{   IMPLIES   } \abs{f(x) - f(y)} < a}
%\]
%}

\ppart{4}
(Fermat's Last Theorem) There are no nontrivial solutions
to the equation:
\[
x^n + y^n = z^n
\]
over the nonnegative integers when $n > 2$.

\solution{
The domain is $\mathbb{N}$.
\[
\forall x \forall y \forall z \forall n.
%\paren{(x > 0 \QAND\ y > 0 \QAND\ z > 0 \QAND\ n > 2)
%    \QIMP\ \QNOT (x^n + y^n = z^n)}
\paren{(x > 0 \footnotesize{\textsc{   AND   }} y > 0 \footnotesize{\textsc{   AND   }} z > 0 \footnotesize{\textsc{   AND   }} n > 2)
\footnotesize{\textsc{  IMPLIES } \textsc{NOT }} (x^n + y^n = z^n)}
\]
}

\ppart{4}
There are infinitely many primes.

\solution{
The domain is $\mathbb{Z}$.

\[
%\QNOT \paren{\exists p (\text{Prime}(p) \QAND\ (\forall q (\text{Prime}(q) \QIMP\ p \geq q)))}
\footnotesize{\textsc{NOT}} \paren{\exists p (\textsc{Prime}(p) \footnotesize{\textsc{   AND   }}(\forall q (\textsc{Prime}(q) \footnotesize{\textsc{   IMPLIES   }} p \geq q)))}
\]
}

%\ppart{4}
%(Bertrand's Postulate) If $n > 1$, then there is always
%at least one prime $p$ such that $n < p < 2n$.

%\solution{
%The domain is $\mathbb{Z}$.
%\[
%\forall n.
%\paren{ (n > 1) \QIMP\ (\exists p ( \text{Prime}(p)  \QAND\ (n < p) \QAND\ (p < 2n))) }
%\paren{ (n > 1) \text{   IMPLIES   } (\exists p ( \text{Prime}(p)  \text{   AND   } (n < p) \text{   AND   } (p < 2n))) }
%\]

%}

\ppart{4} If integers $a$ and $b$ are coprime, then there exist integers $x$ and $y$ such that $ax + by = 1$.

\solution{
The domain is $\mathbb{Z}$.

\[
\forall a, b ~~ (\forall d ~~ ((d | a, d | b) \footnotesize{\textsc{  IMPLIES  }} d=1)) \footnotesize{\textsc{  IMPLIES  }} \exists x,y ~~ (ax + by = 1)
\]
}


\eparts
\end{problem}

%%%%%%%%%%%%%%%%%%%%%%%%%%%%%%%%%%%%%%%%%%%%%%%%%%%%%%%%%%%%%%%%%%%%%%%%%%%%%%% 

\begin{problem}{16}
Translate the following sentences from English to predicate logic. Let $S$ denote the set of all students and let $T$ denote the set of all TAs.  You may use the functions $R(x, y)$, meaning that ``x is in y's recitation," $P(x)$, meaning that ``x will pass 6.042," $H(x)$, meaning that ``x does their homework regularly," and $U(x)$, meaning that ``x is an undergraduate."
\bparts
\ppart{4} All the students in at least one TA's recitation will pass 6.042.
\solution{
\[
	\exists t \in T. \text{ } \forall s \in S.\text{ }R(s, t) \rightarrow P(s)
\]
}
\ppart{4} All the students in 6.042 who do their homework will pass 6.042.
\solution{
\[
	\forall s \in S.\text{ }H(s) \rightarrow P(s)
\]
}
\ppart{4} Every TA will have a student in their recitation pass 6.042.
\solution{
\[
	\forall t \in T. \text{ } \exists s \in S.\text{ }R(s, t) \rightarrow P(s)
\]
}
\ppart{4} There are at least three undergraduate TAs.
\solution{
\[
	\exists a, b, c \in T. a \neq b \land b \neq c \land a \neq c \land  U(a)   \land  U(b)  \land  U(c)
\]
}
\eparts
\end{problem}

%%%%%%%%%%%%%%%%%%%%%%%%%%%%%%%%%%%%%%%%%%%%%%%%%%%%%%%%%%%%%%%%%%%%%%%%%%%%%%% 


\begin{problem}{20} Suppose that $w^2 + x^2 + y^2 = z^2$, where $w,x,y,$ and $z$
always denote positive integers. (\textit{Hint:} It may be helpful to represent even
integers as $2i$ and odd integers as $2j+1$, where $i$ and $j$ are integers.)

Prove the proposition: $z$ is even if and only if $w, x,$ and $y$ are even. Do
this by considering all the cases of $w,x,y$ being odd or even.
\end{problem}

\solution{
As the problem suggests, we will build a truth table to figure out what $z$ is
in each case of $x,y,z$ being even.

\begin{itemize}

    \item $w,x,y$ are all even. In this case, we can write $w,x,y$ as $2i,2j,2k$,
        and the sum of squares is $4i^2 + 4j^2 + 4k^2 = 4(i^2 + j^2 + k^2)$. In
        this case, $z$ must be any even integer when $i^2 + j^2 + k^2 = l^2$ for
        some $l$.
    \item $w,x,y$ are all odd. In this case, each of their squares are odd, and the
        sum of three odd numbers is odd. However, if $z$ is even, then $z^2$ is
        also even. So it cannot be that $z^2 = w^2 + x^2 + y^2$
    \item One of $w,x,y$ is odd. This case is the same as above. Assume $x$ is odd.
        The $x^2$ is odd, but $w^2$ and $y^2$ are even. So the sum is once again
        odd, which cannot equal the square of an even integer. So $z$ cannot be even
        in that case either. The same argument can be repeated for when $w$ and $y$
        are odd as well.
    \item Two of $w,x,y$ are odd. Assume that $w,x$ are odd and $y$ is even. Then
        we can write the sum as $w^2 + x^2 + y^2 = (2i + 1)^2 + (2j + 1)^2 + (2k)^2$.
        This can be rewritten as $4i^2 + 4i + 1 + 4j^2 + 4j + 1 + 4k^2 = 4(i^2 + i + j^2 + j) + 2$.
        However if $z$ is even, then $z^2$ is a multiple of $4$. So $z$ cannot be an
        even integer in this case. The same argument can be used when $w,y$ are odd
        or when $x,z$ are odd.
\end{itemize}

\[
\begin{array}{ccc|c}
\text{w even} & \text{x even} & \text{y even} & \text{z even} \\ \hline
T & T & T & T \\
T & T & F & F \\
T & F & T & F \\
T & F & F & F \\
F & T & T & F \\
F & T & F & F \\
F & F & T & F \\
F & F & F & F
\end{array}
\]
}


\end{document}
