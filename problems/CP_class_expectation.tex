\documentclass[problem]{mcs}

\begin{pcomments}
  \pcomment{CP_class_expectation}
  \pcomment{subsumes FP_class_expectation}
  \pcomment{from: S01.practice final, prob3; modified from S07.ps13, prob3}
  \pcomment{adapted by Steven F09}
  \pcomment{edited ARM 5/17/15, 5/9/16}
\end{pcomments}

\pkeywords{
  expectation
  random
  permutation
  indicator
}

%%%%%%%%%%%%%%%%%%%%%%%%%%%%%%%%%%%%%%%%%%%%%%%%%%%%%%%%%%%%%%%%%%%%%
% Problem starts here
%%%%%%%%%%%%%%%%%%%%%%%%%%%%%%%%%%%%%%%%%%%%%%%%%%%%%%%%%%%%%%%%%%%%%
\begin{problem}
There are $n$ students who are both taking Math for Computer Science
(MCS) and Introduction to Signal Processing (SP) this term.  To make
it easier on themselves, the Professors in charge of these classes
have decided to randomly permute their class lists and then assign
students grades based on their rank in the permutation (just as many
students have suspected).  Assume the permutations are equally likely
and independent of each other.  What is the expected number of
students that have in rank in SP that is higher by $k$ than their rank
in MCS?  \iffalse Explain your reasoning.\fi

\hint Let $X_r$ be the indicator variable for the $r$th ranked student
in CS having a rank in SP of at least $r+k$.

\begin{solution}
The number $X$ of students with $k$-higher SP rank is
\[
X = X_1+X_2+\cdots+X_n.
\]

If a student ranks $r$th in MCS, then there are $r - k$ ranks in SP
that are higher by $k$ or more, where $r > k$.  Since the $r$th
ranked CS is equally likely to have each rank in SP,
\[
\expect{X_r} = \pr{X_r = 1} = \begin{cases} \frac{r-k}{n} &\text{if}\ n \geq r > k,\\
                                            0            & \text{otherwise},
                               \end{cases}
\]
so
\begin{align*}
\expect{X}
  & =\expect{X_1}+\expect{X_2}+\cdots+\expect{X_n}\\
  & =\expect{X_{k+1}}+\expect{X_{k+2}}+\cdots+\expect{X_{n}}\\
  & = \frac{1}{n} + \frac{2}{n} +\cdots+ \frac{n-k}{n}\\
  & = \frac{(n-k)(n-k+1)}{2n}.
\end{align*}

\end{solution}

\end{problem}

%%%%%%%%%%%%%%%%%%%%%%%%%%%%%%%%%%%%%%%%%%%%%%%%%%%%%%%%%%%%%%%%%%%%%
% Problem ends here
%%%%%%%%%%%%%%%%%%%%%%%%%%%%%%%%%%%%%%%%%%%%%%%%%%%%%%%%%%%%%%%%%%%%%

\endinput
