\documentclass[problem]{mcs}

\begin{pcomments}
  \pcomment{MQ_fibonacci_by_induction}
  \pcomment{taken from book/recursive_data.tex staff notes.}
\end{pcomments}

\pkeywords{
  induction
  fibonacci
  recurrence
}

%%%%%%%%%%%%%%%%%%%%%%%%%%%%%%%%%%%%%%%%%%%%%%%%%%%%%%%%%%%%%%%%%%%%%
% Problem starts here
%%%%%%%%%%%%%%%%%%%%%%%%%%%%%%%%%%%%%%%%%%%%%%%%%%%%%%%%%%%%%%%%%%%%%

\begin{problem} 

Fibonacci numbers provide lots of fun for mathematicians because
they satisfy many interesting identities.  Let's prove the following.
\begin{proposition}
  $\forall n \geq 0 (\Sigma_{i=0}^n F_i^2 = F_n F_{n+1})$.
	\end{proposition}

	Example: $n = 4$:
	\[
	0^2 + 1^2 + 1^2 + 2^2 + 3^2 = 15 = 3 \cdot 5.
	\]
	Let's try a proof by induction.  

	\bparts

	\ppart Please give a sensible predicate $P(n)$ for the induction.

	\begin{solution}
	\[
	\sum_{i=0}^n F_i^2 = F_n F_{n+1}.
	\]
	\end{solution}

	\ppart Now prove the base case $P(0)$.

	\begin{solution}

	$\Sigma_{i=0}^0 F_i^2 \eqdef (F_0)^2 = 0 = F_0 F_1$ because
	$F_0 \eqdef 0$.

	\end{solution}

  \ppart Now state $P(n+1)$, and then prove it assuming $P(n)$, thereby proving the induction step.

	\begin{solution}

	$P(n+1) = (\Sigma_{i=0}^{n+1}F_i^2 = F_{n+1}F_{n+2}$) \\
	\begin{proof}
	$\Sigma_{i=0}^{n+1}F_i^2 = \Sigma_{i=0}^nF_i^2 + F_{n+1}^2$ \\
	$ = F_nF_{n+1} + F_{n+1}^2$ by $P(n)$ \\
	$ = F_{n+1}(F_n+F_{n+1})$ \\
	$= F_{n+1}F_{n+2}$ by def. of Fibonacci seq.
	\end{proof}

	\end{solution}

\end{problem}

%%%%%%%%%%%%%%%%%%%%%%%%%%%%%%%%%%%%%%%%%%%%%%%%%%%%%%%%%%%%%%%%%%%%%
% Problem ends here
%%%%%%%%%%%%%%%%%%%%%%%%%%%%%%%%%%%%%%%%%%%%%%%%%%%%%%%%%%%%%%%%%%%%%

\endinput
