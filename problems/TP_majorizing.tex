\documentclass[problem]{mcs}

\begin{pcomments}
  \pcomment{TP_majorizing}
  \pcomment{ARM 10.8.17}
\end{pcomments}

\pkeywords{
  diagonal_argument
  diagonal
  countable
  function
}

%%%%%%%%%%%%%%%%%%%%%%%%%%%%%%%%%%%%%%%%%%%%%%%%%%%%%%%%%%%%%%%%%%%%%
% Problem starts here
%%%%%%%%%%%%%%%%%%%%%%%%%%%%%%%%%%%%%%%%%%%%%%%%%%%%%%%%%%%%%%%%%%%%%

\begin{problem}
Let
\[
f_0, f_1, f_2, \dots, f_k,\dots
\]
be an infinite sequence of total functions $f_k:\nngint \to \reals$.

\bparts

\ppart\label{majorize} Explain how to define a function $g:\nngint \to reals$ that
eventually gets bigger that every one of the $f_k$.

\begin{solution}
Define
\[
g(n) \eqdef \max\set{f_k(j) \suchthat 0 \leq j, k \leq n}.
\]
So for every $k$,
\[
g(n) \geq f_k(n)
\]
for all $n \geq k$.
\end{solution}

\ppart Modify your answer to part~\eqref{majorize}, if necessary, to ensure that
for every $k \in \nngint$,
\[
\lim_{n \to \infty}\frac{f_k(n)}{g(n)} = 0.
\]
Explain why.

\begin{solution}
\[
h(n) \eqdef n (1+ g(n)).
\]

So for all $n \geq k \geq 0$,
\[
\frac{f_k(n)}{h(n)} = \frac{f_k(n)}{n(1+ g(n))} = \frac{1}{n} \cdot
\frac{f_k(n)}{1+g(n)} < \frac{1}{n} \cdot 1.
\]
So for eacj $k \in \nngint$,
\[
\lim_{n \to \infty} \frac{f_k(n)}{h(n)} < \frac{1}{n} = 0.
\]
\end{solution}

\eparts

\end{problem}
\endinput
