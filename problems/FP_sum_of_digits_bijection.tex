\documentclass[problem]{mcs}

\begin{pcomments}
  \pcomment{FP_sum_of_digits_bijection}
  \pcomment{ZDz 12/08/15}
    \pcomment{f15.final}
\end{pcomments}

\pkeywords{
  bijection
  sum_of_digits
}

%%%%%%%%%%%%%%%%%%%%%%%%%%%%%%%%%%%%%%%%%%%%%%%%%%%%%%%%%%%%%%%%%%%%%
% Problem starts here
%%%%%%%%%%%%%%%%%%%%%%%%%%%%%%%%%%%%%%%%%%%%%%%%%%%%%%%%%%%%%%%%%%%%%

\begin{problem}
Consider $n$-digit numbers, including those that start with one or more 0's.
Let $S$ be the set of such numbers whose sum of digits is smaller than
$\frac{9n}{2}$, and $L$ the set of $n$-digit numbers
whose sum of digits is greater than $\frac{9n}{2}$.
Show that $\card{S} = \card{L}$ by defining a bijection between the
two sets. Prove that the function you defined is indeed a bijection.

\examspace[3in]

\begin{solution}
Let $s = \bar{d_1 d_2 \ldots d_n}$ be a $n$-digit number such that $s
\in S$.  Then, $\sum_{i=1}^n d_i < \frac{9n}{2}$.  Define a function
$f : S \rightarrow L$ such that $f(s) = \bar{d'_1 d'_2 \ldots d'_n}$
where $d'_i = 9 - d_i$ for all $1 \leq i \leq n$.  Then, $\sum_{i=1}^n
d'_i = \sum_{i=1}^n (9-d_i) = 9n - \sum_{i=1}^n d_i > 9n -
\frac{9n}{2} = \frac{9n}{2}$.  Therefore, $f(s) \in L$. Since this
holds for any $s \in S$, $f$ is a total function ($[=1 \text{arrow
    out}]$).

On the other hand, if $l \in L$, then it can be similarly shown that
$s = f(l) \in S$, as well as that $f(s) = l$.  So $f$ is surjective
($[\geq 1 \text{arrows in}]$).  In addition, if numbers $s_1$ and
$s_2$ differ at some digit, then $f(s_1)$ and $f(s_2)$ also differ at
that digit.  So, no two different numbers $s_1, s_2 \in S$ can map to
a same number $l \in L$, and therefore $f$ is also injective ($[\leq 1
  \text{arrows in}]$). Note that $f$ is inverse to itself.

Finally, we conclude that $f$ must be bijective since both $[=1
  \text{arrow out}]$ and $[=1 \text{arrow in}]$ hold.
\end{solution}

\end{problem}


%%%%%%%%%%%%%%%%%%%%%%%%%%%%%%%%%%%%%%%%%%%%%%%%%%%%%%%%%%%%%%%%%%%%%
% Problem ends here
%%%%%%%%%%%%%%%%%%%%%%%%%%%%%%%%%%%%%%%%%%%%%%%%%%%%%%%%%%%%%%%%%%%%%
\endinput
