\documentclass[problem]{mcs}

\begin{pcomments}
  \pcomment{CP_bogus_reflexive_proof}
  \pcomment{from: S02.cp3w}
\end{pcomments}

\pkeywords{
  bogus_proof
  reflexive
  relation
}

%%%%%%%%%%%%%%%%%%%%%%%%%%%%%%%%%%%%%%%%%%%%%%%%%%%%%%%%%%%%%%%%%%%%%
% Problem starts here
%%%%%%%%%%%%%%%%%%%%%%%%%%%%%%%%%%%%%%%%%%%%%%%%%%%%%%%%%%%%%%%%%%%%%

\begin{problem} 
Find the flaw in the following false proof, and give a counterexample
to the claim.

\begin{claim*}
Suppose $R$ is a relation on $A$. If $R$ is symmetric and transitive, 
then $R$ is reflexive.
\end{claim*}

\begin{falseproof}
Let $x$ be an arbitrary element of $A$.  Let $y$ be any element of $A$
such that $x\mrel{R}y$.  Since $R$ is symmetric, it follows that
$y\mrel{R}x$.  Then since $x\mrel{R}y$ and $y\mrel{R}x$, we conclude
by transitivity that $x\mrel{R}x$.  Since $x$ was arbitrary, we have
shown that $\forall x \in A\; x\mrel{R}x$, which means that $R$ is
reflexive.
\end{falseproof}

\solution{
The flaw is assuming that $y$ exists.  It is possible that there is an
$x \in A$ that is not related by $R$ to anything.  No such $R$ will be
reflexive.  The simplest such $R$ that is also symmetric and
transitive is the empty relation on any nonempty set $A$.  We can
easily construct other examples, such as letting $A \eqdef \set{a,b,c}$ and
\[
\graph{R_0} \eqdef \set{(a,a),(a,b),(b,a),(b,b)}.
\] 
Now $R_0$ is not reflexive because $\QNOY(c \mrel{R_0}c)$.  So $R_0$
is a counterexamples to the claim.

Note that the theorem can be fixed: $R$ restricted to its domain of
definition is reflexive.%, and hence an equivalence relation.
}
\end{problem} 



%%%%%%%%%%%%%%%%%%%%%%%%%%%%%%%%%%%%%%%%%%%%%%%%%%%%%%%%%%%%%%%%%%%%%
% Problem ends here
%%%%%%%%%%%%%%%%%%%%%%%%%%%%%%%%%%%%%%%%%%%%%%%%%%%%%%%%%%%%%%%%%%%%%
\endinput
