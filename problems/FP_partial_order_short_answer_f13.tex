\documentclass[problem]{mcs}

\begin{pcomments}
  \pcomment{FP_partial_order_short_answer_f13}
  \pcomment{combination of FP_partial_order_short_answer and FP_partial_order_short_answer_S12}
\end{pcomments}

\pkeywords{
  partial_orders
  weak_partial_order  
  strict_partial_order
  transitive
  asymmetric
  linear_order
}

%%%%%%%%%%%%%%%%%%%%%%%%%%%%%%%%%%%%%%%%%%%%%%%%%%%%%%%%%%%%%%%%%%%%%
% Problem starts here
%%%%%%%%%%%%%%%%%%%%%%%%%%%%%%%%%%%%%%%%%%%%%%%%%%%%%%%%%%%%%%%%%%%%%

\begin{problem}

Indicate which of the following relations below are equivalence
relations, (\textbf{E}), strict partial orders (\textbf{St}), weak
partial orders (\textbf{W}).  For each partial order, also indicate
whether it is \emph{linear} (\textbf{L}).

If a relation is none of the above, indicate whether it is

\emph{transitive} (\textbf{T}), \qquad  \emph{symmetric}
(\textbf{Sym}), \qquad  \emph{asymmetric} (\textbf{A}).

\bparts

\iffalse
\ppart The relation $a = b + 1$ between integers, $a$, $b$,
\hfill \examrule[0.7in]

\begin{solution}
\textbf{A}
\end{solution}
\fi

\iffalse
\ppart The superset relation, $\supseteq$ on the power set of the integers.
\hfill \examrule[0.7in]

\begin{solution}
\textbf{W}
\end{solution}
\fi

\ppart The relation $\expect{R} < \expect{S}$ between real-valued
  random variables $R,S$.
\hfill \examrule[0.7in]

\begin{solution}
\textbf{S} %\textbf{Tr, Asym}
\end{solution}

\ppart The relation $\prob{R = S} = 1$ between real-valued
  random variables $R,S$.
\hfill \examrule[0.7in]

\begin{solution}
\textbf{E} Transitivity follows from the easily verified fact that the
intersection of two probability one events must also have probability one.
\end{solution}

\iffalse
\ppart The \idx{empty relation} on the set of rationals.
\hfill \examrule[0.7in]

\begin{solution}
\textbf{St}
\end{solution}
\fi

\ppart The \idx{identity relation} $\ident{\integers}$ on the set of integers.
\hfill \examrule[0.7in]

\begin{solution}
\textbf{E, W}
\end{solution}

\iffalse
\ppart The divides relation on the nonnegative integers, $\naturals$.  \hfill \examrule[0.7in]

\begin{solution}
\textbf{W}
\end{solution}
\fi

\ppart The divides relation on the integers, $\integers$  \hfill \examrule[0.7in]

\begin{solution}
\textbf{T}
\end{solution}

\ppart The divides relation on the positive powers of 4.  \hfill \examrule[0.7in]

\begin{solution}
\textbf{W, L}
\end{solution}

\ppart The relatively prime relation on the nonnegative integers.  \hfill \examrule[0.7in]

\begin{solution}
\textbf{Sym}
\end{solution}

\iffalse
\ppart The less-than, $<$, relation on real-valued functions, $f(x)$,\\
of the form $f(x) = ax+b$ for constants $a,b \in \reals$. \hfill \examrule[0.7in]

\begin{solution}
\textbf{S}
\end{solution}

\ppart The relation ``has the same prime factors'' on the integers.  \hfill \examrule[0.7in]

\begin{solution}
\textbf{E}
\end{solution}
\fi

\eparts

\medskip

For the next parts, let $f,g$ be nonnegative functions from the
integers to the real numbers.

\bparts
\ppart  The ``Big Oh'' relation, $f=O(g)$,  \hfill \examrule[0.7in]

\begin{solution}
\textbf{Tr}
\end{solution}

\ppart  The ``Little Oh'' relation, $f=o(g)$,  \hfill \examrule[0.7in]

\begin{solution}
\textbf{St}
\end{solution}

\iffalse
\ppart The ``asymptotically equal'' relation, $f \sim g$.  \hfill \examrule[0.7in]

\begin{solution}
\textbf{E}
\end{solution}
\fi

\eparts


\end{problem}

%%%%%%%%%%%%%%%%%%%%%%%%%%%%%%%%%%%%%%%%%%%%%%%%%%%%%%%%%%%%%%%%%%%%%
% Problem ends here
%%%%%%%%%%%%%%%%%%%%%%%%%%%%%%%%%%%%%%%%%%%%%%%%%%%%%%%%%%%%%%%%%%%%%

\endinput
