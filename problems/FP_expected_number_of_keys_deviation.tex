\documentclass[problem]{mcs}

\begin{pcomments}
  \pcomment{FP_expected_number_of_keys_deviation}
  \pcomment{subset of PS_n_keys}
  \pcomment{overlaps FP_expected_number_of_keys}
  \pcomment{ARM 12/12/15}
\end{pcomments}

\pkeywords{
  random_variable
  expectation
  indicator_variable
  variance
  Chebyshev
  asymptotic
  little_oh
}

%%%%%%%%%%%%%%%%%%%%%%%%%%%%%%%%%%%%%%%%%%%%%%%%%%%%%%%%%%%%%%%%%%%%%
% Problem starts here
%%%%%%%%%%%%%%%%%%%%%%%%%%%%%%%%%%%%%%%%%%%%%%%%%%%%%%%%%%%%%%%%%%%%%

\begin{problem}
A man has a set of $n$ keys, one of which fits the door to his
apartment.  He tries a key at random, and if it does not fit the door,
he simply puts it back; so he might try the same ill-fitting key
several times.  He continues until he finds the one right key that
fits.

Let $T$ be the number of times he tries keys until he finds the right
key.

\bparts

\ppart Explain why
\[
\expect{T} =n \quad \text{and} \quad \variance{T} = n(n-1).
\]

%\hint Mean time to failure. 

\examspace[0.75in]

\begin{solution}
Take finding the right key to be a ``failure''.  So the probability of
failure on the any given try, given ``success'' on the previous tries,
is $1/n$.  Therefore, using the formula for the mean and variance of
mean time to failure (Lemma~\bref{lem:exp_time_to_fail},
equation~(\bref{var_time_to_fail}))),
\[
\expect{T} = \frac{1}{p} = \frac{1}{1/n} = n,
\]
and
\[
\variance{T}=\frac{1-p}{p^2} =\frac{1-1/n}{(1/n)^2} = (n-1)n.
\]
\end{solution}

\eparts
\medskip

Let 
\[
f_n(a) \eqdef \pr{T \geq an}.
\]

\bparts

\ppart
Use the Chebyshev Bound to show that for any fixed $n>1$,
\begin{equation}\label{T2nO1n2}
f_n(a) \sim \frac{1}{a^2}.
\end{equation}

\examspace[1.7in]

\begin{solution}
\begin{align*}
\pr{T \geq an}
   & = \pr{T - n \geq (a-1) n}\\
   & \leq \pr{\abs{T - n} \geq (a-1)n}\\
   & \leq \frac{\variance{T}}{((a-1)n)^2}
      & \text{(Chebyshev)}\\
   & = \frac{(n-1)n}{((a-1)n)^2}\\
   & = \frac{n-1}{(a-1)^2 n}\\
   & \sim \frac{1}{a^2}.
\end{align*}
\end{solution}

\ppart Derive an upper bound for $f_n(a)$ that for any fixed $n>1$ is
asymptoticaly smaller than Chebyshev's bound~\eqref{T2nO1n2}.

You may assume that $n$ is large enough to use the approximation
\[
\paren{1-\frac{1}{n}}^{cn} \approx \frac{1}{e^c}
\]

\examspace[2.0in]

\begin{solution}
Let $p \eqdef 1/n$ and $q \eqdef 1-p$.
\begin{align*}
\pr{T \geq an}
  & = \sum_{k=0}^\infty \pr{T= an+k}\\
  & = \sum_{k=0}^\infty q^{an+k-1}p\\
  & = q^{an-1}p \sum_{k=0}^\infty q^k\\
  & = q^{an-1}p \frac{1}{1-q}
        & \text{(geometric series)}\\
  & = q^{an-1}\\
  & = q^{-1} (1-p)^{an}\\
  & = q^{-1} \paren{1-\frac{1}{n}}^{an}\\
  & \sim  q^{-1} \frac{1}{e^a}.
\end{align*}
This $\Theta(1/e^a)$ bound is not only little-oh, but is
\emph{exponentially} smaller than the $1/a^2$ bound~\eqref{T2nO1n2}.
\end{solution}

\eparts

\end{problem}

\endinput
