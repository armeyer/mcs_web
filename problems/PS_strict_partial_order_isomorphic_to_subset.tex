\documentclass[problem]{mcs}

\begin{pcomments}
  \pcomment{PS_strict_partial_order_isomorphic_to_subset
  \pcomment{from: S09.ps3}
  \pcomment{revised by ARM 10/4/11}
\end{pcomments}

\pkeywords{
  partial_orders
  isomorphism
  subset
  inverse
}

\newcommand{\srinv}[1]{\text{L}(#1)}

%%%%%%%%%%%%%%%%%%%%%%%%%%%%%%%%%%%%%%%%%%%%%%%%%%%%%%%%%%%%%%%%%%%%%
% Problem starts here
%%%%%%%%%%%%%%%%%%%%%%%%%%%%%%%%%%%%%%%%%%%%%%%%%%%%%%%%%%%%%%%%%%%%%

\begin{problem} Every \idx{partial order} is \idx{isomorphic} to a
  collection of sets under the subset relation (see
  Section~\bref{poset-as-sets_sec}).  In particular, if $R$ is a
  \emph{strict} partial order on a set, $A$, and $a \in A$, define
\begin{equation}\label{srinvadef}
\srinv{a} \eqdef \set{a} \union \set{x \in A \suchthat x \mrel{R} a}.
\end{equation}
Then
\begin{equation}\label{ps3_rgb}
a \mrel{R} b  \qiff  \srinv{a} \subseteq \srinv{b}
\end{equation}
holds for all $a,b \in A$.

\bparts
\ppart
Carefully prove statement~\eqref{ps3_rgb},
starting from the definitions of strict partial order and the
\idx{subset relation}.

\begin{solution}
\iffalse
  \begin{theorem}
    $a = b \qiff R\set{a} = R\set{b}$.
  \end{theorem}
  \begin{proof}
    ($\Rightarrow$) Suppose $a = b$.  Then $\set{a} = \set{b}$, and
    hence $R\set{a} = R\set{b}$. 

    ($\Leftarrow$) Suppose $R\set{a} = R\set{b}$, and let $X \eqdef
    R\set{a} = R\set{b}$.  Since $R$ is a weak partial order, it is
    reflexive (i.e., $a \mrel{R} a$ and $b \mrel{R} b$), and thus $a,b \in X$.
    Since $X = \set{ x \suchthat x \mrel{R} a}$, and $b \in X$, it follows that $b \mrel{R}
    a$.  Similarly, $a \mrel{R} b$ by swapping $a$ and $b$ in the previous
    sentence.  Since $R$ is a weak partial order, it must be
    antisymmetric (i.e., $a \mrel{R} b \QIMPLIES (\QNOT(b \mrel{R} a) \QOR (a = b))$).
    Since $a \mrel{R} b$ and $b \mrel{R} a$, we conclude that $a = b$.
  \end{proof}
\fi

  \begin{theorem}
    $a \mrel{R} b \qiff \srinv{a} \subseteq \srinv{b}$.
  \end{theorem}
  \begin{proof}
    ($\Rightarrow$) Suppose $a \mrel{R} b$.  Then by transitivity of
    partial order, $x \mrel{R} a \qimplies x \mrel{R} b$.  Another way
    to say this is that $x \in \srinv{a} \qimplies x \in \srinv{b}$,
    which means that $\srinv{a} \subseteq \srinv{b}$.  Notice that
    this direction holds for all partial orders.

    ($\Leftarrow$) Suppose $\srinv{a} \subseteq \srinv{b}$.  Since $R$ is a
    weak partial order, $a \mrel{R} a$ is true, so we have $a \in
    \srinv{a}$.  Since $\srinv{a} \subseteq \srinv{b}$, it follows that $a \in
    \srinv{b}$.  This is equivalent by definition to $a \mrel{R} b$.
  \end{proof}

\end{solution}

\ppart Verify that $\srinv{}:A \to \power(A)$ is obviously an
injective function and so defines a bijection from $A$ to
$\range{\prinv{}}$.  Give an example showing that these properties
would not hold if the definition of $\srinv{}$ in
equation~\eqref{srinvadef} had omitted the expression ``$\set{a} \union$.''

\begin{solution}
\TBA{TBA}

\end{solution}

\eparts

\end{problem}

%%%%%%%%%%%%%%%%%%%%%%%%%%%%%%%%%%%%%%%%%%%%%%%%%%%%%%%%%%%%%%%%%%%%%
% Problem ends here
%%%%%%%%%%%%%%%%%%%%%%%%%%%%%%%%%%%%%%%%%%%%%%%%%%%%%%%%%%%%%%%%%%%%%
 \endinput
