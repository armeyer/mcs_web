\documentclass[problem]{mcs}

\begin{pcomments}
  \pcomment{from: S09.cp3t}
  \pcomment{edited by ARM 10/10/09}
%  \pcomment{}
%  \pcomment{}
\end{pcomments}

\pkeywords{
  bijections
  mapping_lemma
  rational
}

%%%%%%%%%%%%%%%%%%%%%%%%%%%%%%%%%%%%%%%%%%%%%%%%%%%%%%%%%%%%%%%%%%%%%
% Problem starts here
%%%%%%%%%%%%%%%%%%%%%%%%%%%%%%%%%%%%%%%%%%%%%%%%%%%%%%%%%%%%%%%%%%%%%

\begin{problem}
  The rational numbers fill in all the spaces between the integers, so a
  first thought is that there must be more of them than the integers, but
  it's not true.  In this problem you'll show that there are the same
  number of nonnegative rational as nonnegative integers.  In short, the
  nonnegative rationals are countable.

  \bparts

  \ppart Describe a bijection between all the integers, $\integers$, and
  the nonnegative integers, $\naturals$.

  \begin{solution}
One such bijection is defined by lining up all the integers and
    the nonnegative integers as follows:
\[\begin{array}{ccccccccl}
0 & 1 & -1 & 2 & -2 & 3 & -3 & 4 & \dots\\
0 & 1 & 2  & 3 & 4  & 5 & 6  & 7 & \dots
\end{array}\]
We can also define this bijection, $f:\integers \to \integers^+$, by a specification rule
\[
f(n) = \begin{cases}
       -2n & \text{for } n\leq 0,\\
       2\abs{n} -1  & \text{for } n> 0.
       \end{cases}
\]

\end{solution}

\ppart Define a bijection between the nonnegative integers and the set,
$\naturals \times \naturals$, of all the ordered pairs of nonnegative
integers:
\[\begin{array}{l}
(0,0), (0,1),(0,2),(0,3),(0,4),\dots\\
(1,0), (1,1),(1,2),(1,3),(1,4),\dots\\
(2,0), (2,1),(2,2),(2,3),(2,4),\dots\\
(3.0), (3,1),(3,2),(3,3),(3,4),\dots\\
\qquad \vdots
\end{array}\]

\begin{solution}
Line up all the pairs by following successive upper-right to
lower-left diagonals along the top row.

That is, start with (0,0) which counts as an initial diagonal of length 1.
Then follow the length 2 second diagonal (0,1), (1,0), then the length 3
third diagonal (0,2), (1,1), (2,0), then the length 4 fourth diagonal
(0,3), (1,2), (2,1), (3,0), \dots.  So the line up would be
\[\begin{array}{cccccccccccl}
(0,0) & (0,1) & (1,0) & (0,2) & (1,1) & (2,0) & (0,3) & (1,2) & (2,1) & (3,0)& \dots\\ 
 0    &   1   & 2     & 3     & 4     & 5     & 6     & 7     & 8     & 9    &\dots
\end{array}\]

It's interesting that this bijection from $\naturals \times \naturals$
to $\naturals$ has a simple formula: the pair $(k,m)$ is the $k$th
element on the diagonal consisting of the pairs whose sum is $k+m$.  The
total number of elements in all the preceding diagonals is
\[
0 + 1 + 2 + \cdots + (k+m) = (k+m+1)(k+m)/2
\]
so the pair $(k,m)$ appears as the $(k+m+1)(k+m)/2 + k$th element in
the line up.
\end{solution}

\ppart Conclude that $\naturals$ is the same size as the set,
$\rationals$, of all nonnegative rational numbers.

\begin{solution}
One way to line up the nonnegative rationals is to take the list
  of all pairs, $(k,m)$, of integers above and replace each remaining pair
  by the rational number $k/m$, skipping the pairs where $m=0$:
\begin{quote}
0/1\ \ \ 0/2\ \ \ 1/1\ \ \ 1/2\ \ \ 0/3\ \ \ 1/2\ \ \ 2/1\ \ \ 0/3\ \ \ 1/2\ \ \ 2/1\ \ \ \dots
\end{quote}
and, going from left to right, delete all the occurrences of numbers
that are already in the list:
\begin{quote}
0\ \ \ 1\ \ \ 1/2\ \ \ 2\ \ \ 3\ \ \ 1/3\ \ \ 1/4 \dots.
\end{quote}

\end{solution}

\eparts
\end{problem}

%%%%%%%%%%%%%%%%%%%%%%%%%%%%%%%%%%%%%%%%%%%%%%%%%%%%%%%%%%%%%%%%%%%%%
% Problem ends here
%%%%%%%%%%%%%%%%%%%%%%%%%%%%%%%%%%%%%%%%%%%%%%%%%%%%%%%%%%%%%%%%%%%%%

\endinput
