\documentclass[problem]{mcs}

\begin{pcomments}
  \pcomment{MQ_RM_subs_M}
  \pcomment{adapted from PS_M_equal_RM 2/28/11}
\end{pcomments}

\pkeywords{
  string
  matched
  bracket
  structural_induction
  induction
  concatenation
  ambiguous
}

%%%%%%%%%%%%%%%%%%%%%%%%%%%%%%%%%%%%%%%%%%%%%%%%%%%%%%%%%%%%%%%%%%%%%
% Problem starts here
%%%%%%%%%%%%%%%%%%%%%%%%%%%%%%%%%%%%%%%%%%%%%%%%%%%%%%%%%%%%%%%%%%%%%

\begin{problem}
The set of strings, $\RM$, is recursively defined as follows:
\begin{itemize}

\item \textbf{Base case:} $\emptystring \in\RM$.

\item \textbf{Constructor case:} If $s,t \in\RM$, then
\[
\lefbrk s\, \rhtbrk t \in \RM.
\]
\end{itemize}

The set of strings, $M$, is recursively defined as follows:
\begin{itemize}

\item \textbf{Base case:} $\emptystring \in M$,

\item \textbf{Constructor cases:} if $s,t \in M$, then
  the strings $\lefbrk s\, \rhtbrk$ and $s\cdot t$ are also in $M$.
\end{itemize}

Prove by structural induction that

\bparts

\ppart  $\RM \subseteq M$

\begin{solution}

\begin{proof}
We prove by structural induction on the definition of $\RM$ that
the predicate
\[
P(x) \eqdef x \in \M
\]
is true for all $x \in \RM$.

\textbf{Base case} ($x = \emptystring$): By definition of $\M$, the empty string is in $\M$.  

\textbf{Constructor case} ($x = \lefbrk s\, \rhtbrk t$ for $s,t \in \RM$):  By
structural induction hypothesis, we may assume that $s, t \in \M$.  By the first constructor case of $M$, it follows that $[s] \in \M$.  Then, by the second constructor case of $\M$, it follows that $\lefbrk s\, \rhtbrk t \in \M$.   

This completes the proof by structural induction, so we conclude that
\[
\forall x.\ x\in \RM \QIMPLIES\ x \in \M
\]
which by definition means that $\RM \subseteq \M$.

\end{proof}

\end{solution}

We won't ask you to prove it here, but it is also true that $M \subseteq \RM$; it follows that these sets are equal.  

\ppart One of these set definitions is \emph{ambiguous}.  Which one is it?

\begin{solution}

TBA

\end{solution}

\ppart What is a disadvantage of having an ambiguous set definition?

\begin{solution}

TBA

\end{solution}

\eparts

\end{problem}

%%%%%%%%%%%%%%%%%%%%%%%%%%%%%%%%%%%%%%%%%%%%%%%%%%%%%%%%%%%%%%%%%%%%%
% Problem ends here
%%%%%%%%%%%%%%%%%%%%%%%%%%%%%%%%%%%%%%%%%%%%%%%%%%%%%%%%%%%%%%%%%%%%%

\endinput
