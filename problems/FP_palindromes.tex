\documentclass[problem]{mcs}

\begin{pcomments}
  \pcomment{FP_palindromes}
  \pcomment{Emanuele and ARM 3/12/16}
\end{pcomments}

\pkeywords{
   recursive
   string
   reverse
   concatenation
   palindrome
}

%\newcommand{\catOK}{\text{cat-OK}}
%\newcommand{\revstr}{\text{rev}}
%\newcommand{\Zerostr}{\text{Zeroes}}


%%%%%%%%%%%%%%%%%%%%%%%%%%%%%%%%%%%%%%%%%%%%%%%%%%%%%%%%%%%%%%%%%%%%%
% Problem starts here
%%%%%%%%%%%%%%%%%%%%%%%%%%%%%%%%%%%%%%%%%%%%%%%%%%%%%%%%%%%%%%%%%%%%%

\begin{problem}
\begin{definition}
Let $r,s,t$ be finite binary strings, \finbin, and let $b,c \in
\set{0,1}$ be binary digits.  This set of \finbin\ and the
\emph{length} of $s \in \finbin$ can be defined recursively:

\inductioncase{Base case}: $\emptystring \in \finbin$, and
$\lnth{\emptystring} \eqdef 0$.

\inductioncase{Constructor case}: If $s \in \finbin$, then so is $sb$
where $b \in \set{0,1}$, and $\lnth{sb} \eqdef
\lnth{s}+1$.\footnote{The concatenation of two strings $x$ and $y$,
  written $xy$, is the string obtained by appending $x$ to the left
  end of $y$.  For example, the concatenation of $01$ and $101$ is
  $01101$.}
\end{definition}

Based on this definition, we can now recursively define
\begin{definition}
The \emph{reversal} function, $\text{rev}: \finbin \to \finbin$ is defined follows:

\inductioncase{Base case}:  $\rev{\emptystring} \eqdef \emptystring$.

\inductioncase{Constructor case}: $\rev{sb} \eqdef b\rev{s}$ for
$s\in \finbin$ and $b\in \set{0,1}$.
\end{definition}

The recursive definition lets is verify properties of the reverse
function by structural induction.  For example,
\begin{lemma*}
\[
\rev{bs} = \rev{s}b
\]
for 
\end{lemma*}
\begin{definition}
The binary \emph{palindromes} are defined recursively by:

\inductioncase{Base case}: $\emptystring$ is a palindrome.

\inductioncase{Constructor case}: If $s$ is a palindrome, then so is $bsb$
where $b \in \set{0,1}$.
\end{definition}

\bparts

\ppart Prove that $s = \rev{s}$ for all palindromes $s$.

\examspace[3.0in]

\begin{solution}
\TBA{needs to be checked!}
\end{solution}

\ppart Prove conversely that if $s = \rev{s}$, then $s$ is a
palindrome.

\hint By induction on $\lnth{s}$.

\begin{solution}
\TBA{needs to be checked!}
\end{solution}

\eparts
\end{problem}

\endinput
