\documentclass[problem]{mcs}

\begin{pcomments}
  \pcomment{CP_degree_constrained_induction}
  \pcomment{ARM 4/8/14}
  \pcomment{from F12 rec 6}
\end{pcomments}

\pkeywords{
  graph
  degree
  regular
  matching
  induction
}

%%%%%%%%%%%%%%%%%%%%%%%%%%%%%%%%%%%%%%%%%%%%%%%%%%%%%%%%%%%%%%%%%%%%%
% Problem starts here
%%%%%%%%%%%%%%%%%%%%%%%%%%%%%%%%%%%%%%%%%%%%%%%%%%%%%%%%%%%%%%%%%%%%%

\begin{problem}
A simple graph is called \term{regular} when every vertex has the same
degree.  Call a graph \emph{balanced} when it is regular and is also a
bipartite graph with the same number of left and right vertices.

Prove that if $G$ is a balanced graph, then the edges of $G$ can be
partitioned into blocks such that each block is a perfect matching.

For example, if $G$ is a balanced graph with vertex degree 3 and $2n$
vertices, then the edges of $G$ can be partitioned into 3 blocks,
where each block consists of $n$ edges, no two of which are incident.

\hint Induction on degree.

\begin{solution}
\begin{proof}
The proof is by induction on the degree $d$ of the vertices in a
balanced graph.  The induction hypothesis is

$P(d) \eqdef$ If $G$ is balanced graph with vertices of degree $d$,
then $\edges{G}$ can be partitioned into $d$ perfect matchings.

\inductioncase{Base case}: $(d=0)$.  If $G$ is regular with degree 0
vertices, then it has no edges, so the empty partition with 0 blocks
which vacuously are perfect matchings.

\inductioncase{Induction step} Let $G$ be a balanced graph with
vertices of degree $d+1$.  We need to prove that $\edges{G}$ can be
partitioned into $d+1$ perfect matchings.

Since all vertices have the same degree, $G$ is
\idx{degree-constrained} and so has a perfect matching $M$ by
Theorem~\bref{lem:no_bottleneck_degree_constrained}.

Let $G-M$ be the graph obtained by removing the edges in $M$ from $G$.
Now $G-M$ is balanced with degree $d$, since each vertex is incident
to exactly one edge in $M$, by definition of perfect matching.  By
induction, we may assume that $\edges{G-M}$ can be partitioned into
$d$ perfect matchings, and $M$, together with these $d$ matchings, is
a partition of $\edges{G}$ into $d+1$ perfect matchings.

This proves $P(d+1)$, and so by induction, $P(d)$ holds for all $d
\geq 0$.
\end{proof}
\end{solution}

\end{problem}
%%%%%%%%%%%%%%%%%%%%%%%%%%%%%%%%%%%%%%%%%%%%%%%%%%%%%%%%%%%%%%%%%%%%%
% Problem ends here
%%%%%%%%%%%%%%%%%%%%%%%%%%%%%%%%%%%%%%%%%%%%%%%%%%%%%%%%%%%%%%%%%%%%%

\endinput

