\documentclass[problem]{mcs}

\begin{pcomments}
  \pcomment{MQ_pigeonhole_cards_3_of_a_kind}
\end{pcomments}

\pkeywords{
  counting
  pigeonhole principle
}

%%%%%%%%%%%%%%%%%%%%%%%%%%%%%%%%%%%%%%%%%%%%%%%%%%%%%%%%%%%%%%%%%%%%%
% Problem starts here
%%%%%%%%%%%%%%%%%%%%%%%%%%%%%%%%%%%%%%%%%%%%%%%%%%%%%%%%%%%%%%%%%%%%%

\begin{problem}
In a standard 52-card deck, what is the smallest $k$ such that
every size $k$ subset of the 52 cards contains a 3-of-a-kind?
(3-of-a-kind is defined as 3 cards of the same rank).
Briefly explain your answer.

\begin{solution}
By the Pigeonhole Principle, there are 13 ranks (holes),
so $(13)(3-1)+1=27$ cards (pigeons) will guarantee that
at least 3 cards have the same rank.
\end{solution}


\end{problem}

%%%%%%%%%%%%%%%%%%%%%%%%%%%%%%%%%%%%%%%%%%%%%%%%%%%%%%%%%%%%%%%%%%%%%
% Problem ends here
%%%%%%%%%%%%%%%%%%%%%%%%%%%%%%%%%%%%%%%%%%%%%%%%%%%%%%%%%%%%%%%%%%%%%

\endinput
