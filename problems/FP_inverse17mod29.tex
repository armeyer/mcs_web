\documentclass[problem]{mcs}

\begin{pcomments}
  \pcomment{FP_inverse17mod29}
  \pcomment{variant of S11.ps4.prob2}
  \pcomment{by Tigran Sloyan 5/12/11}
\end{pcomments}

\pkeywords{
  number_theory
  Pulverizer
  modular_arithmetic
  inverses
  Fermat_theorem
  remainder
}

%%%%%%%%%%%%%%%%%%%%%%%%%%%%%%%%%%%%%%%%%%%%%%%%%%%%%%%%%%%%%%%%%%%%%
% Problem starts here
%%%%%%%%%%%%%%%%%%%%%%%%%%%%%%%%%%%%%%%%%%%%%%%%%%%%%%%%%%%%%%%%%%%%%

\begin{problem}
Find the inverse of 17 modulo 29 in the interval $[1,28]$.

\begin{solution}
\TBA{update for 17, 29}

We first use the Pulverizer to find $s,t$ such that $\gcd(23,13) = s\cdot 23 +
t\cdot 13$, namely,
\[
1 = 4 \cdot 23 - 7 \cdot 13.
\]
This implies that $-7$ is an inverse of 13 modulo 23.

Here is the Pulverizer calculation:
\[
\begin{array}{ccccrcl}
x & \quad & y & \quad & \rem{x}{y} & = & x - q \cdot y \\ \hline
23 && 13 && 10  & = &   23 - 13 \\
13 && 10 && 3   & = &   13 - 10 \\
&&&&            & = &   13 - (23 - 13)\\
&&&&            & = &  (-1)\cdot 23 + 2 \cdot 13\\
10 && 3  && 1   & = &   10 - 3 \cdot 3 \\
&&&&            & = &   (23 - 13) - 3 \cdot ((-1)\cdot 23 + 2 \cdot 13))\\
&&&&            & = &   \fbox{$4 \cdot 23 - 7 \cdot 13$} \\
3  && 1  && 0   & = &
\end{array}
\]

To get an inverse in the specified range, simply find
$\rem{-7}{23}$, namely \textbf{16}.
\end{solution}

\end{problem}

%%%%%%%%%%%%%%%%%%%%%%%%%%%%%%%%%%%%%%%%%%%%%%%%%%%%%%%%%%%%%%%%%%%%%
% Problem ends here
%%%%%%%%%%%%%%%%%%%%%%%%%%%%%%%%%%%%%%%%%%%%%%%%%%%%%%%%%%%%%%%%%%%%%

\endinput
