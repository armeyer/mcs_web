%CP_mating_ritual_example

\documentclass[problem]{mcs}

\begin{pcomments}
  \pcomment{from: S08 cp5f; S06, cp4f; F03 rec4}
\end{pcomments}

\pkeywords{
 stable_matching
 optimal
 pessimal
 Mating_ritual
}


%%%%%%%%%%%%%%%%%%%%%%%%%%%%%%%%%%%%%%%%%%%%%%%%%%%%%%%%%%%%%%%%%%%%%
% Problem starts here
%%%%%%%%%%%%%%%%%%%%%%%%%%%%%%%%%%%%%%%%%%%%%%%%%%%%%%%%%%%%%%%%%%%%%

\newcommand{\Jeff}{Rich}
\newcommand{\Tina}{Megumi}
\newcommand{\Jay}{Justin}

\begin{problem} Four Students want separate assignments to four VI-A
Companies.  Here are their preference rankings:
\begin{center}
\begin{tabular}{r|c} 
Student & Companies \\ \hline
Albert:  & HP, Bellcore, AT\&T, Draper \\
\Jeff:     & AT\&T, Bellcore, Draper, HP \\
\Tina:   & HP, Draper, AT\&T, Bellcore \\
\Jay:    & Draper, AT\&T, Bellcore, HP
\end{tabular}
\end{center}
\begin{center}
\begin{tabular}{r|c}
Company   & Students \\ \hline
AT\&T:    & \Jay, Albert, \Tina, \Jeff \\
Bellcore: & \Tina, \Jeff, Albert, \Jay \\
HP:   & \Jay, \Tina, Albert, \Jeff \\
Draper:   & \Jeff, \Jay, \Tina, Albert
\end{tabular}
\end{center}

\bparts

\ppart Use the Mating Ritual to find \emph{two} stable assignments of
Students to Companies.

\begin{solution}
Treat Students as Boys and the result is the following
assignment:
\begin{center}
\begin{tabular}{r|c|c} 
Student & Companies & Rank in the original list \\ \hline
Albert:  & Bellcore & 2\\
\Jeff:     & AT\&T & 1\\
\Tina:   & HP & 1\\
\Jay:    & Draper &1
\end{tabular}
\end{center}

Treat Companies as Boys and the result is the following assignment:
\begin{center}
\begin{tabular}{r|c|c}
Company & Students & Rank in the original list\\ \hline
AT\&T:    & Albert & 2\\
Bellcore: & \Jeff & 2\\
HP:   & \Tina & 2\\
Draper:   & \Jay & 2
\end{tabular}
\end{center}

\end{solution}

\ppart Describe a simple procedure to determine whether any given stable
marriage problem has a unique solution, that is, only one possible stable
matching.

\begin{solution}
See if the Mating Ritual with Boys as suitors yields the same
solution as the algorithm with Girls as suitors.  These two marriage
assignments are boy-optimal and boy-pessimal, respectively, so they agree
iff there is a unique solution.

To see exactly why this is so, suppose there are two stable matchings.
Then at least one boy, call him Brad, must have different wives, Angelina
and Jen, in these two matchings.  Since Brad will prefer one of these
spouses, say Angelina over Jen, then in his optimal matching he must have
a wife at least as desirable as Angelina and in his pessimal matching a
wife at most as desirable as Jen.  So Brad must have different wives in
the boy-optimal and boy-pessimal matchings.  It follows that if the
boy-pessimal and boy-optimal matchings agree, there can't be any other
stable matching.
\end{solution}

\eparts

\end{problem}

