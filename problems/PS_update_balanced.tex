\documentclass[problem]{mcs}
\begin{pcomments}
  \pcomment{PS_update_balanced}
  \pcomment{ARM 10/8/17}
\end{pcomments}

\pkeywords{
  trees
  binary_tree
  search_tree
  insertion
  deletion
  }

%%%%%%%%%%%%%%%%%%%%%%%%%%%%%%%%%%%%%%%%%%%%%%%%%%%%%%%%%%%%%%%%%%%%%
% Problem starts here
% %%%%%%%%%%%%%%%%%%%%%%%%%%%%%%%%%%%%%%%%%%%%%%%%%%%%%%%%%%%%%%%%%%%%
\begin{problem}

This problem will show that if $T$ is a search tree, and $U$ is a
search tree of the \emph{same shape} as $T$ for the same values as $T$
after one deletion and one insertion, then $T$ and $U$ have no subtrees
in common.  So if we tried to maintain the shape of a search tree,
then after just one deletion and one insertion we might have to find
$\sz{T}$ new subtrees.  This dramatically contrasts with the case of
AVL trees, where by allowing the shape of $U$ to differ somewhat from
the shape of $T$, the search trees $T$ and $U$ can share all but
proportional to $\log_2(\sz{T})$ subtrees.

We assume there is a fixed numerical tree labelling $\nlbl{}: \rectr
\to \reals$.  We'll say that $R,S \in \rectr$ \emph{clash} when only
one of them is a leaf, or else both are branching and
\begin{align*}
          \nlbl{(R)} & \neq \nlbl{(S)},\ \text{or}\\
\nlbl{(\leftsub{(R)})} & \neq \nlbl{(\leftsub{(S)})},\ \text{or}\\
\nlbl{(\rightsub{(R)})}& \neq \nlbl{(\rightsub{(S)})}.
\end{align*}
If $R$ and $S$ clash, then of course $R \neq S$.

The \emph{isomorphism} relation between recursive trees $T,U \in
\rectr$ is defined recursively

\inductioncase{Base case}: ($T \in \leafset$).  $T$ is isomorphic to
$U$ iff $U$ is a leaf.

\inductioncase{Constructor case}: ($T \in \brnchng$).  $T$ is
isomorphic to $U$ iff
\[
U \in \brnchng\ \QAND\
 \leftsub{(U)}\ \text{isomorphic to}\ \leftsub{(T)}\ \QAND\
\rightsub{(U)}\ \text{isomorphic to}\ \rightsub{(T)}.
\]

\bparts

\ppart Prove that if $T$ and $U$ are isomorphic, then $\sz{T} = \sz{U}$.

\begin{solution}
Trivial by structural induction:

\inductioncase{Base case}: ($T \in \leafset$).  If $T$ is isomorphic
to $U$, then $U$ must also be a leaf, so $\sz{T} = 1 = \sz{U}$.

\inductioncase{Constructor case}: ($T \in \brnchng$).  If $U$
isomorphic to $T$, then by def.\ of isomorphism, $\leftsub{(T)}$ is isomorphic to $\leftsub{(U)}$, and so 
\[
\sz{\leftsub{(T)}} = \sz{\leftsub{(U)}}
\]
by induction hypothesis.  Likewise, 
\[
\sz{\rightsub{((T))}} = \sz{\rightsub{(U)}}.
\]
Therefore,
\begin{align*}
\sz{T} & = 1 + \sz{\leftsub{(T)}} + \sz{\rightsub{(T)}}\\
       & = 1 + \sz{\leftsub{(U)}} + \sz{\rightsub{(U)}} = \sz{U}.
\end{align*}
\end{solution}

\ppart\label{uniqnum} Show that if $T$ and $U$ are isomorphic search
trees and $T$ and $U$ are search trees for the same set of values,
that is, $\nlbls{(T)} = \nlbls{(U)}$, then $\nlbl{(T)} = \nlbl{(U)}$.

\hint By definition of isomorphism and search tree.  No induction needed.

\begin{solution}

\inductioncase{Base case}: ($T \in \leafset$).  There is only one
possible labelling of $T$ and $U$ with the same value.

\inductioncase{Constructor case}: ($T \in \brnchng$).  Since $T$ is a
search tree, all the values $\nlbls{(\leftsub{(T)})}$ in
$\leftsub{(T)}$ must be less than the label $\nlbl{(T)}$ of $T$, and
likewise all the labels in $\nlbls{(\rightsub{(T)})}$ must be greater
than $\nlbl{(T)}$.  So $\nlbls{(\leftsub{(T)})}$ must be the smallest
$\sz{\leftsub{(T)}}$ values in $\nlbls{(T)}$, and $\nlbl{(T)}$ must be
the $\sz{\leftsub{(T)}}+1$st smallest value.  Since $T$ and $U$ are
isomorphic, the sizes of their corresponding branches are the same, so
by the same reasoning, $\nlbl{(U)}$ must also be the same
$\sz{\leftsub{(T)}}+1$st smallest value.
\end{solution}

\ppart Let $T \in \rectr$ be a search tree.  Say that \emph{$U$ is a
  shift of $T$} when $U \in \rectr$ is another search tree that is
isomorphic to $T$, and $\nlbls{(U)}$ is the set that results from
deleting $\min(T)$ from $\nlbls{(T)}$ and then inserting some element
$m > \max(T)$, that is,
\[
\nlbls{(U)} \eqdef (\nlbls{(T)} - \set{\min(T)}) \union \set{m}.
\]

%\begin{equation}\tag{shiftnums}
%\end{equation}

Prove that every subtree of $T$ clashes with with every subtree of a
shift of $T$.

\hint Structural induction on $T$.

\begin{solution} 
Suppose $U$ is a shift of $T$.  We prove that all the subtrees of $T$
and $U$ clash by structural induction on the definition of $T \in
\rectr$, with induction hypothesis
\[
P(T) \eqdef 
\forall R \in \subbrn{T}, S \in \subbrn{U}.\, R\ \text{clashes with}\ S.
\]
\begin{proof}

\inductioncase{Base case}: ($T \in \leafset$).  We know $\nlbl{(U)} =
m > \nlbl{(T)}$ by definition of shift, so $T$ and $U$ clash.  There
are no other subtrees of $T$ and $U$, so $P(T)$ holds.

\inductioncase{Constructor case}: ($T \in \brnchng$).  By definition
of isomorphism, $\leftsub{(T)}$ will be isomorphic to $\leftsub{(U)}$.
Moreover, both will be search trees by definition of search tree.  Finally,
\begin{equation}\tag{subnums}
\nlbls{(\leftsub{(U)})} = (\nlbls{(\leftsub{(T)})} - \min\set{T}) \union \nlbl{(T)},
\end{equation}
by definition of shift.  This means that $\leftsub{(U)}$ is a shift
of $\leftsub{(T)}$.  So by induction hypothesis, every subtree of
$\leftsub{(T)}$ clashes with every subtree of $\leftsub{(U)}$.
Likewise for subtrees of $\rightsub{(T)}$ and $\rightsub{(U)}$.

Next, we observe that since $U$ is a shift of $T$, all the labels in
$\leftsub{(U)}$ are less than all the labels in $\rightsub{(T)}$, and
likewise all the labels in $\rightsub{(U)}$ are larger than those in
$\leftsub{(T)}$.  So any pair of subtrees from opposite branches must
have different labels and therefore clash.

Finally, we argue that $T$ itself clashes with the unique subtree $S
\in \subbrn{U}$ such that $\nlbl{(S)} = \nlbl{(T)}$.

From part~\eqref{uniqnum}, we know $\nlbl{(U)}$ is the next largest
number in $\nlbls{(T)}$ after $\nlbl{(T)}$, so $\nlbl{(U)} >
\nlbl{(T)}$.  Therefore $S \neq U$.  We can assume wlog that $S \in
\leftsub{(U)}$.  Now~(subnums) implies that every label in $S$ is less
than $\nlbl{(U)}$ and therefore is $\leq \nlbl{(T)}$.  In particular,
\[
\nlbl{(\rightsub{(S)})} \leq \nlbl{(T)} < \nlbl{(\rightsub{(T)})},
\]
proving that $T$ and $S$ clash.  A symmetrical argument shows that $U$
clashes with the unique subtree of $T$ with the same label.

This proves that all the subtrees of $T$ clash with all the subtrees
of $U$, that is, $P(T)$.
\end{proof}

\end{solution}

\eparts

\end{problem}

%%%%%%%%%%%%%%%%%%%%%%%%%%%%%%%%%%%%%%%%%%%%%%%%%%%%%%%%%%%%%%%%%%%%%
% Problem ends here
%%%%%%%%%%%%%%%%%%%%%%%%%%%%%%%%%%%%%%%%%%%%%%%%%%%%%%%%%%%%%%%%%%%%%

\endinput
