\documentclass[problem]{mcs}
\begin{pcomments}
  \pcomment{PS_update_balanced}
  \pcomment{ARM 10/8/17}
\end{pcomments}

\pkeywords{
  trees
  binary_tree
  search_tree
  insertion
  deletion
  }

%%%%%%%%%%%%%%%%%%%%%%%%%%%%%%%%%%%%%%%%%%%%%%%%%%%%%%%%%%%%%%%%%%%%%
% Problem starts here
% %%%%%%%%%%%%%%%%%%%%%%%%%%%%%%%%%%%%%%%%%%%%%%%%%%%%%%%%%%%%%%%%%%%%

\begin{problem}

\begin{staffnotes}
Parts~\eqref{isosz} and~\eqref{uniqnum} are easy but informative
structural inductions, requiring only familiarity with the recursive
definition of search trees from Section~\bref{search_tree} of the
text.  The main challenge in part~\eqref{shiftleft} is undertanding
the definitions.
\end{staffnotes}


This problem will show that if $T$ is a search tree, and $U$ is a
search tree of the \emph{same shape} as $T$ for the same values as $T$
after one deletion and one insertion, then $T$ and $U$ have no subtrees
in common.  So if we tried to maintain the shape of a search tree,
then after just one deletion and one insertion we might have to find
$\sz{T}$ new subtrees.  This contrasts dramatically with the case of
AVL trees, where by allowing the shape of $U$ to differ somewhat from
the shape of $T$, the search trees $T$ and $U$ can share all but
proportional to $\log_2(\sz{T})$ subtrees.

\bigskip

The \emph{isomorphism} relation between recursive trees $T,U \in
\rectr$ is defined recursively

\inductioncase{Base case}: ($T \in \leafset$).  $T$ is isomorphic to
$U$ iff $U$ is a leaf.

\inductioncase{Constructor case}: ($T \in \brnchng$).  $T$ is
isomorphic to $U$ iff
\[
U \in \brnchng\ \QAND\
 \leftsub{(U)}\ \text{isomorphic to}\ \leftsub{(T)}\ \QAND\
\rightsub{(U)}\ \text{isomorphic to}\ \rightsub{(T)}.
\]

\bparts

\ppart\label{isosz} Prove that if $T$ and $U$ are isomorphic, then $\sz{T} = \sz{U}$.

\begin{solution}
Trivial by structural induction:

\inductioncase{Base case}: ($T \in \leafset$).  If $T$ is isomorphic
to $U$, then $U$ must also be a leaf, so $\sz{T} = 1 = \sz{U}$.

\inductioncase{Constructor case}: ($T \in \brnchng$).  If $U$
isomorphic to $T$, then by def.\ of isomorphism, $\leftsub{(T)}$ is
isomorphic to $\leftsub{(U)}$, and so
\[
\sz{\leftsub{(T)}} = \sz{\leftsub{(U)}}
\]
by induction hypothesis.  Likewise, 
\[
\sz{\rightsub{((T))}} = \sz{\rightsub{(U)}}.
\]
Therefore,
\begin{align*}
\sz{T} & = 1 + \sz{\leftsub{(T)}} + \sz{\rightsub{(T)}}\\
       & = 1 + \sz{\leftsub{(U)}} + \sz{\rightsub{(U)}} = \sz{U}.
\end{align*}
\end{solution}
\eparts

\bigskip
We now assume there is a fixed numerical tree labelling $\nlbl{}: \rectr
\to \reals$.

\bparts \ppart\label{uniqnum} Show that if $T$ and $U$ are isomorphic
search trees for the same set of values, that is, $\nlbls{(T)} =
\nlbls{(U)}$, then $\nlbl{(T)} = \nlbl{(U)}$.

\hint By definition of isomorphism and search tree.  No induction needed.

\begin{solution}

\inductioncase{Case}: ($T \in \leafset$).  There is only one
possible labelling of $T$ and $U$ with the same value.

\inductioncase{Case}: ($T \in \brnchng$).  Since $T$ is a
search tree, all the values $\nlbls{(\leftsub{(T)})}$ in
$\leftsub{(T)}$ must be less than the label $\nlbl{(T)}$ of $T$, and
likewise all the labels in $\nlbls{(\rightsub{(T)})}$ must be greater
than $\nlbl{(T)}$.  So $\nlbls{(\leftsub{(T)})}$ must be the smallest
$\sz{\leftsub{(T)}}$ values in $\nlbls{(T)}$, and $\nlbl{(T)}$ must be
the $\sz{\leftsub{(T)}}+1$st smallest value.  Since $T$ and $U$ are
isomorphic, the sizes of their corresponding branches are the same, so
by the same reasoning, $\nlbl{(U)}$ must also be the same
$\sz{\leftsub{(T)}}+1$st smallest value.
\end{solution}
\eparts
\bigskip

Let $T \in \rectr$ be a search tree whose numerical labels are
integers in some interval
\[
\nlbls{(T)} = \Zintvoc{k}{(k + \sz{T}} 
\]
where
\[
\Zintvoc{k}{n} \eqdef \set{k+1,\dots,n-1,n}.
\]

A \emph{shift} of $T$ is another search tree $U$ isomorphic to $T$
whose labels are one larger, that is
\[
\nlbls{(U)} = \Zintvoc{(k+1)}{((k + 1)+\sz{T})}.
\]

\begin{staffnotes}
Have students work out an example first, say a size
11 search tree:
\begin{center}
\begin{verbatim}
 T:        7                  shift T       8       
          / \                              / \      
         /   \                            /   \     
        3     9                          4     10    
       / \   / \                        / \   / \   
      /   \ 8  10                      /   \ 9  11
     1     5                          2     6       
    / \   / \                        / \   / \      
    0 2   4 6                        1 3   5 7      
\end{verbatim}         
\end{center}

\end{staffnotes}

\bparts

\ppart\label{shiftleft} Show that if $U$ is a shift of $T$, then
$\nlbl{(U)} = \nlbl{(T)} + 1$ and $\leftsub{(U)}$ is a shift of
$\leftsub{(T)}$.

\hint By definition of search tree, isomorphism and shift.  No
induction is needed.

\begin{solution}
By definition of isomorphism, $\leftsub{(T)}$ will be isomorphic to
$\leftsub{(U)}$.  By definition of search tree, $\leftsub{(T)}$ will
be a search tree on the $\sz{\leftsub{(T)}}$ smallest numbers in
  $\nlbls{(T)}$, namely,
\[
\nlbls{(\leftsub{(T)})} = \Zintvoc{k}{(k+\sz{\leftsub{(T)}})}
\]
and $\nlbl{(T)}$ will be the $\sz{(\leftsub{(T)})} +1$st element of
$\nlbls{(T)}$, namely,
\[
\nlbl{(T)} = (k+\sz{\leftsub{(T)}})+1.
\]
Likewise $\leftsub{(U)}$ will be a search tree on the
$\sz{\leftsub{(U)}}$ smallest elements in $\nlbls{(U)}$, namely,
\begin{align*}
\nlbls{(\leftsub{(U)})}
& = \Zintvoc{(k+1)}{((k+1)+ \sz{\leftsub{(U)}})}\\
& = \Zintvoc{(k+1)}{((k+1)+ \sz{\leftsub{(T)}})}.
\end{align*}
This shows that $\leftsub{(U)}$ is a shift of $\leftsub{(T)}$.
Finally, $\nlbl{(U)}$ will be the $\sz{\leftsub{(U)}} +1$st element
of $\nlbls{(U)}$, namely
\begin{align*}
\nlbl{(U)} & = [(k+1)+ \sz{\leftsub{(U)}}]+1\\
           & = [(k+1)+ \sz{\leftsub{(T)}}]+1\\
           & = \nlbl{(T)} +1.
\end{align*}
\end{solution}

\eparts
\bigskip

We'll say that $R,S \in \rectr$ \emph{clash} when only one of them is
a leaf, or else both are branching and
\begin{align*}
          \nlbl{(R)} & \neq \nlbl{(S)},\ \text{or}\\
\nlbl{(\leftsub{(R)})} & \neq \nlbl{(\leftsub{(S)})},\ \text{or}\\
\nlbl{(\rightsub{(R)})}& \neq \nlbl{(\rightsub{(S)})}.
\end{align*}
If $R$ and $S$ clash, then of course $R \neq S$.

\begin{staffnotes}
For part~\eqref{subclashes}, make sure students understand what's
being claimed, and how this connects to AVL trees, as explained in the
problem introduction.

Rather than attempting a proof, it could be more productive to walk
students through the proof below.  This proof is checkable pretty much
line by line, though the intuition gets obscured by the equations.  It
is also OK to skip the proof.

Students may be able to come up with a better explanation.
\end{staffnotes}

\bparts

\ppart\label{subclashes} Prove that every subtree of $T$ clashes with
every subtree of a shift of $T$.

\hint Suppose $U$ is a shift of $T$, and $R$ is a subtree of $T$ and
$S$ is a subtree of $U$.  Show that $R$ and $S$ clash by
cases:

Case 1: $R \in \leftsub{(T)}$ and $S \in \leftsub{(U)}$.

Case 2: $R \in \leftsub{(T)}$ and $S \in \rightsub{(U)}$.

Case 3: $R = T$ and $S \in \leftsub{(U)}$.

\begin{solution} 
Suppose $U$ is a shift of $T$.  We prove that all the subtrees of $T$
and $U$ clash by structural induction on the definition of $T \in
\rectr$, with induction hypothesis
\[
P(T) \eqdef 
\forall R \in \subbrn{T}, S \in \subbrn{U}.\, R\ \text{clashes with}\ S.
\]
\begin{proof}

\inductioncase{Base case}: ($T \in \leafset$).  We know $\nlbl{(U)} =
\nlbl{(T)}+1$, so $T$ and $U$ clash.  There are no other subtrees of
$T$ and $U$, so $P(T)$ holds.

\inductioncase{Constructor case}: ($T \in \brnchng$).  By
part~\eqref{shiftleft}, $\leftsub{(U)}$ is a shift of $\leftsub{(T)}$.

Case 1: By induction hypothesis, every subtree of $\leftsub{(T)}$
clashes with every subtree of $\leftsub{(U)}$.  Likewise for subtrees
of $\rightsub{(T)}$ and $\rightsub{(U)}$.

Case 2: Next, we observe that since $U$ is a shift of $T$, the labels
in $\leftsub{(U)}$ are one larger than those in $\leftsub{(T)}$, and
so are $\leq \nlbl{(T)}$.  But $\nlbl{(T)}$ is less than all the
labels in $\rightsub{(T)}$.  So every label in $\leftsub{(U)}$ is less
than every label in $\rightsub{(T)}$.  So all the subtrees in
$\leftsub{(U)}$ clash with all the subtrees $\rightsub{(T)}$.
Similarly, all the subtrees in $\rightsub{(U)}$ clash with those in
$\leftsub{(T)}$.

Case 3: Next we show that $T$ clashes with every $S \in \subbrn{U}$.
\iffalse
First, $T$ clashes with $U$ because $\nlbl{(U)} = \nlbl{(T)}+1$.  Now
suppose $S$ is the proper subtree of $U$ with the same label as $T$.
We can assume wlog that $S \in \leftsub{(U)}$.
\fi

Since $U$ is a search tree, $\nlbl{(S)} \leq \nlbl{(U)} =
\nlbl{(T)}+1$.  

If $S$ is a leaf, then it clashes with $T \in \brnchng$.  Otherwise,
$S$ is a branching subtree of $U$ by definition of search tree, and
\[
\nlbl{(\rightsub{(S)})} < \nlbl{(S)} \leq \nlbl{(T)} +1 \leq \nlbl{(\rightsub{(T)})},
\]
proving that $T$ and $S$ clash.  A symmetrical argument shows that $U$
clashes with the subtree of $T$ with the same label.

This proves that all the subtrees of $T$ clash with all the subtrees
of $U$, that is, $P(T)$.
\end{proof}

\end{solution}

\eparts

\end{problem}

%%%%%%%%%%%%%%%%%%%%%%%%%%%%%%%%%%%%%%%%%%%%%%%%%%%%%%%%%%%%%%%%%%%%%
% Problem ends here
%%%%%%%%%%%%%%%%%%%%%%%%%%%%%%%%%%%%%%%%%%%%%%%%%%%%%%%%%%%%%%%%%%%%%

\endinput
