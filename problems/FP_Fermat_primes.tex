\documentclass[problem]{mcs}

\begin{pcomments}
  \pcomment{FP_Fermat_primes}
  \pcomment{ARM 5/9/14}
\end{pcomments}

\pkeywords{
  number_theory
  modular_arithmetic
  Fermat_theorem
  Z_n
}

%%%%%%%%%%%%%%%%%%%%%%%%%%%%%%%%%%%%%%%%%%%%%%%%%%%%%%%%%%%%%%%%%%%%%
% Problem starts here
%%%%%%%%%%%%%%%%%%%%%%%%%%%%%%%%%%%%%%%%%%%%%%%%%%%%%%%%%%%%%%%%%%%%%

\begin{problem}
Suppose $p=2^r+1$ is prime for some positive integer $r$.  For
example, $r=1,2,4$ have this property.  Prove that $r$ must always be
a power of 2.

\hint Verify that the order of 2 in $\Zmod{p}$ is $2r$.

\begin{solution}
We have $2^r = p-1$ by definition, and $p-1 = -1 \inzmod{p}$, so the
successive positive powers of 2 up to the $2r$th power in $\Zmod{p}$
are
\[
2^1,2^2,2^3,\dots, 2^r, -2, -2^2, \dots, -(2^r) = -(-1) = 1 \inzmod{p}
\]
and only the last power equals 1 in $\Zmod{p}$.  This shows that the
order of 2 in $\Zmod{p}$ is $2r$.  But by Fermat's Little Theorem, the
order of any element in $\Zmod{p}$ divides $p-1$.  We conclude that
\[
2r \divides 2^r,
\]
which implies that $r$ divides of power of 2, and so must itself be a
power of 2.

Numbers of the form $2^{2^k}+1$ are called \emph{Fermat numbers}, so
we can rephrase the conclusion above as saying that any prime of the
form $2^r+1$ must actually be a Fermat number.  The Fermat numbers are
prime for $k=1,2,3,4$, but not for $k=5$.  In fact, it is not known if
any Fermat number with $k > 4$ is prime.

Another proof\footnote{Adapted from Wikipedia, \emph{Fermat number},
  May 9, 2014.} (that does not follow the hint) is by contradiction:
suppose $r = mn$ and $n>1$ is odd.  Now $2^m = -1 \inzmod{2^m+1}$, so
\[
2^r = (2^m)^n = (-1)^n = -1 \inzmod{2^m+1},
\]
and therefore $p = 2^r+1 = 0 \inzmod{2^m+1}$.  That is, $p$ has a
factor $2^m+1 < p$, and therefore is not prime.

\end{solution}


\end{problem}

%%%%%%%%%%%%%%%%%%%%%%%%%%%%%%%%%%%%%%%%%%%%%%%%%%%%%%%%%%%%%%%%%%%%%
% Problem ends here
%%%%%%%%%%%%%%%%%%%%%%%%%%%%%%%%%%%%%%%%%%%%%%%%%%%%%%%%%%%%%%%%%%%%%

\endinput
