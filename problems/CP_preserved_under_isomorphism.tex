\documentclass[problem]{mcs}

\begin{pcomments}
\  \pcomment{CP_preserved_under_isomorphism}
  \pcomment{overlaps FP_graphs_short_answer, FP_multiple_choice_unhidden}
  \pcomment{by ARM 3/29/13 for first simple graph lecture}
\end{pcomments}

\pkeywords{
  simple_graph
  isomorphism
  vertices
  preserved
  degree
  edge
  path
}

%%%%%%%%%%%%%%%%%%%%%%%%%%%%%%%%%%%%%%%%%%%%%%%%%%%%%%%%%%%%%%%%%%%%%
% Problem starts here
%%%%%%%%%%%%%%%%%%%%%%%%%%%%%%%%%%%%%%%%%%%%%%%%%%%%%%%%%%%%%%%%%%%%%
\begin{problem}
Which simple-graph properties below are preserved under graph
isomorphism?

\bparts

\ppart The vertices can be numbered 1 through 7.

\ppart There is a cycle that includes all the vertices.

\begin{staffnotes}
If asked, explain that simple graph cycles can be defined in the
essemtailly same way as for digraphs.  The only diff is that going
back and forth on the same edge---a length 2 ``cycle''---is not
considered to be a cycle.
\end{staffnotes}

\ppart There are two degree 8 vertices.

\ppart\label{eqlen} Two edges are of equal length.

%\ppart There are exacty two spanning trees.

\ppart No matter which edge is removed, there is a path between any two
  vertices.

\ppart There are two cycles that do not share any vertices. % too easy

%\ppart There are two connected components. % too easy

\ppart\label{edgesubset} One edge is a subset of another one.

\begin{staffnotes}
Always false, so is preserved.
\end{staffnotes}

\ppart The graph can be pictured in a way that all the edges have the same length.

%\ppart The graph is 4-colorable. % too easy

%\ppart Adding an edge between any two vertices creates a cycle. % spanning tree one better

\ppart The $\QOR$ of two properties that are preserved under isomorphism.

\ppart The negation of a property that is preserved under isomorphism.

\eparts

\begin{solution}
All but~\eqref{eqlen} and~\eqref{edgesubset} are preserved.
\end{solution}

\end{problem}

\endinput
