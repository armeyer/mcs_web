\documentclass[problem]{mcs}

\begin{pcomments}
  \pcomment{MQ_uncountable_infinite_sequences}
  \pcomment{from: S10 by wing, revised 3/19/10 by ARM}
  \pcomment{special case of PS_N_to_A_diagonal_argument}
\end{pcomments}

\pkeywords{
  Russells_paradox
  uncountable
}

%%%%%%%%%%%%%%%%%%%%%%%%%%%%%%%%%%%%%%%%%%%%%%%%%%%%%%%%%%%%%%%%%%%%%
% Problem starts here
%%%%%%%%%%%%%%%%%%%%%%%%%%%%%%%%%%%%%%%%%%%%%%%%%%%%%%%%%%%%%%%%%%%%%

\begin{problem}
  
  Let $[\naturals\to \set{1,2,3}]$ be the set of all
  sequences containing only the numbers 1, 2, and 3, for example,
\begin{align*}
(1, 1, 1, 1 ...),\\
(2, 2, 2, 2 ...),\\
(3, 2, 1, 3 ...).
\end{align*}
  For any sequence, $s$, let $s[m]$ be its $m$th element.  

  Prove that $[\naturals \to \set{1,2,3}]$ is \idx{uncountable}.
  
  \hint Suppose there was a list
\[
\mathcal{L} = \text{sequence}_0, \text{sequence}_1, \text{sequence}_2,
\dots
\]
of sequences in $[\naturals\to \set{1,2,3}]$ and show that there is a
``diagonal'' sequence $\text{diag} \in [\naturals \to \set{1,2,3}]$
that does not appear in the list.  Namely,
  \[
  \text{diag} \eqdef r(\text{sequence}_0[0]), r(\text{sequence}_1[1]),
  r(\text{sequence}_2[2]), \dots,
  \]
where $r:\set{1,2,3}\to \set{1,2,3}$ is some function such that $r(i)
\neq i$ for $i=1,2,3$.

\iffalse
  \hint Try to find a total injective relation from the power set
  $\power(\naturals)$ to $[\naturals\to \set{1,2,3}]$.
\fi

\begin{solution}
  
  \begin{proof}
  Assume for the purpose of contradiction that $[\naturals \to \set{1,2,3}]$
  is countable. This means that $\naturals$ is at least as big as the set
  $[\naturals \to \set{1,2,3}]$ (since they are both countable) and there 
  must therefore exist some surjective function $\sigma$ mapping $\naturals$ to 
  $[\naturals \to \set{1,2,3}]$.
  
  To show that $\sigma$ is in fact not a surjection, let us define a sequence
  $\text{diag}$ such that:
  
  \[
  \text{diag} \eqdef r(\text{sequence}_0[0]), r(\text{sequence}_1[1]),
  r(\text{sequence}_2[2]), \dots,
  \]
  where $r:\set{1,2,3}\to \set{1,2,3}$ is some function such that $r(i)
  \neq i$ for $i=1,2,3$.

  So, by definition,
  \[ \forall a \in \naturals. \text{diag}[a] \neq \text{sequence}_a[a] \]
  In other words, diag is a sequence that is not in the image $\naturals \sigma$,
  contradicting our assumption of $\sigma$ being a surjective function.
  \end{proof}
  
  \begin{proof}
  Alternatively, we may also find a total injective relation from the power
  set $\power(\naturals)$ to $[\naturals\to \set{1,2,3}]$. To do so, we can
  define a function that maps each subset of $\naturals$ to a sequence that 
  represents the existence of each integer in the subset:
  
  \[ f(S) \eqdef r_S(0), r_S(1), \dots, \]
  where $r_S:\naturals \to \set{1,2,3}$ is defined as:
  \[
    r_S(n) \eqdef \begin{cases} 1 \text{ if } n \in S,\\
                                2 \text{ otherwise }.
                  \end{cases}
  \]
  
  For instance, f({0,1,4,5,6}) = (1,1,2,2,1,1,1,2,2...2).
  
  $f$ is by definition a total function that is injective since only a subset of 
  infinite sequences are mapped (ones that include only 1 and 2s). Also note 
  that each sequence is mapped at most once since each sequence that only includes
  1 and 2s uniquely identifies one subset of $\naturals$.
  
  We have shown that there is at least one total injective relation from 
  $\power(\naturals)$ to $[\naturals\to \set{1,2,3}]$. Since we know that 
  the power set of any set is strictly bigger than that set, this also shows that
  $[\naturals\to \set{1,2,3}]$ is strictly bigger than $\naturals$ and is
  therefore uncountable.
    
  \end{proof}
  
 
\end{solution}

\end{problem}

\endinput
