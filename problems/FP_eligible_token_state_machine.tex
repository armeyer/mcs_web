\documentclass[problem]{mcs}

\begin{pcomments}
  \pcomment{FP_eligible_token_state_machine}
  \pcomment{ARM 3/10/16}
\end{pcomments}

\pkeywords{
  state_machine
  invariant
  preserved_invariant
  reachable
  induction
  remainder
}

%%%%%%%%%%%%%%%%%%%%%%%%%%%%%%%%%%%%%%%%%%%%%%%%%%%%%%%%%%%%%%%%%%%%%
% Problem starts here
%%%%%%%%%%%%%%%%%%%%%%%%%%%%%%%%%%%%%%%%%%%%%%%%%%%%%%%%%%%%%%%%%%%%%

\begin{problem}

\emph{Token replacing} is a single player game using a set of tokens,
each colored black or white.  In each move, a player can replace a
black token with three white tokens, or replace a white token with
three black tokens.  We can model this game as a state machine whose
states are pairs $(b,w)$ of nonnegative integers, where $b$ is the
number of black tokens and $w$ the number of white ones.

The game has two possible start states: $(5,4)$ or $(4,3)$.

We call a state $(b,w)$ \emph{eligible} when
\begin{align}
\rem{b-w}{4} & = 1, \QAND\\
\min\set{b,w} & \geq 3.
\end{align}
This problem examines the connection between eligible states and
states that are \emph{reachable} from either of the possible start
states.

\bparts

\ppart Give an example of a reachable state that is not eligible.\hfill\examrule
\examspace[0.3in]

\begin{solution}
The state $(b,w) = (7,2)$ is directly reachable from the start state
$(4,3)$, but it is not eligible because $\min\set{7,2} < 3$
\end{solution}

\ppart Show that the derived variable $b+w$ is strictly increasing.
Conclude that state $(3,2)$ is not reachable.

\examspace[0.8in]
    
\begin{solution}
Each transition increases $b+w$ by $2$, so $b+w$ is strictly
increasing.  Also, $b+w$ for the start states is $\geq 7$, so all
reachable states have $b+w \geq 7$. But $3+2 < 7$, so $(3,2)$ is not
reachable.
\end{solution}
  
\ppart\label{bwebg6} Suppose $(b,w)$ is eligible and $b \geq 6$.
Verify that $(b-3,w+1)$ is eligible.

\begin{solution}
We have $w \geq 3$ and $\rem{b-w}{4} = 1$ since $(b,w)$ is eligible.
Therefore,
\begin{align*}
\min\set{{b-3,w+1}} & \geq 3
        & \text{since}\ b \geq 6,\\
\rem{(b-3) - (w+1)}{4} & = \rem{(b-w) -4}{4} = \rem{b-w}{4} = 1,
\end{align*}
which proves that $(b-3,w+1)$ is eligible.
\end{solution}

\eparts

\medskip

For the rest of the problem, you may---and should---\textbf{assume}
the following Fact:
\begin{fact*}
If $\max{(b,w)} \leq 5$ and $(b,w)$ is eligible, then
$(b,w)$ is reachable.
\end{fact*}
(This is easy to verify since there are only nine states with $(5 \geq
b,w \geq 3)$, but don't waste time verifying.)

\bparts

\ppart\label{indhypP} Define the predicate $P(n)$ to be:
\[
\forall (b,w). [b+w = n \QAND\ (b,w)\ \text{is eligible}] \QIMPLIES (b,w)\ \text{is reachable}.
\]
Prove that $P(n-1) \QIMPLIES P(n+1)$ for all $n \geq 1$.

\examspace[2.5in]

\begin{solution}
\begin{proof}
To verify $P(n+1)$, we must show that for any $(b,w)$ with $b+w = n+1$,
if $(b,w)$ is eligible, then $(b,w)$ is reachable.

The Fact means we may assume $\max\set{b,w} \geq 6$.
There are two cases, $b \geq 6$ and $w \geq 6$.

In the case that $b \geq 6$, part~\eqref{bwebg6} implies that
\[
(b-3,w+1) \text{ is eligible}.
\]
Moveover $(b-3)+(w+1) = (b+w-2) = n-1$.  Now $P(n-1)$ implies that
$(b-3,w+1)$ is reachable.  But $(b-3,w+1)$ transitions to $(b,w)$ in
one step, proving that $(b,w)$ is reachable.

In the case that $w \geq 6$, the same reasoning above shows that
$(b+1, w-3)$ is reachable, and therefore $(b,w)$ is reachable.

So in any case, $(b,w)$ is reachable, as required.
\end{proof}
\end{solution}

\ppart Conclude that all eligible states are reachable.

\examspace[0.3in]

\begin{solution}
This follows by Strong Induction, with the Fact above giving the base
cases, and part~\eqref{indhypP} proving the induction step.
\end{solution}

\ppart Prove that $(4^7 +3 , 4^5 +1)$ is \emph{not} reachable.

\examspace[1.0in]

\begin{solution}
Note that the derived variable $\rem{b-w}{4)}$ is a constant: if there
is a transition from $(b,w)$ to $(b',w')$, then $(b'-w') = (b-w) \pm
4$, which implies that $\rem{b'-w'}{4} = \rem{b-w}{4}$.

This derived variable equals 1 for both start states and therefore
equals 1 for all states reachable fromeither of these start states.

On the other hand, this derived variable equals 2 for the state $(4^7 +3 , 4^5 +1)$:
\[
\rem{4^7 +3 - (4^5 +1)}{4} = \rem{(4^7 - 4^5) + 2}{4} = 2.
\]
which means $(4^7 +3 , 4^5 +1)$ is not reachable.
\end{solution}

\eparts

\end{problem}

\endinput
