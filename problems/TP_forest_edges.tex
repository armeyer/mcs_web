\documentclass[problem]{mcs}

\begin{pcomments}
  \pcomment{TP_forest_edges}
  \pcomment{subsumed by problem PS_forestsize}
  \pcomment{ARM 4/11/14; revised 11/10/17}
\end{pcomments}

\pkeywords{
  trees
  graphs
  connected
  component
  vertices
  egdes
}

%%%%%%%%%%%%%%%%%%%%%%%%%%%%%%%%%%%%%%%%%%%%%%%%%%%%%%%%%%%%%%%%%%%%%
% Problem starts here
%%%%%%%%%%%%%%%%%%%%%%%%%%%%%%%%%%%%%%%%%%%%%%%%%%%%%%%%%%%%%%%%%%%%%
\begin{problem}
Prove that if $G$ is a forest, then %make it IFF?
\begin{equation}\tag{v=e+c}
\card{\vertices{G}} = \card{\edges{G}} + \card{\text{components of $G$}}.
\end{equation}

\begin{solution}
Each connected component of $G$ is a tree.  In a tree
\begin{equation}\tag{v=e+1}
\card{\vertices{T}} = \card{\edges{T}} -1.
\end{equation}
by Theorem~\bref{thm:iffe=v-1}.

Summing the left and right-hand sides of equation~(v=e+1) over the
components of $G$ implies~(v=e+c).
\end{solution}

\end{problem}

%%%%%%%%%%%%%%%%%%%%%%%%%%%%%%%%%%%%%%%%%%%%%%%%%%%%%%%%%%%%%%%%%%%%%
% Problem ends here
%%%%%%%%%%%%%%%%%%%%%%%%%%%%%%%%%%%%%%%%%%%%%%%%%%%%%%%%%%%%%%%%%%%%%

\endinput
