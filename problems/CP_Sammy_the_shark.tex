\documentclass[problem]{mcs}

\begin{pcomments}
  \pcomment{PS_Sammy_the_shark} 
  \pcomment{from S07, pset7}
\end{pcomments}

\pkeywords{
  asymptotics
  geometric_sum
  geometric_series
}

%%%%%%%%%%%%%%%%%%%%%%%%%%%%%%%%%%%%%%%%%%%%%%%%%%%%%%%%%%%%%%%%%%%%%
% Problem starts here
%%%%%%%%%%%%%%%%%%%%%%%%%%%%%%%%%%%%%%%%%%%%%%%%%%%%%%%%%%%%%%%%%%%%%
\begin{problem}
Sammy the Shark is a financial service provider who offers loans on
the following terms.

\begin{itemize}

\item Sammy loans a client $m$ dollars in the morning.  This puts the
client $m$ dollars in debt to Sammy.

\item Each evening, Sammy first charges a service fee which
  increases the client's debt by $f$ dollars, and then Sammy charges
  interest, which multiplies the debt by a factor of $p$.  For
  example, Sammy might charge a ``modest'' ten cent service fee and
  1\% interest rate per day, and then $f$ would be $0.1$ and $p$ would
  be $1.01$.

\end{itemize}

\bparts

\ppart What is the client's debt at the end of the first day?

\examspace[0.5in]

\begin{solution}At the end of the first day, the client owes
  Sammy $(m + f) p = m p + f p$ dollars.
  \end{solution}

\ppart What is the client's debt at the end of the second day?

\examspace[0.5in]

\begin{solution}
\[
((m + f) p + f) p = m p^2 + f p^2 + f p
\]
\end{solution}

\ppart Write a formula for the client's debt after $d$ days and find
an equivalent closed form.

\begin{solution}
The client's debt after three days is
%
\[
(((m + f) p + f) p + f) p  =  m p^3 + f p^3 + f p^2 + f p.
\]
%
Generalizing from this pattern, the client owes
%
\[
m p^d + \sum_{k=1}^d fp^k
\]
%
dollars after $d$ days.  Applying the formula for a geometric sum gives:
%
\[
m p^d + f \cdot \paren{\frac{p^{d+1} - 1}{p - 1} - 1}
\]
\end{solution}

\ppart If you borrowed \$10 from Sammy for a year, how much would you
owe him?

\begin{staffnotes}
Calculator use expected.
\end{staffnotes}

\begin{solution}
\textbf{\$749.35}

\begin{verbatim}
(define (sammy m p d f)
  (+ (* m (expt p d))
     (* f  (- (/ (- (expt p (+ d 1)) 1) (- p 1)) 1))))

(sammy 10 1.01 365 .1)

;Value: 749.3470300910349
\end{verbatim}
\end{solution}
\eparts

\end{problem}


%%%%%%%%%%%%%%%%%%%%%%%%%%%%%%%%%%%%%%%%%%%%%%%%%%%%%%%%%%%%%%%%%%%%%
% Problem ends here
%%%%%%%%%%%%%%%%%%%%%%%%%%%%%%%%%%%%%%%%%%%%%%%%%%%%%%%%%%%%%%%%%%%%%

\endinput
