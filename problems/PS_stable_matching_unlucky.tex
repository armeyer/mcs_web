%PS_stable_matching_unlucky.tex

\documentclass[problem]{mcs}

\begin{pcomments}
  \pcomment{PS_stable_matching_unlucky}
  \pcomment{from: F08.ps4; F03.ps4; F00.ps5; revised by ARM 10/15/09 based in part F08}
  \pcomment{revise after October 09 using F08 and Justin Zhang's suggestion}
\end{pcomments}

\pkeywords{
  stable_matching
  state_machines
  termination
  partial_correctness
  invariant
}

\providecommand{\boys}{\text{Boys}}
\providecommand{\girls}{\text{Girls}}

%%%%%%%%%%%%%%%%%%%%%%%%%%%%%%%%%%%%%%%%%%%%%%%%%%%%%%%%%%%%%%%%%%%%%
% Problem starts here
%%%%%%%%%%%%%%%%%%%%%%%%%%%%%%%%%%%%%%%%%%%%%%%%%%%%%%%%%%%%%%%%%%%%%


\begin{problem}
  In a stable matching between $n$ boys and girls produced by the Mating
  Ritual, call a person \term*{lucky} if they are matched up with one of
  their $\ceil{n/2}$ top choices.  We will prove:
\begin{theorem*}%\label{luckyperson}
  There must be at least one lucky person.
\end{theorem*}

To prove this, define the following derived variables for the Mating Ritual:
\begin{description}
\item $r(B) = j$, when the boy $B$ is courting the $j$th girl
on his list.

\item $r(G)$ is the number of boys that girl $G$ has rejected.
\end{description}

\begin{problemparts}

\ppart\label{Ssame}
Let
\begin{equation}\label{Seqsumb}
S \eqdef \sum_{B \in \boys} r(B) - \sum_{G \in \girls} r(G).
\end{equation}
Show that $S$ remains the same from one day to the next in the Mating
Ritual.

\begin{solution}
  Suppose the Mating Ritual is under way.  If the girl $G$ has $m>1$
  suitors then she will reject $m-1$ of them, so $r(G)$ will increase by
  $m-1$.  But $r(B)$ increases by 1 for each of these rejected suitors,
  $B$, so the right hand side of~\eqref{Seqsumb} will be the same
  tomorrow.
\end{solution}

\ppart Prove the Theorem above.                        %~\ref{luckyperson}
(You may assume for simplicity that $n$ is even.)

\hint A girl is sure to be lucky if she has rejected half the boys.

\begin{solution}
Note that on the first morning, the $n$ boys are courting their first
choice girl and the girls haven't rejected anyone yet.  So $S=n$.
By part~\eqref{Ssame}, $S$ remains equal to $n$ every day.

Now suppose to the contrary that no person is lucky.  Since boy $B$ is, by
definition, lucky iff $r(B) \leq \ceil{n/2}$, we have
\[
r(B) \geq \ceil{\frac{n}{2}} + 1
\]
for all boys, $B$.

Now the rank of a girl's suitor will be higher than the ranks of any she
has rejected, so she will certainly be lucky if she has rejected at least
$\ceil{n/2}$ suitors, so
\[
r(G)  \leq \ceil{\frac{n}{2}} - 1,
\]
for all girls, $G$.

It follows that $2n \leq \sum_{B \in \boys} r(B) - \sum_{G \in \girls}
r(G) = S$, contradicting the fact that $S=n$.
\end{solution}

\end{problemparts}

\end{problem}


%%%%%%%%%%%%%%%%%%%%%%%%%%%%%%%%%%%%%%%%%%%%%%%%%%%%%%%%%%%%%%%%%%%%%
% Problem ends here
%%%%%%%%%%%%%%%%%%%%%%%%%%%%%%%%%%%%%%%%%%%%%%%%%%%%%%%%%%%%%%%%%%%%%


\endinput
