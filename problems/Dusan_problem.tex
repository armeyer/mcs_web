\documentclass[12pt]{article}

\title{Problem Proposal for the Final Exam} \author{Dusan
  Milijancevic, MAy 18, 2013}

\begin{document}
\maketitle

\textbf{1. (Pigeonhole principle)} The aim of this problem is to prove
that there exist a natural number $n$ such that $3^n$ has at least
$2013$ consecutive zeros in its decimal expansion.\\

\textbf{a)} Prove that there exist a natural number $n$ such that
$3^n\equiv 1 \; \; mod \; \; 10^{2014}$.\\

\textit{Hint:} Use pigeonhole principle or Euler's theorem.\\

\textbf{Solution:} There are two ways to solve this problem as
indicated in the hint.\\

Pigeonhole principle. Let $A=\{3^i; i=0,...,10^{2014}\}$. At least two
number from $A$ have the same reminder when divided by $10^{2013}$ by
the pigeon hole principle. Call them $3^i$ and $3^j$ for $i>j$. Then
$10^{2014}|3^j(3^{i-j}-1)$, so $10^{2014}|3^{i-j}-1$. Hence for
$n=i-j$, $3^n\equiv 1 \; \; mod \; \; 10^{2014}$.\\

Euler's theorem. As $gcd(3,10^{2014})=1$ for $n=\phi(10^{2014})$ we
have $3^{n}\equiv \; \; 1 \; \; mod \; \; 10^{2014}$.\\

 

\textbf{b)} Conclude that there exist a natural number $n$ such that
$3^n$ has at least $2013$ consecutive zeros.\\

\textbf{Solution:} For $n$ in part a) we have

$$3^n=***...\underbrace{000....000}_{2013}1,$$

so $3^n$ has at least 2013 consecutive zeros.



\end{document}
