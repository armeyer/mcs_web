\documentclass[problem]{mcs}

\begin{pcomments}
    \pcomment{CP_conditional_convergence}
    \pcomment{F01.ps7}
    \pcomment{Christos Kapoutsis, S02}
\end{pcomments}

\pkeywords{
  sum
  absolute
  convergence
  series
  reorder
  Riemann
  conditional_convergence
}

\begin{problem}
An infinite sum of nonnegative terms will converge to the same
value---or diverge---no matter the order in which the terms are
summed.  This may not be true when there are an infinite number of
both nonnegative and negative terms.  An extreme example is
\[
\sum_{i=0}^{\infty} (-1)^i = 1 + (-1) + 1 + (-1) + \cdots
\]
because by regrouping the terms we can deduce:
\begin{align*}
[ 1 + (-1) ] + [ 1 + (-1) ] + \cdots = 0 + 0 + \cdots & = 0,\\
1 + [(-1) + 1] + [(-1) + 1] + \cdots  = 1 + 0 + 0 + \cdots & = 1.
\end{align*}

The problem here with this infinite sum is that the sum of the first
$n$ terms oscillates between 0 and 1, so the sum does not approach any
limit.

But even for convergent sums, rearranging terms can cause big changes
when the sum contains positive and negative terms.  To illustrate the
problem, we look at the Alternating Harmonic Series:
\[
1 - 1/2 + 1/3 - 1/4 + \cdots \pm.
\]
A standard result of elementary calculus, \cite{Apostol67}, p.403, is
that this series converges to $\ln 2$, but things change if we reorder
the terms in the series.

Explain for example how to reorder terms in the Alternating Harmonic
Series so that the reordered series converges to 7.  Then explain how
to reorder so it diverges.

\begin{solution}
Note that the sum of the negative terms goes to negative infinity, because
the sum
\[
-1/2 - 1/4 - 1/6 -  \dots - 1/2n = -H_n/2 \sim (-\ln n)/ 2
\]
and $\ln n$ goes to infinity with $n$.  Similarly, the sum of the positive
terms goes to positive infinity.

To reorder the terms to converge to 7---or any number $r \in
\reals$---begin by taking the positive terms in decreasing order until
their sum first exceeds $r$.  Next, take the negative terms in order
of decreasing magnitude until the sum of all the terms
taken---positive and negative---first goes below $r$.  Now take the
remaining positive terms in decreasing order until the sum again
exceeds $r$, then the remaining negative terms till the sum goes below
$r$, and so on.  It is always possible to move the sum above or below
$r$ in this way, because the remaining positive terms sum to positive
infinity and the remaining negative terms sum to negative infinity.
Also, the running sum of the positive and negative terms taken in this
way goes above and below $r$ at most by the magnitude of the last term
taken.  Since the magnitude of the successive terms goes to zero, it
follows that the sum goes above and below $r$ by an amount that
decreases to zero.  But this is exactly what it means to say the sum
converges to $r$.

Finally, by repeatedly taking a single remaining negative terms
followed by enough remaining positive terms to increase the current
sum by one, yields an ordering in which the sum diverges to positive
infinity.

This argument applies to \emph{any} convergent series whose sum of
positive terms and whose sum of negative terms each diverge---a fact
known as \emph{Riemann's Series Theorem}, \cite{Apostol67}, p.413.
The moral: when reordering terms in a series, be sure to verify
convergence of either the positive terms or the negative terms.
\end{solution}

\end{problem}

\endinput
