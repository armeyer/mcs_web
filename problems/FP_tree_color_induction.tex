\documentclass[problem]{mcs}

\begin{pcomments}
  \pcomment{FP_tree_color_induction} \pcomment{ARM 4/8/14}
  \pcomment{simplification of probability question on final F12}
  \pcomment{soln slightly edited, ARM 4/20/17}
\end{pcomments}

\pkeywords{
  coloring
  tree
  induction
}

%%%%%%%%%%%%%%%%%%%%%%%%%%%%%%%%%%%%%%%%%%%%%%%%%%%%%%%%%%%%%%%%%%%%%
% Problem starts here
%%%%%%%%%%%%%%%%%%%%%%%%%%%%%%%%%%%%%%%%%%%%%%%%%%%%%%%%%%%%%%%%%%%%%

\begin{problem}
Using colors red, white and blue, prove \emph{by induction} that there
are $3\cdot 2^{n-1}$ different 3-colorings\inhandout{\footnote{That
    is, an assignment of one of three possible colors to each vertex
    so that no two adjacent vertices are assigned the same color.}} of
any tree with $n$ vertices.

\inhandout{
As usual, carefully indicate your
\begin{itemize}
\item induction hypothesis
\item base case(s)
\item induction step.
\end{itemize}
}

\begin{solution}
\begin{proof}
By induction on the number of vertices $n$.  The induction hypothesis is

$P(n) \eqdef$ For all $n$-vertex trees $T$ there are $3\cdot 2^n$ different 3-colorings of $T$.

\inductioncase{Base case}: ($n=1$).  There are $3 = 3 \cdot 2^{1-1}$
ways to color one vertex.

\inductioncase{Induction step}: Let $T$ be a tree with $n+1$ vertices
for some $n \geq 1$.  Let $v$ be a leaf of $T$.  Then removing $v$ and
its incident edges, we obtain a tree $T-v$ with $n$ vertices.

We may assume by induction that there are $3 \cdot 2^{n-1}$
3-colorings of $T-v$.  Each such 3-ooloring of $T-v$ can be extended
to a 3-ooloring of $T$ by assigning $v$ to have one of the two colors
that differs from the color of its father.  So there are $2 (3 \cdot
2^{n-1}) = 3 \cdot 2^n$ colorings of $T$, which proves $P(n+1)$.
\end{proof}

\end{solution}

\end{problem}

%%%%%%%%%%%%%%%%%%%%%%%%%%%%%%%%%%%%%%%%%%%%%%%%%%%%%%%%%%%%%%%%%%%%%
% Problem ends here
%%%%%%%%%%%%%%%%%%%%%%%%%%%%%%%%%%%%%%%%%%%%%%%%%%%%%%%%%%%%%%%%%%%%%

\endinput
