\documentclass[problem]{mcs}

\begin{pcomments}
\end{pcomments}

\pkeywords{
  strong induction
  induction
}

%%%%%%%%%%%%%%%%%%%%%%%%%%%%%%%%%%%%%%%%%%%%%%%%%%%%%%%%%%%%%%%%%%%%%
% Problem starts here
%%%%%%%%%%%%%%%%%%%%%%%%%%%%%%%%%%%%%%%%%%%%%%%%%%%%%%%%%%%%%%%%%%%%%

\begin{problem}
The purpose of this question is to prove the validity of strong induction using regular induction.  If $P$ is a predicate on the non-negative integers, then strong induction can be expressed as the following implication:
\begin{equation}
\label{strong induction}
\left\{ \forall n. \; [\forall k . \; k < n \implies P(k)] \implies P(n) \right\} \implies \forall m. \; P(m)
\end{equation}

Prove, by induction, the validity of \eqref{strong induction}.

\smallskip

\hint Let $P(n)$ be an arbitrary predicate over the non-negative integers and assume that the left half of the \eqref{strong induction} holds using this predicate.
Using that assumption, prove, by induction, that the right half of \eqref{strong induction} holds. The inductive hypothesis for that proof, $H(n)$, will have to be in terms of $P(n)$.

\examspace[8in]

\begin{solution}

\begin{proof}

Let $P(n)$ be an arbitrary predicate over the non-negative integers, and assume that the left half of \eqref{strong induction} holds for this predicate.  Define an inductive hypothesis $H(n)$ by
\[
H(n) ::= \forall m. \; m \leq n \implies P(m)
\]

\textbf{Base Case:} Setting $n=0$ makes \mbox{$[\forall k . \; k < n \implies P(k)] \implies P(n)$} logically equivalent to $P(0)$ (because no $k$ is less than 0), which is equivalent to $H(0)$.

\textbf{Inductive Step:} Assume $H(n)$, that is \mbox{$\forall m. \; m \leq n \implies P(m)$}.
$H(n)$ implies \mbox{$\forall k. \; k < n+1 \implies P(k)$}, then using the left side of $\eqref{strong induction}$, we get $P(n+1)$.  So we have $H(n) \text{ AND } P(n+1)$, which implies $H(n+1)$.

\end{proof}

Having proved the base case and inductive step for $H(n)$, we get, by induction, that $\forall n. \; H(n)$.  That is \mbox{$\forall n. \forall m. \; m < n \implies P(n)$}, but this is equivalent to $\forall m. \; P(m)$.  So the right side of $\eqref{strong induction}$ holds.

\end{solution}

\end{problem}

%%%%%%%%%%%%%%%%%%%%%%%%%%%%%%%%%%%%%%%%%%%%%%%%%%%%%%%%%%%%%%%%%%%%%
% Problem ends here
%%%%%%%%%%%%%%%%%%%%%%%%%%%%%%%%%%%%%%%%%%%%%%%%%%%%%%%%%%%%%%%%%%%%%

\endinput
 
