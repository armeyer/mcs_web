\documentclass[problem]{mcs}

\begin{pcomments}
  \pcomment{CP_remainder_repeat}
  \pcomment{similar to CP_remainder_computation_ practice}
  \pcomment{by Stephan Boyer 9/26/13, edited ARM}
\end{pcomments}

\pkeywords{
  number_theory
  modular_arithmetic
  exponent
  remainder
}

%%%%%%%%%%%%%%%%%%%%%%%%%%%%%%%%%%%%%%%%%%%%%%%%%%%%%%%%%%%%%%%%%%%%%
% Problem starts here
%%%%%%%%%%%%%%%%%%%%%%%%%%%%%%%%%%%%%%%%%%%%%%%%%%%%%%%%%%%%%%%%%%%%%

\begin{problem}
Compute the remainder
\[
\rem{2498754270^{184638463} \cdot 9234673466^{844759364}}{22},
\]
carefully explaining the steps in your computation and their
rationale.  (A correct explanation will get nearly full credit even if
there is an arithmetic error in the final answer.)

\hint $18^9 \equiv 8 \pmod {22}$
\begin{solution}
First, we replace the bases of the exponents with their remainders:
\[
\rem{12^{184638463} \cdot 18^{844759364}}{22}.
\]
Now, let's examine the remainders of the first few powers of $12$.

\begin{align*}
  \rem{12}{22} &= 12 \\
  \text{rem}{12^2}{22} &= 12 \\
  \rem{12^3}{22} &= 12 \\
  \vdots
\end{align*}

The remainder is always $12$.  So the original expression has the same
remainder as
\[
\rem{12\cdot 18^{844759364}}{22}.
\]

Now let's look at the remainders of powers of $18$.
\begin{align*}
  \rem{18}{22} &= 18 \\
  \rem{18^2}{22} &= 16 \\
  \rem{18^3}{22} &= 2 \\
  \rem{18^4}{22} &= 14 \\
  \rem{18^5}{22} &= 10 \\
  \rem{18^6}{22} &= 4 \\
  \rem{18^7}{22} &= 6 \\
  \rem{18^8}{22} &= 20 \\
  \rem{18^9}{22} &= 8 \\
  \rem{18^{10}}{22} &= 12 \\
  \rem{18^{11}}{22} &= 18 \\
  \cdots
\end{align*}

So it repeats after $11$ steps.  Therefore,
\[
\rem{18^{844759364}}{22} = \rem{18^{(11 \cdot 76796305) +9}}{22} = \rem{18^{9}}{22},
\]
and we conclude that the original term has remainder
\[
  \rem{12 \cdot 18^9}{22} = \rem{(12)(8)}{22} = 8.
\]

\end{solution}
\end{problem}


%%%%%%%%%%%%%%%%%%%%%%%%%%%%%%%%%%%%%%%%%%%%%%%%%%%%%%%%%%%%%%%%%%%%%
% Problem ends here
%%%%%%%%%%%%%%%%%%%%%%%%%%%%%%%%%%%%%%%%%%%%%%%%%%%%%%%%%%%%%%%%%%%%

\endinput
