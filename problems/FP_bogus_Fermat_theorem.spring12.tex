\documentclass[problem]{mcs}

\begin{pcomments}
  \pcomment{FP_bogus_Fermat_theorem}
  \pcomment{renamed from FP_Fermat_theorem_bogus}
  \pcomment{from S09.final.prob6 by ARM 1/31/12}
  \pcomment{formatted ARM 3/13/12}
\end{pcomments}

\pkeywords{
  modular_arithmetic
  Euler_function
  Euler_theorem
  Fermat_theorem
  remainder
  bogus_proof
}

%%%%%%%%%%%%%%%%%%%%%%%%%%%%%%%%%%%%%%%%%%%%%%%%%%%%%%%%%%%%%%%%%%%%%
% Problem starts here
%%%%%%%%%%%%%%%%%%%%%%%%%%%%%%%%%%%%%%%%%%%%%%%%%%%%%%%%%%%%%%%%%%%%%

\begin{problem}

 There is a mistake in the following proof,
 where all the congruences are taken with modulus 29:
  \begin{align}
    35^{86}
    & \equiv 6^{86} 
       & \text{(since $35 \equiv 6 \pmod {29} $)} \\
    & \equiv 6^{28} 
       & \text{(since $86 \equiv 28 \pmod{29} $)}\label{bug} \\
    & \equiv 1 
       & \text{(by Fermat's Little Theorem)}
  \end{align}

  Identify the exact line containing the mistake and explain the 
  logical error.

 \examspace[2in]

\begin{solution}
The mistake occurs at line~\eqref{bug}.

Exponents can be replaced by their remainders on division by $28$,
not on division by 29.  So the ``explanation'' that $86 \equiv 28
\pmod {29}$ on the third line is true, but does not justify that
mistaken step.
\end{solution}

\end{problem}

\endinput

