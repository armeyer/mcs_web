\documentclass[problem]{mcs}

\begin{pcomments}
  \pcomment{CP_bogus_asymptotics_proof}
  \pcomment{formerly CP_false_asymptotics_proof}
  \pcomment{from: S09.cp9t}
\end{pcomments}

\pkeywords{
  asymptotics
  induction
  false_proof
}

%%%%%%%%%%%%%%%%%%%%%%%%%%%%%%%%%%%%%%%%%%%%%%%%%%%%%%%%%%%%%%%%%%%%%
% Problem starts here
%%%%%%%%%%%%%%%%%%%%%%%%%%%%%%%%%%%%%%%%%%%%%%%%%%%%%%%%%%%%%%%%%%%%%

\begin{problem}
\begin{falseclm*}
\begin{equation}\label{2n1}
2^n = O(1).
\end{equation}
\end{falseclm*}

Explain why the claim is false.  Then identify and explain the mistake in
the following bogus proof.

\begin{bogusproof} The proof is by induction on $n$ where the induction
hypothesis $P(n)$ is the assertion~\eqref{2n1}.

\textbf{base case:}  $P(0)$ holds trivially.

\textbf{inductive step:} We may assume $P(n)$, so there is a constant $c >0$
such that $2^n \leq c \cdot 1$.  Therefore,
\[
2^{n+1} = 2 \cdot 2^n \leq (2c) \cdot 1,
\]
which implies that $2^{n+1} = O(1)$.  That is, $P(n+1)$ holds, which
completes the proof of the inductive step.

We conclude by induction that $2^n = O(1)$ for all $n$.  That is, the
exponential function is bounded by a constant.

\end{bogusproof}

\begin{solution}
The mistake in proof the hinges on a misinterpretation of
equation~\eqref{2n1}.  To begin with, asymptotic relations are
relations between \emph{functions}.  When we write $O(1)$, we really
mean ``big-Oh of the constant function whose value is 1.''  That is,
if we define $c_a$ to be the constant function equal to $a$:
\[
c_a(x) \eqdef a,\qquad \text{for all $x$}.
\]
we should really have phrased~\eqref{2n1} as:
\begin{equation}\label{2nOc1}
2^n = O(c_1).
\end{equation}

But we still have the same issue with $2^n$ in~\eqref{2nOc1}.  Does it
refer to a constant that is a power of two, or does it refer to the
exponential function?  Now~\eqref{2nOc1} is intended to refer to the
exponential function.  That is, it means
\begin{equation}\label{expOc1}
\text{exp} = O(c_1),
\end{equation}
where exp is the function given by
\[
\text{exp}(n) \eqdef 2^n.
\]
This intended interpretation~\eqref{expOc1} is false, but the bogus
proof claims to verify it.

The blunder is in misreading~\eqref{2nOc1} as though it meant
\begin{equation}\label{c2nOc1}
\forall n \in \nngint.\quad c_{2^n} = O(c_1).
\end{equation}
Assertion~\eqref{c2nOc1} is true, but uninteresting since all positive
constant functions are $O()$ of each other.  This is not what
\emph{not} what~\eqref{2n1} was intended to mean.

\iffalse
A function on the nonnegative integers is $O(1)$ iff it is bounded by
a constant.\footnote{Things are a bit more complicated for functions
  on the positive real numbers: $1/x$ is $O(1)$ since it is bounded by
  1 for $x \geq 1$, but no constant bounds it for all $x >0$.}  Since
the function $2^n$ grows unboundedly with $n$, it is not $O(1)$.
\fi

So the mistake in the bogus proof is in its misinterpretation
of~\eqref{2n1} as~\eqref{c2nOc1}.  The bogus proof then is a
\emph{correct} silly proof by an unnecessary induction of the
true uninteresting assertion~\eqref{c2nOc1}.  Then in the last line,
the bogus proof switches from the misinterpretation~\eqref{c2nOc1} to
the intended interpretation~\eqref{expOc1}.

\iffalse
  The intended
interpretation of~\eqref{2n1} is
\begin{equation}\label{f=exp}
\text{Let $f$ be the function defined by the rule $f(n) \eqdef 2^n$.  Then
$f = O(1)$.}
\end{equation}
But the bogus proof treats~\eqref{2n1} as an assertion $P(n)$ about $n$.
Namely, it misinterprets~\eqref{2n1} as meaning:
\begin{quote}
  Let $f_n$ be the constant function equal to $2^n$.  That is, $f_n(k)
  \eqdef 2^n$ for all $k \in \nngint$.  Then
\begin{equation}\label{fn=c}
f_n = O(1).
\end{equation}
\end{quote}
Now~\eqref{fn=c} is true since every constant function is $O(1)$, \fi

It would be reasonable to say that the exact place where the bogus
proof goes wrong is in its first line, where it defines $P(n)$ based
on misinterpretation~\eqref{c2nOc1}.  But we would describe this a
(very serious) \emph{strategic} mistake, but not yet a specific
\emph{mathematical} mistake because the induction proof is correct.
The exact place where the bogus proof makes a mathematical mistake is
in its last line, when it switches from the
misinterpretation~\eqref{c2nOc1} and mistakenly claims to have proved
the false assertion~\eqref{expOc1}.
\end{solution}

\end{problem}

%%%%%%%%%%%%%%%%%%%%%%%%%%%%%%%%%%%%%%%%%%%%%%%%%%%%%%%%%%%%%%%%%%%%%
% Problem ends here
%%%%%%%%%%%%%%%%%%%%%%%%%%%%%%%%%%%%%%%%%%%%%%%%%%%%%%%%%%%%%%%%%%%%%
