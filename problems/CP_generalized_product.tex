\documentclass[problem]{mcs}

\begin{pcomments}
  \pcomment{CP_generalized_product}
  \pcomment{from: S07 cp10w}
\end{pcomments}

\pkeywords{
  generalized_product
  product_rule
}

%%%%%%%%%%%%%%%%%%%%%%%%%%%%%%%%%%%%%%%%%%%%%%%%%%%%%%%%%%%%%%%%%%%%%
% Problem starts here
%%%%%%%%%%%%%%%%%%%%%%%%%%%%%%%%%%%%%%%%%%%%%%%%%%%%%%%%%%%%%%%%%%%%%

\begin{problem} Answer the following quesions using the Generalized
Product Rule.

\bparts

\ppart{Next week, I'm going to get really fit!  On day 1, I'll
exercise for 5 minutes.  On each subsequent day, I'll exercise 0, 1,
2, or 3 minutes more than the previous day.  For example, the number
of minutes that I exercise on the seven days of next week might be 5,
6, 9, 9, 9, 11, 12.  How many such sequences are possible?}

\begin{solution}
The number of minutes on the first day can be selected in 1 way.
The number of minutes on each subsequent day can be selected in 4 ways.
Therefore, the number of exercise sequences is $1 \cdot 4^6$ by the
generalized product rule.
\end{solution}

\ppart An \term{$r$-permutation} of a set is a sequence of $r$
distinct elements of that set.  For example, here are all the
2-permutations of $\set{a, b, c, d}$:
%
\[
\begin{array}{ccc}
(a, b) & (a, c) & (a, d) \\
(b, a) & (b, c) & (b, d) \\
(c, a) & (c, b) & (c, d) \\
(d, a) & (d, b) & (d, c)
\end{array}
\]
%
How many $r$-permutations of an $n$-element set are there?  Express
your answer using factorial notation.

\begin{solution}There are $n$ ways to choose the first element, $n - 1$ ways to
choose the second, $n - 2$ ways to choose the third, \dots, and
$n - r + 1$ ways to choose the $r$-th element.  Thus, there are:
\[
n \cdot (n-1) \cdot (n-2) \cdots (n - r + 1)  =  \frac{n!}{(n - r)!}
\]
$r$-permutations of an $n$-element set.
\end{solution}

\ppart How many $n \times n$ matrices are there with \emph{distinct}
entries drawn from $\set{1, \ldots, p}$, where $p \geq n^2$?

\begin{solution}There are $p$ ways to choose the first entry,
$p - 1$ ways to choose the second for each way of choosing the first,
$p - 2$ ways of choosing the third, and so forth.  In all there are
%
\[
p (p - 1) (p - 2) \cdots (p - n^2 + 1) = \frac{p!}{(p - n^2)!}
\]
%
such matrices.  Alternatively, this is the number of
$n^2$-permutations of a $p$ element set, which is $p! / (p - n^2)!$.
\end{solution}

\eparts

\end{problem}

\endinput
