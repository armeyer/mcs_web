\documentclass[problem]{mcs}

\begin{pcomments}
\pcomment{PS_max_algorithm_expectation}
\pcomment{F95.ps11}
\pcomment{edited by ARM 5/9/12}
\end{pcomments}

\pkeywords{
 probability
 expectation
 maximum
 harmonic
 indicator_variable
}


%%%%%%%%%%%%%%%%%%%%%%%%%%%%%%%%%%%%%%%%%%%%%%%%%%%%%%%%%%%%%%%%%%%%%
% Problem starts here
%%%%%%%%%%%%%%%%%%%%%%%%%%%%%%%%%%%%%%%%%%%%%%%%%%%%%%%%%%%%%%%%%%%%%

\begin{problem}
We are given a random vector of $n$ distinct numbers.  We then
determine the maximum of these numbers by the following procedure:

Pick the first number.  Call it the ``current maximum''.  Go through
the rest of the vector (in order) and each time we find a number $x$
that exceeds our ``current maximum,'' we update the ``current maximum''
with $x$.

What is the expected number of times we update the ``current maximum''?

\hint Let $X_i$ be the indicator variable for the event that the $i$th
element in the vector is larger than all the previous elements.

\begin{solution}
Let's fix the $n$ numbers we are given.  We can assume that we are
given a random permutation of the $n$ numbers.  For $i \in [1,n]$, let
$X_i$ be the indicator variable for the event that the $i$th element
in the vector is larger than all the previous elements.

Note that the number of times we update the current maximum is
precisely $X_1 + \cdots + X_n$.  Since expectation is a linear
operator, we can compute $\expect{X_1 + \cdots + X_n}$ by finding
$\expect{X_i}$ for each $i$ and summing them up.  

Since $X_i$ is an indicator, we only have to find $\prob{X_i=1}$.  In
a random permutation, this happens with probability $1/i$.
Why?  If you take $i$ distinct numbers and randomly permute them, the
probability that the largest one occupies the last (or any) position
is $1/i$.  Now
\[
\prob{X_i = 1} = \sum_{n_1,\ldots,n_i} \prcond{X_i = 1}{\mbox{first
    $i$ elements are $n_1, \dots, n_i$})\cdot \prob(\mbox{first $i$
    elements are $n_1$,\dots,$n_i$)},
\]
and since we now know that the conditional probability term is $1/i$,
we can conclude that $\prob{X_i = 1} = 1/i$.  Thus
\begin{align*}
\expect{X} & =  \sum_i P(X_i=1)\\
& = \sum_{i=1}^n 1/i \\
& =  H_n \approx \ln n,
\end{align*}
where $H_n$ is the $n$th Harmonic number.
\end{solution}

\end{problem}

%%%%%%%%%%%%%%%%%%%%%%%%%%%%%%%%%%%%%%%%%%%%%%%%%%%%%%%%%%%%%%%%%%%%%
% Problem ends here
%%%%%%%%%%%%%%%%%%%%%%%%%%%%%%%%%%%%%%%%%%%%%%%%%%%%%%%%%%%%%%%%%%%%%

\endinput
