\documentclass[problem]{mcs}

\begin{pcomments}
  \pcomment{from: MQ_count_double_deck}
  \pcomment{from: F03.Quiz2}
\end{pcomments}

\pkeywords{
  counting
  division_rule
  card
}

%%%%%%%%%%%%%%%%%%%%%%%%%%%%%%%%%%%%%%%%%%%%%%%%%%%%%%%%%%%%%%%%%%%%%
% Problem starts here
%%%%%%%%%%%%%%%%%%%%%%%%%%%%%%%%%%%%%%%%%%%%%%%%%%%%%%%%%%%%%%%%%%%%%

\begin{problem}

\begin{problemparts}

\problempart
  Suppose that two identical 52-card decks are mixed together.  Write a
  simple expression for the number of arrangements of the 104 cards that 
could
  possibly result from such a mixing.

\examspace[3in]

\begin{solution}
In the mixed deck, there are precisely two copies of each of 52 distinct cards.
By the Bookkeeper Rule and the definition of multinomial coefficients, the 
number of possible arrangements of cards in the mixed deck is therefore just 
%  If the cards were all distinct, there would be $104!$ ``distinct" mixes.
%  Call two distinct mixes \emph{similar} if they differ only in that the
%  positions of the two distinct copies of the same card are exchanged.  So
%  two distinct mixes correspond to the same double-deck mix iff they are
%  similar.  Therefore, the map from distinct mixes to double-deck mixes is
%  $(2!)^{52}$ to one, and so by the division rule, there are
\[
\frac{104!}{(2!)^{52}}.
\]
\end{solution}

\problempart
Using only integers from the interval $[1,n]$, how many strictly increasing length-$m$ sequences can be formed?

\begin{solution}
\[
\binom{n}{m}
\]

\textbf{Justification:} Given any $m$-element subset, $S$, of 
$\set{1,2,...,n}$, 
listing its elements in increasing order yields a unique sequence, 
$s$, that is strictly increasing and has length $m$.  Collecting the terms of $s$ in a set yields a unique set, which is just $S$.
Thus there is a bijection between the set of all strictly increasing length-$m$ sequences with terms drawn from $\set{1,2,...,n}$
and the set of all size-$m$ subsets of $\set{1,2,...,n}$.  

%This could also be done via bijection with $0^{a_1} 1 0^{a_2} 1$ \dots $0^{a_m} 1 0^k$ with $k$ chosen so the number of 0's is $n-m$.

\end{solution}

\end{problemparts}

\end{problem}

%%%%%%%%%%%%%%%%%%%%%%%%%%%%%%%%%%%%%%%%%%%%%%%%%%%%%%%%%%%%%%%%%%%%%
% Problem ends here
%%%%%%%%%%%%%%%%%%%%%%%%%%%%%%%%%%%%%%%%%%%%%%%%%%%%%%%%%%%%%%%%%%%%%

\endinput

