\documentclass[problem]{mcs}

\begin{pcomments}
  \pcomment{CP_mating_ritual_proof}
  \pcomment{class or exam problem only: proof in book}
  \pcomment{from: S07.cp5m, S08.cp5f(?)}
\end{pcomments}

\pkeywords{
 stable_matching
 Mating_ritual
 invariant
}


%%%%%%%%%%%%%%%%%%%%%%%%%%%%%%%%%%%%%%%%%%%%%%%%%%%%%%%%%%%%%%%%%%%%%
% Problem starts here
%%%%%%%%%%%%%%%%%%%%%%%%%%%%%%%%%%%%%%%%%%%%%%%%%%%%%%%%%%%%%%%%%%%%%

\begin{problem}
A preserved invariant of the Mating Ritual is:
\begin{quote}
For every girl $G$ and every boy $B$, if $G$ is crossed off $B$'s
list, then $G$ has a favorite suitor whom she prefers more than $B$.
\end{quote}

Let Brad be some boy and Jen be any girl who is not his wife on the
last day of the Mating Ritual.  You may assume that everyone is
married on this last day.

Use the invariant to explain why Brad and Jen are not a rogue couple.
Conclude that the Mating Algorithm produces stable
marriages.\footnote{This proof was already given in the text.  We
   think that someone with a basic understanding of the Mating Ritual
   will be able to reconstruct the proof for themselves.  If you
   memorized that proof (we hope you didn't; memorization is not a
   sensible approach to learning 6.042 class material) or already
   copied it onto your crib sheet, then you have lucked out and will
   nail this question.}

\begin{solution}

\begin{proof}
To prove the claim, we consider two cases:

\emph{Case} 1.  Jen is not on Brad's list.  Then by the invariant, we
know that Jen prefers her husband to Brad.  So she's not going to run
off with Brad: the claim holds in this case.

\emph{Case} 2.  Otherwise, Jen is on Brad's list.  But since Brad
works his way down his preference list until he finds a wife, his wife
must be higher on his preference list than Jen.  This means that Brad
prefers his wife to Jen, so he's also not going to run off with Jen:
the claim also holds in this case.

So in any case, Brad will not be part of a rogue couple.  Since Brad
is an arbitrary boy, it follows that no boy is part of a rogue couple.
Hence the marriages on the last day are stable.
\end{proof}

\end{solution}

\end{problem}

\endinput
