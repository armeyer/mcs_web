\documentclass[problem]{mcs}

\begin{pcomments}
  \pcomment{TP_squareroot_of_prime}
  \pcomment{should be renamed TP_squareroot_size_factor}
  \pcomment{by ARM 2/1/11, tweaked 9/6/13}
\end{pcomments}

\pkeywords{
  contradiction
  primes
  square_root
  factor
}


%%%%%%%%%%%%%%%%%%%%%%%%%%%%%%%%%%%%%%%%%%%%%%%%%%%%%%%%%%%%%%%%%%%%%
% Problem starts here
%%%%%%%%%%%%%%%%%%%%%%%%%%%%%%%%%%%%%%%%%%%%%%%%%%%%%%%%%%%%%%%%%%%%%


\begin{problem}
Prove that if $a\cdot b =n$, then either $a$ or $b$ must be $\le
\sqrt{n}$, where $a,b$, and $n$ are nonnegative real numbers.  \hint
by contradiction, Section~\bref{contradiction_sec}.

\begin{solution}

\begin{proof}
Suppose to the contrary that $a > \sqrt{n}$ and $b > \sqrt{n}$.  Then
\[
a \cdot b > \sqrt{n} \cdot \sqrt{n} = n,
\]
contradicting the fact that $a\cdot b = n$.

Here we have used the rule that for nonnegative real numbers $u,v,w,x$,
\[
[u > v \QAND w > x] \QIMPLIES u\cdot w > v\cdot x.
\]
This rule applies since $a,b$, and $\sqrt{n}$ are nonnegative.

Note that we are using the fact that if $n$ is nonnegative, then
$\sqrt{n}$ denotes the nonnegative square root.  If we had used the
negative square root of $n$, the above rule would not hold, and the
proof would not be correct.  Indeed, if we used the negative square
root of 2 with $a=b=1$, the claim to be proved would be false.

\end{proof}

\end{solution}
\end{problem}

%%%%%%%%%%%%%%%%%%%%%%%%%%%%%%%%%%%%%%%%%%%%%%%%%%%%%%%%%%%%%%%%%%%%%
% Problem ends here
%%%%%%%%%%%%%%%%%%%%%%%%%%%%%%%%%%%%%%%%%%%%%%%%%%%%%%%%%%%%%%%%%%%%%

\endinput
