\documentclass[problem]{mcs}

\begin{pcomments}
  \pcomment{FP_cnf}
  \pcomment{not worth using--ARM}
  \pcomment{by Justin Zhang, for 2009 fall final}
  \pcomment{minor edit by ARM 9/29/14}
\end{pcomments}

\pkeywords{
  logic
  proposition
  clause
  normal_form
  conjunction
  literal
}

%%%%%%%%%%%%%%%%%%%%%%%%%%%%%%%%%%%%%%%%%%%%%%%%%%%%%%%%%%%%%%%%%%%%%
% Problem starts here
%%%%%%%%%%%%%%%%%%%%%%%%%%%%%%%%%%%%%%%%%%%%%%%%%%%%%%%%%%%%%%%%%%%%%

\begin{problem}
  ``In boolean logic, a formula is in \term{conjunctive normal form}
  (\term{CNF}) if it is a conjunction of clauses, where a clause is a
  disjunction of literals.''---Wikipedia (the following examples are
  also taken from Wikipedia).  A literal is either a boolean variable or
  the negation of a boolean variable. In other words:
  \[
   CNF = clause \QAND clause \QAND ... \QAND clause
   \]
   \[
    clause = literal \QOR literal \QOR ... \QOR literal 
   \]
   \[
    literal = variable | \QNOT(variable)
   \]

   For example, the following formulas are in CNF:

  \begin{align}
    (\neg A) \QAND (B \QOR C)  \nonumber \\
    (A \QOR B) \QAND (\neg B \QOR C \QOR D) \QAND (D \QOR \neg E)
    \nonumber \\
    (A) \QAND (B) \nonumber
  \end{align}

  The following formulas are not in CNF:
  \begin{align*}
    \QNOT(B \QOR C)\\  
    (A \QAND B) \QOR C \\
    A \QAND (B \QOR (D \QAND E))
  \end{align*}

  However, they are respectively equivalent to the following formulas
  in CNF:
  \begin{align*}
    \bar{B} \QAND \bar{C}\\
    (A \QOR C) \QAND (B \QOR C) \\
    (A) \QAND (B \QOR D) \QAND (B \QOR E)
  \end{align*}
  
\bparts
\ppart
Please tell which of the following are in CNF, where $x_i$ are boolean
variables:
  \begin{align*}
    (\bar{x_1} \QAND \bar{x_2}) \QOR x_3 \\
    (x_1 \QOR x_2) \QAND (x_3 \QOR x_4) \QAND (x_5 \QOR x_6 \QOR x_7)\\
    (x_6) \QAND (\neg x_6 \QOR x_7) \QAND \QNOT(\bar{x_7} \QAND \bar{x_6})
  \end{align*}

\begin{solution}
  not CNF;  CNF;  not CNF.
\end{solution}


\ppart If there are any formulas in part (a) that are not in CNF,
please choose one of them and convert it into CNF using whatever ways
you can (laws, truth table, etc.)

\begin{solution}
  We use distributive law to convert the first formula into CNF and De
  Morgan's law to convert the third. 
  \begin{align}
    (\neg x_1 \QAND \neg x_2) \QOR x_3  =  (\neg x_1 \QOR x_3) \QAND
    (\neg x_2 \QOR x_3)  \nonumber \\
    (x_6) \QAND (\neg x_6 \QOR x_7) \QAND \neg (\neg x_7 \QAND \neg x_6) = 
    (x_6) \QAND (\neg x_6 \QOR x_7) \QAND (x_7 \QOR x_6) \nonumber
  \end{align}

\end{solution}

\eparts

\end{problem}

%%%%%%%%%%%%%%%%%%%%%%%%%%%%%%%%%%%%%%%%%%%%%%%%%%%%%%%%%%%%%%%%%%%%%
% Problem ends here
%%%%%%%%%%%%%%%%%%%%%%%%%%%%%%%%%%%%%%%%%%%%%%%%%%%%%%%%%%%%%%%%%%%%%

\endinput
