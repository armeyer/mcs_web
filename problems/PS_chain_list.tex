\documentclass[problem]{mcs}

\begin{pcomments}
  \pcomment{PS_chain_list}
  \pcomment{Mildly edited from FP_chain_list by zabel, 10/19/17}
\end{pcomments}

\pkeywords{
  partial_order
  maximum
  linear
  chain
  comparable
}

%%%%%%%%%%%%%%%%%%%%%%%%%%%%%%%%%%%%%%%%%%%%%%%%%%%%%%%%%%%%%%%%%%%%%
% Problem starts here
%%%%%%%%%%%%%%%%%%%%%%%%%%%%%%%%%%%%%%%%%%%%%%%%%%%%%%%%%%%%%%%%%%%%%

\begin{problem} 
Let $R$ be a weak partial order on a set $A$.  Suppose $C$ is a
finite, nonempty chain.

\bparts

\ppart\label{hasmax} Prove that $C$ has a maximum element.  \hint Induction on the
size of $C$.
\examspace[3.0in]

\begin{solution}
As hinted, we give a proof by induction on the size of $C$.

\begin{proof}
The induction hypothesis is:
\begin{quote}
$P(n) :=$ If $C$ is a chain of size $n$, then $C$ has a maximum
  element.
\end{quote}

\inductioncase{Base case}: ($n=1$).  The one element is $C$ is the
maximum (ands also minimum) element, bym definition of maximum.

\inductioncase{Induction step}: To prove $P(n+1)$ for $n \geq 1$, let
$C_{n+1}$ be a chain of size $n+1$ and let $x$ be an arbitrary element
in $C_{n+1}$.  Then $C_{n+1}-\set{x}$ is a chain of size $n$, so it
has a maximum element $m$ by induction hypothesis.  Now compare $x$
and $m$.  If $x \mrel{R} m$, then $m$ is the maximum element in
$C_{n+1}$.  On the other hand, $m \mrel{R} x$, then (by transitivity
of $R$), $x$ is a maximum element of $C_{n+1}$.  In any case, 
$C_{n+1}$ has a maximum element, which proves $P(n+1)$.
\end{proof}

\end{solution}

\ppart Conclude that there is a unique sequence of all the elements of
$C$ that is strictly increasing.

\hint Induction on the size of $C$, using part~\eqref{hasmax}.

\examspace[3.0in]

\begin{solution}
As hinted, we give a proof by induction on the size of $C$.

\begin{proof}
The induction hypothesis is:
\begin{quote}
$Q(n) :=$ If $C$ is a chain of size $n$, then there is a unique
  sequence of all the elements of $C$ that is strictly increasing.
\end{quote}

\inductioncase{Base case}: ($n=1$).  Immediate.

\inductioncase{Induction step}: To prove $Q(n+1)$ for $n \geq 1$, let
$C_{n+1}$ be a chain of size $n+1$.  By part~\eqref{hasmax}, $C_{n+1}$
has a maximum element $m$.  Then $C_{n+1}-\set{m}$ is a chain of size
$n$, so there is a unique strictly increasing sequence,
$\overrightarrow{C_{n+1}-\set{m}}$, of all the elements of $C_{n+1}-\set{m}$.
Then $\overrightarrow{C_{n+1}-\set{m}}$ followed by $m$ is a strictly increasing
sequence of the elements of $C_{n+1}$.  Moreover, this sequence is
unique, because any strictly increasing sequence elements in $C_{n+1}$
can only consist of a strictly sequence of elements in
$C_{n+1}-\set{m}$, which is unique by hypothesis, followed by $m$.
\end{proof}

\end{solution}

\iffalse
\ppart Show that part~\ref{} may fail when $C$ is infinite: give an example of a weak partial order and some infinite chain $C$ with no maximum element.

\ppart Show that even when part~\ref{} holds, part~\ref{} may fail for infinite chains $C$. In particular, give an example of a weak partial order $R$ on a set $A$ an an infinite chain $C$ such that
\begin{itemize}
\item every nonempty chain (finite or infinite) in $R$ has a maximal element, but
\item there is no strictly increasing sequence $c_0 \prec c_1 \prec c_2 \prec \cdots$ of all of $C$'s elements.
\end{itemize}
\hint The negative integers.
\fi

\eparts

\end{problem} 


%%%%%%%%%%%%%%%%%%%%%%%%%%%%%%%%%%%%%%%%%%%%%%%%%%%%%%%%%%%%%%%%%%%%%
% Problem ends here
%%%%%%%%%%%%%%%%%%%%%%%%%%%%%%%%%%%%%%%%%%%%%%%%%%%%%%%%%%%%%%%%%%%%%

\endinput
