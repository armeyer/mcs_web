\documentclass[problem]{mcs}

\begin{pcomments}
  \pcomment{MQ_partial_order_on_12345}
  \pcomment{variation/improvement of MQ_partial_order_on_123}
  \pcomment{ARM 11/3/13}
\end{pcomments}

\pkeywords{
  partial_orders
  chain
  antichain
  parallel_time
  processors
}

%%%%%%%%%%%%%%%%%%%%%%%%%%%%%%%%%%%%%%%%%%%%%%%%%%%%%%%%%%%%%%%%%%%%%
% Problem starts here
%%%%%%%%%%%%%%%%%%%%%%%%%%%%%%%%%%%%%%%%%%%%%%%%%%%%%%%%%%%%%%%%%%%%%

\begin{problem} 

Suppose the precedence partial order on a set of unit time tasks was
isomorphic to the powerset, $\power([1,5])$, of the integers from 1 to 5
under the strict subset relation $\subset$.

\bparts

\ppart What is the minimum parallel time to complete these
tasks?\hfill\examrule[0.5in]

\begin{solution}
\textbf{6 time units}.

One maximum length chain is $\emptyset, \set{1}, \set{1,2}, \set{1,2,3},
\set{1,2,3,4}, \set{1,2,3,4,5}$.  There are altogether 120 such chains
determined by the permutations of $[1,5]$.
\end{solution}

\ppart Describe a maximum size antichain is this partial order.

\examspace[1in]

\begin{solution}
  The size 2 subsets of $[1,5]$ are a maximum antichain, and there are 10
  such subsets.

 The size 3 subsets are another maximum size antichain.
\end{solution}

\ppart Briefly explain why the minimum number of processors required to
complete these tasks in minimum parallel time is equal to the size of the
maximum antichain.

\examspace[2in]

\begin{solution}
We know the size of the maximum antichain is always sufficient.

No smaller number is possible in this case since all the size 2 subsets
must be scheduled at time 3 in a minimum time schedule because each of
them is the third element in a maximum size chain.
\end{solution}

\eparts

\end{problem}

%%%%%%%%%%%%%%%%%%%%%%%%%%%%%%%%%%%%%%%%%%%%%%%%%%%%%%%%%%%%%%%%%%%%%
% Problem ends here
%%%%%%%%%%%%%%%%%%%%%%%%%%%%%%%%%%%%%%%%%%%%%%%%%%%%%%%%%%%%%%%%%%%%%

\endinput
