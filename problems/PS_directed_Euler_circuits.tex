\documentclass[problem]{mcs}

\begin{pcomments}
  \pcomment{PS_directed_Euler_circuits}
  \pcomment{digraph version of PS_Euler_circuits}
  \pcomment{adapted to digraphs by ARM 3/7/11}
\end{pcomments}

\pkeywords{
  Euler
  Euler_circuit
  Euler_tours
  cycle
  degree
  closed_walk
  digraph
}

%%%%%%%%%%%%%%%%%%%%%%%%%%%%%%%%%%%%%%%%%%%%%%%%%%%%%%%%%%%%%%%%%%%%%
% Problem starts here
%%%%%%%%%%%%%%%%%%%%%%%%%%%%%%%%%%%%%%%%%%%%%%%%%%%%%%%%%%%%%%%%%%%%%

\begin{problem}
  In this problem we'll consider some special cycles in graphs called
  \term{Euler tours}, named after the famous mathematician Leonhard
  Euler.  (Same Euler as for the constant $e\approx 2.718$---he did a
  lot of stuff.)

\begin{definition}
  An Euler tour of a graph is a closed walk that includes every edge
  exactly once.
\end{definition}
\iffalse

Does the graph in the following figure contain an Euler tour?

\begin{figure}
\graphic{example}
\end{figure}

Well, if it did, the edge $(E, F)$ would need to be included.  If the path
does not start at $F$ then at some point it traverses edge $(E,F)$, and
now it is stuck at $F$ since $F$ has no other edges incident to it and an
Euler tour can't traverse $(E,F)$ twice.  But then the path could not
be a tour.  On the other hand, if the path starts at $F$, it must then
go to $E$ along $(E,F)$, but now it cannot return to $F$.  It again cannot
be a tour. This argument generalizes to show that if a graph has a
vertex of degree $1$, it cannot contain an Euler tour.

\begin{staffnotes}
On the other hand, it is easy to see that any cycle has an Euler
tour. You can just start at any vertex and walk around back to it.
\end{staffnotes}
\fi

So how do you tell in general whether a graph has an Euler tour?  At
first glance this may seem like a daunting problem (the similar sounding
problem of finding a cycle that touches every vertex exactly once is one
of those million dollar NP-complete problems known as the \term{Traveling
  Salesman Problem})---but it turns out to be easy.

\bparts

\ppart Show that if a graph has an Euler tour, then the in-degree
of each vertex equals its out-degree.

\begin{solution}
Let
\[
C \eqdef v_1\ \diredge{v_1}{v_2}\ v_2\ \dots\ \diredge{v_r}{v_1}\ v_1
\]
be an Euler tour.  Except for the initial and final occcurrences of
$v_1$, each occurrence of a vertex $v$ in the tour is immediately
preceded by an edge $\diredge{u}{v}$ and immediately followed by an
edge $\diredge{v}{w}$.  It follows that if $v\neq v_1$ occurs $s$
times in $C$, then $\indegr{v} = \outdegr{v} = s$ since every edge incident to $v$
occurs in $C$ exactly once.  For the same reason, if $v_1$ occurs $s$ times on the path, 
then $\indegr{v_1} = \outdegr{v_1} = s-1$.
\end{solution}
\eparts

A digraph is \term{weakly connected} if there is a ``path'' between
any two vertices that may follow edges backwards or forwards.  More
precisely, a digraph, $G$ is weakly connected iff there is a path from
each vertex to every other vertex in the digraph $G \union G^{-1}$.

In the remaining parts, we'll work out the converse: if a graph is
weakly connected and if the in-degree of every vertex equals its
out-degree, then the graph has an Euler tour.  To do this, let's
define an Euler \term{walk} to be a walk that traverses each edge
\emph{at most} once.

\bparts

\ppart\label{conn} Suppose that an Euler path in a connected graph
does not include every edge.  Explain why there must be an edge not on
the path whose head or tail is on the path.

\begin{solution}
  If an edge is not on the path but its head or tail is on the Euler
  path, that already is the desired edge.  So suppose there's an edge,
  $e$, not on the path and both ends of $e$ are not on the Euler path.
  Since there is a path in $G \union G^{-1}$ each endpoint of $e$ and
  a vertex on the Euler path, there must be a shortest path from an
  endpoint of $e$ to a vertex on the Euler path.  All the edges on
  this shortest path must also not be on the Euler path or the
  shortest could be shortened, so the last edge on the shortest path,
  or its reversal, will be the desired edge of $G$.
\end{solution}

\eparts

In the remaining parts, let $W$ be the \emph{longest} Euler path in
the graph.

\bparts

\ppart\label{cycle-circuit} Show that if $W$ is a cycle, then it must be
an Euler tour.

\hint part~\eqref{conn}

\begin{solution}
Suppose $W$ is a cycle and some edge is not in $W$.  By
part~\eqref{conn}, there must be a vertex $w$ in $W$ that is the head
or tail of an edge not in $W$.  Suppose $w$ is the tail of that edge.
Then starting at $w$, go around $W$ back to vertex $w$, and then the
follow the edge.  This makes a longer Euler path, contradicting the
maximality of $W$.  Similarly, if $w$ is the head of an edge not in
$W$, then starting at the tail of that edge, following it to $w$ and
then going around $W$ toget back to $w$ also makes a longer Euler
path, again contradicting maximality.

So no edge can be missing from $W$.
\end{solution}

\ppart\label{already} Explain why all the edges incident to the end of $W$
must be on $W$.

\begin{solution}
Otherwise we could extend $W$ to a longer Euler path with any edge
from the end not already in $W$.
\end{solution}

\ppart\label{odd} Show that if the end of $W$ was not equal to the start
of $W$, then the degree of the end would be odd.

\hint part~\eqref{already}

\begin{solution}
Let $v$ be the end vertex of $W$.  Given that $v$ is not the start of
$W$, it follows that at any occurrence of $v$ in $W$ other than at the
end, $W$ includes an edge whose head is $v$ that appears just before
that occurrence of $v$, and another edge whose tail is $v$ that
appears just after that occurrence of $v$.  Since $W$ is an Euler
path, all the edges in all these pairs are distinct.  In addition, the
final edge traversed by $W$ as it ends at $v$ is distinct from all the
paired edges.  Altogether, this imples that there are an odd number of
edges traversed by $W$ and incident to $v$.  But by
part~\eqref{already}, these are all the edges incident to $v$, proving
that $v$ has odd degree.
\end{solution}

\ppart Conclude that if every vertex of a finite, connected graph has even
degree, then it has an Euler tour.

\begin{solution}
If all vertices in $G$ have even degree, then by
  part~\eqref{odd}, the only possibility is that the end of $W$ equals the
  start, that is, $W$ is a cycle.  So by part~\eqref{cycle-circuit}, $W$ is
  an Euler tour.
\end{solution}

\eparts
\end{problem}

%%%%%%%%%%%%%%%%%%%%%%%%%%%%%%%%%%%%%%%%%%%%%%%%%%%%%%%%%%%%%%%%%%%%%
% Problem ends here
%%%%%%%%%%%%%%%%%%%%%%%%%%%%%%%%%%%%%%%%%%%%%%%%%%%%%%%%%%%%%%%%%%%%%

\endinput
