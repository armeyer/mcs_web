\documentclass[problem]{mcs}

\begin{pcomments}
  \pcomment{PS_union_bound_infinite}
  \pcomment{Subsumes PS_union_bound}
  \pcomment{ARM 4/23/15}
\end{pcomments}

\pkeywords{
  probability
  union_bound
  countable
  union
}

%%%%%%%%%%%%%%%%%%%%%%%%%%%%%%%%%%%%%%%%%%%%%%%%%%%%%%%%%%%%%%%%%%%%%
% Problem starts here
%%%%%%%%%%%%%%%%%%%%%%%%%%%%%%%%%%%%%%%%%%%%%%%%%%%%%%%%%%%%%%%%%%%%%

\begin{problem}
Prove the following probabilistic inequality, referred to as the
\emph{Union Bound}.

Let $A_1, A_2,\dots, A_n, \dots$ be events.  Then
\[
\Prob{\lgunion_{n\in\nngint}A_n} \leq \sum_{n\in\nngint} \prob{A_n}.
\]
\hint Replace the $A_n$'s by pairwise disjoint events and use the Sum Rule.

\begin{solution}
The trick is to convert the union of the $A_i$'s into a union of
disjoint events events $E_i$.  This is easy to do: just define $E_{i}$
to be the new elements that $A_{i}$ adds to the union of the earlier
$A_j$'s.  That is, define
\begin{align*}
E_0    & \eqdef A_0\\
E_{n+1} & \eqdef A_{n+1} - \lgunion_{i=0}^n A_n.
\end{align*}
So by definition,
\begin{align}
E_n & \subseteq A_n\label{ensuban}\\
\lgunion_{i = 0}^n {A_i} &  = \lgunion_{i = 0}^n {E_i},\label{finuEiAi}
\end{align}
and $E_0,E_1,\dots,E_n,\dots$ are pairwise disjoint events.

But if all the finite unions are equal, then so is the infinite union,
namely,

\begin{staffnotes}
Discuss: properties of finite unions don't always hold for infinite
unions.  For example, any finite union of finite sets is finite, but
infinite unions of finite sets certainly need not be finite.  So why
in this case does the equality of the finite unions imply equality of
the infinite unions?

The answer is that if there was an element in one infinite union and
not the other, that element must not be in some finite union, so the
finite unions would differ.
\end{staffnotes}

\begin{equation}\label{lgunionen=an}
\lgunion_{n\in\nngint} A_n = \lgunion_{n\in\nngint} E_n.
\end{equation}

Now we have
\begin{align*}
\Prob{\lgunion_{n\in\nngint} A_n}
  & = \Prob{\lgunion_{n\in\nngint} E_n}
          & \text{(by~\eqref{lgunionen=an})}\\
  & = \sum_{n\in\nngint} \prob{E_n} & \text{(Sum Rule)}\\
  & \leq \sum_{n\in\nngint} \prob{A_n}
          & \text{by~\eqref{ensuban}}.
\end{align*}
\end{solution}

\end{problem}

%%%%%%%%%%%%%%%%%%%%%%%%%%%%%%%%%%%%%%%%%%%%%%%%%%%%%%%%%%%%%%%%%%%%%
% Problem ends here
%%%%%%%%%%%%%%%%%%%%%%%%%%%%%%%%%%%%%%%%%%%%%%%%%%%%%%%%%%%%%%%%%%%%%

\endinput
