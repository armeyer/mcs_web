\documentclass[problem]{mcs}

\begin{pcomments}
  \pcomment{MQ_3_and_5_cent_stamps_by_induction}
  \pcomment{from: F09.mq3}
  \pcomment{similar to CP_10_and_15_cent_stamps_by_WOP but slightly different and uses induction}
\end{pcomments}

\pkeywords{
  induction
  strong_induction
}

%%%%%%%%%%%%%%%%%%%%%%%%%%%%%%%%%%%%%%%%%%%%%%%%%%%%%%%%%%%%%%%%%%%%%
% Problem starts here
%%%%%%%%%%%%%%%%%%%%%%%%%%%%%%%%%%%%%%%%%%%%%%%%%%%%%%%%%%%%%%%%%%%%%

\begin{problem}
   Prove that using only 3\textcent\ and 5\textcent\ stamps, every
   amount of postage $\geq 8\textcent$ can be made.

\hint Let $S(n)$ mean that exactly $n+8$\textcent\ postage can be paid
using only 5\textcent\ and 3\textcent\ stamps.
  
\begin{solution}
  
  \begin{proof}
    The following proof is by strong induction on $n$.
    
    We can begin by observing that the following postage amounts can 
    be made by 5 and 3 cent stamps:

    \begin{align*}
%    3 & = 3 \\
%    5 & = 5 \\
    8 & = 3 + 5 \\
    9 & = 3 + 3 + 3 \\
    10 & = 5 + 5
    \end{align*}

    The the three postage values $8\textcent, 9\textcent,
    10\textcent$ will be our base cases for the induction proof.

    \textbf{Base cases:} $S(8)$, $S(9)$ and $S(10)$ were shown to hold
    above.

    \textbf{Inductive step:} For all $n \geq 10$, we assume that
    $P(8), \dots, P(n)$ are true in order to prove that $P(n+1)$ is
    true.

    By the assumption that $P(n-2)$ is true, we know that the postage
    value $n - 2$ can be paid with 3\textcent\ and 5\textcent\ stamps.
    By adding one 3 cent stamp to that postage, we will be able to pay
    for a postage of $n - 2 + 3 = n + 1$ cents, showing that $P(n+1)$
    is true.  This completes the inductive step.  It follows by strong
    induction that $P(n)$ holds for all $n \geq 8$.
    
    We have therefore shown that all postage values of eight or more
    cents can be paid by 3\textcent\ and 5\textcent\ stamps.
  \end {proof}

\begin{editingnotes}  
  Therefore, we have shown that the postage amounts 3, 5, and any 
  $k \geq 8$ can be paid by 5 and 3 cent stamps.

  Note that we have actually seen a proof for this using WOP in the 
  \href{http://courses.csail.mit.edu/6.042/fall09/slides2m.pdf}{Well Ordering Principle}
  lecture.
\end{editingnotes}
\end{solution}

\end{problem}

%%%%%%%%%%%%%%%%%%%%%%%%%%%%%%%%%%%%%%%%%%%%%%%%%%%%%%%%%%%%%%%%%%%%%
% Problem ends here
%%%%%%%%%%%%%%%%%%%%%%%%%%%%%%%%%%%%%%%%%%%%%%%%%%%%%%%%%%%%%%%%%%%%%

\endinput
