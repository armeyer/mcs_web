\documentclass[problem]{mcs}

\begin{pcomments}
  \pcomment{PS_probabilistic_proof}
  \pcomment{\textbf{Round Robin Tournaments}}
  \pcomment{from S01.tut12}
  \pcomment{added to F02 repository by Tina Wang, S02}
  \pcomment{formatted by ARM 5/7/12}
\end{pcomments}

\pkeywords{
  probability
  probabilistic_method
  Boole
  Booles_inequality
  union_bound
  tournament
}

%%%%%%%%%%%%%%%%%%%%%%%%%%%%%%%%%%%%%%%%%%%%%%%%%%%%%%%%%%%%%%%%%%%%%
% Problem starts here
%%%%%%%%%%%%%%%%%%%%%%%%%%%%%%%%%%%%%%%%%%%%%%%%%%%%%%%%%%%%%%%%%%%%%
                                                                         
\begin{problem}           
The results of a round robin tournament in which every two people play
each other and one of them wins can be modelled a \term{tournament
  digraph}---a digraph with exactly one edge between each pair of
distinct vertices, but we'll continue to use the language of players
beating each other.

For $k \in [0,n)$, a tournament is $k$-neutral if for every set of $k$
  players, there is another player who beats them all.  For example,
  being 1-neutral is the same as not having a ``best'' player who
  beats everyone else.  This problem shows that for any fixed $k$, if
  $n$ is large enough, there will be a $k$-neutral tournament of $n$
  players.  We will do this by reformulating the question in terms of
  probabilities.  In particular, for any fixed $n$, we assign uniform
  probability to all $n$-vertex tournament digraphs.

\begin{problemparts}

\iffalse
\problempart
 numbering the sets of $k$ contestants.  How many such sets are
there?

\examspace[0.5in]

\begin{solution}
\[
\binom{n}{k}
\]
\end{solution}
\fi

\problempart For any set $S$ of players, let $B_S$ be the event that
no contestant beats everyone in $S$.  Express $\prob{B_S}$ in terms of
$n$ and $\card{S}$.

\examspace[1in]

\begin{solution}
The probability that a player outside $S$ beats everyone in $S$ is
$(1/2)^{\card{S}}$, so the probability they did not beat everyone in
the group is $1 - (1/2)^{\card{S}$.  There are $n-\card{S}$ people
  outside of the group. Thus, $B_S$ has probability
\[
\brac{1 - \paren{\frac{1}{2}}^k}^{n-\card{S}}
\]
\end{solution}

\problempart For $k \in [0,n)$, give an upper bound on
\[
\prob{\lgunion_{\card{S} = k} B_S}$.

\examspace[1in]

\begin{solution}
Use Boole's inequality:
\[
\Prob{\lgunion B_i} \leq \sum \Prob{B_i}.
\]

In other words, $\prob{\lgunion B_i}$ can be no greater than if all of
the $B_i$ are disjoint.  (Overlap will merely reduce the total
probability of the union).  Since the expression for $B_i$ does not
depend on $i$---the probability is the same for each $k$-sized group
$i$---the sum for all of the $B_i$ is simply
\[
\binom{n}{k} \brac{1 - \paren{\frac{1}{2}}^k }^{n-k}
\]
\end{solution}

\problempart
Explain why this result can be used to prove the existence of the desired
tournament outcome.

\examspace[1.5in]

\begin{solution}
Probabilistic proof.  If the overall probability of $\lgunion B_i$ is
less than 1, then there must be an outcome that is not in $\lgunion
B_i$.  It does not matter that we chose the probabilities arbitrarily;
the fact that there is \emph{any} positive probability at all means
some outcome that we had not accounted for is possible.

\medskip 
For interest, some numbers that work are:
\begin{align*}
k = 1, n = 3; k = 2, n = 21;  k = 3, n = 33; k = 4, n = 46; \\
k = 5, n = 59; k = 6, n = 72; k = 7, n = 85; k = 8, n = 98
\end{align*}
\end{solution}

\end{problemparts}
\end{problem}
                                 

%%%%%%%%%%%%%%%%%%%%%%%%%%%%%%%%%%%%%%%%%%%%%%%%%%%%%%%%%%%%%%%%%%%%%
% Problem ends here
%%%%%%%%%%%%%%%%%%%%%%%%%%%%%%%%%%%%%%%%%%%%%%%%%%%%%%%%%%%%%%%%%%%%%

\endinput
