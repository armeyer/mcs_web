%CP_smallest_infinite_set

\documentclass[problem]{mcs}

\begin{pcomments}
  \pcomment{from: Ch. ``sets'', notes problem, adapted by ARM 9/20/09}
  \pcomment{needs def of \surj in Appendix}
\end{pcomments}

\pkeywords{
  surjection
  as big as
  axiom of choice}

%%%%%%%%%%%%%%%%%%%%%%%%%%%%%%%%%%%%%%%%%%%%%%%%%%%%%%%%%%%%%%%%%%%%%
% Problem starts here
%%%%%%%%%%%%%%%%%%%%%%%%%%%%%%%%%%%%%%%%%%%%%%%%%%%%%%%%%%%%%%%%%%%%%

\begin{problem}\mbox{}

\begin{center}

\begin{quote}
\textbox{
\begin{lemma}\label{AUb}
  Let $A$ be a set and $b \notin A$.  If $A$ is infinite, then there is a
  bijection from $A \union \set{b}$ to $A$.
\end{lemma}

\begin{proof}
Here's how to define the bijection: since $A$ is infinite, it certainly has
at least one element; call it $a_0$.  But since $A$ is infinite, it has at
least two elements, and one of them must not be equal to $a_0$; call this
new element $a_1$.  But since $A$ is infinite, it has at least three
elements, one of which must not equal $a_0$ or $a_1$; call this new
element $a_2$.  Continuing in the way, we conclude that there is an
infinite sequence $a_0,a_1,a_2,\dots,a_n,\dots$ of different elements of
$A$.  Now we can define a bijection $f: A \union \set{b} \to A$:
\begin{align*}
f(b) & \eqdef a_0,\\
f(a_n) & \eqdef a_{n+1}  &\text{ for } n \in \naturals,\\
f(a) & \eqdef a & \text{ for } a \in A - \set{b,a_0,a_1,\dots}.
\end{align*}
\end{proof}
}

\end{quote}
\end{center}


\bparts

\ppart
Several students felt the proof of Lemma~\ref{AUb} was worrisome, if not
circular.  What do you think?

\begin{solution}
There is no ``solution'' for this discussion problem, since it depends on
what seems bothersome.

It may be bothersome that the proof asserts that $f$ is bijection without
spelling out a proof.  But the bijection property really does follow
directly from definition of $f$, so it shouldn't be much burden for a
bothered reader to fill in such a proof.

Another possibly bothersome point is that the proof assumes that if a set
is infinite, it must have more than $n$ elements, for every nonnegative
integer $n$.  But really that's the definition of infinity: a set is
finite iff it has $n$ elements for some nonnegative integer, $n$, and a
set is infinite iff it is \emph{not} finite.

A possibly worrisome point is how you find an element $a_{n+1} \in A$
given $a_0,a_1,\dots,a_n$.  But you don't have to \emph{find} a specific
one: there must be an element in $A -\set{a_0,a_1,\dots,a_n}$ ---so just
pick any one.  Actually, the justification for this step is the
set-theoretic Axiom of Choice described in the Notes chapter first-order
logic, and some logicians do consider it worrisome.
\end{solution}


\ppart Use the proof of Lemma~\ref{AUb} to show that if $A$ is an infinite
set, then there is surjective function from $A$ to $\naturals$, that is,
every infinite set is ``as big as'' the set of nonnegative integers.

\begin{solution}

  By the proof of Lemma~\ref{AUb}, there is an infinite sequence
  $a_0,a_1,a_2,\dots,a_n,\dots$ of different elements of $A$.  Then we can
  define a surjective function $f:A \to \naturals$ by defining
\[
f(a) \eqdef \begin{cases}
               n, & \text{if } a= a_n,\\
               \text{undefined}, & \text{otherwise}.
              \end{cases}
\]

---A total surjective function is not required, but if you want one define
$f':A \to \naturals$, by
\[
f'(a) \eqdef \begin{cases}
               n, & \text{if } a= a_n,\\
               a_0, & \text{otherwise}.
              \end{cases}
\]
\end{solution}

\eparts

\end{problem}

\endinput
