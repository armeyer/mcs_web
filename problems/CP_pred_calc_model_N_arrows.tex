\documentclass[problem]{mcs}

\begin{pcomments}
  \pcomment{CP_pred_calc_model_N_arrows}
  \pcomment{ARM 2/17/6}
\end{pcomments}

\pkeywords{
  predicate
  logic
  nonnegative
  model
  arrow
}

\newcommand{\nextel}{\emph{next}}
\newcommand{\prev}{\emph{prev}}
\newcommand{\numerl}[1]{\textbf{\~#1}}

%%%%%%%%%%%%%%%%%%%%%%%%%%%%%%%%%%%%%%%%%%%%%%%%%%%%%%%%%%%%%%%%%%%%%
% Problem starts here
%%%%%%%%%%%%%%%%%%%%%%%%%%%%%%%%%%%%%%%%%%%%%%%%%%%%%%%%%%%%%%%%%%%%%

\begin{problem}
Let $\numerl{0}$ be a constant symbol, $\nextel()$ and $\prev()$ be
function symbols taking one argument.

The aim of this problem is to develop a series of predicate formulas
using only these symbols whose models must contain contain a ``copy''
of the consecutive nonnegative integers $\naturals$.  Moreover, the
model must assign $\nextel()$ and $\prev()$ to be the plus-one and
minus-one functions on the copy.

Before we start, it helps to picture a model for these symbols by
representing each domain element $e$ as a point, with an arrow
labelled $\nextel$ going out of $e$ and going into $\nextel(e)$, and
another arrow labelled $\prev$ going out of $e$ and into $\prev(e)$.
The model must interpret the symbols $\next$ and $\prev$ as total
functions on the domain.  This means that
\begin{quote}
every point has exactly one $\next$-arrow, and exactly one
$\prev$-arrow, going out of it.
\end{quote}

Our first objective will be to find some elements that make up the
copy of $\naturals$.  We want them to look like a infinite sequence of
distinct points each of which has a $\next$-arrow going to the next
point in the sequence.  Since $\next$ is supposed to act like plus one
and $\prev$ like minus one, we require that if a $\next$-arrow goes
into an element $e$, then the $\prev$-arrow going out of $e$ goes back
to the beginning of the $\next$-arrow.  There is a simple way to
express this requirement as a predicate formula:
\begin{equation}\label{previnvnext}
\forall x.\, \prev(\nextel(x)) = x\ .
\end{equation}
Notice that if more than one $\next$-arrow went into an element, the
$\prev$-arrow out of a point could go back to the beginning of only to
one of them, which means the beginning of the other arrow would be a
counter-example to~\eqref{previnvnext}.  So~\eqref{previnvnext}
implies that at most one $\next$-arrow can go into any element.
Another way to say this is that $\next$-arrows that go out from
different points have to go into different points:
\begin{equation}\label{next1to1}
\forall x,y.\, \nextel(x) = \nextel(y)\ \QIMPLIES\ x = y.
\end{equation}

So we have just deduced that~\eqref{previnvnext}
implies~\eqref{next1to1}.  In other words
\[
\text{\eqref{previnvnext}} \QIMPIES\ \text{\eqref{next1to1}}
\]
is a valid formula.

The elements in the copy of $\naturals$ we are looking for are easy to
specify: they will be elements that correspond to starting at zero and
repeatedly adding one.  That is, the elements in our copy of
$\naturals$ will be obtained by starting at the $\numerl{0}$ element
then by following consecutive $\next$-arrows.  More precisely, lets
define some convenient abbreviations for some actual terms that use
the given symbols:
\begin{align*}
\numerl{1} & \eqdef \nextel(\numerl{0})\\
\numerl{2} & \eqdef \nextel(\numerl{1})\\
\numerl{3} & \eqdef \nextel(\numerl{2})\\
           &\vdots
\end{align*}
For example, $\numerl{3}$ is an abbreviation for an actual term:
\[
\nextel(\nextel(\nextel(\numerl{0}))).
\]

The values that a model gives to these numerals will be called its
\emph{numeral values}.  We will use the numeral values to be the
elements in the copy of $\naturals$ we are trying to create, with the
integer $n \in \naturals$ corresponding to the value of the numeral
$\numerl{n}$, and with $\next$ and $\prev$ acting like plus one and
minus one functions.

But this won't work yet.  There is nothing so far that prevents all
the numerals having the same value.  In other words, the formulas
above do not imply that $\next(x) \neq x$.  (You should check this
yourself right now.)  We might try to fix this by requiring that no
$\next$-arrow can begin and end at the same element, but there still
could be a pair of elements, each of which was ``plus one'' of the
other.  That is, the model might satisfy:
\[
\exists x, y.\, \nextel(x) = y \QAND \nextel(y) = x.
\]
We could go on to forbid a length two cycle of $\next$-arrows like this, but
then there might be a cycle of three:
\[
\exists x, y,z.\, \nextel(x) = y \QAND\ \nextel(y) = z \QAND\ \nextel(z) = x\ .
\]
It turns out that something we want to do anyway will fix this
problem.

\bparts

\ppart\label{zero-part} So far we haven't said anything that will make
the interpretion of $\numerl{0}$ behave like zero, namely, that zero
is not plus-one of anything.  Also, by convention, subtracting one
from zero has no effect.  Write a predicate formula expressing the
fact that the interpretion of $\numerl{0}$ has these properties.
\begin{solution}
\[
\QNOT(\exists x.\, \next(x) = \numerl{0}) \QAND
\prev(\numerl{0}) = \numerl{0}).
\]
\end{solution}
\eparts

At this point, we actually have ensured that any model satisfying just
the formulas~\eqref{previnvnext},~\eqref{zeronosuccessor}, and
part~\eqref{zero-part} must have the desired copy of $\naturals$.

\bparts

\ppart Explain why two different numerals must have different values
in any model satisfying
formulas~\eqref{previnvnext},~\eqref{zeronosuccessor}, and
part~\eqref{zero-part}.

\begin{solution}
There are a couple of ways to confirm this claim, one using arrows and
the other well-ordering and the given formulas.

An arrow explanation goes as follows:

\begin{proof}
Suppose we start at the point
denoted by $\numerl{0}$ and proceed to generate a sequence of points
by following successive $\next$-arrows out of each point and into the
next.  These points are, by definition, the values of all the
numerals.  But the formula of part~\eqref{zero-part} implies that no
no $\next$-arrow goes into the starting point.  Also, no $\next$-arrow
between points in the sequence can go into any other earlier point in
the sequence, because all the earlier nonzero points already a
$\next$-arrow going into it.  So each $\next$-arrow goes into a new
point and therefore no point has a repeat occurrence.
\end{proof}

A more formal argument can be based on the Well Ordering Principle.
Suppose to the contrary that two different numerals had the same
value.  There there is a minimum $n \in \naturals$ such that
$\numerl{n}$ has the same value as numeral $\numnerl{m}$ for some $m
> n$.

\iffalse
=========================
Since $m>0$, $\next(\numnerl{m-1}) = \numnerl{m}$ by definition.

But $n \neq 0$ because 


Since $\numerl{n}$ and $\numerl{m}$ have equal values so do
$\prev(\numerl{n})$ and $\prev(\numerl{mn})$.  But
now~\eqref{previnvnext} implies $\numerl{n}$ and $\numerl{m}$ have the
same interpretation.

\end{solution}

We also make $\prev$ act like the ``subtract one'' function on these
values, namely subtracting one from zero has no effect:
\begin{equation}\label{prev00}
\prev(\numerl{0})= \numerl{0},
\end{equation}
and otherwise adding one undoes subtracting one.





====================
At this point, we actually have ensured that any model satisfying just
the two formulas~\eqref{previnvnext} and~\eqref{zeronosuccessor} has
the desired copy of $\naturals$.






Now we will make $\nextel$ act like the ``add one'' function on values
of these terms.  In particular, adding one to two different elements
will result in two different elements:
\begin{equation}\label{next1to1}
\forall x,y.\, x \neq y \QIMPLIES\ \nextel(x) \neq \nextel(y)\ .
\end{equation}
We also make $\prev$ act like the ``subtract one'' function on these
values, namely subtracting one from zero has no effect:
\begin{equation}\label{prev00}
\prev(\numerl{0})= \numerl{0},
\end{equation}
and otherwise adding one undoes subtracting one.

\bparts

Write a predicate formula expressing the requirement that adding one
undoes subtracting one from nonzero elements.
\begin{solution}
\begin{equation}\label{previnvnext}
\forall x.\, \prev(\nextel(x)) = x\ .
\end{equation}




To start, we can introduce some abbreviations for certain terms called
\emph{numerals} built up using these symbols: namely, let
\begin{align*}
\numerl{1} & \eqdef \nextel(\numerl{0})\\
\numerl{2} & \eqdef \nextel(\numerl{1})\\
\numerl{3} & \eqdef \nextel(\numerl{2})\\
           &\vdots
\end{align*}

Now we will make $\nextel$ act like the ``add one'' function on values
of these terms.  In particular, adding one to two different elements
will result in two different elements:
\begin{equation}\label{next1to1}
\forall x,y.\, x \neq y \QIMPLIES\ \nextel(x) \neq \nextel(y)\ .
\end{equation}
We also make $\prev$ act like the ``subtract one'' function on these
values, namely subtracting one from zero has no effect:
\begin{equation}\label{prev00}
\prev(\numerl{0})= \numerl{0},
\end{equation}
and otherwise adding one undoes subtracting one.

\bparts

\ppart\label{previnvnext-part} Write a predicate formula expressing
the requirement that adding one undoes subtracting one from nonzero elements.
\begin{solution}
\begin{equation}\label{previnvnext}
\forall x\neq \numerl{0}.\, \nextel(\prev(x)) = x\ .
\end{equation}
\end{solution}

\iffalse
\ppart Write a formula expressing the fact that subtracting one from
two different nonzero elements will result in two different elements.

I was going to add ``Explain why this formula is implied the earlier
ones,'' but no time to check and problem long enough already.

\begin{solution}
\begin{equation}\label{prev1to1}
\forall x,y \neq \numerl{0}.\, x \neq y \QIMPLIES\ \prev(x) \neq \prev(y)\ .
\end{equation}
\end{solution}
\fi


\eparts

Now we are pretty much done: any interpretation that satisifies these
formulas must assign different domain elements to each of the
numerals, and interpretations of $\nextel$ and $\prev$ must act like
plus one and minus one on these elements.

But although the values of the numerals in every model satisfying the
formulas of~\eqref{next1to1}, \eqref{prev00}, and
part~\eqref{previnvnext-part} form a copy of $\naturals$, a model may
have other elements that lead to strange behavior.  For example, a
model satisfying all the above formulas could have \emph{two} copies
that act like $\naturals$, with $\nextel$ and $\prev$ acting like add
one and subtract one on each copy.  In this model, there would be two
elements that act like zero.

So let's fix this:
\bparts

\ppart Write a formula that forces there to be only one copy of
$\naturals$ in the model.

\begin{solution}
It's enough to say there is only one element that can start a copy:
\begin{equation}\label{uniqz}
\forall x.\, \prev(x) = x \QIMPLIES x = \numerl{0}.
\end{equation}
\end{solution}
\eparts

But there still might be other elements with funny properties.  For
example, there might be two elements, each of which was ``plus one''
of the other:
\[
\exists x, y.\, \nextel(x) = y \QAND \nextel(y) = x.
\]
or a cycle of three:
\[
\exists x, y,z.\, \nextel(x) = y \QAND\ \nextel(y) = z \QAND\ \nextel(z) = x\ .
\]
We could easily write a formula forbidding such cycles of length
three, or forbidding cycles of any given length, but we would need an
infinite number of formulas to forbid cycles of all lengths.

To forbid all finite cycles using only a fixed number of formulas, we
will need something further: we will allow a relation symbol
$L(\ ,\ )$ and make it act like a less-than relation.  For this
purpose, it's suggestive to write $x\, L\, y$ instead of $L(x,y)$.
Then one requirement will be that for all $n,m \in \naturals$, if $n <
m$, then $\numerl{n}\, L\, \numerl{m}$ must be true.  But we also want
$L$ to define a relation that has the ``less-than'' properties on all
domain elements.  For example, we will say that adding one makes
numbers bigger with the formula
\begin{equation}\label{Lnext}
\forall x.\ x\,L\,\nextel(x).
\end{equation}

\iffalse  CHECK THIS:
\bparts

\ppart Explain why the formulas above imply 
\begin{equation}\label{Lprev}
\forall x\neq \numerl{0}.\ \prev(x)\,L\,x\ .
\end{equation}

\begin{solution}
If $x\,L\,\nextel(x)$ and $x \neq \numerl{0}$, then
\end{solution}
\eparts
\fi

Then we can forbid all finite cycles by requiring
\begin{equation}\label{irrform}
\forall x.\, \QNOT(x\, L\, x)\ .
\end{equation}

\bparts

\ppart Write down formulas of predicate calculus with only the symbols
above whose models must have properties as much like $\naturals$ as
you can manage.  In particular, make sure that all models of your
formulas force $L$ to mean $<$ on the numeral values and ensure
that~\eqref{irrform} implies there are no finite cycles of
$\nextel$'s.  (If you think you have formulas whose only models are
exactly like $\naturals$, look again, because that is impossible.)

\begin{solution}
Assert that $L$ is \emph{transitive}
\[
\forall x,y,z.\, (x\,L\,y \QAND y\,L\,z) \QIMPLIES x\,L\,z,
\]

Further every two different elements are related by $L$ one way or the
other:
\begin{equation}\label{trichot}
\forall x,y.\, x\,L\,y \QOR y\,L\,x \QOR x = y.
\end{equation}

Now the only models of all these formulas has the a copy of
$\naturals$ along with some two-way infinite chains
\[
\dots c_{-2}\, L \, c_{-1}\, L\, c_0\, L\, c_1\, L\, c_2\, \dots.
\]

All the elements in one of these chains must be $L$-bigger than all
the numeral values.  Also,~\eqref{trichot} ensures all these chains
will have the property that every element of one chain will be $L$ of
every member of the other chain.  So the chains themselves are
``ordered'' by $L$, but there is nothing further that can be said of
how chains are ordered: there might be none, some finite number, or
even an infinite number of them.  And the chains might be ordered
among themselves like the integers, or nonnegative integers, or real
numbers,\dots.

\end{solution}

\ppart Describe a model that satisfies all your formulas but has
elements that are not the numeral values.

\begin{solution}
The numeral values along with a single two-way infinite chain of
successive elements ordered by $L$ is an example for the formulas
given above.
\end{solution}
\fi

\eparts

\end{problem}

%%%%%%%%%%%%%%%%%%%%%%%%%%%%%%%%%%%%%%%%%%%%%%%%%%%%%%%%%%%%%%%%%%%%%
% Problem ends here
%%%%%%%%%%%%%%%%%%%%%%%%%%%%%%%%%%%%%%%%%%%%%%%%%%%%%%%%%%%%%%%%%%%%%

\endinput
