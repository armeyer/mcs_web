\documentclass[problem]{mcs}

\begin{pcomments}
  \pcomment{PS_add_countable_elements}
  \pcomment{add by ARM 4/5/11}
\end{pcomments}

\pkeywords{
  countable
  bijection
  union
}

%%%%%%%%%%%%%%%%%%%%%%%%%%%%%%%%%%%%%%%%%%%%%%%%%%%%%%%%%%%%%%%%%%%%%
% Problem starts here
%%%%%%%%%%%%%%%%%%%%%%%%%%%%%%%%%%%%%%%%%%%%%%%%%%%%%%%%%%%%%%%%%%%%%

\begin{problem}
\bparts

%% \ppart \textbf{Circle} the correct completions (there may be more than
%% one)

%% $A \strict \naturals \ \QIFF$ \dots
%% \begin{itemize}
%% \item $\card{A}$ is undefined.

%% \item $A$ is countably infinite.

%% \item $A$ is uncountable.

%% \item $A$ is finite.

%% \item $\naturals \surj A$.

%% \item $\forall n \in \naturals$, $\card{A} \leq n$.

%% \item $\forall n \in \naturals$, $\card{A} \geq n$.

%% \item $\exists n \in \naturals.\, \card{A} \leq n$.

%% \item $\exists n \in \naturals.\, \card{A} < n$.

%% \end{itemize}

%% \begin{solution}
%% $A$ is finite;  $\exists n \in \naturals.\, \card{A} \leq n$;
%% $\exists n \in \naturals.\, \card{A} < n$.
%% \end{solution}

\ppart\label{Abijb0bn} We know \inbook{from Lemma~\bref{AUb}} that if $A$ is an
infinite set, then $A \bij (A \union \set{b_0})$ for any element $b_0$.
It follows by induction that $A \bij (A \union
\set{b_0,b_1,\dots,b_n})$ for any finite set
$\set{b_0,b_1,\dots,b_n}$.  Briefly explain why this does \emph{not
  prove} the claim that $A \bij (A \union B)$ for a countably infinite set $B
\eqdef \set{b_0,b_1,\dots,b_n,\dots}$.

\hint Illustrate the fallacy with a predicate $P$ such that
$P(\set{b_0,b_1,\dots,b_n})$ for any finite set of elements
$b_0,b_1,\dots,b_n$, but $\QNOT(P(B))$.

\examspace[0.5in]

\begin{solution}
Let $P(A) \eqdef A \text{ is finite}$.
\end{solution}

\ppart Prove the claim of part~\eqref{Abijb0bn}: if $A$ is an infinite
set and $B$ is a countably infinite set that has no elements in common with $A$, then
\[
A \bij (A \union B).
\]
\emph{Reminder}: You may assume any of the results from class, MITx,
or the text as long as you state them explicitly.

\examspace[4.0in]

\begin{staffnotes}
\hint See Problem~\bref{CP_smallest_infinite_set}.
\end{staffnotes}

\begin{solution}

Since $A$ is infinite, we can find an infinite sequence $a_0, a_1, a_2, \ldots$ of distinct elements of $A$.  This sequence may cover all of $A$ (if $A$ is countable) or not.  Since $B$ is countably infinite, it may be enumerated similarly as $b_0, b_1, b_2, \ldots$.  Now it's easy to define the bijection we need.
\begin{eqnarray*}
  f(a_{2i}) &=& a_i \textrm{ (for all $i \in \mathbb N$)} \\
  f(a_{2i+1}) &=& b_i \textrm{ (for all $i \in \mathbb N$)} \\
  f(a) &=& a \textrm{ (where $\forall i. \; a \neq a_i$)}
\end{eqnarray*}

We check that $f$ is really a bijection.

First, $f$ is a \textbf{function}, since, if all $a_i$ are distinct, then no two different $a_{2i}$ or $a_{2i+1}$ can be equal, because no two $2i$ or $2i+1$ can be equal, by basic properties of arithmetic.  Also, the final clause doesn't overlap the first two, pretty much by definition.

Second, $f$ is \textbf{total}, since every $n \in \mathbb N$ can be written as either $2i$ or $2i+1$ for an appropriate $i$, and the last clause covers those $a$ values that don't match some $a_i$.

Third, $f$ is \textbf{1-to-1}, since clearly all the righthand sides of $f$'s defining equations are distinct: we assumed that $B$ does not overlap $A$, as well as that the sequences of $a_i$ and $b_i$ contain no duplicates.  Also, by definition, the last clause fires only for an $a$ that matches no $a_i$.

Finally, $f$ is \textbf{onto}.  Any element of $A \cup B$ is in either $A$ or $B$.  For $a \in A$, if $a = a_i$ for some $i$, then we have $f(a_{2i}) = a$.  Otherwise, the last clause gives us $f(a) = a$.  For $b \in B$, we have $b = b_i$ for some $i$, so $f(a_{2i+1}) = b$.

\end{solution}

\eparts
\end{problem}

%%%%%%%%%%%%%%%%%%%%%%%%%%%%%%%%%%%%%%%%%%%%%%%%%%%%%%%%%%%%%%%%%%%%%
% Problem ends here
%%%%%%%%%%%%%%%%%%%%%%%%%%%%%%%%%%%%%%%%%%%%%%%%%%%%%%%%%%%%%%%%%%%%%

\endinput
