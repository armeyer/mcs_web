\documentclass[problem]{mcs}

\begin{pcomments}
  \pcomment{PS_add_countable_elements}
  \pcomment{add by ARM 4/5/11}
\end{pcomments}

\pkeywords{
  countable
  bijection
  union
}

%%%%%%%%%%%%%%%%%%%%%%%%%%%%%%%%%%%%%%%%%%%%%%%%%%%%%%%%%%%%%%%%%%%%%
% Problem starts here
%%%%%%%%%%%%%%%%%%%%%%%%%%%%%%%%%%%%%%%%%%%%%%%%%%%%%%%%%%%%%%%%%%%%%

\begin{problem}
\bparts

\ppart \textbf{Circle} the correct completions (there may be more than
one)

$A \strict \naturals \QIFF$ \dots
\begin{itemize}
\item $\card{A}$ is undefined.

\item $A$ is countably infinite.

\item $A$ is uncountable.

\item $A$ is finite.

\item $\naturals \surj A$.

\item $\forall n \in \naturals$, $\card{A} \leq n$.

\item $\forall n \in \naturals$, $\card{A} \geq n$.

\item $\exists n \in \naturals.\, \card{A} \leq n$.

\item $\exists n \in \naturals.\, \card{A} < n$.

\end{itemize}

\begin{solution}
$A$ is finite;  $\exists n \in \naturals.\, \card{A} \leq n$;
$\exists n \in \naturals.\, \card{A} < n$.
\end{solution}

\ppart\label{Abijb0bn} We know \inbook{from Lemma~\bref{AUb}} that if $A$ is an
infinite set, then $A \bij (A \union \set{b_0})$ for any element $b_0$.
It follows by induction that $A \bij (A \union
\set{b_0,b_1,\dots,b_n})$ for any finite set
$\set{b_0,b_1,\dots,b_n}$.  Briefly explain why this does \emph{not
  prove} the claim that $A \bij (A \union B)$ for a countably infinite set $B
\eqdef \set{b_0,b_1,\dots,b_n,\dots}$.

\hint Illustrate the fallacy with a predicate $P$ such that
$P(\set{b_0,b_1,\dots,b_n})$ for any finite set of elements
$b_0,b_1,\dots,b_n$, but $\QNOT(P(B))$.

\examspace[0.5in]

\begin{solution}
Let $P(A) \eqdef A \text{ is finite}$.
\end{solution}

\ppart Prove the claim of part~\eqref{Abijb0bn}: if $A$ is an infinite
set and $B$ is a countably infinite set, then
\[
A \bij (A \union B).
\]
\emph{Reminder}: You may assume any of the results from class, MITx,
or the text as long as you state them explicitly.

\examspace[4.0in]

\begin{staffnotes}
\hint See Problem~\bref{CP_smallest_infinite_set}.
\end{staffnotes}

\begin{solution}

\TBA{soln}

\end{solution}

\eparts
\end{problem}

%%%%%%%%%%%%%%%%%%%%%%%%%%%%%%%%%%%%%%%%%%%%%%%%%%%%%%%%%%%%%%%%%%%%%
% Problem ends here
%%%%%%%%%%%%%%%%%%%%%%%%%%%%%%%%%%%%%%%%%%%%%%%%%%%%%%%%%%%%%%%%%%%%%

\endinput
