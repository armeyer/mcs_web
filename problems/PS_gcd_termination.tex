\documentclass[problem]{mcs}

\begin{pcomments}
  \pcomment{PS_gcd_termination}
  \pcomment{by ARM 2/28/11}
\end{pcomments}

\pkeywords{
  Euclidean
  algorithm
  GCD
  Fibonacci
  transition
  golden_ratio
}

%%%%%%%%%%%%%%%%%%%%%%%%%%%%%%%%%%%%%%%%%%%%%%%%%%%%%%%%%%%%%%%%%%%%%
% Problem starts here
%%%%%%%%%%%%%%%%%%%%%%%%%%%%%%%%%%%%%%%%%%%%%%%%%%%%%%%%%%%%%%%%%%%%%

\begin{problem}
Prove that the smallest positive integers $a \ge b$ for which,
starting in state $(a,b)$, the Euclidean state machine will make $n$
transitions are $F(n+1)$ and $F(n)$, where $F(n)$ is the $n$th
Fibonacci number.

\hint Induction.

In a later chapter, we'll show that $F(n) \le \varphi^n$ where
$\varphi$ is the \idx{golden ratio} $(1 + \sqrt{5})/2$.  This implies
that the Euclidean algorithm halts after at most $\log_\varphi(a)$
transitions.  This is a somewhat smaller than the $2 \log_2 a$ bound
derived from equation~(\bref{rxylx2})\inhandout{ in the text}.

\begin{solution}
Lame' argument TBA.
\end{solution}
\end{problem}


%%%%%%%%%%%%%%%%%%%%%%%%%%%%%%%%%%%%%%%%%%%%%%%%%%%%%%%%%%%%%%%%%%%%%
% Problem ends here
%%%%%%%%%%%%%%%%%%%%%%%%%%%%%%%%%%%%%%%%%%%%%%%%%%%%%%%%%%%%%%%%%%%%%

\endinput
