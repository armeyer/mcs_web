\documentclass[problem]{mcs}

\begin{pcomments}
    \pcomment{TP_divides_n_square_then_n}
    \pcomment{see also TP_divides_n_square_not_n}
    \pcomment{ARM 9/7/13}
\end{pcomments}

\pkeywords{
divisor
square
prime
}

\begin{problem}
Let $n$ be a nonnegative integer.

\bparts

\ppart\label{even_square} Explain why if $n^2$ is even---that is, a
multiple of 2---then $n$ is even.

\begin{solution}
We know that the product of two odd numbers is odd, and the product of
an even number and an odd number is even.  So the claim follows from
by proof by contradiction: if $n$ was odd, then $n \cdot n$ would also
be odd, contradicting the fact that $n^2$ is even.
\end{solution}

\ppart Explain why if $n^2$ is a multiple of 3, then
$n$ must be a multiple of 3.

\begin{solution}
We know that every integer greater than 1 has a \emph{unique}
factorization into primes (see the Fundamental Theorem of Arithmetic,
Section~\ref{fundamental_theorem_sec}).  One way to factor $n^2$ into
primes is to use two copies of the unique factorization of $n$ into
primes.  By uniqueness, this is the \emph{only} way to factor $n^2$
into primes.  In particular, any prime in the factorization of $n^2$
must have come from one of the copies of the prime factorization of
$n$, which means that the prime divides $n$.

Of course this argument holds for any prime in place of 3.  In
particular, it holds for the prime 2, which provides another
explanation for part~\ref{even_square}.
\end{solution}

\eparts

\end{problem}

\endinput
