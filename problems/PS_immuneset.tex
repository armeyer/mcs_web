\documentclass[problem]{mcs}

\begin{pcomments}
  \pcomment{PS_immuneset}
  \pcomment{PARTIAL DRAFT ARM 10.9.17}
\end{pcomments}

\pkeywords{
  diagonal_argument
  diagonal
  countable
  function
  immune
  range
}

%%%%%%%%%%%%%%%%%%%%%%%%%%%%%%%%%%%%%%%%%%%%%%%%%%%%%%%%%%%%%%%%%%%%%
% Problem starts here
%%%%%%%%%%%%%%%%%%%%%%%%%%%%%%%%%%%%%%%%%%%%%%%%%%%%%%%%%%%%%%%%%%%%%

\begin{problem}

\begin{center}
\large \textbf{DRAFT }
\end{center}

Let
\[
f_0, f_1, f_2, \dots, f_k,\dots
\]
be an infinite sequence of total functions $f_k:\nngint \to \nngint$.

Define a total function $g: \nngint \to \nngint$ and a set $C_n
\subset \nngint$ recursively:
\[
g(0) \eqdef f_0(0), \quad C_0 \eqdef \set{0}.
\]
Let
\[
k_{n+1} \eqdef \min\set{k \notin C_n \suchthat f_k(n) \geq 2k}, 
\]
Then,
\[
g(n+1) \eqdef f_{k_{n+1}}(n), \quad C_{n+1} \eqdef C_n \union \set{k_{n+1}}.
\]

Let $U$ to be the complement of $\range{G}$, that is,
\[
U \eqdef \bar{\range{g}}.
\]

\bparts

\ppart\label{Uinf} Show that $U$ is infinite.

\begin{solution}
By definition, for every $n \in \nngint$, $\range{g}$ can contain at
most half the elements in $\Zintv{0}{n}$.\inhandout{\footnote{$\Zintv{0}{n}
    \eqdef \set{0,1,\dots,n}$.}}  So $U$ must contain
    \emph{at least} half the elements in $\Zintv{0}{n}$ for every $n$.

\TBA{MUCH MORE EXPLANATION NEEDED: how $C_n$ works?}
\end{solution}


\ppart Show that for all $k \in \nngint$, if $\range{f_k}$ is infinite, then
\[
\range{f_k} \not\subseteq U.
\]

\begin{solution}
We just need to show that
\[
\range{f_k} \intersect \range{g} \neq \emptyset
\]
for all $k \in \nngint$.

\TBA{MORE about $C_n$}

\end{solution}

\ppart Tweak part~\eqref{Uinf} so
\[
\lim_{n \to \infty} \frac{\card{U \intersect \Zintv{0}{n}}}{n} = 1.
\]

\begin{solution}
Replace $2k$ by $k^2$, or by any other function of $k$ that grows more
than linearly.
\end{solution}

\begin{staffnotes}
Contrast with a majorizing function $h$ for $f_0,f_1,\dots$.  Now like
$U$, $\range{h}$ does not contain any infinite $\range{f_k}$.  But
$range{h}$ accomplishes this only by being very sparse.
\end{staffnotes}

\eparts

\end{problem}

\endinput
