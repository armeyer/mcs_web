%CP_prerequisite_relation

\documentclass[problem]{mcs}

\begin{pcomments}
  \pcomment{from: S09.cp3r}
  \pcomment{There is some pdf error that needs to be addressed}
%  \pcomment{}
\end{pcomments}

\pkeywords{
  relations
  scheduling
  partial_orders
}

%%%%%%%%%%%%%%%%%%%%%%%%%%%%%%%%%%%%%%%%%%%%%%%%%%%%%%%%%%%%%%%%%%%%%
% Problem starts here
%%%%%%%%%%%%%%%%%%%%%%%%%%%%%%%%%%%%%%%%%%%%%%%%%%%%%%%%%%%%%%%%%%%%%

\begin{problem} \mbox{}  %LaTeX artifact to position the table

\begin{center}
\begin{tabular}{|l|l|}
\hline
Direct Prerequisites & Subject\\ \hline
 18.01 & 6.042\\ \hline
 18.01 & 18.02\\ \hline
 18.01 & 18.03\\ \hline
 8.01  & 8.02\\ \hline
 8.01  & 6.01\\ \hline
 6.042 & 6.046\\ \hline
 18.02, 18.03, 8.02, 6.01 & 6.02\\ \hline
 6.01, 6.042 & 6.006\\ \hline
 6.01 & 6.034\\ \hline
 6.02 & 6.004\\ \hline
\end{tabular}
\end{center}

\bparts

\ppart\label{prereqtable} For the above table of MIT subject
prerequisites, draw a diagram showing the subject numbers with a line
going down from each subject to each of its (direct) prerequisites.

\begin{solution}
\TBA{pdf bug in figure}
\iffalse
\begin{center}%causes latex error
\includegraphics[height=2.5in]{prereq-poset.pdf}
\end{center}
\fi
\end{solution}

\eparts

For each of the following binary relations, describe a collection of sets
whose proper subset relation, $\subset$, has the same shape.

\bparts

\ppart the prerequisite relation among MIT subjects from part\eqref{prereqtable}.
Explain what would go wrong if the set corresponding to a subject consisted only
of its indirect prerequisites (that is, the subject itself was not
included in the corresponding set)?  \hint Consider 18.01 and 6.042.

\begin{solution}
For each subject, let the corresponding set be the subject itself along
with all the subjects that are indirect prerequisites of that subject.
Remember that a direct prerequisite is considered to be a special case of
an indirect one.

If the subject itself was not included, then since 18.01 and 6.042 have
the same prerequisites, for example, thef would both correspond to the
same set, so the correspondence between subjects and sets would not be a
bijection.
\end{solution}

\ppart  the ``empty'' relation on 5 elements.  That is, the relation
under which no element is related to anything.

\begin{solution}
According to the recipe of taking the sets to be preimages of elements,
the five sets would be $\set{1},\set{2},\set{3},\set{4},\set{5}$.

Of course any 5 sets none of which is contained in any of the others will
also work, for example, all the size 4 subsets of $\set{1,2,3,4,5}$
\end{solution}

\ppart the "properly contains'' relation, $\supset$, on
$\power{\set{1, 2, 3,4}}$.

\begin{solution}
  The standard inverse image solution involves sets of subsets.  A more
  elegant correspondence is to let each set $A \subseteq \set{1, 2, 3,4}$
  correspond to its complement.  That is,
\[
f(A) \eqdef \set{1, 2, 3,4} - A.
\]

\end{solution}

\eparts

\end{problem}

%%%%%%%%%%%%%%%%%%%%%%%%%%%%%%%%%%%%%%%%%%%%%%%%%%%%%%%%%%%%%%%%%%%%%
% Problem ends here
%%%%%%%%%%%%%%%%%%%%%%%%%%%%%%%%%%%%%%%%%%%%%%%%%%%%%%%%%%%%%%%%%%%%%

\endinput
