%PS_binomial_problem

\documentclass[problem]{mcs}

\begin{pcomments}
  \pcomment{PS_linear_recur_closed_form}
  \pcomment{from: F06.ps8}
\end{pcomments}

\pkeywords{
  generating_functions
  linear_recurrence
}

%%%%%%%%%%%%%%%%%%%%%%%%%%%%%%%%%%%%%%%%%%%%%%%%%%%%%%%%%%%%%%%%%%%%%
% Problem starts here
%%%%%%%%%%%%%%%%%%%%%%%%%%%%%%%%%%%%%%%%%%%%%%%%%%%%%%%%%%%%%%%%%%%%%

\begin{problem}

  Find closed-form solutions to the following linear recurrences.

  % \bparts

%   \ppart $x_{n} = 12 x_{n-2} - 16 x_{n-3} \quad
%     (x_0 = 1, x_1 = 2, x_2 = 3)$
% 
%   Hint: 2 is a root.
% 
% \begin{solution}
%   The characteristic equation is $r^3 - 12 r +  16 = 0$.
%   Solving a cubic equation can be messy process, but in this case the
%   roots are easy to find:
%   \begin{align*}
%   r_1 & =  2 \\
%   r_2 & =  2 \\
%   r_3 & = -4 
%   \end{align*}
% 
%   Therefore a general form for a solution is
%   \[
%   x_n  = A 2^n + B n 2^n + C (-4)^n.
%   \]
% 
%   Substituting the initial conditions into this general form gives a
%   system of linear equations.
%   \begin{eqnarray*}
%   1 & = & A + C \\
%   2 & = & 2 A + 2 B -4 C \\
%   3 & = & 4 A + 8 B + 16 C
%   \end{eqnarray*}
% 
%   The solution to this linear system is $A = 37/36$, $B = -1/12$, and
%   $C = -1/36$.  The complete solution to the recurrence is therefore
%   \[
%   x_n  =  \frac{37}{36}2^{n} - \frac{1}{12}n 2^{n} - \frac{1}{36}(-4)^{n}.
%   \]
% \end{solution}

  % \ppart
  \[
  x_{n} = 3x_{n-1} - 2 x_{n-2} + n \quad (x_0 = 0, x_1 = 1)
  \]
  
  \hint The solution should be of the form 
  \[
  x_n = A \cdot 2^n + B \codt n^2 + C \cdot n + D.
  \]

\begin{solution}
  First, we find the general solution to the homogenous recurrence.  The
  characteristic equation is $r^2 - 3r + 2 = 0$.  The roots of this
  equation are $r_1 = 1$ and $r_2 = 2$.  Therefore, the general solution
  to the homogenous recurrence is

  \begin{eqnarray*}
  x_n & = & A + B 2^n.
  \end{eqnarray*}

  Now we must find a particular solution to the recurrence, ignoring the
  boundary conditions.  Since the inhomogenous term is linear, we guess
  there is a linear solution, that is, a solution of the form $an + b$.  If
  the solution is of this form, we must have
  \[
  an+b = 3(a(n-1) + b) -2(a(n-2) + b) +n,
  \]
  which implies the absurd conclusion that
  \[
  n = -(3b +a).
  \]

  So we make another guess, this time that there is a quadratic solution of
  the form $an^2 + bn +c$.  If the solution is of this form, we must have
  \[
  an^2+bn+c = 3(a(n-1)^2+b(n-1)+c) - 2(a(n-2)^2+b(n-2)+c) + n,
  \]
  which simplifies to
  \[
  (2a+1)n + (b - 5a) = 0.
  \]
  This last equation is satisfied only if the coefficient of $n$ and the
  constant term are both zero, which implies $a = -1/2$ and $b = -5/2$.
  Apparently, any value of $c$ gives a valid particular solution.  For
  simplicity, we choose $c = 0$ and obtain the particular solution:
  \[
  x_n = -\frac{1}{2}n^2 - \frac{5}{2} n.
  \]

  The complete solution to the recurrence is the homogenous solution
  plus the particular solution:
  \[
  x_n = A + B 2^n -\frac{1}{2}n^2 - \frac{5}{2} n
  \]
  Substituting the initial conditions gives a system of linear equations:
  \begin{eqnarray*}
  0	& = &	A + B \\
  1	& = &	A + 2 B - 3
  \end{eqnarray*}

  The solution to this linear system is $A = -4$ and $B = 4$.
  Therefore, the complete solution to the recurrence is
  \[
  x_n = 2^{n+2} -\frac{1}{2}n^2 - \frac{5}{2} n - 4.
  \]
\end{solution}

  % \eparts



\end{problem}

%%%%%%%%%%%%%%%%%%%%%%%%%%%%%%%%%%%%%%%%%%%%%%%%%%%%%%%%%%%%%%%%%%%%%
% Problem ends here
%%%%%%%%%%%%%%%%%%%%%%%%%%%%%%%%%%%%%%%%%%%%%%%%%%%%%%%%%%%%%%%%%%%%%

\endinput
