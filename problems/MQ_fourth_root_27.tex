\documentclass[problem]{mcs}

\begin{pcomments}
  \pcomment{MQ_fourth_root_27_irrational}
  \pcomment{riff on CP_generalize_root_2_proof}
  \pcomment{ARM for MQ 9/12/11}
\end{pcomments}

\pkeywords{
  root_4
  irrational
  rational
  contradiction
}

%%%%%%%%%%%%%%%%%%%%%%%%%%%%%%%%%%%%%%%%%%%%%%%%%%%%%%%%%%%%%%%%%%%%%
% Problem starts here
%%%%%%%%%%%%%%%%%%%%%%%%%%%%%%%%%%%%%%%%%%%%%%%%%%%%%%%%%%%%%%%%%%%%%

\begin{problem}\label{generprob}
   Prove that $\sqrt[4]{27}$ is irrational.

\begin{solution}
\begin{proof}
  Assume for the sake of contradiction that $\sqrt[4]{27}$ is rational.
  Under this assumption, there exist integers $a$ and $b$ such that
\[
\sqrt[4]{27} =  \frac{a}{b},
\] 
where $a$ and $b$ have no common factor.  Now we prove that $a$ and
$b$ have 3 as a common factor, a contradiction.

\begin{align*}
\sqrt[4]{27} & = \frac{a}{b}, & \text{(by assumption)}\\
27           & = \frac{a^{4}}{b^{4}}, & \text{(taking 4th powers)}\\
27b^{4}      & = a^{4}.
\end{align*}
The lefthand side of this equation has 3 as a factor, so 3 is a
factor of the right hand side, $a^{4}$, which, by unique
factorization, implies that 3 is also a factor of $a$.

In particular, $a = 3c$ for some integer $c$.  Thus,
\begin{align*}
27b^{4} & = (3c)^4 = 3^4c^{4},\\
b^{4}  & = 3c^{4} & \text{(dividing by 27)}.
\end{align*}
The righthand side of the last equation has 3 as a factor, so the
lefthand side, $b^{4}$, does as well, which implies that $b$ does too.
\end{proof}

\end{solution}

\end{problem}
%%%%%%%%%%%%%%%%%%%%%%%%%%%%%%%%%%%%%%%%%%%%%%%%%%%%%%%%%%%%%%%%%%%%%
% Problem ends here
%%%%%%%%%%%%%%%%%%%%%%%%%%%%%%%%%%%%%%%%%%%%%%%%%%%%%%%%%%%%%%%%%%%%%

\endinput
