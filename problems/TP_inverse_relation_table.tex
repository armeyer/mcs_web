\documentclass[problem]{mcs}

\begin{pcomments}
  \pcomment{TP_inverse_relation_table}
  \pcomment{excerpted from CP_relational_properties_table
            do not use with that problem}
\end{pcomments}

\pkeywords{
  relations
  total
  relational_properties
  functions
  injections
  surjections
  bijections
}

%%%%%%%%%%%%%%%%%%%%%%%%%%%%%%%%%%%%%%%%%%%%%%%%%%%%%%%%%%%%%%%%%%%%%
% Problem starts here
%%%%%%%%%%%%%%%%%%%%%%%%%%%%%%%%%%%%%%%%%%%%%%%%%%%%%%%%%%%%%%%%%%%%%

\begin{problem}
  The \emph{inverse}, $\inv{R}$, of a binary relation, $R$, from $A$ to
  $B$, is the relation from $B$ to $A$ defined by:
\[
b \mrel{\inv{R}} a \qiff a \mrel{R} b.
\]
In other words, you get the diagram for $\inv{R}$ from $R$ by ``reversing
the arrows'' in the diagram describing $R$.  Now many of the relational
properties of $R$ correspond to different properties of $\inv{R}$.  For
example, $R$ is \emph{total} iff $\inv{R}$ is a \emph{surjection}.

Fill in the remaining entries is this table:
\begin{center}
\begin{tabular}{l|cl}
$R$ is  & iff & $\inv{R}$ is \\ \hline
total                    && a surjection\\
a function\\
a surjection\\
an injection\\
a bijection
\end{tabular}
\end{center}

\hint Explain what's going on in terms of ``arrows'' from $A$ to $B$ in
the diagram for $R$.

\begin{solution}\mbox{}
\begin{center}
\begin{tabular}{l|cl}
$R$ is  & iff & $\inv{R}$ is \\ \hline
total                    && a surjection\\
a function               && \insolutions{an injection}\\
a surjection             && \insolutions{total}\\
an injection             && \insolutions{a function}\\
a bijection              && \insolutions{a bijection}
\end{tabular}
\end{center}
The first line of the table follows from the fact that \emph{total} means
$[\geq 1\ \text{out}]$, so reversing the arrows gives $[\geq 1\
\text{in}]$ which is the definition of \emph{surjection}.

The second line follows from the fact that \emph{function} means
$[\leq 1\ \text{out}]$, so reversing the arrows gives $[\leq 1\
\text{in}]$ which is the definition of \emph{injection}.

The third and fourth lines follow respectively from the first and second
lines.

The fifth line follows from the fact that \emph{bijection} means $[=1\
\text{out}, =1\ \text{in}]$, so reversing the arrows gives $[=1\ \text{in},
=1\ \text{out}]$ which is the same.

\end{solution}

\inhandout{
\examspace[0.6in]
\textbox{
\textboxheader{Arrow Properties}

\begin{definition*}
A binary relation, $R$ is
\begin{itemize}

\item is a \emph{function} when it has the $[\le 1\ \text{arrow
    \textbf{out}}]$ property.

\item is \emph{surjective} when it has the $[\ge 1\ \text{arrows
    \textbf{in}}]$ property.  That is, every point in the righthand,
     codomain column has at least one arrow pointing to it.

\item is \emph{total} when it has the $[\ge 1\ \text{arrows
       \textbf{out}}]$ property.

\item is \emph{injective} when it has the $[\le 1\ \text{arrow
    \textbf{in}}]$ property.

\item is \emph{bijective} when it has both the $[=1\ \text{arrow
    \textbf{out}}]$ and the $[=1\ \text{arrow \textbf{in}}]$ property.
\end{itemize}
\end{definition*}
}}

\end{problem}


%%%%%%%%%%%%%%%%%%%%%%%%%%%%%%%%%%%%%%%%%%%%%%%%%%%%%%%%%%%%%%%%%%%%%
% Problem ends here
%%%%%%%%%%%%%%%%%%%%%%%%%%%%%%%%%%%%%%%%%%%%%%%%%%%%%%%%%%%%%%%%%%%%%

\endinput
