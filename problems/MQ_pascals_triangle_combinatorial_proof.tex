\documentclass[problem]{mcs}

\begin{pcomments}
  \pcomment{adapted from MQ_pascals_triangle}
  \pcomment{from: S10}
\end{pcomments}

\pkeywords{
  combinatorial_proof
  Pascal
  Triangle
  algebraic
}

%%%%%%%%%%%%%%%%%%%%%%%%%%%%%%%%%%%%%%%%%%%%%%%%%%%%%%%%%%%%%%%%%%%%%
% Problem starts here
%%%%%%%%%%%%%%%%%%%%%%%%%%%%%%%%%%%%%%%%%%%%%%%%%%%%%%%%%%%%%%%%%%%%%

\begin{problem}

The following identity is known as \idx{Pascal's Triangle}: 
\begin{equation*}
{{n-1}\choose{k-1}} + {{n-1}\choose k} = {n \choose k}
\end{equation*}

Give a \emph{\idx{combinatorial proof}} for Pascal's Triangle identity.

\examspace[3in]

\begin{solution}
This comes directly from Section~\bref{subsec:pascal} of the text,
reproduced below:

Jay figures that $n$ people (including himself)
are competing for spots on a team and only $k$ will be selected.  As
part of maneuvering for a spot on the team, he needs to work out how
many different teams are possible.  There are two cases to consider:

\begin{itemize}

\item Jay \emph{is} selected for the team, and his $k - 1$
  teammates are selected from among the other $n - 1$ competitors.
  The number of different teams that can be formed in this way is:
\[
\binom{n-1}{k-1}.
\]

\item Jay is \emph{not} selected for the team, and all $k$ team
members are selected from among the other $n - 1$ competitors.  The
number of teams that can be formed this way is:
%
\[
\binom{n - 1}{k}.
\]

\end{itemize}

All teams of the first type contain Jay, and no team of the second
type does; therefore, the two sets of teams are disjoint.  Thus, by
the Sum Rule, the total number of teams is:
%
\[
\binom{n-1}{k-1} + \binom{n - 1}{k}.
\]

Another way of thinking about it is that $n$ people (including
himself) are trying out for $k$ spots.  Thus, the number of ways to
select the team is simply:
%
\[
\binom{n}{k}.
\]

Each method correctly counts the number of teams, so the 
 answers must be equal.  So we know:
\[
\binom{n-1}{k-1} + \binom{n - 1}{k} = \binom{n}{k}
\]

\end{solution}


\end{problem}

%%%%%%%%%%%%%%%%%%%%%%%%%%%%%%%%%%%%%%%%%%%%%%%%%%%%%%%%%%%%%%%%%%%%%
% Problem ends here
%%%%%%%%%%%%%%%%%%%%%%%%%%%%%%%%%%%%%%%%%%%%%%%%%%%%%%%%%%%%%%%%%%%%%

\endinput
