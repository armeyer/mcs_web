\documentclass[problem]{mcs}

\begin{pcomments}
  \pcomment{CP_10_and_15_cent_stamps_by_WOP}
  \pcomment{same as F09: cp2m with 10 replacing 6}
\end{pcomments}

\pkeywords{
  well_ordering
  WOP
  postage_stamps
}

%%%%%%%%%%%%%%%%%%%%%%%%%%%%%%%%%%%%%%%%%%%%%%%%%%%%%%%%%%%%%%%%%%%%%
% Problems start here
%%%%%%%%%%%%%%%%%%%%%%%%%%%%%%%%%%%%%%%%%%%%%%%%%%%%%%%%%%%%%%%%%%%%%

\begin{problem}
  The (flawed) proof below uses the Well Ordering Principle to prove that every
  amount of postage that can be paid exactly, using only 10 cent and 15
  cent stamps, is divisible by 5.  Let $S(n)$ mean that exactly $n$ cents
  postage can be paid using only 10 and 15 cent stamps.  Then the proof
  shows that
%
\begin{equation}\tag{*}
S(n)\ \QIMPLIES\ 5 \divides n, \quad \text{for all nonnegative integers $n$}.
\end{equation}
Fill in the missing portions (indicated by ``\dots'') of the following
proof of~(*), and at the final line point out where the error in the proof is.

\begin{quote}
Let $C$ be the set of \emph{counterexamples} to~(*), namely
\[
C \eqdef \set{n \suchthat S(n)\text{ and } \QNOT(5 \divides n)}
\]

Assume for the purpose of obtaining a contradiction that $C$ is
nonempty.  Then by the WOP, there is a smallest number, $m \in C$.
Then $S(m-10)$ or $S(m-15)$ must hold, because the $m$ cents postage is made from 10 and 15 cent
stamps, so we remove one.

So suppose $S(m-10)$ holds.  Then $5 \divides (m-10)$, because\dots

\examspace[0.3in]
\instatements{\examrule[6in]}

\begin{solution}
\dots if $\QNOT(5 \divides (m-10))$, then $m-10$ would be
  a counterexample smaller than $m$, contradicting the minimality of
  $m$.
\end{solution}

But if $5 \divides (m-10)$, then  $5 \divides m$, because\dots

\examspace[0.3in]
\instatements{\examrule[6in]}

\begin{solution}
\dots $ 5 \divides (m-10) $ and $5 \divides 10$, so $5 \divides (m - 10  + 10)$.
\end{solution}

contradicting the fact that $m$ is a counterexample.

Next suppose $S(m-15)$ holds.  Then the proof for $m-10$ carries over
directly for $m-15$ to yield a contradiction in this case as well.
Since we get a contradiction in both cases, we conclude that  $C$ must be empty.  That is, there are no
counterexamples to~(*), which proves that (*) holds.

\end{quote}

What was wrong/missing in the argument? Your answer should fit in the line below.

\begin{solution}
We didn't check $m>0$, if $m = 0$ neither $S(m - 10)$ nor $S(m -15)$ hold.
\end{solution}


\examspace[0.3in]
\instatements{\examrule[6in]}


\end{problem}

%%%%%%%%%%%%%%%%%%%%%%%%%%%%%%%%%%%%%%%%%%%%%%%%%%%%%%%%%%%%%%%%%%%%%
% Problems end here
%%%%%%%%%%%%%%%%%%%%%%%%%%%%%%%%%%%%%%%%%%%%%%%%%%%%%%%%%%%%%%%%%%%%%
\examspace[3in]
\endinput
