\documentclass[problem]{mcs}

\begin{pcomments}
  \pcomment{TP_sub_one_with_two_of_opposite_color}
  \pcomment{proposed for midterm II, Fall15}
  \pcomment{author: Zoran Dzunic}
\end{pcomments}

\pkeywords{
  state_machines
  derived_variable
  induction
  congruences
}

%%%%%%%%%%%%%%%%%%%%%%%%%%%%%%%%%%%%%%%%%%%%%%%%%%%%%%%%%%%%%%%%%%%%%
% Problem starts here
%%%%%%%%%%%%%%%%%%%%%%%%%%%%%%%%%%%%%%%%%%%%%%%%%%%%%%%%%%%%%%%%%%%%%

\begin{problem}
``Token replacing'' is a single player game using a set of tokens,
  each colored black or white.  Except for color, the tokens are
  indistinguishable.  The game starts with one black token.
  In each move, a player can replace one black token with two
  white tokens, or replace one white token with two black tokens.

% Zoran: I somehow prefer to introduce the start state here, as it may
% be hinting too much if introduced in part (b). Also, I came up
% with another part of the problem for which it may be better to
% have the start state fixed as part of rules.

  We can model this game as a state machine whose states are pairs
  $(n_b,n_w)$ where $n_b \geq 0$ equals the number of black tokens and
  $n_w \geq 0$ equals the number of white tokens.

\iffalse
There is no further labeling of the tokens of the same color (e.g., we
don't care which of the black tokens appeared first, second, etc.),
and the order of tokens does not matter (e.g., sequence 'bbw' is
the same as sequence 'bwb').
\fi

\bparts

\iffalse
\ppart
Model this game as a state machine, carefully defining the set of
states, the start state and the possible state transitions.
\hint Be sure to state the conditions of the state transitions.
\begin{solution}
A state of this game is fully described with a pair of nonnegative integer
numbers $(n_b, n_w)$, where $n_b$ and $n_w$ are the numbers of black and
white tokens, respectively. State transitions are of the form
$(n_b, n_w) \rightarrow (n_b-1, n_w+2)$ if $n_b > 0$, and
$(n_b, n_w) \rightarrow (n_b+2, n_w-1)$ if $n_w > 0$ 
\end{solution}

\examspace[2.0in]
\fi

%{\color{red} VARIANT 1}

\ppart
\iffalse
Let $n_b$ be the number of black tokens, and $n_w$ the number of
white tokens \iffalse, and $n \eqdef n_b + n_w$ be the total number of
tokens\fi at any step. 
\fi
Which of the following predicates are preserved invariants (list their numbers):
\begin{align}
n_b + n_w  & \equiv 1 \pmod{3}\\
n_w - n_b  & \equiv 2 \pmod{3}\\
n_b - n_w  & \equiv 2 \pmod{3}\\
2n_b + n_w & \equiv 2 \pmod{3}\\
n_b+n_w & > 5 \\
n_b+n_w & < 5
\end{align}
%\begin{align*}
%P_1 & \eqdef [n_b + n_w  \equiv 1 \pmod{3}] \\
%P_2 & \eqdef [n_w - n_b  \equiv 2 \pmod{3}] \\
%P_3 & \eqdef [n_b - n_w  \equiv 2 \pmod{3}] \\
%P_4 & \eqdef [2n_b + n_w \equiv 2 \pmod{3}] \\
%P_5 & \eqdef [n_b+n_w > 5] \\
%P_6 & \eqdef [n_b+n_w < 5]
%\end{align*}
%\begin{enumerate}
%\item $n_b + n_w \equiv 1 \pmod{3}$ \hfill\examrule[0.5in]
%\item $n_w - n_b \equiv 2 \pmod{3}$ \hfill\examrule[0.5in]
%\item $n_b - n_w \equiv 2 \pmod{3}$ \hfill\examrule[0.5in]
%%\item $2 n_b - n \equiv 1 \pmod{3}$ \hfill\examrule[0.5in]
%%\item $n_b + n \equiv 2 \pmod{3}$ \hfill\examrule[0.5in]
%\item $2n_b + n_w \equiv 2 \pmod{3}$ \hfill\examrule[0.5in]
%\end{enumerate}

\examspace[0.5in]

\begin{solution}
Predicates 2, 3, 4, and 5 are preserved invariants.

Detailed explanation:
\begin{enumerate}
\item {\bf False.}
E.g., predicate is true for state $(1, 0)$, but not for state $(0, 2)$ that
it transitions to.
\item {\bf True.}
Let $(n_b, n_w)$ be an arbitrary state for which $n_w - n_b \equiv 2 \pmod{3}$.
The only two possible transitions are into state $(n'_b, n'_w) = (n_b - 1, n_w + 2)$
or state $(n'_b, n'_w) = (n_b + 2, n_w - 1)$. Note that
$n'_w - n'_b = n_w - n_b \pm 3 \equiv n_w - n_b \pmod{3} \equiv 2 \pmod{3}$,
and thus the property is preserved.
\item {\bf True.}
Similar as with predicate 2.
\item {\bf True.}
Note that $2n_b + n_w = n_w - n_b + 3n_b \equiv n_w - n_b \pmod{3}$,
and thus this predicate is equivalent to predicate 2.
\item {\bf True.}
Note that $n_b + n_w$ increases by 1 at each move. Thus, if
$n_b + n_w > 5$, the same must hold for all states it can transition to.
\item {\bf False.}
E.g., predicate is true for state $(1, 3)$, but not for state $(3, 2)$ that
it transitions to.
\end{enumerate}
%\begin{enumerate}
%\item $n_b + n_w \equiv 1 \pmod{3}$ -- {\bf False}
%\item $n_w - n_b \equiv 2 \pmod{3}$ -- {\bf True}
%\item $n_b - n_w \equiv 2 \pmod{3}$ -- {\bf True}
%%\item $2 n_b - n \equiv 1 \pmod{3}$ -- {\bf True}
%\item $2n_b + n_w \equiv 2 \pmod{3}$ -- {\bf True}
%\end{enumerate}
\end{solution}

\iffalse
\ppart
Which of the preceding are true for the start state?

\begin{solution}
\begin{enumerate}
\item $n_b + n_w \equiv 1 \pmod{3}$ -- {\bf True}
\item $n_w - n_b \equiv 2 \pmod{3}$ -- {\bf True}
\item $n_b - n_w \equiv 2 \pmod{3}$ -- {\bf False}
\item $2 n_b - n \equiv 1 \pmod{3}$ -- {\bf True}
\item $n_b + n \equiv 2 \pmod{3}$ -- {\bf True}
\end{enumerate}
\end{solution}
\fi

\ppart
Which of the predicates above are true for all reachable states?

\begin{staffnotes}
If needed, provide the following hint.
\hint{Which are true of the start state?}
%Use the invariant principle.}
\end{staffnotes}

\examspace[0.5in]

\begin{solution}
Predicates 2 and 4 are true for all reachable states.

Detailed explanation:
\begin{enumerate}
\item {\bf False.}
Not true for state $(0, 2)$ that is reachable:
$(1, 0) \rightarrow (0, 2)$.
\item {\bf True.}
True in the start state and is also a preserved invariant,
and is therefore true for all reachable states by the invariant principle.
\item {\bf False.}
Not true in the start state.
\item {\bf True.}
The same as with predicate 2.
\item {\bf False.}
Not true in the start state.
\item {\bf False.}
Note that $n_b + n_w$ increases by 1 at each move and will eventually become $\ge 5$.
E.g., $(1, 0) \rightarrow (0, 2) \rightarrow (2, 1) \rightarrow (1, 3) \rightarrow (3, 2)$.
\end{enumerate}
%By the invariant principle, a predicate is true for all reachable states if it is both
%a preserved invariant and true for the start state.
%
%\begin{enumerate}
%\item $n_b + n_w \equiv 1 \pmod{3}$ -- {\bf False}
%\item $n_w - n_b \equiv 2 \pmod{3}$ -- {\bf True}
%\item $n_b - n_w \equiv 2 \pmod{3}$ -- {\bf False}
%%\item $2 n_b - n \equiv 1 \pmod{3}$ -- {\bf True}
%\item $2n_b + n_w \equiv 2 \pmod{3}$ -- {\bf True}
%\end{enumerate}
\end{solution}

\ppart
%Let $n \eqdef n_b + n_w$ be the total number of tokens at any step.
%Let $R(n)$ be the reachable sates
Define
\[
R(n) \eqdef \set{(n_b, n_w) \in \integers^2 \suchthat n_b + n_w = n,\,
n_b \ge 0,\, n_w \ge 0 \QAND
n_w - n_b \equiv 2 \pmod{3}}.
\]
Prove by induction on $n$ that $R(n)$ is the set of all reachable states for
which $n_b+n_w = n$.

\examspace[4.0in]

\begin{solution}
The induction hypothesis is
that if $n_b + n_w = n$, then $(n_b, n_w)$ is reachable if and only if $(n_b, n_w) \in R(n)$.
%\[
%P(n) \eqdef \text{if} \, n_b + n_w = n, (n_b, n_w) \, \text{is reachable} \iff (n_b, n_w) \in R(n).
%\]
Note that this hypothesis is clearly true when $n_b + n_w = n \le 0$, since no such state
is reachable and $R(n)$ is empty in such case.

\inductioncase{Base case}: ($n = 1$).
Since $n$ increases as every step, the only reachable state with $n=1$ is
the start state $(1,0)$.  Since $R(1) = \set{(1,0)}$, we conclude that
this base case is true.

%The only valid state for $n=1$ is $(n_b, n_w) = (1,0)$, which is the starting
%state and is therefore reachable ($P((1,0))$ must hold by definition).

\inductioncase{Inductive step}:
Assume that the induction hypothesis holds for $n \ge 1$.
Let $(n'_b, n'_w)$ be a state for which $n' = n'_b + n'_w = n+1$.
We need to show two directions.

First let us show that if $(n'_b, n'_w)$ is reachable, then $(n'_b, n'_w) \in R(n)$.
Note that the total number of tokens increases by one at each step.
Therefore, $(n'_b, n'_w)$ is reachable only if there exists a reachable state
$(n_b, n_w)$ for which $(n_b, n_w) = n$ and a transition from
$(n_b, n_w)$ to $(n'_b, n'_w)$.
Let $(n_b, n_w)$ be such a state.
By inductive assumption, $(n_b, n_w) \in R(n)$, and therefore
$n_w - n_b \equiv 2 \pmod{3}$.
Note that there are two possible transitions from state $(n_b, n_w)$:
$(n_b, n_w) \rightarrow (n_b-1, n_w+2)$ if $n_b > 0$, and
$(n_b, n_w) \rightarrow (n_b+2, n_w-1)$ if $n_w > 0$.
In either case $n'_b + n'_w = n + 1$, $n'_b \ge 0$ and $n'_w \ge 0$
are true. In addition,
$n'_w - n'_b = n_w - n_b \pm 3 \equiv n_w - n_b \pmod{3} \equiv 2 \pmod{3}$.
Therefore, $(n'_b, n'_w) \in R(n)$ must hold.

Now let us show that if $(n'_b, n'_w) \in R(n)$ then $(n'_b, n'_w)$ is reachable.
Either $n'_b \ge 2$ or $n'_w \ge 2$ is true, which we prove by cases. If $n+1=2$,
then $R(n) = \set{(0, 2)}$. If $n+1 > 2$, then by assuming $n'_b \le 1$ and
$n'_w \le 1$, $n+1 = n'_b + n'_w \le 2$ must hold, which is a contradiction.
As a consequence, either state $(n_b, n_w) = (n'_b-2, n'_w+1)$ or
$(n_b, n_w) = (n'_b+1, n'_w-2)$ satisfies $n_b \ge 0$ and $n_w \ge 0$.
Pick such $(n_b, n_w)$.
In addition, $n_b + n_w = n$ and
$n_w - n_b = n'_w - n'_b \pm 3 \equiv n'_w - n'_b \pmod{3} \equiv 2 \pmod{3}$.
Clearly, $(n_b, n_w) \in R(n)$ and is therefore reachable by the hypothesis.
Now, if $(n_b, n_w) = (n'_b-2, n'_w+1)$, then$(n'_w, n'_b)$ is reachable
\iffalse by a transition \fi from $(n_b, n_w)$ by replacing a white token,
while if $(n_b, n_w) = (n'_b+1, n'_w-2)$, then $(n'_w, n'_b)$ is reachable
\iffalse by a transition\fi from $(n_b, n_w)$ by replacing a black token.
\end{solution}

\iffalse
\ppart
The proof above would hold regardless of which predicate from part (b) that you selected
as an answer was used in the definition of $R(n)$. Why do you think that is the case?
%Why do you think the proof above holds regardless of which predicate above you chose?

\examspace[1.0in]

\begin{solution}
Any of the given predicates is preserved in a ``reverse'' move.
\end{solution}
\fi

\iffalse
\ppart
For each predicate above, how would you modify the game such that
it is true for all reachable states?

\examspace[2.0in]

\begin{solution}
\begin{enumerate}
\item Change the rules such that each token is replace with 4 tokens of opposite color instead.
\item No change.
\item Start with any state for which this predicate is satisfied, e.g., $(0, 1)$ or $(2, 0)$.
\item No change.
\item Start with any state for which $n_b + n_w > 5$, e.g., $(3, 3)$.
\item Change the rules such that each token is replace with 1 token of opposite color instead.
\end{enumerate}
\end{solution}
\fi

\eparts
\end{problem}

%%%%%%%%%%%%%%%%%%%%%%%%%%%%%%%%%%%%%%%%%%%%%%%%%%%%%%%%%%%%%%%%%%%%%
% Problem ends here
%%%%%%%%%%%%%%%%%%%%%%%%%%%%%%%%%%%%%%%%%%%%%%%%%%%%%%%%%%%%%%%%%%%%%

\endinput
