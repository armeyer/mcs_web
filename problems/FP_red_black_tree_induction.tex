\documentclass[problem]{mcs}

\begin{pcomments}
  \pcomment{FP_red_black_tree_induction}
  \pcomment{same as PS_ but with hint}
  \pcomment{F15.final-conflict2}
  \pcomment{12/12/15 by Zoran & ARM}
\end{pcomments}

\pkeywords{
  structural_induction
  tree
  red_black
}

%%%%%%%%%%%%%%%%%%%%%%%%%%%%%%%%%%%%%%%%%%%%%%%%%%%%%%%%%%%%%%%%%%%%%
% Problem starts here
%%%%%%%%%%%%%%%%%%%%%%%%%%%%%%%%%%%%%%%%%%%%%%%%%%%%%%%%%%%%%%%%%%%%%

\newcommand{\redl}{\textbf{red}}
\newcommand{\blackl}{\textbf{black}}
\newcommand{\RBT}{\text{RBT}}

\begin{problem} \mbox{}
The set \RBT\ of \emph{Red-black trees} is defined recursively as follows:

\inductioncase{Base case(s)}:
\[
\ang{\redl}, \ang{\blackl} \in \RBT.
\]

\inductioncase{Constructor case(s)}:
$A, B$ are \RBT's, then
\begin{itemize}
\item if $A, B$ start with \blackl, then $\ang{\redl, A, B}$ is an \RBT.
\item if $A, B$ start with \redl, then $\ang{\blackl, A, B}$ is an \RBT.
\end{itemize}

Prove by structural induction that if an \RBT\ $T$ has $r_T$
\redl\ labels and $b_T$ \blackl\ labels, and $T$ starts with a
\redl\ label, then
\begin{equation}\label{nT3leqrT}
\frac{n_T}{3} \leq r_T \leq \frac{2n_T +1}{3},
\end{equation}
where $n_T \eqdef r_A+b_A$.
\medskip

\hint \[n/3 \leq r\quad \QIFF\quad (2/3)n \geq n-r\]

\begin{solution}
The induction hypothesis will be
\begin{align*}
[T \text{ starts with } & \redl \QIMPLIES\ \frac{n_T}{3} \leq r_T \leq \frac{2n_T + 1}{3}]\quad \QAND\\
[T \text{ starts with } & \blackl \QIMPLIES\ \frac{n_T}{3} \leq b_T \leq \frac{2n_T + 1}{3}].
\end{align*}

\inductioncase{Base case(s)}:
\begin{itemize}
\item ($T = \ang{\redl}$).  Then $r=1, b=0, n=1$ and
\[
\frac{1}{3} \leq 1 = \frac{2\cdot 0 0+1}{3}
\]
as required.
\item ($T = \ang{\blackl}$).  Now $r=0, b=1, n=1$ and the required
  inequality holds similarly.
\end{itemize}

\inductioncase{Constructor case(s)}:
\begin{itemize}

\item ($T = \ang{\redl, A, B}$).  The constructor rules imply that
  $A,B$ are \RBT's that start with \blackl.  Now we have by induction that
\begin{align}
\frac{n_A}{3} & \leq b_A \leq \frac{2n_A + 1}{3}]\label{nA3bA},\\
\frac{n_B}{3} & \leq b_B \leq \frac{2n_B + 1}{3}]\label{nB3bB}.
\end{align}
Adding these, we get
\[\begin{array}{rlcl}
\frac{n_A}{3}+ \frac{n_B}{3} & \leq b_A+b_B & \leq & \frac{2n_A + 1}{3}+\frac{2n_b + 1}{3},\notag\\
\frac{n_A+n_b}{3}            & \leq b_A+b_B & \leq & \frac{2(n_A + n_b + 1)}{3},\notag\\
\frac{n_T-1}{3}              & \leq b_T    & \leq & \frac{2n_T}{3}.\label{nT13}
\end{array}\]

Subtracting each of the terms in~\eqref{nT13} from $n_T$ reverses the
inequalities and yields
\begin{align*}
n_T -\frac{n_T - 1}{3} & \geq n_T - b_T  \geq n_T -\frac{2n_T}{3},\\
\frac{2n_T + 1}{3} &  \geq r_T  \geq \frac{n_T}{3},
\end{align*}
which proves~\eqref{nT3leqrT}.

\item ($T = \ang{\blackl, A, B}$).  Follows from the previous case By a symmetric argument.

\end{itemize}

\end{solution}

\end{problem}

\endinput
