\documentclass[problem]{mcs}

\begin{pcomments}
    \pcomment{MQ_translate_set_formulas_S13}
    \pcomment{based on TP_basic_set_formulas}
    \pcomment{ARM 3/17/13}
\end{pcomments}

\pkeywords{
  logic
  sets
  set_theory
  predicate
  formula
  subset
  power_set
  union
}

\begin{problem}

Six assertions about sets are bulleted below.  There are seven
predicate formulas that express some of these assertions.  Write the
number of each formula next to the bulleted assertion it expresses.
For example, you should write ``2'' next to the first assertion, since
formula 2.\ expresses the assertion that $x = y$.  Write ``none'' next
to an assertion that no formula expresses.

Variables $x,y,z\dots$ range over sets.  More than one formula may
express the same bulleted assertion.

\begin{quote}

\begin{itemize}

\item $x = y$. \hfill\examrule[1in]
\begin{solution}
2
\end{solution}

%\item $x = \emptyset$. \hfill\examrule[1in]

\item $x = \set{y,z}$. \hfill\examrule[1in]
\begin{solution}
1
\end{solution}

\item $x \subseteq y$. \hfill\examrule[1in]
\begin{solution}
3,7
\end{solution}

\item $x = y \union z$. \hfill\examrule[1in]
\begin{solution}
4
\end{solution}

%\item $x = y - z$. \hfill\examrule[1in]

%\item $x = \power(y)$. \hfill\examrule[1in]

\item $\card{x} \leq 3$. \hfill\examrule[1in]
\begin{solution}
5
\end{solution}

\item $\card{x} > 3$. \hfill\examrule[1in]
\begin{solution}
6
\end{solution}

\end{itemize}

\end{quote}


\begin{enumerate}
%\item $\forall z.\, \QNOT(z\in x)$.   %x empty

\item $\forall w.\, w \in x \QIFF (w=y \QOR\ w = z)$. %x = {y,z}

\item $\forall z.\, (z \in x\ \QIFF\ z \in y)$.  %x=y

\item $\forall z.\, z \in x\ \QIMPLIES\ z \in y$.  %x \subseteq y

\item $\forall w.\, w \in x \QIFF (w \in y\ \QOR\ w \in z)$.  %x = y U z

%\item $\forall w.\, w \in x\ \QIFF\  (w \in y\ \QAND\ \QNOT(w \in z))$. %x = y - z

%\item $\exists z.\, (y = x \union z)$.  %$x \subseteq y$

\item $\exists x_1,x_2,x_3.\, \forall z.\, z \in x \QIMPLIES (z = x_1 \QOR z = x_2 \QOR z = x_3)$.
 %$\card{x} \leq 3$

\item
$\forall x_1,x_2,x_3.\, \exists z.\, z \in x \QAND\ z \neq
  x_1\ \QAND\ z \neq x_2\ \QAND\ z \neq x_3$  %$\card{x} > 3$.

\item $(x - y) = \emptyset$. %$x \subseteq y$

%\item $\forall z.\, z \in x \QIFF z \subseteq y$. % $x = \power(y)$.

\end{enumerate}

\end{problem}

\endinput
