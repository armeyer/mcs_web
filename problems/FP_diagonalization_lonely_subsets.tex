\documentclass[problem]{mcs}

\begin{pcomments}
  \pcomment{FP_diagonalization_lonely_subsets}
  \pcomment{zabel, 10/28/17 for f17 midterm3conflict}
\end{pcomments}

\pkeywords{
  countable
  union
  powerset
  uncountable
  diagonalization
}

%%%%%%%%%%%%%%%%%%%%%%%%%%%%%%%%%%%%%%%%%%%%%%%%%%%%%%%%%%%%%%%%%%%%%
% Problem starts here
%%%%%%%%%%%%%%%%%%%%%%%%%%%%%%%%%%%%%%%%%%%%%%%%%%%%%%%%%%%%%%%%%%%%%

\begin{problem}
  Call a subset of $\mathbb{N}$ \emph{lonely} if it doesn't contain any pair of consecutive integers. For example, the set $\{1,5,25,125,\ldots\}$ of powers of $5$ is lonely, but the set of primes $\{2,3,5,7,11,\ldots\}$ is not lonely because it contains both $2$ and $3$.

  If $F$ is the set of lonely subsets of $\mathbb{N}$, show that $F$ is uncountable.
\end{problem}

\begin{solution}
 
  \begin{proof}
    We'll argue by diagonalization with ``slope -1/2'' as we saw in class. For the sake of contradiction, assume $F$ is countable. Since $F$ is certainly infinite, it must be in bijection with $\mathbb{N}$, so we can write $F = \{s_0, s_1, s_2, \ldots\}$. Define a new subset $t\subseteq\mathbb{N}$ of \emph{only even integers} as follows:
    \begin{equation*}
      t = \{2n\in\mathbb{N} \mid n\in\mathbb{N} \QAND 2n\notin s_n\}.
    \end{equation*}
    Because $t$ only has even integers, it is lonely, so $t\in F$. On the other hand, $t\ne s_k$ for $k\in\mathbb{N}$, because one of $t$ and $s_k$ contains $2k$ and the other does not, by construction. Thus, $t\notin\{s_0,s_1,s_2,\ldots\}$, i.e., $t\notin F$. This is a contradiction, so our assumption that $F$ is countable must be false, i.e., $F$ is countable.
  \end{proof}

  A similar, alternate proof does not rely on diagonalization at all.

  \newcommand\powN{\pow(\mathbb{N})}
  
  \begin{proof}
    We'll show that $\powN \inj F$, which suffices because we already know that $\powN$ is uncountable. Define the function $g: \powN\to F$ that simply doubles all entries of a given set: for any $s\subseteq\mathbb{N}$,
    \begin{equation*}
      g(s) \eqdef \{2k \mid k\in s\}.
    \end{equation*}
    The set $g(s)$ contains only even integers and therefore belongs to $F$. We can see that $g$ is total by definition. Furthermore, $g$ is injective, because if $s,t\subset\mathbb{N}$ are different, then there is some integer $n$ with $n\in s$ and $n\notin t$ (or vice versa), which implies $2n\in g(s)$ and $2n\notin g(t)$ (or vice versa), so $g(s)\ne g(t)$. So $g$ is a total injective relation, meaning $\powN \inj F$, as required.
  \end{proof}
\end{solution}
  


%%%%%%%%%%%%%%%%%%%%%%%%%%%%%%%%%%%%%%%%%%%%%%%%%%%%%%%%%%%%%%%%%%%%%
% Problem ends here
%%%%%%%%%%%%%%%%%%%%%%%%%%%%%%%%%%%%%%%%%%%%%%%%%%%%%%%%%%%%%%%%%%%%%

\endinput
