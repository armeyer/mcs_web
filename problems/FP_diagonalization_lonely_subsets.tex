\documentclass[problem]{mcs}

\begin{pcomments}
  \pcomment{FP_diagonalization_lonely_subsets}
  \pcomment{references PS_off_diagonal_arguments}
  \pcomment{overlaps FP_diagonalization_lonely_sequences FP_uncountable_sparse1s}
  \pcomment{zabel, 10/28/17}
\end{pcomments}

\pkeywords{
  countable
  union
  powerset
  uncountable
  diagonalization
}

 \newcommand\powN{\power(\nngint)}

%%%%%%%%%%%%%%%%%%%%%%%%%%%%%%%%%%%%%%%%%%%%%%%%%%%%%%%%%%%%%%%%%%%%%
% Problem starts here
%%%%%%%%%%%%%%%%%%%%%%%%%%%%%%%%%%%%%%%%%%%%%%%%%%%%%%%%%%%%%%%%%%%%%

\begin{problem}
  A subset of the nonnegative integers \nngint\ is called
  \emph{lonely} when it doesn't contain any pair of consecutive
  integers.  For example, the set $\set{1,5,25,125,\dots}$ of powers
  of $5$ is lonely, but the set of primes $\set{2,3,5,7,11,\dots}$ is
  not lonely because it contains both $2$ and $3$.

  Let $L$ be the set of lonely subsets of $\nngint$.  Show that $L$ is
  uncountable.
\end{problem}

\inhandout{\hint Instead of a $45^o$ diagonal, try $30^o$;
  alternatively, describe a \inj\ or \surj\ relation with
  $\powN$.}

\begin{solution}
 
  \begin{proof}
   We'll argue by diagonalization with a $30^o$ diagonal, that is the
   line $y=2x$, as in Problem~\bref{PS_off_diagonal_arguments}.  For
   the sake of contradiction, assume $L$ is countable.  Since $L$ is
   certainly infinite, it must be in bijection with $\nngint$, so we
   can write $L = \set{s_0, s_1, s_2, \dots}$.  Define a new subset
   $t \subset \nngint$ of \emph{only even integers} as follows:
   \begin{equation*}
    t = \set{2n \suchthat n \in \nngint \QAND 2n \notin s_n}.
   \end{equation*}
   Because $t$ only has even integers, it is lonely and so is in $L$.
   On the other hand, $t \ne s_k$ for $k\in\nngint$, because one of
   $t$ and $s_k$ contains $2k$ and the other does not, by
   construction.  Thus, $t\notin \set{s_0,s_1,s_2,\dots}$, that is,
   $t\notin L$.  This is a contradiction, so our assumption that $L$
   is countable must be false, proving that $L$ is uncountable.
  \end{proof}

  A similar, alternate proof does not rely on diagonalization at all.

  \begin{proof}
   We'll show that $\powN \inj L$, which suffices because we already
   know that $\powN$ is uncountable.  Define the function $g:\powN\to
   L$ that simply doubles all entries of a given set: for any
   $s \subseteq \nngint$,
   \[
   g(s) \eqdef \set{2k \suchthat k\in s}.
   \]
   The function $g$ is total by definition, and the set $g(s)$
   contains only even integers and therefore belongs to $L$.
   Furthermore, $g$ is injective, because if $s,t \subseteq \nngint$
   are different, then there is some integer $n$ with $n\in
   s\ QIFF\ n\notin t$, which implies $2n \in g(s) \QIFF\ 2n \notin
   g(t)$, and therefore $g(s)\ne g(t)$.  So $g$ is a total injective
   relation, meaning $\powN \inj L$, as required.
  \end{proof}
\end{solution}


%%%%%%%%%%%%%%%%%%%%%%%%%%%%%%%%%%%%%%%%%%%%%%%%%%%%%%%%%%%%%%%%%%%%%
% Problem ends here
%%%%%%%%%%%%%%%%%%%%%%%%%%%%%%%%%%%%%%%%%%%%%%%%%%%%%%%%%%%%%%%%%%%%%

\endinput
