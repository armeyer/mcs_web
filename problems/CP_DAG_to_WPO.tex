\documentclass[problem]{mcs}

\begin{pcomments}
  \pcomment{from: S09.cp7r}
  \pcomment{from: F06.cp5m}
  \pcomment{commented out in S09 - proofread before using}
  \pcomment{This problem could potentially be revised to use the
            positive path relation and strict partial orders instead
            or in addition to the path relation and weak POs.}
\end{pcomments}

\pkeywords{
  DAGs
  partial_orders
  transitive_closure
  digraphs
  relations
}

%%%%%%%%%%%%%%%%%%%%%%%%%%%%%%%%%%%%%%%%%%%%%%%%%%%%%%%%%%%%%%%%%%%%%
% Problem starts here
%%%%%%%%%%%%%%%%%%%%%%%%%%%%%%%%%%%%%%%%%%%%%%%%%%%%%%%%%%%%%%%%%%%%%

\begin{problem}
Prove that if $R$ is the graph of a DAG, then the path relation, $R^*$, is
a weak partial order on the vertices.

\solution{ \emph{reflexive}: Every vertex, $a$, is connected to itself by
a 0-length path, so $aR^*a$ holds for all $a$.

\emph{antisymmetric}: Suppose $a\neq b$ and $aR^*b$, that is, there is a
(necessarily positive length) path from $a$ to $b$.  If $bR^*a$ also held,
then there is also a path from $b$ to $a$, and the two paths would be a
cycle (though not necessarily a simple cycle).  Since the graph is
acyclic, we conclude that $\neg(bR^*a)$, which proves antisymmetry.

\emph{transitive}: $aR^*b$ means there is a path from $a$ to $b$, and
likewise, $bR^*c$ means there is a path from $b$ to $c$.  Then the first
path followed by the second would form a path from $a$ to $c$, proving
that $aR^*c$, which proves transitivity.}

\end{problem}

%%%%%%%%%%%%%%%%%%%%%%%%%%%%%%%%%%%%%%%%%%%%%%%%%%%%%%%%%%%%%%%%%%%%%
% Problem ends here
%%%%%%%%%%%%%%%%%%%%%%%%%%%%%%%%%%%%%%%%%%%%%%%%%%%%%%%%%%%%%%%%%%%%%
