\documentclass[problem]{mcs}

\begin{pcomments}
  \pcomment{FP_infinite_sequence_injection}
  \pcomment{slightly related to PS_unit_interval}
\end{pcomments}

\pkeywords{
  bijection
  surjection
  infinite_sequence
}

%%%%%%%%%%%%%%%%%%%%%%%%%%%%%%%%%%%%%%%%%%%%%%%%%%%%%%%%%%%%%%%%%%%%%
% Problem starts here
%%%%%%%%%%%%%%%%%%%%%%%%%%%%%%%%%%%%%%%%%%%%%%%%%%%%%%%%%%%%%%%%%%%%%

\begin{problem}
Let $\mathcal{S} \eqdef \infstrings{\set{\texttt{1},\texttt{2},\texttt{3}}}$ be the set of
infinite sequences of the digits \texttt{1},\texttt{2}, and
\texttt{3}.  Some representative sequences in $S$ are
\begin{align*}
&\texttt{123123123}\dots\\
&\texttt{1223331111222223}\dots\\
&\texttt{222222222222}\dots
\end{align*}

\bparts

\ppart\label{interlv} Define a total injective function $f:
\mathcal{S} \times \mathcal{S} \to \mathcal{S}$.

\examspace[1.5in]

\begin{solution}
Let $f$ be the interleaving operation on two sequences, that is
\[
f(x_0x_1x_2x_3x_4x_5\dots, y_0y_1y_2y_3y_4y_5) \eqdef x_0 y_0 x_1 y_1 x_2 y_2 x_3 \dots.
\]
\end{solution}

\ppart Explain why these is a bijection between $\mathcal{S}$ and
$\mathcal{S} \times \mathcal{S}$.  (You need not explicitly define the
bijection.)

\examspace[1.0in]

\begin{solution}
We have $\mathcal{S} \times \mathcal{S} \inj \mathcal{S}$ by
part~\eqref{interlv}.  Also, it is easy to see that $\mathcal{S} \inj
\mathcal{S} \times \mathcal{S}$; mapping a sequence $s$ to the pair
$(s,\texttt{1111}\dots)$ is a total injective relation from
$\mathcal{S}$ to $\mathcal{S} \times \mathcal{S}$.  The
Schr\"oder-Bernstein Theorem~\bref{S-B_thm} now implies that
$\mathcal{S} \times \mathcal{S} \bij \mathcal{S}$.
\end{solution}

\eparts

\end{problem}
%%%%%%%%%%%%%%%%%%%%%%%%%%%%%%%%%%%%%%%%%%%%%%%%%%%%%%%%%%%%%%%%%%%%%
% Problem ends here
%%%%%%%%%%%%%%%%%%%%%%%%%%%%%%%%%%%%%%%%%%%%%%%%%%%%%%%%%%%%%%%%%%%%%

\endinput
