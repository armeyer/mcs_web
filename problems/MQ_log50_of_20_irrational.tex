\documentclass[problem]{mcs}

\begin{pcomments}
  \pcomment{MQ_log50_of_20_irrational}
  \pcomment{variant of MQ_log12_of_18_irrational}
  \pcomment{for conflict midterm 9-24-15}
\end{pcomments}

\pkeywords{
  irrational
  power 
  contradiction
  log
}

%%%%%%%%%%%%%%%%%%%%%%%%%%%%%%%%%%%%%%%%%%%%%%%%%%%%%%%%%%%%%%%%%%%%%
% Problem starts here
%%%%%%%%%%%%%%%%%%%%%%%%%%%%%%%%%%%%%%%%%%%%%%%%%%%%%%%%%%%%%%%%%%%%%

\begin{problem} Prove that $\log_{50} 20$ is irrational.
%\hint Proof by contradiction.

\begin{solution}
\begin{proof}
  Suppose to the contrary that
  \[
  \textcolor{red}{\log_{50} 20 = \frac{m}{n}}
  \]
  for some integers $m, n$ where $n>0$.  So we have
\begin{align}
%  \log_9 12 & = m/n, \notag\\
  50^{\log_{50} 20} & = 50^{m/n}  & \text{(raising 50 to equal powers)},\notag\\
  20 & = 50^{m/n} & \text{(def of $\log_{50}$)},\notag\\
  20^n & = 50^m & \text{(raising both sides to the $n$th power)}.\notag\\
  (5\cdot 2^{2})^n & = (5^2 \cdot 2)^m & \text{(factoring 20 \& 50 into primes)}.\notag\\
  5^n\cdot 2^{2n} & = 5^{2m}\cdot 2^m. \label{factoring}
\end{align}
Now we have two cases:

\inductioncase{Case 1}: $(n \neq 2m)$.  There are different numbers of
five's on the left and right hand sides of equation~\eqref{factoring},
which contradicts the Unique Factorization Theorem.

\inductioncase{Case 2}: $(n = 2m)$.  There are $4m$ two's on the
left hand side of~\eqref{factoring} and $m$ two's on the right
hand side, which contradicts the Unique Factorization Theorem.

In any case there is a contradiction, which implies that $\log_{50}
20$ must be irrational.

\medskip

An alternative to the argument by cases is to assume (for the sake of
contradiction) that $\textcolor{red}{n = 2m}$.  Then
$\textcolor{red}{\log_{50} 20 = 1/2}$, which is false (since $50^{1/2}
\neq 20$).  This implies that Case 1 is in fact the only possibility.
\end{proof}
\end{solution}
\end{problem}

%%%%%%%%%%%%%%%%%%%%%%%%%%%%%%%%%%%%%%%%%%%%%%%%%%%%%%%%%%%%%%%%%%%%%
% Problem ends here
%%%%%%%%%%%%%%%%%%%%%%%%%%%%%%%%%%%%%%%%%%%%%%%%%%%%%%%%%%%%%%%%%%%%%

\endinput
