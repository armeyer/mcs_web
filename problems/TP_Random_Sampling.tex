\documentclass[problem]{mcs}

\begin{pcomments}
    \pcomment{TP_Random_Sampling}
    \pcomment{Converted from sampling-conceptual.scm
              by scmtotex and dmj
              on Sun 13 Jun 2010 05:25:50 PM EDT}
\end{pcomments}

\begin{problem}

%% type: multi-part
%% title: Random Sampling

You want to estimate the fraction $p$
of voters in the nation who will vote to re-elect the current president in the upcoming election.  You do this by random sampling (with replacement).  Specifically, you select
$n$ voters independently and randomly, ask them who they are going to
vote for, and use the fraction $P$ of those that say they will vote
for the current president as an estimate for $p$.

\bparts

\ppart
%% type: short-answer
%% title: Facts

Our theorems about sampling and distributions allow us to calculate
how confident we can be that the random variable $P$ takes a value
near the constant $p$.  This calculation uses some facts about voters
and the way they are chosen.  Which of the following facts are true?

\begin{enumerate}

\item
Given a particular voter, the probability of that voter preferring the president is~$p$.

\begin{solution}
False.  Each voter will either prefer or not prefer the president.
\end{solution}

\item
Given a particular voter, the probability of that voter preferring the
President is 1 or~0.

\begin{solution}
True.  Each voter will either prefer or not prefer the president.
\end{solution}

\item
The probability that some voter is chosen more than once in the
sequence goes to zero as $n$ increases.

\begin{solution}
False.  It approaches 1 (the Birthday paradox).
\end{solution}

\item
All voters are equally likely to be selected as the third in our
sequence of $n$ choices of voters (assuming $n \ge 3$).

\begin{solution}
True.  This is true by definition of random selection from a set (with replacement).
\end{solution}

\item
The probability that the second voter chosen will favor the President,
given that the first voter chosen prefers the President, is greater
than~$p$.

\begin{solution}
False. The first and second voters' preferences are independent (given $p$).
\end{solution}

\item
The probability that the second voter chosen will favor the President,
given that the second voter chosen is from the same state as the
first, may not equal~$p$.

\begin{solution}
False.  This would be true if we were given that the first voter was from a particular state, but we are not.
\end{solution}

\item
The probability that the second voter chosen will favor the President,
given that the second voter chosen is from Massachusetts, may not equal~$p$.

\begin{solution}
True.  The distribution of preferences in Massachusetts may not be the same as the national distribution.
\end{solution}

\end{enumerate}

\ppart
%% type: short-answer
%% title: What do you say?

Suppose that, according to your calculations the following is true
about your polling:

\begin{equation*}
    \pr{|P-p| \le 0.04} \ge 0.95
\end{equation*}

\ppart
You do the asking, you count how many said they will vote for the
President, you divide by $n$, and find that $P=0.53$.  You call the President to give him your results.  Which of the following are true?

\begin{enumerate}

\item Mr.\ President, $p=0.53$!

\begin{solution}
False.  The only way to know the exact value of the
constant $p$ is to ask all 250,000,000 voters.
\end{solution}

\item\label{probability}
Mr.\ President, with probability at least 95\%, $p$ is within 0.04 of
0.53.

\begin{solution}
False.  $p$ is a \emph{constant} which can either
be or not be within 0.04 of 0.53.  If it is, then the probability that
it is is~1, and thus at least 0.95, and therefore (2) will be true. If
it is not, then the probability that it is is~0, and thus smaller than
0.95, and therefore (2) will be false.
\end{solution}

\item
Mr.\ President, either $p$ is within 0.04 of 0.53 or something very
strange (5-in-100) has happened.

\begin{solution}
True. To see why, start with the statement
\begin{quote}
    \textbf{\emph{Either}}  $|0.53 - p| \le  0.04$  \textbf{\emph{or}}  $|0.53 - p| > 0.04$.
\end{quote}
which is obviously true.  Now read it as follows: \emph{Either} $p$ is
within 0.04 of 0.53 \emph{or} it is not and therefore my random
variable $P$ took a value from a set that is hit only 5 times in~100.
So, clearly, \emph{either} $p$ is within 0.04 of 0.53 \emph{or}
something strange has happened.
\end{solution}

\item
Mr.\ President, we can be 95\% confident that $p$ is within 0.04 of 0.53.

\begin{solution}
True. By rephrasing \ref{probability} as ``confidence'' rather than
probability, you are correctly indicating that you are talking about
the probable behavior of your methodology for sampling $p$, not the
actual value of~$p$
\end{solution}

\end{enumerate}

% Give the list of numbers that correspond to statements that you are
% justified to say.

\eparts

\end{problem}

\endinput
