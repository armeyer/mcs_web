\documentclass[problem]{mcs}

\begin{pcomments}
\pcomment{PS_roulette_expectation}
\pcomment{F95.ps11}
\pcomment{edited by ARM 5/9/12}
\end{pcomments}

\pkeywords{
 probability
 expectation
 linearity
}


%%%%%%%%%%%%%%%%%%%%%%%%%%%%%%%%%%%%%%%%%%%%%%%%%%%%%%%%%%%%%%%%%%%%%
% Problem starts here
%%%%%%%%%%%%%%%%%%%%%%%%%%%%%%%%%%%%%%%%%%%%%%%%%%%%%%%%%%%%%%%%%%%%%

\begin{problem}
A wheel-of-fortune has the numbers from $1$ to $2n$ arranged in a
circle.  The wheel has a spinner, and a spin randomly determines the
two numbers at the opposite ends of the spinner.  How would you
arrange the numbers on the wheel to maximize the expected value of:

\bparts
\ppart the sum of the numbers chosen?

\begin{solution}
It makes no difference how you arrange the numbers, the expected value
is the same.  Why?  If we label the ends of the spinner we can define
$2$ random variables $X$ and $Y$.  $X$ is the number pointed to by one
end of the spinner and $Y$ is the other number.  We wish to maximize
$\expect{X+Y}$.  But since $\expect{X+Y} = \expect{X} + \expect{Y}$
and $\expect{X} = \expect{Y} = (2n+1)/2$, all arangements give the
same expected value of $2n+1$.
\end{solution}

\ppart the product of the numbers chosen?

\begin{solution}
The best way to do this is to ensure that successive integers are
opposite each other, that is, $1$ is opposite $2$, $3$ is opposite $4,
\dots$, and $2n-1$ is opposite $2n$.  Let's call this the successor
arrangement.  We claim the successor arrangement maximizes the
expected value of the product.  To prove this, we'll show that any
arrangement that differs from the successor arrangement can be altered
to have a larger expected product.  This implies that the successor
arrangement must be the unique arrangement with maximum expected
product.

Given any arrangement, let's say a number is OK if it is opposite the
same number as in the successor arrangement.  So in any arrangement,
opposite numbers are both OK or neither is OK.  Likewise, any other
arrangement, numbers that are opposite in the successor arrangement
are either both OK or neither is OK.

Now any given arrangement not equal to the successor arrangement must
have a smallest number, $k$, that is not OK.  Since $k$ is the
smallest number that is not OK, it must be oppposite some larger not
OK number $p$.  Also, $k$ must be odd, or else $k-1$ would be a
smaller number that is not OK.  Since $k$ is odd, it is opposite $k+1$
in the successor arrangement, and therefore $k+1$ in not OK and $p
\neq k+1$, which implies $p>k+1$.  This also means that $k+1$ is
opposite some $q$ that is also not OK.

Now create a new arrangement by putting $k$ opposite $k+1$ and $p$
opposite $q$, leaving all the other opposites unchanged.  The new
arrangement will have a larger expected product than the given one.

To prove this, note that since all pairs are equally likely when we
turn the spinner, it suffices to show that the sum of the products of
the switched numbers is larger in the second arrangement, that is,
that
\[
kp + (k+1)q < k(k+1)+pq.
\]
This holds iff
\[
k(k+1)+pq - (kp + (k+1)q) > 0.
\]
But the left hand term simplifies to $(k-q)(k-(p-1))$, which is
greater than zero since $k$ is smaller than both $q$ and $p-1$.
\end{solution}

\eparts
\end{problem}


%%%%%%%%%%%%%%%%%%%%%%%%%%%%%%%%%%%%%%%%%%%%%%%%%%%%%%%%%%%%%%%%%%%%%
% Problem ends here
%%%%%%%%%%%%%%%%%%%%%%%%%%%%%%%%%%%%%%%%%%%%%%%%%%%%%%%%%%%%%%%%%%%%%

\endinput
