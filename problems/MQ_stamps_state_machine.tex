\documentclass[problem]{mcs}

\begin{pcomments}
  \pcomment{MQ_stamps_state_machine}
  \pcomment{from: S10.mq3 by Rich}
\end{pcomments}

\pkeywords{
  state_machines
  unreachable_states
  increasing_decreasing_variables
}

%%%%%%%%%%%%%%%%%%%%%%%%%%%%%%%%%%%%%%%%%%%%%%%%%%%%%%%%%%%%%%%%%%%%%
% Problem starts here
%%%%%%%%%%%%%%%%%%%%%%%%%%%%%%%%%%%%%%%%%%%%%%%%%%%%%%%%%%%%%%%%%%%%%

\begin{problem}
  A student-made stamp vending machine that only sells 4 and 7 cent 
  stamps has a rather strange interface for selecting how many stamps
  to output. The user may choose to:
  
  % Let $a$ be the number of 3-cent stamps and $b$ be the number of 7-cent
  %   stamps. The user may choose to:
  
  \begin{enumerate}
    \item[(i)]  Add \textit{one} 4-cent stamp, or
    \item[(ii)] remove \textit{two} 4-cent AND \textit{two} 7-cent
      stamps (when this is possible).
  \end{enumerate}
  
  \bparts
  \ppart
  
  Let $A$ be the number of 4-cent stamps; and $B$ be the number of
  7-cent stamps.  The chart below indicates properties of some derived
  variables; fill it in.
  
  \[\begin{array}{| l | c | c | c | c |}
      \hline
      \text{derived variables:}    & 4A+7B & 4*A-7*B & \rem{B}{2} & \rem{4*A+7*B}{2}\\
      \hline
      \textbf{weakly increasing}   & & & &  \\ \hline
      \textbf{strictly increasing} & & & &  \\ \hline
      \textbf{weakly decreasing}   & & & &  \\ \hline
      \textbf{strictly increasing} & & & &  \\ \hline
      \textbf{constant}            & & & &  \\ \hline
      \hline
  \end{array}\]
  
  \begin{solution}
    TBA
  \end{solution}
  
  \ppart\label{stampinvars}
  
  Circle the properties below that are preserved invariants:
  
  \begin{enumerate}
    \item The number of 7-cent stamps ($B$) must be even.
    \item The number of 7-cent stamps ($B$) must be greater than 0.
    \item The total cost of stamps ($4*A+7*B$) must be odd.
    \item $4A > 7B$.

  \end{enumerate}
  
  \begin{solution}
    TBA
  \end{solution}
  
  \ppart
  
  Using the Invariant Principle, show that it is impossible to output
  stamps with a total value of exactly 90 cents when the machine
  starts with exactly 211 7-cent stamps.  (You may use without proof
  the preserved invariance of some of the properties from
  part~\eqref{stampinvars}.)
  
  \begin{solution}
    TBA
  \end{solution}
  
  \eparts
\end{problem}

%%%%%%%%%%%%%%%%%%%%%%%%%%%%%%%%%%%%%%%%%%%%%%%%%%%%%%%%%%%%%%%%%%%%%
% Problem ends here
%%%%%%%%%%%%%%%%%%%%%%%%%%%%%%%%%%%%%%%%%%%%%%%%%%%%%%%%%%%%%%%%%%%%%

\endinput
