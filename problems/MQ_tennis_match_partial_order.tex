\documentclass[problem]{mcs}

\begin{pcomments}
  \pcomment{MQ_tennis_match_partial_order}
  \pcomment{don't use with MQ_reindeer_games_partial_order}
\end{pcomments}
%%%%%%%%%%%%%%%%%%%%%%%%%%%%%%%%%%%%%%%%%%%%%%%%%%%%%%%%%%%%%%%%%%%%%
% Problem starts here
%%%%%%%%%%%%%%%%%%%%%%%%%%%%%%%%%%%%%%%%%%%%%%%%%%%%%%%%%%%%%%%%%%%%%

\begin{problem}
Tennis tournaments consists of a series of matches between two
players.  Usually the objective is to determine a single best player
among a set of players.  The organizers of the Math for Computer
Science tournament want to do more: they want to find a linear ranking
of all the players.  To avoid controversy, they want to avoid the
awkward situation of having a sequence of players each of whom beats
the next player in the sequence and then having last player beat the
first.  So the organizers will keep a running record of who beat whom
during the tournament and never allow simultaneous matches whose
outcomes could lead to an awkward situation.

Knowledge of binary relations can help the organizers in arranging the
tournament.  Namely, at any stage of the tournament, the organizers
have a record of who lost to whom.  Mathematically, we can say that
there is a binary relation, $L$, on players where $p\mrel{L}q$ means
that player $p$ lost a match to player $q$.  No awkward situations
means that the positive length path relation, $L^+$, of $L$ is a
strict partial order.  Which of the partial order properties listed
below correspond to the following properties of a stage of the
tournament:

Partial Order properties:
\begin{enumerate}
\item incomparable
\item comparable
\item maximal
\item maximum
\item minimal
\item minimum
\item a chain
\item a total order
\item a topological sort
\item an empty partial order
\item an antichain
\item a maximal antichain
\item irreflexive
\item reflexive
\end{enumerate}

\bparts
\ppart An unbeaten player is a \brule{1in} element of the partial order.
\begin{solution}
maximal
\end{solution}

\ppart A player who has lost every match he was in is a \brule{1in}
element of the partial order $L^+$.
\begin{solution}
minimal
\end{solution}

\ppart A player who is sure to rank first at the end of the tournament is
a \brule{1in} element of $L^+$.
\begin{solution}
maximum
\end{solution}

\ppart Two players who should not be matched in the next stage of the
tournament are \brule{1in} elements of $L^+$.
\begin{solution}
comparable
\end{solution}

\ppart A group of players whose rankings relative to each other is
completely known is \brule{1in}.
\begin{solution}
a chain, or a total order.
\end{solution}

\ppart An set of players any two of whom could be paired up to play
the next match is \brule{1in}.
\begin{solution}
an antichain
\end{solution}

\ppart The fact that no player loses to himself correspond to $L^+$ being
a(n) \brule{1in} partial order.  
\begin{solution}
irreflexive
\end{solution}

\ppart The final ranking at the end of the tournament must be
\brule{1in} of $L^+$.

\begin{solution}
a topological sort
\end{solution}

\eparts

For the following parts, assume there are 256 players in the tournament.

\bparts

\ppart What is the smallest number of matches that could possibly have been
played in a completed tournament? 
\begin{solution}
255
\end{solution}

\ppart Assuming each match takes an hour and matches are scheduled to be
played simultaneously on the hour, what is the smallest number of hours the
tournament could take? 
\begin{solution}
8
\end{solution}
\eparts
\end{problem}

\endinput
