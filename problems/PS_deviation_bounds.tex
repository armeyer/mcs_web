\documentclass[problem]{mcs}

\begin{pcomments}
  \pcomment{PS_deviation_bounds}
  \pcomment{from Velleman}
  \pcomment{S98.ps11. F97.ps11, S98.ps11, F96/ps12, F01.TP12}
\end{pcomments}

\pkeywords{
  Markov
  Chebyshev
  Chernoff
  variance
  expectation
  random_variable
}

%%%%%%%%%%%%%%%%%%%%%%%%%%%%%%%%%%%%%%%%%%%%%%%%%%%%%%%%%%%%%%%%%%%%%
% Problem starts here
%%%%%%%%%%%%%%%%%%%%%%%%%%%%%%%%%%%%%%%%%%%%%%%%%%%%%%%%%%%%%%%%%%%%%

\begin{problem}
Suppose you flip a fair coin 100 times.  The coin flips are all
mutually independent.  What is an upper bound on the probability that
the number of heads is at least 70\ldots
\begin{problemparts}

\problempart
\dots according to Markov's Inequality?

\begin{solution}
The expected number of heads of $50$.  So the probability that the
number of heads is at least $70$ is at most $50/70 = 0.71$.
\end{solution}

\problempart
\dots according to Chebyshev's Inequality?

\begin{solution}
Let $X_i$ be the random variable whose value is $1$ if the $i$th coin
flip is heads.  Then $\Var[X_i] = 1/2 - (1/2)^2 = 1/4$.  So $\Var[X_1
+ \cdots + X_100] = 100/4 = 25$.  The variance of the number of heads
is $100/4 = 25$, so the standard deviation is $5$.  So $70$ is four
times the standard deviation from the mean.  Since the distribution is
symmetric, the probability is at most $\frac{1}{2} \cdot \frac{1}{4^2}
= \frac{1}{32}$
\end{solution}

\problempart
\dots according to Chernoff's Bound?

\begin{solution}

We apply Chernoff`s bound with $c = 70/50 = 1.4$.  This gives us that
$\beta(1.4) = \ln(1.4) + 1/1.4 - 1 = 0.05076$ and that the probability is
at most $e^{-0.05076 \cdot 1.4 \cdot 50} = 0.0286$.

\end{solution}

\iffalse
\problempart according to the upper bound for binomial distributions
derived in lecture? 

\hint Get a strict upper bound by using the multiplicative factors of
$\frac{1-\alpha}{1-2\alpha}$ to account for the case that there are
strictly more than 70 heads, and $\paren{\frac{n}{e}}^{1/12n}$ to
adjust Stirling's approximation.)

\begin{solution}
\TBA{SOLN}
\end{solution}
\fi

\end{problemparts}

\end{problem}

\endinput
