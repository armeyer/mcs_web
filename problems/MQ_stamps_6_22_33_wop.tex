\documentclass[problem]{mcs}

\begin{pcomments}
  \pcomment{MQ_stamps_6_14_21_wop}
  \pcomment{F16.midterm1}
  \pcomment{ARM 2/24/16}
\end{pcomments}

\pkeywords{
  WOP
  stamp
  makeable
}


%%%%%%%%%%%%%%%%%%%%%%%%%%%%%%%%%%%%%%%%%%%%%%%%%%%%%%%%%%%%%%%%%%%%%
% Problem starts here
%%%%%%%%%%%%%%%%%%%%%%%%%%%%%%%%%%%%%%%%%%%%%%%%%%%%%%%%%%%%%%%%%%%%%
\begin{problem}
Prove using the Well Ordering Principle that, using 6\textcent,
22\textcent, and 33\textcent\ stamps, it is possible to make any
amount of postage over 72\textcent.  To save time, you may specify
\emph{assume without proof} that 72textcent, 73\textcent, \dots
100\textcent\ are all makeable, but you should clearly indicate which
of these assumptions your proof depends on.

\begin{solution}
\begin{proof}
Assume to the contrary that some amount of postage of
$72$\textcent\ or more is not makeable.  So by WOP, there will be a
\emph{least} unmakeable amount $m \geq 72$.  If we assume
72--77\textcent\ is makeable, then we can conclude that $m \geq 78$.
So $m-6 \geq 72$ and therefore is makeable, because if $m > k \geq
72$, then $k$ is makeable by definition of $m$.  Now since $m-6$ is
makeable, we can add a 6\textcent\ stamp and make $(m-6)+6 =
m$\textcent, contradicting the fact that $m$ is unmakeable.  So there
cannot be such a minimum $m$, which proves that all amounts $\geq
72$\textcent\ are makeable.
\end{proof}
\end{solution}

\end{problem}

%%%%%%%%%%%%%%%%%%%%%%%%%%%%%%%%%%%%%%%%%%%%%%%%%%%%%%%%%%%%%%%%%%%%%
% Problem ends here
%%%%%%%%%%%%%%%%%%%%%%%%%%%%%%%%%%%%%%%%%%%%%%%%%%%%%%%%%%%%%%%%%%%%%
\endinput
