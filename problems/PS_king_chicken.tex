\documentclass[problem]{mcs}

\begin{pcomments}
  \pcomment{PS_king_chicken}
  \pcomment{edited by ARM from FTL digraph chapter, soln revised 3/4/17}
  \pcomment{S15.ps6, S17.cp8m}
\end{pcomments}

\pkeywords{
  king_chicken
  digraph
  tournament
  out-degree
}

%%%%%%%%%%%%%%%%%%%%%%%%%%%%%%%%%%%%%%%%%%%%%%%%%%%%%%%%%%%%%%%%%%%%%
% Problem starts here
%%%%%%%%%%%%%%%%%%%%%%%%%%%%%%%%%%%%%%%%%%%%%%%%%%%%%%%%%%%%%%%%%%%%%

\begin{problem}
Chickens are rather aggressive birds that tend to establish dominance
over other chickens by pecking them---hence the term ``pecking
order.''  So for any two chickens in a farmyard, either the first
pecks the second, or the second pecks the first, but not both.  We say
that chicken~$u$ \emph{virtually pecks} chicken~$v$ if either:
\begin{itemize}

\item Chicken $u$ pecks chicken~$v$, or

\item Chicken $u$ pecks some other chicken~$w$ who in turn pecks
  chicken~$v$.

\end{itemize}
A chicken that virtually pecks every other chicken is called a
\emph{king chicken}.

We can model this situation with a \emph{chicken digraph} whose
vertices are chickens, with an edge from chicken~$u$ to chicken~$v$
precisely when $u$ pecks $v$.  In the graph in Figure~\ref{fig:6EE3},
three of the four chickens are kings.  Chicken~$c$ is not a king in
this example since it does not peck chicken~$b$ and it does not peck
any chicken that pecks chicken~$b$.  Chicken~$a$ \emph{is} a king
since it pecks chicken~$d$, who in turn pecks chickens $b$ and~$c$.

In general, a \emph{\idx{tournament digraph}} is a digraph with exactly one
edge between each pair of distinct vertices.

\begin{figure}[h]

\graphic{Fig_EE3}

\caption{A 4-chicken tournament in which chickens $a$, $b$ and~$d$
  are kings.}

\label{fig:6EE3}.

\end{figure}

\bparts

\ppart Define a 10-chicken tournament graph with a king chicken that
has outdegree 1.

\begin{solution}
1 pecks 2 and 2 pecks 3--10 and 3--10 peck 1. The directions of edges
amongst 3-10 are irrelevant.
\end{solution}

\ppart Describe a 5-chicken tournament graph in which every player is a king.

\begin{solution}
In the 5-chicken tournament graph illustrated in
Figure~\ref{fig:6EE4}, every vertex has out-degree 2.  (There are lots
of other such 5-vertex graphs where all vertices have out-degree 2.)

By the King Chicken Theorem of part~\eqref{kingthm} below, when all
vertices have the same out-degree, they are all kings.

\begin{figure}

\graphic{Fig_EE4}

\caption{A 5-chicken tournament in which every chicken is a king.}

\label{fig:6EE4}

\end{figure}

\end{solution}

\ppart\label{kingthm} Prove
\begin{theorem*}[King Chicken Theorem]%\label{thm:king_chicken}
Any chicken with maximum out-degree in a tournament is a king.
\end{theorem*}

\begin{staffnotes}
Students generally can prove this without help, but if they get stuck,
give them the Lemma to prove.
\end{staffnotes}

\begin{solution}
We will prove
\begin{lemma*}
If vertex $v$ is \emph{not} virtually pecked by vertex $u$, then
$\outdegr{v} > \outdegr{u}$.
\end{lemma*}
Notice that the Lemma immediately implies that if $u$ has maximum
out-degree, then there is no $v$ that is not virtually pecked by $u$,
that is, $u$ must be a king.

\begin{proof}
Let $P_u$ be the vertices pecked by vertex $u$.  So $\outdegr{u} =
\card{P_u}$.

If vertex $v$ is not virtually pecked by $u$, then it is not pecked by
$u$, and it is not pecked by any vertex in $P_u$.  This means $v$ must
peck $u$, and also $v$ must peck every vertex in $P_u$.  Therefore,
\[
\outdegr{v} \geq \card{\set{u} \union P_u} = 1+\outdegr{u} >
\outdegr{u}.
\]
\end{proof}
\end{solution}

The King Chicken Theorem means that if the player with the most
victories is defeated by another player~$x$, then at least he/she
defeats some third player that defeats~$x$.  In this sense, the player
with the most victories has some sort of bragging rights over every
other player.  Unfortunately, as Figure~\ref{fig:6EE3} illustrates,
there can be many other players with such bragging rights, even some
with fewer victories.
\eparts  

\end{problem}

%%%%%%%%%%%%%%%%%%%%%%%%%%%%%%%%%%%%%%%%%%%%%%%%%%%%%%%%%%%%%%%%%%%%%
% Problem ends here
%%%%%%%%%%%%%%%%%%%%%%%%%%%%%%%%%%%%%%%%%%%%%%%%%%%%%%%%%%%%%%%%%%%%%
 
