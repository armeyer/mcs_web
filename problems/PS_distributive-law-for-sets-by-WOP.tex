\documentclass[problem]{mcs}

\begin{pcomments}
  \pcomment{PS_distributive-law-for-sets-by-WOP}
  \pcomment{fall 07 ps2, variant of S05, ps2}
  \pcomment{revised by ARM to use WOP}
  \pcomment{revised ARM 2/17/13 to cite problem CP_distribute_union_over_intersection}
\end{pcomments}

\pkeywords{
  WOP
  sets
  intersection
  union
  distributive
}

%%%%%%%%%%%%%%%%%%%%%%%%%%%%%%%%%%%%%%%%%%%%%%%%%%%%%%%%%%%%%%%%%%%%%
% Problem starts here
%%%%%%%%%%%%%%%%%%%%%%%%%%%%%%%%%%%%%%%%%%%%%%%%%%%%%%%%%%%%%%%%%%%%%

\begin{problem}
Union distributes over the intersection of two sets:
\begin{equation}\label{distr}
A \union (B \intersect C) = (A \union B) \intersect (A \union C)
\end{equation}
(see Problem~\bref{CP_distribute_union_over_intersection}).

Use~\eqref{distr} and the Well-ordering Principle to prove the
Distributive Law of union over the intersection of $n$ sets:
\begin{equation}\label{AUBn}
  A \union (B_1 \intersect \cdots  \intersect B_{n-1} \intersect B_n)
  = (A \union B_1)  \intersect \cdots \intersect (A \union B_{n-1} ) \intersect (A \union B_n)
\end{equation}

Extending formulas to an arbitrary number of terms is a common (if
mundane) application of the WOP.

\begin{solution}
The proof is by contradiction.

Suppose to the contrary that~\eqref{AUBn} was not true for some $n \geq 2$ and
sets $A,B_1,\dots,B_n$.   Then by the WOP, there is a least such $m \in
\naturals$.  Now $m > 2$ by~\eqref{distr}.  Since $m> m-1 \geq 2$, we
have~\eqref{AUBn} at $m-1$, namely,
\begin{equation}\label{AUBm-1}
A \union (B_1 \intersect \cdots  \intersect B_{m-2} \intersect B_{m-1})
    = (A \union B_1) \intersect \cdots \intersect  (A \union B_{m-2})
    \intersect (A \union B_{m-1}).
\end{equation}
for all $A,B_1,\dots,B_{m-1}$.  But now

\begin{align*}
\lefteqn{A \union (B_1 \intersect \cdots \intersect B_{m-1} \intersect B_m)}\\
    & = A \union (B_1 \intersect \cdots \intersect B_{m-2} \intersect (B_{m-1} \intersect B_m)) & \text{(by def of $\intersect$)}\\
    & = A \union (B_1 \intersect \cdots
\intersect B_{m-2}\intersect C_{m-1}) & (\text{for } C_{m-1} \eqdef B_{m-1} \intersect B_m)\\
    & = (A \union B_1) \intersect \cdots
    \intersect (A \union B_{m-2}) \intersect (A \union C_{m-1}) &
    \text{(by~\eqref{AUBm-1} with $B_{m-1} = C_{m-1}$)}\\
    & = (A \union B_1) \intersect \cdots \intersect (A \union B_{m-2})
    \intersect (A \union (B_{m-1} \intersect B_m)) & \text{(def of $C_{m-1}$)}\\
    & = (A \union B_1) \intersect \cdots \intersect (A \union B_{m-2})
    \intersect ((A \union B_{m-1}) \intersect (A \union B_m)) & \text{(by~\eqref{distr})}\\
    & = (A \union B_1)  \intersect \cdots \intersect (A \union B_{m-1})
    \intersect (A \union B_m) & \text{(def of $\intersect$)}.
\end{align*}
This proves that~\eqref{AUBn} does hold for $n = m$, contradicting the
definition of $m$.

This contradiction implies that there cannot be an  $n \geq 2$ for
which~\eqref{AUBn} fails.  That is~\eqref{AUBn} holds for all $n \geq 2$.

\end{solution}
\end{problem}
