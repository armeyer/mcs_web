\documentclass[problem]{mcs}

\begin{pcomments}
  \pcomment{FP_simple_graphs_TF}
  \pcomment{from FP_multiple_choice_unhidden_fall13}
  \pcomment{revised from FP_multiple_choice_unhidden by ARM 12/13/13}
  \pcomment{overlaps FP_graphs_short_answer}  
\end{pcomments}

\pkeywords{
  vertices
  edge
  coloring
  chromatic_number
  big_Oh
}

%%%%%%%%%%%%%%%%%%%%%%%%%%%%%%%%%%%%%%%%%%%%%%%%%%%%%%%%%%%%%%%%%%%%%
% Problem starts here
%%%%%%%%%%%%%%%%%%%%%%%%%%%%%%%%%%%%%%%%%%%%%%%%%%%%%%%%%%%%%%%%%%%%%

\begin{problem} \mbox{}

\textbf{\large Circle \textbf{true} or \textbf{false} for the
  following statements about finite \textbf{simple graphs} $G$.}

\bparts

\ppart  $G$ has a spanning tree.  \hfill \textbf{true} \qquad
  \textbf{false}

\begin{solution}
\textbf{false.  Any disconnected graph is a counterexample.}
\end{solution}

\ppart $\card{\vertices{G}} = O(\card{\edges{G}})$ for connected $G$.  \hfill \textbf{true} \qquad
  \textbf{false}

\begin{solution}
\textbf{true.}
\end{solution}

\iffalse
\ppart $\card{\edges{G}} = O(\card{\vertices{G}})$.  \hfill \textbf{true} \qquad
  \textbf{false}

\begin{solution}
\textbf{false.}  $\card{\vertices{K_n}} = n = o(n^2)$, but
  $\card{\edges{K_n}} = \Theta(n^2)$.

\end{solution}
\fi

\ppart $\chi(G) \leq \max\set{\degr{v} \suchthat
  v \in \vertices{G}}$.\footnote{$\chi(G)$ is the chromatic number of $G$.}  \hfill \textbf{true} \qquad
  \textbf{false}

\begin{solution}
\textbf{false.}  $\chi(K_n) = n > n-1 = \text{ max degree of } v \in \vertices{K_n}$
\end{solution}

\ppart $\card{\vertices{G}} = O(\chi(G))$.  \hfill \textbf{true} \qquad
  \textbf{false} 

\begin{solution}
\textbf{false.}

There are arbitrarily large graphs, for example trees, with $\chi = 2$.
\end{solution}

\eparts

\end{problem}

\endinput
