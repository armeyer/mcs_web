\documentclass[problem]{mcs}

\begin{pcomments}
  \pcomment{TP_psets_and_laundry}
  \pcomment{formerly called PS_psets_and_laundry}
  \pcomment{from F08, pset12}
\end{pcomments}

\pkeywords{
  probability
  random_variable
  expectation
}

%%%%%%%%%%%%%%%%%%%%%%%%%%%%%%%%%%%%%%%%%%%%%%%%%%%%%%%%%%%%%%%%%%%%%
% Problem starts here
%%%%%%%%%%%%%%%%%%%%%%%%%%%%%%%%%%%%%%%%%%%%%%%%%%%%%%%%%%%%%%%%%%%%%
\begin{problem}
MIT students sometimes delay laundry for a few days.  Assume all
random values described below are mutually independent.

\bparts

\ppart A \term{busy} student must complete 3 problem sets before doing
laundry.  Each problem set requires 1 day with probability $2/3$ and 2
days with probability $1/3$.  Let $B$ be the number of days a busy
student delays laundry.  What is $\expect{B}$?

Example: If the first problem set requires 1 day and the second and
third problem sets each require 2 days, then the student delays for $B
= 5$ days.

\begin{solution} The expected time to complete a problem
set is:
%
\[
1 \cdot \frac{2}{3} + 2 \cdot \frac{1}{3} = \frac{4}{3}
\]
%
Therefore, the expected time to complete all three problem sets is:
%
\begin{align*}
\expect{B}
    & = \expect{\text{pset1}} + \expect{\text{pset2}} + \expect{\text{pset3}} \\
    & = \frac{4}{3} + \frac{4}{3} + \frac{4}{3} \\
    & = 4
\end{align*}
\end{solution}

\ppart A \term{relaxed} student rolls a fair, 6-sided die in the
morning.  If he rolls a 1, then he does his laundry immediately (with
zero days of delay).  Otherwise, he delays for one day and repeats the
experiment the following morning.  Let $R$ be the number of days a
relaxed student delays laundry.  What is $\expect{R}$?

Example: If the student rolls a 2 the first morning, a 5 the second
morning, and a 1 the third morning, then he delays for $R = 2$ days.

\begin{solution}If we regard doing laundry as a failure, then the mean time
to failure is $1 / (1/6) = 6$.  However, this counts the day laundry
is done, so the number of days delay is $6 - 1 = 5$.  Alternatively,
we could derive the answer as follows:
%
\begin{align*}
\expect{R}
    & = \sum_{k=0}^{\infty} \pr{R > k} \\
    & = \frac{5}{6} + \paren{\frac{5}{6}}^2 + \paren{\frac{5}{6}}^3 + \ldots \\
    & = \frac{5}{6} \cdot
            \paren{1 + \frac{5}{6} + \paren{\frac{5}{6}}^2 + \ldots} \\
    & = \frac{5}{6} \cdot \frac{1}{1 - 5/6} \\
    & = 5
\end{align*}
\end{solution}

\ppart Before doing laundry, an \term{unlucky} student must recover
from illness for a number of days equal to the product of the numbers
rolled on two fair, 6-sided dice.  Let $U$ be the expected number of
days an unlucky student delays laundry.  What is $\expect{U}$?

Example: If the rolls are 5 and 3, then the student delays for $U =
15$ days.

\begin{solution}Let $D_1$ and $D_2$ be the two die rolls.  Recall that a die
roll has expectation $7/2$.  Thus:
%
\begin{align*}
\expect{U}
    & = \expect{D_1 \cdot D_2} \\
    & = \expect{D_1} \cdot \expect{D_2} \\
    & = \frac{7}{2} \cdot \frac{7}{2} \\
    & = \frac{49}{4}
\end{align*}
\end{solution}

\ppart A student is \term{busy} with probability $1/2$, \term{relaxed}
with probability $1/3$, and \term{unlucky} with probability $1/6$.
Let $D$ be the number of days the student delays laundry.  What is
$\expect{D}$?

\begin{solution}
\[
\expect{D} = \frac{1}{2} \expect{B} + \frac{1}{3} \expect{R} + \frac{1}{6} \expect{U}
\]
\end{solution}

\eparts

\end{problem}

%%%%%%%%%%%%%%%%%%%%%%%%%%%%%%%%%%%%%%%%%%%%%%%%%%%%%%%%%%%%%%%%%%%%%
% Problem ends here
%%%%%%%%%%%%%%%%%%%%%%%%%%%%%%%%%%%%%%%%%%%%%%%%%%%%%%%%%%%%%%%%%%%%%

\endinput
