\documentclass[problem]{mcs}

\begin{pcomments}
  \pcomment{PS_combinatorial_proof_2}
  \pcomment{adapted from fall10 recitation 16.}
  \pcomment{hint added by ARM 11/16/13}
\end{pcomments}

\pkeywords{
  combinatorial_proof
  binomial_coefficient
}

%%%%%%%%%%%%%%%%%%%%%%%%%%%%%%%%%%%%%%%%%%%%%%%%%%%%%%%%%%%%%%%%%%%%%
% Problem starts here
%%%%%%%%%%%%%%%%%%%%%%%%%%%%%%%%%%%%%%%%%%%%%%%%%%%%%%%%%%%%%%%%%%%%%

\begin{problem}
Give a combinatorial proof for this identity:
\[
\sum_{i=0}^n \binom{k+i}{k} = \binom{k+n+1}{k+1}
\]

\hint Let $S_i$ be the set of binary sequences with exactly $n$
zeroes, $k + 1$ ones, and a total of exactly $i$ occurrences of zeroes
appearing before the rightmost occurrence of a one.

\begin{solution}
\begin{proof}
Let $S$ be the set of binary sequences with exactly $n$ zeroes and $k
+ 1$ ones.

We know that
\[
\card{S}= \binom{k + n + 1}{k+1}.
\]

On the other hand, since the total number of occurrences of zeroes
appearing before the rightmost occurrence of a one can range from 0 to
$n$, we have
\[
\card{S}= \sum_{i = 0}^n \card{S_i} .
\]

But every string in $S_i$ must have $n-i$ zeroes after the last
occurrence of a one.  Any pattern of $k$ ones and $i$ zeroes can occur
before the rightmost occurrence of a one, so it follows that
\[
\card{S_i} =\binom{k + i}{k},
\]
which implies
\[
\card{S}=  \sum_{i = 0}^n \binom{k + i}{k} .
\]
Equating these two expressions for $\card{S}$ proves the theorem.
\end{proof}

\end{solution}
\end{problem}
\endinput
