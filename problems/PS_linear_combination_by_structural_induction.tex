\documentclass[problem]{mcs}

\begin{pcomments}
\pcomment{PS_linear_combination_by_structural_induction}
\pcomment{based on S07.ps3.prob1 by ARM 9/20/11}
\end{pcomments}

\pkeywords{
  gcd
  linear_combination
  structural_induction
}

%%%%%%%%%%%%%%%%%%%%%%%%%%%%%%%%%%%%%%%%%%%%%%%%%%%%%%%%%%%%%%%%%%%%%
% Problem starts here
%%%%%%%%%%%%%%%%%%%%%%%%%%%%%%%%%%%%%%%%%%%%%%%%%%%%%%%%%%%%%%%%%%%%%

\begin{problem}
Let $m,n$ be integers, not both zero.  Define a set of integers,
$L_{m,n}$, recursively as follows:
\begin{itemize}
\item \textbf{Base cases}: $m,n \in L_{m,n}$.
\item \textbf{Constructor steps}:
If $j,k \in L_{m,n}$, then
\begin{enumerate}
\item $-j \in  L_{m,n}$,
\item $j+k \in L_{m,n}$.
\end{enumerate}
\end{itemize}
Let $L$ be an abbreviation for $L_{m,n}$ in the rest of this problem.

\begin{problemparts}

  \problempart \label{common} Prove \emph{by structural induction} that
  every common divisor of $m$ and $n$ also divides every member of $L$.

\begin{solution}
TBA (really trivial)
\end{solution}

\ppart\label{intmultL} Prove that any integer multiple of an element of
$L$ is also in $L$.

\begin{solution}
Let $a$ be an element of $L$ and define the predicate
\[
P(k) \eqdef ka \in L.
\]
We prove that $P(k)$ holds for all $k \in \naturals$ by induction on $k$,
with induction hypothesis, $P$.

\textbf{Base case} $(k=0)$: $0a = 0 = m + (-m) \in L$.

\textbf{Induction step}: Assume $P(k)$, so $ ka\in L$.  But we are given
that $a \in L$, so $(k+1)a = ka+a \in L$ by the second recursive step in
the definition of $L$.  This proves $P(k+1)$, completing the induction
proof.

Now since $ka \in L$ implies $-ka \in L$ by the first recursive step in
the definition of $L$, we conclude that $ka \in L$ for all $k \in \integers$
\end{solution}

\problempart\label{rem-closed} Show that if $j,k \in L$ and $k\neq 0$, then
$\rem{j}{k} \in L$.

\begin{solution}
Say $r$ is the remainder of $j$ divided by $k$.  That is, $j =
qk+r$ for some quotient $q \in \integers$ and remainder, $r$, where $0\leq
r < \card{k}$.

Now that $(-q)k \in L$ by part~\eqref{intmultL}, so $j + (-q)k \in L$ the
second recursive step in the definition of $L$.  That is, $r \in L$.
\end{solution}

\problempart \label{smallestg}
Show that there is a positive integer $g \in L$ which divides every
member of $L$.  \hint The least positive integer in $L$.

\begin{solution}
At least one of the integers $m, -m, n, -n \in L$ must be
positive.  Hence by the Well Ordering Principle, there is a least positive
integer $g \in L$.

Suppose $a \in L$.  We must show that $g \divides a$.

We prove this by contradiction: if $g$ does not divide $a$, then
$\rem{j}{g}$ is a positive integer which is in $L$ by
part~\eqref{rem-closed}. But $\rem{j}{g}$ is by definition less than $g$,
contradicting the minimality of $g$.
\end{solution}


\ppart Conclude that $g= \GCD(m,n)$ for $g$ from part~\eqref{smallestg}.
\begin{solution}
TBA
\end{solution}

\iffalse

\problempart \label{Lmx+ny}  Prove \emph{by structural induction} that
\[
L \subseteq \set{mx+ny \suchthat x,y \in \integers}.
\]

\begin{solution}
We need to prove that every number in $L$ is of the form $mx+ny$.
We do this by structural induction on the definition of $L$.

\textbf{Base cases}: The base cases are of the required form because
\begin{align*}
m & = m\cdot 1+ n \cdot 0,\\
n & = m\cdot 0+ n \cdot 1.
\end{align*}

\textbf{Constructor steps}: We must prove that $-j$ and $j+k$ are of the
required form for $j,k \in L$, where by structural induction hypothesis,
we may assume $j,k$ are of the required form.  But this follows
immediately since
\begin{align*}
-j  & = -(mx+ny) = m(-x)+ n(-y)\\
j+k & = (mx+ny) + (mx'+ny') = m(x+x')+n(y+y')
\end{align*}
This completes the structural induction.
\end{solution}

\problempart \label{mx+nyL}  Show that
\[
\set{mx+ny \suchthat x,y \in \integers} \subseteq L.
\]

\begin{solution}
 We must show that $mx+ny \in L$ for all integers $x,y$.  To
begin, suppose $j\in L$ and let
\[
P_j(k) \eqdef jk \in L.
\]
We prove that $P_j(k)$ holds for all $k \in \naturals$ by induction on $k$,
with induction hypothesis, $P_j$.

\textbf{Base case} $(k=0)$: $j0 = 0 = m + (-m) \in L$.

\textbf{Induction step}: Assume $P(k)$, so $jk \in L$.  But we are given
that $j \in L$, and $j(k+1) = jk+j \in L$ by the second recursive step in
the definition of $L$.  This proves $P(k+1)$, completing the induction
proof.

Now since $jk \in L$ implies $-jk \in L$ by the first recursive step in
the definition of $L$, we conclude
\begin{lemma*}
If $j\in L$, then $jx \in L$ for all $x \in \integers$.
\end{lemma*}
But $m,n \in L$ by definition of $L$, so the Lemma implies that $mx \in L$ for
$x \in \integers$, and also $ny \in L$ for all $y \in \integers$.  Now by
the second recursive step in the definition of $L$, we conclude that
$mx+ny \in L$ for all integers $x,y$.
\end{solution}


\problempart \label{common}
Conclude that any common divisor of $m$ and $n$ also divides every member
of $L$.

\begin{solution}
If $k \in L$, then by part~(\ref{Lmx+ny}), $k = mx+ny$.
Now if $d$ is a common divisor of $m$ and $n$, then it divides $mx$ and
$ny$, and hence divides their sum $mx+ny$.  So $d$ also divides $k$.
\end{solution}


\problempart Show that if $j,k \in L$ and $k\neq 0$, then the remainder of
$j$ divided by $k$ is in $L$.

\begin{solution}
Say $r$ is the remainder of $j$ divided by $k$.  That is, $j =
qk+r$ for some quotient $q \in \integers$ and remainder, $r$, where $0\leq
r < \card{k}$.

Now the Lemma in the solution to part~\ref{mx+nyL} implies that $(-q)k \in
L$, and then the second recursive step in the definition of $L$ implies
that $j + (-q)k \in L$.  That is, $r \in L$.
\end{solution}

\problempart \label{g}

Show that there is a positive integer $g \in L$ which divides every
member of $L$.  \hint The least positive integer in $L$.

\begin{solution}
At least one of the integers $m, -m, n, -n \in L$ must be
positive.  Hence by the Well Ordering Principle, there is a least positive
integer $g \in L$.

Suppose $j \in L$.  We must show that $g \divides j$.

We prove this by contradiction: if $g$ does not divide $j$, then the
remainder of $j$ divided by $g$ is a positive number smaller than $g$,
which by the previous part is in $L$, contradicting the fact that
$g$ is the smallest such integer.
\end{solution}

\problempart Conclude that $\gcd(m,n) = mx+ny$ for some $x,y \in \integers$.

\begin{solution}
From part~(\ref{g}), we have a positive integer $g \in L$ that
divides every element of $L$.  In particular, $g$ is a common divisor of
$m,n$, since $m,n \in L$.

Further, $g =mx+ny$ for some $x,y \in \integers$ by part~(\ref{Lmx+ny}), and
any common divisor of $m$ and $n$ divides $g$, by part~(\ref{common}).
So $g$ is a common divisor that is divided by, and hence at least as large
as, any common divisors.  So $g$ must be the \emph{greatest} common divisor
of $m$ and $n$.
\end{solution}
\fi

\end{problemparts}

\end{problem}


%%%%%%%%%%%%%%%%%%%%%%%%%%%%%%%%%%%%%%%%%%%%%%%%%%%%%%%%%%%%%%%%%%%%%
% Problem ends here
%%%%%%%%%%%%%%%%%%%%%%%%%%%%%%%%%%%%%%%%%%%%%%%%%%%%%%%%%%%%%%%%%%%%%

\endinput
