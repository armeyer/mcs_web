\documentclass[problem]{mcs}

\newcommand{\pairset}[2]{\text{pair}(#1,#2)}

\begin{pcomments}
  \pcomment{TP_set_pairing}
  \pcomment{ARM 2/18/13}
\end{pcomments}

\pkeywords{
  logic
  sets
  set_theory
  predicate
  formula
  union
}

\begin{problem}
Forming a pair $(a,b)$ of items $a$ and $b$ is a mathematical
operation that we can safely take for granted.  But when we're trying
to show how all of mathematics can be reduced to set theory, we need a
way to represent the pair $(a,b)$ as a set.

\begin{problemparts}

\ppart Explain why representing $(a,b)$ by $\set{a,b}$ won't work.

\begin{solution}
The order of the elements gets lost: $(a,b)$ and $(b,a)$ would have
the same representation.
\end{solution}

\ppart Explain why representing $(a,b)$ by $\set{a,\set{b}}$ won't work either.
\hint Suppose $a = \set{\set{b}}$.

\begin{solution}
$(a, b)$ and $(\set{b}, a)$ would have the same representation.
\end{solution}

\ppart Define
\[
\pairset{a}{b} \eqdef \set{a,\set{a,b}}.
\]
Explain why representing $(a,b)$ as $\pairset{a}{b}$ uniquely determines
$a$ and $b$.

\begin{solution}
Notice that $\set{a,b} \notin a$ because otherwise $a$ would
indirectly be a member of itself, namely $a \in \set{a,b} \in a$,
which sets don't do.\footnote{By the Foundation Axiom,
  Section~\bref{ZFC_sec}}.  So of the two elements in $\pairset{a}{b}$,
$a$ must be the element that is a member of the other one, and $b$
must be the element in the other one.
\end{solution}

\end{problemparts}
\end{problem}

\endinput
