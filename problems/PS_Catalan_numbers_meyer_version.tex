\documentclass[problem]{mcs}

\begin{pcomments}
  \pcomment{PS_Catalan_numbers_meyer_version}
  \pcomment{revised from S08.ps10 by ARM 11/20/09}
\end{pcomments}

\pkeywords{
  generating_functions
  Catalan
  Taylor_expansion
  Taylor_theorem
  recursive
  parentheses
}

%%%%%%%%%%%%%%%%%%%%%%%%%%%%%%%%%%%%%%%%%%%%%%%%%%%%%%%%%%%%%%%%%%%%%
% Problem starts here
%%%%%%%%%%%%%%%%%%%%%%%%%%%%%%%%%%%%%%%%%%%%%%%%%%%%%%%%%%%%%%%%%%%%%

\begin{problem}

  Generating functions provide an interesting way to count the number
  of strings of matched parentheses.  To do this, we'll use the
  description of these strings given in Definition~\bref{gc-def} as
  the set, $\GC$, of strings of parentheses with a \idx{good count}.
  Let $c_n$ be the number of strings in $\GC$ with exactly $n$ left
  parentheses, and let $C(x)$ be the generating function for these
  numbers:
  \[
  C(x) \eqdef c_0 + c_1 x + c_2 x^2 + \cdots.
  \]
  
\bparts

\ppart Call a string \emph{enclosed} if it starts with a left
parenthesis and ends with a right parenthesis.  Explain why the
generating function for the enclosed strings with a good count is
$xC(x)$.

\hint The mapping which takes a string $s$ to the enclosed string
$\mtt{(}s\mtt{)}$ is a bijection.

\begin{solution}
The mapping which takes a string $s$ to the enclosed string
$\mtt{(}s\mtt{)}$ is obviously a bijection from the set of strings to
the set of enclosed strings.  But if $s$ has a good count, then so
does $\mtt{(}s\mtt{)}$, so this mapping also defines a bijection from
strings with good counts that have exactly $n$ left parentheses to
enclosed strings with good counts that have exactly $n+1$ left
parentheses.  So letting $e_{n+1}$ be the number of enclosed
strings with a good count that have exactly $n$ left parentheses, we have
$e_{n+1} = c_n$.  Therefore,
\begin{align*}
xC(x) & = 0+ c_0x +c_1x^2 +\cdots+ c_nx^{n+1} +\cdots\\
      & = e_0 + e_1x +e_2x^2 +\cdots+ e_{n+1}x^{n+1} +\cdots,
\end{align*}
which shows that $xC(x)$ is the generating function for the enclosed
strings with a good count.
\end{solution}

\ppart\label{unique-enclosed-parse} Explain why, for every string, $s$,
with a good count, there is a unique sequence of enclosed strings
$s_1,\dots,s_k$ with good counts such that $s = s_1\cdots s_k$.  For
example, the string $r \eqdef \mtt{(())()(()())} \in \GC$ equals
$s_1s_2s_3$ where $s_1= \mtt{(())}, s_2 = \mtt{()}, s_3 = \mtt{(()())}$,
and this is the only way to express $r$ as a sequence of enclosed strings
with good counts.

\begin{solution}
TBA
\end{solution}

\ppart Conclude that
\begin{equation}\label{1xcxc2}
C = 1 + xC + (xC)^2 + \cdots + (xC)^n + \cdots,
\end{equation}
so
\begin{equation}\label{C11-xC}
C = \frac{1}{1-xC}\, ,
\end{equation}
and hence
\begin{equation}\label{Cpmsqrt}
C= \frac{1 \pm \sqrt{1 -4x}}{2x}\, .
\end{equation}
\begin{solution}
  Equation~\eqref{1xcxc2} follows from part~\eqref{unique-enclosed-parse}.
  Then~\eqref{C11-xC} follows since the irght hand side of~\eqref{1xcxc2}
  is a geometric series in $xC$.  Then~\eqref{Cpmsqrt} follows
  from~\eqref{C11-xC} using the quadratic formula.
\end{solution}

\eparts

Let $D(x) \eqdef 2xC(x)$.  Expressing $D$ as a power series
\[
D(x) = d_0 + d_1x +d_2 x^2 + \cdots,
\]
we have
\begin{equation}\label{cndn+1}
c_n  = \frac{d_{n+1}}{2}.
\end{equation}

\bparts

\ppart Use~\eqref{C11-xC}, \eqref{cndn+1}, and the value of $c_0$ to conclude that
\[
D(x)= 1 - \sqrt{1 -4x}.
\]

\begin{solution}
We have from~\eqref{C11-xC} that
\begin{equation}\label{Cpmsqrt}
C= \frac{1 \pm \sqrt{1 -4x}}{2x},
\end{equation}
so $D(x)$ is either $1 + \sqrt{1 -4x}$ or $1 - \sqrt{1 -4x}$.  Thus
$D'(x)$ is either $\frac{-2}{\sqrt{1 -4x}}$ or $\frac{2}{\sqrt{1 -4x}}$.
It follows that
  \[
  d_1 = D'(0) = \begin{cases}
    -2 & \text{if } D(x) = 1 + \sqrt{1 -4x},\\
    2 & \text{if } D(x) = 1 - \sqrt{1 -4x}.
  \end{cases}
  \]

  But $c_0 = 1$ since $\lambda$ is the only string in $\GC$ with $0$ left
  parentheses, and so $d_1 = 2$ by~\eqref{cndn+1}.  So the second case
  holds, namel $D(x)=1 - \sqrt{1 -4x}$.

\end{solution}

\ppart Prove that
\[
d_n = \frac{(2n-3) \cdot (2n-5) \cdots 5 \cdot 3 \cdot 1 \cdot 2^n}{n!}.
\]

\hint $d_n = D^{(n)}(0)/n!$

\begin{solution}
By Taylor's theorem, the coefficient $d_n$ equals $D^{(n)}(0)/n!$.
Now,
\begin{align*}
D'(x)      & = 2 \cdot (1-4x)^{-\frac{1}{2}} \\
D''(x)     & = 1 \cdot 2^2 \cdot (1-4x)^{-\frac{3}{2}} \\
D'''(x)    & = 3 \cdot 1 \cdot 2^3 \cdot (1-4x)^{-\frac{5}{2}} \\
D^{(4)}(x) & = 5 \cdot 3 \cdot 1 \cdot 2^4 \cdot (1-4x)^{-\frac{7}{2}} \\
D^{(5)}(x) & = 7 \cdot 5 \cdot 3 \cdot 1 \cdot 2^5 \cdot (1-4x)^{-\frac{9}{2}} \\
\vdots \quad    &\qquad \vdots\\
D^{(n)}(x) & = (2n-3) \cdot (2n-5) \cdots 5 \cdot 3 \cdot 1 \cdot 2^n \cdot (1-4x)^{\frac{1-2n}{2}}
\end{align*}
Here the dots reflect a routine induction argument that we're omitting.

A similar alternative argument uses Newton's formula
\[
(1+x)^\alpha = \binom{\alpha}{0} + \binom{\alpha}{1}x + \binom{\alpha}{2}
x^2 + \cdots \binom{\alpha}{k}x^k + \cdots
\]
where
\[
\binom{\alpha}{k}  \eqdef \frac{\alpha (\alpha -1) (\alpha -2) \cdots
  (\alpha -k +1)}{k!}.
\]
\end{solution}

\ppart Conclude that
\[
c_n = \frac{1}{n+1} \binom{2n}{n}.
\] 

\begin{solution}
Since $D(x)=2 x C(x)$, 
\begin{align*}
c_n   & = \frac{1}{2} d_{n+1}\\
      & = (2n-1) \cdot (2n-3) \cdots 5 \cdot 3 \cdot 1 \cdot \frac{2^n}{(n+1)!}\\
      & = \frac{(n) \cdot (2n-1) \cdot (n-1) \cdot (2n-3) \cdot (n-2) \cdots 3 \cdot 1 \cdot 1}
               {(n) \cdot (n-1) \cdot (n-2) \cdots 1} \cdot
          \frac{2^n}{(n+1)!}\\
      & = \frac{2(n) \cdot (2n-1) \cdot 2(n-1) \cdot (2n-3) \cdot 2(n-2) \cdots 3 \cdot 2(1) \cdot 1}{n!} \cdot
          \frac{1}{(n+1)!}\\
      & = \frac{2n!}{n!(n+1)!} = \frac{2n!}{n!\,n!}\frac{1}{n+1}.
\end{align*}
\end{solution}

\eparts

\iffalse
The method of partial fractions can only be applied to rational functions
(i.e. quotients of polynomials) and there are many interesting generating
functions that are not of this form.  Catalan numbers come up frequently
in counting the sizes of various recursively defined sets -- there are in
fact hundreds of interpretations of these
numbers!\footnote{\href{http://www-math.mit.edu/~rstan/ec/catadd.pdf}
  {http://www-math.mit.edu/~rstan/ec/catadd.pdf}}

The $n$th Catalan number, $c_n$, is equal to the number of strings in 
$\GC$ having exactly $n$ left parentheses.

\fi

\end{problem}

%%%%%%%%%%%%%%%%%%%%%%%%%%%%%%%%%%%%%%%%%%%%%%%%%%%%%%%%%%%%%%%%%%%%%
% Problem ends here
%%%%%%%%%%%%%%%%%%%%%%%%%%%%%%%%%%%%%%%%%%%%%%%%%%%%%%%%%%%%%%%%%%%%%

\endinput
