\documentclass[problem]{mcs}

\begin{pcomments}
  \pcomment{CP_magic_trick_27_cards_no_counting}
  \pcomment{Modified from CP_magic_trick_27_cards by zabel, 10/28/17}
  \pcomment{from: S09.cp10t, S06.cp9w with first prob shortened, 2nd problem
    omitted; F02.cp9W which was taken from F99 tut16 & S99 tut9}
  \pcomment{follow up to inclass Magic Trick demo}
\end{pcomments}

\pkeywords{
 magic_trick
 degree-constrained
 Halls_theorem
}

%%%%%%%%%%%%%%%%%%%%%%%%%%%%%%%%%%%%%%%%%%%%%%%%%%%%%%%%%%%%%%%%%%%%%
% Problem starts here
%%%%%%%%%%%%%%%%%%%%%%%%%%%%%%%%%%%%%%%%%%%%%%%%%%%%%%%%%%%%%%%%%%%%%

\begin{problem}

  A Magician wishes to perform a card trick with just the 26 red cards from a deck of cards. The magician's Assistant will have the audience choose four cards from the deck, and the Assistant will then choose three of these four cards and reveal them---in an order the Assistant chooses---to the Magician. The Magician then announces what the fourth card will be.

  For this trick to work, the Magician and Assistant must have decided on a \emph{strategy} in advance. Specifically, for every possible set of 4 cards, the Assistant must have decided which three of those four cards to reveal and in what order. Likewise, for each 3-card sequence the Magician might see, the Magician must have decided what fourth card to announce. The Magician and Assistant are hoping to find a strategy that guarantees the Magician will always announce the correct card.

  \bparts

  \ppart
  Model this as a bipartite matching problem, where the left vertices of $G$ are sets of 4 cards and the right vertices are ordered sequences of 3 cards. What are the edges of $G$? Explain how you could turn a matching in $G$ into a strategy (set of decisions) for the Magician and Assistant, and explain why this strategy guarantees the Magician will always be right.

  \begin{solution}
    Draw an edge in our graph $G$ between a set $f\in L(G)$ of 4 cards and a sequence $t\in R(G)$ of 3 cards if all of $t$'s cards belong to $f$. Suppose we have a matching $M$, which consists of a set of edges where each node in $L(G)$ belongs to exactly one edge of $M$ and each node in $R(G)$ belongs to at most one edge of $M$. When handed a 4-card set $f$, the Assistant will find the unique edge $\edge{f}{t}$ in $M$ and reveal $t$, which is possible since $t$'s cards belong to $f$ (this is how we chose $G$'s edges). When shown a 3-card sequence $t$, the Magician will find the unique edge $\edge{f}{t}$ in $M$ and announce the card of $f$ not in $t$. Because the right endpoints of $M$'s edges are distinct, the Assistant only reveals this sequence $t$ when presented with the set $f$, so the Magician will be right.
  \end{solution}

  \examspace[2in]
  
  \ppart
  Prove that your graph has a bipartite matching, and therefore a working strategy exists for executing this magic trick with a \textbf{26-card deck}. (You don't need need to find such a strategy explicitly---proving it exists is enough.)

  \hint For each set of 4 cards, the Assistant has $4\cdot 3\cdot 2 = 24$ choices about which 3-card sequence to reveal.

  \begin{solution}
    As in the hint, each $f\in L(G)$ has $\deg(f) = 24$. For each 3-card sequence $t\in R(G)$, the edges at $t$ correspond to the 4-card sets containing $t$'s cards; there are $26-3 = 23$ other cards that could be added to $t$ to make a 4-card set, so $\deg(t) = 23$. This means $G$ is degree constrained, so it must have a matching by the degree-constrained corollary of Hall's Theorem.
  \end{solution}
  
  \eparts


\end{problem}
 
\endinput
