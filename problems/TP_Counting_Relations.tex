\documentclass[problem]{mcs}

\begin{pcomments}
  \pcomment{TP_Counting_Relations}
  \pcomment{May be OK for tutor, but omit from text---ARM}
  \pcomment{corrected, edited ARM 3.18/13}
  \pcomment{Converted from ./00Convert/probs/practice3/prob7.scm
    by scmtotex and drewe
    on Thu 21 Jul 2011 12:05:47 PM EDT}
\end{pcomments}

\begin{problem}

How many relations are there on a set of size $n$ when:
\bparts

\ppart
$n=1$?

\begin{solution}
2.  Either the element is related to itself or not.
\end{solution}

\ppart
$n=2$?
\begin{solution}
16. 

Letting the two elements br $a,b$, there are four pairs---(a,a), (a,b)
(b,a) (b,b)---each of which can be present or absent in thegraph of
the relation, giving a total of $2^4=16$ possible binary relations.
\end{solution}

\ppart
$n=3$?

\begin{solution}
$2^9$. 

Same reasoning as above, but now there are 9 pairs.
\end{solution}

\eparts

\end{problem}

\endinput
