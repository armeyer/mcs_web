\documentclass[problem]{mcs}

\begin{pcomments}
  \pcomment{CP_2_layer_array_network}
  \pcomment{from: S09.final}
\end{pcomments}

\pkeywords{
  networks
  congestion
  routing
}

%%%%%%%%%%%%%%%%%%%%%%%%%%%%%%%%%%%%%%%%%%%%%%%%%%%%%%%%%%%%%%%%%%%%%
% Problem starts here
%%%%%%%%%%%%%%%%%%%%%%%%%%%%%%%%%%%%%%%%%%%%%%%%%%%%%%%%%%%%%%%%%%%%%

\begin{problem}


“Token replacing” is a single player game using a set of tokens, each colored black or white. Except for color, the tokens are indistinguishable. In each move, a player can replace a black token with three white tokens, or replace a white token with three black tokens. We can model this game like a state machine, where the state is described by a pair $(b,w)$ that counts the number of black and white tokens. 

The game has two possible start states: $(5,4)$ or $(4, 3)$.

We call a state $(b,w)$ ELIGIBLE if and only if rem$(b-w,4) = 1$ and $\min{(b,w)} \geq 3$.
This problem examines the connection between eligible states and
states that are REACHABLE from a start state.

\bparts


\ppart Give an example of a reachable state that is not eligible.

\begin{solution}
The state $(7,2)$ is directly reachable from the start state $(4,3)$, but it is not eligible.
\end{solution}

\ppart Show that $b+w$ is a strictly increasing derived variable.  Conclude
    that state $(3,2)$ is not reachable.
    
\begin{solution}
Each transition increases $b+w$ by $2$, so $b+w$ is strictly
   increasing.  Also, $b+w$ for the start states is $\geq 7$, so all
   reachable states have $b+w \geq 7$. $3+2 < 7$, so $(3,2)$ is not reachable.
   \end{solution}
   
\medskip
   
For the rest of the problem, you may find it useful to use the following fact: if $\max{(b,w)} \leq 5$ and $(b,w)$ is eligible, then $(b,w)$ is reachable.

  It's easy to verify this fact by checking the nine cases with $(5 \geq
  b,w \geq 3)$.  You may (and should) \textbf{assume} this fact
  \textbf{without proof}.

\ppart Define

  $P(n) ::=$ "for all $(b,w). [b+w = n$ and $(b,w)$ is eligible$]$ $\Rightarrow (b,w)$ is reachable."

Prove that $P(n-1) \Rightarrow P(n+1)$ for all $n \geq 1$.

\begin{solution}

Assuming that $b+w = n+1$ and that $(b,w)$ is eligible, we must show that $(b,w)$
         is reachable.  The fact given above means we may assume $\max{(b,w)} \geq 6$.
         We can assume without loss of generality that $b \geq 6$.

         Now, $(b-3, w+1)$ will be eligible since $\min{(b-3,w+1)} \geq 3$, and $(b-3)-(w+1) = (b-w) - 4$. Then, if $b-w$ has remainder $1$ on division by $4$, so does $(b-3)-(w+1)$,
         
         Also, $(b-3)+(w+1) = n-1$.  Now $P(n-1)$ implies that $(b-3,w+1)$ is
         reachable, and therefore so is $(b,w)$.  QED.

\ppart Conclude that all eligible states are reachable.

\begin{solution}
    This follows by Strong Induction, with the fact given above giving the
    base cases, and part (d) proving the induction step.
    \end{solution}

\ppart Prove that $(4^7 +3 , 4^5 +1)$ is NOT reachable.
\begin{solution}
$\rem{(b-w,4)}$ is a constant derived variable because,
          if there is a transition from $(b,w)$ to $(b',w')$, then $(b'-w') = (b-w) +- 4$. This implies that $b-w$ and $b'-w'$ have the same remainder on division by $4$.

          Since $(4^7 +3 - (4^5 +1)) = (4^7 - 4^5) + 2$ leaves a
          remainder of $2$ on division by $4$, and both start states have
          a remainder of $1$, the state $(4^7 +3 , 4^5 +1)$ cannot be reachable.

  \eparts
\end{problem}

\endinput
