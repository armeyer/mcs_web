 \documentclass[problem]{mcs}

\begin{pcomments}
\pcomment{PS_tree_degree_sequence}
\pcomment{by Diego Cifuentes 3/17/14; edits by ARM S14 and 12/6/15}
\end{pcomments}

\pkeywords{
 tree
 degree
 handshaking
 induction
 bipartite
}

%%%%%%%%%%%%%%%%%%%%%%%%%%%%%%%%%%%%%%%%%%%%%%%%%%%%%%%%%%%%%%%%%%%%%
% Problem starts here
%%%%%%%%%%%%%%%%%%%%%%%%%%%%%%%%%%%%%%%%%%%%%%%%%%%%%%%%%%%%%%%%%%%%%

\begin{problem}

%  The \emph{degree sequence} of a simple graph is the weakly
%  decreasing sequence of degrees of its vertices.  For example, the
%  degree sequence for the 5-vertex numbered tree pictured in the
%  Figure~\bref{codetrees} in Problem~\bref{CP_numbered_trees} is
%  $(2,2,2,1,1)$ and for the 7-vertex tree it is $(3,3,2,1,1,1,1)$.
%
%\inhandout{
%\begin{figure}[htb]
%\graphic{n-2}
%\caption{}
%\label{codetrees}
%\end{figure}
%}

%Let $D = (d_1,d_2,\dots,d_n)$ be a sequence of positive integers where $n\geq 2$.

Suppose $D = (d_1,d_2,\dots,d_n)$ is a list of the vertex degrees of
some $n$-vertex tree $T$ for $n \geq 2$.  That is, we assume the
vertices of $T$ are numbered, and $d_i>0$ is the degree of the $i$th
vertex of $T$.

\bparts
\ppart Explain why
\begin{equation}\label{eq:tree_degree_seq}
\sum_{i=1}^n d_i = 2(n-1).
\end{equation}

\begin{solution}
By the Handshaking Lemma\inbook{~\bref{sumedges}}, the sum of the
degrees of the vertices of $T$ is twice the number of edges, and since
$T$ is a tree with $n$ vertices, it has $n-1$ edges.
\end{solution}

\ppart Prove conversely that if $D$ is a sequence of positive integers
satisfying equation~\eqref{eq:tree_degree_seq}, then $D$ is a list of
the degrees of the vertices of some $n$-vertex tree.  \hint Induction.

\begin{solution}
The proof will be by induction on $n$ with induction hypothesis
\[
P(n) \eqdef \forall D.\, D\
\text{ satisfies }~\eqref{eq:tree_degree_seq}\ \QIMPLIES\ \exists\text{ tree }
 T.\, D \text{ is a list of the vertex degrees of } T.
 \]

\inductioncase{Base case} ($n = 2$): $D$ must be the length two
sequence $(1,1)$, which is the degree sequence of a 2-vertex tree.

\inductioncase{Induction step}: Let $D_{n+1} \eqdef
(d_1,d_2,\dots,d_n,d_{n+1})$ for $n \geq 2$,and suppose
\begin{equation}\label{sumdn+1}
\sum_1^{n+1} d_i = 2n.
\end{equation}
We want to prove that $D_{n+1}$ is a list of the degrees of the
vertices of some tree $T_{n+1}$ with $n+1$ vertices.

First notice that there must be an $i$ such that $d_i = 1$, otherwise
we would have that $d_{i}\geq 2$ for all $i$ and the
sum~\eqref{sumdn+1} would be too large.  Similarly, there must be a
$j$ such that $d_j = 1$, otherwise the sum would be $n+1$, which is
less than $2n$ since $n\geq 2$.  Now we can assume without loss of
generality that $d_{n+1}=1$ and $d_{n} \geq 2$.

Let $D \eqdef (d_1,d_2,\dots, d_{n-1},d_n - 1)$.  Compared to
$D_{n+1}$, the sequence $D$ omits $d_{n+1} = 1$ and replaces $d_{n}$
by $d_{n}-1$.  So the sum of $D$ is two less than the
sum~\eqref{sumdn+1}, namely $2n-2 = 2(n-1)$.  So $D$ satisfies
equation~\eqref{eq:tree_degree_seq}.  Now by induction hypothesis, $D$
is the sequence of degrees of the vertices of some $n$ vertex tree
$T$.

Let $T_{n+1}$ be the $(n+1)$-vertex tree obtained from $T$ by adding a
new leaf to the $n$th vertex.  Then $D_{n+1}$ is a list of the vertex
degrees of $T_{n+1}$, which completes the proof of the inductive step.
\end{solution}

\ppart\label{pat-decode-dlist} Assume that $D$ satisfies equation
\eqref{eq:tree_degree_seq}.  Show that it is possible to partition $D$
into two sets $S_1, S_2$ such that the sum of the elements in each set
is the same. \hint Trees are bipartite.

\begin{solution}
Using the previous part we know that there is a tree $T$ with degree
sequence $D$.  Since trees are bipartite graphs, there exists a partition
of the vertices into sets $V_1,V_2$ such that any edge connects a
vertex in $V_1$ with a vertex in $V_2$.  We argue that the sum of the
degrees of the vertices in $V_1$ is equal to the number of edges of
the graph.  The reason is that any edge in the graph contributes to
$1$ in the degree in exactly one of the vertices in $V_1$.  Similarly,
the sum of the degrees of the vertices in $V_2$ is also the number of
edges.  Thus the degrees corresponding to the partition $V_1,V_2$
determine the sets $S_1,S_2$ we were looking for.

A proof by induction similar to part~\eqref{pat-decode-dlist} is also
possible.
\end{solution}

\eparts

\end{problem}

%%%%%%%%%%%%%%%%%%%%%%%%%%%%%%%%%%%%%%%%%%%%%%%%%%%%%%%%%%%%%%%%%%%%%
% Problem ends here
%%%%%%%%%%%%%%%%%%%%%%%%%%%%%%%%%%%%%%%%%%%%%%%%%%%%%%%%%%%%%%%%%%%%%

\endinput
 
