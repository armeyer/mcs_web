\documentclass[problem]{mcs}

\begin{pcomments}
\pcomment{PS_more_numbered_trees}
\pcomment{from S08, ps9; f07.ps9;prob4 edited by ARM 11/14/09}
\pcomment{solution can and should be explained more simply---ARM 11/14/09}
\end{pcomments}

\pkeywords{
 multinomial_coefficient
 numbered_trees
 bijection
 degree_of_a_vertex
}

%%%%%%%%%%%%%%%%%%%%%%%%%%%%%%%%%%%%%%%%%%%%%%%%%%%%%%%%%%%%%%%%%%%%%
% Problem starts here
%%%%%%%%%%%%%%%%%%%%%%%%%%%%%%%%%%%%%%%%%%%%%%%%%%%%%%%%%%%%%%%%%%%%%

\begin{problem}

%  The \emph{degree sequence} of a simple graph is the weakly
%  decreasing sequence of degrees of its vertices.  For example, the
%  degree sequence for the 5-vertex numbered tree pictured in the
%  Figure~\bref{codetrees} in Problem~\bref{CP_numbered_trees} is
%  $(2,2,2,1,1)$ and for the 7-vertex tree it is $(3,3,2,1,1,1,1)$.
%
%\inhandout{
%\begin{figure}[htb]
%\graphic{n-2}
%\caption{}
%\label{codetrees}
%\end{figure}
%}

Let $D = (d_1,d_2,\ldots,d_n)$ be a  sequence of positive integers, for some $n\geq 2$.

\bparts
\ppart\label{occnum}

Assume that there is a tree $T$ with degree sequence $D$ (i.e. if $v_i$ denote the vertices of $T$, then the degree of $v_i$ is $d_i$, for $i=1,\ldots,n$).
Show that  
\begin{align}\label{eq:tree_degree_seq}
\sum_{i=1}^n d_i = 2(n-1)
\end{align}


\begin{solution}
   We know that the sum of the degrees of all vertices is twice the number of edges. As $T$ is a tree with $n$ vertices, it has $n-1$ edges. This shows equation \eqref{eq:tree_degree_seq}.
\end{solution}

\eparts


\bparts

\ppart\label{7pats} 
Assume that $D$ satisfies equation \eqref{eq:tree_degree_seq}. Show that there exists a tree $T$ with degree sequence $D$.
(\hint Use induction.)

\begin{solution}
    We show it by induciton on $T$.

    The base case is when $n=2$. In such case $D = (d_1,d_2) = (1,1)$ so that it satisfies equation \eqref{eq:tree_degree_seq}. The path $P_2$ consisting of two vertices connected by an edge satisfies the conditions.

    We now prove the induction step. We assume that the statement holds for $n=k-1$, and we consider now a sequence $D = (d_1,d_2,\ldots,d_{k})$ with sum $2(k-1)$. First notice that there must be an $i$ such that $d_i = 1$, otherwise we would have that $d_{i}\geq 2$ for all $i$ and the sum would be too large. Similarly, there must be a $j$ such that $d_j\geq 2$, otherwise the sum would be too small. We can assume then that $d_{k}=1$ and $d_{k-1}\geq 2$.

    Consider the sequence $D' = (d_1,d_2,\ldots,d_{k-2},d_{k-1}-1)$ consisting of $k-1$ positive integers. Observe that the sum of $D'$ is equal to $2(k-1)-2 = 2(k-2)$, so it satisfies equation \eqref{eq:tree_degree_seq}. By induction hypothesis we know that there is a tree $T'$ with degree sequence $D'$.  Let $T$ be the tree obtained by adding a new vertex to $T'$ which is adjacent to the vertex corresponding to $d_{k-1}-1$. Such $T$ has degree sequence $D$, concluding the proof.
\end{solution}
\eparts

\bparts
\ppart\label{pat-decode}
Assume that $D$ satisfies equation \eqref{eq:tree_degree_seq}. Show that it is possible to partition $D$ into two sets $S_1, S_2$ such that the sum of the elements in both parts is $n-1$. (\hint Trees are bipartite.)

\begin{solution}
    Using the previous part we know that there is a tree $T$ with degree sequence $D$. As trees are bipartite graphs, there exists a partition of the vertices into sets $V_1,V_2$ such that any edge connects a vertex in $V_1$ with a vertex in $V_2$. We argue that the sum of the degrees of the vertices in $V_1$ is equal to the number of edges of the graph. The reason is that any edge in the graph contributes to $1$ in the degree in exactly one of the vertices in $V_1$. Similarly, the sum of the degrees of the vertices in $V_2$ is also the number of edges. Thus the degrees corresponding to the partition $V_1,V_2$ determine the sets $S_1,S_2$ we were looking for.
\end{solution}

\eparts

\end{problem}

%%%%%%%%%%%%%%%%%%%%%%%%%%%%%%%%%%%%%%%%%%%%%%%%%%%%%%%%%%%%%%%%%%%%%
% Problem ends here
%%%%%%%%%%%%%%%%%%%%%%%%%%%%%%%%%%%%%%%%%%%%%%%%%%%%%%%%%%%%%%%%%%%%%

\endinput
