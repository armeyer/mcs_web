\documentclass[problem]{mcs}

\begin{pcomments}
\pcomment{PS_starfree_language}
\pcomment{ARM 2/10/15}
\end{pcomments}

\pkeywords{
  sequence
  word
  complement
  union
  concatenation
}

%%%%%%%%%%%%%%%%%%%%%%%%%%%%%%%%%%%%%%%%%%%%%%%%%%%%%%%%%%%%%%%%%%%%%
% Problem starts here
%%%%%%%%%%%%%%%%%%%%%%%%%%%%%%%%%%%%%%%%%%%%%%%%%%%%%%%%%%%%%%%%%%%%%

\begin{problem}
A \emph{binary word} is a finite sequence of \texttt{0}'s and
  \texttt{1}'s.  For example, $(\texttt{1},\texttt{1},\texttt{0})$ and
  $(\texttt{1})$ are words of length three and one, respectively, over
  this alphabet.  We usually omit the parentheses and commas in the
  descriptios of words, so the preceding binary words would just be
  written as $\texttt{110}$ and $\texttt{1}$.

The basic operation of making one word immediately follow another is
called \term{concatentation}.  For example, the concatentation of
$\texttt{110}$ and $\texttt{1}$ is $\texttt{1101}$, and the
concatentation of $\texttt{110}$ with itself is $\texttt{110110}$.  If
$R$ and $S$ are sets of words, then $R \cdot S$, is the set of all
words you can get by concatenating a word from $R$ with a word from
$S$.  For example $T \eqdef \set{\texttt{1}, \texttt{0}} cdot
\set{\texttt{1}, \texttt{0}}$ is the set
$\set{\texttt{00},\texttt{01},\texttt{10},\texttt{11}}$ of length two
binary words, and, $T \cdot T \cdot T$, which we abbreviate as
$T^3$, is the set of length six binary words.

If $S$ is a set of words, the set of all words you can get by
concatenating copies of any number of words in $S$ is called $S^*$.
By convention, the empty word $\emptystring$ always included in $S^*$.
For example, $T^*$ would be the set of all even length binary words.
So the set, $B$, of all binary words is
$\set{\texttt{0},\texttt{1}}^*$.

A set of binary words is called \emph{concatenation-definable}
(\emph{c-d}) if it can be constructed by starting from finite sets of
words and then combining them using concatenations, unions, and
complements (relative to $B$).  For example, the empty set of words is
c-d because it equals the intersection of any two disjoint sets, and
by DeMorgan's Law for sets, intersection can be defined in terms of
union and complement.  That is,
\[
\emptyset = \set{11} \intersect \set{00} =  \bar{\bar{\set{11}} \union \bar{\set{00}}}.
\]
Therefore the set $B$ of all binary words is c-d because
\[
B = \bar{\emptyset}.
\]
An interesting example of a c-d set is the set of all binary words
that include three consecutive \texttt{1}'s:
\[
B\cdot \texttt{111} \cdot B.
\]

\begin{problemparts}
\ppart Show that $\set{\texttt{0}}^*$ is c-d over the binary alphabet.
\begin{solution}
This is simply the set of binary words that do not contain a \texttt{1}
\[
\set{0}^* = \bar{B \cdot \texttt{1} \cdot B}.
\]
\end{solution}

\ppart Show that the set of binary words that start with \texttt{0}
and end with \texttt{1} is c-d.

\begin{solution}
$0B \intersect B1$
\end{solution}

\ppart Show that $\set{\texttt{01}}^*$ is c-d.

\begin{solution}
This is the set of words that do not include \texttt{00} or
\texttt{11} and start with \texttt{0} and end with \texttt{1}.
\[
\set{\texttt{01}}^* = \bar{B\set{\texttt{00},\texttt{11}}B} \intersect 0B \intersect B1.
\]
\end{solution}

\eparts

A set $S$ of binary words is called \emph{\texttt{0}-simple}
iff either $S \intersect \set{\texttt{0}}^*$ or $\bar{S} \intersect
\set{\texttt{0}}^*$ has only a finite number of words in it.

\bparts

\ppart\label{uc0simp} Verify that if $R$ and $S$ are
\texttt{0}-simple, then so are $R \union S$ and $R \cdot S$.
\begin{solution}
TBA
\end{solution}

\ppart\label{cdimp0s} Prove that all c-d sets are \texttt{0}-simple.
\begin{solution}
The starting c-d sets are of size one because all finite sets are
clearly \texttt{0}-simple.  Also, $S$ is \texttt{0}-simple iff
$\bar{S}$ is \texttt{0}-simple because $S = \bar{\bar{S}}$.  By
part~\eqref{uc0simp}, we conclude that all the operations for
constructing c-d sets preserve \texttt{0}-simplicity.  Hence all c-d
sets are \texttt{0}-simple.
\end{solution}

\ppart Conclude that $\set{\texttt{00}}^*$ is not a c-d set.

\begin{solution}
It's easy to see that $\set{\texttt{00}}^*$ is not \texttt{0}-simple:
it is the infinite set of even length strings of \texttt{0}'s and its
complement includes all the odd length strings of \texttt{0}'s, so its
intersection with $\set{\texttt{0}}^*$ is also infinite.  So
part~\eqref{cdimp0s} implies it is not c-d.

\end{solution}

\end{problemparts}


\end{problem}

%%%%%%%%%%%%%%%%%%%%%%%%%%%%%%%%%%%%%%%%%%%%%%%%%%%%%%%%%%%%%%%%%%%%%
% Problem ends here
%%%%%%%%%%%%%%%%%%%%%%%%%%%%%%%%%%%%%%%%%%%%%%%%%%%%%%%%%%%%%%%%%%%%%

\endinput


\iffalse

\eparts

Let $w$ be a binary word and $n$ a nonnegative integer.  A set $S$ of
words \emph{$n$-repeats $w$}, a word with $n$ consective occurrences
of $w$ is in $S$ iff that word with an extra occurrence of $w$ is also
in $S$.  That is,
\[
xw^ny \in S \QIFF xw^nwy \in S.
\]
We say a set of words is \emph{$n$-repeating} if it $n$-repeats every
word $w$.  For example, $\texttt{0}^*$ is 1-repeating and $B$ is actually 0-repeating.

\begin{problemparts}
\ppart Verify that $\set{\texttt{01}}^*$ is 2-repeating.

\begin{solution}
TBA
\end{solution}

\ppart Explain why $\set{\texttt{00}}^*$ is not $n$-repeating for any $n$.

\begin{solution}
$\set{\texttt{00}}^*$ does not $n$-repeat \texttt{0} for any $n$,
  because \texttt{0}^n is in $\set{\texttt{00}}^*$ but $\texttt{0}^n
  \texttt{0}$ is not.
\end{solution}
\fi

