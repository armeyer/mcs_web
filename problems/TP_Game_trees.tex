\documentclass[problem]{mcs}

\begin{pcomments}
  \pcomment{TP_game_trees}
  \pcomment{Converted from ./00Convert/probs/practice6/prob4.scm
    by scmtotex and drewe on Thu 28 Jul 2011 01:18:22 PM EDT}
  \pcomment{revised by ARM 10/16/17 to match section ''recursive_games''}
\end{pcomments}

\begin{problem}
In the game Tic-Tac-Toe, there are nine first-move games
corresponding to the nine boxes that player-1 could mark with
an ``X''.

Each of these nine games will themselves have eight first-move games
corresponding to where the second player can mark his ``O'', for a
total of 72 ``second-move'' games.

Answer the following questions about the subsequent Tic-Tac-Toe games.
\bparts

\ppart
How many third-move games are there where player-1 can mark his second ``X''?

\begin{solution}
\textbf{252}.

Player-1 can place an ``X'' in seven different available squares, so
each second-move game contributes seven of it own first-move games to
the set of third-move Tic-Tac-Toe games.  However, each third move
game is contributed by two different second move games since the
either of the two ``X'''s can be the one chosen on the first move.
So there are $(72 \cdot 7)/2 = 252$ distinct third move games.
\end{solution}

\ppart What is the first level where this simple pattern of
calculating how many first moves are possible in each subsequent game
stops working?

\begin{solution}
\textbf{Level 6.}  At the fifth level, it's possible that player-1 has
won, and so the number of first-move games of each fifth-level game
may be either zero or four.
\end{solution}

\eparts

\end{problem}

\endinput
