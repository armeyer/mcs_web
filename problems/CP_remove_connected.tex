\documentclass[problem]{mcs}

\begin{pcomments}
  \pcomment{CP_remove_connected}
  \pcomment{renamed from TP_remove_connected}
  \pcomment{ARM 4/9/14, revised 4/6/15}
  \pcomment{subsumes PS_connected_induction}
  \pcomment{F12.rec6 by clumsy induction; commented out of F12 final}
\end{pcomments}

\pkeywords{
  graph
  connected
  component
  induction
}

%%%%%%%%%%%%%%%%%%%%%%%%%%%%%%%%%%%%%%%%%%%%%%%%%%%%%%%%%%%%%%%%%%%%%
% Problem starts here
%%%%%%%%%%%%%%%%%%%%%%%%%%%%%%%%%%%%%%%%%%%%%%%%%%%%%%%%%%%%%%%%%%%%%

\begin{problem}
A simple graph $G$ is \emph{2-removable} iff it contains two vertices
$v \neq w$ such that $G-v$ is connected, and $G-w$ is also connected.
Prove that every connected graph with at least two vertices is
2-removable.

\hint Consider a maximum length path.

\begin{solution}
\begin{proof}
If $G$ is connected and has at least two vertices, the maximum length
of a path in $G$ will be at least two, and in particular, the
endpoints of a maximum length path must be different.

Let $v$ be one of the endpoints of a maximum length path, $\walkv{p}$.
We need only show that $G-v$ is connected.  We will prove this by
showing that any two vertices connected by a path going through $v$
are also connected by a walk that does not include $v$.

So suppose $\walkv{q}$ is a path that goes through $v$, that is,
$\walkv{q}$ has an edge $\edge{a}{v}$ followed by an edge
$\edge{v}{b}$ for some vertices $a \neq b$.  Now $b$ must occur on the
path $\walkv{p}$ or else $\catv{\walkv{p}}{v}{\edge{v}{b}}$ would be a
longer path, contradicting the fact the maximum length of $\walkv{p}$.
Likewise, $a$ must occur on $\walkv{p}$.  So we can replace the
consecutive edges $\edge{a}{v}$ and $\edge{v}{b}$ in $\walkv{q}$ by
the segment of $\walkv{p}$ between $a$ and $b$ to obtain a walk with
the same endpoints as $\walkv{q}$ that does not go through $v$.
\end{proof}

Note that this argument applies to the endpoints of any maxim\emph{al}
path---meaning a path that is not a subpath of any other path---not
just maximum length paths.  In fact, it's not hard to see that $G-v$
is connected iff $v$ is the endpoint of a maximal path.

Another immediate proof follows from the fact that every connected
graph has a spanning tree \inbook{(Theorem~\bref{th:spantree})} and
every tree with at least two vertices has at least two leaves\inbook{
  (Theorem~\bref{th:treeprops})}.
\end{solution}

\end{problem}


%%%%%%%%%%%%%%%%%%%%%%%%%%%%%%%%%%%%%%%%%%%%%%%%%%%%%%%%%%%%%%%%%%%%%
% Problem ends here
%%%%%%%%%%%%%%%%%%%%%%%%%%%%%%%%%%%%%%%%%%%%%%%%%%%%%%%%%%%%%%%%%%%%%

\endinput

