\documentclass[problem]{mcs}

\begin{pcomments}
  \pcomment{FP_numbers_short_answer2_F15}
  \pcomment{conflict variant of FP_numbers_short_answer_F15}
\end{pcomments}

\pkeywords{
  gcd
  modulo
  relatively_prime
  inverse
  mod_n
  Fermat
  prime
}

%%%%%%%%%%%%%%%%%%%%%%%%%%%%%%%%%%%%%%%%%%%%%%%%%%%%%%%%%%%%%%%%%%%%%
% Problem starts here
%%%%%%%%%%%%%%%%%%%%%%%%%%%%%%%%%%%%%%%%%%%%%%%%%%%%%%%%%%%%%%%%%%%%%

\begin{problem} \mbox{}

Parts~\eqref{cdabcda} to~\eqref{ainvmb} offer a collection of number
theoretic assertions.  Most of the assertions require some
side-conditions to be correct.  A selection of possible
side-conditions (\textbf{1})--(\textbf{8}) are listed below.

Next to each of assertions, write \True\ if the assertion is correct
as stated, write the number of the \emph{weakest} of the
side-conditions necessary to make the assertion true\footnote{For
  example, writing that a true assertion required a side-condition
  would be a mistake.}, or write \False\ if none of the
side-conditions imply the assertion is correct.

Assume all variables have appropriate integer values---for example, in
the context of $\Zmod{k}$, numbers are in $\Zintvco{0}{k}$ and $k>1$.

\[\begin{array}{llll}
(\mathbf{1}).\ \QNOT(a \text{ divides  } b)
 & (\mathbf{2}).\ \gcd(a, b) = 1
 & (\mathbf{3}).\ \gcd(a, b) > 1\\
   (\mathbf{4}).\ \gcd(a, b) > 2
 & (\mathbf{5}).\ \gcd(a,b) = 1 \text{ and} &\!\! \gcd(a,c) = 1\\
   (\mathbf{6}).\ a \text{ is prime}
 & (\mathbf{7}).\ b \text{ is prime}
 & (\mathbf{8}).\ a \text{ is prime} \text{ and } b \text{ is prime}
\end{array}\]

\bparts

\ppart\label{cdabcda} If $c \divides ab$, then $c \divides a$.  \hfill\examrule

\begin{solution}
\textbf{8}. $a$ is prime and $b$ is prime
\end{solution}

\iffalse
\ppart $\gcd(a^m,b^m) = (\gcd(a,b))^m$.\hfill\examrule

\begin{solution}
\True
\end{solution}
\fi

\iffalse %on main
\ppart $\gcd(1 + a, 1 + b) = 1 + \gcd(a, b)$.  \hfill\examrule

\begin{solution}
\False
\end{solution}
\fi

\iffalse
\ppart If $\gcd(a, b) \neq 1$ and $\gcd(b, c) \neq 1$, then $\gcd(a, c) \neq 1$. \hfill 

\begin{solution}
\False}  % $a=2\cdot 3, b=3\cdot 5, c=5\cdot 7$
\end{solution}

\ppart Some integer linear combination of $a^2$ and $b^2$ equals
one. \hfill \examrule

\begin{solution}
$\gcd(a, b) = 1$.
\end{solution}

\ppart No integer linear combination of $a^2$ and $b^2$ equals
one. \hfill \examrule

\begin{solution}
$\gcd(a, b) > 1$.
\end{solution}
\fi

\ppart No integer linear combination of $a^2$ and $b^2$ equals three. \hfill \examrule

\begin{solution}
$\gcd(a, b) > 1$.  If gcd is not one, it must be a square.
\end{solution}

\ppart No integer linear combination of $a^2$ and $b^2$ equals four. \hfill \examrule

\begin{solution}
  \False.  (Let $a=b=2$.)
\end{solution}

\iffalse
\ppart If $m a = n a \inzmod{b}$ then $m = n \inzmod{b}$.\hfill\examrule

\begin{solution}
$\gcd(a,b)=1$
\end{solution}
\fi

\ppart  If $b^{-1} = c^{-1} \inzmods{a}$, then $b = c \inzmods{a}$.\hfill\examrule

\begin{solution}
\True, where we take the use of inverse notation to imply the inverses
exist.  The condition that the inverses exist, namely, $\gcd(a,b)=1$
and $\gcd(a,c) = 1$, is also a good answer.
\end{solution}

\iffalse
\ppart If $m = n \inzmod{\phi(a)}$, then $m^b = n^b \inzmod{a}$.\hfill\examrule

\begin{solution}
\False
\end{solution}
\fi

\ppart If $m = n \inzmod{\phi(a)}$, then $b^m = b^n \inzmod{a}$.\hfill\examrule

\begin{solution}
$\gcd(a,b)=1$
\end{solution}

\iffalse
\ppart If $m = n \inzmod{ab}$, then $m = n \inzmod{b}$.\hfill\examrule

\begin{solution}
\True
\end{solution}
\fi

\ppart $[m = n \inzmod{a} \QAND\ m = n \inzmod{b}]$ iff $[m  = n \inzmod{ab}]$. \hfill\examrule

\begin{solution}
$\gcd(a, b) = 1$: the Chinese Remainder Theorem
  (Problem~\bref{CP_chinese_remainder}).
\end{solution}

\ppart\label{abazb} $a^{b} = a \inzmod{b}$. \hfill\examrule

\begin{solution}
$b$ is prime.
\end{solution}


\ppart\label{ainvmb}
 $a$ has in inverse in $\Zmod{b}$  iff $b$ has an inverse in $\Zmod{a}$. \hfill\examrule

\begin{solution}
\True:  $a$  has in inverse in $\Zmod{b}$  iff $\gcd(a,b)=1$.
\end{solution}

\eparts

\end{problem}

\endinput
