\documentclass[problem]{mcs}

\begin{pcomments}
\pcomment{PS_labeled_binary-trees}
\pcomment{revised from S07.ps3.prob2 by ARM 3/20/11}
\end{pcomments}

\pkeywords{
  structural_induction
  tree
  leaf
  binary_tree
}

\newcommand{\lbt}{\text{LBT}}
  

%%%%%%%%%%%%%%%%%%%%%%%%%%%%%%%%%%%%%%%%%%%%%%%%%%%%%%%%%%%%%%%%%%%%%
% Problem starts here
%%%%%%%%%%%%%%%%%%%%%%%%%%%%%%%%%%%%%%%%%%%%%%%%%%%%%%%%%%%%%%%%%%%%%


\begin{problem}
  Let $L$ be some convenient set whose elements will be called
  \emph{labels}.  The labeled binary trees, $\lbt$'s, are defined
  recursively as follows:
\begin{definition*}
If $l$ is a label, 

\inductioncase{Base case}:  $\ang{l, \texttt{leaf}}$ is an $\lbt$, and

\inductioncase{Constructor case}: if $B$ and $C$ are $\lbt$'s, then
$\ang{l,B,C}$ is an $\lbt$.

\end{definition*}
The \emph{leaf-labels} and \emph{internal-labels} of an $\lbt$, are defined
recursively in the obvious way:

\begin{definition*}

  \inductioncase{Base case}: The set of leaf-labels of the $\lbt$ $\ang{l,
    \texttt{leaf}}$ is $\set{l}$ and its set of internal-labels is the
  empty set.

  \inductioncase{Constructor case}: The set of leaf labels of the $\lbt$
  $\ang{l, B, C}$ is the union of the leaf-labels of $B$ and of $C$; the
  set of internal-labels is the union of $\set{l}$ and the sets of
  internal-labels of $B$ and of $C$.

\end{definition*}

The set of \emph{labels} of an $\lbt$ is the union of its leaf- and
internal-labels.

The $\lbt$'s with \emph{unique} labels are also defined recursively:
\begin{definition*}

  \inductioncase{Base case}: The $\lbt$ $\ang{l,\texttt{leaf}}$ has \emph{unique
    labels}.

  \inductioncase{Constructor case}: The $\lbt$ $\ang{l, B, C}$ has \emph{unique
    labels} iff $l$ is not a label of $B$ or $C$, and no label is a label
  of both $B$ and $C$.

\end{definition*}

If $B$ is an $\lbt$, let $n_B$ be the number of i\textbf{n}ternal-labels
appearing in $B$ and $f_B$ be the number of lea\textbf{f} labels of $B$.

Prove by \idx{structural induction} that
\begin{equation}\label{fbnb1}
f_B = n_B+1
\end{equation}
for all $\lbt$'s \emph{with unique labels}.  This equation can obviously
fail if labels are not unique, so your proof had better use uniqueness of
labels at some point; be sure to indicate where.

\begin{solution}
\inductioncase{Base case}: If $B = \ang{l,\texttt{leaf}}$, then $f_B=1$
and $n_B = 0$ so~\eqref{fbnb1} holds.

\inductioncase{Constructor case}: If $A \eqdef \ang{l, B,C}$ has unique labels then

\begin{align*}
f_A & =  \card{\text{leaf-labels}(B) \union \text{leaf-labels}(C)}
         & \text{(by def. of leaf labels)}\\
 & = f_{B} + f_{C} & \text{(no label appears in both $B$ and $C$)}\\
 & = (n_{B}+1) + (n_{C}+1)
          & \text{(by structural induction hypothesis)}\\
 &= (n_{B} + n_{C}) +1) +1\\
 & = \card{\set{l} \union \text{internal-labels}(B) \union \text{internal
     labels}(C)} + 1
         &  \text{(uniqueness of labels)}\\
 & = n_A +1 & \text{(by def. of $n_A$)}.
\end{align*}

This proves~\eqref{fbnb1} holds for $A$, completing the proof of the
Constructor case.  It follows by structural induction that ~\eqref{fbnb1}
holds for all $\lbt$'s with unique labels.
\end{solution}

\end{problem}


%%%%%%%%%%%%%%%%%%%%%%%%%%%%%%%%%%%%%%%%%%%%%%%%%%%%%%%%%%%%%%%%%%%%%
% Problem ends here
%%%%%%%%%%%%%%%%%%%%%%%%%%%%%%%%%%%%%%%%%%%%%%%%%%%%%%%%%%%%%%%%%%%%%

\endinput
