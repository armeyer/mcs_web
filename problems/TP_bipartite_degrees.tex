 \documentclass[problem]{mcs}

\begin{pcomments}
\pcomment{TP_bipartite_degrees}
\pcomment{excerpted PS_tree_degree_sequence_S16}
\end{pcomments}

\pkeywords{
 tree
 degree
 handshaking
 induction
}

%%%%%%%%%%%%%%%%%%%%%%%%%%%%%%%%%%%%%%%%%%%%%%%%%%%%%%%%%%%%%%%%%%%%%
% Problem starts here
%%%%%%%%%%%%%%%%%%%%%%%%%%%%%%%%%%%%%%%%%%%%%%%%%%%%%%%%%%%%%%%%%%%%%

\begin{problem}
Let $B$ be a bipartite graph with vertex sets $\leftbi{B},
\rigthbi{B}$.  Explain why the sum of the degrees of the vertices in
$\leftbi{B}$ equals the sum of the degrees of the vertices in
$\rightbi{B}$.

\begin{solution}
Both sums equal the number of edges in $B$.

The reason is that any edge in the graph contributes to $1$ to the
degree of its endpoint in $\leftbi{B}$ and likewise $1$ to the degree
of its endpoint in $\rightbi{B}$.
\end{solution}

\end{problem}

%%%%%%%%%%%%%%%%%%%%%%%%%%%%%%%%%%%%%%%%%%%%%%%%%%%%%%%%%%%%%%%%%%%%%
% Problem ends here
%%%%%%%%%%%%%%%%%%%%%%%%%%%%%%%%%%%%%%%%%%%%%%%%%%%%%%%%%%%%%%%%%%%%%

\endinput
 
