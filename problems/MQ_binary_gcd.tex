\documentclass[problem]{mcs}

\begin{pcomments}
  \pcomment{MQ_binary_gcd}
  \pcomment{lighter version of PS_binary_gcd}
  \pcomment{ARM 3/17/13}
\end{pcomments}

\pkeywords{
  preserved_invariant
  GCD
  binary_GCD
  state_machine
}

%%%%%%%%%%%%%%%%%%%%%%%%%%%%%%%%%%%%%%%%%%%%%%%%%%%%%%%%%%%%%%%%%%%%%
% Problem starts here
%%%%%%%%%%%%%%%%%%%%%%%%%%%%%%%%%%%%%%%%%%%%%%%%%%%%%%%%%%%%%%%%%%%%%

\begin{problem}
  The following \idx{Binary GCD} state machine computes the GCD of
  integers $a>b>0$:

\begin{align*}
\text{states} & \eqdef \naturals^3\\
\text{start state} & \eqdef (a, b, 1)\\
\text{transitions} & \eqdef \text{ if } \min(x,y) > 0, \text{ then } (x,  y, e) \movesto\\
     &\qquad \text{the first possible state according to the rules:}\\
     &\qquad
       \begin{cases}
       (1, 0, ex)     & \text{(if $x = y$)}\\
       (1, 0, e)      & \text{(if $y = 1$)},\\
       (x/2, y/2, 2e) & \text{(if $2 \divides x$ and $2 \divides y$)},\\
       (x/2, y, e)    & \text{(if $2 \divides x$)}\\
       (x, y/2, e)    & \text{(if $2 \divides y$)}\\
       (y, x, e)      & \text{(if $y>x$)}\\
       (x-y,y,e)      & \text{(otherwise)}.
       \end{cases}
\end{align*}

The predicate
\[
\gcd(a,b) = e\gcd(x,y)
\]
is claimed to be a preserved invariant of this state machine.

\bparts 

\ppart Verify that this predicate is a preserved invariant for the
3rd: $(x/2,y/2,e)$, 4th: $(x/2,y,e)$ and last: $(x-y,y,e)$ of the
above transition rules.  You may assume without proof elementary
properties of GCD, but be sure to state your assumptions explicitly.

\examspace[3.5in]

\begin{solution}
To verify preserved invariance, we assume the invariant holds for
state $(x,y,e)$ and show that if $(x,y,e) \movesto (x',y',e')$, then
$\gcd(a,b) = e'\gcd(x',y')$.

The proof is by cases according to which kind of transition occurs.

\inductioncase{Case}: ($2 \divides x$ and $2 \divides y$).  
In this case, $(x',y',e') = (x/2, y/2, 2e)$.

We use the easily verified fact
\begin{equation}\label{auavauv}
\gcd(au,av) = a\gcd(u,v).
\end{equation}

Now
\begin{align*}
\gcd(a,b)
  & =  e\gcd(x,y)
        & \text{(by the invariant for $(x,y,e)$)}\\
  & = e 2\gcd(x/2,y/2)
        & \text{(by~\eqref{auavauv})}\\
  & = e'\gcd(x',y'),
\end{align*}
which shows that the invariant holds for $(x',y',e')$.

\inductioncase{Case}: ($2 \divides x$ and 2 does not divide $y$).
In this case, $(x',y',e') = (x/2,y, e)$.

We use the easily verified fact
\begin{equation}\label{auvvuv}
\gcd(au,v) = \gcd(u,v)
\end{equation}
for $a$ relatively prime to $v$.

Now
\begin{align*}
\gcd(a,b)
  & = e\gcd(x,y)
      & \text{(invariant for $(x,y,e)$)}\\
  & = e\gcd(x/2,y)
      & \text{(by~\eqref{auvvuv})}\\
  & = e'\gcd(x',y'),
\end{align*}
which shows that the invariant holds for $(x',y',e')$.

\inductioncase{Case}: (otherwise clause).
In this case $(x',y',e') = (x-y,y,e)$.

We use the easily verified fact that
\begin{equation}\label{gcdu-v}
\gcd(u-v,v) = \gcd(u,v).
\end{equation}

Now,
\begin{align*}
\gcd(a,b)
  & = e\gcd(x,y) & \text{(invariant for $(x,y,e)$)}\\
  & = e\gcd(x-y,y) & \text{(by~\eqref{gcdu-v})}\\
  & = e'\gcd(x',y'),
\end{align*}
proving that the invariant holds for $(x',y',e')$.

Verification of the remaining cases follows similarly.
\end{solution}

\iffalse
\ppart Explain why the only possible stopped states (states where no
transition is possible) are of the form $(1,0,e')$.

\begin{solution}
The start state is not stopped since $\min(a,b)= b > 0$.  Likewise,
the final transition rule to state $(x-y,y,e)$ is also only possible
if $\min(x,y) = y > 0$, so another transition will be possible after
applying this rule.  Moves by the third through sixth rules do not
lead to stopped states, since the minimum of the new $x$ and $y$
values remains positive.

So only the first two rules can lead to stopped states, which in both
cases are of the form $(1,0,e')$.
\end{solution}
\fi

\ppart Use the Invariant Principle to conclude that if this machine
reaches a stopped state $(1,0,e')$, then $e' = \gcd(a,b)$.

%\examspace[1in]
\begin{solution}
We first observe that the preserved invariant holds trivially in the
start state $(a,b,1)$ because $\gcd(a,b) = 1\cdot\gcd(a,b)$.  The
Invariant Principle allows us to conclude that the preserved invariant
holds in every reachable state.

If a stopped state $(1,0,e')$ is reachable, then the invariant implies
\[
\gcd(a,b) = e'\gcd(1,0) = e',
\]
as required.
\end{solution}

\eparts

\end{problem}

%%%%%%%%%%%%%%%%%%%%%%%%%%%%%%%%%%%%%%%%%%%%%%%%%%%%%%%%%%%%%%%%%%%%%
% Problem ends here
%%%%%%%%%%%%%%%%%%%%%%%%%%%%%%%%%%%%%%%%%%%%%%%%%%%%%%%%%%%%%%%%%%%%%

\endinput
