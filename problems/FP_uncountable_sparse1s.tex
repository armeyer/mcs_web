\documentclass[problem]{mcs}

\begin{pcomments}
  \pcomment{FP_uncountable_sparse1s}
  \pcomment{overlaps PS_off_diagonal_arguments}
  \pcomment{ARM 4/2/17}
\end{pcomments}

\pkeywords{
  uncountable
  infinite
  string
}

%%%%%%%%%%%%%%%%%%%%%%%%%%%%%%%%%%%%%%%%%%%%%%%%%%%%%%%%%%%%%%%%%%%%%
% Problem starts here
%%%%%%%%%%%%%%%%%%%%%%%%%%%%%%%%%%%%%%%%%%%%%%%%%%%%%%%%%%%%%%%%%%%%%
\begin{problem}
An infinite binary string $b \eqdef b_0b_1b_2\dots b_n\dots$ is called
\emph{OK} when the \mtt{1}'s in $b$ occur only at perfect square
positions.  That is, $b$ is OK when
\[
b_i = \begin{cases}
  \quad \mtt{0} & \text{ if } i \notin \set{0,1,4,9,\dots,n^2,(n+1)^2\dots},\\
  \mtt{0} \text{ or } \mtt{1}  & \text{otherwise}.
\end{cases}
\]

\bparts

\ppart\label{uncountOK} Prove that the set of OK strings is uncountable.

\examspace[2in]

\begin{solution}
An OK string is determined by the perfect square positions where it
has \mtt{1}'s.  So the function
\[
f(b) \eqdef \set{n \in \nngint \suchthat b_{n^2} = \mtt{1}}
\]
is a bijection from OK strings to $\power(\nngint)$.  Since
$\power(\nngint)$ is uncountable by Cantor's Theorem~\bref{powbig},
and uncountability is preserved by bijections (Corollary~\bref{UinjAu}), 
it follows that the OK strings are uncountable.

Alternatively, the function
\[
g(b) \eqdef b_0b_1b_4\dots b_{n^2}b_{(n+1)^2}\dots
\]
is a bijection from OK strings to the set $\binw$ of all infinite
binary sequences, which was shown to be uncountable as a Corollary to
Cantor's Theorem.
\end{solution}

\ppart\label{uncountsubset} Prove that a set with an uncountable
subset must itself be uncountable.

\examspace[2in]

\begin{solution}
If $A \subseteq B$ and there is an $a_0 \in A$, then $B \surj A$.
This follows because there is a trivial surjective function $f:B \to
A$:
\[
f(b) =  \begin{cases}
      b & \text{ if } b \in A,\\
      a_0 & \text{otherwise}.
\end{cases}
\]
But we know
\begin{fact*}
If $A$ is uncountable and $B \surj A$, then $B$ must uncountable.
\end{fact*}
So a set $B$ with an uncountable subset $A$ must be uncountable.

The above fact was given in Corollary~\bref{UinjAu}, but it also
follows immediately by contradiction.  Namely, if $B$ was countable,
then by definition $\nngint \surj B$.  Together with $B \surj A$, this
implies $\nngint \surj A$, contradicting the uncountability of $A$.
\end{solution}

\ppart Let \emph{Sparse} be the set of infinite binary strings whose
fraction of \mtt{1}'s approaches zero.  Conclude that Sparse is
uncountable.

\begin{solution}
Every OK string $b$ is sparse since
\[
\frac{\text{\#\mtt{1}'s in } b_0b_1\dots b_n}{n} \leq  \frac{\sqrt{n}}{n},
\]
and
\[
\lim_{n\to \infty} \frac{\sqrt{n}}{n} = 0.
\]
This implies that the OK strings are a subset of Sparse, so Sparse is
uncountable by parts~\eqref{uncountOK} and~\eqref{uncountsubset}.
\end{solution}

\eparts

\end{problem}

\endinput
