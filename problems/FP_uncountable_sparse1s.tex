\documentclass[problem]{mcs}

\begin{pcomments}
  \pcomment{FP_uncountable_sparse1s}
  \pcomment{easy special case of PS_off_diagonal_arguments}
  \pcomment{ARM 3/29/17}
\end{pcomments}

\pkeywords{
  uncountable
  infinite
  string
}

%%%%%%%%%%%%%%%%%%%%%%%%%%%%%%%%%%%%%%%%%%%%%%%%%%%%%%%%%%%%%%%%%%%%%
% Problem starts here
%%%%%%%%%%%%%%%%%%%%%%%%%%%%%%%%%%%%%%%%%%%%%%%%%%%%%%%%%%%%%%%%%%%%%
\begin{problem}
Call an infinite binary string $b \eqdef b_0b_1b_2\dots b_n\dots \in
\binw$ ``OK,'' if the \mtt{1}'s in $b$ occur only at perfect square
positions.  That is, $b$ is OK when
\[
b_i = 1 \QIMPLIES i \in \set{0,1,4,9,\dots,n^2,\dots}.
\]
Prove that the set of OK strings is uncountable.

\begin{solution}
The function
\[
f(b) \eqdef \set{n \suchthat b_{n^2} = 1}
\]
is a bijection from OK strings to $\power(\nngint)$.  Since
$\power(\nngint)$ is uncountable by Cantor's Theorem~\bref{powbig},
and uncountability is preserved by bijections (Corollary~\bref{UinjAu}), 
it follows that the OK strings are uncountable.

Alternatively, the function
\[
g(b) \eqdef b_0b_1b_4\dots b_{n^2}\dots
\]
is a bijection from OK strings to the set $\binw$ of all infinite
binary sequences, which was shown to be uncountable as a Corollary to
Cantor's Theorem.
\end{solution}

\end{problem}

\endinput

\iffalse
A \emph{prefix} of an infinite string $b$ is a finite string $w$ such
that $b=wc$ for some infinite string $c$.  An infinite sequence $b$,
is called \mtt{0}-\emph{sparse} iff
\[
\lim_{p \in \text{prefixes}(b)}  \frac{\#\mtt{0}'s\text{ in }p}{\length{p}} = 0.
\]
Show that the set of \mtt{0}-\emph{sparse} infinite binary strings $b
\in \binw$ is uncountable.
\fi
