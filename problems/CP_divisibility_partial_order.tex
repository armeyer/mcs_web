\documentclass[problem]{mcs}

\begin{pcomments}
  \pcomment{CP_divisibility_partial_order}
  \pcomment{9/25/09 from partial order notes problem, edited by ARM}
\end{pcomments}

\pkeywords{
 partial_order
 divisibility
 antisymmetry
 antisymmetric
 reflexive
 transitive
}

%%%%%%%%%%%%%%%%%%%%%%%%%%%%%%%%%%%%%%%%%%%%%%%%%%%%%%%%%%%%%%%%%%%%%
% Problems start here
%%%%%%%%%%%%%%%%%%%%%%%%%%%%%%%%%%%%%%%%%%%%%%%%%%%%%%%%%%%%%%%%%%%%%

\begin{problem}
\bparts

\ppart
Verify that the divisibility relation on the set of nonnegative integers is
a weak partial order.
\begin{solution}
Divisibility is reflexive since $n \divides n$.

It is transitive by Lemma~\bref{lem:divtrans}.

It is anti-symmetric since if $n \divides m$ implies $n \leq m$ for nonnegative integers $n,m$.

\end{solution}

\ppart What about the divisibility relation on the set of integers?

\begin{solution}
Divisibility is not antisymmetric on the integers, since $n \divides -n$.
\end{solution}

\eparts

\end{problem}


%%%%%%%%%%%%%%%%%%%%%%%%%%%%%%%%%%%%%%%%%%%%%%%%%%%%%%%%%%%%%%%%%%%%%
% Problems end here
%%%%%%%%%%%%%%%%%%%%%%%%%%%%%%%%%%%%%%%%%%%%%%%%%%%%%%%%%%%%%%%%%%%%%

\endinput
