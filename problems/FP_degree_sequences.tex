\documentclass[problem]{mcs}

\begin{pcomments}
  \pcomment{FP_degree_sequences}
  \pcomment{CH Spring '14}
\end{pcomments}

\pkeywords{
   degree
   undirected graphs
   handshaking lemma
}


%%%%%%%%%%%%%%%%%%%%%%%%%%%%%%%%%%%%%%%%%%%%%%%%%%%%%%%%%%%%%%%%%%%%%
% Problem starts here
%%%%%%%%%%%%%%%%%%%%%%%%%%%%%%%%%%%%%%%%%%%%%%%%%%%%%%%%%%%%%%%%%%%%%

\begin{problem}

A \emph{degree sequence} is a set of $n$ non-negative integers
$\{ d_1, d_2, \ldots, d_n \}$ that correspond to degrees of vertices
in some simple graph $G$
with $n$ nodes. Determine whether the following are valid degree
sequences, and provide your reasoning.

\bparts

\ppart $\{ 4, 4, 4, 4, 5, 1 \}$
\begin{solution}
This is a \textbf{valid} degree sequence. This can be realized by augmenting
the complete graph with 5 vertices, $K_5$, with an ``extra'' vertex
connected to any one of the 5 vertices.
\end{solution}

\examspace[0.8in]

\ppart $\{1, 2, 3, 4, 5, 6, 7\}$
\begin{solution}
This is \textbf{not a valid} degree sequence. There are only 7
vertices, so the maximum degree of any vertex is 6.

\end{solution}

\examspace[0.8in]

\ppart $\{ 4, 3, 4, 4, 3, 1 \}$
\begin{solution}
This is \textbf{not a valid} degree sequence. By the Handshaking
Lemma, the sum of degrees in any simple graph must be even, which is
not true in this case (since $4 + 3 + 4 + 4+ 3 + 1 = 19$.)
\end{solution}

\examspace[0.8in]

\ppart $\{1, 2, 3, 4, 4\}$
\begin{solution}
This is \textbf{not a valid} degree sequence. There are two vertices
with degree 4, which means that both of them have to be connected to
each of the \emph{other} 4 vertices. Therefore, the degree of every
vertex has to be at least 2.
\end{solution}

\ppart $\{1, 2, 3, 4, 5, 6, 7\}$
\begin{solution}
This is \textbf{not a valid} degree sequence. There are only 7
vertices, so the maximum degree of any vertex is 6.

\end{solution}

\examspace[0.8in]

\eparts

\end{problem}

%%%%%%%%%%%%%%%%%%%%%%%%%%%%%%%%%%%%%%%%%%%%%%%%%%%%%%%%%%%%%%%%%%%%%
% Problem ends here
%%%%%%%%%%%%%%%%%%%%%%%%%%%%%%%%%%%%%%%%%%%%%%%%%%%%%%%%%%%%%%%%%%%%%

\endinput
