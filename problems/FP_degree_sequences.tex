\documentclass[problem]{mcs}

\begin{pcomments}
  \pcomment{FP_degree_sequences}
  \pcomment{CH Spring '14}
\end{pcomments}

\pkeywords{
   degree
   undirected graphs
   handshaking lemma
}


%%%%%%%%%%%%%%%%%%%%%%%%%%%%%%%%%%%%%%%%%%%%%%%%%%%%%%%%%%%%%%%%%%%%%
% Problem starts here
%%%%%%%%%%%%%%%%%%%%%%%%%%%%%%%%%%%%%%%%%%%%%%%%%%%%%%%%%%%%%%%%%%%%%

\begin{problem}

A \emph{degree sequence} is a set of $n$ non-negative integers
$\{ d_1, d_2, \ldots, d_n \}$ that correspond to degrees of vertices
in some simple connected graph $G$ with $n$ nodes. 

Argue why the following are \textbf{not} valid degree
sequences. 

\bparts


\ppart $\{1, 2, 3, 4, 5, 6, 7\}$
\begin{solution}
There are only 7 vertices, so the degree of any vertex can be at most 6.
\end{solution}

\examspace[0.8in]

\ppart $\{0, 2, 2, 2, 2\}$
\begin{solution}
There are a vertex with degree 0. This means that the graph is not connected.
\end{solution}

\examspace[0.8in]

\ppart $\{1, 3, 3, 4, 4, 4\}$
\begin{solution}
By the Handshaking Lemma, the sum of degrees in any simple graph must be even, which is
not true in this case (since $1 + 3 + 3+ 4 +  4 + 4 = 19$.)
\end{solution}

\examspace[0.8in]

\ppart $\{1, 2, 3, 4, 4\}$
\begin{solution}
There are two vertices with degree 4, which means that both of them have to be connected to
each of the \emph{other} 4 vertices. Therefore, the degree of every
vertex has to be at least 2, which is violated in this case.

\end{solution}

\examspace[0.8in]

\ppart Any sequence $s = \{d_1, d_2, \ldots, d_n \}$ that satisfies:
\[
d_1 + d_2 + \ldots + d_n = 2n - 3.
\] 
\begin{solution}
There are too few edges. The sum of the degrees is twice the number of edges (by the
Handshaking Lemma). However, the number of edges in any $n$-vertex connected
graph is at least $n-1$. Therefore, the sum of the degrees must be at
least $2n-2$. 
\end{solution}

\examspace[0.8in]


\eparts

\end{problem}

%%%%%%%%%%%%%%%%%%%%%%%%%%%%%%%%%%%%%%%%%%%%%%%%%%%%%%%%%%%%%%%%%%%%%
% Problem ends here
%%%%%%%%%%%%%%%%%%%%%%%%%%%%%%%%%%%%%%%%%%%%%%%%%%%%%%%%%%%%%%%%%%%%%

\endinput
