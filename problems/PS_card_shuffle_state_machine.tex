\documentclass[problem]{mcs}

\begin{pcomments}
  \pcomment{PS_card_shuffle_state_machine}
  \pcomment{S03.ps7}
\end{pcomments}

\pkeywords{
  state_machines
  termination
  partial_correctness
  invariant
  shuffle
  cards
}

%%%%%%%%%%%%%%%%%%%%%%%%%%%%%%%%%%%%%%%%%%%%%%%%%%%%%%%%%%%%%%%%%%%%%
% Problem starts here
%%%%%%%%%%%%%%%%%%%%%%%%%%%%%%%%%%%%%%%%%%%%%%%%%%%%%%%%%%%%%%%%%%%%%

\begin{problem} Suppose that you have a regular deck of cards arranged as
follows, from top to bottom:
\[
A \heartsuit\ 2 \heartsuit \dots K \heartsuit\
A \spadesuit\ 2 \spadesuit \dots K \spadesuit\
A \clubsuit\ 2 \clubsuit \dots K \clubsuit\
A \diamondsuit\ 2 \diamondsuit \dots K \diamondsuit
\]

Only two operations on the deck are allowed: \emph{inshuffling} and
\emph{outshuffling}.  In both, you begin by cutting the deck exactly in
half, taking the top half into your right hand and the bottom into
your left.  Then you shuffle the two halves together so that the cards
are perfectly interlaced; that is, the shuffled deck consists of one
card from the left, one from the right, one from the left, one from
the right, etc.  The top card in the shuffled deck comes from the
right hand in an outshuffle and from the left hand in an inshuffle.

\begin{problemparts}

\ppart Model this problem as a state machine.

\begin{solution}
Let the set of states $Q$ be all the possible orderings of
the deck of cards.  Let the set of start states $Q_0$ consist of the
single ordering listed above.  For each state $(c_1, \ldots c_{52}) \in
Q$, there are two transitions in $\delta$:

\begin{align*}
(c_1, \dots c_{52})
	& \rightarrow
		(c_1, c_{27}, c_2, c_{28}, \dots, c_{26}, c_{52}) \\
(c_1, \dots c_{52})
	& \rightarrow
		(c_{27}, c_1, c_{28}, c_2, \dots, c_{52}, c_{26})
\end{align*}
\end{solution}

\ppart Use the Invariant Principle to prove that you cannot make
the entire first half of the deck black through a sequence of
inshuffles and outshuffles.

\begin{solution}
Define two cards to be \emph{opposites} if the sum of their positions
is 53.  For example, the top card is opposite the bottom card, the
second card from the top is opposite the second from the bottom, etc.

Let $P$ be the property that $A \heartsuit$ is opposite $K
\diamondsuit$, $2 \heartsuit$ is opposite $Q \diamondsuit$, etc., as
in the initial configuration.  We claim that $P$ is a preserved
invariant.  Suppose that $P$ holds for a given state and consider the
two types of transition out of that state.  Note that, for both types
of transition, cards in opposite positions are mapped to opposite
positions.  Therefore, the property $P$ still holds.  Thus, $P$ is
preserved.

If the above explanation doesn't convince you, consider this
arithmetical justification: Consider two cards at positions $i$ and
$j$ that are correctly paired in some configuration.  Without loss of
generality, let $i$ be the earlier of the two positions.  We know
$i + j = 53$, so we can simply talk about positions $i$ and $53 - i$.
Considering the first of the transitions shown  in part (a)'s solution,
we can see that:
\begin{itemize}
\item A card at position $k$ in the first half of the deck is moved
  to position $2k - 1$
\item A card at position $k$ in the second half of the deck is moved
  to position $2(k - 26)$
\end{itemize}

Thus, the new position of the card from position $i$ is $2i - 1$, and
the position of the other card is $2(j - 26) = 2(53 - i - 26) = 54 - 2i$.
It is easy to see that the two new positions sum to 53, so the invariant
is maintained for this pair of matched cards.  Since we imposed no
further conditions on the pair we chose, we see that any shuffle of
this variety preserves the invariant; and the reasoning for the other
sort of shuffle is analogous.

We finish by noting that $P$ holds for the start state by definition.  However,
$P$ cannot hold when all black cards are in the top half of the deck;
in that case, every black card is opposite a red card, which was \emph{not}
true for any cards in the original configuration.  Therefore, by
the Invariant Principle, that state is not reachable.
\end{solution}

\end{problemparts}

Note: Discovering a suitable invariant can be difficult!  The standard
approach is to identify a bunch of reachable states and then look for
a pattern, some feature that they all share.
\iffalse
\footnote{If this does not work, consider twitching and drooling until
  someone takes the problem away.}
\fi

\end{problem}

\endinput
