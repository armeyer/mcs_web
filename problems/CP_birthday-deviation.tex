\documentclass[problem]{mcs}

\begin{pcomments}
  \pcomment{CP_birthday-deviation}
  \pcomment{new S10 revised from book}
\end{pcomments}

\pkeywords{
  deviation
  pairwise_independent
  birthday
}

%%%%%%%%%%%%%%%%%%%%%%%%%%%%%%%%%%%%%%%%%%%%%%%%%%%%%%%%%%%%%%%%%%%%%
% Problem starts here
%%%%%%%%%%%%%%%%%%%%%%%%%%%%%%%%%%%%%%%%%%%%%%%%%%%%%%%%%%%%%%%%%%%%%
\begin{staffnotes}
This problem comes pretty directly from
Section~\bref{bday_match_subsec}.  If a team gets stuck, point them
there.
\end{staffnotes}

\begin{problem}
Suppose there are $n$ students and $d$ days in the year, and let $D$
be the number of pairs of students with the same birthday.  Let
$B_1,B_2,\dots,B_n$ be the birthdays of $n$ independently chosen
people, and let $E_{i,j}$ be the indicator variable for the event
$[B_i = B_j]$.  

\bparts What is $\expect{E_{i,j}}$?
\begin{solution}
  Also, the expectations of $E_{i,j}$ for $i \neq j$
equals the probability that $B_i = B_j$, namely, $1/d$.
\end{solution}

Now, $D$, the number of matching pairs of birthdays among the $n$
choices is simply the sum of the $E_{i,j}$'s:
\begin{equation}\label{Vn}
D \eqdef \sum_{1\le i < j \le n} E_{i,j}.
\end{equation}
So by linearity of expectation
\[
\expect{D} = \expect{\sum_{1\le i < j \le n} E_{i,j}} = 
               \sum_{1\le i < j \le n} \expect{E_{i,j}} =
               \binom{n}{2}\cdot \frac{1}{d}.
\]
Also, by Theorem~\ref{th:varsum}, the variances of pairwise independent
variables are additive, so
\[
\variance{D} = \variance{\sum_{1\le i < j \le n} E_{i,j}} = 
               \sum_{1\le i < j \le n} \variance{E_{i,j}} =
               \binom{n}{2} \cdot \frac{1}{d}\paren{1-\frac{1}{d}}.
\]

In particular, for a class of $n= 85$ students with $d=365$ possible
birthdays, we have $\expect{D} \approx 9.7$ and $\variance{D} < 9.7 (1-
1/365) < 9.7$.  So by Chebyshev's Theorem
\[
\pr{\abs{D - 9.7} \geq x} < \frac{9.7}{x^2}.
\]

Letting $x=5$, we conclude that there is a better than 50\% chance that in
a class of 85 students, the number of pairs of students with the same
birthday will be between 5 and 14.

%In fact, there turned out to be
%\emph{exactly} the 16 matches expected in the class this term!

\iffalse
Add another calculation, say for a 6.001 class of 400, or for uniform
selection of numbers from 1 to 1,000,000.
\fi


\end{problem}

%%%%%%%%%%%%%%%%%%%%%%%%%%%%%%%%%%%%%%%%%%%%%%%%%%%%%%%%%%%%%%%%%%%%%
% Problem ends here
%%%%%%%%%%%%%%%%%%%%%%%%%%%%%%%%%%%%%%%%%%%%%%%%%%%%%%%%%%%%%%%%%%%%%

\endinput
