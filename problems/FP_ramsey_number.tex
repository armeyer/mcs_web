\documentclass[problem]{mcs}

\begin{pcomments}
  \pcomment{FP_ramsey_number}
  \pcomment{combinatorial proof of friends-and-strangers theorem}
  \pcomment{probably more suited for CP or PS?}
  \pcomment{CH, Spring '14}
\end{pcomments}

\pkeywords{
  combinatorics
  pigeon hole principle
  ramsey number
  friends and strangers
}

%%%%%%%%%%%%%%%%%%%%%%%%%%%%%%%%%%%%%%%%%%%%%%%%%%%%%%%%%%%%%%%%%%%%%
% Problem starts here
%%%%%%%%%%%%%%%%%%%%%%%%%%%%%%%%%%%%%%%%%%%%%%%%%%%%%%%%%%%%%%%%%%%%%

\begin{problem}

The \emph{complete graph}, $K_n$, is a simple graph with $n$ vertices
and an edge between every pair of vertices.   We'll refer to length-3
cycles as \emph{triangles}.

\bparts

\ppart Show that the number of triangles in $K_6$ equals 20.

\begin{solution}
Every set of three vertices in $K_n$ determines a triangle, so there are
\iffalse
Denote the vertices of $K_6$ as $V = \{v_1, v_2, \ldots, v_6\}$.   There
is a bijection between triangles in $K_6$ and subsets of $V$ of size
3.  Therefore, the number of triangles in $K_6$ is
\fi
\[
\binom{6}{3} = 20
\]
triangles.
\end{solution}

\examspace[1.5in]

\ppart

Now suppose that every edge in $K_6$ is colored either red or blue.  A
\emph{red-blue sequence} is a sequence of three distinct vertices
$(x,v,y)$ such that $\edge{x}{v}$ is red and $\edge{v}{y}$ is blue.
Show that the number of red-blue sequences is at most 36.

\begin{solution}
Fix the middle vertex $v$.  Suppose $v$ is adjacent $r$ vertices through a red
edge and $b$ vertices through a blue edge.  So $r+b = 5$, the
degree of $v$.  The number of choices for red-blue sequences with $v$
as the middle vertex is $rb$, which attains a maximum value of 6 for $(r,b) =
(2, 3)$ or $(3,2)$.  Summing over different $v$, the total number of red-blue sequences
is at most $6 \times 6 = 36$.
\end{solution}

\examspace[1.5in]

\ppart A triangle in is \emph{2-colored} if it has both a red and a
blue edge.  Describe a 2-to-1 mapping between red-blue sequences and
the set of 2-colored triangles.

\begin{solution}
We claim that the edges of each 2-color triangle appear in exactly two
red-blue sequences, so mapping each such triangle to its two red-blue
sequences defines the 2-to-1 mapping.

To see why a 2-color triangle has exactly two red-blue sequences,
suppose the vertices are $\set{x,v,y}$.  We have two separate cases:

\begin{enumerate}
\item Only one edge is colored blue, say edge $\edge{x}{y}$.
  Then $(v,x,y)$ and $(v,y,x)$ are the only red-blue sequences.
\item Only one edge is colored red, say $\edge{x}{y}$.  Then
  $(x,y,v)$ and $(y,x,v)$ are the only red-blue sequences.
\end{enumerate}
\end{solution}

\examspace[1.5in]

\ppart Use the facts proved above to prove the following statement:

\begin{quote}
Every collection of 6 people always includes a group of 3 mutual
acquaintances, or a group of 3 mutual strangers.
\end{quote}

\begin{solution}
Form the complete graph with 6 vertices and color edge $\edge{i}{j}$
red if the $i$th and $j$th people are acquainted, and color it blue if
they are strangers.  Then, the statement is equivalent to proving that
$K_6$ always includes a \emph{monochromatic} triangle, that is, a
triangle that is not 2-colored.

But we showed above that there are at most thirty-six red-blue triples
and every 2-color triangle maps to exactly 2 such triples.  Therefore,
there are at most eighteen 2-color triangles.  The total number of
triangles is $\bin{6}{3} = 20$, and therefore $K_6$ contains at least
$20 -18 = 2$ monochromatic triangles.  So in fact, we have proved a
that there is more than one group of all friends or all strangers.

\end{solution}

\eparts

\end{problem}

%%%%%%%%%%%%%%%%%%%%%%%%%%%%%%%%%%%%%%%%%%%%%%%%%%%%%%%%%%%%%%%%%%%%%
% Problem ends here
%%%%%%%%%%%%%%%%%%%%%%%%%%%%%%%%%%%%%%%%%%%%%%%%%%%%%%%%%%%%%%%%%%%%%

\endinput
