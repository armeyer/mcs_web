\documentclass[problem]{mcs}

\begin{pcomments}
  \pcomment{PS_dense_pigeon} \pcomment{ARM 5/16/15}
  \pcomment{Parts(a-b) from suggestion by R.C. Spence}
\end{pcomments}

\pkeywords{
  pigeon_hole
  irrational
  dense
  circle
  clockwise
}

%%%%%%%%%%%%%%%%%%%%%%%%%%%%%%%%%%%%%%%%%%%%%%%%%%%%%%%%%%%%%%%%%%%%%
% Problem starts here
%%%%%%%%%%%%%%%%%%%%%%%%%%%%%%%%%%%%%%%%%%%%%%%%%%%%%%%%%%%%%%%%%%%%%

\begin{problem}
Let's start by marking a point on a circle of length one.  Next, mark
the point that is distance $\sqrt{2}$ clockwise around the circle.  So
you wrap around once and actually mark the point at distance $\sqrt{2}
- 1$ clockwise from the start.  Now repeat with the newly marked point
as the starting point.  In other words, the marked points are those at
clockwise distances
\[
0, \sqrt{2},\ 2\sqrt{2},\ 3\sqrt{2},\dots,\ n\sqrt{2},\dots,
\]
from the start.

We will use a pigeonhole argument to prove that marked points are
\emph{dense} on the circle: for any point $p$ on the circle, and any
$\epsilon >0$, there is a marked point within distance $\epsilon$ of
$p$.

\bparts

\ppart\label{nmark2} Prove that no point gets marked twice.  That is,
the points at clockwise distance $k\sqrt{2}$ and $m\sqrt{2}$ are the
same iff $k = m$.

\begin{solution}
The points going $k\sqrt{2}$ and $m \sqrt{2}$ clockwise around the
circle coincide when they differ by some multiple of the
circumference.  In particular, they can coincide iff $k\sqrt{2} -
m\sqrt{2}$ is an integer $j$.  But if $k \neq m$, then we would have
\[
\sqrt{2} = \frac{j}{k-m},
\]
contradicting the fact that $\sqrt{2}$ is irrational.
\end{solution}

\ppart\label{markclose} Prove that among the first $n>1$ marked
points, there have to be two that are at most distance $1/n$ from each
other.

\begin{solution}
Let the pigeon holes be the $n$ half-open real intervals of the form
\[
\left[\right.k/n,\ (k+1)/n\left.\right)
\]
for $k=0,1,\dots,n-1$.  Let the pigeons be the points on the circle
that are distances $m\sqrt{2}$ clockwise from the starting point, for
$m = 0, 1,\dots, n$.  By part~\eqref{nmark2}, there are $n+1$ pigeons.
Map pigeon $r$ to the hole that contains $r-\floor{r}$.  By the
pigeonhole principle, there are two pigeons in the same hole, which
means these two pigeons are within clockwise distance $1/n$.

There is also a simple alternative argument based on length: cutting
the circumference at $n>1$ marked points yields $n$ positive length
arcs whose total length is one.  But if all the arcs are longer than
$1/n$, their total length would exceed one.  So there must be an arc
of length $\leq 1/n$, and its endpoints are the desired two marked
points.
\end{solution}

\ppart Prove that every point on the circle is within $1/n$ of a
marked point.  This implies the claim that the marked points are dense
on the circle.

\begin{solution}
By part~\eqref{markclose}, there are two marked points at distances
$k\sqrt{2}$ and $m \sqrt{2}$ around the circle that are clockwise
distance at most $1/n$ apart, where $0 \leq k < m\leq n-1$.  So
starting at one of the points and going clockwise distance
$(m-k)\sqrt{n}$ arrives at the second point.  Then again going
distance $(m-k)\sqrt{n}$ from the second point arrives at a third
point a further distance $1/n$.  Repeating this around the whole
circle will yield a sequence of marked points, each distance at most
$1/n$ from the next, going completely around the circle.  Now every
point on the circle will be within $1/n$ of one of these marked
points.
\end{solution}

\eparts

\end{problem}

\endinput
