\documentclass[problem]{mcs}

\begin{pcomments}
  \pcomment{CP_gambler_fair_expected_time}
  \pcomment{revised from: S08.cp13m by ARM, 5/9/10}
\end{pcomments}

\pkeywords{
  probability
  expectation
  random_walk
  gamblers_ruin
  linear-recurrence
}

%%%%%%%%%%%%%%%%%%%%%%%%%%%%%%%%%%%%%%%%%%%%%%%%%%%%%%%%%%%%%%%%%%%%%
% Problem starts here
%%%%%%%%%%%%%%%%%%%%%%%%%%%%%%%%%%%%%%%%%%%%%%%%%%%%%%%%%%%%%%%%%%%%%

\begin{problem}
In the fair Gambler's Ruin game with initial stake of $n$ dollars and
target of $T$ dollars, let $e_n$ be the number of \$1 bets the gambler makes
until the game ends (because he reaches his target or goes broke).
\begin{problemparts}
\ppart  Describe constants $a,b,c$ such that
\begin{equation}\label{enaen-1}
e_n = ae_{n-1} + be_{n-2} + c.
\end{equation}
for $1 < n < T$.
\begin{solution}
\[
e_{n}  = 2e_{n-1} - e_{n-2} - 2.
\]
That is, $a=2, b = -1, c = -2$.

This follows \inbook{from equation~(\ref{expected-bets-recurrence}) with $p=q=1/2$.}\inhandout{because
\begin{align*}
e_{n-1} & = \expcond{e_{n-1}}{\text{gambler wins first bet}}
             \cdot \pr{\text{gambler wins first bet}}\\
       & \quad + \expcond{e_{n-1}}{\text{gambler loses first bet}}
             \cdot \pr{\text{gambler loses first bet}}
                       & \text{(Total Expectation)}\\
    & =  (1+ e_n) \cdot \frac{1}{2} + (1+ e_{n-2}) \cdot \frac{1}{2}\\
    & = \frac{2 + e_n + e_{n-2}}{2}.
\end{align*}}
\end{solution}

\ppart Let $e_n$ be defined by~\eqref{enaen-1} for all $n > 1$, where
$e_0 = 0$ and $e_1 = d$ for some constant $d$.  Derive a closed form
(involving $d$) for the generating function $E(x) \eqdef \sum_0^\infty
e_nx^n$.

\begin{solution}
\[\begin{array}{rcrrrr}
E        & = & e_0 + &   e_1x + &  e_2x^2 + & \dots\\
-2xE     & = &     - &  2e_0x - & 2e_1x^2 - & \dots\\
 x^2E    & = &       &          &  e_0x^2 + & \dots\\
2/(1-x) & = &  2  + &     2x + &    2x^2 + & \dots\\
\hline
E - 2x E + x^2E + 2/(1-x)
         & = &  2  + & (d + 2)x + &  0x^2 + & \dots
\end{array}\]
so
\begin{align*}
E(x) & = \frac{2 + (d+2)x - 2/(1-x)}{1-2x+x^2}\\
     & = \frac{(1-x)(2 + (d+2)x) - 2}{(1-x)^3}\\
     & = \frac{(dx - (d+2)x^2}{(1-x)^3}
\end{align*}
By partial fractions,
\begin{equation}\label{ExABC}
E(x) = \frac{A}{1-x} + \frac{B}{(1-x)^2} +\frac{C}{(1-x)^3}
\end{equation}
so multiplying both sides by the denominator $(1-x)^3$ yields
\[
dx - (d+2)x^2 = A(1-x)^2 + B(1-x) + C.
\]
Letting $x =1$ yields
\[
-2 = d - (d+2) = C.
\]

Letting $x =0$ yields
\[
2 = A+B.
\]

Letting $x=-1$ yields
\[
-2(d+1) = 4A+2B+C = 2A + 2(A+B) -2 = 2A + 2
\]
so $A = -(d+2)$ and $B = d+4$.

\end{solution}

\ppart\label{closeden} Find a closed form (involving $d$) for $e_n$.

\begin{solution}
  It follows from~\eqref{ExABC} that
\begin{align}
e_n & =  A + B(n+1) + C\binom{n+2}{2}\notag\\
    & = -(d+2) + (d+4)(n+1) - (n+2)(n+1)\notag\\
%    & = -d -2 + dn +4n + d +4 - n^2 -3n -2\\
%    & = dn +n - n^2
     & = n(d +1 -n). \label{enndn+2}
\end{align}
\end{solution}

\ppart Use part~\eqref{closeden} to solve for $d$.

\begin{solution}
Letting $n=T$ in~\eqref{enndn+2},
\[
0 = e_T = T(d+1-T)
\]
so
\begin{equation}\label{dn2n1}
d = T-1.
\end{equation}
\end{solution}

\ppart Prove that $e_n = n(T-n)$.

\begin{solution}
\begin{align*}
e_n & = n((T-1) +1 - n) & \text{(by~\eqref{enndn+2} \&~\eqref{dn2n1})}\\
    & = n(T-n).
\end{align*}
\end{solution}

\end{problemparts}

\end{problem}

\endinput
