\documentclass[problem]{mcs}

%PS_mathematician_set

\begin{pcomments}
  \pcomment{TP_mathematician_set}
  \pcomment{renamed from PS_mathematician_set}
  \pcomment{from: F09.ps2, S02.ps3}
  \pcomment{vague ``describe'' question.  Not suitable for pset}
  \pcomment{edited ARM 2.15.12}
\end{pcomments}
\pkeywords{
  set_theory
  relations
  relational_properties
  composition
  path
  walk
}

%%%%%%%%%%%%%%%%%%%%%%%%%%%%%%%%%%%%%%%%%%%%%%%%%%%%%%%%%%%%%%%%%%%%%
% Problem starts here
%%%%%%%%%%%%%%%%%%%%%%%%%%%%%%%%%%%%%%%%%%%%%%%%%%%%%%%%%%%%%%%%%%%%%

\begin{problem}

Let $R$ be the relation on the set, $M$, of all mathematicians defined
by the rule:
\[
a \mrel{R} b \QIFF \text{mathematicians $a$ and $b$ have 
written a paper together}.
\]

\bparts

\ppart
Describe the relation $R^2 \eqdef R \compose R$.

\begin{solution}
Two mathematicians are related under $R^2$ if and only if
each has written a joint paper with some mathematician, $c$.
\end{solution}

\ppart
Describe the path relation $R^+ \eqdef \union R \union R^2 \union
R^3 \union \cdots$.

\begin{solution}
Two mathematicians are related under $R^+$ if there is a
finite sequence of mathematicians $a=c_0, c_1, c_2,..., c_{m-1},
c_m=b$, with $m \geq 1$, such that for each $i$ from 1 to $m$,
mathematician $c_i$ has written a joint paper with mathematician
$c_{i-1}$.
\end{solution}

\ppart
The \href{http://en.wikipedia.org/wiki/Erd\%C5\%91s_number}{\emph{Erd\H{o}s number}}
of a mathematician is 1 if this mathematician wrote a paper with the 
prolific Hungarian mathematician Paul Erd\H{o}s, it is 2 if this mathematician 
did not write a joint paper with Erd\H{o}s but wrote a joint paper with
someone who wrote a joint paper with Erd\H{o}s, and so on (except that
the Erd\H{o}s number of Erd\H{o}s himself is 0).  Give a definition of
the Erd\H{o}s number in terms of paths in $R$.

\begin{solution}
The Erd\H{o}s number of $a$ is the length of a shortest path
in $R$ from $a$ to Erd\H{o}s, if such a path exists.  (Some
mathematicians have no Erd\H{o}s number).
\end{solution}

\eparts
\end{problem}

%%%%%%%%%%%%%%%%%%%%%%%%%%%%%%%%%%%%%%%%%%%%%%%%%%%%%%%%%%%%%%%%%%%%%
% Problem ends here
%%%%%%%%%%%%%%%%%%%%%%%%%%%%%%%%%%%%%%%%%%%%%%%%%%%%%%%%%%%%%%%%%%%%%

\endinput
