\documentclass[problem]{mcs}

\begin{pcomments}
  \pcomment{MQ_stamps_6_14_21_wop}
  \pcomment{F16, midterm1}
  \pcomment{ARM 2/24/16}
  \pcomment{was CP_TBA1}
\end{pcomments}

\pkeywords{
  WOP
  stamp
  makeable
}

\begin{problem}

\inbook{
\bparts
\ppart
}

Prove using the Well Ordering Principle that, using 6\textcent,
14\textcent, and 21\textcent\ stamps, it is possible to make any
amount of postage over 50\textcent.  To save time, you may specify
\emph{assume without proof} that 50\textcent, 51\textcent, \dots
100\textcent are all makeable, but you should clearly indicate which
of these assumptions your proof depends on.

\begin{solution}
\begin{proof}
Assume to the contrary that some amount of postage of
$50$\textcent\ or more is not makeable.  So by WOP, there will be a
\emph{least} unmakeable amount $m \geq 50$.  If we assume $k=6$, then
we can conclude that $m \geq 56$.  So $m-6 \geq 50$ and therefore is
makeable because anything $\geq 50$ and less than $m$ is makeable by
definition of $m$.  Now since $m-6$ is makeable, we can add a
6\textcent\ stamp and make $(m-6)+6 = m$\textcent, contradicting the
fact that $m$ is unmakeable.  So there cannot be such a minimum $m$,
which proves that all amounts $\geq 50$\textcent\ are makeable.
\end{proof}
\end{solution}

\iffalse
\ppart What is the smallest value of $k$ for which your proof will
work?
\begin{solution}
The proof above uses $k=6$, which is the smallest $k$ that will work
for a proof organized in this way.  Of course, $k = 14$ or $k=21$ will
allow the same proof to go through adding a 14 or 21\textcent\ stamp
instead of 6
\end{solution}
\fi

\inbook{
\ppart Show that 49\textcent\ is not makeable amount.

\begin{solution}
\begin{tabular}{}
at & most one 21.
   & if & one 21, then at most one 15.
   &    & if one 21 and one 15,
   &    & then 49-(21+15) = 13 is not a multiple of 6, so not makeable.
   &    & if one 21 and no 15,
   &    & then 49-21 = 28 is not a multiple of 6, so not makeable.
   &if no 21,
   &    & then at most three 15's,
   &    & but none of (49-45), (49-30), and (49-15) are multiples of 6, so not makeable.
\end{tabular}
\end{solution}
\eparts
}

\end{problem}

\endinput
