\documentclass[problem]{mcs}

\begin{pcomments}
  \pcomment{PS_binary_gcd}
  \pcomment{By ARM 4/24/11 based on Shoup}
  \pcomment{soln edited ARM 3/16/13}
  \pcomment{lighter version in MQ_binary_gcd}
\end{pcomments}

\pkeywords{
  preserved_invariant
  GCD
  binary_GCD
  state_machine
}

%%%%%%%%%%%%%%%%%%%%%%%%%%%%%%%%%%%%%%%%%%%%%%%%%%%%%%%%%%%%%%%%%%%%%
% Problem starts here
%%%%%%%%%%%%%%%%%%%%%%%%%%%%%%%%%%%%%%%%%%%%%%%%%%%%%%%%%%%%%%%%%%%%%

\begin{problem}
  The binary-GCD state machine computes the GCD of $a$ and $b$ using
  only division by 2 and subtraction, which makes it run very
  efficiently on hardware that uses binary representation of numbers.
  In practice, it runs more quickly than the more famous Euclidean
  algorithm described in Section~\bref{sec: Euclid}.

\begin{align*}
\text{states} & \eqdef \naturals^3\\
\text{start state} & \eqdef (a, b, 1)
             & \text{(where $a > b >0$)}\\
\text{transitions} & \eqdef \text{ if } \min(x,y) > 0, \text{ then } (x,  y, e) \movesto\\
     &\qquad \text{the first possible state according to the rules:}\\
     &\qquad
       \begin{cases}
       (1, 0, ex)     & \text{(if $x = y$)}\\
       (1, 0, e)      & \text{(if $y=1$)},\\
       (x/2, y/2, 2e) & \text{(if $2 \divides x$ and $2 \divides
           y$)},\\
       (x/2, y, e)    & \text{(if $2 \divides x$)}\\
       (x, y/2, e)    & \text{(if $2 \divides y$)}\\
       (y, x, e)      & \text{(if $y>x$)}\\
       (x-y,y,e)      & \text{(otherwise)}.
       \end{cases}
\end{align*}

\bparts 

\ppart Use the Invariant Principle \iffalse
(Section~\bref{subsec:invariant})\fi to prove that if this machine
stops, that is, reaches a state $(x,y,e)$ in which no transition is
possible, then $e = \gcd(a,b)$.

\hint There are only three ways the machine can stop.

\begin{solution}
We claim that a preserved invariant of this machine is
\[
\gcd(a,b) = e\gcd(x,y).
\]
To show this, we assume the invariant holds for state $(x,y,e)$ and
show that if $(x,y,e) \movesto (x',y',e')$, then $\gcd(a,b) =
e'\gcd(x',y')$.

The proof is by cases according to which kind of transition occurs.

\inductioncase{Case}: ($x=y$).  Since $x=y$, we have $\gcd(x,y) = x$,
and since the invariant holds for $(x,y,e)$, we conclude that
\begin{equation}\label{gcdabex}
\gcd(a,b) = ex.
\end{equation}
In this case, $(x',y',e') = (1,0,ex)$, so
\begin{align*}
e'\gcd(x',y')
   & = ex\gcd(1,0)\\
   & = ex\\
   & = \gcd(a,b) & \text{(by~\eqref{gcdabex})},
\end{align*}
which shows that the invariant holds for $(x',y',e')$.

\inductioncase{Case}: ($2 \divides x$ and $2 \divides y$).  We use the
easily verified fact that $\gcd(au,av) = a\gcd(u,v)$.  So in this
case, we have $\gcd(x,y) = 2\gcd(x/2,y/2)$.  Since the invariant holds
for $(x,y,e)$, we conclude that
\begin{equation}\label{gcdabe2x2}
\gcd(a,b) = e 2\gcd(x/2,y/2).
\end{equation}
Now $(x',y',e')= (x/2, y/2, 2e)$, so
\begin{align*}
e'\gcd(x',y')
 & = 2e\gcd(x/2,y/2)\\
& = \gcd(a,b) & \text{(by~\eqref{gcdabe2x2})},
\end{align*}
which shows that the invariant holds for $(x',y',e')$.

\inductioncase{Case}: $2 \divides x$ and 2 does not divide $y$.  Now
\begin{equation}\label{gcdxyx2}
\gcd(x,y) = \gcd(x/2,y),
\end{equation}
and $(x',y',e') = (x/2,y, e)$.  So
\begin{align*}
\gcd(a,b)
  & = e\gcd(x,y)
      & \text{(invariant for $(x,y,e)$)}\\
  & = e\gcd(x/2,y)
      & \text{(by~\eqref{gcdxyx2})}\\
  & = e'\gcd(x',y'),
\end{align*}
proving that the invariant holds for $(x',y',e')$.

\inductioncase{Case}: (otherwise clause).
We use the easily verified fact that
\begin{equation}\label{gcdu-v}
\gcd(u-v,v) = \gcd(u,v).
\end{equation}
In this case $(x',y',e') = (x-y,y,e)$, so
\begin{align*}
\gcd(a,b)
  & = e\gcd(x,y) & \text{(invariant for $(x,y,e)$)}\\
  & = e\gcd(x-y,y) & \text{(by~\eqref{gcdu-v })}\\
  & = e'\gcd(x',y'),
\end{align*}
proving that the invariant holds for $(x',y',e')$.

Verification of the remaining cases follows similarly.

To apply the Invariant Principle, we now first observe that the
preserved invariant holds trivially in the start state $(a,b,1)$
because $\gcd(a,b) = 1\cdot\gcd(a,b)$, allowing us to conclude that
the preserved invariant holds in every reachable state.

The final transition rule to state $(x-y,y,e)$ is only possible if
$x>y>0$, so another transition will be possible after applying this
rule.  A transition is also possible from the start state.  It follows
that there are only two ways to reach a stopped state:
\begin{align}
(x,x,e) & \movesto (1, 0, ex) &\text{ or} \label{xxe10ex}\\
(x,1,e) & \movesto (1, 0, e).  \label{x1e10ex}
\end{align}

In case~\eqref{xxe10ex}, we have
\begin{align*}
\gcd(a,b)
   & = e\gcd(x,x)
       & (invariant for $(x,x,e)$)\\
   & = ex = e',
\end{align*}
as required.

In case~\eqref{xxe0xe}, we have
\begin{align*}
\gcd(a,b)
   & = e\gcd(x,x)
       & (invariant for $(x,x,e)$)\\
   & = e\cdot 1 = e',
\end{align*}
as required.
\end{solution}

\ppart Prove that the machine reaches a final state in at most
$1+3(\log a +\log b)$ transitions.  (This is a coarse bound; you may
be able to get a better one.)

%\hint Strong induction on $\max(a,b)$.

\begin{solution}
Either $x$ or $y$ gets halved after at most three transitions---the
worst case is when $x$ and $y$ are both odd and $y > x > 1$.  In that
case, the first transition switches $x$ and $y$, the next subtracts
one from the other yielding an even value of $x$, and the third
transition will halve $x$.  So after at most $3(\log a + \log b)$
transitions, one of $x$ and $y$ must have been reduced to 1, after
which there can be at most one more transition.
\end{solution}

\eparts

\end{problem}

%%%%%%%%%%%%%%%%%%%%%%%%%%%%%%%%%%%%%%%%%%%%%%%%%%%%%%%%%%%%%%%%%%%%%
% Problem ends here
%%%%%%%%%%%%%%%%%%%%%%%%%%%%%%%%%%%%%%%%%%%%%%%%%%%%%%%%%%%%%%%%%%%%%

\endinput

