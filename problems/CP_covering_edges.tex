\documentclass[problem]{mcs}

\begin{pcomments}
  \pcomment{CP_covering_edges}
  \pcomment{from: S09.cp7r. S07.cp7r}
  \pcomment{edited by ARM 10/17/09}
\end{pcomments}

\pkeywords{
  relations
  digraphs
  covering_edges
  transitive_closure
  DAGs
}

%%%%%%%%%%%%%%%%%%%%%%%%%%%%%%%%%%%%%%%%%%%%%%%%%%%%%%%%%%%%%%%%%%%%%
% Problem starts here
%%%%%%%%%%%%%%%%%%%%%%%%%%%%%%%%%%%%%%%%%%%%%%%%%%%%%%%%%%%%%%%%%%%%%

\begin{problem}
If $a$ and $b$ are distinct nodes of a digraph, then $a$ is said to
\term{cover} $b$ if there is an edge from $a$ to $b$ and every path from
$a$ to $b$ traverses this edge.  If $a$ covers $b$, the edge from $a$ to
$b$ is called a \term{covering edge}.

\bparts

\ppart Show that if two digraphs have the same positive path relation,
then they have the same covering edges.

\begin{solution}
  Suppose two digraphs, $C$ and $D$, have the same positive path
  relation, and consider any covering edge $\diredge{a}{b}$ in graph $C$.
  Since this edge by itself is a positive length path from $a$ to $b$
  in $C$, and $D$ has the same positive length path relation, there must
  be a positive length path from $a$ to $b$ in $D$.

  Now suppose this positive length path did not traverse $\diredge{a}{b}$.
  Then there must be a path from $a$ to $c$ to $b$ for some $c$ not equal
  to $a$ or $b$.  \textbf{NOT DONE YET.}

\end{solution}


\ppart\label{12pospath} Describe two graphs with vertices $\set{1,2}$
which have the same set of covering edges, but not the same positive path
relation (\hint Self-loops.)

\begin{solution}
  Let one graph have edges $\set{(1,2), (1,1)}$ and the other
  $\set{(1,2),(2,2)}$.  They have the same set of covering edges, namely,
  $(1,2)$.  But in the second there is a positive length path from 2 to 2,
  namely a path of length one but there is no positive length path from 2
  to 2 in the first graph.
\end{solution}

\ppart\label{nocoveringedges} What are the covering edges of the
\term{complete digraph} with vertices $1,2,3$ which has edges between every
two distinct vertices?

\begin{solution}
 There are no covering edges, since there is always a length two path from
 $a$ to $b$ that does not use the edge $\diredge{a}{b}$.

 So this graph has the same covering edges as the empty relation on
 $\set{1,2,3}$.
\end{solution}

\ppart What are the covering edges in the following DAG?

%\mfigure{!}{3in}{div-dag}

\mfigure{!}{2.5in}{divisibility}
\begin{solution}
TBA
\end{solution}

\ppart\label{+path=} In contrast to the case illustrated in
parts~\eqref{12pospath} and~\eqref{nocoveringedges} of digraphs with
cycles, show that if two \emph{DAG}'s have the same set of covering edges,
then they have have the same positive path
relation.

\newcommand{\covering}[1]{\text{covering}\paren{#1}}

\ppart\label{cover-ok} Let $\covering{D}$ be the subgraph of $D$ consisting
of only the covering edges.  For any \idx{DAG}, $D$, explain why $\covering{D}$
has the same positive path relation as $D$.

\hint Consider \emph{longest} paths between a pair of vertices.

\begin{solution}
What we need to show is that if there is a path in $D$ between
vertices $a \neq b$, then there is a path consisting only of covering
edges from $a$ to $b$.  But since $D$ is a finite DAG, there must be a
\emph{longest} path from $a$ to $b$.  Now every edge on this path must be a
covering edge or it could be replaced by a path of length 2 or more,
yielding a longer path from $a$ to $b$.
\end{solution}

\ppart Conclude that $\covering{D}$ is the \emph{unique} DAG with the smallest
number of edges among all digraphs with the same positive path relation as
$D$.

\begin{solution}
  By part~\eqref{+path=}, any DAG with the same positive path relation as
  $D$ must contain all the edges of $\covering{D}$, so the unique
  minimality of $\covering{D}$ follows immediately from
  part~\eqref{cover-ok}.

\end{solution}

\iffalse
\ppart Show that the previous result is not true in general infinite
DAG's.

\hint Consider the DAG for the total order on the rational numbers.

\begin{solution}
In the DAG for $<$ on the $\rationals$, there are no covering
edges, so $\widehat{<}$ has no edges.
\end{solution}
\fi


\eparts
\end{problem}

%%%%%%%%%%%%%%%%%%%%%%%%%%%%%%%%%%%%%%%%%%%%%%%%%%%%%%%%%%%%%%%%%%%%%
% Problem ends here
%%%%%%%%%%%%%%%%%%%%%%%%%%%%%%%%%%%%%%%%%%%%%%%%%%%%%%%%%%%%%%%%%%%%%



\iffalse
Let $D$ be a finite Directed Acyclic Graph (\idx{DAG}).

\bparts

\ppart Explain in one (maybe two) sentences why the \idx{positive path
  relation} of $D$ is obviously a strict partial order.

\begin{solution}
Since following a path from $x$ to $y$ and then a path from $y$ to $z$ is
itself a path from $x$ to $z$, transitivity of the path relation (positive
or not) is immediate.  Also, a path from $x$ to $y$ combined with a path
from $y$ to $x$ would be a cycle, so if a graph is acyclic, its path
relation must be asymmetrical.
\end{solution}

\eparts
\fi
