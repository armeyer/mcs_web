\documentclass[problem]{mcs}

\begin{pcomments}
  \pcomment{FP_structural_linear_product}
  \pcomment{ARM 5/5/16}
\end{pcomments}

\pkeywords{
  structural_induction
  linear_function
}

\providecommand{\wtav}{\text{N23}}

%%%%%%%%%%%%%%%%%%%%%%%%%%%%%%%%%%%%%%%%%%%%%%%%%%%%%%%%%%%%%%%%%%%%%
% Problem starts here
%%%%%%%%%%%%%%%%%%%%%%%%%%%%%%%%%%%%%%%%%%%%%%%%%%%%%%%%%%%%%%%%%%%%%

\begin{problem}
The \emph{2-3-averaged numbers} are a subset, \wtav, of the real
interval $[0,1]$ defined recursively as follows:

\inductioncase{Base cases:} $0,1 \in \wtav$.

\inductioncase{Constructor case:}
If $a,b$ are in \wtav, then so is $L(a,b)$ where
\[
L(a,b) \eqdef \frac{2a+3b}{5}.
\]

\bparts

\ppart Use ordinary induction or the Well-Ordering Principle to prove
that
\[
\paren{\frac{3}{5}}^n \in \wtav
\]
for all nonnegative integers $n$.

\examspace[2.5in]

\begin{solution}
\TBA{proof}
\end{solution}

\ppart Prove by Structural Induction that the product of two
2-3-averaged numbers is also a 2-3-averaged number.

\hint
\[
P(c) \eqdef\quad \forall d \in \wtav.\ \ c \cdot d \in \wtav.
\]

\begin{solution}
\inductioncase{Base cases:} $P(0)$ holds since $0\cdot b = 0 \in
\wtav$.  Likewise, $P(1)$ holds since $1 \cdot b = b \in \wtav$.

\inductioncase{Constructor case:} ($c = L(a,b)$ for $a,b \in \wtav$).
So
\[
c \cdot d = \frac{2a+3b}{5} \cdot d = \frac{2(ad)+3(bd)}{5} = L(ad, bd).
\]
Now $ad, bd \in \wtav$ by structural induction hypothesis, and
therefore $c\cdot d = L(ad,bd) \in \wtav$ by the Constructor step.
\end{solution}

\eparts

\end{problem}

%%%%%%%%%%%%%%%%%%%%%%%%%%%%%%%%%%%%%%%%%%%%%%%%%%%%%%%%%%%%%%%%%%%%%
% Problem ends here
%%%%%%%%%%%%%%%%%%%%%%%%%%%%%%%%%%%%%%%%%%%%%%%%%%%%%%%%%%%%%%%%%%%%%

\endinput


%% \ppart Prove using the Well-Ordering Principle that
%% \[
%% \paren{\frac{2}{5}}^n \in \wtav
%% \]
%% for all nonnegative integers $n$.

%% \textbf{Do not overlook part~(c) on the next page.}

%% \begin{solution}
%% \TBA{proof}
%% \end{solution}

%% \examspace

