\documentclass[problem]{mcs}

\begin{pcomments}
  \pcomment{PS_relation_matrices}
  \pcomment{from: S02.cp3m}
\end{pcomments}

\pkeywords{
  relation
  matrix multiplication
  composition
}

%%%%%%%%%%%%%%%%%%%%%%%%%%%%%%%%%%%%%%%%%%%%%%%%%%%%%%%%%%%%%%%%%%%%%
% Problem starts here
%%%%%%%%%%%%%%%%%%%%%%%%%%%%%%%%%%%%%%%%%%%%%%%%%%%%%%%%%%%%%%%%%%%%%

\begin{problem}

Recall that the \idx{composition} of relations $R: A \to B$ and $S: B
\to C$ is the relation $S \compose R: A \to C$ defined by the rule
\[
a \mrel{(S \compose R)} c\quad \QIFF\quad \exists b.\;  (a \mrel{R} b) \QAND
(b \mrel{S} c).
\]

We can represent a relation $S$ between two sets $A$ and $B$ of size $n$ as
an $n \times n$ square matrix $M_S$, where the elements of $M_S$ are defined by
the rule
\[
i \mrel{S} j \quad\QIFF\quad  M_S (i, j)=1.
\] 

If we represent relations as matrices in this fashion, then we can compute the
composition of two relations by a ``boolean'' matrix multiplication of
their matrices.  Boolean matrix multiplication is the same as matrix
multiplication except that ``$+$'' is replaced by $\QOR$
and ``$\times$'' is replaced by $\QAND$.

Prove that the matrix representation of $S \compose R$ is equal to the
boolean product of $M_R$ and $M_S$ (note the reversal of $R$ and $S$),
where $M_R$ is the matrix representing $R$ and $M_S$ is the matrix
representing $S$.

\begin{solution}
\begin{proof}

Let $M_P$ be the boolean product of $M_R$ and $M_S$ (notice
that $M_P$, $M_R$ and $M_S$ are all $n\times n$ square matrices). What
we want to prove is that
\[
i \mrel{(S\compose R)} j \quad\QIFF\quad  M_P (i, j)=1.
\]

Recall that by the definition of composition, $i \mrel{(S\compose R)} j$
iff there exists a $k$ such that $i\mrel{R}k$ and $k\mrel{S}j$.  Also,
by the definition of boolean matrix multiplication,
\[
M_P (i,j) = \underbrace{\left[M_R(i,k_1) \QAND
M_S(k_1,j)\right]}_{k_1 \text{ is the ``link''}} \QOR
\underbrace{\left[M_R(i,k_2) \QAND
M_S(k_2,j)\right]}_{k_2 \text{ is the ``link''}}\QOR
\dots \QOR \underbrace{\left[M_R(i,k_n) \QAND
M_S(k_n,j)\right]}_{k_n \text { is the ``link''}}
\]

\begin{description}
\item[Case 1:($\QIMPLIES$)] If $i \mrel{(S\circ R)} j$, then for at
  least one $k$, say $k'$, $i \mrel{R} k'$ and
  $k'\mrel{S}j$.  Consequently, $M_R(i,k')=1$ and $M_S(k',j)=1$.  This
  turns $\left[M_R(i,k') \QAND M_S(k',j)\right]$ true, and hence
  $M_P(i, j)=1$.

\item[Case 2: ($\Longleftarrow$)] If $M_P(i, j)=1$, then there is at
  least one $k$, say $k'$, for which $\left[M_R(i,k') \QAND
    M_S(k',j)\right] =1$. This means that both $M_R(i,k')=1$ and
  $M_S(k',j)=1$.  Since $M_R$ and $M_S$ are the matrix representations
  of $R$ and $S$, we can conclude that $i\mrel{R}k'$ and
  $k'\mrel{S}j$, and so, by the definition of composition, $i
  \mrel{(S\compose R)} j$.
\end{description}

\end{proof} 
\end{solution}

\end{problem}



%%%%%%%%%%%%%%%%%%%%%%%%%%%%%%%%%%%%%%%%%%%%%%%%%%%%%%%%%%%%%%%%%%%%%
% Problem ends here
%%%%%%%%%%%%%%%%%%%%%%%%%%%%%%%%%%%%%%%%%%%%%%%%%%%%%%%%%%%%%%%%%%%%%

\endinput
