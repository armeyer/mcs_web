\documentclass[problem]{mcs}

\begin{pcomments}
  \pcomment{PS_bogus_Cantor}
  \pcomment{ARM 3.14.17}
\end{pcomments}

\pkeywords{
  diagonal_argument
  diagonal
  Cantor
  powerset
  countable
  uncountable
  subset
}

%%%%%%%%%%%%%%%%%%%%%%%%%%%%%%%%%%%%%%%%%%%%%%%%%%%%%%%%%%%%%%%%%%%%%
% Problem starts here
%%%%%%%%%%%%%%%%%%%%%%%%%%%%%%%%%%%%%%%%%%%%%%%%%%%%%%%%%%%%%%%%%%%%%

\begin{problem}
Cantor's Powerset Theorem\inbook{~\bref{powbig}} implies that for any
set $A$, no total function $g:A \to \power(A)$ is a surjection.  In
particular, the theorem is proved by showing that the ``diagonal'' set
\[
A_g \eqdef \set{a \in A \suchthat a \notin g(a)}
\]
is an element of $\power(A)$ that is not in the range of $g$, because
for every $a \in A$ it differs at the element $a$ from the set $g(a)$
in the range of $g$.

But there's no need to stick to the diagonal.  If $p:A \to A$ is some
total function, we can differ from $g(a)$ at $p(a)$ instead of ``on
the diagonal'' at $a$.  That is, define
\[
A_{g,p} \eqdef \set{p(a) \suchthat a \in A \QAND p(a) \notin g(a)}.
\]
So $A_{g,p}$ is an element of $\power(A)$, and
\begin{falseclm*}
$A_{g,p}$ is not in the range of $g$,
\end{falseclm*}
because for every $a \in A$, the set $A_{g,p}$ differs at the element
$p(a)$ from the set $g(a)$ in the range of $g$.

\bparts

\ppart\label{agn0} Show that the claim is false by letting $A$ be
$\set{0,1}$, and $g(n)= \set{n}$ and $p(n) = 0$ for all $n \in
\nngint$.

\begin{solution}
In this case, 
\begin{align*}
g(1) & = \set{1}  &\text{ so } p(1) = 0 \notin g(1),\text{ and}\\
A_{g,p} & = \set{0} = g(0).
\end{align*}
\end{solution}

\ppart Identify the mistake in the argument above and show that it is
fixed by requiring that $p$ be an injection.

\begin{solution}
The mistake is in the assertion that $A_{g,p}$ differs at the element
$p(a)$ from the set $g(a)$.  The problem is that $p(a)$ may equal
$p(b)$ for some $b in g(b)$.  This is exactly what goes wrong in the
counterexample of part~\eqref{agn0}, where $p(0) = p(1) = 0$.  Since $p(0)
\in g(0)$, it is not supposed to be in $A_{g,p}$, but $p(1) \notin
g(1)$ so $p(1)$ is in $A_{g,p}$.

On the other hand, if $p$ is an injection, then\iffalse $z \in
A_{g,p}$ implies that $z \notin g(p^{-1}(z))$, so \fi $A_{g,p}$ and
$g(a)$ really are guaranteed to differ at $p(a)$.
\end{solution}

\eparts

\end{problem}
\endinput


\iffalse
 we can prove by
contradiction that $A_{g,p}$ is not in the range of $g$.  Namely:

\begin{bogusproof}
Suppose to the contary that $A_{g,p}$ is in the range of $g$.
Then by defination of range, there is a element $a_0 \in A$ such that
$A_{g,p} = g(a_0)$.  Therefore
\begin{equation}\tag{*}
p(a) \in A_{g,p} \QIFF p(a) \in g(a_0)
\end{equation}
is trivially true for all $a \in A$.  But
\begin{equation}\tag{\#}
p(a) \notin A_{g,p} \QIFF p(a) \in g(a)
\end{equation}
for all $a \in A$ by definition of $A_{g,p}$.  Letting $a$ be $a_0$
in~(*) and~(\#) gives the contradiction
\[
p(a_0) \in A_{g,p} \QIFF p(a_0) \in g(a_0) \QIFF p(a_0) \notin  A_{g,p}.
\]

Since assuming $A_{g,p}$ is in the range of $g$ led to this
contradiction, we have proved that
\begin{falseclm*}
$A_{g,p}$ is not in the range of $g$.
  \end{falseclm*}
\end{bogusproof}
\fi
