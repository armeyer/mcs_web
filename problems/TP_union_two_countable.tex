\documentclass[problem]{mcs}

\begin{pcomments}
  \pcomment{TP_union_two_countable}
  \pcomment{by ARM 2/11/11; soln edited 3/2/17}
\end{pcomments}


\pkeywords{
  countable
  union
  duplicates
  list
}

%%%%%%%%%%%%%%%%%%%%%%%%%%%%%%%%%%%%%%%%%%%%%%%%%%%%%%%%%%%%%%%%%%%%%
% Problem starts here
%%%%%%%%%%%%%%%%%%%%%%%%%%%%%%%%%%%%%%%%%%%%%%%%%%%%%%%%%%%%%%%%%%%%%

\begin{problem}
%\begin{lemma}\label{countable-union}

Prove that if $A$ and $B$ are countable sets, then so is $A \union B$.

\begin{staffnotes}
  If students get stuck, give them the hint that it's just like the
  bijection between $\nngint$ and $\integers$ given in the
  notes~(\bref{intlist}).
\end{staffnotes}

\begin{solution}

\begin{proof}
If there is a list $a_0,a_1,\dots$ of the elements of $A$ and a list
$b_0,b_1, \dots$ of the elements of $B$, then
\begin{equation}\label{FP_a0b0list}
a_0,b_0,a_1,b_1, \dots a_n,b_n, \dots.
\end{equation}
is a list of the elements of in  $A\union B$.
\end{proof}

\end{solution}

\end{problem}

%%%%%%%%%%%%%%%%%%%%%%%%%%%%%%%%%%%%%%%%%%%%%%%%%%%%%%%%%%%%%%%%%%%%%
% Problem ends here
%%%%%%%%%%%%%%%%%%%%%%%%%%%%%%%%%%%%%%%%%%%%%%%%%%%%%%%%%%%%%%%%%%%%%

\endinput
