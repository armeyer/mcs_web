\documentclass[problem]{mcs}

\begin{pcomments}
  \pcomment{MQ_limit_quantifiers}
  \pcomment{DRAFT}
  \pcomment{10/16/17 ARM}
\end{pcomments}

\pkeywords{
 quantifier
 forall
 exists
 limit
 continuous
}

%%%%%%%%%%%%%%%%%%%%%%%%%%%%%%%%%%%%%%%%%%%%%%%%%%%%%%%%%%%%%%%%%%%%%
% Problem starts here
%%%%%%%%%%%%%%%%%%%%%%%%%%%%%%%%%%%%%%%%%%%%%%%%%%%%%%%%%%%%%%%%%%%%%

\begin{problem}
Let $f:\reals \to \reals$ be a total function on the real numbers.
The limit $\lim_{x \to r} f(x)$ of $f(x)$ as $x$ approaches some real
number $r$ is a number $a \in \reals$ such that $f(x)$ will be as
close as you want (within distance $\epsilon > 0$) to $a$ as long as
$x$ is close enough (within distance $\delta > 0$) to $r$.

This can be expressed by a logical formula of the form
\begin{equation}\tag{limitf}
Q_1 \epsilon \in \reals^+, Q_2 x \in \reals, Q_3 \delta \in \reals^+.\,
        \abs{x-r} \leq \alpha \QIMP \abs{f(x) - a} \leq \beta,
\end{equation}
where $Q_1,Q_2,Q_3$ are quantifiers $\forall, \exists$ and
$\alpha, \beta$ are each one $\delta$ or $\epsilon$.  Indicate
which\inhandout{ with a circle}:

\[\begin{array}{rcrr}
Q_1 &\text{ is } &  \forall & \exists\\
Q_2 &\text{ is } &  \forall & \exists\\
Q_3 &\text{ is } &  \forall & \exists\\
\alpha & =       &  \delta  & \epsilon\\
\beta  & =       &  \delta  & \epsilon
\end{array}\]

\begin{solution}
\begin{definition*}
\[
\lim_{x \to r} f(x) = a
\] 
means that
\[
\forall \epsilon \in \reals^+, x \in \reals \exists \delta \in \reals^+.\, \abs{x-r} \leq \delta \QIMP \abs{f(x) - a} \leq \epsilon.
\]
\end{definition*}
Therefore,
\[\begin{array}{rcl}
Q_1 &\text{ is } &  \forall\\
Q_2 &\text{ is } &  \forall\\
Q_3 &\text{ is } &  \exists\\
\alpha & =       &  \delta\\
\beta  & =       &  \epsilon
\end{array}\]
\end{solution}

\begin{staffnotes}
Related material for possible future use:

$f$ is \emph{continuous} at $r$ iff $\lim_{x \to r} = f(r)$.

A real number $b$ is a \term{lower bound} of a set $A$ of real
numbers, written ``$b \leq A$,'' when $b \leq a$ for all numbers $a
\in A$.  A real number $u$ is an \term{upper bound} of a set $A$ of
real numbers, written ``$A \leq u$,'' when $a \leq u$ for all numbers
$a \in A$.

\begin{axiom}
Every nonempty set of real numbers with a lower bound has a greatest lower bound (\glb).
\end{axiom}

\begin{corollary}
Every nonempty set of real numbers with an upper bound has a least
upper bound (\lub).
\end{corollary}

\begin{proof}
Two proofs:

\begin{enumerate}(i)
\item $\lub A =  \glb\set{u \such A \leq u}$.
\item $\lub A = - (\glb -A)$.
\end{enumerate}
\end{proof}

Let $A \subset \reals$ be a nonempty set with an upper bound, and let
\[
g \eqdef \glb\set{u \suchthat A \leq u}.
\]

Prove that $A \leq g$.

\TBA{insert soln}
\end{staffnotes}

\end{problem}

%%%%%%%%%%%%%%%%%%%%%%%%%%%%%%%%%%%%%%%%%%%%%%%%%%%%%%%%%%%%%%%%%%%%%
% Problem ends here
%%%%%%%%%%%%%%%%%%%%%%%%%%%%%%%%%%%%%%%%%%%%%%%%%%%%%%%%%%%%%%%%%%%%%

\endinput
