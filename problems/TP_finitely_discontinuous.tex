\documentclass[problem]{mcs}

\begin{pcomments}
  \pcomment{TP_finitely_discontinuous}
  \pcomment{ARM 3/17/15}
\end{pcomments}

\pkeywords{
  countable
  finite
  predicate
  bijection
  union
}

%%%%%%%%%%%%%%%%%%%%%%%%%%%%%%%%%%%%%%%%%%%%%%%%%%%%%%%%%%%%%%%%%%%%%
% Problem starts here
%%%%%%%%%%%%%%%%%%%%%%%%%%%%%%%%%%%%%%%%%%%%%%%%%%%%%%%%%%%%%%%%%%%%%

\begin{problem}
Let $A$ to be some infinite set and $B$ to be some countable set.  We
know \inbook{from Lemma~\bref{AUb}} that 
\[
A \bij (A \union \set{b_0})
\]
for any element $b_0 \in B$.  An easy induction implies that
\begin{equation}\label{AbijAubn}
A \bij (A \union \set{b_0,b_1,\dots,b_n})
\end{equation}
for any finite subset $\set{b_0,b_1,\dots,b_n} \subset B$.

Students sometimes think that~\eqref{AbijAubn} shows that $A \bij (A
\union B)$.  Now it's true that $A \bij (A \union B)$ for all such $A$
and $B$ for any countable set $B$
(Problem~\bref{PS_add_countable_elements}), but the facts above do not
prove it.

To explain this, let's say that a predicate $P(C)$ is \emph{finitely
  discontinuous} when $P(A \union F)$ is true for every \emph{finite}
subset $F \subset B$, but $P(A \union B)$ is false.  The hole in the
claim that~\eqref{AbijAubn} implies $A \bij (A \union B)$ is the
assumption (without proof) that the predicate
\[
P_0(C) \eqdef [A \bij C]
\]
is not finitely discontinuous.  This assumption about $P_0$ is
correct, but it's not completely obvious and takes some proving.

To illustrate this point, let $A$ be the nonnegative integers and $B$
be the nonnegative rational numbers, and remember that both $A$ and
$B$ are countably infinite.  Some of the predicates $P(C)$ below are
finitely \textbf{d}iscontinuous and some are \textbf{n}ot.  Briefly
explain which is which.

\begin{enumerate}
\item $C$ is finite.  
\item $C$ is countable.
\item $C$ is uncountable.
\item $C$ contains only finitely many non-integers.
\item $C$ contains the rational number 2/3.
\item There is a maximum non-integer in $C$.
\item There is an $\epsilon > 0$ such that any two elements of $C$ are $\epsilon$ apart.
\item $C$ is countable.
\item $C$ is uncountable.
\item $C$ has no infinite decreasing sequence $c_0 > c_1 > \cdots$.
\item Every nonempty subset of $C$ has a minimum element.
\item $C$ has a maximum element.
\item $C$ has a minimum element.
\end{enumerate}

\begin{solution}
\begin{enumerate}
\item $C$ is finite.       This is always false, so \hfill\textbf{n}.
\item $C$ is countable.    This is always true,  so \hfill\textbf{n}.
\item $C$ is uncountable.  This is always false, so \hfill\textbf{n}.
\item $C$ contains only finitely many non-integers. \hfill\textbf{d}.
\item $C$ contains the rational number 2/3.  Not true for $A \union F$
  unless $2/3 \in F$, and so \hfill\textbf{n}.
\item There is a maximum non-integer in $C$.        \hfill\textbf{d}.
\item There is an $\epsilon > 0$ such that any two elements of $C$ 
  are $\epsilon$ apart.\hfill\textbf{d}
\item $C$ has no infinite decreasing sequence $c_0 > c_1 > \cdots$.\hfill\textbf{d}
\item Every nonempty subset of $C$ has a minimum element.\hfill\textbf{d}
\item $C$ has a maximum element.  This is always false, so \hfill\textbf{n}.
\item $C$ has a minimum element.  This is always true,  so \hfill\textbf{n}.

\end{enumerate}

\end{solution}
\end{problem}

%%%%%%%%%%%%%%%%%%%%%%%%%%%%%%%%%%%%%%%%%%%%%%%%%%%%%%%%%%%%%%%%%%%%%
% Problem ends here
%%%%%%%%%%%%%%%%%%%%%%%%%%%%%%%%%%%%%%%%%%%%%%%%%%%%%%%%%%%%%%%%%%%%%

\endinput

