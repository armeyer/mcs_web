\documentclass[problem]{mcs}

\newcommand{\pairset}[2]{\text{pair}(#1,#2)}

\begin{pcomments}
  \pcomment{PS_equality_axioms}
  \pcomment{DRAFT: needs solns, explanation of F(x) versus F(y)}
  \pcomment{ARM 6/2/17}
\end{pcomments}

\pkeywords{
  equality
  logic
  sets
  set_theory
  predicate
  formula
}

\begin{problem}
By definition\inbook{~\bref{xsetequaly}}, the set-formula $x = y$ is an abbrevation for
\[
\forall z.\; (z \in x \QIFF z \in y).
\]
It is informative to verify that with this definition, equality of
sets has all the expected properties.  For example, the formula
\[
(x = y \QIFF y=x)  \qquad (\text{symmetry of $\eq$}),
\]
which asserts that equality has the property of
being\term{symmetric}\index{binary relation!symmetric}, is a valid
formula.  This follows because expanding the abbreviation for equality
we get the formula
\begin{equation}\label{sym=}
[\forall z.\; (z \in x \QIFF z \in y)] \QIFF [\forall z.\; (z \in y \QIFF z \in x)].
\end{equation}
Now it follows from the truth table for \QIFF\ that
\[
[P \QIFF Q] \QIFF [Q \QIFF P]
\]
is a valid formula for any formulas $P,Q$, so letting $P$ be ``$z \in
x$'' and $Q$ be ``$z \in y$,'' we conclude that~\eqref{sym=} is valid,
as claimed.

\bparts

\ppart Prove that equality of sets is a reflexive relation.
\begin{solution}
\TBA{proof of =reflex}
\end{solution}

\ppart Prove that equality of sets is a transitive relation.

\TBA{proof of =transitive}

\eparts

\bigskip

The most important property of equality is the \emph{substitution
  property}: if $x = y$, then it's OK to replace $x$ by $y$ in
formulas.  More precisely, if $F(x)$ is a formula of set theory
asserting some property of the set $x$, then
\begin{equation}\label{subst=}
[x = y \QAND F(x)] \QIMP F(y)
\end{equation}
is valid formula.

\bparts Prove that~\eqref{subst=} is valid for all set theory formulas
$F(x)$.

\hint Induction of the size of $F$.

\begin{solution}
\TBA{proof of =substitution}
\end{solution}

\eparts

\end{problem}

\endinput
