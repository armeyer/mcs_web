\documentclass[problem]{mcs}

\begin{pcomments}
  \pcomment{MQ_hypercube}
  \pcomment{earlier version of MQ_hypercube_nocount}
  \pcomment{needs revision or deletion}
\end{pcomments}

\pkeywords{
       hypercube
       tree
       spanning_tree
}

%%%%%%%%%%%%%%%%%%%%%%%%%%%%%%%%%%%%%%%%%%%%%%%%%%%%%%%%%%%%%%%%%%%%%
% Problem starts here
%%%%%%%%%%%%%%%%%%%%%%%%%%%%%%%%%%%%%%%%%%%%%%%%%%%%%%%%%%%%%%%%%%%%%

\begin{problem}
% 

The $n$-dimensional hypercube, $H_n$, is a graph whose vertices are
the binary strings of length $n$. Two vertices are adjacent if and
only if they differ in exactly 1 bit. For example, in $H_3$, vertices
\texttt{111} and \texttt{011} are adjacent because they differ only in
the first bit, while vertices \texttt{101} and \texttt{011} are not
adjacent because they differ at both the first and second bits.

\begin{problemparts}

\problempart
Draw $H_2$.
\examspace[2in]
\begin{solution}
\begin{figure}[h]
\graphic{MQ_H2}
\caption{$H_2$.}
\label{fig:H2}
\end{figure}
\end{solution}

\problempart
How many vertices are in $H_n$?  Show that your answer is
valid for $n=2$.
\examspace[2in]

\begin{solution}
There are $2^n$ distinct binary strings of length $n$.  So $H_n$ has
$2^n$ vertices.  In particular, $H_2$ has $2^2=4$ vertices:
\texttt{00}, \texttt{01}, \texttt{10}, and \texttt{11}.
\end{solution}

\problempart
There are $n2^{n-1}$ edges in $H_n$.  Show that this is valid for $n=2$.
\examspace[1.5in]

\begin{solution}
There are indeed $2(2^{2-1})=4$ edges in $H_2$:
$\edge{\tt{00}}{\tt{01}}$, $\edge{\tt{00}}{\tt{10}}$,
$\edge{\tt{01}}{\tt{11}}$, and $\edge{\tt{10}}{\tt{11}}$.
\end{solution}

\problempart Explain why it is impossible to find two spanning trees
of $H_3$ that have no edges in common.

\hint Use the count of vertices \& edges.

\examspace[3in]

\begin{solution}
$H_3$ has $2^3=8$ vertices, so any spanning tree must have $8-1=7$
  edges. But $H_3$ has only $3(2^{3-1})=12$ edges, so any two sets of
  7 edges must overlap.
\end{solution}

\end{problemparts}

\end{problem}

\endinput
