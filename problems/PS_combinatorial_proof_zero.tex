\documentclass[problem]{mcs}

\begin{pcomments}
  \pcomment{PS_combinatorial_proof_zero}
  \pcomment{from: new as of S10}
\end{pcomments}

\pkeywords{
 combinatorial_proof
 binomial_coefficient
}

%%%%%%%%%%%%%%%%%%%%%%%%%%%%%%%%%%%%%%%%%%%%%%%%%%%%%%%%%%%%%%%%%%%%%
% Problem starts here
%%%%%%%%%%%%%%%%%%%%%%%%%%%%%%%%%%%%%%%%%%%%%%%%%%%%%%%%%%%%%%%%%%%%%

\begin{problem}

The following two questions ask for \emph{combinatorial} proofs
of equalities.  A combinatorial proof works by demonstrating that
both sides of the equality are valid ways of counting the same
thing, not by using algebraic manipulation.

\bparts

\ppart
Find a combinatorial (\emph{not} algebraic) proof that
\[
\sum_{i=0}^n -1^i\binom{n}{i} = 0.
\]

%\hint Think about the binomial theorem.
\begin{solution}
%Both sides are the sum of the coefficients of $(x + y)^n$.

  We use the binomial theorem to expand $(1 - 1)^n$.  The left-hand
side of the equation is the sum of the terms, while the righthand side
uses the shortcut that $(0)^n = 0$.
\end{solution}

\ppart
Fina a combinatorial proof that
\[
\sum_{i=0}^n -1^{i}2^{n-i}\binom{n}{i} = 1
\]
\begin{solution}
Once again we use the binomial theorem, this time expanding $(2 - 1)^n$ to 
prove a less trivial identity.
\end{solution}

\eparts
\end{problem}

%%%%%%%%%%%%%%%%%%%%%%%%%%%%%%%%%%%%%%%%%%%%%%%%%%%%%%%%%%%%%%%%%%%%%
% Problem ends here
%%%%%%%%%%%%%%%%%%%%%%%%%%%%%%%%%%%%%%%%%%%%%%%%%%%%%%%%%%%%%%%%%%%%%

\endinput
