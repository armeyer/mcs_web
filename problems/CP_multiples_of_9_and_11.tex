\documentclass[problem]{mcs}

\begin{pcomments}
  \pcomment{from: S09.cp8t, S06.cp7m, S04.ps7}
%  \pcomment{}
%  \pcomment{}
\end{pcomments}

\pkeywords{
  series
  number_theory
  divides
  remainders
}

%%%%%%%%%%%%%%%%%%%%%%%%%%%%%%%%%%%%%%%%%%%%%%%%%%%%%%%%%%%%%%%%%%%%%
% Problem starts here
%%%%%%%%%%%%%%%%%%%%%%%%%%%%%%%%%%%%%%%%%%%%%%%%%%%%%%%%%%%%%%%%%%%%%

\begin{problem}
\bparts

\ppart Why is a number written in decimal evenly divisible by 9 if and
only if the sum of its digits is a multiple of 9?  \hint $10 \equiv 1
\pmod{9}$.

\begin{solution}
Since $10 \equiv 1 \pmod{9}$, so is
\begin{equation}\label{10k}
10^k \equiv 1^k \equiv 1 \pmod{9}.
\end{equation}
Now a number in decimal has the form:
\[
d_k \cdot 10^k + d_{k-1} \cdot 10^{k-1} + \ldots + d_1 \cdot 10 + d_0.
\]

From~\eqref{10k}, we have
\begin{eqnarray*}
d_k \cdot 10^k + d_{k-1} \cdot 10^{k-1} + \ldots + d_1 \cdot 10 + d_0
    & \equiv & d_k + d_{k-1} + \ldots + d_1 + d_0 \pmod{9} \\
\end{eqnarray*}

This shows something stronger than what we were asked to show, namely, it
shows that the remainder when the original number is divided by 9 is equal
to the remainder when the sum of the digits is divided by 9.  In
particular, if one is zero, then so is the other.
\end{solution}

%S04 ps7

\ppart Take a big number, such as 37273761261.  Sum the digits, where
every other one is negated:
\[
3 + (-7) + 2 + (-7) + 3 + (-7) + 6 + (-1) + 2 + (-6) + 1  =  -11
\]
Explain why the original number is a multiple of 11 if and only
if this sum is a multiple of 11.
%\hint $10 \equiv -1 \pmod{11}$.

\begin{solution}
A number in decimal has the form:
\[
d_k \cdot 10^k + d_{k-1} \cdot 10^{k-1} + \ldots + d_1 \cdot 10 + d_0
\]

Observing that $10 \equiv -1 \pmod{11}$, we know:
\begin{align*}
\lefteqn{d_k \cdot 10^k + d_{k-1} \cdot 10^{k-1} + \ldots + d_1 \cdot 10 + d_0} \\
    & \equiv d_k \cdot (-1)^k + d_{k-1} \cdot (-1)^{k-1} + \ldots \cdot + d_1 \cdot (-1)^1 + d_0 \cdot (-1)^0 \pmod{11} \\
& \equiv d_k - d_{k-1} + \ldots \cdot - d_1 + d_0  \pmod{11}
\end{align*}
assuming $k$ is even.  The case where $k$ is odd is the same with signs
reversed.  

The procedure given in the problem computes $\pm$ this alternating sum of
digits, and hence yields a number divisible by 11 ($\equiv 0 \pmod {11}$)
iff the original number was divisible by 11.
\end{solution}

\eparts

\end{problem}

%%%%%%%%%%%%%%%%%%%%%%%%%%%%%%%%%%%%%%%%%%%%%%%%%%%%%%%%%%%%%%%%%%%%%
% Problem ends here
%%%%%%%%%%%%%%%%%%%%%%%%%%%%%%%%%%%%%%%%%%%%%%%%%%%%%%%%%%%%%%%%%%%%%

\endinput
