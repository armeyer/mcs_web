\section*{Gamblers Ruin}

A gambler aims to gamble until he reaches a \emph{goal} of $T$ dollars
or until he runs out of money, in which case he is said to be
``ruined.''  He gambles by making a sequence of 1 dollar bets.  If he
wins an individual bet, his stake increases by one dollar.  If he
loses, his stake decreases by one dollar.  In each bet, he wins with
probability $p>0$ and loses with probability $q \eqdef 1-p >0$.  He is
an overall \emph{winner} if he reaches his goal and is an overall
\emph{loser} if he gets ruined.

In a \emph{fair} game, $p = q = 1/2$.  The gambler is more likely to win
if $p>1/2$ and less likely to win if $p<1/2$.

With $T$ and $p$ fixed, the gambler's probability of winning will depend
on how much money he starts with.  Let $w_n$ be the probability that he is
a winner when his initial stake in $n$ dollars.

\begin{problem}
\bparts

\ppart  What are $w_0$ and $w_T$?

\solution{$w_0 = 0$ and $w_T=1$.}

\ppart Note that $w_n$ satisfies a linear recurrence
\begin{equation}\label{recab}
w_{n+1} = aw_{n}+bw_{n-1}
\end{equation}
for some constants $a,b$ and $0 < n < T$.  Write simple expressions for
$a$ and $b$ in terms of $p$.

\solution{By Total Probability
\begin{align}
w_n & = \prcond{\text{win game}}{\text{win the first bet}}\pr{\text{win the first
   bet}} +\\
    & \quad \prcond{\text{win game}}{\text{lose the first bet}}\pr{\text{lose the first bet}}\notag\\
   & = pw_{n+1}+q\pr{w_{n-1}}, & \text{so}\notag\\
pw_{n+1} & = w_n - qw_{n-1}\notag\\
w_{n+1} & = \frac{w_n}{p} - \frac{qw_{n-1}}{p}.\label{wrec}
\end{align}
So
\[
a = \frac{1}{p}, \qquad b= - \frac{q}{p}.
\]
}

\ppart For $n>T$, let $w_n$ be defined by the recurrence~\eqref{recab},
and let $W(x) \eqdef \sum_{n=1}^\infty w_nx^n$ be the generating function
for the sequence $w_0,w_1,\dots$.  Verify that
\begin{equation}\label{gx}
W(x) = \frac{w_1 x}{(1-x)(1-(q/p)x)}.
\end{equation}

\solution{
\[\begin{array}{rclclclclc}
W(x)          & = & w_0 & + & w_1x   & + & w_2x^2       & + & w_3x^3       & + \cdots\\
xW(x)/p       & = &     &   & w_0x/p & + & w_1x^2/p     & + & w_2x^3/p     & + \cdots\\
(q/p)x^2W(x) & = &     &   &        &    & (q/p)w_0x^2  & + & (q/p)w_1x^3 & + \cdots
\end{array}\]
so
\begin{align}
W(x) - \paren{\frac{xW(x)}{p} - \frac{qx^2W(x)}{p}} & = w_0 + w_1x - w_0 x/p =
w_1x,\notag\\
W(x)\paren{ 1 - \frac{x}{p} + \frac{qx^2}{p}} & = w_1 x.\label{gfw}
\end{align}
But
\begin{equation}\label{fac}
1 - \frac{x}{p} + \frac{qx^2}{p} = (1-x)(1-(q/p)x)
\end{equation}
Combining~\eqref{fac} and~\eqref{gfw} yields~\eqref{gx}.%
}

\ppart Use partial fractions to show that 
\begin{equation}\label{wnd}
w_n = w_1 \left( \frac{(q/p)^n - 1}{(q/p)-1}\right) \ .
\end{equation}

\solution{
  Express $W(x)$ as the partial fraction 
  \[ W(x) = \frac{A}{1-x} + \frac{B}{1-(q/p)x} \ . \]
  Given~\eqref{gx}, we want $c,d$ such that
  Substituting $W(x)$ according to~\eqref{gx}, we get
\begin{align*}
    \frac{w_1 x}{(1-x)(1-(q/p)x)} &= \frac{A}{1-x} + \frac{B}{1-(q/p)x}
    \\
    w_1 x &= A(1-(q/p)x) + B(1-x)
\end{align*}
  Let $x=1$, we get $A = w_1/(1-(q/p))$, and letting $x=p/q$, we get
  $B = -w_1/(1-(q/p))$, so
  \[
W(x) = \frac{w_1}{1-q/p}\left(\frac{1}{1-x} -
    \frac{1}{1-(q/p)x}\right) \ ,
\]
  which implies
 \[
w_n = \frac{w_1}{(q/p)-1}\left(\left(\frac{q}{p}\right)^n -
    1\right) .
\]
}

\ppart  Show that in an unfair game,
\[
w_n=\frac{(q/p)^n-1}{(q/p)^T-1}.
\]
(Hint: solve for $w_1$.)

\solution{
We can solve for $w_1$, by letting $n=T$ in~\eqref{wnd}:
\[
1=w_T = \frac{w_1}{q/p-1} \paren{\paren{\frac{q}{p}}^T -1}
\]
so
\[
w_1 = \frac{\paren{q/p -1}}{\paren{q/p}^T - 1}.
\]
Combining this with~\eqref{wnd} yields
\[
w_n = \frac{\paren{\paren{q/p}^n -1}}{\paren{q/p}^T -1}.
\]
}

\ppart Verify that if $0 < a < b$, then
\[
\frac{a}{b} < \frac{a+1}{b+1}.
\]
Conclude that if $p < 1/2$, then
\[
w_n < \paren{\frac{p}{q}}^{T-n}.
\]

\solution{
\[
\frac{a}{b} = \frac{a(1+1/b)}{b(1+1/b)} = \frac{a+a/b}{b+1} < \frac{a+1}{b+1}.
\]
So from the previous part, we have
\[
w_n=\frac{(q/p)^n-1}{(q/p)^T-1} < \frac{(q/p)^n}{(q/p)^T} =
\paren{\frac{q}{p}}^{n-T} = \paren{\frac{p}{q}}^{T-n}.
\]
}

\eparts
\end{problem}
