\documentclass[problem]{mcs}

\begin{pcomments}
  \pcomment{FP_structural_induction_arithmetic_composition}
  \pcomment{ARM 12/13/11}
\end{pcomments}

\pkeywords{
  structural_induction
  polynomial
  arithmetic
  derivative}

%%%%%%%%%%%%%%%%%%%%%%%%%%%%%%%%%%%%%%%%%%%%%%%%%%%%%%%%%%%%%%%%%%%%%
% Problem starts here
%%%%%%%%%%%%%%%%%%%%%%%%%%%%%%%%%%%%%%%%%%%%%%%%%%%%%%%%%%%%%%%%%%%%%

\def\ArF{\text{ArF}}

\begin{problem}

\begin{definition*}
  The set, $\ArF$, of \term{arithmetic functions} of one argument is
  defined recursively as follows:

\begin{itemize}
\item \textbf{Base cases:}

\begin{enumerate}

\item The identity function, $\ide_\reals$ on real numbers is an $\ArF$.

\item\label{AFc} Every real-valued constant function is in $\ArF$.

\end{enumerate}

\item \textbf{Constructor cases:} If $e,f \in \ArF$, then
\begin{enumerate}
\setcounter{enumi}{2}

\item\label{AF+} $e + f \in \ArF$,

\item\label{AF*} $e \cdot f \in \ArF$.

\end{enumerate}
\end{itemize}
\end{definition*}

Prove by structural induction that $\ArF$ is closed
under composition.  That is, using the induction hypothesis,
\[
P(f) \eqdef \forall g \in \ArF. g \compose f \in \ArF],
\]
prove that $P(f)$ holds for all $fh \in \ArF$.  Make sure to indicate
explicitly each of the base cases and the constructor cases of the
structural induction.

\examspace[4in]
\begin{solution}

\begin{proof}
  \inductioncase{base cases}: We must show $P(\ide_\reals)$ and
  $P(\text{constant-function})$.  But this follows immediately from
  the fact that $g \compose \ide_\reals = g$ and the composition of
  $g$ with a the constant function is a constant function.
  
  \inductioncase{constructor cases}: Given $e, f \in \ArF$, we may assume by
  structural induction that $P(e)$ and $P(f)$ both hold, and must prove
  $P(h)$ where

\emph{case} $h= e + f$:  In this case,
\[
g \compose h = (g \compose e) + (g \compose f)
\]
and since $(g \compose e), (g \compose f) \in \ArF$ by hypothesis, so is their
sum by the constructor rule~\eqref{AF+}.  This proves $P(h)$ in this
case.

\emph{case} $h= f \cdot g$:
$P(h)$ follows exactly as in the previous case with ``$+$'' replaced
by ``$\cdot$.''

\end{proof}

\end{solution}

\end{problem}

%%%%%%%%%%%%%%%%%%%%%%%%%%%%%%%%%%%%%%%%%%%%%%%%%%%%%%%%%%%%%%%%%%%%%
% Problem ends here
%%%%%%%%%%%%%%%%%%%%%%%%%%%%%%%%%%%%%%%%%%%%%%%%%%%%%%%%%%%%%%%%%%%%%


\endinput
