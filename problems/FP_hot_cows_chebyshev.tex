\documentclass[problem]{mcs}

\begin{pcomments}
  \pcomment{FP_hot_cows_chebyshev}
  \pcomment{variant of FP_hot_cows_markov}
  \pcomment{Original: ARM May 19, 2013}
\end{pcomments}

\pkeywords{
  average
  Markov_bound
  deviation
  sample_space
  outcome
  probability_space
}

%%%%%%%%%%%%%%%%%%%%%%%%%%%%%%%%%%%%%%%%%%%%%%%%%%%%%%%%%%%%%%%%%%%%%
% Problem starts here
%%%%%%%%%%%%%%%%%%%%%%%%%%%%%%%%%%%%%%%%%%%%%%%%%%%%%%%%%%%%%%%%%%%%%


\begin{problem}

First, let us define a notion called \emph{collection-variance}. Let $A = \set{a_1, a_2, \ldots, a_n}$ be a
set of real numbers, and $\mu$ be their average. Then, the
collection-variance is defined as the quantity:
\[
\text{ColVar}(A) = \frac{\sum_{i=1}^n (a_i - \mu)^2}{n} .
\]


A herd of cows is stricken by an outbreak of \emph{wacky cow disease},
which tends to severely affect body temperatures. 
A cow will die if its body temperature goes above $110$ degrees or
below $90$ degrees. The average of the body temperatures of the
infected herd is measured to be $100$ degrees, while the collection-variance of the body
temperatures is $20$. 

\bparts

\ppart\label{2/10cows} Show that at most 20\% of the cows will die due
to this disease.

\examspace[2in]

\begin{solution}
Let $A$ be the set of body temperatures of the herd. 
Let $T$ be the temperature of a cow chosen \emph{uniformly at
  random}. Then it is not hard to prove formally that 
\[
\expect{T} = 100, \variance{T} = 20 .
\]
We need to upper bound $\prob{(T \geq 110) \union (T \leq 90)}$. We do this as follows:
\begin{align*}
\prob{(T \geq 110) \union (T \leq 90)}& \leq \prob{T - 100 \geq 10} + \prob{T -100 \leq -10 } &
\text{sum rule} \\
& = \prob{ | T - 100 | \leq 10} \\
& \leq \frac{\variance{T}}{10^2} &\text{(Chebyshev's bound)} \\
&= \frac{20}{100} .
\end{align*}

Since the cows were chosen randomly, this implies that at most 20\% of the
herd can have a temperature that is either greater than 110 degrees or
less than 90 degrees. 
\end{solution}

\ppart Tthe conclusion of Part~\eqref{2/10cows} is a purely
arithmetic facts about averages, not about probabilities.  But you
proved the claim of Part~\eqref{2/10cows} by bounding the deviation of
a random variable.  

Justify this approach by explaining how to define a random variable, $T$, for the temperature
of a cow.  Carefully specify the probability space on which $T$ is
defined: what are the outcomes? what are their probabilities?  Explain
the precise connection between properties of $T$, average herd
temperature, and fractions of the herd with various temperatures.

\begin{solution}
The sample space for $T$ is the set of cows in the herd, that is, each
cow is an outcome.  The probabilities are defined to be
\emph{uniform}---the probability of any cow, $c$, is $1/n$ where $n$
is the size of the herd---and $T(c)$ is the temperature of cow $c$.
Since the probabilities are uniform, we can formally prove that

\begin{itemize}

\item the average temperature of the herd equals $\expect{T}$.

\item the collection variance of the herd equals $\variance{T}$. 

\item the fraction of cows with temperatures $\geq t$ is the
  probability that $T \geq t$.

\end{itemize}

So the fact that $\prob{T \geq 110} \leq 0.2$ implies that at most 20\%
of the herd could have survived.

\end{solution}
\eparts

\end{problem}

%%%%%%%%%%%%%%%%%%%%%%%%%%%%%%%%%%%%%%%%%%%%%%%%%%%%%%%%%%%%%%%%%%%%%
% Problem ends here
%%%%%%%%%%%%%%%%%%%%%%%%%%%%%%%%%%%%%%%%%%%%%%%%%%%%%%%%%%%%%%%%%%%%%

\endinput
