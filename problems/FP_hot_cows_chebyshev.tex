\documentclass[problem]{mcs}

\begin{pcomments}
  \pcomment{FP_hot_cows_chebyshev}
  \pcomment{variant of FP_hot_cows_markov}
  \pcomment{Original: ARM May 19, 2013}
\end{pcomments}

\pkeywords{
  average
  Markov_bound
  deviation
  sample_space
  outcome
  probability_space
}

%%%%%%%%%%%%%%%%%%%%%%%%%%%%%%%%%%%%%%%%%%%%%%%%%%%%%%%%%%%%%%%%%%%%%
% Problem starts here
%%%%%%%%%%%%%%%%%%%%%%%%%%%%%%%%%%%%%%%%%%%%%%%%%%%%%%%%%%%%%%%%%%%%%


\begin{problem}

The \emph{collection-variance} $\text{CVar}(A)$ of a set $A =
\set{a_1, a_2, \dots, a_n}$ of real numbers is defined as
\[
\text{CVar}(A) \eqdef \frac{\sum_{i=1}^n (a_i - \mu)^2}{n},
\]
where $\mu$ is the average value of the numbers in $A$.

There is a herd of cows whose average of body temperatures turns out
to be $100$ degrees, while the collection-variance of the body
temperatures is $20$.  The herd is stricken by an outbreak of
\emph{wacky cow disease} which will kill cows whose body temperature
differs from the average by $10$ degrees or more.

\bparts

\ppart\label{2/10cows} Apply the Chebyshev bound to the temperature
$T$ of a random cow to show that at most 20\% of the cows will die due
to this disease outbreak.

\examspace[2in]

\begin{solution}
Let $A$ be the set of body temperatures of the herd.  Let $T$ be the
temperature of a random cow.  Then
\begin{align*}
\prob{ \abs{T - 100} \geq 10} & \leq \frac{\variance{T}}{10^2} &\text{(Chebyshev's bound)} \\
&= \frac{20}{100} .
\end{align*}

So at most 20\% of the herd can have a temperature that differs from
the average by as much as 10 degrees.
\end{solution}

\ppart The conclusion of part~\eqref{2/10cows} is anarithmetic bound
on the fraction of the herd with some property.  But you proved the
claim of part~\eqref{2/10cows} by bounding the deviation of a random
variable.

Justify this approach by explaining how to define a random variable,
$T$, for the temperature of a cow.  Carefully specify the probability
space on which $T$ is defined: what are the outcomes? what are their
probabilities?  Explain the precise connection between properties of
$T$ and the properties of the actual herd that justify this
application of the Chebyshev bound.

\begin{solution}
The sample space for $T$ is the set of cows in the herd, that is, each
cow is an outcome.  The probabilities are defined to be
\emph{uniform}---the probability of any cow, $c$, is $1/n$ where $n$
is the size of the herd---and $T(c)$ is the temperature of cow $c$.
Since the probabilities are uniform, it is easy to prove that

\begin{itemize}

\item the average temperature of the herd equals $\expect{T}$.

\item the collection variance of the herd equals $\variance{T}$. 

\item the fraction of cows with temperatures $\geq t$ is the
  probability that $T \geq t$.

\end{itemize}

So the fact that $\prob{\abs{T - 100} \geq 10} \leq 0.2$ implies that
at most 20\% of the herd will die from the outbreak.

\end{solution}
\eparts

\end{problem}

%%%%%%%%%%%%%%%%%%%%%%%%%%%%%%%%%%%%%%%%%%%%%%%%%%%%%%%%%%%%%%%%%%%%%
% Problem ends here
%%%%%%%%%%%%%%%%%%%%%%%%%%%%%%%%%%%%%%%%%%%%%%%%%%%%%%%%%%%%%%%%%%%%%

\endinput
