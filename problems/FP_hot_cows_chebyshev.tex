\documentclass[problem]{mcs}

\begin{pcomments}
  \pcomment{FP_hot_cows_chebyshev}
  \pcomment{variant of FP_hot_cows_markov}
  \pcomment{Original: ARM May 19, 2013}
\end{pcomments}

\pkeywords{
  average
  Markov_bound
  deviation
  sample_space
  outcome
  probability_space
}

%%%%%%%%%%%%%%%%%%%%%%%%%%%%%%%%%%%%%%%%%%%%%%%%%%%%%%%%%%%%%%%%%%%%%
% Problem starts here
%%%%%%%%%%%%%%%%%%%%%%%%%%%%%%%%%%%%%%%%%%%%%%%%%%%%%%%%%%%%%%%%%%%%%


\begin{problem}
If $A$ is a finite set of real numbers, then the
\emph{collection-variance} $\text{CVar}(A)$ of $A$ is defined as $A$'s
average square deviation from its mean:
\[
\text{CVar}(A) \eqdef \frac{\sum_{a \in A} (a - \mu)^2}{\card{A}},
\]
where $\mu$ is the average value of the numbers in $A$.

There is a herd of cows whose average body temperature turns out to be
$100$ degrees, while the collection-variance of all the body
temperatures is $20$.  Our thermometer produces such sensitive readings
that no two cows have exactly the same body temperature.

The herd is stricken by an outbreak of \emph{wacky cow disease} which
will eventually kill any cow whose body temperature differs from the average by
$10$ degrees or more.

\bparts

\ppart\label{2/10cows} Apply the Chebyshev bound to the temperature
$T$ of a random cow to show that at most 20\% of the cows will be
killed by this disease outbreak.

\examspace[2in]

\begin{solution}
Let $A$ be the set of body temperatures of the herd.  Let $T$ be the
temperature of a random cow.  Then
\begin{align*}
\prob{ \abs{T - 100} \geq 10} & \leq \frac{\variance{T}}{10^2} &\text{(Chebyshev's bound)} \\
&= \frac{20}{100} .
\end{align*}

So at most 20\% of the herd can have a temperature that differs from
the average by as much as 10 degrees.
\end{solution}

\ppart The conclusion of part~\eqref{2/10cows} is a bound on the
fraction of the herd with some property that was derived by bounding
the deviation of a random variable.  Justify this approach by
explaining how to define a random variable, $T$, for the temperature
of a cow.  Carefully specify the probability space on which $T$ is
defined: what are the outcomes? what are their probabilities?  Explain
the precise connection between properties of $T$ and the properties of
the actual herd that justify the application of the Chebyshev bound to
reach the conclusion about the herd.

\begin{solution}
The sample space for $T$ is the set of cows in the herd, that is, each
cow is an outcome.  (Alternatively, the outcomes could be chosen to be
the temperatures, since our sensitive thermometer ensures there is a
bijection between cows and temperatures.)  The probabilities are
defined to be \emph{uniform}---the probability of any outcome, $c$, is
$1/n$ where $n$ is the size of the herd---and $T(c)$ is the
temperature of cow $c$.  Since the probabilities are uniform, it is
easy to prove that:
\begin{itemize}

\item The average temperature of the herd equals $\expect{T}$.

\item The collection variance of the herd equals $\variance{T}$. 

\item The fraction of cows whose temperatures have any given property,
  $P$, is the probability that $T \text{ has property } P$.

\end{itemize}

So the fact that $\prob{\abs{T - 100} \geq 10} \leq 0.2$ implies that
at most 20\% of the herd will die from the outbreak.

\end{solution}
\eparts

\end{problem}

%%%%%%%%%%%%%%%%%%%%%%%%%%%%%%%%%%%%%%%%%%%%%%%%%%%%%%%%%%%%%%%%%%%%%
% Problem ends here
%%%%%%%%%%%%%%%%%%%%%%%%%%%%%%%%%%%%%%%%%%%%%%%%%%%%%%%%%%%%%%%%%%%%%

\endinput
