\documentclass[problem]{mcs}

\begin{pcomments}
    \pcomment{Converted from ../problems/00Convert/probs/practice13/prob1-spiders.scm
              by scmtotex and drewe
              on Fri 15 Jul 2011 12:09:19 PM EDT}
\end{pcomments}

\begin{problem}

%% type: multi-part
%% title: Spiders and Flies

The spider is expecting dinner guests and wants to catch 500 flies.  100 flies pass by her web every hour.  60 of these flies are
 quite small and are caught with probability 1/6 each.  40 of the flies
 are big and are caught with probability 3/4 each.  Assume all fly
 interceptions are mutually independent. She has 10 hours before the dinner party is set to begin. 

You work in a bobblehead doll factory, in which you have two broken assembly machines.  One of them makes 60 dolls per hour, but because it is broken, each of them has a probability of only $1/6$ of being assembled correctly.  The other machine is slightly less broken (each of its dolls has a probability of $3/4$ of being assembled correctly) but is slower, producing only 40 dolls per hour.  Your production quota is 500 dolls per day.% Using this information, the
% methods from lecture can show
% that the poor spider has only about 1 chance in 100,000 of catching 500
% flies within 10 hours.
\bparts

\ppart\label{ppart:first}
%% type: multiple-choice
%% title: 
Briefly comment on the utility of each of the following methods for estimating the chances that your factory will produce 500 dolls in a 10-hour workday.


\begin{enumerate}
\item Estimation of the binomial density $F_{n,p}$
\item Markov's bound
\item Chebyshev's bound
\item Chernoff's bound
\end{enumerate}
\begin{solution} \begin{enumerate}
These happen to be ordered from least to most relevant:
\item Estimation of the binomial density $F_{n,p}$ is not at all useful. The sum of two binomial distributions with different values of $p$ is not binomial.  We're considering a random variable that is the sum of
$1000$ Bernoulli variables, $600$ with $p = 1/6$ and $400$ with $p = 3/4$.
\item Markov's bound is also not useful.  It merely tells us that the probability is at most 0.8, which is an absurd overestimate.
\item Yes, we can use Chebyshev's bound here: it depends on the standard deviation, which we have (since we know this distribution is the sum of two binomial distributions).
\item Yes, Chernoff's bound is useful too -- we'll calculate it in the next part.
\end{enumerate}
\end{solution}
\ppart
%% type: multiple-choice
%% title: 
Which one produces the tightest bound, and what is it? 

\begin{solution}

$e^{-(5/4 ln(5/4) - 5/4 + 1)400}\approx9.4\times10^{-6}$


This is the Chernoff bound.  The expected number of flies caught every hour is $(1/6)60+(3/4)40 = 40$, so the expected number of flies in 10 hours is
400.  So, 
\begin{equation*}
  \prob{X\geq500} \leq Prob{X\geq5/4*400} \leq e^{-(5/4 ln(5/4) - 5/4 + 1)400} \approx 9.4\times10^{-6}
\end{equation*}
\end{solution}

%% </font>

\eparts


\end{problem}

\endinput
