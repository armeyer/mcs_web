\documentclass[problem]{mcs}

\begin{pcomments}
  \pcomment{PS_easy_impossible_count}
  \pcomment{ARM 1/12/13}
  
\end{pcomments}

\pkeywords{
  generating functions
  symbolic_calculation
  convolution
}

%%%%%%%%%%%%%%%%%%%%%%%%%%%%%%%%%%%%%%%%%%%%%%%%%%%%%%%%%%%%%%%%%%%%%
% Problem starts here
%%%%%%%%%%%%%%%%%%%%%%%%%%%%%%%%%%%%%%%%%%%%%%%%%%%%%%%%%%%%%%%%%%%%%

\begin{problem}
The answer derived by generating functions for the ``absurd'' counting
problem in Section~\bref{sec:impossible_counting} is not impossibly
complicated at all.  Describe a direct simple counting argument to
derive this answer without using generating functions.

\begin{solution}
The following argument was offered by Rebecca Freund when she was a
student in the MIT Mathematics for Computer Science class in Fall '09.
\begin{quote}
Note that the fruits can be divided into two groups, the
apples-and-pears and the bananas-and-oranges.  Once you know how many
are apples-and-pears, there's only one way to distribute them: use a
pear if the number is odd, otherwise don't; then make the rest apples.
Similarly, once you've decided on the number $k$ of
bananas-and-oranges, you have to throw in $k - \rem{k}{5}$ bananas and
make the rest oranges.  So the number of apples-and-pears exactly
determines the arrangement.  You can have $0--n$ apples-and-pears, so
there are $n+1$ possibilities.
\end{quote}
\end{solution}

\end{problem}
%%%%%%%%%%%%%%%%%%%%%%%%%%%%%%%%%%%%%%%%%%%%%%%%%%%%%%%%%%%%%%%%%%%%%
% Problem ends here
%%%%%%%%%%%%%%%%%%%%%%%%%%%%%%%%%%%%%%%%%%%%%%%%%%%%%%%%%%%%%%%%%%%%%

\endinput
