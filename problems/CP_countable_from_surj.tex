\documentclass[problem]{mcs}

\begin{pcomments}
  \pcomment{CP_countable_from_surj}
  \pcomment{subsumes CP_countable_surjection}
  \pcomment{ARM 3/2/13, revised 5/15/14}
\end{pcomments}

\pkeywords{
  bijection
  surjection
  countable
  finite
}

%%%%%%%%%%%%%%%%%%%%%%%%%%%%%%%%%%%%%%%%%%%%%%%%%%%%%%%%%%%%%%%%%%%%%
% Problem starts here
%%%%%%%%%%%%%%%%%%%%%%%%%%%%%%%%%%%%%%%%%%%%%%%%%%%%%%%%%%%%%%%%%%%%%
\begin{problem}

\bparts

\ppart Prove that if a nonempty set, $C$, is countable, then there is
a \emph{total} surjective function $f: \naturals \to C$.

\begin{solution}
\begin{proof}
There are two cases:

\inductioncase{Case}: ($C$ is finite.)  Suppose $C =
\set{c_0,c_1,\dots, c_n}$.  Then we can define a total surjective
function $f:\naturals \to C$ by the rule
\[
f(k) = \begin{cases}  
       c_k & \text{if } 0 \leq k \leq n,\\
       c_n & \text{otherwise}.
\end{cases}
\]

\inductioncase{Case}: ($C$ is infinite.)  If $C$ is countably
infinite, then the bijection from $\naturals$ to $C$ is a total
surjective function.

\ppart Conversely, suppose that $\naturals \surj D$, that is, there is
a not necessarily total surjective function $f:\naturals D$.  Prove
that $D$ is countable.

\begin{solution}
If $D$ is countable by definition if it is finite, so we may assume
that $D$ is infinite.  Now to prove that $D$ is countable, we need to find
a bijection from $\naturals$ to $D$.  We do this simply by filtering
out any duplicates and ignoring any undefined items in the list $f(0),
f(1),\dots$.  Namely, define $g: \naturals \to C$ by the rule
\[
g(n) \eqdef \text {the $n$th distinct element in the sequence } f(0),
f(1), f(2),\dots.
\]
Then $g$ is the desired bijection.
\end{proof}

\end{solution}

\eparts

\end{problem}

\endinput
