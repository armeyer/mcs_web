\documentclass[problem]{mcs}

\begin{pcomments}
  \pcomment{CP_countable_from_surj}
  \pcomment{variant of CP_countable_surjection}
  \pcomment{ARM 3/2/13}
\end{pcomments}

\pkeywords{
  bijection
  surjection
  countable
  finite
}

%%%%%%%%%%%%%%%%%%%%%%%%%%%%%%%%%%%%%%%%%%%%%%%%%%%%%%%%%%%%%%%%%%%%%
% Problem starts here
%%%%%%%%%%%%%%%%%%%%%%%%%%%%%%%%%%%%%%%%%%%%%%%%%%%%%%%%%%%%%%%%%%%%%
\begin{problem}
Prove that a nonempty set, $C$, is countable iff there is a total
surjective function $f: \naturals \to C$.
\begin{solution}

\begin{proof}
There are two cases:

\inductioncase{Case}: ($C$ is finite.)  $C$ is countable by
definition.  Suppose $C=\set{c_0,c_1,\dots, c_n}$.  Then we can define
a total surjective function $f:\naturals \to C$ by the rule
\[
f(k) = \begin{cases}  
       c_k & \text{if } 0 \leq k \leq n,\\
       c_n & \text{otherwise}.
\end{cases}
\]

\inductioncase{Case}: ($C$ is infinite.)  If $C$ is countably
infinite, then the bijection from $\naturals$ to $C$ is a total
surjective surjection.

Conversely, suppose there is a total surjective function $f:\naturals
C$.  To show that $C$ is countably infinite, we need to find a
bijection from $\naturals$ to $C$.  So we simply filter out the
duplicates from the list $f(0), f(1),\dots$.  Namely, defining
\[
g(n) \eqdef \text {the $n$th disinct element in the sequence } f(0),
f(1), f(2),\dots,
\]
gives the desired bijection $g: \naturals \to C.$

\end{proof}


\end{solution}

\end{problem}

\endinput
