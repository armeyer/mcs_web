\documentclass[problem]{mcs}

\begin{pcomments}
  \pcomment{FP_token_state_machine_conflict}
  \pcomment{proposed for midterm II, Fall15}
  \pcomment{author: Zoran Dzunic, edited ARM}
\end{pcomments}

\pkeywords{
  state_machines
  derived_variable
  induction
  congruences
}

%%%%%%%%%%%%%%%%%%%%%%%%%%%%%%%%%%%%%%%%%%%%%%%%%%%%%%%%%%%%%%%%%%%%%
% Problem starts here
%%%%%%%%%%%%%%%%%%%%%%%%%%%%%%%%%%%%%%%%%%%%%%%%%%%%%%%%%%%%%%%%%%%%%

\begin{problem}
  ``Token replacing'' is a single player game using a set of tokens,
  each colored black or white.  Except for color, the tokens are
  indistinguishable.  %The game starts with one black token.
  In each move, a player can replace one black token with two
  white tokens, or replace one white token with two black tokens.

% Zoran: I somehow prefer to introduce the start state here, as it may
% be hinting too much if introduced in part (b). Also, I came up
% with another part of the problem for which it may be better to
% have the start state fixed as part of rules.

  We can model this game as a state machine whose states are pairs
  $(n_b,n_w)$ where $n_b \geq 0$ equals the number of black tokens,
  and $n_w \geq 0$ equals the number of white tokens.
  
 Let us assume that the game starts with one black token, that is,
 state $(1,0)$.

\bparts

\ppart Define the predicate $T(n_b,n_w)$ by the rule:
\[
T(n_b,n_w) \eqdef\  [n_w - n_b  \equiv 2 \pmod{3}].
\]
Prove that $T$ is true for all reachable states.

\examspace[1.5in]

\begin{solution}
$T$ is true for the start state since $n_w - n_b = 0-1 \equiv
2 \pmod{3}$.

Next we show that $T$ is a preserved invariant.

So suppose $T(n_b, n_w)$ holds.  The only two possible transitions are
into a state $(n'_b,n'_w)$ equal to $(n_b - 1, n_w + 2)$ or to $(n_b +
2, n_w - 1)$.  But since
\[
n'_w - n'_b = n_w - n_b \pm 3 \equiv n_w - n_b \pmod{3} \equiv 2 \pmod{3},
\]
we have $T(n'_b,n'_w)$, which proves that $T$ is invariant.

Since $T$ is a preserved invariant that is true for the start state,
it is true for all reachable states by the invariant principle.
\end{solution}

\ppart
Does the same hold for the predicate $n_b - n_w \equiv 2 \pmod{3}$? Explain.

\examspace[0.5in]

\begin{solution}
No. This predicate is not true for the start state.
\end{solution}

\ppart
\begin{claim*}
If $T(n_b, n_w)$, then state $(n_b, n_w)$ is reachable from the start
state $(1, 0)$.
\end{claim*}

Note that this claim is different from the fact that $T$ is a
preserved invariant.

The proof will be by induction in $n$ using induction hypothesis $P(n) \eqdef$
\[
 \forall (n_b, n_w) \in \naturals^2.\,
 [(n_b + n_w = n) \QAND  T(n_b,n_w)]
     \QIMPLIES (n_b,n_w) \text{ is reachable}.
\]

The base cases will be when $n \leq 2$.  Assuming that the base cases
have been verified, complete the \inductioncase{Induction Step}:

\examspace[3in]

\begin{solution}

\begin{proof}

\inductioncase{Inductive step}: We need to show that if $T(n_b,n_w)$
holds, then $(n_b, n_w)$ is reachable.

By the base cases, we may assume that.  Suppose $n_b + n_w = n+1$ and $n_w - n_b
\equiv 2 \pmod{3}$.  We want to show that 

By the observation above, $(n_b, n_w)$ is reachable if both $n_b < 2$
and $n_w < 2$, so we need only consider the case when either $n_b
\geq 2$ or $n_w \geq 2$.

In the case that $n_b \geq 2$, we have $n_b-2 \geq 0$,
$(n_b-2)+(n_w+1) = n$, and
\[
(n_b-2)-(n_w+1) = n_b-n_w -3 \equiv n_b-n_w \equiv 2 \pmod{3}.
\]
So by induction hypothesis $P(n)$, we have that $(n_b-2,n_w+1)$ is
reachable.  But $(n_b-2,n_w+1)$ transitions in one step to
$(n_b,n_w)$, so $(n_b,n_w)$ is reachable.

The same argument applies in the case that $n_w \geq 2$.

We conclude that in any case $(n_b,n_w)$ is reachable, which completes
the induction step.

\end{proof}

\end{solution}

Now verify the base cases by proving $P(n)$ for $n \leq 2$.

\examspace[3in]

\iffalse
\ppart
For each predicate above, how would you modify the game such that
it is true for all reachable states?

%\examspace[2.0in]

\begin{solution}
\begin{enumerate}
\item Change the rules such that each token is replace with 4 tokens of opposite color instead.
\item No change.
\item Start with any state for which this predicate is satisfied, e.g., $(0, 1)$ or $(2, 0)$.
\item No change.
\item Start with any state for which $n_b + n_w > 5$, e.g., $(3, 3)$.
\item Change the rules such that each token is replace with 1 token of opposite color instead.
\end{enumerate}
\end{solution}
\fi

\eparts
\end{problem}

%%%%%%%%%%%%%%%%%%%%%%%%%%%%%%%%%%%%%%%%%%%%%%%%%%%%%%%%%%%%%%%%%%%%%
% Problem ends here
%%%%%%%%%%%%%%%%%%%%%%%%%%%%%%%%%%%%%%%%%%%%%%%%%%%%%%%%%%%%%%%%%%%%%

\endinput
