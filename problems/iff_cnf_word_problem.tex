\documentclass[problem]{mcs}

\begin{pcomments}
  \pcomment{Inspired by textbook problem 3.2.1}
  \pcomment{Edited by dganelin Fall 2017}
\end{pcomments}

\pkeywords{
  cnf
  propositional algebra
  iff
}

%%%%%%%%%%%%%%%%%%%%%%%%%%%%%%%%%%%%%%%%%%%%%%%%%%%%%%%%%%%%%%%%%%%%%
% Problem starts here
%%%%%%%%%%%%%%%%%%%%%%%%%%%%%%%%%%%%%%%%%%%%%%%%%%%%%%%%%%%%%%%%%%%%%
\begin{problem}

\begin{enumerate}
\item{Express the formula 

$A$ IFF $B$

in conjunctive normal form (AND of OR's), showing your work. (Hint: start by breaking the IFF into implications.)}


\item{
Five of the 6.042 TA's - Akhil, Daniela, Maggie, Preksha, and Sibo - are deciding whether to go to the Propositional Party together this weekend.

Define the following propositions:

A: Akhil will go to the Propositional Party.

D: Daniela will go to the Propositional Party.

M: Maggie will go to the Propositional Party.

P: Preksha will go to the Propositional Party.

S: Sibo will go to the Propositional Party.

Using these propositions, translate the following sentence into a propositional formula in conjunctive normal form:

\textit{Either all five TA's will go to the Propositional Party, or none of them will.}

(Hint: this can be done with a five-clause formula.)
}
\end{enumerate}




\begin{solution}

\begin{enumerate}
\item{
$A$ IFF $B$

($A$ IMPLIES $B$) AND ($B$ IMPLIES $A$)

(NOT $A$ OR $B$) AND (NOT $B$ OR $A$)
}

\item{
The idea is the same as in Part A, forming a cycle:

(NOT $A$ OR $D$) AND (NOT $D$ OR $M$) AND (NOT $M$ OR $P$) AND (NOT $P$ OR $S$) AND (NOT $S$ OR $A$)  
}
\end{enumerate}
\end{solution}

\end{problem}

%%%%%%%%%%%%%%%%%%%%%%%%%%%%%%%%%%%%%%%%%%%%%%%%%%%%%%%%%%%%%%%%%%%%%
% Problem ends here
%%%%%%%%%%%%%%%%%%%%%%%%%%%%%%%%%%%%%%%%%%%%%%%%%%%%%%%%%%%%%%%%%%%%%
\endinput