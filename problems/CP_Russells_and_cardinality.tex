\documentclass[problem]{mcs}

\begin{pcomments}
  \pcomment{CP_Russells_and_cardinality}
  \pcomment{from: S09.ps1, S07.ps1, F05.ps1, commented out in S09}
  \pcomment{revised ARM 3/13/18}
  \pcomment{doubtful problem: done in the notes---better...}
\end{pcomments}
        
\pkeywords{
  set_theory
  Russells_paradox
  bijections
}

%%%%%%%%%%%%%%%%%%%%%%%%%%%%%%%%%%%%%%%%%%%%%%%%%%%%%%%%%%%%%%%%%%%%%
% Problem starts here
%%%%%%%%%%%%%%%%%%%%%%%%%%%%%%%%%%%%%%%%%%%%%%%%%%%%%%%%%%%%%%%%%%%%%

\begin{problem}
Cantor's Theorem implies there is no bijection $f:\power(A) \to A$.  Without appeal
to Cantor's Theorem, prove this by contradiction.  In particular, assume there
was such an $f$, and define
\[
W_f \eqdef \set{a \in A \suchthat a \notin f^{-1}(a)}.
 \]
Show that
\begin{equation}\tag{*}
f(W_f) \in f^{-1}(f(W_f)\ \ \QIFF\ \ f(W_f) \notin f^{-1}(f(W_f)).
\end{equation}

\begin{solution}  
By the definition of $W_f$, we know that for all $a \in A$:
\begin{equation}\label{xw}
a \in W_f\ \ \QIFF\ \ a \notin f^{-1}(a) .
\end{equation}
But $W_f \in \power(A)$ by definition, so $W_f = f^{-1}(f(W_f))$.  So we have
from~\eqref{xw}, that
\begin{equation}\label{xf}
a \in f^{-1}(f(W_f)\ \ \QIFF\ \ a \notin f^{-1}(a)
\end{equation}
for all $a \in A$.  Substituting $f(W_f)$ for $a$ in~\eqref{xf} yields the
contradiction~(*), which proves that there cannot be such an bijection $f$.
\end{solution}

\end{problem}

%%%%%%%%%%%%%%%%%%%%%%%%%%%%%%%%%%%%%%%%%%%%%%%%%%%%%%%%%%%%%%%%%%%%%
% Problem ends here
%%%%%%%%%%%%%%%%%%%%%%%%%%%%%%%%%%%%%%%%%%%%%%%%%%%%%%%%%%%%%%%%%%%%%

\endinput
