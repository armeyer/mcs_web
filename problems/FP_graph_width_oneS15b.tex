\documentclass[problem]{mcs}

\begin{pcomments}
  \pcomment{FP_graph_width_oneS15b}
  \pcomment{part (b) of FP_graph_width_one}
\end{pcomments}

\pkeywords{
 graph theory
 colorable
 width
}

\begin{problem}
A simple graph, $G$, is said to have \emph{width} $1$ iff there is a
way to list all its vertices so that each vertex is adjacent to at
most one vertex that appears earlier in the list.  All the graphs
mentioned below are assumed to be finite.

Prove that every finite tree has width one.

\begin{solution}
By induction on the number of vertices, $n$.  The induction hypothesis
is
\begin{center}
$Q(n) \eqdef$ all $n$-vertex trees $T$ have width one.
\end{center}

\inductioncase{Base case}: ($n=1$).  Trivial.

\inductioncase{Induction step}.  Assume that that $Q(n)$ is true for
some $n \geq 1$ and let $T$ be an $(n+1)$-vertex tree.  We need only
show that $T$ has width one.

By Theorem~\bref{th:treeprops}, every tree with at least two vertices has a leaf.  Let $v$
be a leaf of $T$.  Then $T-v$ has width one by Induction Hypothesis,
so its vertices can be listed with each vertex adjacent to at most one
vertex earlier in the list.  Since $v$ has degree one, we can add it
to the end of the list of vertices for $T-v$ to obtain the required
list for $T$.  Hence $T$ has width one, as claimed.

This proves $Q(n+1)$ and completes the induction step.

\end{solution}

\end{problem}
\endinput
