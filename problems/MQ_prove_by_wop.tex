\documentclass[problem]{mcs}

\begin{pcomments}
  \pcomment{MQ_prove_by_wop}
  \pcomment{from: S10}
\end{pcomments}

\pkeywords{
  wop proof
}

%%%%%%%%%%%%%%%%%%%%%%%%%%%%%%%%%%%%%%%%%%%%%%%%%%%%%%%%%%%%%%%%%%%%%
% Problem starts here
%%%%%%%%%%%%%%%%%%%%%%%%%%%%%%%%%%%%%%%%%%%%%%%%%%%%%%%%%%%%%%%%%%%%%

\begin{problem}

Prove by the Well Ordering Principle that for all nonnegative integers, $n$: 
\[
\sum_{i=0}^{n} i^3 = \left(\frac{n(n+1)}{2}\right)^2.
\]

\begin{solution}

The proof is by contradiction.

Suppose to the contrary that this failed for some $n
\geq 0$.  Then by the WOP, there is a \emph{smallest} nonnegative integer,
$m$, such this formula does not hold when $n = m$.

But it clearly holds when $n = 0$, which means that $m \geq
1$.  So $m-1$ is nonnegative, and since it is smaller than $m$,
the formula must be true for $n = m-1$.  That is,
\begin{equation}\label{sum-to-m-1}
\sum_{i=0}^{m-1} i^3 = \left(\frac{(m-1)m}{2}\right)^2.
\end{equation}
Now add $m^3$ to both sides of equation~\eqref{sum-to-m-1}.
Then the left hand side equals
\[
\sum_{i=0}^{m} i^3
\]
and the right hand side equals
\[
\left(\frac{(m-1)m}{2}\right)^2 + m^3 
\]
Now a little algebra %(given below) 
shows that the right hand side equals
\[
\left(\frac{m(m+1)}{2}\right)^2.
\]
That is,
\[
\sum_{i=0}^{m} i^3 = \left(\frac{m(m+1)}{2}\right)^2,
\]
contradicting the fact that our formula does not hold for
$m$.

\end{solution}

\end{problem}


%%%%%%%%%%%%%%%%%%%%%%%%%%%%%%%%%%%%%%%%%%%%%%%%%%%%%%%%%%%%%%%%%%%%%
% Problem ends here
%%%%%%%%%%%%%%%%%%%%%%%%%%%%%%%%%%%%%%%%%%%%%%%%%%%%%%%%%%%%%%%%%%%%%
\endinput
