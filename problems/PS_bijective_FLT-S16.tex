\documentclass[problem]{mcs}

\begin{pcomments}
  \pcomment{PS_bijective_FLT-S16}
  \pcomment{ARM revise PS_bijective_FLT, 4/15/16}
  \pcomment{S16.ps7, S13.ps10, F11.MQ}
\end{pcomments}

\pkeywords{
  counting
  Fermat
  congruence
  equivalence
}

%%%%%%%%%%%%%%%%%%%%%%%%%%%%%%%%%%%%%%%%%%%%%%%%%%%%%%%%%%%%%%%%%%%%%
% Problem starts here
%%%%%%%%%%%%%%%%%%%%%%%%%%%%%%%%%%%%%%%%%%%%%%%%%%%%%%%%%%%%%%%%%%%%%

\begin{problem}
Fermat's Little Theorem~\bref{fermat_little}\footnote{This Theorem is usually stated as
\[
 a^{p-1}\equiv 1\pmod{p},
\]
for all primes $p$ and integers $a$ not divisible by $p$.  This
follows immediately from~\eqref{apeqvap} by canceling $a$.}
asserts that
\begin{equation}\label{apeqvap}}
 a^p\equiv a\pmod{p}
\end{equation}
for all primes $p$ and nonnegative integers $a$.  This is immediate
for $a=0,1$ so we assume that $a \geq 2$.

This problem offers a proof of the Theorem by counting strings over a
fixed alphabet with $a$ characters.


\bparts

\ppart How many length-$k$ strings $a$ are there? \hfill\examrule.
\begin{solution}
$a^k$.
\end{solution}

How many of these are strings use more than one character? \hfill\examrule.
\begin{solution}
$a^k-a$.

For each of the $a$ characters, there is exactly one length-$k$
string using only that character.
\end{solution}

\begin{solution}
There are $a^p$ different colored sequences of beads by the Product
Rule.  Of these sequences, there is one monochromatic sequence for
each of the $a$ colors, so there are $a^p-a$ sequences of at least two
colors.
\end{solution}
\eparts

Let $z$ be a length $k$ string.  The \emph{length-$n$ rotation} of $z$
is the string $yx$, where $z = xy$ and the length, $lnth{x}$, of $x$
is $\remainder(n,k)$.

\bparts

\ppart\label{rotateagain} Verify that if $u$ is a length-$n$ rotation of $z$, and $v$ is
a length-$m$ rotation of $u$, then $v$ is a length-$n+m$ rotation of
$z$.
\begin{solution}
Suppose $z = wxy$ where $\lngth{w} = n, \lngth{x} = m$.  Then $u =
(xy)w$ and so $v = (yw)x= y(wx)$ which is the length-$n+m$ rotation of
$z$.  This argument applies to $n,m,n+m > \lnth{z}$ by taking
remainders on division by $\lnth{z}$. 
\begin{staffnotes}
\TBA{Explain about remainders more clearly?}
\end{staffnotes}
\end{solution}

\ppart Let $\approx$ be the ``is a rotation of'' relation on strings.
That is,
\[
v \approx z \quad \QIFF\quad v\ \text{is a length-$n$ rotation of}\ z
\]
for some $n \in \nngint$.  Prove that $\approx$ is an equivalence
relation.
\begin{solution}
\begin{proof}
Reflexivity follows since everything is a length 0 rotation of itself.
Symmetry follows because $v$ is a length $m$ rotation of $z$ iff $z$
is a length $\lnth{z} - \remainder(m,\lnth{z})$ rotation of $v$.
Transitivity follows from part~\eqref{rotateagain}.
\end{proof}
\end{solution}

\ppart\label{xyyxm*} Prove that if $xy=yx$ then $x$ and $y$ each consist of
repetitions of some string $m$.  That is, if $xy=yx$, then $x,y \in
m^*$ for some string $m$.

\hint Let $m$ be the shortest positive length string such that $my = ym$.

\begin{solution}
\TBA{fill in}.
\end{solution}

\ppart\label{exactlyp} Conclude that if $p$ is prime and $z$ is a length-$p$ string
containing at least two different characters, then $z$ is equivalent
under $\approx$ to exactly $p$ other strings.

\begin{solution}
\TBA{By part\eqref{xyyxm*},....}
\end{solution}

\ppart Conclude from part~\eqrevf{exactlyp} that $p \divides a^p -a$, which proves
Fermat's Little Theorem.

\begin{solution}
\TBA{}
\end{solution}
\eparts

\end{problem}

%%%%%%%%%%%%%%%%%%%%%%%%%%%%%%%%%%%%%%%%%%%%%%%%%%%%%%%%%%%%%%%%%%%%%
% Problem ends here
%%%%%%%%%%%%%%%%%%%%%%%%%%%%%%%%%%%%%%%%%%%%%%%%%%%%%%%%%%%%%%%%%%%%%

\endinput
