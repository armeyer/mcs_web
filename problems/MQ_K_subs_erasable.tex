\documentclass[problem]{mcs}

\begin{pcomments}
\pcomment{MQ_K_subs_erasable}
\pcomment{by NDJ, based on CP_erasable_strings and PS_M_equals_RM, 2/27/11}
\end{pcomments}

\pkeywords{
induction
matching_brackets
recursive_data
strings
structural_induction
}

%%%%%%%%%%%%%%%%%%%%%%%%%%%%%%%%%%%%%%%%%%%%%%%%%%%%%%%%%%%%%%%%%%%%%
% Problem starts here
%%%%%%%%%%%%%%%%%%%%%%%%%%%%%%%%%%%%%%%%%%%%%%%%%%%%%%%%%%%%%%%%%%%%%

\begin{problem}
Let $p$ be the string $\lefbrk\rhtbrk$.  A string of
brackets is said to be \term{erasable} iff it can be reduced to
the empty string by repeatedly erasing occurrences of $p$.  For
example, here's how to erase the string
$\lefbrk\lefbrk\lefbrk\rhtbrk\rhtbrk\lefbrk\rhtbrk\rhtbrk\lefbrk\rhtbrk$:
\[
\lefbrk\lefbrk\lefbrk\rhtbrk\rhtbrk\lefbrk\rhtbrk\rhtbrk\lefbrk\rhtbrk
\rightarrow \lefbrk\lefbrk\rhtbrk\rhtbrk
\rightarrow \lefbrk\rhtbrk
\rightarrow \emptystring.
\]
On the other hand the string $\lefbrk\rhtbrk\rhtbrk\lefbrk\lefbrk\lefbrk\lefbrk\lefbrk\rhtbrk\rhtbrk$ is not erasable because
when we try to erase, we get stuck: %at $\rhtbrk\lefbrk\lefbrk\lefbrk$:
\[
\lefbrk\rhtbrk\rhtbrk\lefbrk\lefbrk\lefbrk\lefbrk\lefbrk\rhtbrk\rhtbrk
\rightarrow \rhtbrk\lefbrk\lefbrk\lefbrk\lefbrk\rhtbrk
\rightarrow \rhtbrk\lefbrk\lefbrk\lefbrk
\not\rightarrow
\]

Let $\ES$ be the set of erasable strings of brackets.  

The set of strings, $K$, is recursively defined as follows:
\begin{itemize}

\item \textbf{Base case:} $\emptystring \in K$,

\item \textbf{Constructor cases:} if $s,t \in K$, then
the string $s\, \lefbrk t\, \rhtbrk s$ is also in $K$.
\end{itemize}

Prove by structural induction that $K \subseteq \ES$.

\begin{solution}

\begin{proof}
We prove by structural induction on the definition of $\K$ that
  the predicate
	\[
	P(x) \eqdef x \in \ES
	\]
	is true for all $x \in \K$.

	\textbf{Base case} ($x = \emptystring$): The empty string is erasable by
	definition of $\ES$---it can be reduced to itself by erasing the
	substring $\lefbrk$ 0 times.

	\textbf{Constructor case 1} ($x = s\, \lefbrk t\, \rhtbrk s$ for $s,t \in \K$):  By
	structural induction hypothesis, we may assume that $s \in \ES$.  So to
	erase $x$, erase $s$ to be left with the substring
	$\lefbrk\rhtbrk$, and one more erasure leads to the empty string.

	This completes the proof by structural induction, so we conclude that
	\[
	\forall x.\ x\in \K\ \QIMPLIES\ x \in \ES
	\]
	which by definition means that $\K \subseteq \ES$.

	\end{proof}

\end{solution}

\end{problem}

%%%%%%%%%%%%%%%%%%%%%%%%%%%%%%%%%%%%%%%%%%%%%%%%%%%%%%%%%%%%%%%%%%%%%
% Problem ends here
%%%%%%%%%%%%%%%%%%%%%%%%%%%%%%%%%%%%%%%%%%%%%%%%%%%%%%%%%%%%%%%%%%%%%

\endinput
