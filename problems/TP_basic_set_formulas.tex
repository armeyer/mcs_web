\documentclass[problem]{mcs}

\begin{pcomments}
    \pcomment{TP_basic_set_formulas}
    \pcomment{ARM 2/17/13}
\end{pcomments}

\pkeywords{
  logic
  sets
  set_theory
  predicate
  formula
  subset
  power_set
  union
}

\begin{problem}
\inhandout{A \emph{formula of \idx{set theory}}\footnote{Technically this is
  called a \term{first-order predicate formula} of set theory} is a
predicate formula that only uses the predicate ``$x \in y$.''  The
domain of discourse is the collection of sets, and ``$x \in y$'' is
interpreted to mean that $x$ and $y$ are variables that range over
sets, and $x$ is one of the elements in $y$.

For example, since $x$ and $y$ are the same set iff they have the same
members, here's how we can express equality of $x$ and $y$ with a
formula of set theory:
\begin{equation}\label{x=xAz}
(x = y) \eqdef\ \forall z.\, (z \in x\ \QIFF\ z \in y).
\end{equation}
}

Express each of the following assertions about sets by a formula of
set theory.\inbook{\footnote{See Section~\bref{ZFC_sec}.}}

\bparts

\ppart $x = \emptyset$.

\begin{solution}
$\forall z.\, \QNOT(z\in x)$.
\end{solution}

\ppart $x = \set{y,z}$.

\begin{solution}
$\forall w.\, w \in x \QIFF (w=y \QOR\ w = z)$.
\end{solution}

\ppart $x \subseteq y$.  ($x$ is a subset of $y$ that might equal $y$.)

\begin{solution}
$\forall z.\, z \in x\ \QIMPLIES\ z \in y$.
\end{solution}

\ppart $ x = y \union z$.

\begin{solution}
$\forall w.\, w \in x \QIFF (x \in y\ \QOR\ w \in z)$.
\end{solution}

\ppart $x = y - z$.

\begin{solution}
$\forall w.\, w \in x\ \QIFF\  (w \in y\ \QAND\ \QNOT(w \in z))$.
\end{solution}

\eparts

Now we can explain how to express ``$x$ is a \idx{proper
  subset} of $y$''---$x \subset y$---as a set theory formula using
things we already know how to express.  Namely, $x$ is a \idx{proper
  subset} of $y$ iff $x$ is a subset of $y$ that is not equal to $y$:
\[
(x \subset y)\ \eqdef\ (x \subseteq y\ \QAND \QNOT(x = y)).
\]

\iffalse
Another way to express $x \subset y$ is
\[
\exists z.\, (y = x \union z\ \QAND \QNOT(z = \emptyset)) 
\]
\fi

From here on, feel free to use any previously expressed property in
describing formulas for the following:

\bparts

\ppart $x = \power(y)$.

\begin{solution}
$\forall z.\, z \in x \QIFF z \subseteq y$.
\end{solution}

\ppart $x = \lgunion_{z \in y} z$.

This means that $y$ is supposed to be a collection of sets, and $x$ is
the union of all them.  A more concise notation for ``$\lgunion_{z \in
  y} z$' is simply ``$\lgunion y$.''

\begin{solution}
$\forall w.\, w \in x \QIFF \exists z.\, (z \in y\ \QAND\ w \in z)$.
\end{solution}

\eparts

\end{problem}

\endinput
