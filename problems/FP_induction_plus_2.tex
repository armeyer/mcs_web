\documentclass[problem]{mcs}

\begin{pcomments}
  \pcomment{FP_induction_plus_2}
  \pcomment{from: F05.final}
  \pcomment{by adamc, 2012/01/13}
\end{pcomments}

%%%%%%%%%%%%%%%%%%%%%%%%%%%%%%%%%%%%%%%%%%%%%%%%%%%%%%%%%%%%%%%%%%%%%
% Problem starts here
%%%%%%%%%%%%%%%%%%%%%%%%%%%%%%%%%%%%%%%%%%%%%%%%%%%%%%%%%%%%%%%%%%%%%

\begin{problem}

Suppose $S(n)$ is a predicate on natural numbers, $n$, and suppose
\begin{equation}\tag{*}
\forall k \in \naturals \; S(k) \implies S(k+2).
\end{equation}
If~(*) holds, some of the assertions below must \emph{always}
(\textbf{A}) hold, some \emph{can} (\textbf{C}) hold but not always,
and some can \emph{never} (\textbf{N}) hold.  Indicate which case
applies for each of the assertions \inbook{and briefly explain why}\inhandout{by \textbf{circling} the correct letter}.

\begin{problemparts}

\iffalse

\problempart[3]  \hspace{.2in} {\bf A C N \hspace{.2in}} 
$\forall n \ge 0\; S(n)$ 

\solution[\vspace{0.25in}]{\textbf{C}.  The assertion means that $S$ is
always true.  So $S(k+2)$ is always true, and therefore $S(k) \implies
S(k+2)$ is always true.  So this case is possible.  But~\ref{Sk2} also
holds when $S$ is always false, so the asertion does not always hold when
\ref{Sk2} does.}

\problempart[3] \hspace{.2in} {\bf A\quad N\quad C \hspace{.2in}} $\neg{S(0)}
\land \forall n \ge 1\; S(n)$ 

\solution[\vspace{0.25in}]{\textbf{C}.  This time $S$ is false at 0, but
true everywhere else.  So $S(k) \implies S(k+2)$ still always holds
because $S(k+2)$ is still always true.  So this assertion can hold, but
not always, since~(\ref{Sk2}) can hold when $S(0)$ is true.}

\problempart[3]  \hspace{.2in} {\bf A\quad N\quad C \hspace{.2in}} 
$\forall n \ge 0\; \neg{S(n)}$

\solution[\vspace{0.25in}]{\textbf{C.}  Now $S$ is always false.  So $S(k)
\implies S(k+2)$ is always true because $S(k)$ is false.  So this case is
possible, but again does not always hold.}  \fi


\problempart[3]  \inhandout{\hspace{.2in} {\bf A\quad N\quad C \hspace{.2in}}}
$(\forall n \le 100\; S(n)) \land (\forall n > 100\; \neg{S(n)})$

\begin{solution}
  \textbf{N}.  In this case, $S$ is true for $n$
  up to 100 and false from 101 on.  So $S(99)$ is true, but $S(101)$ is
  false.  That means that $S(k) \not \implies S(k+2)$ for $k = 99$.  This
  case is impossible.
\end{solution}

\iffalse
\problempart[3] \hspace{.2in} {\bf A\quad N\quad C \hspace{.2in}} $(\forall n
\le 100\; \neg{S(n)}) \land (\forall n > 100\; S(n))$

\solution[\vspace{0.25in}]{\textbf{C}.  In this case, $S$ is false for $n$
up to 100 and true from 101 on.  So $S(k) \implies S(k+2)$ for $k \le 100$
because $S(k)$ is false, and $S(k) \implies S(k+2)$ for $k \ge 99$ because
$S(k+2)$ is true.  So this case is possible, but again does not always
hold}

\problempart[2]  \hspace{.2in} {\bf A\quad N\quad C \hspace{.2in}} 
$S(0) \implies \forall n\ S(n+2)$

\solution[\vspace{0.25in}]{\textbf{C}.  If $S(n)$ is always true this
assertion holds.  So this case is possible.  If $S(n)$ is true only for
even $n$~(\ref{Sk2}) still holds, but $S(1+2)$ is false.  So this case
does not always hold.}  \fi

\problempart[3]  \inhandout{\hspace{.2in} {\bf A\quad N\quad C \hspace{.2in}}}
$S(1) \implies \forall n\ S(2n+1)$

\begin{solution}
  \textbf{A}.  This assertion says that if $S(1)$
  holds, then $S(n)$ holds for all odd $n$.  This case is always true.
\end{solution}

\problempart[3]  \inhandout{\hspace{.2in} {\bf A\quad N\quad C \hspace{.2in}}}
$[\exists n\, S(2n)] \implies \forall n \ S(2n+2)$

\begin{solution}
  \textbf{C}.  If $S(n)$ is always true, this
  assertion holds.  So this case is possible.  If $S(n)$ is true only for even $n$ greater than
4,~(\ref{Sk2}) holds, but this assertion is false.  So this case does not
always hold.
\end{solution}

\problempart[3]  \inhandout{\hspace{.2in} {\bf A\quad N\quad C \hspace{.2in}}}
$\exists n\, \exists m > n\, [S(2n) \land \neg{S(2m)}]$

\begin{solution}
  \textbf{N}.  This assertion says that $S$ holds for some even
number ,$2n$, but not for some other larger even number, $2m$.  However,
if $S(2n)$ holds, we can apply~(\ref{Sk2}) $n-m$ times to conclude $S(2m)$
also holds.  This case is impossible.
\end{solution}

\problempart[3]  \inhandout{\hspace{.2in} {\bf A\quad N\quad C \hspace{.2in}}}
$[\exists n\, S(n)] \implies \forall n\, \exists m> n\, S(m)$

\begin{solution}
  \textbf{A}.  This assertion says that if $S$
holds for some $n$, then for every number, there is a larger number, $m$,
for which $S$ also holds.  Since~(\ref{Sk2}) implies that if there is one
$n$ for which $S(n)$ holds, there are an infinite, increasing chain of
$k$'s for which $S(k)$ holds, this case is always true.
\end{solution}

\iffalse
\problempart[2]  \hspace{.2in} {\bf A\quad N\quad C \hspace{.2in}} 
$\neg{S(0)} \implies \forall n \ \neg{S(2n)}$
\fi

\end{problemparts}

\end{problem}

%%%%%%%%%%%%%%%%%%%%%%%%%%%%%%%%%%%%%%%%%%%%%%%%%%%%%%%%%%%%%%%%%%%%%
% Problem ends here
%%%%%%%%%%%%%%%%%%%%%%%%%%%%%%%%%%%%%%%%%%%%%%%%%%%%%%%%%%%%%%%%%%%%%

\endinput
