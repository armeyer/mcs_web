\documentclass[problem]{mcs}

\begin{pcomments}
  \pcomment{MQ_tree_diameter}
  \pcomment{by ARM 3/16/13}
\end{pcomments}

\pkeywords{
  tree
  connected
  diameter
}

%%%%%%%%%%%%%%%%%%%%%%%%%%%%%%%%%%%%%%%%%%%%%%%%%%%%%%%%%%%%%%%%%%%%%
% Problem starts here
%%%%%%%%%%%%%%%%%%%%%%%%%%%%%%%%%%%%%%%%%%%%%%%%%%%%%%%%%%%%%%%%%%%%%

\begin{problem}
The \term{diameter} of a connected graph is the largest distance
between any two vertices.

\bparts

\ppart What is the largest possible diameter in any connected graph
with $n$ vertices?  Describe a graph with this maximum diameter.

\examspace[3in]

\begin{solution}
The longest path in an $n$-vertex graph contains all the vertices and
so is of length $n-1$.  The line graph $L_n$
\inbook{(Section~\bref{sec:common_graphs})} has diameter $n-1$.
\end{solution}

\ppart What is the smallest possible diameter of an $n$-vertex tree?
Describe an $n$-vertex tree with this minimum diameter.

\begin{solution}
A tree with at least three vertices must have diameter at least two,
since there cannot be an edge between every pair of vertices.  The
$n$-vertex \idx{star graph} \inbook{see Figure~\bref{fig:5T}}
  \inhandout{illustrated in Figure~\ref{fig:star7}} has diameter 2.

\begin{figure}

\graphic{star-graph}

\caption{A 7-node star graph.}

\label{fig:star7}

%\label{fig:5T}

\end{figure}

\end{solution}

\eparts

\end{problem}

\endinput
