\documentclass[problem]{mcs}

\begin{pcomments}
\pcomment{from: S09.final; S08 final}
\pcomment{adapted by ARM 12/8/09}
\end{pcomments}

\pkeywords{
  structural_induction
  polynomial
  arithmetic}

%%%%%%%%%%%%%%%%%%%%%%%%%%%%%%%%%%%%%%%%%%%%%%%%%%%%%%%%%%%%%%%%%%%%%
% Problem starts here
%%%%%%%%%%%%%%%%%%%%%%%%%%%%%%%%%%%%%%%%%%%%%%%%%%%%%%%%%%%%%%%%%%%%%

\begin{problem}

\fbox{\
\begin{minipage}[t]{6.5in}
\vspace{.1in}

\begin{definition*}\label{A}
The set, $\aexp$, of \emph{Arithmetic expressions} in the variable, $x$,
is defined recursively as follows:

\begin{itemize}
\item \textbf{Base cases:}

\begin{enumerate}

\item The variable, $x$, is in $\aexp$.

\item The arabic numeral, $\mtt{k}$, for any nonnegative integer, $k$, is
  in $\aexp$.

\end{enumerate}

\item \textbf{Constructor cases:} If $e,f \in \aexp$, then
\begin{enumerate}
\setcounter{enumi}{2}

\item $(e \sumsym f) \in \aexp$.  The expression $(e \sumsym f)$ is called a
  \term{sum}.  The \aexp's $e$ and $f$ are called the \term{components} of
  the sum; they're also called the \term{summands}.

\item $(e \prodsym f) \in \aexp$.  The expression $(e \prodsym f)$ is called a
  \term{product}.  The \aexp's $e$ and $f$ are called the
  \term{components} of the product; they're also called the
  \term{multiplier} and \term{multiplicand}.

\item $\minussym(e) \in \aexp$.  The expression $\minussym(e)$ is called a
  \term{negative}.
\end{enumerate}
\end{itemize}
\end{definition*}

\end{minipage}}

\examspace[0.1in]

You are asked to prove by structural induction that $\aexp$ is closed
under taking derivatives.  That is, using the induction hypothesis,
\[
P(h) \eqdef [h^{\prime} \in \aexp],
\]
where $h^{\prime} \eqdef d\,h(x)/dx$, prove that $P(h)$ holds for all
functions, $h \in \aexp$.

\bparts

\ppart[2] Prove the base cases of the structural induction.

\examspace[1.5in]
\begin{solution}

\begin{proof}
We must show $P(\ide(x))$ and $P(\text{constant-function})$.  But
$\ide^{\prime}$ is the constant function 1, and the derivative of a
constant function is the constant function 0, and these are rational
functions of $x$ by definition.

This proves that the induction hypothesis holds in the Base cases.
\end{proof}
\end{solution}

\ppart[7] Prove the constructor cases of the structural induction.

\examspace[3.5in]

\begin{solution}

\begin{proof}
  Assuming $f,g$ are rational functions of $x$ for which $P(f)$ and $P(g)$
  both hold, we must prove $P(h)$ where

\textbf{Case $h= f + g$}:  In this case,
\[
h^{\prime} = f^{\prime} + g^{\prime},
\]
and since $f^{\prime}$ and $g^{\prime}$ are rational functions of $x$ by
hypothesis, so is their sum by the constructor rules, which proves $P(h)$.

\textbf{Case $h= f \cdot g$}:

The Product Rule of derivatives states that:
\begin{equation}\label{fgderiv}
h^{\prime} =  f^{\prime} \cdot g + f \cdot g^{\prime},
\end{equation}
and since $f, f^{\prime}, g, g^{\prime}$ are rational functions of $x$ by
hypothesis, so is the right hand side of~\eqref{fgderiv} by the
constructor rules, which proves $P(h)$.

\textbf{Case $h= \dfrac{1}{f}$}:

The Chain Rule gives: 
\begin{equation}\label{1/fderiv}
h^{\prime} = \frac{-1}{f^2} \cdot f^{\prime},
\end{equation}
and since $f$ and $f^{\prime}$ are rational by hypothesis, so is the right
hand side of~\eqref{1/fderiv} by the constructor rules, which proves
$P(h)$.

We have shown that the induction hypothesis holds in all Constructor cases.
This completes the proof by structural induction.
\end{proof}

\end{solution}

\eparts
\end{problem}

\endinput
