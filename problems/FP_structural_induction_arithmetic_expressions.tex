\documentclass[problem]{mcs}

\begin{pcomments}
\pcomment{from: S09.final; S08 final}
\pcomment{adapted by ARM 12/8/09}
\end{pcomments}

\pkeywords{
  structural_induction
  polynomial
  arithmetic}

%%%%%%%%%%%%%%%%%%%%%%%%%%%%%%%%%%%%%%%%%%%%%%%%%%%%%%%%%%%%%%%%%%%%%
% Problem starts here
%%%%%%%%%%%%%%%%%%%%%%%%%%%%%%%%%%%%%%%%%%%%%%%%%%%%%%%%%%%%%%%%%%%%%

\def\AF{\AF}

\begin{problem}

\fbox{\
\begin{minipage}[t]{6.5in}
\vspace{.1in}

\begin{definition*}\label{A}
The set, $\AF$, of \term{arithmetic functions} of one argument
is defined recursively as follows:

\begin{itemize}
\item \textbf{Base cases:}

\begin{enumerate}

\item The identity function, $\id_\reals$ on real numbers is an $\AF$.

\item\label{AFc} Every real-valued constant function is in $\AF$.

\end{enumerate}

\item \textbf{Constructor cases:} If $e,f \in \AF$, then
\begin{enumerate}
\setcounter{enumi}{2}

\item\label{AF+} $e + f \in \AF$,

\item\label{AF*} $e \cdot f \in \AF$.

\end{enumerate}
\end{itemize}
\end{definition*}

\end{minipage}}

Prove by structural induction that $\AF$ is closed
under taking derivatives.  That is, using the induction hypothesis,
\[
P(h) \eqdef [h^{\prime} \in \AF],
\]
where $h^{\prime} \eqdef d\,h(x)/dx$, prove that $P(h)$ holds for all
functions, $h \in \AF$.  Make sure to indicate explicitly each of the
base cases and the constructor cases of the structural induction.

\examspace[4in]
\begin{solution}

\begin{proof}
  \textbf{base cases}: We must show $P(\id_\reals)$ and
  $P(\text{constant-function})$.  But $(\id_\reals)^{\prime}$ is the
  constant function 1, and the derivative of a constant function is
  the constant function 0, and these are in $\AF$ by~\eqref{AFc} in
  the definition of $\AF$.
  
  \textbf{constructor cases}: Given $f, g \in \AF$, we may assume by
  structural induction that $P(f)$ and $P(g)$ both hold, and must prove
  $P(h)$ where

\textbf{Case $h= f + g$}:  In this case,
\[
h^{\prime} = f^{\prime} + g^{\prime},
\]
and since $f^{\prime},g^{\prime} \in \AF$ by hypothesis, so is their
sum by the constructor rule~\eqref{AF+}.  This proves $P(h)$ in this
case.

\textbf{Case $h= f \cdot g$}:

The Product Rule of derivatives states that:
\begin{equation}\label{fgderiv}
h^{\prime} =  f^{\prime} \cdot g + f \cdot g^{\prime},
\end{equation}
and since $f, f^{\prime}, g, g^{\prime} \in \AF$ by hypothesis, so is
the right hand side of~\eqref{fgderiv} by the constructor
rules~\eqref{AF+} and~\eqref{AF*}.  This proves $P(h)$ in this case.

Thus the induction hypothesis holds in all Constructor cases, which
completes the proof by structural induction.  Hence, $P(h}$ holds for
all $h \in \AF$.
\end{proof}

\end{solution}

\end{problem}

\endinput
