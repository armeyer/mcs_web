\documentclass[problem]{mcs}

\begin{pcomments}
  \pcomment{CP_size_of_sample_vs_population}
  \pcomment{from: S06.cp12f}
\end{pcomments}

\pkeywords{
}

%%%%%%%%%%%%%%%%%%%%%%%%%%%%%%%%%%%%%%%%%%%%%%%%%%%%%%%%%%%%%%%%%%%%%
% Problem starts here
%%%%%%%%%%%%%%%%%%%%%%%%%%%%%%%%%%%%%%%%%%%%%%%%%%%%%%%%%%%%%%%%%%%%%

\begin{problem}

% TODO: need to make sure fractions match! and that we are doing fish in lecture.
In lecture you learned how many fish you must sample to yield a
fraction that, 95\% of the time, will be within 0.04 of the actual
fraction of contaminated fish.  Notice that the size of the fish 
population was never considered because \emph{it did not matter}.

It seems remarkable that, whether there are a thousand, a million, or a
billion fish in the river, polling only a hundred is sufficient to
be confident of an accurate estimation of the total \% contaminated.

Suppose you were going to serve as an expert witness in a trial.  How
would you explain why the number of people necessary to poll \emph{does
not depend on the population size}?  Remember that juries do not
understand formulas, so you have to provide an intuitive explanation,
which is not quantitative.

\begin{solution}
This was intended to be a thought-provoking, conceptual question. 
In past terms, although most of the class could follow the
derivations and crank through the formulas to calculate poll size and
confidence levels, many students couldn't articulate, and indeed didn't
really believe that the derived sample sizes were actually adequate to
produce reliable estimates.

Here's a way to explain why we model polling people as
independent coin tosses that a jury might be able to follow:

\begin{quote}
Of the approximately 100,000,000 people in the US, there are some
\emph{unknown} number, say 51,000,000, who votes ``A''.  So in this case,
the \emph{fraction} of voters that says ``A'' would be
51,000,000/100,000,000 = 0.51.

To estimate this unknown fraction, we randomly select one person from the
100,000,000 in such a way that \emph{everyone has an equal chance of being
picked}.  Picking a person this way amounts to flipping a coin that had
a chance of coming up ``A'' that was equal to the unknown fraction. 

After we have picked a person and learned their vote, we perform the
procedure again, making sure that everyone is equally likely to be picked 
the second time, and so on, for picking a third, fourth, \etc
person.  Each pick is like flipping a coin whose probability of coming up
``A'' is the same unknown fraction.

Now we all understand that if we keep flipping a coin with a 51\% chance
of coming up Heads, then the more we flip, the closer the fraction of
Heads flipped will be to 51\%.  Mathematical theory lets us calculate us
how many times to flip coins to make the fraction of Heads very likely
close to 51\%, but we needn't go into the details of the calculation.

Now suppose we had two coins, say a penny and a nickel, which had the same
51\% probability of coming up Heads.  Then it's not going to make any
difference which coin we use in our coin flips: the number of flips we
need to get the fraction of Heads flipped being very likely close to 51\%
will be the same whether we flip pennies or nickels.

Different size populations correspond to different coins: the nickel might
correspond to selecting a voter from a population 100,000,000 people, and
the dime might correspond to selecting one from a population of 100.  The
same number of flips of pennies or nickels will allow us to estimate the
probability of ``Heads,'' and hence to estimate the fraction of voters
favoring ``A''.  All that mattered is that the ``penny'' population had the
same probability of Heads, namely, 51,000,000 out of a population of
100,000,000, as the ``nickel'' population, namely, 51 out of 100.

So the number of ``flips'' needed does not depend on whether we're
flipping a ``51\% penny'' or a ``51\% nickel.''  That is, if two
populations have the same fraction of voters saying ``A'', then \emph{the
number of people we need to poll is the same}, even if the populations are
of very different sizes.
\end{quote}

\end{solution}

\end{problem}

%%%%%%%%%%%%%%%%%%%%%%%%%%%%%%%%%%%%%%%%%%%%%%%%%%%%%%%%%%%%%%%%%%%%%
% Problem ends here
%%%%%%%%%%%%%%%%%%%%%%%%%%%%%%%%%%%%%%%%%%%%%%%%%%%%%%%%%%%%%%%%%%%%%

\endinput
