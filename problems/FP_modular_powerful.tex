\documentclass[problem]{mcs}

\begin{pcomments}
  \pcomment{FP_modular_powerful}
  \pcomment{from F07.final.prob7}
  \pcomment{slightly modified by Rich Chan 5/11/10}
\end{pcomments}

\pkeywords{
  eulers_theorem
  phi_function
  eulers_fuc
  modular_arithmetic
  relatively_prime
  inverse
  prime_divisor
}

%%%%%%%%%%%%%%%%%%%%%%%%%%%%%%%%%%%%%%%%%%%%%%%%%%%%%%%%%%%%%%%%%%%%%
% Problem starts here
%%%%%%%%%%%%%%%%%%%%%%%%%%%%%%%%%%%%%%%%%%%%%%%%%%%%%%%%%%%%%%%%%%%%%

\begin{problem}

\bparts

\ppart
Explain why $(-12)^{526}$ has a multiplicative inverse modulo 175.

\examspace[2in]

\begin{solution}
  A number has a multiplicative inverse modulo 175 iff it is relatively
  prime to 175.  But the only prime divisors of 175 are 7 and 5, and the
  only prime divisors of $(-12)^{482}$ are the prime divisors of 12, namely
  2 and 3, so $(-12)^{482}$ is relatively prime to 175.
\end{solution}

\ppart
What is the value of $\phi(175)$, where $\phi$ is Euler's function?

\examspace[1in]

\begin{solution}
  Noting that $175 = 5^2 \cdot 7$.  It follows
  that $\phi(175) = (5^2 - 5^1)(7-1) = 20 \cdot 6 = 120$.
\end{solution}

\ppart
  Call a number from 0 to 174 \emph{powerful} iff some power
  of the number is congruent to 1 modulo 175.  What is the probability
  that a random number from 0 to 174 is powerful?

\examspace[0.7in]

\begin{solution}
  $8/15$.

  Note that $x^k \equiv 1 \pmod n$ for some $k$ iff $x$ has an inverse
  modulo $n$ iff $x$ is relatively prime to $n$.  So being powerful is
  equivalent to being relatively prime to 175.  There are
  $\phi(175) = 120$ numbers from 0 to 174 that are relatively prime to
  175, so
  \[
  \pr{\text{powerful}} = \frac{120}{175} = \frac{24}{35}.
  \]
\end{solution}

\ppart
What is the remainder of $(-12)^{482}$ divided by 175?

\examspace[2in]

\begin{solution}
  144.

  Since -12 and 175 are relatively prime, we have by Euler's
  Theorem that $(-12)^{\phi(175)} \equiv 1 \pmod {175}$, and so
  \[
  (-12)^{482} = \paren{(-12)^{120}}^{4} \cdot (-12)^{2}
              \equiv 1^{4} \cdot (144)
              \equiv 144 \pmod {175}.
  \]
\end{solution}

\eparts
\end{problem}

%%%%%%%%%%%%%%%%%%%%%%%%%%%%%%%%%%%%%%%%%%%%%%%%%%%%%%%%%%%%%%%%%%%%%
% Problem ends here
%%%%%%%%%%%%%%%%%%%%%%%%%%%%%%%%%%%%%%%%%%%%%%%%%%%%%%%%%%%%%%%%%%%%%

\endinput
