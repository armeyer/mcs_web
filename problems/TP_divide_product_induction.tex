\documentclass[problem]{mcs}

\begin{pcomments}
  \pcomment{TP_divide_product_induction}
  \pcomment{minor revision of part(b) of PS_induction_mod_proof}
  \pcomment{by ARM 3/17/13}
\end{pcomments}

\pkeywords{
  divides
  prime
  induction
}

%%%%%%%%%%%%%%%%%%%%%%%%%%%%%%%%%%%%%%%%%%%%%%%%%%%%%%%%%%%%%%%%%%%%%
% Problem starts here
%%%%%%%%%%%%%%%%%%%%%%%%%%%%%%%%%%%%%%%%%%%%%%%%%%%%%%%%%%%%%%%%%%%%%

\begin{problem}
Let $p$ be a prime number and $a_1,\dots,a_n$ integers.  Prove the
following Lemma \emph{by induction}:
\begin{lemma*}
\begin{equation*}
\text{If } p \text{ divides a product } a_1 \cdot a_2 \cdots a_n,\text{ then }\text{ p  divides some } a_i.\tag{*}
\end{equation*}
\end{lemma*}
You may assume the case for $n=2$ which \inhandout{was proved in the
  text}\inbook{was given by Lemma~\bref{lem:prime-divides}}.
  
Be sure to clearly state and label your Induction Hypothesis, Base
case(s), and Induction step.

\examspace

\begin{solution}
We proceed by induction on $n$ with induction hypothesis\footnote{Note
  that the assertion~(*) defines a property of integers $p, a_1 \cdot
  a_2 \cdots a_n,$ and $n$.  Referring to~(*) as a property, $P(n)$,
  really means that~(*) holds \emph{for all} primes $p$ and integers,
  $a_1 \cdot a_2 \cdots a_n$.  That is,
\[
P(n) \eqdef \forall\, \text{primes}\, p.\, \forall a_1,\dots,a_n \in
\integers.\ (p \divides a_1 \cdot a_2 \cdots a_n) \QIMPLIES \exists i
\in \Zintvcc{1}{n}.\, p \divides a_i.
\]
By the same convention, asserting~(*) as a Lemma, really means that
\[
\forall n \geq 1,\, P(n).
\]}
\[
P(n) \eqdef \text{assertion~(*)}.
\]

\inductioncase{Base case}: ($n = 1$).  $P(1)$ asserts that if $p
\divides a_1$, then $p \divides a_1$, which is trivially true.

\inductioncase{Inductive step}: We assume $P(n)$ holds for some $n
\geq 1$ and prove $P(n + 1)$.

So suppose that
\[
p \divides a_1 \cdot a_2 \cdots a_n \cdot a_{n+1}.
\]
Let $b \eqdef a_1 \cdots a_{n}$, so $p \divides b\cdot a_{n+1}$, and
it was already proved that this implies $p \divides a_{n+1}$ or $p
\divides b$.  Now there are two cases:
\begin{itemize}
\item \inductioncase{case} ($p \divides a_{n+1}$): $P(n+1)$ follows
  immediately by letting $i = n+1$.

\item \inductioncase{case} ($p \divides b$): The induction hypothesis
  $P(n)$ implies that $p \divides a_i$ for some $i \in
  \Zintvcc{1}{n}$, which also implies $P(n+1)$.
\end{itemize}
So in either case, $P(n+1)$ holds, which completes the inductive step.

By induction, $P(n)$ holds for all $n \geq 1$, which proves the Lemma.
\end{solution}
  
\end{problem}

%%%%%%%%%%%%%%%%%%%%%%%%%%%%%%%%%%%%%%%%%%%%%%%%%%%%%%%%%%%%%%%%%%%%%
% Problem ends here
%%%%%%%%%%%%%%%%%%%%%%%%%%%%%%%%%%%%%%%%%%%%%%%%%%%%%%%%%%%%%%%%%%%%%

\endinput
