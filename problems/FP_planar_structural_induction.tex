\documentclass[problem]{mcs}

\begin{pcomments}
  \pcomment{FP_structural_induction}
  \pcomment{NOT READY. Probably unusable because follows from v-e+f w/o
    induction --ARM 10/11/09}
  \pcomment{Might be interesting in shared faces are represented, then get
    v-e+f-c=1. --ARM 10/11/09}
  \pcomment{from: Megumi F09}
\end{pcomments}

\pkeywords{
  planar embedding
  structural induction
  Euler_formula
  faces
}

%%%%%%%%%%%%%%%%%%%%%%%%%%%%%%%%%%%%%%%%%%%%%%%%%%%%%%%%%%%%%%%%%%%%%
% Problem starts here
%%%%%%%%%%%%%%%%%%%%%%%%%%%%%%%%%%%%%%%%%%%%%%%%%%%%%%%%%%%%%%%%%%%%%

\begin{problem}  \textbf{Structural Induction}

Recall the definition of a planar graph reproduced in the appendix.
Consider the following additional constructor case as you answer the following questions. 

\fbox{\
\begin{minipage}[t]{6.5in}
\vspace{.1in}

\textbf{Constructor Case:}
(combine disjoint graphs) Suppose $G1$ and $G2$ are disjoint
graphs with planar embeddings. Then the union of their embeddings
is an embedding of $G1$ union $G2$. 
\vspace{.1in}

\end{minipage}}

\examspace[0.1in]

You are asked to prove by structural induction that $v-e+f-2c = 0$, where $v$ is the number of vertices, $e$ is the number of edges, $f$ is the number of faces, and $c$ is the number of connected components.  

%\inhandout{\instatements{\newpage}}

\bparts
\ppart Prove the base case of the structural induction.

\begin{solution}
TBA
\end{solution}

\examspace[2in]

\ppart Prove the constructor cases of the structural induction.

\begin{solution}
\begin{proof}
TBA
%  Assuming $f,g$ are rational functions of $x$ for which $P(f)$ and $P(g)$
%  both hold, we must prove $P(h)$ where

%\textbf{Case $h= f + g$}:  In this case,
%\[
%h^{\prime} = f^{\prime} + g^{\prime},
%\]
%and since $f^{\prime}$ and $g^{\prime}$ are rational functions of $x$ by
%hypothesis, so is their sum by the constructor rules, which proves $P(h)$.

%\textbf{Case $h= f \cdot g$}:

%The Product Rule of derivatives states that:
%\begin{equation}\label{fgderiv}
%h^{\prime} =  f^{\prime} \cdot g + f \cdot g^{\prime},
%\end{equation}
%and since $f, f^{\prime}, g, g^{\prime}$ are rational functions of $x$ by
%hypothesis, so is the right hand side of~\eqref{fgderiv} by the
%constructor rules, which proves $P(h)$.

%\textbf{Case $h= \dfrac{1}{f}$}:

%The Chain Rule gives: 
%\begin{equation}\label{1/fderiv}
%h^{\prime} = \frac{-1}{f^2} \cdot f^{\prime},
%\end{equation}
%and since $f$ and $f^{\prime}$ are rational by hypothesis, so is the right
%hand side of~\eqref{1/fderiv} by the constructor rules, which proves
%$P(h)$.

%We have shown that the induction hypothesis holds in all Constructor cases.
%This completes the proof by structural induction.
\end{proof}
\end{solution}

\eparts
\end{problem}


%%%%%%%%%%%%%%%%%%%%%%%%%%%%%%%%%%%%%%%%%%%%%%%%%%%%%%%%%%%%%%%%%%%%%
% Problem ends here
%%%%%%%%%%%%%%%%%%%%%%%%%%%%%%%%%%%%%%%%%%%%%%%%%%%%%%%%%%%%%%%%%%%%%
