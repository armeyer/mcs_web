\documentclass[problem]{mcs}

\begin{pcomments}
  \pcomment{MQ_modular_arithmetic_afternoon}
  \pcomment{from spring06 Quiz2}
  \pcomment{modified for the afternoon}
\end{pcomments}

\pkeywords{
  modular_arithmetic
  multiplicative_inverse
 }


%%%%%%%%%%%%%%%%%%%%%%%%%%%%%%%%%%%%%%%%%%%%%%%%%%%%%%%%%%%%%%%%%%%%%
% Problem starts here
%%%%%%%%%%%%%%%%%%%%%%%%%%%%%%%%%%%%%%%%%%%%%%%%%%%%%%%%%%%%%%%%%%%%%

\begin{problem}
 
\bparts

-

\ppart
Prove that $33^{60001}$ has a multiplicative inverse modulo 
175.

\examspace[2in]
\begin{solution}
Since $\gcd(33^{60001},175) =
       \gcd(3^{60001} \cdot 11^{60001},5^{2} \cdot 7) =
       1$, $33^{60001}$ has a multiplicative inverse modulo 175.
}
\end{solution}

\ppart What is the value of $\phi(175)$, where $\phi$ is Euler's
function?

\examspace[2in]
\begin{solution}


Noting that $175=5^2\cdot 7$.  It follows
that $\phi(175)=(5^2 - 5^1)(7-1)= 20 \cdot 6 =120$.
\end{solution}

\ppart What is the remainder of $33^{60001}$ divided by 175?

\begin{solution}Since 33 and 175 are relatively prime, we have by Euler's
Theorem that $33^{120} \equiv 1 \pmod {175}$, and so
\[
33^{60001} = \paren{33^{120}}^{500}\cdot 33 \equiv 1^{500}\cdot 33 \equiv 33 \pmod {175}.
\]
\end{solution}

\eparts
\end{problem}

\endinput
