\documentclass[problem]{mcs}

\begin{pcomments}
  \pcomment{FP_logic_relations}
  \pcomment{from: F08.final}
\end{pcomments}

\pkeywords{
  asymptotic bounds
  partial order
}

%%%%%%%%%%%%%%%%%%%%%%%%%%%%%%%%%%%%%%%%%%%%%%%%%%%%%%%%%%%%%%%%%%%%%
% Problem starts here
%%%%%%%%%%%%%%%%%%%%%%%%%%%%%%%%%%%%%%%%%%%%%%%%%%%%%%%%%%%%%%%%%%%%%

%Logic&RElations
\instatements{\newpage}
\begin{problem}
We define a  relation $R$ on pairs of the Boolean values $\true$ and $\false$ 
by
$$ (p_1,p_2)R(q_1,q_2) \mbox{ iff } \left[
\begin{array}{ll}
(p_1\vee p_2) \Rightarrow (q_1\vee q_2) & \mbox{ and } \\
(p_1\wedge p_2) \Rightarrow (q_1\wedge q_2) & 
\end{array} \right].$$
For example, since $(\false \vee \true) \Rightarrow (\true \vee \false)$
and $(\false \wedge \true) \Rightarrow (\true \wedge \false)$,
$$(\false,\true)R(\true,\false).$$

\bparts

\ppart Give another couple of pairs of Boolean values $(p_1,p_2)$ and $(q_1,q_2)$ for which $(p_1,p_2)R(q_1,q_2)$.

\begin{solution}[\vspace{1.5cm}]
For example, $(\false,\false)R(\true,\true)$.
\end{solution}

\ppart Give a couple of pairs of Boolean values $(p_1,p_2)$ and $(q_1,q_2)$ for which $(p_1,p_2)R(q_1,q_2)$ does not hold.

\begin{solution}[\vspace{1.5cm}]
For example, $(\true,\true)R(\false,\false)$.
\end{solution}

For the following parts provide a brief explanation (e.g., by giving a counter example):

\ppart Is $R$ reflexive? 

\begin{solution}[\vspace{3cm}]
Yes, $(p_1\vee p_2) \Rightarrow (p_1\vee p_2)$ and $(p_1\wedge p_2) \Rightarrow (p_1\wedge p_2)$.
\end{solution}

\ppart Is $R$ symmetric? 

\begin{solution}[\vspace{3cm}]
No, $(\true, \false)R(\true, \true)$ but not $(\true, \true)R(\true, \false)$.
\end{solution}

\ppart Is $R$ antisymmetric? 

\begin{solution}
No, $(\true, \false)R(\false,\true)$, $(\false,\true)R(\true,\false)$, and $(\true,\false)\neq (\false,\true)$.
\end{solution}

%\ppart Is $R$ transitive? 

%\begin{solution}
%Yes, this follows directly from the transitivity of $\Rightarrow$. If $(p_1\vee p_2) \Rightarrow (q_1\vee q_2)$ and $(q_1\vee q_2) \Rightarrow (z_1\vee z_2)$, then $(p_1\vee p_2) \Rightarrow (z_1\vee z_2)$. If $(p_1\wedge p_2) \Rightarrow (q_1\wedge q_2)$ and $(q_1\wedge q_2) \Rightarrow (z_1\wedge z_2)$, then $(p_1\wedge p_2) \Rightarrow (z_1\wedge z_2)$. So, $(p_1,p_2)R(q_1,q_2)$ and $(q_1,q_2)R(z_1,z_2)$ implies $(p_1,p_2)R(z_1,z_2)$
%\end{solution}

\eparts

\end{problem}

\endinput
