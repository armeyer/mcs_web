\documentclass[problem]{mcs}

\begin{pcomments}
  \pcomment{TP_preserved_under_isomorphism}
  \pcomment{overlaps FP_graphs_short_answer, FP_multiple_choice_unhidden}
  \pcomment{by ARM 3/29/13 for first simple graph lecture}
  \pcomment{solns updates ARM 10/31/15}
\end{pcomments}

\pkeywords{
  simple_graph
  isomorphism
  vertices
  preserved
  degree
  edge
  path
}

%%%%%%%%%%%%%%%%%%%%%%%%%%%%%%%%%%%%%%%%%%%%%%%%%%%%%%%%%%%%%%%%%%%%%
% Problem starts here
%%%%%%%%%%%%%%%%%%%%%%%%%%%%%%%%%%%%%%%%%%%%%%%%%%%%%%%%%%%%%%%%%%%%%
\begin{problem}
Which of the items below are simple-graph properties preserved under
isomorphism?

\bparts

\ppart There is a cycle that includes all the vertices.

\begin{staffnotes}
If asked, explain that simple graph cycles can be defined in the
essentially same way as for digraphs.  The only difference is that going
back and forth on the same edge---a length 2 ``cycle''---is not
considered to be a cycle.
\end{staffnotes}

\begin{solution}
Preserved.
\end{solution}

\ppart The vertices are numbered 1 through 7.

\begin{solution}
Not a property of simple graphs.

Numbers or labels may be assigned to vertices for various purposes,
but vertex numbering is something added to a simple graph but is not a
``built in'' property of simple graphs.
\end{solution}

\ppart The vertices can be numbered 1 through 7.

\begin{solution}
Preserved.

The vertices \emph{can} be numbered 1 through 7 iff the graph has 7
vertices, which is a property that is preserved under isomorphism.
\end{solution}

\ppart There are two degree 8 vertices.

\begin{solution}
Preserved.
\end{solution}

\ppart\label{eqlen} Two edges are of equal length.

\begin{solution}
``Length'' is not a property of edges in a simple graph.  (Edges in a
  \emph{drawing} of the graph may or may not have the same length,
  depending on how they are drawn.)
\end{solution}

%\ppart There are exacty two spanning trees.

\ppart No matter which edge is removed, there is a path between any two
  vertices.

\begin{solution}
Preserved.
\end{solution}

\ppart There are two cycles that do not share any vertices. % too easy

\begin{solution}
Preserved.
\end{solution}

%\ppart There are two connected components. % too easy

%\ppart\label{edgesubset} One edge is a subset of another one.

\ppart\label{vertexsubset} The vertices are sets.

\begin{solution}
NOT Preserved.

Isomorphism does not preserve what things are made of, only how they
fit together.

However, this question was intended to remind you of \inhandout{the
  Theorem}\inbook{Theorem~\bref{thm:posetrepsets}} that every DAG is
isomorphic to a DAG whose vertices are sets.
\end{solution}

\ppart The graph can be drawn in a way that all the edges have the
same length.

\begin{solution}
Preserved.

If you can draw a simple graph in a certain way, then you can draw any
isomorphic graph in the same way.
\end{solution}


\ppart\label{edgecross} No two edges cross.

\begin{solution}
``Crossing'' is not a property of edges in a simple graph.

On the bother hand, edges in a \emph{drawing} of a graph may or may
not cross, depending on how they are drawn.  The property that a graph
\emph{can be drawn} in the plane without edges crossing \textbf{is}
preserved under isomorphism.  It is an important enough property that
\inhandout{a whole chapter} \inbook{Chapter~\bref{planar_graphs_chap}}
is devoted entirely to such \emph{planar} graphs.
\end{solution}

%\ppart The graph is 4-colorable. % too easy

%\ppart Adding an edge between any two vertices creates a cycle. % spanning tree one better

\ppart\label{isoOR} The $\QOR$ of two properties that are preserved
under isomorphism.

\begin{solution}
Preserved.
\begin{staffnotes}
When problem is done, have students come back a\textbf{ write out a proof of this case}:
\end{staffnotes}

\begin{proof}
Suppose $P$ and $Q$ are graph properties preserved under isomorphism,
and $G$ and $H$ are isomorphic simple graphs.  Then the truth values
of both $P$ and $Q$ will be the same in $G$ and $H$, and so any
propositional combination of $P$ and $Q$ will also have the same truth
value in $G$ and $H$.
\end{proof}

Second, more pedantic proof:
\begin{proof}
  Let $R \eqdef P \QOR
Q$.  Then
\[\begin{array}{rcll}
R(G) 
  & \QIFF & P(G) \QOR Q(G)
     & \text{(def of $R$)}\\
  & \QIFF & P(H) \QOR Q(H) 
     & \text{(since $P$, $Q$ are preserved)}\\
  & \QIFF & R(H)
     & \text{(def of $R$)},
\end{array}\]
so $R$ is preserved, as claimed.
\end{proof}

\end{solution}

\ppart The negation of a property that is preserved under isomorphism.

\begin{solution}
Preserved, by the same reasoning as for the $\QOR$ of two properties.
\end{solution}

\eparts

\end{problem}

\endinput
