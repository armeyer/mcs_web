\documentclass[problem]{mcs}

\begin{pcomments}
  \pcomment{CP_murphys_law}
  \pcomment{F02.cp14f}
\end{pcomments}

\pkeywords{
  murphy
  mutually_independent
  independent
  indicator
  sum
}

%%%%%%%%%%%%%%%%%%%%%%%%%%%%%%%%%%%%%%%%%%%%%%%%%%%%%%%%%%%%%%%%%%%%%
% Problem starts here
%%%%%%%%%%%%%%%%%%%%%%%%%%%%%%%%%%%%%%%%%%%%%%%%%%%%%%%%%%%%%%%%%%%%%

\begin{problem}
In this problem you will check a proof of:
\begin{theorem*}[Murphy's Law]
Let $A_1, A_2, \dots A_n$ be mutually independent events, and let $T$ be
the number of these events that occur.  The probability that none of the
events occur is at most $e^{-\expect{T}}$.
\end{theorem*}

To prove Murphy's Law, note that
\begin{equation}\label{Tsum}
T = T_1 + T_2 + \dots + T_n,
\end{equation}
where $T_i$ is the indicator variable for the event $A_i$.  Also, remember
that
\begin{equation} \label{1+xeleq}
1 + x \leq e^x
\end{equation}
for all $x$\iffalse
and
\begin{equation}
1 + x \approx e^x \label{1+xesim},
\end{equation}
for $0 \leq x \leq 1$.  Both~(\ref{1+xeleq}) and~(\ref{1+xesim}) follow
from the Taylor's expansion of $e^x$\fi.

%\bparts

%\ppart
Justify each line in the following derivation (without looking it up
in the text):

\instatements{\begin{proof}
\begin{align*}
\pr{T = 0}
  & = \bar{A_1 \union A_2 \union \cdots \union A_n}\\
  & =  \pr{\bar{A_1} \cap \bar{A_2} \cap \dots \cap \bar{A_n}}\\
  & =  \prod_{i=1}^n \pr{\bar{A_i}}\\
  & =  \prod_{i=1}^n 1 - \pr{A_i}\\
  & \leq  \prod_{i=1}^n e^{-\pr{A_i}}\\
  & =  e^{-\sum_{i=1}^n \pr{A_i}}\\
  & =  e^{-\sum_{i=1}^n \expect{T_i}}\\
  & =  e^{-\expect{T}}.
\end{align*}
\end{proof}
}

\begin{solution}

\begin{proof}
\begin{align*}
\pr{T = 0}
  & = \bar{A_1 \union A_2 \union \cdots \union A_n} & \text{(def. of $T$)}\\
  & =  \pr{\bar{A_1} \cap \bar{A_2} \cap \dots \cap \bar{A_n}} &
          \text{(De Morgan's law)}\\
  & =  \prod_{i=1}^n \pr{\bar{A_i}} & \text{(mutual independence of $A_i$'s)}\\
  & =  \prod_{i=1}^n 1 - \pr{A_i} & \text{(complement rule)}\\
  & \leq  \prod_{i=1}^n e^{-\pr{A_i}} & \text{(by~(\ref{1+xeleq}))}\\
  & =  e^{-\sum_{i=1}^n \pr{A_i}} & \text{(exponent algebra)}\\
  & =  e^{-\sum_{i=1}^n \expect{T_i}} & \text{(expectation of indicator variable)}\\
  & =  e^{-\expect{T}}. & \text{((\ref{Tsum}) \& linearity of expectation)}
\end{align*}
\end{proof}

\end{solution}

\end{problem}

%%%%%%%%%%%%%%%%%%%%%%%%%%%%%%%%%%%%%%%%%%%%%%%%%%%%%%%%%%%%%%%%%%%%%
% Problem ends here
%%%%%%%%%%%%%%%%%%%%%%%%%%%%%%%%%%%%%%%%%%%%%%%%%%%%%%%%%%%%%%%%%%%%%

\endinput
 
