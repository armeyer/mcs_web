\documentclass[problem]{mcs}

\begin{pcomments}
  \pcomment{FP_validity_by_cases}
  \pcomment{S17.mid1}
  \pcomment{ARM 2/17/17}
%  \pcomment{variant of 2003 final problem 1(a)}
\end{pcomments}

\pkeywords{
  validity
  cases
  truth_table
}

%%%%%%%%%%%%%%%%%%%%%%%%%%%%%%%%%%%%%%%%%%%%%%%%%%%%%%%%%%%%%%%%%%%%%
% Problem starts here
%%%%%%%%%%%%%%%%%%%%%%%%%%%%%%%%%%%%%%%%%%%%%%%%%%%%%%%%%%%%%%%%%%%%%

\begin{problem}%\label{generprob}
The formula
\begin{align*}
\QNOT(\bar{A} \QIMPLIES B) & \QAND A \QAND C\\
                           & \QIMPLIES\\
         D \QAND E \QAND F & \QAND G \QAND H \QAND I \QAND J \QAND K \QAND L \QAND M
\end{align*}
turns out to be valid.

\begin{problemparts}
\problempart Explain why verifying the validity of this formula
\emph{by truth table} would be very hard for one person to do with
pencil and paper (no computers).

\examspace[1.0in]

\begin{solution}
The number of entries in a truth table here would be $2^{14}$ or about
16,000 since there are 14 variables.  This could take weeks for one
person to do by hand.
\end{solution}

\problempart Verify that the formula is valid, reasoning by
cases according to the truth value of $A$.

\begin{proof}
\inductioncase{Case}: ($A$ is \True).
\examspace[2.0in]

\begin{solution}
Since the implication $\overline{\True} \QIMPLIES B$ is \True, the
expression $\QNOT(\overline{A} \QIMPLIES B)$ evaluates to \False.  So
the hypothesis (upper formula) of the main implication is \False,
making the whole implication formula \True.  So the formula is
\True\ in all assignments with $A$ assigned \True.
\end{solution}

\inductioncase{Case}: ($A$ is \False). 
\examspace[2.0in]

\begin{solution}
The hypothesis (upper formula) side of the formula is now equivalent
to $(\False\ \QAND \dots)$, and so is immediately \False.  As in the
previous case, this makes the whole implication formula \True.  So the
formula is \True\ in all assignments with $A$ assigned \False.

\medskip
Since $A$ must be \True\ or \False\ in any truth assignment, the
formula is \True\ in any case.  That is, it is valid.
\end{solution}

\end{proof}
\end{problemparts}

\end{problem}

%%%%%%%%%%%%%%%%%%%%%%%%%%%%%%%%%%%%%%%%%%%%%%%%%%%%%%%%%%%%%%%%%%%%%
% Problem ends here
%%%%%%%%%%%%%%%%%%%%%%%%%%%%%%%%%%%%%%%%%%%%%%%%%%%%%%%%%%%%%%%%%%%%%

\endinput
