%% Various permutations of proofs inside solutions.

\documentclass[problem]{mcs}

\begin{problem}
  The proof below uses the Well Ordering Principle to prove that every
  amount of postage that can be paid exactly, using only 10 cent and 15
  cent stamps, is divisible by 5.  Let $S(n)$ mean that exactly $n$ cents
  postage can be paid using only 10 and 15 cent stamps.  Then the proof
  shows that

%%%%%%%%%%%%%%%%%%%%%%%%%%%%%%%%%%%%%%%%%%%%%%%%%%%%%%%%%%%%%%%%%%%%%%%%%%%

Proof outside of a solution:

\begin{proof}
 $n$ is a counterexample to~(*) if $n$ cents postage can be
  made and $n$ is not divisible by 5, so the predicate
\[
S(n)\text{ and } \QNOT(5 \divides n)
\]
defines the set, $C$, of counterexamples.
\end{proof}

%%%%%%%%%%%%%%%%%%%%%%%%%%%%%%%%%%%%%%%%%%%%%%%%%%%%%%%%%%%%%%%%%%%%%%%%%%%

Solution without a proof:

\begin{solution}
 $n$ is a counterexample to~(*) if $n$ cents postage can be
  made and $n$ is not divisible by 5, so the predicate
\[
S(n)\text{ and } \QNOT(5 \divides n)
\]
defines the set, $C$, of counterexamples.
\end{solution}

%%%%%%%%%%%%%%%%%%%%%%%%%%%%%%%%%%%%%%%%%%%%%%%%%%%%%%%%%%%%%%%%%%%%%%%%%%%

Proof inside solution:

\begin{solution}

\begin{proof}
 $n$ is a counterexample to~(*) if $n$ cents postage can be
  made and $n$ is not divisible by 5, so the predicate
\[
S(n)\text{ and } \QNOT(5 \divides n)
\]
defines the set, $C$, of counterexamples.
\end{proof}
\end{solution}

\begin{solution}

%%%%%%%%%%%%%%%%%%%%%%%%%%%%%%%%%%%%%%%%%%%%%%%%%%%%%%%%%%%%%%%%%%%%%%%%%%%

Proof inside solution, with extra commentary after proof.

\begin{proof}
 $n$ is a counterexample to~(*) if $n$ cents postage can be
  made and $n$ is not divisible by 5, so the predicate
\[
S(n)\text{ and } \QNOT(5 \divides n)
\]
defines the set, $C$, of counterexamples.
\end{proof}
Some final commentary on this solution.
\end{solution}

\end{problem}
