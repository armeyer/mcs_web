\documentclass[problem]{mcs}

\begin{pcomments}
  \pcomment{PS_conditional_space}
  \pcomment{from: new S10.ps11 by Rich}
\end{pcomments}

\pkeywords{
  probability
}

%%%%%%%%%%%%%%%%%%%%%%%%%%%%%%%%%%%%%%%%%%%%%%%%%%%%%%%%%%%%%%%%%%%%%
% Problem starts here
%%%%%%%%%%%%%%%%%%%%%%%%%%%%%%%%%%%%%%%%%%%%%%%%%%%%%%%%%%%%%%%%%%%%%

\newcommand{\prb}[1]{\mathop{\textup{Pr}_B}\nolimits\left\{#1\right\}}

\begin{problem}

  Let there be a sample space $\sspace$ and a function 
  $\pr{}: \sspace\to [0,1]$ so $(\sspace, \pr{})$ together forms
  a \textit{probability space}. (see Definition \bref{LN12:probsp})

  Let $\prb{}$ be defined as:
  \[
  \prb{A} \eqdef \frac{\pr{A \intersect B}}{\pr{B}}
  \]
  
  Prove that $(\sspace, \prb{})$ is a \textit{probability space}.
  
\begin{solution}
  
  To show that $(\sspace, \prb{})$ is a \textit{probability space},
  we will need to show that $\prb{}$ is a function mapping $\sspace$
  to $[0,1]$ and that the sum
  \[
  \sum_{w \in \sspace} \prb{w} = 1.
  \]
  
  We can show that $\forall A. \prb{A} \geq 0$, since both the numerator
  $\pr{A \intersect B}$ and the denominator $\pr{B}$ are nonnegative.
  
  We can show that $\forall A. \prb{A} \leq 1$ by using the \bref{LN12:subsetbound}
  property of probability function. That is, since we know that 
  $(A \intersect B) \subseteq B$, we know that $\pr{A \intersect B} \leq \pr{B}$.
  Because of that, we see that the numerator is always less than or equal to
  the denominator so $\prb{A} \leq 1$.
  
  Lastly, we need to show that
  \[
  \sum_{w \in \sspace} \prb{w} = 1.
  \]
  In other words, if we let $E$ be the event consisting of every outcome $w$ in $\sspace$,
  this is equivalent to showing that
  \[
  \prb{E} = 1.
  \]
  We may divide E into the disjoint sets $B$ and $(E-B)$, so
  \begin{align*}
    \prb{E} &= \prb{B \union (E-B)} \\
            &= \frac{\pr{(B \union (E-B)) \intersect B}}{\pr{B}} \\
            &= \frac{\pr{B}}{\pr{B}} \\
            &= 1
  \end{align*}

\end{solution}

\end{problem}

%%%%%%%%%%%%%%%%%%%%%%%%%%%%%%%%%%%%%%%%%%%%%%%%%%%%%%%%%%%%%%%%%%%%%
% Problem ends here
%%%%%%%%%%%%%%%%%%%%%%%%%%%%%%%%%%%%%%%%%%%%%%%%%%%%%%%%%%%%%%%%%%%%%

\endinput
