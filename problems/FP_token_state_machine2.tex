\documentclass[problem]{mcs}

\begin{pcomments}
  \pcomment{FP_token_state_machine2}
  \pcomment{F15.final}
  \pcomment{variation of FP_token_state_machine}
\end{pcomments}

\pkeywords{
  state_machine
  invariant
  preserved_invariant
  induction
  congruence
  reachable
}

%%%%%%%%%%%%%%%%%%%%%%%%%%%%%%%%%%%%%%%%%%%%%%%%%%%%%%%%%%%%%%%%%%%%%
% Problem starts here
%%%%%%%%%%%%%%%%%%%%%%%%%%%%%%%%%%%%%%%%%%%%%%%%%%%%%%%%%%%%%%%%%%%%%

\begin{problem}
``Double Token Replacing'' is a process for updating a set of tokens,
  each colored black or white, starting from a single black token.  At
  each step,
\begin{itemize}
\item one black token can be replaced with two white tokens, or
\item one white token can be replaced with two black tokens, or
\item the number of both black and white tokens can be doubled (multiplied by two).
\end{itemize}

  We can model this game as a state machine whose states are pairs
  $(n_b,n_w)$ where $n_b \geq 0$ equals the number of black tokens,
  and $n_w \geq 0$ equals the number of white tokens.  So the start
  state is $(1,0)$.

\bparts
  
\ppart
Define the predicate $F(n_b,n_w)$ by the rule:
\[
F(n_b,n_w) \eqdef\  n_b > 0 \QAND\ [n_b - n_w  = 1\text{ or } 2 \inzmod{3}].
\]

Prove that if $F(n_b,n_w)$ is true, the state $(n_b,n_w)$ is reachable
from the state state.

\examspace[3.5in]

\begin{solution}
The proof of the Claim will be by induction in $n$ using induction
hypothesis $P(n) \eqdef$
\[
 \forall (n_b, n_w).\,
 [(n_b + n_w = n) \QAND  F(n_b,n_w)]
     \QIMPLIES (n_b,n_w) \text{ is reachable}.
\]

The base cases will be when $n \leq 2$.

\begin{proof}
\inductioncase{Base case} ($n \leq 2$):
There are only two states with $n \leq 2$ that satisfy $F$:
\[
(1,0),\ (2,0).
\]
But $(1,0)$ is reachable since it is the start state, and $(2,0)$ is
reachable by doubling the start state.  This proves $P(n)$ for $n \leq 2$. 

\inductioncase{Inductive step}: Suppose $n \ge 2$ and $n_b + n_w =
n+1$ and $F(n_b,n_w)$ holds.  We want to show that $(n_b, n_w)$ is
reachable.

Since $n+1 \geq 3$, either $n_b \geq 2$ or $n_w \geq 2$.

In the case that $n_b \geq 2$, we have $n_b-2 \geq 0$, so
$(n_b-2,n_w+1)$ is a state.  If $n_B - 2 >0$, then $F(n_b-2,n_w+1)$ holdss
because
\[
(n_b-2)-(n_w+1) = n_b-n_w -3 = n_b-n_w  = 1\text{ or } 2 \inzmod{3}.
\]
Since
\[
(n_b-2)+(n_w+1) = n_b+n_w - 1 = n,
\]
we conclude by induction hypothesis $P(n)$ that $(n_b-2,n_w+1)$ is
reachable.  But $(n_b-2,n_w+1)$ transitions in one step to
$(n_b,n_w)$, which proves that $(n_b,n_w)$ is reachable.

In the case that $n_b-2 = 0$, \TBA{$n_b = 0$}

The same argument applies in the case that $n_w \geq 2$.

We conclude that in any case $(n_b,n_w)$ is reachable, which completes
the induction step.
\end{proof}

To understand the difference between this problem and proving that $T$
is a preserved invariant, it may be helpful to think about the problem
of making different postage amounts, for example using 6 and 15 cent
stamps\inbook{ (Problem~\bref{FP_6_and_15_cent_stamps_by WOP})}.  An
easy invariant proof confirms the fact that every makeable postage
amount is divisible by three.  This preserved invariant for postage
corresponds to invariance of $T$ for reachable token machine states.

A very different fact is that \emph{every} (large enough) amount of
postage that is divisible by three is actually makeable from 6 and 15
cent stamps.  Postage \emph{makeability} for amounts divisible by
three corresponds to \emph{reachability} of token machine states
satisfying $T$.
\end{solution}


\eparts

\end{problem}

%%%%%%%%%%%%%%%%%%%%%%%%%%%%%%%%%%%%%%%%%%%%%%%%%%%%%%%%%%%%%%%%%%%%%
% Problem ends here
%%%%%%%%%%%%%%%%%%%%%%%%%%%%%%%%%%%%%%%%%%%%%%%%%%%%%%%%%%%%%%%%%%%%%

\endinput
