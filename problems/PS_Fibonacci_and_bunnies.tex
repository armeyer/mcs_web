%PS_binomial_problem

\documentclass[problem]{mcs}

\begin{pcomments}
  \pcomment{from: S09.ps10}
\end{pcomments}

\pkeywords{
  generating functions
}

%%%%%%%%%%%%%%%%%%%%%%%%%%%%%%%%%%%%%%%%%%%%%%%%%%%%%%%%%%%%%%%%%%%%%
% Problem starts here
%%%%%%%%%%%%%%%%%%%%%%%%%%%%%%%%%%%%%%%%%%%%%%%%%%%%%%%%%%%%%%%%%%%%%

\begin{problem}

  The famous mathematician, Fibonacci, has decided to start a rabbit farm
  to fill up his time while he's not making new sequences to torment
  future college students.  Fibonacci starts his farm on month zero (being
  a mathematician), and at the start of month one he recieves his first
  pair of rabbits.  Each pair of rabbits takes a month to mature, and
  after that breeds to produce one new pair of rabbits each month.
  Fibonacci decides that in order never to run out of rabbits or money,
  every time a batch of new rabbits is born he'll sell a number of newborn
  pairs equal to the total number of pairs he had three months prior.
  Fibonacci is convinced that this way he'll never run out of stock.

\bparts

\ppart Define the number of pairs, $f_n$, of rabbits Fibonacci has in
month $n$, using a recurrence relation.  That is, define $f_n$ in terms of
various $f_i$ where $i<n$.

\begin{solution}$f_0$ is defined as 0, and $f_1$ as 1 by the problem text.
  After that, $f_n$, the number of rabbits in month $n$, is equal to the
  number of rabbits in the previous month, $f_{n-1}$, plus the number of
  newborn rabbits, which is equal to the number of living rabbits at least
  one month old at the start of the last month, $f_{n-2}$, minus the
  number he sells, which is defined as $f_{n-3}$.  Thus
\[
f_n = f_{n-1} + (f_{n-2} - f_{n-3}).
\]
\end{solution}

\ppart Building on the method used to find a generating function for the
Fibonacci sequence in Lecture Notes 11, let $F(x)$ be the generating
function
\[
F(x) \eqdef f_0+f_1x+f_2x^2+\cdot.
\]
Express $F(x)$ as a quotient of polynomials.

\begin{solution}$F(x) = x + x F(x) + x^2 F(x) - x^3 F(x)$.
\[
F(x) = \frac{x}{x^3 - x^2 - x + 1}.
\]
\end{solution}

\ppart
Find a partial fraction decomposition of the generating function you found in
the previous section.

\hint
Factoring the expression in the denominator of your generating function should
be easy, but be sure to remember or look up how to do a partial fraction
decomposition of an expression with repeated roots.

\begin{solution}Setting $F(x)$ equal to
\[
\frac{A}{x+1} + \frac{B}{x-1} + \frac{C}{(x-1)^2}
\]
and solving for $A, B$ and $C$ gives us
\[
A = -\frac{1}{4} \qquad B  = \frac{1}{4} \qquad C  = \frac{1}{2}.
\]
\end{solution}

\ppart
Finally, use the partial fraction decomposition and the rules outlined in
Lecture Notes 11 to come up with a closed form expression for the number of
pairs of rabbits Fibonacci has on his farm on month $n$.

\begin{solution}We find the coefficient as the sum of the coefficients for each term
in the partial fraction expansion.  $A(1 - (-x))$ expands out to $A(
1 - x + x^2 - x^3...)$, $-B/(1 - x)$ expands out to $-B(1 + x + x^2 + 
x^3 + ...)$, and $C/(1 - x)^2$ expands to $C(1 + 2x + 3x^2 + ...)$.
Summing these all up we get for even $n$, the coefficient of $x^n$ is
\[
f_n =
\begin{cases}
 \dfrac{n}{2} & \text{if $n$ is even},\\
 \dfrac{n+1}{2}  & \text{if $n$ is odd}.
\end{cases}
\]
.
\end{solution}

\eparts

\end{problem}

%%%%%%%%%%%%%%%%%%%%%%%%%%%%%%%%%%%%%%%%%%%%%%%%%%%%%%%%%%%%%%%%%%%%%
% Problem ends here
%%%%%%%%%%%%%%%%%%%%%%%%%%%%%%%%%%%%%%%%%%%%%%%%%%%%%%%%%%%%%%%%%%%%%

\endinput
