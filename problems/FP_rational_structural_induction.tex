\documentclass[problem]{mcs}

\begin{pcomments}
  \pcomment{FP_rational_structural_induction}
  \pcomment{verbatim on 12/5/11 by kazerani from S09 Final P2}
\end{pcomments}


\pkeywords{
  rational
  structural induction
  polynomial algebra
  differentiation
  proof
}

%%%%%%%%%%%%%%%%%%%%%%%%%%%%%%%%%%%%%%%%%%%%%%%%%%%%%%%%%%%%%%%%%%%%%
% Problem starts here
%%%%%%%%%%%%%%%%%%%%%%%%%%%%%%%%%%%%%%%%%%%%%%%%%%%%%%%%%%%%%%%%%%%%%

\begin{problem}

\examspace
\begin{problem}[9]  \textbf{Structural Induction}

\fbox{\
\begin{minipage}[t]{6.5in}
\vspace{.1in}
\begin{definition*}
The {\it rational functions} of a single variable, $x$, are defined
recursively as follows:

\vspace{.1in}
\textbf{Base cases:}
\begin{itemize}
\item The identity function, $\ide(x) \eqdef x$, and
\item any constant function
\end{itemize}
are rational functions of $x$.

\vspace{.1in}
\textbf{Constructor cases:}

If $f,g$ are rational functions of $x$, then so are $f + g, f
\cdot g, \text{ and } 1/f$.
\end{definition*}
\end{minipage}}

\examspace[0.1in]

You are asked to prove by structural induction that the rational functions
of $x$ are closed under taking derivatives.  That is, using the induction
hypothesis,
\[
P(h) \eqdef [h^{\prime} \text{ is a rational function}],
\]
where $h^{\prime} \eqdef d\,h(x)/dx$, prove that $P(h)$ holds for all
functions, $h$, that are rational functions of a single variable, $x$.

\bparts

\ppart[2] Prove the base cases of the structural induction.

\solution[\vspace{1.5in}]{
\begin{proof}
We must show $P(\ide(x))$ and $P(\text{constant-function})$.  But
$\ide^{\prime}$ is the constant function 1, and the derivative of a
constant function is the constant function 0, and these are rational
functions of $x$ by definition.

This proves that the induction hypothesis holds in the Base cases.
\end{proof}
}

\ppart[7] Prove the constructor cases of the structural induction.

\solution[\vspace{3.5in}]{
\begin{proof}
  Assuming $f,g$ are rational functions of $x$ for which $P(f)$ and $P(g)$
  both hold, we must prove $P(h)$ where

\textbf{Case $h= f + g$}:  In this case,
\[
h^{\prime} = f^{\prime} + g^{\prime},
\]
and since $f^{\prime}$ and $g^{\prime}$ are rational functions of $x$ by
hypothesis, so is their sum by the constructor rules, which proves $P(h)$.

\textbf{Case $h= f \cdot g$}:

The Product Rule of derivatives states that:
\begin{equation}\label{fgderiv}
h^{\prime} =  f^{\prime} \cdot g + f \cdot g^{\prime},
\end{equation}
and since $f, f^{\prime}, g, g^{\prime}$ are rational functions of $x$ by
hypothesis, so is the right hand side of~\eqref{fgderiv} by the
constructor rules, which proves $P(h)$.

\textbf{Case $h= \dfrac{1}{f}$}:

The Chain Rule gives:
\begin{equation}\label{1/fderiv}
h^{\prime} = \frac{-1}{f^2} \cdot f^{\prime},
\end{equation}
and since $f$ and $f^{\prime}$ are rational by hypothesis, so is the right
hand side of~\eqref{1/fderiv} by the constructor rules, which proves
$P(h)$.

We have shown that the induction hypothesis holds in all Constructor cases.
This completes the proof by structural induction.
\end{proof}
}

\eparts

\end{problem}

%%%%%%%%%%%%%%%%%%%%%%%%%%%%%%%%%%%%%%%%%%%%%%%%%%%%%%%%%%%%%%%%%%%%%
% Problem ends here
%%%%%%%%%%%%%%%%%%%%%%%%%%%%%%%%%%%%%%%%%%%%%%%%%%%%%%%%%%%%%%%%%%%%%

\endinput
