\documentclass[problem]{mcs}

\begin{pcomments}
  \pcomment{FP_chromatic_union}
  \pcomment{by Harry Sanabria, May 2013}
%  \pcomment{from: S13.final}
\end{pcomments}

\pkeywords{
  chromatic_number
  graph_coloring
  union
}

%%%%%%%%%%%%%%%%%%%%%%%%%%%%%%%%%%%%%%%%%%%%%%%%%%%%%%%%%%%%%%%%%%%%%
% Problem starts here
%%%%%%%%%%%%%%%%%%%%%%%%%%%%%%%%%%%%%%%%%%%%%%%%%%%%%%%%%%%%%%%%%%%%%


\begin{problem}
For simple graphs $G$ and $H$ with the same vertices, explain why the
chromatic number $\chi(G \union H)$ of the union of $G$ and $H$ is at
most $\chi(G)\cdot \chi(H)$.

\begin{solution}
Color each vertex $v \in \vertices{G}$, with a new ``color'' that is a
pair consisting of the color of $v$ in a minimal coloring of $G$ and
its color in a minimal coloring of $H$.  Clearly the number of colors
used will be at most $\chi(G)\cdot \chi(H)$.

Now let's prove that this is a valid coloring.  Let $u,v$ be two
adjacent vertices in $G \union H$.  This means by definition that they
must be adjacent in $G$ or in $H$.  Suppose wlog that $u$ and $v$ are
adjacent in $G$.  Then the colors of $u$ and $v$ must differ in any
coloring of $G$, so the colors of $u$ and $v$ in the specified
pair-coloring of $G \union H$ will differ because they differ in their
first coordinate.  So adjacent vertices in $G \union H$ will be
assigned different colors, as required.
\end{solution}

\end{problem}

%%%%%%%%%%%%%%%%%%%%%%%%%%%%%%%%%%%%%%%%%%%%%%%%%%%%%%%%%%%%%%%%%%%%%
% Problem ends here
%%%%%%%%%%%%%%%%%%%%%%%%%%%%%%%%%%%%%%%%%%%%%%%%%%%%%%%%%%%%%%%%%%%%%

\endinput
