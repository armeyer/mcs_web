\documentclass[problem]{mcs}

\begin{pcomments}
  \pcomment{PS_relation_transitive_properties}
  \pcomment{from: S04.ps3 (by wing; edited buy ARM 3/13/10)}
\end{pcomments}

\pkeywords{
  transitive
  composition
  relation
  union_of_relations
  intersection_of_relations
  inverse
}

%%%%%%%%%%%%%%%%%%%%%%%%%%%%%%%%%%%%%%%%%%%%%%%%%%%%%%%%%%%%%%%%%%%%%
% Problem starts here
%%%%%%%%%%%%%%%%%%%%%%%%%%%%%%%%%%%%%%%%%%%%%%%%%%%%%%%%%%%%%%%%%%%%%

\begin{problem} 
  
  Let $R$ and $S$ be transitive relations on a set, $A$.  For each of
  the relations below, either prove that it is transitive, or give a
  counter-example showing that it may \emph{not} be transitive.

  \begin{itemize}

  \item $R^{-1}$

  \begin{solution}
    Transitive.

    Take arbitrary $a,b,c$ such that $(a, b), (b,c) \in
    R^{-1}$.  Then $(b,a), (c,b) \in R$.  Because $R$ is
    transitive, $(c,a) \in R$. By definition of inverse, $(a,c) \in
    R^{-1}$.  Therefore, $R^{-1}$ is transitive.
  \end{solution}

  \item $R \intersect S$

  \begin{solution}
  Transitive.

  Suppose $(a,b), (b,c) \in R \intersect S$.  Thus, we have $(a,b),
  (b,c) \in R \QAND (a,b), (b,c) \in S$.  Because $R$ and $S$ are
  transitive, we have $(a,c) \in R \QAND (a,c) \in S$.  Therefore,
  $(a,c) \in R \intersect S$.
  \end{solution}
  
  \item $R \union S$ 

  \begin{solution}
    \emph{Not} transtiive.  

    Here is a counterexample: let $R$ and $S$ be relations on the set
    $\set{1, 2, 3}$ where
    \begin{align*}
    R \eqdef & \set{(1,1) (2,2) (3,3) (1,2) (2,1)},\\
    S \eqdef & \set{(1,1) (2,2) (3,3) (2,3) (3,2)}.
    \end{align*}
    It's easy to check that $R$ and $S$ are both transitive.
    But $R\cup S$ is not transitive, because $(1,2),(2,3) \in R\cup S$
    and $(1,3) \notin R\cup S$.  Therefore $R\cup S$ is not transitive.
  \end{solution}

  \item $R \composition S$
  
  \begin{solution}
   \emph{Not} transitive.

    Here is a counterexample:
    let $R$ and $S$ be relations on the set $\set{1,2,3,4,5}$ where
    \begin{align*}
      S \eqdef & \set{(1,4) (2,5)},\\
      R \eqdef & \set{(4,2) (5,3)}.
    \end{align*}

    Now $R$ and $S$ are transitive (vacuously, since no two pairs of
    related elements overlap).  Also, $(1,2),(2,3) \in R \composition
    S$, but $(1,3) \notin R \composition S$. 
  \end{solution}

  \end{itemize}
  
\end{problem}

%%%%%%%%%%%%%%%%%%%%%%%%%%%%%%%%%%%%%%%%%%%%%%%%%%%%%%%%%%%%%%%%%%%%%
% Problem ends here
%%%%%%%%%%%%%%%%%%%%%%%%%%%%%%%%%%%%%%%%%%%%%%%%%%%%%%%%%%%%%%%%%%%%%

\endinput
