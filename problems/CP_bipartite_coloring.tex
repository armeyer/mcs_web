\documentclass[problem]{mcs}

\begin{pcomments}
  \pcomment{CP_bipartite_coloring}
  \pcomment{by student Piazza 5/19/14: probably from FTL}
  \pcomment{added by ARM 5/19/14}
\end{pcomments}

\pkeywords{
 coloring
 2-coloring
 bipartite
 connected
 components
}

%%%%%%%%%%%%%%%%%%%%%%%%%%%%%%%%%%%%%%%%%%%%%%%%%%%%%%%%%%%%%%%%%%%%%
% Problem starts here
%%%%%%%%%%%%%%%%%%%%%%%%%%%%%%%%%%%%%%%%%%%%%%%%%%%%%%%%%%%%%%%%%%%%%

\begin{problem}
Suppose an $n$-vertex bipartite graph has exactly $k$ connected
components, each of which has two or more vertices.  How many ways are
there color it using a given set of two colors?

\begin{solution}
$\mathbf{2^{n-k}}$.

The key is notice that there are exactly $2^{m-1}$ ways to color a
connected bipartite graph with $m>0$ vertices, because once one of the
the two colors is chosen for a vertex, the colors of all the other
vertices in a 2-coloring are uniquely determined.  So if there are $k$
components, and the $i$th one has size $m_i$, then the number of
colorings is
\[
\prod_1^k 2^{m_i-1} = \prod_1^k 2^{m_i} = 2^{-k}2^{\sum_1^k m_i} = 2^{-k}2^n = 2^{n-k}.
\]
\end{solution}

\end{problem}

%%%%%%%%%%%%%%%%%%%%%%%%%%%%%%%%%%%%%%%%%%%%%%%%%%%%%%%%%%%%%%%%%%%%%
% Problem ends here
%%%%%%%%%%%%%%%%%%%%%%%%%%%%%%%%%%%%%%%%%%%%%%%%%%%%%%%%%%%%%%%%%%%%%

\endinput
