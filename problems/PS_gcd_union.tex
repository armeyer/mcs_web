\documentclass[problem]{mcs}

\begin{pcomments}
  \pcomment{PS_gcd_union}
  \pcomment{subsumes mistaken FP/TP_gcd_linear_combination_induction/wop, 3/16/16}
  \pcomment{ARM 3/16/16}
\end{pcomments}

\pkeywords{
  induction
  gcd
  linear_combination
}

%%%%%%%%%%%%%%%%%%%%%%%%%%%%%%%%%%%%%%%%%%%%%%%%%%%%%%%%%%%%%%%%%%%%%
% Problem starts here
%%%%%%%%%%%%%%%%%%%%%%%%%%%%%%%%%%%%%%%%%%%%%%%%%%%%%%%%%%%%%%%%%%%%%

\begin{problem}
For any set, $A$, of integers,
\[
\gcd(A) \eqdef \text{the greatest common divisor of the elements of $A$.}
\]

The following useful property of gcd's of sets is easy to take for
granted:
\begin{theorem*}
\begin{equation}
\gcd(A \union B) = \gcd(\gcd(A),\gcd(B))\tag{AuB},
\end{equation}
For all finite sets $A,B \subset \integers$.
\end{theorem*}

Theorem~(AuB) has an easy proof as a Corollary of the Unique
Factorization Theorem.  In this problem we develop a proof of the
theorem just induction and repeated use of
\inbook{Lemma~\bref{lem:gcd-hold}.\bref{gcd1}}
\inhandout{the fact that}
\begin{equation}
(d \divides a \QAND d \divides b)\ \QIFF\ d \divides \gcd(a,b). \label{gcddiv}
\end{equation}

The key to proving~(AuB) will be generalizing~(gcddiv) to finite sets.
\begin{definition*}
For any subset $A\subseteq \integers$,
\begin{equation}
d \divides A \eqdef \forall a \in A,\, d \divides a.\tag{divdef}
\end{equation}
\end{definition*}

\begin{lemma*}
\begin{equation}
d \divides A \QIFF\ d \divides \gcd(A).
\end{equation}
for all $d \in \integers$ and finite sets $A subset \integers$.b
\end{lemma*}

\bparts

\ppart\label{gcdassocpart} Prove that
\begin{equation}
\gcd(a,gcd(b,c)) = \gcd(\gcd(a,b),c) \tag{gcd-associativity}
\end{equation}
for all integers $a,b,$.

\begin{solution}
It is sufficient to prove fact that both sides of the equality have
the same divisors.  This follows by repeated use of~(gcddiv):
\begin{align*}
d \divides \gcd(a,gcd(b,c))
  & \QIFF\ d \divides a  \QAND d \divides \gcd(b,c)\\
  & \QIFF  d \divides a  \QAND (d \divides b \QAND d \divides c)
  & \QIFF (d \divides a \QAND d \divides b)  \QAND d \divides c\\
  & \QIFF d \divides \gcd(a,b) \QAND d \divides c\\
  & \QIFF d \divides \gcd(\gcd(a,b), c).\\
\end{align*} 
\end{solution}
\eparts

\medskip
From here on we write ``$a \union A$'' as an abbreviation for
``$\set{a} \union A$.''  

\bparts
\ppart\label{abCpart}  Prove that
\begin{equation}
d \divides (a \union b \union C) \QIFF\ d \divides (\gcd(a,b) \union C)\tag{abCgcd}
\end{equation}
for all $a,b,d \in \integers$, and $C\subseteq \integers$.

\begin{proof}
\begin{align*}
d \divides (a \union b \union C)
  & \QIFF (d \divides a) \QAND (d \divides b) \QAND (d \divides C)
        & \text{(def~(divdef) of divides)}\\
  & \QIFF (d \divides \gcd(a,b)) \QAND (d \divides C)
        & \text{by~(gcddiv)}\\
  & \QIFF d \divides (\gcd(a,b) \union C)
        & \text{(def~(divdef) of divides)}.
\end{align*}
\end{proof}

\ppart  Using parts~\ref{gcdassocpart} and~\ref{abCpart} prove by
induction on the size of $A$, that
\begin{equation}
d \divides \set{a} \union A} \QIFF d \divides \gcd(a,\gcd(A)),
\end{equation}
for all integers $a,d$ and finite sets $A \subset \integers$.

\begin{solution}
The induction hypothesis will be
\[
P(n) \eqdef \card{A} = n \QIMPLIES (d \divides (\set{a} \union A)} \QIFF d \divides \gcd(a,\gcd(A)),
\]
for all integers $a,d$ and finite sets $A \subset \integers$.

\inductioncase{Base cases}: ($n \leq 3$).  The case ($n=0$) holds
vacuously and the case ($n=1$) is immediate.  The case ($n=2$) follows
from the fact that $\gcd(a,b) = \gcd(\set{a,b})$.  The case ($n=3$) follows from~(gcd-associativity).

The right to left implication follow immediately from the fact that a
divisor of a divisor of $n$ is divisor of $n$.

We prove the left to right implication by induction on $\card{A}$,
using induction hypothesis
\[
P(n) \eqdef \card{A} = n \QIMPLIES~(*).
\]

\begin{proof}
\inductioncase{Base case}: $\card{A} = \emptyset$.  Immediate.

\inductioncase{Induction step}: We prove~(*) by showing that the left
and right hand sides of the equality have the same divisors, namely,
\begin{equation}
d \divides (\set{a} \union A) \QIFF\ d \divides \gcd(a,\gcd(A))\tag{**},
\end{equation}
assuming 


\begin{align*}
d \divides (\set{a} \union A)
   & \QIFF d \divides a \QAND d \divides A
        & \text{(def of $d \divides {a} \union A$)}\\
   & \QIFF d \divides a \QAND d \divides \gcd(A)
        & \text{(by~(*) for $A$)}\\
   & \QIFF d \divides \gcd(a, \gcd(A))
        & \text{(Lemma~\bref{lem:gcd-hold}.\bref{gcd3})}\\
   & 
\end{align*}
\end{proof}

\ppart Prove that $\gcd(A)$ is a linear combination of the elements in $A$.

\ppart
Prove that
\begin{equation}
\gcd(A \union B) = \gcd(\gcd(A),\gcd(B)),\tag{*}
\end{equation}  
for any finite sets $A,B$ of integers.

\begin{staffnotes}
\hint Induction on the size of $A$.
\end{staffnotes}

\begin{solution}
For the base case $\card{A}$, we need
\begin{lemma*}
\begin{equation}
\gcd(a,B) = \gcd(a,\gcd(B))\tag{**}
\end{equation}

\end{solution}

\begin{proof}
We simply show that the left hand side and the right hand side of
equation~(**) have the same divisors:
\begin{align*}
d \divides \gcd(a,B)
  & \QIFF\  d \divides a\ \QAND\ d \divides B\\
\end{align*}
\end{proof}
\end{lemma*}

\end{solution}  


\end{problem}

%%%%%%%%%%%%%%%%%%%%%%%%%%%%%%%%%%%%%%%%%%%%%%%%%%%%%%%%%%%%%%%%%%%%%
% Problem ends here
%%%%%%%%%%%%%%%%%%%%%%%%%%%%%%%%%%%%%%%%%%%%%%%%%%%%%%%%%%%%%%%%%%%%%

\endinput


To prove the left to right implication, suppose $d divides A$.

g = dq + r,    r<d

g div a, d div a implies lin(g,d) div a

a = kg,  a = jd  implies 

lin(g,d) = 
so  

d div a & d div b imply d div (lin a,b)

g-dq = r
