\documentclass[problem]{mcs}

\begin{pcomments}
  \pcomment{from: S09.cp8m, S06.cp6w}
  \pcomment{has some commented out parts that should be proofread and 
            added or removed}
%  \pcomment{}
\end{pcomments}

\pkeywords{
  gcd
  linear_combinations
  divides
  number_theory
}

%%%%%%%%%%%%%%%%%%%%%%%%%%%%%%%%%%%%%%%%%%%%%%%%%%%%%%%%%%%%%%%%%%%%%
% Problem starts here
%%%%%%%%%%%%%%%%%%%%%%%%%%%%%%%%%%%%%%%%%%%%%%%%%%%%%%%%%%%%%%%%%%%%%

\begin{problem}
  Use the fact that $\gcd(a, b)$ is an integer linear combination of $a$
  and $b$ to prove the following properties of divisibility and gcd's.
  (You may not appeal to uniqueness of prime factorization ---aka the
  Fundamental Theorem of Arithmetic ---because these properties are needed
  to \emph{prove} the Fundamental Theorem.)

\bparts

\ppart\label{ecd} Every common divisor of $a$ and $b$ divides $\gcd(a, b)$.

\solution{For some $s$ and $t$, $\gcd(a, b) = s a + tb$.  Let $c$ be a
common divisor of $a$ and $b$.  Since $c \divides a$ and $c \divides b$,
we have $a =kc, b=k'c$ so
\[
s a + t b = skc + tk'c = c(sk+tk')
\]
so $c \divides s a + t b$.}


\iffalse

\ppart $\gcd(k a, k b) = k \cdot \gcd(a, b)$ for all $k > 0$.

\solution{
We prove that each divides the other, which implies that one is $\pm$ the
other.  Since both are nonnegative, this implies they are equal.

Since $k\gcd(a,b) =k(sa+tb) = s(ka)+t(kb)$ for some $s,t$, any common
divisor of of $ka$ and $kb$ will divide $k\gcd(a,b)$, so in particular,
\[
\gcd(ka,kb) \divides k\gcd(a,b)
\]

Conversely, $\gcd(k a, k b) = s'(ka)+t'(kb) = (s'k)a+(t'k)b$ for some
$s',t'$, so is a linear combination of $a$ and $b$ and therefore,
\[
\gcd(a,b) \divides \gcd(k a, k b)
\]
}
\fi


\ppart\label{adb} If $a \divides b c$ and $\gcd(a, b) = 1$, then $a \divides c$.

\solution{
Since $\gcd(a, b) = 1$, we have $sa+tb=1$ for some $s,t$.  Multiplying by
$c$, we have
\[
sac+tbc=c
\]
but $a$ divides the second term of the sum since $a \divides b c$, and it
obviously divides the first term, and therefore it divides the sum, which
equals $c$.
}

\ppart If $p \divides ab$ for some prime, $p$, the $p \divides a$ or $p
\divides b$.

\solution{ If $p$ does not divide $a$, then since $p$ is prime, $\gcd(p,a)
  =1$.  By part~\eqref{adb}, we conclude that $p \divides b$.  }

\ppart $\gcd(a, b) = \gcd(b, \rem{a}{b})$

\solution{
Let $r = \rem{a}{b}$.

Since $r=a-qb$ for some $q$, we have that $r$ is a linear combination of
$a$ and $b$ and is therefore divisible by $\gcd(a,b)$.  So any linear
combination of $r$ and $b$ is divisible by $\gcd(a,b)$.  Hence,
\[
\gcd(a,b) \divides \gcd(b,r)
\]

Conversely, $a = qb+r$ is a linear combination of $b$ and $r$ and is
therefore divisible by $\gcd(b,r)$.  Since $\gcd(b,r)$ divides both $a$
and $b$, we conclude from part~\eqref{ecd} that
\[
\gcd(b,r) \divides \gcd(a,b).
\]
}

\eparts

\end{problem}

%%%%%%%%%%%%%%%%%%%%%%%%%%%%%%%%%%%%%%%%%%%%%%%%%%%%%%%%%%%%%%%%%%%%%
% Problem ends here
%%%%%%%%%%%%%%%%%%%%%%%%%%%%%%%%%%%%%%%%%%%%%%%%%%%%%%%%%%%%%%%%%%%%%

\endinput
