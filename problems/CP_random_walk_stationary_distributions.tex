\documentclass[problem]{mcs}

\begin{pcomments}
  \pcomment{CP_random_walk_stationary_distributions}
  \pcomment{from: S08.cp13m}
\end{pcomments}

\pkeywords{
  random_walk
  stationary_distributions
}

%%%%%%%%%%%%%%%%%%%%%%%%%%%%%%%%%%%%%%%%%%%%%%%%%%%%%%%%%%%%%%%%%%%%%
% Problem starts here
%%%%%%%%%%%%%%%%%%%%%%%%%%%%%%%%%%%%%%%%%%%%%%%%%%%%%%%%%%%%%%%%%%%%%

\begin{problem}

\bparts

\ppart
Find a stationary distribution for the random walk graph in Figure~\ref{fig:bipart}.

\begin{solution}
$d(x) = d(y) = 1/2$
\end{solution}

\begin{figure}[h]
\graphic[height=1in]{randomWalkFigs/bipart}
\caption{}
\label{fig:bipart}
\end{figure}

\ppart If you start at node $x$ in Figure~\ref{fig:bipart} and take a
(long) random walk, does the distribution over nodes ever get close to
the stationary distribution?  Explain.

\begin{solution}
No! you just alternate between nodes $x$ and $y$.
\end{solution}

\ppart Find a stationary distribution for the random walk graph in Figure~\ref{fig:stable}.

\begin{figure}[h]
\graphic[height=1in]{randomWalkFigs/stable}
\caption{}
\label{fig:stable}
\end{figure}

\begin{solution}
$d(w) = 9/19$, $d(z) = 10/19$.  You can derive this by setting $d(w) =
  (9/10)d(z)$, $d(z) = d(w) + (1/10)d(z)$, and $d(w) + d(z) = 1$.
  There is a unique solution.
\end{solution}

\ppart If you start at node $w$ Figure~\ref{fig:stable} and take a
(long) random walk, does the distribution over nodes ever get close to
the stationary distribution?  You needn't prove anything
here, just write out a few steps and see what's happening.

\begin{solution}
Yes, it does.
\end{solution}


\ppart Find a stationary distribution for the random walk graph in Figure~\ref{fig:sinky}.

\begin{figure}[h]
\graphic[height=1in]{randomWalkFigs/sinky}
\caption{}
\label{fig:sinky}
\end{figure}

\begin{solution}
There are infinitely many, with $d(b)=d(c)=0$, and $d(a)
  = p$ and $d(d) = 1-p$ for any $p$.
\end{solution}

\ppart If you start at node $b$ in Figure~\ref{fig:sinky} and take a
long random walk, the probability you are at node $d$ will be close to
what fraction?  Explain.

\begin{solution}
1/3.
\end{solution}

\eparts
\end{problem}

\endinput
