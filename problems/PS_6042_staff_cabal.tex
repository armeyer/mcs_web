\documentclass[problem]{mcs}

\begin{pcomments}
  \pcomment{PS_6042_staff_cabal}
  \pcomment{from: S09.ps1}
  \pcomment{commented out in S09}
  \pcomment{Might be better to use fixed names to avoid newcommand's.}
  \pcomment{names edited F11}
\end{pcomments}

\pkeywords{
  logic
  translating_english_statements
  predicate_calculus
  domain_of_discourse
  quantifiers
}

\newcommand{\cabal}{\mopt{cabal}}

\newcommand{\Eric}{Eric}
\newcommand{\Tom}{Tom}
\newcommand{\Albert}{Albert}
\newcommand{\Oshani}{Drew}
\newcommand{\Edmond}{Ali}
\newcommand{\Florent}{Oscar}
\newcommand{\Nick}{Nick}

%%%%%%%%%%%%%%%%%%%%%%%%%%%%%%%%%%%%%%%%%%%%%%%%%%%%%%%%%%%%%%%%%%%%%
% Problem starts here
%%%%%%%%%%%%%%%%%%%%%%%%%%%%%%%%%%%%%%%%%%%%%%%%%%%%%%%%%%%%%%%%%%%%%

\begin{problem}
A certain cabal within the Math for Computer Science course staff is
plotting to make the final exam \emph{ridiculously hard}.
(``Problem 1.  Prove the Poincare Conjecture starting from the axioms
of ZFC.  Express your answer in khipu---the knot language of the
Incas.'')  The only way to stop their evil plan is to determine
exactly who is in the cabal.  The course staff consists of seven
people:
%
\[
\set{\Eric, \Tom, \Albert, \Oshani, \Edmond, \Florent, \Nick }
\]
The cabal is a subset of these seven.  A membership roster has been
found and appears below, but it is deviously encrypted in logic
notation.  The predicate $\cabal$ indicates who is in the cabal; that is,
$\cabal(x)$ is true if and only if $x$ is a member.  Translate each
statement below into English and deduce who is in the cabal.

\begin{enumerate}%[\upshape (i)]

\item\label{eee} $\exists x \ \exists y \ \exists z \
    (x \neq y \wedge
     x \neq z \wedge
     y \neq z \wedge
     \cabal(x) \wedge \cabal(y) \wedge \cabal(z))$

\begin{solution}
A direct English paraphrase would be ``There
exist people we'll call $x,y$, and $z$, who are all different, such that
$x,y$ and $z$ are each in the cabal.''  A better version would use the
fact that there's no need in this case to give names to the people.
Namely, a better paraphrase is ``There are 3 different people in the
cabal.''  Perhaps a simpler way to say this is: ``The cabal is of size at
least 3.''
\end{solution}

\item\label{nNC} $\neg (\cabal(\text{\Nick}) \wedge \cabal(\text{\Oshani}))$

\begin{solution}
\Nick and \Oshani are not both in the cabal.
Equivalently: at least one of \Nick and \Oshani is not in the cabal.
\end{solution}

\item\label{Fall} $\cabal(\text{\Florent}) \rightarrow \forall x \ \cabal(x)$

\begin{solution}
If \Florent is in the cabal, then everyone is.
\end{solution}

\item\label{CN} $\cabal(\text{\Oshani}) \rightarrow \cabal(\text{\Nick})$

\begin{solution}
If \Oshani is in the cabal, then \Nick is also.
\end{solution}

\item\label{EAnT}
$(\cabal(\text{\Edmond}) \vee \cabal(\text{\Albert})) \rightarrow \neg \cabal(\text{\Tom})$

\begin{solution}
If either of \Edmond or \Albert is in the cabal,
then \Tom is not.  Equivalently, if \Tom \emph{is} in the cabal, the neither
\Albert nor \Edmond is.
\end{solution}

\item\label{ENnE}
$(\cabal(\text{\Edmond}) \vee \cabal(\text{\Nick})) \rightarrow \neg \cabal(\text{\Eric})$

\begin{solution}
If either of \Edmond or \Nick is in the cabal,
then \Eric is not.  Equivalently, if \Eric \emph{is} in the cabal, the
neither \Edmond nor \Nick is.  
\end{solution}
\end{enumerate}

\insolutions{So much for the translations.  We now argue that the only
cabal satisfying all six propositions above is one whose members are
exactly \Nick, \Edmond, and \Albert.

We first observe that by~\eqref{nNC}, there must be someone---either \Nick
or \Oshani---who is not in the cabal.  But if Flo were in the cabal, then
by~\eqref{Fall}, everyone would be.  So we conclude by contradiction
that:

\begin{equation}\label{nF}
\text{\Florent is not in the cabal.}
\end{equation}

Next observe that if \Oshani was in the cabal, then by~\eqref{CN}, \Nick would
be too, contradicting~\eqref{nNC}.  So by again contradiction, we conclude:
\begin{equation}\label{nC}
\text{\Oshani is not in the cabal.}
\end{equation}

Now suppose \Tom is in the cabal.  Then by~\eqref{EAnT}, \Edmond and \Albert
are not, and we already know \Florent and \Oshani are not, so only three remain
who could be in the cabal, namely, \Tom, \Nick, and \Eric.  But
by~\eqref{eee} the cabal must have at least three members, so it follows
that the cabal must consist of exactly these three.  This proves:
\begin{lemma}\label{TNE}
\text{If \Tom is in the cabal, then \Nick and \Eric are in the cabal.}
\end{lemma}

But by~\eqref{ENnE}, if \Nick is the cabal, then \Eric is not.  That is, 
\begin{lemma}\label{NnE}
\text{\Nick and \Eric cannot both be in the cabal.}
\end{lemma}
Now from Lemma~\ref{NnE} we conclude that the conclusion of
Lemma~\ref{TNE} is false.  So by contrapositive, the hypothesis of
Lemma~\ref{TNE} must also be false, namely,
\begin{equation}\label{nT}
\text{\Tom is not in the cabal.}
\end{equation}

Finally, suppose \Eric is in the cabal.  Then by~\eqref{ENnE}, \Edmond
and \Nick are not, and we already know \Florent, \Oshani and \Tom are not. So
the cabel must consist of at most two people (\Albert and \Eric). This
contradicts~\eqref{eee}, and we conclude by contradiction that
\begin{equation}\label{nE}
\text{\Eric is not in the cabal.}
\end{equation}
So the only remaining people who could be in the cabal are \Albert, \Edmond,
and \Nick.  Since the cabal must have at least three members, we conclude
that
\begin{lemma}
The only possible cabal consists of \Albert, \Edmond, and \Nick.
\end{lemma}

But we're not done yet: we haven't shown that a cabal consisting of
\Albert, \Edmond, and \Nick actually does satisfy all six conditions.  So let
$\mathcal{A} =\set{\text{\Albert}, \text{\Edmond}, \text{\Nick}}$, and let's quickly
check that $\mathcal{A}$ satisfies~\eqref{eee}--\eqref{ENnE}:

\begin{itemize}

\item $\size{A} = 3$, so $A$ satisfies~\eqref{eee}.
\item \Oshani is not in $A$, so $A$ satisfies~\eqref{nNC} and~\eqref{CN}.
\item \Florent is not in $A$, so the hypothesis of~\eqref{Fall} is false, which
means that $A$ satisfies~\eqref{Fall}.
\item Finally, \Tom and \Eric are not in $A$, so the conclusions of
both~\eqref{EAnT} and~\eqref{ENnE} are true, and so $A$
satisfies~\eqref{EAnT} and ~\eqref{ENnE}.

\end{itemize}

So now we have proved
\begin{proposition*}
$\set{\text{\Albert}, \text{\Edmond}, \text{\Nick}}$ is the \emph{unique} cabal
satisfying conditions~\eqref{eee}--\eqref{ENnE}.
\end{proposition*}}
\end{problem}

%%%%%%%%%%%%%%%%%%%%%%%%%%%%%%%%%%%%%%%%%%%%%%%%%%%%%%%%%%%%%%%%%%%%%
% Problem ends here
%%%%%%%%%%%%%%%%%%%%%%%%%%%%%%%%%%%%%%%%%%%%%%%%%%%%%%%%%%%%%%%%%%%%%
