\documentclass[problem]{mcs}

\begin{pcomments}
  \pcomment{FP_structural_induction_rational_composition_s16}
  \pcomment{variant of FP & CP_structural_induction_rational_composition}
  \pcomment{ARM 3/12/16}
  \pcomment{rational composition on S15.mid2, F15.mid2}
\end{pcomments}

\pkeywords{
  recursive
  structural_induction
  functions
  composition
  rational_function}

%%%%%%%%%%%%%%%%%%%%%%%%%%%%%%%%%%%%%%%%%%%%%%%%%%%%%%%%%%%%%%%%%%%%%
% Problem starts here
%%%%%%%%%%%%%%%%%%%%%%%%%%%%%%%%%%%%%%%%%%%%%%%%%%%%%%%%%%%%%%%%%%%%%

\providecommand{\RAF}{\text{RAF}}

\begin{problem}

\begin{definition*}
The set \RAF\ of \emph{rational functions} of one real variable is
the set of functions defined recursively as follows:

\inductioncase{Base cases:}
\begin{itemize}
\item The identity function, $\ide(r) \eqdef r$ for $r \in
  \reals$ (the real numbers), is an $\RAF$,
\item any constant function on $\reals$ is an $\RAF$.
\end{itemize}

\inductioncase{Constructor cases:}
If $f,g$ are $\RAF$'s, then so is
$f \circledast g$, where $\circledast$ is one of the operations
\begin{enumerate}%\label{RAF+*}
\item addition, $+$,
\item multiplication, $\cdot$, or
\item division $/$.
%\item the inverse function $f^{(-1)}$,\label{L:inversefunc}
%\item the composition $f \compose g$.\label{cmp}
\end{enumerate}
\end{definition*}

\bparts
\ppart Describe how to construct functions $e,f,g \in \RAF$ such that

\begin{equation}\label{ecf+gecf}
e \compose (f+g) \neq (e \compose f)+(e \compose g).
\end{equation}

\examspace[1.0in]

\begin{solution}
Saying that $e$ is ``nonlinear'' means that~\eqref{ecf+gecf} fails.
So in particular, let $e(x) = x^2$.  Then $e \in RAF$ since it is the
identity function multiplied by itself.  Let $f,g$ be the constant
function equal to 1.  Then $f \compose (g + h)$ is the constant
function equal to $(1+1)^2 = 4$, but $(e \compose f) + (e \compose f)$
is the constant function equal to $(1^2+1^2)=2$.
\end{solution}

\ppart Prove that for all real-valued functions $e,f,g$ (not just
those in \RAF):
\begin{equation}\label{ecdfcomp}
   (e \circledast f) \compose g = (e \compose g) \circledast (f \compose g),
\end{equation}

\hint $(e \circledast f)(x) \eqdef e(x) \circledast f(x)$.

\examspace[1.5in]

\begin{solution}
Equation~\eqref{ecdfcomp} follows directly from the definition of
composition and the hint:

\begin{align*}
[(e\circledast  f) \compose g](x)
 & = [e\circledast f](g(x))
       & \text{(by def of composition)}\\
 & =  e(g(x))\circledast  f(g(x))
       & \text{(by the hint)}\\
 & = (e \compose g)(x)\circledast  (f \compose g)(x)
       & \text{(by def of composition)}\\
 & = [(e \compose g)\circledast (f \compose g)](x)
       & \text{(by the hint)}.
\end{align*}
\end{solution}

\examspace

\ppart Let predicate $P(h)$ be the following predicate on functions $h
\in \RAF$:
\[
P(h) \eqdef \forall g \in \RAF.\ h \compose g \in \RAF.
\]
Prove by structural induction on the definition of \RAF\ that $P(h)$
holds for all $h \in \RAF$.

Make sure to indicate explicitly
\begin{itemize}
\item each of the base cases, and
\item each of the constructor cases.
\end{itemize}

%\examspace[3.5in]

\begin{solution}

\begin{proof}
  \inductioncase{Base case}: ($h = \ide_\reals$).  $P(h)$ holds
  because $\ide_\reals \compose g = g$ for all real-valued functions
  $g$, and so $h \in \RAF$ for all $g \in \RAF$.

  \inductioncase{Base case}: ($h = \text{a constant-function, } c$).
  $P(h)$ holds because $c \compose g = c$ for all real-valued
  functions $g$, and $c \in \RAF$ by definition.
  
  \inductioncase{Constructor cases}: We are given $h = e \circledast
  f$ for $e, f \in \RAF$.  We may assume by structural induction that
  $P(e)$ and $P(f)$ both hold, and must prove $P(h)$.
\iffalse
, that is
  \[
  \forall g \in \RAF.\, h \compose g \in \RAF.
  \]
\fi

Now by~\eqref{ecdfcomp},
\[
h \compose g = (e \compose g) \circledast (f \compose g).
\]
Also, if $g \in \RAF$, then $(e \compose g)$ is in $\RAF$ by induction
hypothesis $P(e)$.  Likewise, $(f \compose g)$ in $\RAF$.

Therefore
\[
(e \compose g) \circledast (f\compose g) \in \RAF
\]
by the Constructor rules for $\RAF$.  That is, $h \compose g \in
\RAF$.

This proves $P(h)$ for all three of the constructor cases, which
completes the proof by structural induction.  We conclude
   \[
   \forall h \in \RAF.\, P(h).
   \]
\end{proof}

\end{solution}

\eparts

\end{problem}
%%%%%%%%%%%%%%%%%%%%%%%%%%%%%%%%%%%%%%%%%%%%%%%%%%%%%%%%%%%%%%%%%%%%%
% Problem ends here
%%%%%%%%%%%%%%%%%%%%%%%%%%%%%%%%%%%%%%%%%%%%%%%%%%%%%%%%%%%%%%%%%%%%%

\endinput
