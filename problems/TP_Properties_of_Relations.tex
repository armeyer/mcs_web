\documentclass[problem]{mcs}

\begin{pcomments}
    \pcomment{Converted from ./00Convert/probs/practice3/prob8.scm
              by scmtotex and drewe
              on Thu 21 Jul 2011 12:06:00 PM EDT}
\end{pcomments}

\begin{problem}

%% type: multiple-choice
%% title: Properties of Relations

Let $R$ and $S$ be relations on the set $A$. Determine whether each of the following statements is true or false and prove your answer.
\bparts
\ppart
If $R$ is reflexive then the complement of $R$ is also reflexive.
\begin{solution}
False.

Counterexample: Let $R$ be the equality relation. 
\end{solution}
\ppart
If $R$ is symmetric then the complement of $R$ is symmetric.
\begin{solution}
True.

$\mathcal{R}$ is symmetric means: $\forall a,b \in A, a \mathcal{R} b \implies b \mathcal{R} a$.
This implies that $\forall a,b \in A, \lnot (a \mathcal{R} b) \implies \lnot(b \mathcal{R} a)$,
so the complement of a symmetric relation is symmetric.
\end{solution}
\ppart
If $R$ and $S$ are reflexive then $R \intersect S$ is also reflexive.
\begin{solution}
True.

$\forall a\in A a\mathcal{R}a \wedge \forall a \in A a\mathcal{S}a \implies \forall a \in A a(\mathcal{R}\intersect\mathcal{S})a$.

Note that the converse is \emph{not} true: if $\mathcal{R}\intersect\mathcal{S}$ is reflexive, it's possible that $\mathcal{R}$ or $\mathcal{S}$ is not.
\end{solution}
\eparts
\end{problem}

\endinput
