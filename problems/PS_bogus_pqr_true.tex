\documentclass[problem]{mcs}

\begin{pcomments}
  \pcomment{PS_bogus_pqr_true}
  \pcomment{S95/S98/F14/F16/S17.ps1}
  \pcomment{revised ARM 2/3/17 for S17.ps1}
\end{pcomments}

\pkeywords{
  logic
  formula
  proposition
  implies
}

%%%%%%%%%%%%%%%%%%%%%%%%%%%%%%%%%%%%%%%%%%%%%%%%%%%%%%%%%%%%%%%%%%%%%
% Problems start here
%%%%%%%%%%%%%%%%%%%%%%%%%%%%%%%%%%%%%%%%%%%%%%%%%%%%%%%%%%%%%%%%%%%%%

\begin{problem}
Sloppy Sam is trying to prove a certain proposition $P$.  He defines
two related propositions $Q$ and $R$, and then proceeds to prove three
implications:
\[
P\ \QIMPLIES\ Q, \qquad Q\ \QIMPLIES\ R, \qquad R\ \QIMPLIES\ P.
\]
He then reasons as follows:
\begin{quote}
If $Q$ is true, then since I proved $(Q \QIMPLIES R)$, I can conclude
that $R$ is true.  Now, since I proved $(R \QIMPLIES P)$, I can
conclude that $P$ is true.  Similarly, if $R$ is true, then $P$ is
true and so $Q$ is true.  Likewise, if $P$ is true, then so are $Q$
and $R$.  So any way you look at it, all three of $P,Q$ and $R$ are
true.
\end{quote}

\bparts
\ppart Exhibit truth tables for
\begin{equation}\tag{*}
(P\ \QIMPLIES\ Q) \QAND (Q\ \QIMPLIES\ R) \QAND (R\ \QIMPLIES\ P)
\end{equation}
and for 
\begin{equation}\tag{**}
P \QAND Q \QAND R.
\end{equation}
Use these tables to find a truth assignment for $P,Q,R$ so that~(*)
is~\true\ and~(**) is~\false.

\begin{solution}
\[
\begin{array}{ccc|rclcrclcrcl}
  P    &    Q   &    R   & (( P  &\QIMPLIES& Q) &  \QAND  & (Q &\QIMPLIES& R)) &  \LGQAND  & (R &\QIMPLIES& P)\\
\hline
\true  & \true  & \true  &       &  \true  &    &  \true  &    &  \true  &     &  \lgtrue  &    &  \true  &   \\
\true  & \true  & \false &       &  \true  &    &  \false &    &  \false &     &  \lgfalse &    &  \true  &   \\
\true  & \false & \true  &       &  \false &    &  \false &    &  \true  &     &  \lgfalse &    &  \true  &   \\
\true  & \false & \false &       &  \false &    &  \false &    &  \true  &     &  \lgfalse &    &  \true  &   \\
\false & \true  & \true  &       &  \true  &    &  \true  &    &  \true  &     &  \lgfalse &    &  \false &   \\
\false & \true  & \false &       &  \true  &    &  \false &    &  \false &     &  \lgfalse &    &  \true  &   \\
\false & \false & \true  &       &  \true  &    &  \true  &    &  \true  &     &  \lgfalse &    &  \false &   \\
\false & \false & \false &       &  \true  &    &  \true  &    &  \true  &     &  \textcolor{red}{\lgtrue}
                                                                                           &    &  \true  &
\end{array}
\]

\[
\begin{array}{ccc|rclrcl}
   P   &   Q    &    R   &   (P & \QAND  & Q) &  \LGQAND    & R\\
\hline
\true  & \true  & \true  &      & \true  &    & \lgtrue     & \\
\true  & \true  & \false &      & \true  &    & \lgfalse    & \\
\true  & \false & \true  &      & \false &    & \lgfalse    & \\
\true  & \false & \false &      & \false &    & \lgfalse    & \\
\false & \true  & \true  &      & \false &    & \lgfalse    & \\
\false & \true  & \false &      & \false &    & \lgfalse    & \\
\false & \false & \true  &      & \false &    & \lgfalse    & \\
\false & \false & \false &      & \false &    & \textcolor{red}{\lgfalse} &   
\end{array}
\]





So ~(*) is~\true\ and~(**) is~\false\ when all three of $P,Q,R$ are \false.

\end{solution}

\ppart You show these truth tables to Sloppy Sam and he says ``OK, I'm
wrong that $P,Q$ and $R$ all have to be true, but I still don't see
the mistake in my reasoning.  Can you help me understand my mistake?''
How would you explain to Sammy where the flaw lies in his reasoning?

\begin{solution}
The ``any way you look at it'' is not a sound way of reasoning.  It
really means that in all the cases that \emph{Sammy could think of},
$P,Q$ and $R$ all came out to have truth-value \true.  But each of
Sammy's cases started with the assumption that one of $P,Q$ or $R$ had
the value \true.  Sammy forgot to consider the case when \emph{none}
of $P,Q$ or $R$ take the value \true.  The truth table shows that when
none of $P,Q$ or $R$ are \true, all the implications take the value
\true.  So Sammy neglected to consider this one case where his
conclusion was mistaken.
\end{solution}
\eparts
\end{problem}

%%%%%%%%%%%%%%%%%%%%%%%%%%%%%%%%%%%%%%%%%%%%%%%%%%%%%%%%%%%%%%%%%%%%%
% Problems end here
%%%%%%%%%%%%%%%%%%%%%%%%%%%%%%%%%%%%%%%%%%%%%%%%%%%%%%%%%%%%%%%%%%%%%

\endinput

