        \problemdata       % Takes 5 *mandatory* arguments
        {set-formulas}             % latex-friendly label for the prob.
        {sets}  % The topic(s) of the problem content.
        {}               % Source (if known)
        {F01 PS1-4, S00} % Usage (list ones you are aware of).
        {}        % Last revision info (author, date).

\begin{problem}
\begin{problemparts}

The standard notation for the proposition that a set, $x$, is a member of
a set, $y$, is
\[
x \in y.
\]
Formulas built up from membership propositions of this form using logical
connectives and quantifiers are called the logical \emph{formulas of set
theory}.  For example, to say $x \neq y$ with a formula of set theory, we
could write
\[
\exists z\; (z \in x) \iff \neg (z \in y).
\]

\problempart Write a formula of set theory that means that $x$ is the
empty set.
\solution{$\forall y\; \neg (y \in x)$.}

\problempart Write a formula of set theory that means that $x \subset y$, that
is, $x$ is a \emph{proper} subset of $y$.

\solution{$(\forall z\; (z \in x) \implies z \in y)$ means that $x$ is a
subset of $y$, that is, $x \subseteq y$.  To say $x \subset y$, form the
conjuction ($\land$) of this formula that means $(x \subseteq
y)$ and the formula above that means $(x \neq y)$.}

\problempart Write a formula of set theory that means that $y$ is the powerset
of $x$.

\solution{$\forall z\; P \implies (z \in y)$, where $P$ is a formula
of set theory that means $(z \subseteq x)$.
}

\problempart Write a formula of set theory that means that $y$ has at least
three elements.

\solution{$\exists u \exists v \exists w\; (u \in y) \land (v \in y)
\land (w \in y) \land (u \neq v) \land (u \neq w) \land (v \neq w).$}

\problempart Write a formula of set theory that means that $y$ has exactly two
elements.

\solution{It is easy to write a formula, $T(y)$, that means that $y$ has
at least two elements.  The desired formula is just the conjunction of
$T(y)$ and the formula that says that $y$ does \emph{not} have at least
three elements.}

\end{problemparts}

\end{problem}