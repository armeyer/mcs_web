\documentclass[problem]{mcs}

\begin{pcomments}
  \pcomment{CP_proving_basic_congruence_properties}
  \pcomment{from: S09.cp8t, S06.cp7m}
\end{pcomments}

\pkeywords{
  modular_arithmetic
  number_theory
  divides
  congruence
  remainders
}

%%%%%%%%%%%%%%%%%%%%%%%%%%%%%%%%%%%%%%%%%%%%%%%%%%%%%%%%%%%%%%%%%%%%%
% Problem starts here
%%%%%%%%%%%%%%%%%%%%%%%%%%%%%%%%%%%%%%%%%%%%%%%%%%%%%%%%%%%%%%%%%%%%%

\begin{problem}

\bparts

The following properties of equivalence mod $n$ follow directly from
its definition and simple properties of divisibility.  See if you can
prove them without looking up the proofs in the text.

\ppart\label{multc} If $a \equiv b \pmod{n}$, then $a c \equiv b c
\pmod{n}$.

\begin{solution}
The condition $a \equiv b \pmod{n}$ is equivalent to the
assertion $n \divides (a - b)$.  This implies that $n \divides (a - b)c$,
and so $n \divides (ac - bc)$.  This is equivalent to $ac \equiv bc
\pmod{n}$.
\end{solution}

\ppart\label{transitive} If $a \equiv b \pmod{n}$ and $b \equiv c
\pmod{n}$, then $a \equiv c \pmod{n}$.

\begin{solution}
Assume $a \equiv b \pmod{n}$ and $b \equiv c \pmod{n}$, that is,
$n \divides (a - b)$ and $n \divides (b - c)$.  Then $n \divides (a - b) +
(b - c) = (a-c)$, so $a \equiv c \pmod{n}$.
\end{solution}

\ppart If $a \equiv b \pmod{n}$ and $c \equiv d \pmod{n}$, then $a c
\equiv b d \pmod{n}$.

\begin{solution}
$a \equiv b \pmod{n}$ implies $ac \equiv bc \pmod{n}$ by
part~\eqref{multc}; likewise, $c \equiv d \pmod{n}$ implies $bc \equiv bd
\pmod{n}$.  So $a c \equiv b d \pmod{n}$ by part~\eqref{transitive}.
\end{solution}

\ppart $\rem{a}{n} \equiv a \pmod{n}$.

\begin{solution}
The remainder $\rem{a}{n}$ is equal to $a - qn$ for some integer
$q$.  However, for every integer $q$:
\begin{align*}
n \divides qn & \QIFF  n \divides ((a - qn) - a) \\
              & \QIMPLIES  n \divides (\rem{a}{n} - a)\\
              & \QIFF  \rem{a}{n} \equiv a \pmod{n}.
\end{align*}
\end{solution}
\eparts

\end{problem}

%%%%%%%%%%%%%%%%%%%%%%%%%%%%%%%%%%%%%%%%%%%%%%%%%%%%%%%%%%%%%%%%%%%%%
% Problem ends here
%%%%%%%%%%%%%%%%%%%%%%%%%%%%%%%%%%%%%%%%%%%%%%%%%%%%%%%%%%%%%%%%%%%%%

\endinput
