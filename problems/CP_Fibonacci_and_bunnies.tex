\documentclass[problem]{mcs}

\begin{pcomments}
  \pcomment{CP_Fibonacci_and_bunnies}
  \pcomment{from: S09.ps10}
\end{pcomments}

\pkeywords{
  generating_functions
  Fibonacci
  linear_recurrence
  partial_fraction
}

%%%%%%%%%%%%%%%%%%%%%%%%%%%%%%%%%%%%%%%%%%%%%%%%%%%%%%%%%%%%%%%%%%%%%
% Problem starts here
%%%%%%%%%%%%%%%%%%%%%%%%%%%%%%%%%%%%%%%%%%%%%%%%%%%%%%%%%%%%%%%%%%%%%

\begin{problem}

  The famous mathematician, Fibonacci, has decided to start a rabbit farm
  to fill up his time while he's not making new sequences to torment
  future college students.  Fibonacci starts his farm on month zero (being
  a mathematician), and at the start of month one he receives his first
  pair of rabbits.  Each pair of rabbits takes a month to mature, and
  after that breeds to produce one new pair of rabbits each month.
  Fibonacci decides that in order never to run out of rabbits or money,
  every time a batch of new rabbits is born, he'll sell a number of newborn
  pairs equal to the total number of pairs he had three months earlier.
  Fibonacci is convinced that this way he'll never run out of stock.

\bparts

\ppart Define the number, $r_n$, of pairs of rabbits Fibonacci has in
month $n$, using a recurrence relation.  That is, define $r_n$ in terms of
various $r_i$ where $i<n$.

\begin{solution}
  According to the description above, $r_0 = 0$ and $r_1 = 1$.  Since the
  rabbit pair received at the first month is too young to breed, $r_2=1$
  as well.  After that, $r_n$ is equal to the number, $r_{n-1}$, of rabbit
  pairs in the previous month, plus the number of newborn pairs, minus the
  number, $r_{n-3}$, he sells.  The number of newborn pairs equals to the
  number of breeding pairs from the previous month, which is precisely the
  total number, $r_{n-2}$, of pairs from two months before.

  Thus,
\[
r_n = r_{n-1} + (r_{n-2} - r_{n-3}).
\]
\end{solution}

\ppart Let $R(x)$ be the generating function for rabbit pairs,
\[
R(x) \eqdef r_0+r_1x+r_2x^2+\cdots
\]
Express $R(x)$ as a quotient of polynomials.

\begin{solution}
  Reasoning as in the derivation of the generating
  function for the orginal Fibonacci numbers, we have
\[
\begin{array}{rcrcrcrcrcr}
R(x)     & = & r_0 & + & r_1  x & + & r_2 x^2 & + & r_3 x^3 & + & r_4 x^4 + \cdots.\\
-xR(x)   & = &     & - & r_0  x & - & r_1 x^2 & - & r_2 x^3 & - & r_3 x^4 - \cdots.\\
-x^2R(x) & = &     &   &        & - & r_0 x^2 & - & r_1 x^3 & - & r_2 x^4 - \cdots.\\
 x^3R(x) & = &     &   &        &   &         & + & r_0 x^3 & + & r_1 x^4 + \cdots.
\end{array}
\]
so
\[
\begin{array}{rcrcrcrcrcr}
\lefteqn{R(x)(1-x-x^2+x^3)}\\
         & = & r_0 & + & (r_1-r_0)
                              x & + & (r_2 - r_1 - r_0)
                                          x^2 & + & 0   x^3 & + & 0   x^4 + \cdots\\
         & = &  0  & + & 1    x & + & 0   x^2,
\end{array}
\]
and
\begin{equation}\label{Rfracx1}
R(x) = \frac{x}{1 - x - x^2 + x^3} = \frac{x}{(1+x)(1-x)^2}\, .
\end{equation}
\end{solution}

\ppart Find a partial fraction decomposition of the generating function
$R(x)$.

\begin{solution}
We know
\[
R(x) = \frac{A}{1+x} + \frac{B}{1-x} + \frac{C}{(1-x)^2}
\]
for some numbers $A, B, C$.  Multiplying both sides of this equation by
$(1+x)(1-x)^2$ gives
\[
x = A(1-x)^2 + B(1+x)(1-x) + C(1+x).
\]
Letting $x=1$ gives $C = 1/2$, letting $x=-1$ gives $A= -1/4$, and letting
$x = 0$ then gives $B = -(A+C) = -1/4$.
\end{solution}

\ppart Finally, use the partial fraction decomposition \iffalse and the
rules outlined in~Ch.\bref{generating_function_chap}\fi to come up with a
closed form expression for the number of pairs of rabbits Fibonacci has on
his farm on month $n$.

\begin{solution}
  We find the coefficient as the sum of the coefficients for each term in
  the partial fraction expansion.
\[
\begin{array}{rcrccrcccr}
A/(1+x)  & = & - & 1/4 & - & (1/4)  (-x)  - & \cdots & - & (1/4)       (-x)^n - \cdots,\\
B/(1-x)  & = & - & 1/4 & - & (1/4)  x     - & \cdots & - & (1/4)       x^n - \cdots,\\
C/(1-x)^2& = &   & 1/2 & + & (2/2)  x     + & \cdots & + & ((n+1)/2)   x^n + \cdots,
\end{array}
\]
so
\[
R(x) = 1 x + 1 x^2  + \cdots +  \paren{\frac{n+1}{2} - \frac{(-1)^n +1}{4}}x^n + \cdots,
\]
and
\[
r_n = \ceil{\frac{n}{2}}\, .
\]
\end{solution}

\eparts

\end{problem}

%%%%%%%%%%%%%%%%%%%%%%%%%%%%%%%%%%%%%%%%%%%%%%%%%%%%%%%%%%%%%%%%%%%%%
% Problem ends here
%%%%%%%%%%%%%%%%%%%%%%%%%%%%%%%%%%%%%%%%%%%%%%%%%%%%%%%%%%%%%%%%%%%%%

\endinput
