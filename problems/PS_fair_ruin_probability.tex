\documentclass[problem]{mcs}

\begin{pcomments}
  \pcomment{PS_fair_ruin_probability}   
  \pcomment{DRAFT: from S02 Notes}
\end{pcomments}

\pkeywords{
  probability
  expectation
  random_walk
  gamblers_ruin
  infinite_linearity
  fair
}

%%%%%%%%%%%%%%%%%%%%%%%%%%%%%%%%%%%%%%%%%%%%%%%%%%%%%%%%%%%%%%%%%%%%%
% Problem starts here
%%%%%%%%%%%%%%%%%%%%%%%%%%%%%%%%%%%%%%%%%%%%%%%%%%%%%%%%%%%%%%%%%%%%%

\begin{problem}
Let's begin by considering the case of a fair coin, that is, $p = 1/2$, and
determine the probability, $w$, that the gambler wins.  We can handle this
case by considering the expectation of the random variable $G$ equal to the
gambler's dollar gain.  That is, $G = T-n$ if the gambler wins, and $G= -n$
if the gambler loses, so
\[
\expect{G} = w(T-n) - (1-w)n = wT - n.
\]
Notice that we're using the fact that the only outcomes are those in
which the gambler wins or loses ---there are no infinite games ---so
the probability of losing is $1-w$.

Now let $G_{i}$ be the amount the gambler gains on the $i$th flip:
$G_{i}=1$ if the gambler wins the flip, $G_{i}=-1$ if the gambler loses
the flip, and $G_{i}=0$ if the game has ended before the $i$th flip.
Since the coin is fair, $\expect{G_{i}}=0$.

The random variable $G$ is the sum of all the $G_{i}$'s, so by linearity of
expectation\footnote{We've been stung by paradoxes in this kind of
situation, so we should be careful to check that the condition for infinite
linearity of expectation is satisfied.  Namely, we have to check that
$\sum_{i=1}^\infty \expect{\abs{G_i}}$ converges.

In this case, $\abs{G_i} = 1$ iff the walk is of length at least $i$, and
$\abs{G_i} = 0$ otherwise.  So
\[
\expect{\abs{G_i}} = \pr{\text{the walk is of length } \geq i}.
\]

But we show in an in-class problem that there is a constant $\epsilon >0$
such that
\[
\pr{\text{the walk is of length } \geq i} \leq (1 - \epsilon)^i.
\]
So the $\sum_{i=1}^\infty \expect{\abs{G_i}}$ is bounded term-by-term by a
convergent geometric series, and therefore it also converges.}
\[
wT - n= E(G) = \sum_{i=1}^{\infty}E(G_{i}) = 0,
\]
which proves
\begin{theorem}\label{fairwinthm}
In the unbiased Gambler's Ruin game with probability $p=1/2$ of winning
each individual bet, with initial capital, $n$, and goal, $T$,
\begin{equation}\label{fairwin}
\pr{\text{the gambler is a winner}} = \frac{n}{T}.
\end{equation}
\end{theorem}
\end{problem}

%%%%%%%%%%%%%%%%%%%%%%%%%%%%%%%%%%%%%%%%%%%%%%%%%%%%%%%%%%%%%%%%%%%%%
% Problem ends here
%%%%%%%%%%%%%%%%%%%%%%%%%%%%%%%%%%%%%%%%%%%%%%%%%%%%%%%%%%%%%%%%%%%%%


\endinput
