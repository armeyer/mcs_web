\documentclass[problem]{mcs}

\begin{pcomments}
  \pcomment{CP_isomorphic_graphs}
  \pcomment{from: S09.cp6m, S06.cp5w}
\end{pcomments}

\pkeywords{
  isomorphism
  isomorphic
  digraph
}

%%%%%%%%%%%%%%%%%%%%%%%%%%%%%%%%%%%%%%%%%%%%%%%%%%%%%%%%%%%%%%%%%%%%%
% Problem starts here
%%%%%%%%%%%%%%%%%%%%%%%%%%%%%%%%%%%%%%%%%%%%%%%%%%%%%%%%%%%%%%%%%%%%%

\begin{problem}
For each of the following pairs of simple graphs, either define an
isomorphism between them, or prove that there is none.  (We write $ab$
as shorthand for $\edge{a}{b}$.)

\bparts

\ppart
\begin{align*}
G_1\text{ with }& V_1 = \set{1,2,3,4,5,6},\ E_1= \set{12,23,34,14,15,35,45}\\
G_2\text{ with }& V_2 = \set{1,2,3,4,5,6},\ E_2= \set{12,23,34,45,51,24,25}
\end{align*}

\begin{solution}

Not isomorphic: $G_2$ has a node, 2, of degree 4, but the maximum degree
in $G_1$ is 3.

So if as in this example, two graphs have different degree
``spectra,'' then they are clearly not isomorphic.  On the other hand,
there are simple examples of non-isomorphic graphs with the same
degree spectra.  For example,
\begin{itemize}
\item a (disconnected) graph consisting of two triangles, and
\item  a hexagon, 
\end{itemize}
are non-isomorphic graphs, each with six vertices all of which have
degree two.

Even without finding such a simple counterexample, you could have
concluded that it would be too good to be true for degree spectra to
determine isomorphism.  If having the same degree spectra implied
isomorphism, the isomorphism would be easy to check.  But it is a long
standing open problem to find an easy way to verify isomorphism.
\end{solution}

\ppart
\begin{align*}
G_3\text{ with } &V_3 = \set{1,2,3,4,5,6},\ E_3= \set{12,23,34,14,45,56,26}\\
G_4\text{ with } &V_4 = \set{a,b,c,d,e,f},\ E_4= \set{ab,bc,cd,de,ae,ef,cf}
\end{align*}

\begin{solution}
Isomorphic (four isomorphisms) with the vertex correspondences: \\
$1f, 2c, 3d, 4e, 5a, 6b$ \\
or $1f, 2e, 3d, 4c, 5b, 6a$ \\
or $1d, 2c, 3f, 4e, 5a, 6b$ \\
or $1d, 2e, 3f, 4c, 5b, 6a$

\end{solution}

\eparts

\end{problem}

%%%%%%%%%%%%%%%%%%%%%%%%%%%%%%%%%%%%%%%%%%%%%%%%%%%%%%%%%%%%%%%%%%%%%
% Problem ends here
%%%%%%%%%%%%%%%%%%%%%%%%%%%%%%%%%%%%%%%%%%%%%%%%%%%%%%%%%%%%%%%%%%%%%
\endinput
