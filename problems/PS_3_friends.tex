\documentclass[problem]{mcs}

\begin{pcomments}
  \pcomment{PS_3_friends}
  \pcomment{combinatorial proof of friends-and-strangers theorem}
  \pcomment{CH, Spring '14, edited ARM 4/26/14}
\end{pcomments}

\pkeywords{
  counting
  degree
  complete_graph
  pigeon_hole
  ramsey
  friends
  strangers
}

%%%%%%%%%%%%%%%%%%%%%%%%%%%%%%%%%%%%%%%%%%%%%%%%%%%%%%%%%%%%%%%%%%%%%
% Problem starts here
%%%%%%%%%%%%%%%%%%%%%%%%%%%%%%%%%%%%%%%%%%%%%%%%%%%%%%%%%%%%%%%%%%%%%

\begin{problem}
The \emph{complete graph}, $K_n$, is a simple graph with $n$ vertices
and an edge between every pair of vertices.   We'll refer to length-3
cycles as \emph{triangles}.

\bparts

\ppart\label{20trik6} Show that there are exactly twenty different
triangles in $K_6$.

\begin{solution}
Every set of three vertices in $K_n$ determines a triangle, so there are
\[
\binom{6}{3} = 20
\]
triangles.
\end{solution}

\examspace[0.75in]

\ppart\label{6edgepair} Now suppose that every edge in $K_6$ is
colored either red or blue.  A \emph{2-color edge set} is a set of two
edges of different colors that have a vertex in common; this vertex is
called the \emph{center} of the edge set.  Show that at most six
2-color edge sets can have the same center.

\hint A vertex incident to $r$ red edges and $b$ blue edges is the
center of $r \cdot b$ 2-color edge sets.

\begin{solution}
If vertex $v$ is incident to $r$ red edges and $b$ blue edges, then it
is the center of exactly $r \cdot b$ 2-color edge sets.  But the
degree of $v$ is five, so $r+b =5$.  Under this constraint, it's easy
to see that the maximum value of $r \cdot b$ is six.
\end{solution}

\examspace[1.0in]

\ppart A triangle is \emph{2-colored} if it has two edges of different
colors.  Describe a 2-to-1 mapping between the set of 2-color edge
sets and the set of 2-colored triangles.  Conclude that there are at
most eighteen 2-colored triangles.

\begin{solution}
The edges in a two 2-color edge set are edges of a unique 2-color
triangle, so map the set to the triangle.  Since the edge set of a
2-color triangle contains exactly two 2-color edge sets, this mapping
is 2-to-1.

This implies that the number of 2-colored triangles is exactly half
the number of 2-color edge sets.

Now by part~\eqref{6edgeset}, each vertex is the center of at most
six 2-color edge sets, and since there are six vertices, there are at
most $6 \cdot 6$ 2-color edge sets and hence at most $(6 \cdot 6)/2 =
18$ 2-colored triangles.
\end{solution}

\examspace[1.5in]

\ppart If every pair of people in a group are friends, or if every
pair are strangers, the group is called \emph{uniform}.  Use the facts
established above to prove the following statement:

\begin{quote}
Every set of six people includes \emph{two} uniform three-person
groups.
\end{quote}

\begin{solution}
Label the vertices of $K_6$ with the names of the 6 people.  Color the
edge between two people red if they are friends and blue if they are
strangers.  Then, the statement is equivalent to the claim that there
are at least two \emph{monochromatic} triangles, that is, triangles
that are \emph{not} 2-colored.

But we know there are twenty triangles, and of these at most eighteen
are 2-colored, so there must be at least $20 - 18 = 2$ monochromatic
triangles.
\end{solution}

\eparts

\end{problem}

%%%%%%%%%%%%%%%%%%%%%%%%%%%%%%%%%%%%%%%%%%%%%%%%%%%%%%%%%%%%%%%%%%%%%
% Problem ends here
%%%%%%%%%%%%%%%%%%%%%%%%%%%%%%%%%%%%%%%%%%%%%%%%%%%%%%%%%%%%%%%%%%%%%

\endinput
