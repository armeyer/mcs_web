\documentclass[problem]{mcs}

\begin{pcomments}
  \pcomment{PS_3_friends}
  \pcomment{combinatorial proof of friends-and-strangers theorem}
  \pcomment{CH, Spring '14, edited ARM 4/26/14}
\end{pcomments}

\pkeywords{
  counting
  degree
  complete_graph
  pigeon_hole
  ramsey
  friends
  strangers
}

%%%%%%%%%%%%%%%%%%%%%%%%%%%%%%%%%%%%%%%%%%%%%%%%%%%%%%%%%%%%%%%%%%%%%
% Problem starts here
%%%%%%%%%%%%%%%%%%%%%%%%%%%%%%%%%%%%%%%%%%%%%%%%%%%%%%%%%%%%%%%%%%%%%

\begin{problem}
The \emph{complete graph}, $K_n$, is a simple graph with $n$ vertices
and an edge between every pair of vertices.   We'll refer to length-3
cycles as \emph{triangles}.

\bparts

\ppart\label{20trik6} Show that the number of triangles in $K_6$
equals 20.

\begin{solution}
Every set of three vertices in $K_n$ determines a triangle, so there are
\[
\binom{6}{3} = 20
\]
triangles.
\end{solution}

\examspace[1.5in]

\ppart\label{6edgepair} Now suppose that every edge in $K_6$ is
colored either red or blue.  A \emph{2-color edge pair} is a set of
two edges of different colors that have a vertex in common.  Show that
the number of 2-color pairs which share a given vertex is at most 6.

\begin{solution}
Let $v$ be the given vertex. Suppose $v$ is incident to some number,
$r$, of red edges and $b$ blue edges.  Then the number of 2-color edge
pairs for which $v$ is the shared vertex is exactly $r \cdot b$.  But
$5 = \degr{v} = r+b$, and under this constraint, it's eadsy to see
that the maximum value of $r \cdot b$ is 6.
\end{solution}

\examspace[1.5in]

\ppart A triangle is \emph{2-colored} if it has two edges of different
colors.  Describe a 2-to-1 mapping between the set of the set of all
2-color edge pairs and the set of all 2-colored triangles.  Conclude
that there are at most eighteen 2-colored triangles.

\begin{solution}
The edges in a two 2-color edge pair are edges of a unique 2-color
triangle, so map the pair to the triangle.  Since the edge set of a
2-color triangle contains exactly two 2-color edge pairs, this mapping
is 2-to-1.

This implies that the number of 2-colored triangles is exactly half
the number of 2-color edge pairs.

Now by part~\eqref{6edgepair}, each vertex is the shared vertex of at
most six 2-color edge pairs, and since there are six vertices, there
are at most $6 \cdot 6$ 2-color edge pairs and hence at most $(6 \cdot
6)/2 = 18$ 2-colored triangles.
\end{solution}

\examspace[1.5in]

\ppart Use the facts proved above to prove the following statement:

\begin{quote}
Every collection of 6 people always includes a group of 3 mutual
acquaintances, or a group of 3 mutual strangers.
\end{quote}

\begin{solution}
Label the vertices of $K_6$ with the names of the 6 people.  Color the
edge between two people red if they are acquainted and blue if they
are strangers.  Then, the statement is equivalent to the claim that
there is always includes a \emph{monochromatic} triangle, that is, a
triangle that is \emph{not} 2-colored.

But we know there are twenty triangles of which at most are 2-colored,
so there must be at least $20 -18 = 2$ monochromatic triangles.  So
not only is there a group of three people who are mutually all friends
or all strangers, there are at least two such groups.
\end{solution}

\eparts

\end{problem}

%%%%%%%%%%%%%%%%%%%%%%%%%%%%%%%%%%%%%%%%%%%%%%%%%%%%%%%%%%%%%%%%%%%%%
% Problem ends here
%%%%%%%%%%%%%%%%%%%%%%%%%%%%%%%%%%%%%%%%%%%%%%%%%%%%%%%%%%%%%%%%%%%%%

\endinput
