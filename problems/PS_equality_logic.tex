\documentclass[problem]{mcs}

\begin{pcomments}
  \pcomment{PS_equality_logic}
  \pcomment{generalizes MQ_equality_logic}
  \pcomment{ARM 2/4/16}
\end{pcomments}

\pkeywords{
  predicate
  equality
  logic
}

%%%%%%%%%%%%%%%%%%%%%%%%%%%%%%%%%%%%%%%%%%%%%%%%%%%%%%%%%%%%%%%%%%%%%
% Problem starts here
%%%%%%%%%%%%%%%%%%%%%%%%%%%%%%%%%%%%%%%%%%%%%%%%%%%%%%%%%%%%%%%%%%%%%

\begin{problem}
Predicate Formulas whose only predicate symbol is equality are called
``pure equality'' formulas.  For example,
\begin{equation}
\forall x\, \forall y.\ x = y\tag{1-element}
\end{equation}
is a pure equality formula.  Its meaning is that there is exactly one
element in the domain of discourse.\footnote{Remember, a domain of
  discourse is not allowed to be empty.}

Another pure equality formula is
\begin{equation}
\exists a_1\, \exists a_2\, \exists a_2\, \forall x.\ x=a_1 \QOR x =
a_2 \QOR x = a_3. \tag{$\leq 3$-elements}
\end{equation}
Its meaning is that there are at most three elements in the domain of
discourse.

The formula
\begin{equation}\label{neqabbrev}
\exists x\, \exists y.\ x \neq y
\end{equation}
uses the symbol ``$\neq$,'' which is not allowed in pure formulas.
But $x \neq y$ is just an abbreviation for the pure equality formula
\[
\QNOT(x = y),
\]
so we'll regard~\eqref{neqabbrev} as a pure equality formula, even
though technically it is only an abbreviation for a pure formula.

On the other hand, the formula
\begin{equation}
x \leq y.\tag{\text{not-allowed}}
\end{equation}
is \emph{not} an equality formula because it uses the
less-than-or-equal predicate $\leq$.  Moreover,
formula~(not-allowed) only makes sense when the domain
elements are ordered, while pure equality formulas make sense over
every domain, so~(not-allowed) is not an abbreviation for any
pure equality formula.

\bparts

\ppart For each $n \in \naturals$, show a pure equality formula $L_n$
that means that there are at most $n$ elements in the domain of
discourse.

\examspace[2.0in]

\begin{solution}
\[
L_n \eqdef \exists a_1\, \exists a_2 \dots \exists a_n\, \forall
x.\ x=a_1 \QOR x = a_2 \QOR \dots \QOR x = a_n.
\]
\end{solution}

\ppart Show a pure equality formula $X_n$ that means that there are
\emph{exactly} $n$ elements in the domain of discourse.

\examspace[2.0in]

\begin{solution}
There are exactly $n$ elements iff there are at least $n$ but not at least $n+1$:
\[
X_n \eqdef L_n \QAND\ \QNOT(L_{n+1}).
\]

A more long-winded alternative is
\[
\exists a_1,\dots,\ a_n.\
[\LGQAND_{1 \leq i < j \leq n} a_i \neq a_j\ \QAND  \forall x.\, \LGQOR_{1 \leq i \leq n} x = a_i].
\]
\end{solution}

\begin{staffnotes}

ADD MORE?  Use quantifier elimination to show every pure equality
formula is equavalent to a quantifier-free pure formula.

\end{staffnotes}

\eparts

\end{problem}

\endinput
