\documentclass[problem]{mcs}

\begin{pcomments}
  \pcomment{Based off FP_validity_by_cases}
  \pcomment{Edited by dganelin Fall 2017}
\end{pcomments}

\pkeywords{
  validity
  implication
  truth table
  proof by cases
}

%%%%%%%%%%%%%%%%%%%%%%%%%%%%%%%%%%%%%%%%%%%%%%%%%%%%%%%%%%%%%%%%%%%%%
% Problem starts here
%%%%%%%%%%%%%%%%%%%%%%%%%%%%%%%%%%%%%%%%%%%%%%%%%%%%%%%%%%%%%%%%%%%%%
\begin{problem}

The formula

\noindent{(NOT $A$) OR (NOT $B$) OR (NOT $C$ OR NOT $E$ OR NOT $F$)  AND (NOT $F$ OR $G$)  OR (NOT $H$)   AND (NOT $I$)  AND (NOT $J$)  OR $K$}

IMPLIES 

\noindent{NOT (NOT $X$ AND NOT $Y$ AND NOT($X$ IMPLIES $Z$)) }

turns out to be valid.

\begin{enumerate}
\item{Laborious Lucy decides to verify this formula by making a truth table. How many rows will she need in her truth table? (You may give a close approximation.)}
\item{You decide to be a little more efficient. Verify that the formula is valid by reasoning by cases according to the value of $X$.}
\end{enumerate}



\begin{solution}

\begin{enumerate}
\item{Since there are $14$ variables, there are $2^{14} =  16384$ rows}.
\item{If $X$ is True, then NOT $X$ is False, so the right side of the main implication is True, which makes the statement True.
If $X$ is False, then $X$ IMPLIES $Z$ is True, so NOT($X$ IMPLIES $Z$) is False, so the right side of the main implication is True, which makes the statement True.}
\end{enumerate}

\end{solution}

\end{problem}

%%%%%%%%%%%%%%%%%%%%%%%%%%%%%%%%%%%%%%%%%%%%%%%%%%%%%%%%%%%%%%%%%%%%%
% Problem ends here
%%%%%%%%%%%%%%%%%%%%%%%%%%%%%%%%%%%%%%%%%%%%%%%%%%%%%%%%%%%%%%%%%%%%%
\endinput