\documentclass[problem]{mcs}

\begin{pcomments}
  \pcomment{FP_sum_of_odd_squares_ver1_f15}
  \pcomment{a variant of FP_sum_of_odd_squares}
  \pcomment{ARM 10/29/15, ZDz 11/4/15}
\end{pcomments}

\pkeywords{
  induction
  sum
  squares
}

\begin{problem}
We are interesed in finding a closed form formula for the sum
\begin{equation}\label{sumoddsq}
\sum_{i=1}^n (3i-2)^2.
\end{equation}
Display such a closed form formula for~\eqref{sumoddsq}.
\textbf{Alternatively}, you can get \emph{full credit by giving a
  clear description} of a procedure for deriving such a closed formula
without fully executing your procedure to derive an explicit closed
formula.

\begin{solution}
As in Section~\bref{sec:sum_powers}, we can (correctly) guess that a
formula for~\eqref{sumoddsq} will be a cubic polynomial
\[
an^3 + bn^2 + cn + d.
\]
Then we can determine the parameters $a$, $b$ $c$, and $d$ by
plugging in a few values for $n$ until we get enough equations in
$a,b,c,d$ to solve for their values.  Applying this method to our
example gives:
\begin{align*}
n = 1 & \qimplies  1 = a + b + c + d \\
n = 2 & \qimplies  17 = 8a + 4b + 2c + d \\
n = 3 & \qimplies  66 = 27a + 9b + 3c + d \\
n = 4 & \qimplies  166 = 64a + 16b + 4c + d.
\end{align*}
Solving this system gives the solution $a = 3, b = -3/2, c = -1/2, d = 0$
So the closed form formula for~\eqref{sumoddsq} would be
\begin{equation}\label{4n312n2}
\frac{6n^3 - 3n^2 - n}{2}.
\end{equation}
Of course this remains a guess until we have verified it somehow, for
example, by induction.

Another approach is to rewrite $(3i-2)^2$ as $9i^2-12i +4$ and then use
the known formula for the sum of squares\inbook{~(\bref{eqn:G27})}:
\[
\sum_{i=0}^n i^2 = \frac{(2n+1) (n+1) n}{6}\, .
\]
So,
\begin{align*}
\sum_{i=1}^n (3i-2)^2
   & = \sum_{i=1}^n (9i^2 - 12i + 4)\\
   & = 9 \sum_{i=1}^n i^2 - 12\sum_{i=1}^n i  + \sum_{i=1}^n 4\\
   & = 9 \frac{(2n+1) (n+1) n}{6} - 12 \frac{n(n+1)}{2} + 4n\\
   & = \frac{3}{2}(2n^3+ 3n^2 +n) - 6n^2 - 2n \\
   & = \frac{6n^3+ 9n^2 + 3n}{2} - \frac{12n^2 + 4n}{2}\\
   & = \frac{6n^3 - 3n^2 - n}{2},
\end{align*}
which agrees with~\eqref{4n312n2}.

A third approach involves using generating functions as described in
Chapter~\bref{generating_function_chap}, but let's not get ahead of
ourselves.

\end{solution}
\end{problem}

\endinput
