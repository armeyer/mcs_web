\documentclass[problem]{mcs}

\begin{pcomments}
  \pcomment{PS_prime_divides_integer_product}
  \pcomment{from: S09.ps4}
  \pcomment{from: F08.ps4}
  \pcomment{from: F07.ps3}
\end{pcomments}

\pkeywords{
  false_proof
  primes
  induction
  divides
  strong_induction
}

%%%%%%%%%%%%%%%%%%%%%%%%%%%%%%%%%%%%%%%%%%%%%%%%%%%%%%%%%%%%%%%%%%%%%
% Problem starts here
%%%%%%%%%%%%%%%%%%%%%%%%%%%%%%%%%%%%%%%%%%%%%%%%%%%%%%%%%%%%%%%%%%%%%

\begin{problem}
  The following Lemma is true, but the \emph{proof} given for it below is
  defective.  Pinpoint \emph{exactly} where the proof first makes an
  unjustified step and explain why it is unjustified.

\begin{lemma}
For any prime $p$ and positive integers $n, x_1, x_2,\ldots, x_n$, if
$p \divides x_1x_2\dots x_n$, then $p \divides x_i$ for some $1\leq i\leq n$.
\end{lemma}

\begin{falseproof}
Proof by strong induction on $n$.

\textbf{Base case} $n= 1$: When $n=1$, we have $p\divides x_1$, therefore
we can let $i=1$ and conclude $p\divides x_i$.

\textbf{Induction step}: Now assuming the claim holds for all $k\leq n$, we must
prove it for $n+1$.

So suppose $p\divides x_1x_2\ldots x_{n+1}$.  Let $y_n = x_n x_{n+1}$, so
$x_1x_2\ldots x_{n+1} = x_1x_2\ldots x_{n-1}y_n$.  Since the righthand
side of this equality is a product of $n$ terms, we have by induction that
$p$ divides one of them.  If $p\divides x_i$ for some $i < n$, then we have the
desired $i$.  Otherwise $p\divides y_n$.  But since $y_n$ is a product of the two
terms $x_n, x_{n+1}$, we have by strong induction that $p$ divides one of
them.  So in this case $p \divides  x_i$ for $i = n$ or $i = n+1$.
\end{falseproof}

\begin{solution}
Notice that nowhere in the proof is the fact that $p$ is prime
used.  So if this proof were correct, the Lemma would hold not just for
prime $p$, but for any positive integer $p$.  But of course, the Lemma is
false when $p$ is not prime, for example if $p=6$, $x_1=3$ and $x_2=4$, we
have $p\divides x_1x_2$ but $\QNOT(p \divides x_1)$ and $\QNOT(p \divides
x_2)$.  So there has to be something wrong somewhere.

The statement ``we have by strong induction that $p$ divides
one of them'' is the place where the proof breaks down: it appeals to
strong induction to justify applying the induction hypothesis for $2=k\leq
n$.  But the base case was $n=1$, so we can't assume $2 \leq n$.  Note
that the reasoning above is fine for every $n\geq 2$, so the whole proof
would be fine if we had an argument to prove the claim for $n+1=2$.

Now in fact, if a prime, $p$ divides $x_1x_2$, it must divide $x_1$ or
$x_2$; this follows by prime factorization of integers (and we'll show you
another proof later in the term).  But the proof here never made use of
this fact.
\end{solution}
\end{problem}

%%%%%%%%%%%%%%%%%%%%%%%%%%%%%%%%%%%%%%%%%%%%%%%%%%%%%%%%%%%%%%%%%%%%%
% Problem ends here
%%%%%%%%%%%%%%%%%%%%%%%%%%%%%%%%%%%%%%%%%%%%%%%%%%%%%%%%%%%%%%%%%%%%%
