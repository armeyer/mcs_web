\documentclass[problem]{mcs}

\begin{pcomments}
  \pcomment{from: F09.ps3}
  \pcomment{from: S02.ps2}
\end{pcomments}

\pkeywords{
  induction
  false_proof
}

%%%%%%%%%%%%%%%%%%%%%%%%%%%%%%%%%%%%%%%%%%%%%%%%%%%%%%%%%%%%%%%%%%%%%
% Problem starts here
%%%%%%%%%%%%%%%%%%%%%%%%%%%%%%%%%%%%%%%%%%%%%%%%%%%%%%%%%%%%%%%%%%%%%

\begin{problem}
Find the flaw in the following "proof" that $a^n = 1$ for all nonnegative integers $n$, whenever $a$ is a nonzero real number.

\begin{proof}
(by induction)

\textbf{Base Case:} $a^0 =1$ is true by definition of $a^0$.

\textbf{Inductive Step:} Assume that $a^k = 1$ for all nonnegative integers $k$ with $k \leq n$. Then note that
\[
a^{n+1} = \frac{a^n \cdot a^n}{a^{n-1}} = \frac{1 \cdot 1}{1} = 1.
\]
\end{proof}

\begin{solution}
The flaw comes in the inductive step, where we implicitly assume
$n\geq 1$ in order to talk about $a^{n-1}$ in the denominator
(otherwise the exponent is not a nonnegative integer, so we cannot
apply the inductive hypothesis).  We checked the base case only for
$n=0$, so we are not justified in assuming that $n\geq 1$ when we try
to prove the statement for $n+1$ in the inductive step.  Indeed, the
proposition first breaks precisely at $n=1$.
\end{solution}

\end{problem}

%%%%%%%%%%%%%%%%%%%%%%%%%%%%%%%%%%%%%%%%%%%%%%%%%%%%%%%%%%%%%%%%%%%%%
% Problem ends here
%%%%%%%%%%%%%%%%%%%%%%%%%%%%%%%%%%%%%%%%%%%%%%%%%%%%%%%%%%%%%%%%%%%%%
