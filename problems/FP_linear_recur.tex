\documentclass[problem]{mcs}

\begin{pcomments}
  \pcomment{FP_linear_recur}
  \pcomment{slightly more complex version of MQ_gen_2}
  \pcomment{renamed from MQ_gen_3}
  \pcomment{soln edited ARM 4/28/12}
\end{pcomments}

\pkeywords{
  recurrence
  linear_recurrence
  generating_function
}

%%%%%%%%%%%%%%%%%%%%%%%%%%%%%%%%%%%%%%%%%%%%%%%%%%%%%%%%%%%%%%%%%%%%%
% Problem starts here
%%%%%%%%%%%%%%%%%%%%%%%%%%%%%%%%%%%%%%%%%%%%%%%%%%%%%%%%%%%%%%%%%%%%%

\begin{problem}
Alyssa Hacker sends out a video that spreads like wildfire over the
UToob network.  On the day of the release---call it \emph{day
  zero}---and the day following---call it \emph{day one}---the video
doesn't receive any hits.  However, starting with day two, the number
of hits, $r_n$, can be expressed as seven times the number of hits on
the previous day, four times the number of hits the day before that,
and the number of days that has passed since the release of the video
plus one.  So, for example on day 2, there will be $7 \times 0 + 4
\times 0 + 3 = 3$ hits.

\bparts

\ppart
Give a linear a recurrence  for $r_n$.

\begin{solution}
\[
     r_n = 7r_{n-1}  +  4r_{n-2}  + (n+1), \qquad \text{for } n\geq 2.
\]
\end{solution}

\ppart Express the generating function $R(x) \eqdef \sum_0^\infty
r_nx^n$ as a quotient of polynomials or products of polynomials.  You
do \emph{not} have to find a closed form for $r_n$.

\begin{solution}
We have
\[
\begin{array}{rcrcrcrcrcr}
R(x)      & = & r_0 & + & r_1  x & + & r_2 x^2 & + & r_3 x^3 & + & r_4 x^4 + \cdots,\\
-7xR(x)   & = &     & - & 7r_0  x & - & 7r_1 x^2 & - & 7r_2 x^3 & - & 7r_3 x^4 - \cdots.,\\
-4x^2R(x) & = &     &   &        & - & 4r_0 x^2 & - & 4r_1 x^3 & - & 4r_2 x^4 - \cdots,
\end{array}
\]
so
\[
\begin{array}{rcrcrcrcrcr}
R(x)(1-7x-4x^2)
         & = & r_0 & + & (r_1-7r_0)
                              x & + & (r_2 - 7r_1 - 4r_0)
                                          x^2 & + & \cdots\\
         & = &  0  & + & 0     & + & 3 x^2 & + & 4 x^3 + \cdots
\end{array}
\]
Remembering that
\[
1 + 2x + 3x^2+ \cdots + (n+1)x^n + \cdots = \frac{1}{(1-x)^2}
\]
by~(\bref{c-alpha-k-coeff}), we have
\[
R(x)(1-7x-4x^2) = \frac{1}{(1-x)^2} - 1 - 2x,
\]
\iffalse
Let the right hand side be $F(x)$, we have $F(x)=3x^2 + 4x^3 + 5x^4 +
\cdots$.
\[
 F(x) - xF(x) = 3x^2 + x^3 + x^4 + \cdots = G(x)
\]
 \[
  G(x) - xG(x) = 3x^2 - 2x^3  \Longrightarrow G(x) = \frac{3x^2 -
    2x^3}{1-x}
 \]
 We have $F(x) = \frac{G(x)}{1-x} = \frac{3x^2 - 2x^3}{(1-x)^2}$,
so
%\begin{equation}\label{Rfracx1}
\[
R(x) = \frac{F(x)}{1 - 7x - 4x^2} = \frac{3x^2 - 2x^3}{(1-x)^2(1-7x-4x^2)}\, .
\]
%\end{equation}
\fi
so
\[
R(x) = \frac{1}{(1-7x-4x^2)(1-x)^2} - \frac{1 - 2x}{1-7x-4x^2}
= \frac{1+3x^2 -2x^3}{1-7x-4x^2}\ .
\]

\end{solution}

\eparts
\end{problem}


%%%%%%%%%%%%%%%%%%%%%%%%%%%%%%%%%%%%%%%%%%%%%%%%%%%%%%%%%%%%%%%%%%%%%
% Problem ends here
%%%%%%%%%%%%%%%%%%%%%%%%%%%%%%%%%%%%%%%%%%%%%%%%%%%%%%%%%%%%%%%%%%%%%
\endinput
