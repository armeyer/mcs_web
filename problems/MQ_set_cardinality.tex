\documentclass[problem]{mcs}

\begin{pcomments}
  \pcomment{MQ_set_cardinality}
  \pcomment{formerly fall15/M2_sets}
  \pcomment{TP_cardinality_class + TP_uncountable_example}
  \pcomment{F15.mid2}
  \pcomment{Zoran Dzunic 10/10/15; edited ARM 10/20/15}
\end{pcomments}

%%%%%%%%%%%%%%%%%%%%%%%%%%%%%%%%%%%%%%%%%%%%%%%%%%%%%%%%%%%%%%%%%%%%%
% Problem starts here
%%%%%%%%%%%%%%%%%%%%%%%%%%%%%%%%%%%%%%%%%%%%%%%%%%%%%%%%%%%%%%%%%%%%%
\begin{problem}

\bparts

\ppart
For each of the following sets, indicate whether it is
finite\inhandout{ (\textbf{F})},
countably infinite\inhandout{ (\textbf{C})},
or uncountable\inhandout{ (\textbf{U})}.

\renewcommand{\theenumi}{\roman{enumi}}
\renewcommand{\labelenumi}{(\theenumi)}

\begin{enumerate}%{i}

%\item \label{itm:roots} The set of solutions to the equation $x^3 - x
%  = -0.1$. \hfill\examrule[0.4in]

%\item \label{itm:naturals} The set of natural numbers $\nngint$. \instatements{\hfill\examrule[0.4in]}

%\item \label{itm:rationals} The set of rational numbers $\rationals$. \instatements{\hfill\examrule[0.4in]}

%\item \label{itm:reals} The set of real numbers $\reals$. \instatements{\hfill\examrule[0.4in]}

\item %\label{itm:ints}
The set of even integers greater than $10^{100}$.\hfill\examrule[0.4in]

\begin{solution}
\textbf{C}.

The function $f(n) \eqdef 2n + 10^{100} + 1$ defines a bijection with
$\nngint$.
\end{solution}

\item %\label{itm:complex}
The set of ``pure'' complex numbers of the form $ri$ for nonzero real
numbers $r$.\hfill\examrule[0.4in]

\begin{solution}
\textbf{U}

Mapping $ri$ to $r$ defines a bijection to the nonzero real numbers.
But $\reals - \set{0}$ is uncountable because it differs by one
element from the uncountable set $\reals$ of all reals.
\end{solution}

%\item \label{itm:char} The set of words in the English language no more
%  than 20 characters long. \instatements{\hfill\examrule[0.4in]}

\item %\label{itm:self_bij}
The powerset of the integer interval $\Zintvcc{10}{10^{10}}$.\hfill\examrule[0.4in]
%  bijections from $\set{1,2,\dots,10}$ to itself.

\begin{solution}
\textbf{F}.

The powerset of a finite set is exponentially larger, but still finite.
\end{solution}

\item %\label{itm:roots}
The complex numbers $c$ such that $c$ is the root of a quadratic with
integer coefficients, that is, \hfill\examrule[0.4in]
\[
\exists m,n,p \in \integers, m\neq 0.\ mc^2 + nc + p = 0.
\]

\begin{solution}
\textbf{C}.

The quadratics with integer coefficients are a countably infinite set
because there is a countable infinity of triples $(m,n,p) \in
\integers^3$.  Since each quadratic in $c$ has at most two roots, the number
of roots is also countably infinite.
\end{solution}

\iffalse
 $\exists m,n \in \integers^{+}.\, (m+nc)c = 0$. \hfill\examrule[0.4in]
\fi

\medskip Let $\mathcal{U}$ be an uncountable set, $\mathcal{C}$ be a
countably infinite subset of $\mathcal{U}$, and $\mathcal{D}$ be a
countably infinite set.

\item %\label{itm:unionuc}
 $\mathcal{U} \union \mathcal{D}$. \hfill\examrule[0.4in]

\begin{solution}
\textbf{U}.

Adding a countable number of elements to an infinite set $A$ yields a
set of the ``same size,'' that is, a set with a bijection to $A$,
\inhandout{as remarked in the text}\inbook{(see
  Section~\bref{countable_subsec} and
  Problem~\ref{PS_add_countable_elements})}.  So since $U$ is
uncountable, $\mathcal{U} \union \mathcal{D}$ is also uncountable.
\end{solution}

\item %\label{itm:interuc}
$\mathcal{U} \intersect \mathcal{C}$\hfill\examrule[0.4in]

\begin{solution}
\textbf{C}.

Since $\mathcal{C} \subseteq \mathcal{U}$,
\[
\mathcal{U} \intersect \mathcal{C} = \mathcal{C}
\]
\end{solution}

\item %\label{itm:diffuc}
$\mathcal{U} - \mathcal{D}$ \hfill\examrule[0.4in]

\begin{solution}
Adding a countable number of elements to an infinite set $A$ yields a
set with a bijection to $A$, as noted above.  Since we can add the
elements of $D$ back into $\mathcal{U} - \mathcal{D}$ to get the
uncountable set $U$, the set $\mathcal{U} - \mathcal{D}$ must have
been uncountable in the first place.
\end{solution}

\end{enumerate}

\begin{staffnotes}
\TBA{Interleave solns with items. Add Explanations.}

\begin{enumerate}

\item %\label{itm:ints}
(\textbf{C})

\item %\label{itm:complex}
 (\textbf{U})

\item %\label{itm:self_bij}
(\textbf{F})

\item %\label{itm:roots}
(\textbf{C})

\item %\label{itm:unionuc}
(\textbf{U})

\item %\label{itm:interuc}
(\textbf{C})

\item %\label{itm:diffuc}
(\textbf{U})

\end{enumerate}
\end{staffnotes}

\examspace[0.1in]

\ppart
Give an example of sets $A$ and $B$ such that
\[
\reals \strict A \strict B.
\]
%Recall that $A \strict B$ means that $A$ is not ``as big as'' $B$.

\begin{solution}
Let $A$ be $\power(\reals)$ and $B$ be $\power(A)$.
\end{solution}

%\examspace[2.0in]

\eparts

\end{problem}

%%%%%%%%%%%%%%%%%%%%%%%%%%%%%%%%%%%%%%%%%%%%%%%%%%%%%%%%%%%%%%%%%%%%%
% Problem ends here
%%%%%%%%%%%%%%%%%%%%%%%%%%%%%%%%%%%%%%%%%%%%%%%%%%%%%%%%%%%%%%%%%%%%%
\endinput
