\documentclass[problem]{mcs}

\begin{pcomments}
    \pcomment{PS_find_dnf}
    \pcomment{by ARM 2/5/11, revised 2/7/16}
\end{pcomments}

\pkeywords{
  proposition
  formula
  dnf
  cnf
  equivalence
  conjunt
  disjunct}

%%%%%%%%%%%%%%%%%%%%%%%%%%%%%%%%%%%%%%%%%%%%%%%%%%%%%%%%%%%%%%%%%%%%%
% Problem starts here
%%%%%%%%%%%%%%%%%%%%%%%%%%%%%%%%%%%%%%%%%%%%%%%%%%%%%%%%%%%%%%%%%%%%%

\begin{problem}
\begin{staffnotes}
If used as a class problem, do only one part.
\end{staffnotes}

\bparts Use the equivalence axioms of
Section~\bref{propositional_equivalences_sec} to convert the formula
\[
A \QXOR B \QXOR C
\]

\ppart \dots to disjunctive (\QOR\ of \QAND's) form,

\begin{solution}
We begin by noting that
\begin{equation}\label{XORequiv}
A \QXOR B \corresp (A \QAND \bar{B}) \QOR (\bar{A} \QAND B)
\end{equation}

So we have
\begin{align*}
\lefteqn{A \QXOR B \QXOR C}\\
                    & \corresp (A \QXOR B) \QXOR C
     & \text{(by def)} \\
 & \corresp ((A \QAND \bar{B}) \QOR (\bar{A} \QAND B)) \QXOR C
     & \text{(by~\eqref{XORequiv})} \\
 & \corresp 
      (((A \QAND \bar{B}) \QOR (\bar{A} \QAND B)) \QAND \bar{C})\ \QOR\\
   &\qquad\qquad (\QNOT(((A \QAND \bar{B}) \QOR (\bar{A} \QAND B))) \QAND C)
     & \text{(by~\eqref{XORequiv})} \\
   & \corresp
  (A \QAND \bar{B} \QAND \bar{C}) \QOR (\bar{A} \QAND B \QAND \bar{C})\ \QOR\\
   &\qquad\qquad ((\QNOT((A \QAND \bar{B})) \QAND \QNOT((\bar{A} \QAND B))) \QAND C)
      & \text{(by distributivity and DeMorgan for \QOR)}\\
  & \corresp 
 (A \QAND \bar{B} \QAND \bar{C}) \QOR  (\bar{A} \QAND B \QAND \bar{C})\ \QOR\\
  &\qquad\qquad (((\bar{A} \QOR B) \QAND (A \QOR \bar{B})) \QAND C)
      & \text{(by DeMorgan for \QAND)}\\
  & \corresp 
 (A \QAND \bar{B} \QAND \bar{C}) \QOR
 (\bar{A} \QAND B \QAND \bar{C})\ \QOR\\
  &\qquad\qquad ((A \QAND (\bar{A} \QOR B)) \QOR ((\bar{A} \QOR B) \QAND\bar{B})) \QAND C
      & \text{(by distributivity)}\\
  & \corresp
 (A \QAND \bar{B} \QAND \bar{C}) \QOR (\bar{A} \QAND B \QAND \bar{C})\ \QOR\\
 &\qquad\qquad (((A \QAND B) \QOR (\bar{A} \QAND \bar{B})) \QAND C)
      & \text{(by distributivity and contradiction for \QAND)}\\
  & \corresp 
 (A \QAND \bar{B} \QAND \bar{C}) \QOR\ (\bar{A} \QAND B \QAND \bar{C})\ \QOR\\
 &\qquad\qquad (\bar{A} \QAND \bar{B} \QAND C) \QOR\ (A \QAND B \QAND C).
      & \text{(by distributivity)}
\end{align*}

\iffalse
This, applying distributivity and DeMorgan's laws, yields the following:

\begin{gather*}
A \QXOR B \QXOR C\\
\Updownarrow \\
(A \QXOR B) \QXOR C \\
\Updownarrow \\
((A \QAND \bar{B}) \QOR (\bar{A} \QAND B)) \QXOR C\\
\Updownarrow \\
(((A \QAND \bar{B}) \QOR (\bar{A} \QAND B)) \QAND \bar{C}) \QOR (\bar{((A \QAND \bar{B}) \QOR (\bar{A} \QAND B))} \QAND C)\\
\Updownarrow \\
(A \QAND \bar{B} \QAND \bar{C}) \QOR (\bar{A} \QAND B \QAND \bar{C}) \QOR ((\bar{(A \QAND \bar{B})} \QAND \bar{(\bar{A} \QAND B)}) \QAND C)\\
\Updownarrow \\
(A \QAND \bar{B} \QAND \bar{C}) \QOR (\bar{A} \QAND B \QAND \bar{C}) \QOR (((\bar{A} \QOR B) \QAND (A \QOR \bar{B})) \QAND C)\\
\Updownarrow \\
(A \QAND \bar{B} \QAND \bar{C}) \QOR (\bar{A} \QAND B \QAND \bar{C}) \QOR (((A \QAND (\bar{A} \QOR B)) \QOR ((\bar{A} \QOR B) \QAND\bar{B}))) \QAND C)\\
\Updownarrow \\
(A \QAND \bar{B} \QAND \bar{C}) \QOR (\bar{A} \QAND B \QAND \bar{C}) \QOR (((A \QAND B) \QOR (\bar{A} \QAND \bar{B})) \QAND C)\\
\Updownarrow \\
(A \QAND \bar{B} \QAND \bar{C}) \QOR (\bar{A} \QAND B \QAND \bar{C}) \QOR (\bar{A} \QAND \bar{B} \QAND C) \QOR (A \QAND B \QAND C).
\end{gather*}
\fi

\end{solution}

\ppart \dots to conjunctive (\QAND\ of \QOR's) form.

\begin{solution}
We begin by noting that $A \QXOR B$ evaluates to \true\ exactly when
either of $A$ or $B$ is \true\ but not both at the same time.

Also, notice that in the previous solution we proved along the way that 
$$
(A \QOR \bar{B}) \QAND (\bar{A} \QOR B) \Longleftrightarrow (A \QAND B) \QOR (\bar{A} \QAND \bar{B})
$$

Thus, applying distributivity and DeMorgan's laws together with the
fact above, yields the following:

\begin{gather*}
A \QXOR B \QXOR C\\
\corresp \\
(A \QXOR B) \QXOR C \\
\corresp \\
((A \QOR B) \QAND \bar{(A \QAND B)}) \QXOR C\\
\corresp \\
((A \QOR B) \QAND (\bar{A} \QOR \bar{B})) \QXOR C\\
\corresp \\
(((A \QOR B) \QAND (\bar{A} \QOR \bar{B})) \QOR C) \QAND (\bar{((A \QOR B) \QAND (\bar{A} \QOR \bar{B}))} \QOR \bar{C})\\
\corresp \\
(A \QOR B \QOR C) \QAND (\bar{A} \QOR \bar{B} \QOR C) \QAND ((\bar{(A \QOR B)} \QOR \bar{(\bar{A} \QOR \bar{B})}) \QOR \bar{C})\\
\corresp \\
(A \QOR B \QOR C) \QAND (\bar{A} \QOR \bar{B} \QOR C) \QAND (((\bar{A} \QAND \bar{B}) \QOR (A \QAND B)) \QOR \bar{C})\\
\corresp \\
(A \QOR B \QOR C) \QAND (\bar{A} \QOR \bar{B} \QOR C) \QAND (((A \QOR \bar{B}) \QAND (\bar{A} \QOR B)) \QOR \bar{C})\\
\corresp \\
(A \QOR B \QOR C) \QAND (\bar{A} \QOR \bar{B} \QOR C) \QAND (\bar{A} \QOR B \QOR \bar{C}) \QAND (A \QOR \bar{B} \QOR \bar{C}).
\end{gather*}
\end{solution}

\eparts

\end{problem}

\endinput
