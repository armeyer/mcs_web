\documentclass[problem]{mcs}

\begin{pcomments}
  \pcomment{FP_variance_dice_sum}
  \pcomment{ARM 2/12/11, edited 12/13/13, 2/19/15, 5/7/16}
\end{pcomments}

\pkeywords{
  random_variable
  variance
  geometric
  mutual_independence
  additivity
  indicator
}

%%%%%%%%%%%%%%%%%%%%%%%%%%%%%%%%%%%%%%%%%%%%%%%%%%%%%%%%%%%%%%%%%%%%%
% Problem starts here
%%%%%%%%%%%%%%%%%%%%%%%%%%%%%%%%%%%%%%%%%%%%%%%%%%%%%%%%%%%%%%%%%%%%%

\begin{problem}
You are playing a game where you get $n$ turns.  Each of your turns
involves flipping a coin a number of times.  On the first turn, you
have 1 flip, on the second turn you have two flips, and so on until
your $n$th turn when you flip the coin $n$ times.  All the flips are
mutually independent.

The coin you are using is biased to flip Heads with probability $p$.
You \emph{win} a turn if you flip all Heads.  Let $W$ be the number of
winning turns.

\bparts

\ppart Write a closed-form (no summations) expression for
$\expect{W}$.

\begin{center}
\exambox{3.0in}{0.6in}{-0.3in}
\end{center}

\examspace[0.7in]

\begin{solution}
\[
\frac{1-p^{n+1}}{1-p}-1.
\]
This can also be expressed as
\[
\frac{p(1-p^n)}{1-p}.
\].

Let $H_k$ be indicator for winning the $k$th try.  This means
that
\[
W = \sum_1^n H_k.
\]

By independence of coin flips,
\[
p^k = \pr{H_k=1} = \expect{H_k},
\]
so
\[
\expect{W} = \sum_1^n \expect{H_k} = \sum_1^n p^k = \paren{\sum_0^n p^k} - 1 =
\frac{1-p^{n+1}}{1-p} - 1.
\]
\end{solution}

\ppart Write a closed-form expression for $\variance{W}$.

\begin{center}
\exambox{3.0in}{0.6in}{-0.3in}
\end{center}

\examspace[0.7in]

\begin{solution}
\[
\frac{1-p^{n+1}}{1-p} - \frac{1-p^{2(n+1)}}{1-p^2},
\]
which can also be written
\[
p\frac{1-p^n}{1-p} - p^2\frac{1-p^{2n}}{1-p^2}.
\]

We have $\variance{H_k} = p^k(1-p^k)$.  Variances add because of
mutual independence, so
\begin{align*}
\variance{W}
    & = \sum_{k=1}^{n} p^k(1-p^k)\\
    & = \sum_{k=1}^{n} p^k-p^{2k}\\
    & = \sum_{k=0}^{n} p^k-p^{2k}\\
    & = \sum_0^n p^k - \sum^0^np^{2k}\\
    & = \frac{1-p^{n+1}}{1-p} - \frac{1-p^{2(n+1)}}{1-p^2}
\iffalse
    & = p\sum_{k=0}^{n-1} p^k - p^2\paren{\sum_{k=0}^{n-1} \paren{p^2}^k}\\
    & = p\frac{1-p^n}{1-p} - p^2\frac{1-p^{2n}}{1-p^2}.
\fi
\end{align*}

\end{solution}

\begin{editingnotes}
Could make Chebyshev, Chernoff, $\cdf$ estimate comparison.
\end{editingnotes}

\eparts

\end{problem}

%%%%%%%%%%%%%%%%%%%%%%%%%%%%%%%%%%%%%%%%%%%%%%%%%%%%%%%%%%%%%%%%%%%%%
% Problem ends here
%%%%%%%%%%%%%%%%%%%%%%%%%%%%%%%%%%%%%%%%%%%%%%%%%%%%%%%%%%%%%%%%%%%%%

\endinput
