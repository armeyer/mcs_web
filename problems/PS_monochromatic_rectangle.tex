\documentclass[problem]{mcs}

\begin{pcomments}
  \pcomment{PS_monochromatic_rectangles}
  \pcomment{Author: Justin Venezuela (jven@mit.edu)}
  \pcomment{Generalized from USAMTS Problem 4/1/18 (usamts.org).}
\end{pcomments}

\pkeywords{
  counting
  Pigeonhole Principle
  coloring
}

%%%%%%%%%%%%%%%%%%%%%%%%%%%%%%%%%%%%%%%%%%%%%%%%%%%%%%%%%%%%%%%%%%%%%
% Problem starts here
%%%%%%%%%%%%%%%%%%%%%%%%%%%%%%%%%%%%%%%%%%%%%%%%%%%%%%%%%%%%%%%%%%%%%

\begin{problem}
Suppose every point in the plane with integer coordinates
is colored one of $k$ colors, where $k$ is a positive integer. We will
prove that there must exist a rectangle whose vertices have integer
coordinates and are all colored the same color.

\bparts

\ppart Let's start with the case $k=3$: suppose each point with integer 
coordinates is colored, say, red, white, or blue. Consider the points $(x,y)$ 
such that $0\leq x\leq 3$ and $0\leq y\leq 81$. This is a rectangular array of 
points with $82$ rows and $4$ columns. Prove that some two of these rows must 
be colored the same. In other words, prove that there exist $y_1, y_2$, $y_1\neq y_2$, 
$0\leq y_1, y_2\leq 81$ such that $(x,y_1)$ and $(x,y_2)$ have the same color 
for all $0\leq x\leq 3$.

\begin{solution}
Each row consists of $4$ points, one for each $x$ coordinate between $0$ and $3$. 
Each point can be colored one of $3$ colors, and each choice of color is independent 
of the others. Thus the number of ways to color a row is $3^4=81$. Since the 
rectangular array we are considering has $82$ rows, we have by the Pigeonhole 
Principle that there exist two rows with the same coloring.
\end{solution}

\ppart Prove that there must exist a rectangle whose vertices come from these 
two rows and such that all four of the vertices are the same color.

\begin{solution}
We claim that in any row of this rectangular array, there must exist two points 
with the same color. Indeed, there are $4$ points in each row but only $3$ 
colors, so the Pigeonhole Principle gives the desired result.

In particular, we have from part (a) that there exist two rows $y_1, y_2$ 
with the same coloring. In row $y_1$, there must exist $(x_1, y_1)$ and 
$(x_2, y_1)$ with the same color $c$. But since row $y_2$ is colored the 
same as row $y_1$, $(x_1, y_2)$ and $(x_2, y_2)$ must also both be 
colored $c$. The four points $(x_1, y_1), (x_2, y_1), (x_1, y_2), (x_2, y_2)$ 
are the vertices of a rectangle and are all colored $c$, and we're done.
\end{solution}

\ppart Generalize the above argument to $k$ colors.

\begin{solution}
Consider the rectangular array of points $(x,y)$ such that $0\leq x\leq k$ 
and $0\leq y\leq k^{k+1}$. We claim that some two rows are colored the same. 
Indeed, each row has $k+1$ points and we independently choose one of $k$ colors 
for each point. Thus the number of way to color a row is $k^{k+1}$. Since there 
are $k^{k+1}+1$ rows, there must exist two rows $y_1,y_2$ colored the same by 
the Pigeonhole Principle.

Now in row $y_1$, there must exist two points colored the same: indeed, there 
are $k+1$ points in this row but only $k$ colors, so the Pigeonhole Principle 
gives that for some $x_1, x_2$, $(x_1, y_1)$ and $(x_2, y_1)$ have the same 
color $c$. But since row $y_2$ has the same coloring as $y_1$, $(x_1, y_2)$ 
and $(x_2, y_2)$ must also have color $c$. $(x_1, y_1), (x_2, y_1), (x_1, y_2), (x_2, y_2)$ 
are the vertices of a rectangle and are all colored $c$, and we're done.
\end{solution}

\eparts
\end{problem}

%%%%%%%%%%%%%%%%%%%%%%%%%%%%%%%%%%%%%%%%%%%%%%%%%%%%%%%%%%%%%%%%%%%%%
% Problem ends here
%%%%%%%%%%%%%%%%%%%%%%%%%%%%%%%%%%%%%%%%%%%%%%%%%%%%%%%%%%%%%%%%%%%%%

\endinput
