\documentclass[problem]{mcs}

\begin{pcomments}
  \pcomment{PS_Euler_function_multiplicativity}
  \pcomment{depends on CP_chinese_remainder}
  \pcomment{formerly called PS_distributive_law_for_phi}
  \pcomment{from: S09.ps8m, S08.ps7}
  \pcomment{revised by ARM 2/2/11}
\end{pcomments}

\pkeywords{
  primes
  number_theory
  Eulers_theorem
  modular_arithmetic
  bijections
  mulitplicative
  Euler_function
  phi_function
}

%%%%%%%%%%%%%%%%%%%%%%%%%%%%%%%%%%%%%%%%%%%%%%%%%%%%%%%%%%%%%%%%%%%%%
% Problem starts here
%%%%%%%%%%%%%%%%%%%%%%%%%%%%%%%%%%%%%%%%%%%%%%%%%%%%%%%%%%%%%%%%%%%%%

\begin{problem}
  Suppose $a, b$ are relatively prime integers greater than 1.  In
  this problem you will prove the key property of Euler's function
  that $\phi(ab)=\phi(a)\phi(b)$ using the Chinese Remainder Theorem
  of Problem~\bref{CP_chinese_remainder}.

\bparts

\ppart\label{fbij0ab} Conclude from the Chinese Remainder Theorem that
the function $f: [0,ab) \to [0,a) \cross [0,b)$ defined by
\[
f(x) \eqdef (\rem{x}{a},\rem{x}{b})
\]
is a bijection.

\begin{solution}
The Chinese Remainder Theorem says that the congruences
 \begin{align*}
   x &\equiv m \pmod a,\label{xma}\\
   x &\equiv n \pmod b.\label{xnb}
  \end{align*}
have a solution $x \in [0,ab)$, which means that $f$ is surjective,
  and that the solution is unique, which means that $f$ is injective,
  and hence it is a bijection.
\end{solution}

\ppart For any positive integer, $k$, let $k^*$ be the integers in
$[1,k)$ that are relatively prime to $k$.  Prove that the function $f$
  from part~\eqref{fbij0ab} also defines a bijection from $(ab)^*$ to
  $a^* \times b^*$

\begin{solution}
If $g: A \to B$ is a bijection, and $C \subseteq A$, then by
definition of bijection, we know that restricting $g$ to domain $C$
defines bijection from $C$ to $g(C)$.  So all that's needed is to show that
$f((ab)^*) = a^* \times B^*$.

But since $a$ and $b$ are relatively prime, number $x$ is relatively prime
to $ab$ iff $x$ is relatively prime to $a$ and $x$ is relatively prime
to $b$, by Unique Factorization.  This means precisely that $x \in
(ab)^* \qiff f(x) \in a^* \times b^*$, which in turn means $f((ab)^*) = a^*
\times b^*$.

\end{solution}

\ppart Conclude from the preceding parts of this problem
that
\[
\phi(mn)=\phi(m)\phi(n).
\]

\begin{solution}
The mapping $f$ defines a bijection between $mn^*$ and $m^* \cross n^*$.
So
\[
\phi(mn) \eqdef \card{mn^*} = \card{m^* \cross n^*} = \card{m^*}\cdot  \card{n^*}.
\]
\end{solution}

\eparts
\end{problem}

%%%%%%%%%%%%%%%%%%%%%%%%%%%%%%%%%%%%%%%%%%%%%%%%%%%%%%%%%%%%%%%%%%%%%
% Problem ends here
%%%%%%%%%%%%%%%%%%%%%%%%%%%%%%%%%%%%%%%%%%%%%%%%%%%%%%%%%%%%%%%%%%%%%

\endinput
