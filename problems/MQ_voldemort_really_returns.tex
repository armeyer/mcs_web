\documentclass[problem]{mcs}

\begin{pcomments}
  \pcomment{MQ_voldemort_really_returns}
  \pcomment{forked from MQ_voldemort_returns}
  \pcomment{original: ARM 11/24/13}
  \pcomment{CH, Spring '14}
\end{pcomments}

\pkeywords{
  conditional_probability
  tree_diagram
  four-step_method
}

%%%%%%%%%%%%%%%%%%%%%%%%%%%%%%%%%%%%%%%%%%%%%%%%%%%%%%%%%%%%%%%%%%%%%
% Problem starts here
%%%%%%%%%%%%%%%%%%%%%%%%%%%%%%%%%%%%%%%%%%%%%%%%%%%%%%%%%%%%%%%%%%%%%

\begin{problem}
A guard is going to release exactly two of three prisoners, Sauron,
Voldemort, and Bunny Foo Foo, and he is equally likely to release any
pair of prisoners. The prisoners ask him to announce the name of one
of the released prisoners. The guard has three rules for choosing whom
he names:

\begin{itemize}

\item He will never say that Sauron will be released.

\item If Sauron is getting released, he will mention the name of the
  prisoner who is \textbf{not} getting released.

\item If both Foo Foo and Voldemort are getting released, he will say
  ``Foo Foo'' with probability $2/5$ and ``Voldemort'' with probability $3/5$, 

\end{itemize}

We are interested in the following events:
\begin{description}
\item[\ \ $S$:] \emph{S}auron is released.
\item[\ \ ``$V$'':] The guard says that \emph{V}oldemort will be released.
\end{description}

\bparts

\ppart What is the value of $\pr{S}$? \hfill\examrule

\begin{solution}
$\pr{S} = 2/3$, since Sauron is released in 2 out of 3 cases. 
\end{solution}

\examspace[0.75in]

\ppart Use a tree diagram to derive a probability space suitable for
representing the events above.  Indicate with a 'x' mark which outcomes are in each of
the events $S$ and $\text{``V''}$.

\begin{solution}
The tree diagram is as follows.

\begin{center}
\begin{picture}(360,175)(0,-40)
%\put(0,-40){\dashbox(360,175){}} % bounding box
\put(0,60){\line(1,1){60}}
\put(0,60){\line(1,0){60}}
\put(0,60){\line(1,-1){60}}
\put(30,-30){\makebox(0,0){released}}
\put(60,120){\line(1,0){60}}
\put(60,60){\line(2,1){60}}
\put(60,60){\line(2,-1){60}}
\put(60,0){\line(1,0){60}}
\put(90,-30){\makebox(0,0){guard says}}
\put(11,90){\makebox(0,0){$F,S$}}
\put(40,68){\makebox(0,0){$F,V$}}
\put(11,30){\makebox(0,0){$V,S$}}
\put(52,96){\makebox(0,0){$1/3$}}
\put(40,50){\makebox(0,0){$1/3$}}
\put(52,24){\makebox(0,0){$1/3$}}
\put(90,128){\makebox(0,0){``$V$''}}
\put(90,90){\makebox(0,0){``$F$''}}
\put(90,30){\makebox(0,0){``$V$''}}
\put(90,-10){\makebox(0,0){``$F$''}}
\put(90,110){\makebox(0,0){$1$}}
\put(102,70){\makebox(0,0){$2/5$}}
\put(102,50){\makebox(0,0){$3/5$}}
\put(90,8){\makebox(0,0){$1$}}
\put(150,120){\makebox(0,0){$1/3$}}
\put(150,90){\makebox(0,0){$2/15$}}
\put(150,30){\makebox(0,0){$1/5$}}
\put(150,0){\makebox(0,0){$1/3$}}
\put(150,-30){\makebox(0,0){prob.}}
\put(210,120){\makebox(0,0){$\times$}}
\put(210,30){\makebox(0,0){$\times$}}
\put(210,-30){\makebox(0,0){\shortstack{guard says\\"Voldemort''}}}
\put(270,120){\makebox(0,0){$\times$}}
\put(270,0){\makebox(0,0){$\times$}}
\put(270,-30){\makebox(0,0){\shortstack{Sauron\\released}}}
\end{picture}
\end{center}
\end{solution}

\examspace[3.0in]

\problempart Show that the probability that Sauron is released,
conditioned on the event that the guard says Voldemort will be
released, is less than the probability that Sauron is released.

%How does this compare to $\pr{S}$ ?

%\hint Calculate the value of $\prcond{S}{\text{``$V$''}}$.

\begin{solution}
We have
\[
\prcond{S}{\text{``$V$''}} = \frac{ \pr{S \intersect \text{``$V$''} }}{ \pr{\text{``$V$''}} }
       = \frac{1/3}{1/3 + 1/5} = \frac{5}{8} .
\]
But $\pr{S} = 2/3$ and hence $\prcond{S}{\text{``V''}} < \pr{S}$.
\end{solution}

% \problempart\label{saysVo} On the other hand, if the guards tells
% Sauron that Foo Foo will be released, then Sauron's probability of
% release will be increased.  Verify that Sauron is correct by calculating
% $\prcond{S}{\text{``$F$''}}$:
% \examspace[0.5in]

% \begin{solution}
% \[
% \frac{3}{4}\, ,
% \]
% because
% \[
% \prcond{S}{\text{``$F$''}}
%    = \frac{ \pr{S \QAND \text{``$F$''}} }{\pr{\text{``$F$''}} }
%   =\frac{ 1/3 }{ 1/3 + 1/9} = \frac{3}{4} > \frac{2}{3}\, .
% \]
% \end{solution}

% \problempart Show how to use the Law of Total Probability to combine
% your answers to parts~\eqref{saysFF} and~\eqref{saysVo} to verify
% that the result matches~\eqref{Sreleaseprob}.

% \examspace[2in]

% \begin{solution}
% \begin{align*}
% \frac{2}{3}
%  & = \prob{S}
%       & \text{(by~\eqref{Sreleaseprob})}\\
%  & = \prcond{S}{\text{``$F$''}}
%               \cdot \prob{\text{``$F$''}} + \prcond{S}{\text{``$V$''}} \cdot \prob{\text{``$V$''}}
%       & \text{(Total Probability)}\\
%  & = \frac{3}{4} \cdot \frac{4}{9} + \frac{3}{5} \cdot \frac{5}{9}
%        & \text{(parts~\eqref{saysFF} and~\eqref{saysVo})}\\
%  & = \frac{2}{3}\, .
% \end{align*}

%\end{solution}

\eparts

\end{problem}


%%%%%%%%%%%%%%%%%%%%%%%%%%%%%%%%%%%%%%%%%%%%%%%%%%%%%%%%%%%%%%%%%%%%%
% Problem ends here
%%%%%%%%%%%%%%%%%%%%%%%%%%%%%%%%%%%%%%%%%%%%%%%%%%%%%%%%%%%%%%%%%%%%%

\endinput
