\documentclass[problem]{mcs}

\begin{pcomments}
  \pcomment{from: S09.cp9t}
  \pcomment{from: F02.quiz2}
  \pcomment{has a lot of commented out material}
\end{pcomments}

\pkeywords{
  asymptotics
}

%%%%%%%%%%%%%%%%%%%%%%%%%%%%%%%%%%%%%%%%%%%%%%%%%%%%%%%%%%%%%%%%%%%%%
% Problem starts here
%%%%%%%%%%%%%%%%%%%%%%%%%%%%%%%%%%%%%%%%%%%%%%%%%%%%%%%%%%%%%%%%%%%%%

\begin{problem}
Recall that for functions $f,g$ on the natural numbers,
$\naturals$, $f = O(g)$ iff
\begin{equation}\label{Oh}
\exists c \in \naturals\, \exists n_0 \in \naturals\,
\forall n \geq n_0\quad c \cdot g(n) \geq \abs{f(n)}.
\end{equation}

For each pair of functions below, determine whether $f = O(g)$ and whether
$g = O(f)$.  In cases where one function is O() of the other, indicate the
\emph{smallest natural number}, $c$, and for that smallest $c$, the
\emph{smallest corresponding natural number $n_0$} ensuring that
condition~\eqref{Oh} applies.

\begin{problemparts}

\problempart $f(n) = n^2, g(n) = 3n$.

$f = O(g)$ \hspace{.5in}YES \hspace{.5in}NO \hspace{.5in} 
If YES, $c =$ \brule{.5in}, $n_0$ = \brule{.5in}

\solution{NO.}

$g = O(f)$ \hspace{.5in}YES \hspace{.5in}NO \hspace{.5in} 
If YES, $c =$ \brule{.5in}, $n_0$ = \brule{.5in}

\solution{YES, with $c = 1$, $n_0 = 3$, which works
because $3^2 = 9$, $3 \cdot 3 = 9$.}

\problempart $f(n) = (3n - 7) / (n + 4), g(n) = 4$

$f = O(g)$ \hspace{.5in}YES \hspace{.5in}NO \hspace{.5in} 
If YES, $c =$ \brule{.5in}, $n_0$ = \brule{.5in}

\solution{YES, with $c = 1, n_0 = 0$  (because $\abs{f(n)}< 3$).}

$g = O(f)$ \hspace{.5in}YES \hspace{.5in}NO \hspace{.5in} 
If YES, $c =$ \brule{.5in}, $n_0$ = \brule{.5in}

\solution{YES, with $c =2, n_0 = 15.$

Since $\lim_{n \to \infty} f(n) = 3$, the smallest possible $c$ is 2.
For $c = 2$, the smallest possible $n_0 = 15$ which follows from the
requirement that $2f(n_0) \ge 4$.}

\iffalse

\problempart  (NOT USED) $f(n) = 2^{(n + 2 \sin(n))}, g(n) = 2^n$

$f = O(g)$ \hspace{.5in}YES \hspace{.5in}NO \hspace{.5in} 
If yes, $c =$ \brule{.5in} $n_0$ = \brule{.5in}

$g = O(f)$ \hspace{.5in}YES \hspace{.5in}NO \hspace{.5in} 
If yes, $c =$ \brule{.5in} $n_0$ = \brule{.5in}


\solution{
$f = O(g)$    YES

$c = 4, n_0 = 0$    (because $2 \sin (n)$ contributes at worst $-2$ to the power)

$g = O(f)$    YES

$c = 4, n_0 = 0$    (because the $2 \sin (n)$ contributes at worst $+2$ to the power)
}

\fi

\problempart $f(n) = 1 + (n \sin(n\pi/2))^2, g(n) = 3n$

$f = O(g)$ \hspace{.5in}YES \hspace{.5in}NO \hspace{.5in} 
If yes, $c =$ \brule{.5in} $n_0$ = \brule{.5in}

\solution{NO, because $f(2n)=1$, which rules out $g =
O(f)$ since $g=\Theta(n)$.}

$g = O(f)$ \hspace{.5in}YES \hspace{.5in}NO \hspace{.5in} 
If yes, $c =$ \brule{.5in} $n_0$ = \brule{.5in}

\solution{NO, because $f(2n+1) = n^2+1 \neq O(n)$ which rules out
$f = O(g)$.}

\end{problemparts}

\end{problem}

%%%%%%%%%%%%%%%%%%%%%%%%%%%%%%%%%%%%%%%%%%%%%%%%%%%%%%%%%%%%%%%%%%%%%
% Problem ends here
%%%%%%%%%%%%%%%%%%%%%%%%%%%%%%%%%%%%%%%%%%%%%%%%%%%%%%%%%%%%%%%%%%%%%
