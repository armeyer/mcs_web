\documentclass[problem]{mcs}

\begin{pcomments}
  \pcomment{PS_filling_buckets_with_water}
  \pcomment{from: F09.ps4, F02.ps5}
  \pcomment{edited by ARM 2/26/11 and 4/24/11}
\end{pcomments}

\pkeywords{
	state_machines
        invariant
        jug
        die_hard
        gcd
}

%%%%%%%%%%%%%%%%%%%%%%%%%%%%%%%%%%%%%%%%%%%%%%%%%%%%%%%%%%%%%%%%%%%%%
% Problem starts here
%%%%%%%%%%%%%%%%%%%%%%%%%%%%%%%%%%%%%%%%%%%%%%%%%%%%%%%%%%%%%%%%%%%%%

\begin{problem}
Let's extend the jug filling scenario of
Section~\bref{jug_invar_subsubsec} to three jugs and a receptacle.
Suppose the jugs can hold $a$, $b$, and $c$ gallons of water,
respectively.

The receptacle can be used to store an unlimited amount of water, but
has no measurement markings.  Excess water can be dumped into the
drain.  Among the possible moves are:

\begin{enumerate}

\item\label{hose-bucket3} fill a bucket from the hose,

\item pour from the receptacle to a bucket until the bucket is full
or the receptacle is empty, whichever happens first,

\item empty a bucket to the drain,

\item\label{bucket-receptacle} empty a bucket to the receptacle, and

\item pour from one bucket to another until either the first is empty or the second is full.

\end{enumerate}

\bparts

\ppart
Model this scenario with a state machine.  (What are the states?  How does
a state change in response to a move?)

\begin{solution}
The states are quadruples $(w,x,y,z)$ which give the current amount of
water in the three buckets and the receptacle, respectively.
The initial state is $(0,0,0,0)$.

Here are the corresponding transitions for each of the above moves:
\begin{enumerate}
\item fill a bucket from the hose
\[
(w,x,y,z) \rightarrow \left \{ \begin{array}{l}
(a,x,y,z) \mbox{ if filling jug A from the hose } \\
(w,b,y,z) \mbox{ if filling jug B from the hose }\\
(w,x,c,z) \mbox{ if filling jug C from the hose }\\
\end{array} \right.
\]

\item pour from the receptacle to a bucket until the bucket is full
or the receptacle is empty, whichever happens first
\[
(w,x,y,z) \rightarrow \left \{ \begin{array}{l}
(a,x,y,z-(a-w)) \mbox{ if pouring to A and } z \geq (a-w) \\
(w+z,x,y,0) \mbox{ if pouring to A and } z < (a-w) \\
(w,b,y,z-(b-x)) \mbox{ if pouring to B and } z \geq (b-x) \\
(w,x+z,y,0) \mbox{ if pouring to B and } z < (b-x)\\
(w,x,c,z-(c-y)) \mbox{ if pouring to C and } z \geq (c-y) \\
(w,x,y+z,0) \mbox{ if pouring to C and } z < (c-y)
\end{array} \right.
\]

\item empty a bucket to the drain
\[
(w,x,y,z) \rightarrow \left \{ \begin{array}{l}
(0,x,y,z) \mbox{ if emptying A to the drain } \\
(w,0,y,z) \mbox{ if emptying B to the drain }\\
(w,x,0,z)\mbox{ if emptying C to the drain }
\end{array} \right.
\]

\item empty a bucket to the receptacle
\[
(w,x,y,z) \rightarrow \left \{ \begin{array}{l}
(0,x,y,z+w) \mbox{ if emptying A to the receptacle } \\
(w,0,y,z+x) \mbox{ if emptying B to the receptacle }\\
(w,x,0,z+y) \mbox{ if emptying C to the receptacle }
\end{array} \right.
\]


\item Pour from one bucket to another until either the first is empty or the second is full. Note that it is enough to show the transition ``Pouring from $A$ to $B$ until either $A$ is empty or $B$ is full" because the other 5 transitions are symmetrical.

Pour from $A$ to $B$ until either $A$ is empty or $B$ is full, whichever happens first
\[
(w,x,y,z) \rightarrow \left \{ \begin{array}{l}
(0,x+w,y,z) \mbox{ if } w < (b-x) \\
(w-(b-x),b,y,z) \mbox{ if } w \geq (b-x)
\end{array} \right.
\]
\end{enumerate}

\end{solution}


\ppart \label{gcd-iff}
Prove that Bruce can get $k\in\naturals$ gallons of water into the
receptacle using the above operations if $\gcd(a,b,c) \divides k$.

\begin{solution}
\begin{proof}

Since there exist integers $s, t, u$ s.t. gcd($a, b, c$) = $sa+tb+uc$, then if gcd($a, b, c) | k$ then there exists an integer $n$ s.t. $n$gcd($a, b) = k$, and hence $n(sa+tb+uc) = k$. Assume, without loss of generality, that $sa \ge tb \ge uc$  (otherwise, exchange the buckets in the following argument).

We can fill the receptacle with $k$ gallons:
First, we repeat $ns$ times moves 1 and 4, filling the $A$ bucket and pouring its content into the receptacle (note that if $sa \ge tb \ge uc$ then $s \ge 0$). With this series of moves we will get from state (0, 0, 0, 0) to (0, 0, 0, $nsa$). 

Then, if $t = 0$ and $u = 0$, we are already done since $k = nsa$. If $t > 0$, we repeat $nt$ times moves 1 and 4, but now with bucket $B$. If $t < 0$, we instead repeat $n|t|$ times moves 2 and 3 using bucket $B$. In both cases, we transition from (0, 0, 0, $nsa$) to (0, 0, 0, $nsa+ntb$).

Finally, if $u = 0$, we are done since $k = nsa + ntb$. If $u > 0$, we repeat $nu$ times moves 1 and 4, but now with bucket $C$. If $u < 0$, we instead repeat $n|u|$ times moves 2 and 3 using bucket $C$. In both cases, we transition from (0, 0, 0, $nsa+ntb$) to (0, 0, 0, $nsa+ntb+nuc$). 

At the end of this series, we will be in state (0, 0, 0, $nsa+ntb+nuc$) = ($0, 0, 0, k$).
\end{proof}
\end{solution}

%\ppart Prove conversely, that if $\gcd(a,b,c) \divides k$, then Bruce
%can get actually get $k$ gallons of water into the receptacle.

%\begin{editingnotes}
%This part may be out of place because needs gcd = linear combination.
%\end{editingnotes}

%\begin{solution}
%\TBA{solution}
%\end{solution}

\eparts
\end{problem}

%%%%%%%%%%%%%%%%%%%%%%%%%%%%%%%%%%%%%%%%%%%%%%%%%%%%%%%%%%%%%%%%%%%%%
% Problem ends here
%%%%%%%%%%%%%%%%%%%%%%%%%%%%%%%%%%%%%%%%%%%%%%%%%%%%%%%%%%%%%%%%%%%%%

\endinput