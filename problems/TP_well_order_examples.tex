\documentclass[problem]{mcs}

\begin{pcomments}
  \pcomment{TP_well_order_examples}
  \pcomment{ARM 2/4/12}
\end{pcomments}

\pkeywords{
  well_order
  minimum
}

\begin{problem}

Indicate which of the following sets of numbers have a minimum element
and which are well ordered.  For those that are not well ordered, give
an example of a subset with no minimum element.
     
\bparts
    
\ppart\label{geqsqrt2} The integers $\geq \sqrt{2}$.

\begin{solution}
These have a minimum element 2, and are well ordered by Corollary~bref{lower_bound_cor}.
\end{solution}

\ppart The integers $> \sqrt{2}$.

\begin{solution}
\dots same as part~\eqref{geqsqrt2}.
\end{solution}

\ppart The rational numbers $\geq \sqrt{2}$.

\begin{solution}
These have no minimum element and so are not well ordered.

They have no minimum element because $\sqrt{2}$ is irrational
(Theorem~\bref{thm:sqrt2irr_by_contra}).  Therefore, if $r$ is
rational number $\geq \sqrt{2}$, then in fact $r > \sqrt{2}$.  This
means that $r$ is not a minimum element $\geq \sqrt{2}$ because the
open-ended real interval $(r, \sqrt{2})$ has positive length, and
therefore contains a rational number, $q$, which by definition is
$\geq \sqrt{2}$ and less than $r$.
\end{solution}

\ppart A finite set of real numbers.

\begin{solution}
A finite set cannot have an infinite decreasing sequence of elements,
and so is well ordered (see Problem~\bref{CP_well_order_decreasing}).
\end{solution}

\ppart The set of rationals of the form $1/n$ where $n$ is a
positive integer.

\begin{solution}
No minimum element.
\end{solution}

\ppart The set of rationals of the form $1/n$ where $n$ is a positive
integer less than or equal to a $g \eqdef 10^{100}$ (a \term{google}).

\begin{solution}
This is a finite set of size a google, and so is well ordered.  Its
minimum element is $1/g$.
\end{solution}

\ppart The set $G$ of rationals of the form $m/n$ where $m$ and $n$
are positive integers and $n \leq g = 10^{100}$.

\begin{solution}
$G$ has minimum element $1/g$.  Furthermore, there is
  no infinite decreasing sequence of numbers in $G$, so it is well
  ordered (see Problem~\bref{CP_well_order_decreasing}.

To see why there is no infinite decreasing sequence in $G$, note that
every number in $G$ can be written as an integer quotient of plus a
remainder of at most $g$.  So there are at $kg$ numbers in $G$
that are less than any nonnegative integer $k$.
\end{solution}

\eparts

\end{problem}

\endinput
