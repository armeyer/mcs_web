\documentclass[problem]{mcs}

\begin{pcomments}
  \pcomment{PS_equivalence_relation_operations_part_b}
  \pcomment{extracted by Jenn Wong 3/29/13 from
    PS_equivalence_relation_operations and slightly modified}
\end{pcomments}

\pkeywords{
  equivalence_relations
}

%%%%%%%%%%%%%%%%%%%%%%%%%%%%%%%%%%%%%%%%%%%%%%%%%%%%%%%%%%%%%%%%%%%%%
% Problem starts here
%%%%%%%%%%%%%%%%%%%%%%%%%%%%%%%%%%%%%%%%%%%%%%%%%%%%%%%%%%%%%%%%%%%%%

\begin{problem}
Give an example of two equivalence relations $R_1$ and $R_2$ on the
set $\set{1, 2, 3}$ such that $R_1 \cup R_2$ is \textbf{not} an
equivalence relation.  Briefly explain why.

\iffalse
Remember that an equivalence relation must be reflexive, transitive,
and symmetric.
\fi

\begin{solution}
We give a counterexample showing that $R_1 \cup R_2$ may not be
an equivalence relation.  Let $R_1$ and $R_2$ be the relations on
$\set{1, 2, 3}$ where
\begin{align*}
R_1 \eqdef & \set{(1,1) (2,2) (3,3) (1,2) (2,1)},\\
R_2 \eqdef & \set{(1,1) (2,2) (3,3) (2,3) (3,2)}.
\end{align*}
It's easy to check that $R_1$ and $R_2$ are both equivalence relations.
But $R_1\cup R_2$ is not transitive, because $(1,2),(2,3) \in R_1\cup R_2$
and $(1,3) \notin R_1\cup R_2$.  Therefore $R_1\cup R_2$ is not an
equivalence relation.

\iffalse
\vspace{5mm} We can also write a proof instead of providing a
counterexample.  Specifically, we can conclude that transitivity (if
$a$ $R$ $b$ and $b$ $R$ $c$ then $a$ $R$ $c$) can be violated when
$(a, b)$ belongs only to $R_1$ and $(b, c)$ belongs only to
$R_2$.  Then, it is not necessarily true that $a$ $R$ $c$.
\fi

\end{solution}

\end{problem}


%%%%%%%%%%%%%%%%%%%%%%%%%%%%%%%%%%%%%%%%%%%%%%%%%%%%%%%%%%%%%%%%%%%%%
% Problem ends here
%%%%%%%%%%%%%%%%%%%%%%%%%%%%%%%%%%%%%%%%%%%%%%%%%%%%%%%%%%%%%%%%%%%%%
