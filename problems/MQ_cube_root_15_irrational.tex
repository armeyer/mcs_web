\documentclass[problem]{mcs}

\begin{pcomments}
  \pcomment{MQ_cube_root_15_irrational}
  \pcomment{riff on CP_generalize_root_2_proof}
  \pcomment{ARM for MQ 9/12/11}
\end{pcomments}

\pkeywords{
  root_4
  irrational
  rational
  contradiction
}

%%%%%%%%%%%%%%%%%%%%%%%%%%%%%%%%%%%%%%%%%%%%%%%%%%%%%%%%%%%%%%%%%%%%%
% Problem starts here
%%%%%%%%%%%%%%%%%%%%%%%%%%%%%%%%%%%%%%%%%%%%%%%%%%%%%%%%%%%%%%%%%%%%%

\begin{problem}\label{generprob}
   Prove that $\sqrt[3]{15}$ is irrational.

\begin{solution}
\begin{proof}
  Assume for the sake of contradiction that $\sqrt[3]{15}$ is rational.
  Under this assumption, there exist integers $a$ and $b$ such that
\[
\sqrt[3]{15} =  \frac{a}{b},
\]
where $a$ and $b$ have no common factor.  Now we prove that $a$ and $b$
have 5 as a common factor, a contradiction.

\begin{align}
\sqrt[3]{15} & = \frac{a}{b}, & \text{(by assumption)}\notag\\
15           & = \frac{a^{3}}{b^{3}}, & \text{(taking 3rd powers)}\notag\\
15b^{3}       & = a^{3}.\label{15b3}
\end{align}
Since 5 is a factor of the lefthand side of~\eqref{15b3}, it is also
  a factor of the right hand side, $a^{3}$.  By unique factorization,
  this implies that 5 is a factor of $a$.

In particular, $a = 5c$ for some integer $c$.  Thus,
\begin{align}
15b^{3} & = (5c)^3 = 5^3 c^3,\notag\\
3b^{3}  & = 5^2c^{7=3} & \text{(dividing by 5)}.\label{3b3=5}
\end{align}
Since 5 is a factor of the righthand side of~\eqref{3b3=5}, it is
also a factor of the left hand side, $3b^{3}$.  But 5 is not a factor
of 3, so by unique factorization it must be a factor of $b^3$ and
hence of $b$.
\end{proof}

\end{solution}

\end{problem}
%%%%%%%%%%%%%%%%%%%%%%%%%%%%%%%%%%%%%%%%%%%%%%%%%%%%%%%%%%%%%%%%%%%%%
% Problem ends here
%%%%%%%%%%%%%%%%%%%%%%%%%%%%%%%%%%%%%%%%%%%%%%%%%%%%%%%%%%%%%%%%%%%%%

\endinput
