\documentclass[problem]{mcs}

\begin{pcomments}
  \pcomment{FP_chain_list}
  \pcomment{ARM 4/1/16}
  \pcomment{S16.mid3}
  \pcomment{revised versin is PS_chain_list}
\end{pcomments}

\pkeywords{
  partial_order
  maximum
  linear
  chain
  comparable
}

%%%%%%%%%%%%%%%%%%%%%%%%%%%%%%%%%%%%%%%%%%%%%%%%%%%%%%%%%%%%%%%%%%%%%
% Problem starts here
%%%%%%%%%%%%%%%%%%%%%%%%%%%%%%%%%%%%%%%%%%%%%%%%%%%%%%%%%%%%%%%%%%%%%

\begin{problem} 
Let $R$ be a weak partial order on a set $A$.  Suppose $C$ is a
finite chain.\footnote{A set $C$ is a \emph{chain} when it is
  nonempty, and all elements $c,d \in C$ are comparable.  Elements $c$
  and $d$ are \emph{comparable} iff $[c \mrel{R} d\ \QOR\ d \mrel{R}
    c]$.  \iffalse A partial order for which every two different
  elements are comparable is called a \emph{linear order}\fi}

\bparts

\ppart\label{hasmax} Prove that $C$ has a maximum element.  \hint Induction on the
size of $C$.
\examspace[3.0in]

\begin{solution}
As hinted, we give a proof by induction on the size of $C$.

\begin{proof}
The induction hypothesis is:
\begin{quote}
$P(n) :=$ If $C$ is a chain of size $n$, then $C$ has a maximum
  element.
\end{quote}

\inductioncase{Base case}: ($n=1$).  The one element is $C$ is the
maximum (ands also minimum) element, bym definition of maximum.

\inductioncase{Induction step}: To prove $P(n+1)$ for $n \geq 1$, let
$C_{n+1}$ be a chain of size $n+1$ and let $x$ be an arbitrary element
in $C_{n+1}$.  Then $C_{n+1}-\set{x}$ is a chain of size $n$, so it
has a maximum element $m$ by induction hypothesis.  Now compare $x$
and $m$.  If $x \mrel{R} m$, then $m$ is the maximum element in
$C_{n+1}$.  On the other hand, $m \mrel{R} x$, then (by transitivity
of $R$), $x$ is a maximum element of $C_{n+1}$.  In any case, 
$C_{n+1}$ has a maximum element, which proves $P(n+1)$.
\end{proof}

\end{solution}

\ppart Conclude that there is a unique sequence of all the elements of
$C$ that is strictly increasing.

\hint Induction on the size of $C$, using part~\eqref{hasmax}.

\examspace[3.0in]

\begin{solution}
As hinted, we give a proof by induction on the size of $C$.

\begin{proof}
The induction hypothesis is:
\begin{quote}
$Q(n) :=$ If $C$ is a chain of size $n$, then there is a unique
  sequence of all the elements of $C$ that is strictly increasing.
\end{quote}

\inductioncase{Base case}: ($n=1$).  Immediate.

\inductioncase{Induction step}: To prove $Q(n+1)$ for $n \geq 1$, let
$C_{n+1}$ be a chain of size $n+1$.  By part~\eqref{hasmax}, $C_{n+1}$
has a maximum element $m$.  Then $C_{n+1}-\set{m}$ is a chain of size
$n$, so there is a unique strictly increasing sequence,
$\overrightarrow{C_{n+1}-\set{m}}$, of all the elements of $C_{n+1}-\set{m}$.
Then $\overrightarrow{C_{n+1}-\set{m}}$ followed by $m$ is a strictly increasing
sequence of the elements of $C_{n+1}$.  Moreover, this sequence is
unique, because any strictly increasing sequence elements in $C_{n+1}$
can only consist of a strictly sequence of elements in
$C_{n+1}-\set{m}$, which is unique by hypothesis, followed by $m$.
\end{proof}

\end{solution}

\iffalse
We can use part~\eqref{hasmax} to construct a strictly increasing
sequence.  Given $C$, remove the maximum element $m$.  This gives a
chain $C'$, whose maximum element is $m_1$.  Remove $m_1$; the
resulting chain $C''$ has a maximum element $m_2$. Continuing this
process until there is only one element $m_n$ in the set yields a
$m_n, m_{n-1},\dots,m_1, m_0$ is a strictly increasing sequence.

As a last note, since $m_i \mrel{R} m_{j}$ for all $j < i$, any other
sequence is not strictly increasing.  To see this, consider any
sequence different than the one given.  Since the sequence is
different, there must be at least one pair of elements $(m_i, m_j)$
such that $m_i$ occurs after $m_j$ but $m_i \mrel{R} m_{j}$. Clearly,
the sequence cannot be strictly increasing.
\fi

\eparts

\end{problem} 


%%%%%%%%%%%%%%%%%%%%%%%%%%%%%%%%%%%%%%%%%%%%%%%%%%%%%%%%%%%%%%%%%%%%%
% Problem ends here
%%%%%%%%%%%%%%%%%%%%%%%%%%%%%%%%%%%%%%%%%%%%%%%%%%%%%%%%%%%%%%%%%%%%%

\endinput
