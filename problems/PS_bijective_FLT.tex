\documentclass[problem]{mcs}

\begin{pcomments}
  \pcomment{PS_bijective_FLT}
  \pcomment{Author: Justin Venezuela (jven@mit.edu)}
  \pcomment{Problem is well-known.}
  \pcomment{slightly edited ARM 11/3/11}
\end{pcomments}

\pkeywords{
  counting
  Fermat's Little Theorem
  coloring
}

%%%%%%%%%%%%%%%%%%%%%%%%%%%%%%%%%%%%%%%%%%%%%%%%%%%%%%%%%%%%%%%%%%%%%
% Problem starts here
%%%%%%%%%%%%%%%%%%%%%%%%%%%%%%%%%%%%%%%%%%%%%%%%%%%%%%%%%%%%%%%%%%%%%

\begin{problem}
Here is a purely combinatorial proof of Fermat's Little
Theorem~\bref{fermat_little}.

\begin{staffnotes}
Not quite: Fermat says
\[
 a^{p-1}\equiv 1\pmod{p}
\]
for $a$ not divisible by $p$.  Fermat follows from what's proved
below, namely,
\[
a^p\equiv a\pmod{p}
\]
for all integers $a$, by cancelling $a$ from both sides when $a$ not
divisible by $p$.
\end{staffnotes}

\bparts

\ppart Suppose there are beads available in $a$ different colors for
some integer $a >1$, and let $p$ be a prime number.  How many
different colored length $p$ sequences of beads can be strung
together?  How many of them contain beads of at least two different
colors?

\begin{solution}
There are $a^p$ different colored sequences of beads by the Product
Rule.  Of these sequences, there is one monochromatic sequence for
each of the $a$ colors, so there are $a^p-a$ sequences of at least two
colors.
\end{solution}

\ppart Make each string of $p$ beads with at least two colors into a
bracelet by tying the two ends of the string together.  Two bracelets
are the same if one can be rotated to yield the other.  (Note,
however, that you \textbf{cannot} "flip" a bracelet over or reflect
it.)  Show that for every bracelet, there are exactly $p$ strings of
beads that yield it.

\hint Both the fact that $p$ is prime and that the bracelet consists
of at least two colors are needed for this to be true..

\begin{solution}
Given a bracelet, choose any of its beads and cut the string right
before it, thereby choosing that bead as the first bead of a
string.   By construction, this string yields the bracelet.  Now consider
the act of moving the first bead to the end of the string.  Repeating
this operation $p$ times, we generate $p$ strings, all of which
clearly yield the bracelet.  It is also clear that no other string can
yield the bracelet.

\begin{staffnotes}
This part was a bit difficult to make precise (if I even succeeded in
being precise).  In any case, I would suggest being lenient in grading
this section, giving full credit so long as it is clear that the
student seems to have an understanding why we require $p$ prime and at
least two colors.  -JVen}
\end{staffnotes}

It suffices to show that each of these $p$ strings is unique.  Indeed,
suppose for sake of contradiction that some two of the strings were
the same.  That is, given one of these strings, applying the above
operation some number of times less than $p$ yields the same string.
Let $k$ be the smallest such number of times the operation must be
applied to yield the same string.  Note that applying the operation
any multiple of $k$ times must also yield the same string.  Let $tk$
be the greatest multiple of $k$ less than $p$.  Note that applying the
operation $p-tk=\rem{p}{k}$ times must also yield the same string.
But $\rem{p}{k}<k$, so we must have $\rem{p}{k}=0$: that is, $k$ is a
factor of $p$.  \textit{Since} $p$ \textit{is prime,} $k$ must be $1$
or $p$.  But \textit{since the string consists of at least two
  colors}, applying the operation once cannot yield the same string,
so that $k\neq 1$.  It follows that $k=p$, which contradicts the fact
that $k<p$.  We conclude that no two of the $p$ strings are the same,
and we're done.
\end{solution}

\ppart Conclude that $p \divides (a^p-a)$

\begin{solution}
This quantity represents the number of indistinguishable bracelets,
which must clearly be an integer.  Thus $p$ divides $a^p-a$, so
$a^p-a\equiv 0 \pmod{p}$.  Therefore $a^p \equiv a\pmod{p}$, as
desired.
\end{solution}

\eparts
\end{problem}

%%%%%%%%%%%%%%%%%%%%%%%%%%%%%%%%%%%%%%%%%%%%%%%%%%%%%%%%%%%%%%%%%%%%%
% Problem ends here
%%%%%%%%%%%%%%%%%%%%%%%%%%%%%%%%%%%%%%%%%%%%%%%%%%%%%%%%%%%%%%%%%%%%%

\endinput
