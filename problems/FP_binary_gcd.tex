\documentclass[problem]{mcs}

\begin{pcomments}
  \pcomment{FP_binary_gcd}
  \pcomment{lighter version of PS_binary_gcd}
  \pcomment{By ARM 4/24/11 based on Shoup}
  \pcomment{revised ARM 3/17/14; 5/13/14}
\end{pcomments}

\pkeywords{
  preserved_invariant
  GCD
  binary_GCD
  state_machine
}

%%%%%%%%%%%%%%%%%%%%%%%%%%%%%%%%%%%%%%%%%%%%%%%%%%%%%%%%%%%%%%%%%%%%%
% Problem starts here
%%%%%%%%%%%%%%%%%%%%%%%%%%%%%%%%%%%%%%%%%%%%%%%%%%%%%%%%%%%%%%%%%%%%%

\begin{problem}
The following \idx{Binary GCD} state machine computes the GCD of
positive integers $a$ and $b$:
\begin{align*}
\text{states}      & \eqdef \naturals^3\\
\text{start state} & \eqdef (a, b, 1)\\
\text{transitions} & \eqdef \text{ if } \min(x,y) > 0, \text{ then }
\end{align*}
\begin{numcases}{(x,  y, e) \movesto}
       (x/2, y/2, 2e) & \text{(if $2 \divides x$ and $2 \divides y$)}\tag{i1}\\%x2y22e
       (x/2, y, e)    & \text{(else if $2 \divides x$)}\tag{i2}\\%x2ye
       (x, y/2, e)    & \text{(else if $2 \divides y$)}\tag{i3}\\%xy2e
       (x-y, y, e)    & \text{(else if $x > y$)}\tag{i4}\\%x-yx>y
       (y-x, x, e)    & \text{(else if $y > x$)}\tag{i5}\\%y-xy>x
       (1,   0, ex)   & \text{(otherwise ($x=y$))}. \tag{i6}%x0ex
\end{numcases}

The predicate
\begin{equation}\tag{Inv1}
\gcd(a,b) = e\gcd(x,y)
\end{equation}
is claimed to be a preserved invariant of this state machine.

\bparts 

\ppart Verify that equation~(Inv1) is a preserved invariant
under transition rules~(i1), and~(i2)\iffalse ,
and~(i4)\fi.  Explictly state any basic GCD properties you
need, but you do not have to prove them.

\examspace[2.0in]

\begin{solution}
To verify preserved invariance, we assume that the invariant holds for
state $(x,y,e)$ and show that if $(x,y,e) \movesto (x',y',e')$, then
$\gcd(a,b) = e'\gcd(x',y')$.  We do so by proof by cases according to which kind of transition occurs.

\inductioncase{Case}~(i1): ($2 \divides x$ and $2 \divides y$).  
In this case, $(x',y',e') = (x/2, y/2, 2e)$.

We use the the gcd property from Lemma~\bref{lem:gcd-hold}
\begin{equation}\tag{agcd}%auavauv
\gcd(au,av) = a\gcd(u,v).
\end{equation}

Now,
\begin{align*}
\gcd(a,b)
  & =  e\gcd(x,y)
        & \text{(by the invariant for $(x,y,e)$)}\\
  & = e 2\gcd(x/2,y/2)
        & \text{(by~(agcd))}\\
  & = e'\gcd(x',y'),
\end{align*}
which shows that the invariant holds for $(x',y',e')$.

\inductioncase{Case}~(i2): ($2 \divides x$ and $2 \nmid y$).
In this case, $(x',y',e') = (x/2,y, e)$.

We use the easily verified fact
\begin{equation}\tag{acancel}%auvvuv
\gcd(au,v) = \gcd(u,v)
\end{equation}
for all $a$'s relatively prime to $v$.

Now,
\begin{align*}
\gcd(a,b)
  & = e\gcd(x,y)
      & \text{(by the invariant for $(x,y,e)$)}\\
  & = e\gcd(x/2,y)
      & \text{(by~(acancel))}\\
  & = e'\gcd(x',y'),
\end{align*}
which shows that the invariant holds for $(x',y',e')$.

\iffalse

\inductioncase{Case}~(i4): ($x>y, 2 \nmid x,\text{ and } 2 \nmid y$)
In this case $(x',y',e') = (x-y,y,e)$.

We use the easily verified fact that
\begin{equation}\tag{gcdu-v}
\gcd(u-v,v) = \gcd(u,v).
\end{equation}

Now,
\begin{align*}
\gcd(a,b)
  & = e\gcd(x,y) & \text{(by the invariant for $(x,y,e)$)}\\
  & = e\gcd(x-y,y) & \text{(by~\eqref{gcdu-v})}\\
  & = e'\gcd(x',y'),
\end{align*}
proving that the invariant holds for $(x',y',e')$.

\inductioncase{Case}~(i6): ($x=y$).  In this case, $(x',y',e') = (1,0,ex)$.

Now we have
\begin{align*}
\gcd(a,b) 
    & = e\gcd(x,y) 
          & \text{(by the invariant for $(x,y,e)$)}\\
    & = e\gcd(x,x) 
          & \text{($x=y$)}\\    
    & = ex\\
    & = ex\gcd(1,0)
          & \text{(since $\gcd(1,0) = 1$)}\\
    & =  e'\gcd(x',y'),
\end{align*}
which shows that the invariant holds for $(x',y',e')$.

Verification of the remaining cases follows similarly.
%by symmetry between $a$ and $b$, and between $x$ and $y$.
\fi

\end{solution}

\ppart Prove that rule~(i1)
\iffalse
\[
(x, y, e) \to (x/2, y/2, 2e)
\]\fi
is never executed after any of the other rules is executed.

\hint A preserved invariant about the parities of $x$ and $y$.

\begin{solution}
We claim that another preserved invariant is
\begin{equation}\tag{noteven}%xybev
\QNOT(2 \divides x\ \QAND\ 2 \divides y),
\end{equation}
that is, at least one of $x$ and $y$ is odd.

To verify this, suppose a state $(x,y,e)$ satisfies~(noteven).
Then the first rule~(i1) will not be executed.

Suppose the second rule~(i2) gets executed, resulting in state
$(x/2, y ,e)$.  Then $x$ must be even and $y$ must be odd; so this
state satisfies~(noteven) since $y$ is odd.

Suppose the fourth rule gets executed, resulting in state $(x-y,y,e)$.  Then if $y$ is odd this
state satisfies~(noteven), and if $y$ is even, then $x$ must be odd, so this state
satisfies~(noteven) because $x-y$ must be odd.

The verification that~(noteven) is preserved by the other rules follows by similar routine
reasoning.

Now, if the first rule is not executed for some state $(x,y,e)$, then~(noteven) must hold,
and since~(noteven) is preserved, we can conclude that the first rule will never be
executed for any subsequent state.
\end{solution}

\eparts

\end{problem}

%%%%%%%%%%%%%%%%%%%%%%%%%%%%%%%%%%%%%%%%%%%%%%%%%%%%%%%%%%%%%%%%%%%%%
% Problem ends here
%%%%%%%%%%%%%%%%%%%%%%%%%%%%%%%%%%%%%%%%%%%%%%%%%%%%%%%%%%%%%%%%%%%%%

\endinput
