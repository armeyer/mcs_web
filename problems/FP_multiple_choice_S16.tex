\documentclass[problem]{mcs}

\begin{pcomments}
  \pcomment{FP_multiple_choice_S16}
  \pcomment{perturbation of FP_multiple_choice_unhidden_fall13}
  \pcomment{revised from FP_multiple_choice_unhidden by ARM 12/13/13}
  \pcomment{overlaps FP_graphs_short_answer}  
\end{pcomments}

\pkeywords{
  isomorphism
  partial_order  
  total_order
  linear
  asymptotic
  vertices
  gcd
  modular
  mod_n
  stable_distribution
  countable
}

%%%%%%%%%%%%%%%%%%%%%%%%%%%%%%%%%%%%%%%%%%%%%%%%%%%%%%%%%%%%%%%%%%%%%
% Problem starts here
%%%%%%%%%%%%%%%%%%%%%%%%%%%%%%%%%%%%%%%%%%%%%%%%%%%%%%%%%%%%%%%%%%%%%

\begin{problem} \mbox{}

\textbf{\large For each of the statements below in parts~\eqref{gcdtf}
  through~\eqref{simgrpart}, write ``V'' for Valid or ``N'' for Not
  Valid, and \emph{provide counterexamples} for those that are not
  valid.}  All variables $a,b,c,m,n$ range over positive integers.

\bparts

\ppart\label{gcdtf} The following statements about the
\textbf{greatest common divisor}:

\begin{itemize}

\item If $\gcd(a, b) \neq 1$ and $\gcd(b, c) \neq 1$, then $\gcd(a, c) \neq 1$. \hfill\examrule

\examspace[0.4in]

\begin{solution}
Not, $a=2, b=2\cdot 3, c=3$
\end{solution}

\item If $a \divides b c$ and $\gcd(a, b) = 1$, then $a \divides c$.\hfill\examrule

\examspace[0.4in]

\begin{solution}
Valid.
\end{solution}

\item $\gcd(a^n,b^n) =  (\gcd(a,b))^n$  

\examspace[0.4in]

\begin{solution}
Valid.
\end{solution}

\item $\gcd(ab, ac) = a \gcd(b, c)$.\hfill\examrule

\examspace[0.4in]

\begin{solution}
Valid.
\end{solution}

%\item $\gcd(1 + a, 1 + b) = 1 + \gcd(a, b)$.  \instatements{\hfill 
%Valid \qquad  N} \examspace[0.4in]
%\begin{solution}
%N, $a=1,b=2$.
%\end{solution}

\item If some integer linear combination of $a$ and $b$ equals 1, then
  so does some integer linear combination of $a$ and $b^2$. \hfill\examrule

\examspace[0.4in]

\begin{solution}
Valid.
\end{solution}

\item If some integer linear combination of $a$ and $b$ equals 2, then
 so does some integer linear combination of $a$ and
  $b^2$. \hfill\examrule

\examspace[0.4in]

\begin{solution}
Not, let $a=4, b=2$.
\end{solution}

\end{itemize}

\ppart The following statements about \textbf{congruence modulo} $n$, where $n > 1$:

\begin{itemize}

\item If $a c \equiv b c \pmod{n}$ and $n$ does not divide $c$, then $a \equiv b \pmod{n}$.
\hfill\examrule

\examspace[0.4in]

\begin{solution}
Not.  Need $c$ relatively prime to $n$.  Counterexample:
$n=2 \cdot 3, a=0, b=2, c=3$
\end{solution}

%% \item If $a \equiv b \pmod{\phi(n)}$ for $a, b > 0$, then $c^a \equiv c^b
%% \pmod{n}$.  \hfill\examrule
%\examspace[0.4in]

%% \begin{solution}
%%   N.  Need $c$ relatively prime to $n$.  Counterexample:
%%   $n=4$, so $\phi(n) = 2$; $a=1, b=3$, so $a \equiv b
%%   \pmod \phi(n)$, $c = 2$, so $c^a = 2 \not\equiv 0 = c^b \pmod 4$.
%% \end{solution}

%% \item If $a \equiv b \pmod{n}$, then $P(a) \equiv P(b) \pmod{n}$ for any
%% polynomial $P(x)$ with integer coefficients.
%%  \hfill\examrule
%\examspace[0.4in]
%% \begin{solution}
%% true
%% \end{solution}


\item If $a \equiv b \pmod{nm}$, then $a \equiv b \pmod{n}$, for $m,n > 1$. \hfill\examrule

\examspace[0.4in]

\begin{solution}
Valid.
\end{solution}

\item For relatively prime $m,n >1$,\\
$[a \equiv b \pmod{m} \QAND a \equiv b \pmod{n}] \QIFF [a \equiv b \pmod{mn}]$ \hfill\examrule

\examspace[0.4in]

\begin{solution}
Valid.
\end{solution}

%% \item Assuming $a,b$ have inverses modulo $n$, if $a^{-1} \equiv
%%   b^{-1} \pmod{n}$, then $a \equiv b \pmod{n}$. \instatements{\hfill Valid
%%   \qquad N} \examspace[0.4in]

%% \begin{solution}
%% Valid
%% \end{solution}

\item If $a,b >1$, then

 [$a$ has a multiplicative inverse mod $b$ iff $b$ has a
   multiplicative inverse mod $a$].  \hfill\examrule

\examspace[0.4in]

\begin{solution}
Valid.  $a$ has a multiplicative inverse mod $b$ iff
$a,b$ relatively prime iff $b$ has a multiplicative inverse mod $a$.
\end{solution}

%% \item If $\gcd(a,n)=1$, then $a^{n-1} \equiv 1 \pmod{n}$. \instatements{\hfill
%%   Valid \qquad N} \examspace[0.4in]

%% \begin{solution}
%% N  Let $a=5$, $n =6$.
%%\end{solution}

\end{itemize}

\examspace

%% \ppart The following statements about \textbf{trees}:

%% \begin{itemize}

%% \item Any connected subgraph is a tree.  \instatements{\hfill Valid \qquad
%%   N} \examspace[0.4in]
%% \begin{solution}
%% Valid.
%% \end{solution}

%% \item Adding an edge between two nonadjacent vertices creates a
%%   cycle. \hfill\examrule
%\examspace[0.4in]
%% \begin{solution}
%% Valid.
%% \end{solution}

%% %% \item The number of vertices is one less than twice the number of
%% %%   leaves.  \instatements{\hfill Valid \qquad
%% %%   N} \examspace[0.4in]
%% %\begin{solution}
%% %  N.  This property holds for full binary trees, but not
%% %  in general.  A tree with two vertices is a counterexample.
%% %\end{solution}

%% %% \item The number of leaves in a tree is not equal to the number of
%% %%   non-leaf vertices.  \instatements{\hfill Valid \qquad
%% %%   N} \examspace[0.4in]
%% %% \begin{solution}
%% %%   N.  A line graph with 4 vertices has 2 non-leaf
%% %%   vertices and 2 leaves.
%% %% \end{solution}

%% \item Any subgraph of a tree is a tree.  \instatements{\hfill Valid \qquad
%%   N} \examspace[0.4in]
%% \begin{solution}
%% N.  Choose a subgraph with 2 vertices and no edges.
%% \end{solution}

%% \item The number of vertices in a tree is one less than the number of
%%   edges.  \hfill\examrule
%\examspace[0.4in]

%% \begin{solution}
%% N.  This got ``edges'' and ``vertices'' reversed.  Every tree is
%% a counterexample.
%% \end{solution}

%% \end{itemize}

\ppart\label{simgrpart} The following statements about a finite
\textbf{simple graph} $G$:

\begin{itemize}
\item  $G$ has a spanning tree.\hfill\examrule

\examspace[0.4in]

\begin{solution}
Not.  Any disconnected graph is a counterexample.
\end{solution}

\item $\card{\vertices{G}} = O(\card{\edges{G}})$ for connected $G$.\hfill\examrule

\examspace[0.4in]

\begin{solution}
Valid.  To be connected, there must be at least $\card{\vertices{G}}
-1$ edges.
\end{solution}

\item $\card{\edges{G}} = O(\card{\vertices{G}})$.\hfill\examrule

\examspace[0.4in]

\begin{solution}
Not.  $\card{\vertices{K_n}} = n = o(n^2)$, but
  $\card{\edges{K_n}} = \Theta(n^2)$.
\end{solution}

\examspace[0.4in]

%% \item The chromatic number $\chi(G) \leq \max\set{\degr{v} \suchthat
%%    v \in \vertices{G}}$. \hfill\examrule

%% \examspace[0.4in]

%% \begin{solution}
%% N.  $\chi(K_n) = n > n-1 = \text{ max degree of } v \in \vertices{K_n}$
%% \end{solution}

%% \item $\card{\edges{G}} = O(\chi(G))$, where $\chi(G)$ is the
%%   chromatic number of $G$. \instatements{\hfill Valid \qquad
%%     N} \examspace[0.4in]
%% \begin{solution}
%% N.   $\chi(K_n) = n$, but $\card{\edges{K_n}} = \Theta(n^2)$.
%% \end{solution}

\end{itemize}

\ppart \textbf{Circle} all the properties below that are preserved under
\textbf{graph isomorphism}:

\begin{itemize}

%\item There is a cycle that includes all the vertices.

\item The vertices can be numbered 1 through 7.

%\item The graph remains connected if any two edges are removed.

\item There is a cut edge.

%\item The negation of a property that is preserved under isomorphism.

\item Two edges are of equal length.

%\item There are exactly two spanning trees.

\item The $\QXOR$ of two properties that are preserved under isomorphism.

%\item The graph remains connected if a vertex is removed.

\end{itemize}

\begin{solution}
All are preserved except ``Two edges are of equal length.''
\end{solution}

\ppart A \term{sink} in a random walk digraph is a vertex with no
edges leaving it.  \textbf{Circle} all of the following assertions
that are true. \iffalse
of \textbf{stable distributions on random walk graphs}
with exactly two sinks:\fi

\begin{itemize}

\item There is a 2-sink random walk graph without any stable distribution.

\item There is a 2-sink random walk graph with a unique stable distribution.

%\item there are exacty two

%\item there may be a countably infinite number

\item There is a 2-sink random walk graph with an uncountable number stable distributions.

\item Every 2-sink random walk graph has an uncountable number stable distributions.

\end{itemize}

\begin{solution}
The first two choices are false, and the last two are true.
That's because a distribution in which one sink has probability $r \in
[0,1] \subseteq \reals$ and the other sink has probability $1-r$ is
stable, and there are an is uncountable number of real numbers in $[0,1]$.
\end{solution}

\eparts

\end{problem}

\endinput
