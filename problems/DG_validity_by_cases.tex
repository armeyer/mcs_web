\documentclass[problem]{mcs}

\begin{pcomments}
  \pcomment{DG_validity_by_cases}
  \pcomment{variant of 2003 final problem 1(a)}
\end{pcomments}

\pkeywords{
  validity
  cases
}

%%%%%%%%%%%%%%%%%%%%%%%%%%%%%%%%%%%%%%%%%%%%%%%%%%%%%%%%%%%%%%%%%%%%%
% Problem starts here
%%%%%%%%%%%%%%%%%%%%%%%%%%%%%%%%%%%%%%%%%%%%%%%%%%%%%%%%%%%%%%%%%%%%%

\begin{problem}%\label{generprob}
The formula
\begin{align*}
(X \QAND & \QNOT(\bar{X} \QIMPLIES Y) \QAND W )\\
         & \QIMPLIES\\
(Z \QAND &  U \QAND V \QAND R \QAND S \QAND T \QAND A \QAND B \QAND C \QAND D)
\end{align*}
is turns out to be valid.

\begin{problemparts}
\problempart Explain why verifying the validity of this formula
\emph{by truth table} would be very hard for one person to do with
pencil and paper (no computers).

\begin{solution}
The number of entries in a truth table here would be $2^{14}$ or about
16,000 since there are 14 variables.  This could take weeks for one
person to do by hand.
\end{solution}

\problempart Verify that the formula is valid by cases according to
the truth value of $X$.  Briefly explain your reasoning in each case.

\begin{solution}
\begin{proof}
The proof is by cases on the value of $X$.

\inductioncase{Case}: ($X$ is \True).  Since the implication
$\overline{\True)} \QIMPLIES Y$ is \True, the expression
$\QNOT(\overline{X} \QIMPLIES Y)$ evaluates to \False.  So the
hypothesis (upper formula) of the main implication is \False, making
the whole implication formula \True.  So the formula is \True\ in all
assignments with $X$ assigned \True.

\inductioncase{Case}: ($X$ is \False).  The hypothesis (upper formula)
side of the formula is now equivalent to $(\False\ \QAND \dots)$, and
so is immediately \False.  As in the previous case, this makes the
whole implication formula \True.  So the formula is \True\ in all
assignments with $X$ assigned \False.

Since $X$ must be \True\ or \False\ in any truth assignment, the
formula is \True\ in any case.  That is, it is valid.
\end{proof}

\end{solution}
\end{problemparts}

\end{problem}

%%%%%%%%%%%%%%%%%%%%%%%%%%%%%%%%%%%%%%%%%%%%%%%%%%%%%%%%%%%%%%%%%%%%%
% Problem ends here
%%%%%%%%%%%%%%%%%%%%%%%%%%%%%%%%%%%%%%%%%%%%%%%%%%%%%%%%%%%%%%%%%%%%%

\endinput
