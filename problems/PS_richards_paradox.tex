\documentclass[problem]{mcs}

\newcommand{\asciibet}{\text{ASCII}}
\newcommand{\asciistr}{\strings{\asciibet}}
\newcommand{\numstr}{\strings{ \text{ASCII\_NUM} }}

\begin{pcomments}
  \pcomment{PS_infinite_sets_and_number_theory}
  \pcomment{from: F13.ps4}
\end{pcomments}

\pkeywords{
  infinite sets
  number theory
}

%%%%%%%%%%%%%%%%%%%%%%%%%%%%%%%%%%%%%%%%%%%%%%%%%%%%%%%%%%%%%%%%%%%%%
% Problem starts here
%%%%%%%%%%%%%%%%%%%%%%%%%%%%%%%%%%%%%%%%%%%%%%%%%%%%%%%%%%%%%%%%%%%%%

\begin{problem} Ben Bitdiddle is building an online system to process orders for a restaurant.  Ben's brilliant idea is to have customers enter their orders online ahead of time so their food is ready when they arrive.  Ben would like his online system to validate user input so that customers are immediately aware of any problems with their orders.

Ben wants to make his system as user-friendly as possible.  In addition to specifying the items they want to order, customers should be able to input how many of each item they want---in plain English!  For example, a customer should be able to order ``five factorial'' french fries and ``positive square root of four'' hamburgers.  However, a customer should not be able to order ``George Washington'' slices of pizza (because ``George Washington'' does not describe a quantity).  Ben is faced with a challenge---how can a computer program determine whether an English phrase describes a number?

$\asciibet$ is a 256-character alphabet that is often used in computer software.  Let $\asciistr$ be the set of (finite) strings of $\asciibet$ characters, as described in Section~\bref{halting_sec} of the textbook.  Let $\numstr$ be the subset of $\asciistr$ which are English descriptions of nonnegative real numbers (not necessarily integers).

\bparts

\ppart\label{enumerate_numstr} Propose a bijection between $\mathbb{N}$ and $\numstr$.

\begin{solution}
One way to enumerate the strings in $\numstr$ is to sort them first by length and then alphabetically.
\end{solution}

\ppart\label{diagonalization} Consider the numbers $S$ identified by the strings in $\numstr$.  In plain English, describe a number which is different from all of them.

\begin{solution}
We use diagonalization.  Order the numbers in $S$ according to the bijection from part~\eqref{enumerate_numstr}.  Construct a new number $q$ according to the following:  the integer part of $q$ is $1$, and the $n^{\text{th}}$ digit in the decimal part of $q$ is $2$ if the corresponding digit in the $n^{\text{th}}$ member of $S$ is not $2$, and $1$ otherwise.  $q$ differs from every number in $S$ at some digit, so $q \notin S$.
\end{solution}

\ppart\label{no_algorithm_for_numstr} Conclude that $\numstr$ is not a valid set, and that Ben's efforts are futile.

\begin{solution}
The solution to part~\eqref{diagonalization} is an English description of a number, so it must be in $\numstr$.  That number $q$ must therefore be in $S$, but we have just shown otherwise---a contradiction.  So our assumption that $\numstr$ is a valid set is false, and there is no algorithm which determines whether a string is a description of a number.
\end{solution}

\eparts
\end{problem}

%%%%%%%%%%%%%%%%%%%%%%%%%%%%%%%%%%%%%%%%%%%%%%%%%%%%%%%%%%%%%%%%%%%%%
% Problem ends here
%%%%%%%%%%%%%%%%%%%%%%%%%%%%%%%%%%%%%%%%%%%%%%%%%%%%%%%%%%%%%%%%%%%%%

\endinput
