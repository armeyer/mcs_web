\documentclass[problem]{mcs}

\begin{pcomments}
  \pcomment{CP_recursive_prenex}
  \pcomment{ARM 3/4/16}
\end{pcomments}

\pkeywords{
  predicate_formula
  recursive
  free_varianble
  structural_induction
}

\newcommand{\fvrbls}{\ensuremath{\mathop{\mathrm{fvar}}}}
\newcommand{\ANDsym}{\ensuremath{\mathbin{\mathrm{\textbf{And}}}}}
\newcommand{\NOTsym}{\ensuremath{\mathop{\mathrm{\textbf{Not}}}}}
\newcommand{\IMPsym}{\ensuremath{\mathbin{\mathrm{\textbf{Implies}}}}}
\newcommand{\xbod}{\ensuremath{\mathop{\mathrm{bod}}}}
\newcommand{\qnts}{\ensuremath{\mathop{\mathrm{qnt}}}}

\begin{problem}

\bparts

\ppart\label{recpred} Give a recursive definition of the set of all
predicate formulas whose only the propositional connectives are
$\ANDsym, \NOTsym$, and $\IMPsym$, and whose only ``atomic'' formulas
are of the form $R(x)$ and $S(x,y)$ where $R$ and $S$ are fixed
predicate symbols and $x,y$ may be any variable symbols, not
necessarily distinct.

\begin{solution}

\inductioncase{Base cases} The ``atomic formula'' described above are
predicate formula.

\inductioncase{Constructor step}:  If $F$ and $G$ are predicate formulas, then so are
\begin{align*}
 \NOTsym(F),
 (F \ANDsym G),
 (F \IMPsym G),
 (\exists x.\, F)
 (\forall x.\, F)
\end{align*}

\end{solution}

\ppart Based on part~\eqref{recpred}, give a recursive definition of
the set $\fvrbls(F)$ of \emph{free variables} in a predicate formula
$F$.  (A variable $x$ is free if it has an occurrence that is not
inside any subformula of the form $\exists x.[\dots]$ or $\forall
x.[\dots]$.)

\begin{solution}
\inductioncase{Base cases} 
\begin{align*}
\fvrbls(R(x))& \eqdef \set{x},\\
\fvrbls(S(x,y))& \eqdef \set{x,y}.
\end{align*}

\inductioncase{Constructor cases}
\begin{align*}
\fvrbls(\NOTsym(F))   & \eqdef \fvrbls(F),\\
\fvrbls((F \ANDsym G)) & \eqdef \fvrbls(F) \union \fvrbls(G),\\
\fvrbls((F \IMPsym G)) & \eqdef \fvrbls(F) \union \fvrbls(G),\\
\fvrbls(\exists x.\, F) & \eqdef \fvrbls(F) - \set{x},\\
\fvrbls(\forall x.\, F) &  \eqdef \fvrbls(F) - \set{x}.
\end{align*}

\end{solution}

\ppart Assume $F$ has the property that for each variable $x$, there
is at most one occurrence of either a ``$\forall x$'' or an ``$\exists
x$,'' and that if $x$ is a free variable, there is no occurrence of a
``$\forall x$'' or an ``$\exists x$'' anywhere else in the formula.
(That is, $F$ satisfies the unique variable convention of
Problem~\bref{CP_variable_convention}.)  Define a recursive procedure
to convert any predicate formula $F$ into an equivalent \emph{prenex
  formula} of the form
\[
\qnts(F).\,\xbod(F),
\]
where $\xbod(F)$ is a quantier-free formula, and $\qnts{F}$ is a
(possibly empty) sequence of quantifiers
\[
Q_1 x_1.\, Q_2 x_2.\, \dots Q_n x_n.\, 
\]
where $Q_i$ is ``$\forall$'' or ``$\exists$,'' and the $x_i$'s are
distinct variables.  You may use the notation $\overline{\qnts(F)}$
for the sequence
\[
\overline{Q_1} x_1.\, \overline{Q_2} x_2.\, \dots \overline{Q_n} x_n.\, 
\]
were $\overline{\forall} \eqdef \exists$ and $\overline{\exists}
\eqdef \forall$.

For example, the variable convention ensures that if $(Q x. F)$ is a
subformula somewhere, then there are no free occurrences of $x$
anywhere.  Therefore,
\[
(Q x. F) \ANDsym G\quad  \text{is equivalent to} \quad Q x. (F \ANDsym G),
\]
because we can think of $G$ as simply being \True\ or \False.  This
justfies the following definitions of $\xbod$ and $\qnts$ for the
\ANDsym\ case
\begin{align*}
\xbod((F \ANDsym G)) & \eqdef (\xbod(F) \ANDsym \xbod(G)),\\
\qnts((F \ANDsym G)) & \eqdef \qnts(F),\qnts(G)
\end{align*}

which explains why the definition of $\qnts((F \ANDsym G))$ and
$\xbod((F \ANDsym G))$ preserves equivalence. 


\begin{solution}
\inductioncase{Base cases} 
\begin{align*}
\xbod(R(x)) & \eqdef R(x),\\
\qnts(R(x)) & \eqdef \emptystring,\\
\xbod(S(x,y))& \eqdef S(x,y),\\
\qnts(S(x,y)) & \eqdef \emptystring.
\end{align*}

\inductioncase{Constructor cases}
\begin{align*}
\xbod(\NOTsym(F))   & \eqdef \NOTsym(\xbod(F)),\\
\qnts(\NOTsym(F)))   & \eqdef \overline{\qnts}(F),\\
\xbod((F \ANDsym G)) & \eqdef (\xbod(F) \ANDsym \xbod(G)),\\
\qnts((F \ANDsym G)) & \eqdef \qnts(F),\qnts(G)\\
\xbod((F \IMPsym G)) & \eqdef \xbod(F) \IMPsym \xbod(G),\\
\qnts((F \IMPsym G)) & \eqdef \overline{\qnts(F)},\qnts(G)\\
\xbod(\exists x.\, F) & \eqdef \xbod(F)\\
\qnts(\exists x.\, F) & \eqdef \exists x,\qnts(F)\\
\xbod(\forall x.\, F) &  \eqdef \xbod(F)\\
\qnts(\forall x.\, F) & \eqdef \forall x,\qnts(F).
\end{align*}

\end{solution}
\begin{staffnotes}
The variable convention justifies the other cases similarly.
\end{staffnotes}

\eparts

\end{problem}

\endinput 
