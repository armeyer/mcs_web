\documentclass[problem]{mcs}

\begin{pcomments}
  \pcomment{CP_cold_cows_markov}
  \pcomment{revised by ARM 12/5/09}
  \pcomment{from: S06.cp13w}
%  \pcomment{}
%  \pcomment{}
\end{pcomments}

\pkeywords{
  Markov_bound
  deviation
}

%%%%%%%%%%%%%%%%%%%%%%%%%%%%%%%%%%%%%%%%%%%%%%%%%%%%%%%%%%%%%%%%%%%%%
% Problem starts here
%%%%%%%%%%%%%%%%%%%%%%%%%%%%%%%%%%%%%%%%%%%%%%%%%%%%%%%%%%%%%%%%%%%%%


\begin{problem}
A herd of cows is stricken by an outbreak of \emph{cold cow disease}.  The
disease lowers the normal body temperature of a cow, and a cow will die if
its temperature goes below 90 degrees F.  The disease epidemic is so
intense that it lowered the average temperature of the herd to $85$
degrees.  Body temperatures as low as $70$ degrees, \textbf{but no lower},
were actually found in the herd.

\bparts

\ppart\label{3/4cows} Prove that at most 3/4 of the cows could have survived.

\hint Let $T$ be the temperature of a random cow.  Make use of Markov's
bound.

\begin{solution}
  Let $T$ be the temperature of a random cow.  Then the fraction of cows
  that survive is the probability that $T \geq 90$, and $\expect{T}$ is the
  average temperature of the herd.

  Applying Markov's Bound to $T$:
  \[
  \prob{T \geq 90} = \leq \frac{\expect{T}}{90} = \frac{85}{90} =
  \frac{17}{18}\, .
  \]
  But $17/18 > 3/4$, so this bound is not good enough.

  Instead, we apply Markov's Bound to $T-70$:
\[
\prob{T \geq 90} = \prob{T-70 \geq 20} \leq \frac{\expect{T-70}}{20} = (85-70)/20 = 3/4.
\]
\end{solution}

%\vspace{3in}

\ppart Suppose there are 400 cows in the herd.  Show that the bound of
part~\eqref{3/4cows} is best possible by giving an example set of
temperatures for the cows so that the average herd temperature is 85, and
with probability 3/4, a randomly chosen cow will have a high enough
temperature to survive.

\begin{solution}
  Let 100 cows have temperature 70 degrees and 300 have 90 degrees.  So
  the probability that a random cow has a high enough temperature to
  survive is exactly 3/4.  Also, the mean temperature is
\[
(1/4)70 + (3/4)90 = 85.
\]
So this distribution of temperatures satisfies the conditions under which
the Markov bound implies that the probability of having a high enough
temperature to survive cannot be larger than 3/4.
\end{solution}

\eparts

\end{problem}

%%%%%%%%%%%%%%%%%%%%%%%%%%%%%%%%%%%%%%%%%%%%%%%%%%%%%%%%%%%%%%%%%%%%%
% Problem ends here
%%%%%%%%%%%%%%%%%%%%%%%%%%%%%%%%%%%%%%%%%%%%%%%%%%%%%%%%%%%%%%%%%%%%%

\endinput
