\documentclass[problem]{mcs}

\begin{pcomments}
  \pcomment{MQ_tree_plus_edge}
  \pcomment{small part of CP_min_weight_edge}
  \pcomment{by ARM 3/16/13}
\end{pcomments}

\pkeywords{
  spanning_tree
  cycle
  cut_edge
  connected
  acyclic
}

%%%%%%%%%%%%%%%%%%%%%%%%%%%%%%%%%%%%%%%%%%%%%%%%%%%%%%%%%%%%%%%%%%%%%
% Problem starts here
%%%%%%%%%%%%%%%%%%%%%%%%%%%%%%%%%%%%%%%%%%%%%%%%%%%%%%%%%%%%%%%%%%%%%

\begin{problem}
\bparts

\ppart\label{T+e} Let $T$ be a tree and $e$ a new edge between two
vertices of $T$.  Explain why $T+e$ must contain a cycle.

\examspace[3in]

\begin{solution}
Since $T$ is connected, $T+e$ must also be connected.  Since a tree by
definition is a connected, acyclic graph, if we show that $T+e$ is not
a tree, it must not be acyclic.

There are lots of ways to show that $T+e$ is not a tree based on the
various characterizations of trees.

For example, using the fact that trees are graphs with unique paths
between any two vertices, $T+e$ has two different paths between the
endpoints of $e$, namely the path in $T$ between them and the length
one path consiting of edge $e$.  

Alternatively, using the fact that a finite tree is a connected graph
with one fewer edges than vertices, $T+e$ has too many edges to still be a tree.
\end{solution}

\ppart Conclude that $T+e$ must have another spanning tree besides
$T$.

\begin{solution}
By part~\eqref{T+e}, $T+e$ has a cycle.  Since there is no cycle in
$T$, the cycle must include the edge $e$.  Also, the cycle must, by
definition of cycle, contain an edge $f \neq e$.  Removing an edge on
a cycle does not disconnect any vertices (we know edges on cycles are
not \idx{cut-edges}), so $T+e-f$ is a connected spanning graph with
the same number of edges as $T$, and so is another spanning tree of
$T+e$.
\end{solution}

\eparts

\end{problem}

\endinput
