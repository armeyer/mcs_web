\documentclass[problem]{mcs}

\begin{pcomments}
  \pcomment{MQ_committees}
\end{pcomments}

\pkeywords{
}

%%%%%%%%%%%%%%%%%%%%%%%%%%%%%%%%%%%%%%%%%%%%%%%%%%%%%%%%%%%%%%%%%%%%%
% Problem starts here
%%%%%%%%%%%%%%%%%%%%%%%%%%%%%%%%%%%%%%%%%%%%%%%%%%%%%%%%%%%%%%%%%%%%%

\begin{problem}
Twenty people work at CantorCorp, a small, unsuccessful start-up.  Two
six-person committees are to be formed.  The first will work only to prove
the Continuum Hypothesis.  The other will work only to disprove it.  Each employee
will be a member of at most one committee.  However:
\begin{itemize}
\item Two of the workers, Aleph and Beth, refuse to serve on the same committee.
\item Beth also refuses to serve on a committee if both Ferdinand and Georg are on it.
\end{itemize}
In how many ways can these committees be formed so as to keep all these people happy?
(The people in a committee are all of equal standing.)
\begin{solution}
First, number the employees $1,2,\ldots,20$.  Now, represent each ``way to form a committee''
by an ordered pair of sets, $S_1$ and $S_2$: $\paren{S_1,S_2}$.  Call each such pair of sets
a ``setpair''.  $S_1$ contains the numbers of the six employees assigned to the first committee,
while  $S_2$ contains the numbers of the six employees who make up the second committee.  Now, let:
\begin{itemize}
\item $P$ denote the set of setpairs that have Aleph and Beth serving on the same committee.
\item $Q$ denote the set of setpairs that have Beth, Ferdinand, and Georg all serving on the same committee.
\item $D$ denote the set of all possible setpairs.
\item $R$ denote the set of all setpairs that correspond to acceptable committee selections.
\end{itemize} 

Now, $P\cap Q$ is the set of setpairs in which Aleph, Beth, Ferdinand, and Georg all serve on the same committee, while 
$P\cup Q$ is the set of all setpairs in which people who refuse to serve together have ended up in the same 
committee.  Clearly, $D = R\cup \paren{P\cup Q}$, where $R$ and $P\cup Q$ are disjoint, so by the Sum Rule, $|D| = |R| + |P\cup Q|$.

Let's start with $P$: Either Aleph and Beth serve on the first committee or the second.  If the first, then
there are 18 employees left to choose from, and four spots left in the committee: there are $\displaystyle\binom{18}{4}$
ways to do build the first committee.  For each such selection of the first committee, there are 14 employees left for
the second committee.  Of these, six must be chosen.  There are $\displaystyle\binom{14}{6}$ ways to do this.  So altogether
by the Generalized Product Rule, there are $\displaystyle\binom{18}{4}\binom{14}{6}$ ways to choose the committees such that
Aleph and Beth serve together on the first committee.  Similarly, there are  $\displaystyle\binom{18}{4}\binom{14}{6}$ ways 
to choose the committees such that Aleph and Beth serve together on the second committee.  So the number of setpairs in which
Aleph and Beth serve on the same committee is just, by the Sum Rule, $\displaystyle |P|=2\binom{18}{4}\binom{14}{6}$.  (It
should be obvious that the disjointness requirement needed to apply the Sum Rule is met here.)

Similarly, we find that $\displaystyle |Q|=2\binom{17}{3}\binom{14}{6}$ and $\displaystyle |P\cap Q|=2\binom{16}{2}\binom{14}{6}$.

And with no restrictions, have $|D| = \binom{20}{6}\binom{14}{6}$.

Applying inclusion/exclusion, have
\begin{align*}
|P\cup Q| &= |P| + |Q| - |P\cap Q| \\
          &= 2\binom{18}{4}\binom{14}{6} + 2\binom{17}{3}\binom{14}{6} - 2\binom{16}{2}\binom{14}{6} \\
          & = 2\paren{\binom{18}{4}+\binom{17}{3}-\binom{16}{2}}\binom{14}{6}
\end{align*}

So, finally, the number of ways to choose committees so that no one's wishes have been neglected is:
\begin{align*}
|R| &= |D| - |P\cup Q| \\
          &= \binom{20}{6}\binom{14}{6} - 2\binom{18}{4}\binom{14}{6} - 2\binom{17}{3}\binom{14}{6} + 2\binom{16}{2}\binom{14}{6} \\
          & = \paren{\binom{20}{6}-2\binom{18}{4}-2\binom{17}{3}+2\binom{16}{2}}\binom{14}{6}
\end{align*}

\end{solution}

\textbf{THINGS WE CAN DO TO MAKE THIS SLIPPERIER:}
\begin{itemize}
\item Make the committees different sizes.
\item Have another guy, Ludwig, who's an optimist an therefore refuses to serve on the second committee.
\item Have two people who refuse to be separated.
\end{itemize}

\end{problem}

%%%%%%%%%%%%%%%%%%%%%%%%%%%%%%%%%%%%%%%%%%%%%%%%%%%%%%%%%%%%%%%%%%%%%
% Problem ends here
%%%%%%%%%%%%%%%%%%%%%%%%%%%%%%%%%%%%%%%%%%%%%%%%%%%%%%%%%%%%%%%%%%%%%
