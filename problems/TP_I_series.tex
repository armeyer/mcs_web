\documentclass[problem]{mcs}

\begin{pcomments}
  \pcomment{TP_I_series}
  \pcomment{by ARM 10/3/12}
\end{pcomments}

\pkeywords{
  ring
  inverse
  power_series
  generating_function
}

%%%%%%%%%%%%%%%%%%%%%%%%%%%%%%%%%%%%%%%%%%%%%%%%%%%%%%%%%%%%%%%%%%%%%
% Problem starts here
%%%%%%%%%%%%%%%%%%%%%%%%%%%%%%%%%%%%%%%%%%%%%%%%%%%%%%%%%%%%%%%%%%%%%

\begin{problem}
In the context of formal series, a number $r$ may be used
to indicate the sequence
\[
(r,0,0,\dots,0,\dots).
\]
For example the number 1 may be used to indicate the identity series,
$I$, and 0 may indicate to the zero series, $Z$.  Whether ``$r$''
means the number or the sequence is supposed to be clear from context.

Verify that in the ring of formal power series, 
\[
r \otimes (g_0,g_1,g_2, \dots) = (rg_0, rg_1, rg_2, \dots).
\]
In particular,
\[
-(g_0,g_1,g_2, \dots) = -1 \otimes (g_0,g_1,g_2, \dots).
\]

\begin{solution}
\TBA{.}
\end{solution}
\end{problem}


%%%%%%%%%%%%%%%%%%%%%%%%%%%%%%%%%%%%%%%%%%%%%%%%%%%%%%%%%%%%%%%%%%%%%
% Problem ends here
%%%%%%%%%%%%%%%%%%%%%%%%%%%%%%%%%%%%%%%%%%%%%%%%%%%%%%%%%%%%%%%%%%%%%
\endinput





