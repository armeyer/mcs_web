\documentclass[problem]{mcs}

\begin{pcomments}
  \pcomment{FP_propositional_formulas_matching}
  \pcomment{forked from MQ_implies_relation_on_propositional_formulas_modified}
  \pcomment{Original: ARM 9/20/11}
  \pcomment{CH, Spring '14}
\end{pcomments}

\pkeywords{
  propositional formula
  implies
  valid
  relation
  codomain
  image
  inverse
  graph_of_relation
  subset
  matching
  bottleneck
}

%%%%%%%%%%%%%%%%%%%%%%%%%%%%%%%%%%%%%%%%%%%%%%%%%%%%%%%%%%%%%%%%%%%%%
% Problem starts here
%%%%%%%%%%%%%%%%%%%%%%%%%%%%%%%%%%%%%%%%%%%%%%%%%%%%%%%%%%%%%%%%%%%%%

\begin{problem} \mbox{}
  Let $A$ be the set of five propositional formulas shown below
  on the left, and let $C$ be the set of three propositional formulas on
  the right.  Let $R$ be the ``implies'' binary relation from $A$ to $C$ which
  is defined by the rule
\[
F \mrel{R} G \qiff [\text{the formula } (F \QIMPLIES G) \text{ is valid}].
\]
For example, $(P \QAND Q) \mrel{R} P$, because the formula $(P \QAND
Q)$ does imply $P$.  Also, it is not true that $(P \QOR Q) \mrel{R} P$
since $(P \QOR Q)$ does not imply $P$.

  \bparts
  \ppart Fill in the arrows so the following figure describes the graph of
  the relation, $R$:

\[\begin{array}{lcr}
A & \hspace{1in} \text{arrows} \hspace{1in} & C\\
\hline
&\\
M\\
&\\
                                  && Q\\
&\\
P \QOR Q\\
&\\
                                  && \bar{P} \QOR \bar{Q}\\
&\\
P \QXOR Q\\
&\\
                                  && M \QAND (P \QIMPLIES M)\\
&\\
P \QAND Q\\
&\\
&\\
\QNOT(P \QAND Q)\\
&\\
\end{array}\]

\begin{solution}
Four arrows for $R$:
\begin{align*}
M & \qiff & M \QAND (P \QIMPLIES M)\\
P \QXOR Q & \qimplies & \bar{P} \QOR \bar{Q}\\
P \QAND Q & \qimplies & Q\\
\QNOT(P \QAND Q) & \qiff & \bar{P} \QOR \bar{Q}
\end{align*}
\end{solution}

\ppart Circle the properties below possessed by the
 relation $R$:
  \[
  \begin{array}{ccccc}
  \mbox{FUNCTION~~} & 
  \mbox{~~TOTAL~~} &
  \mbox{~~INJECTIVE~~} &
  \mbox{~~SURJECTIVE~~} &
  \mbox{~~BIJECTIVE} 
  \end{array}
  \]

\begin{solution}
From part (a),  the ``implies'' relation, $R$, is a surjective function.  
\end{solution}

  \ppart  Circle the properties below possessed by the relation $\inv{R}$:
  \[
  \begin{array}{ccccc}
  \mbox{FUNCTION~~} & 
  \mbox{~~TOTAL~~} &
  \mbox{~~INJECTIVE~~} &
  \mbox{~~SURJECTIVE~~} &
  \mbox{~~BIJECTIVE~}
  \end{array}
  \]

\begin{solution}
From part (b), the inverse relation, $\inv{R}$, is a total injection.
\end{solution}


  \eparts

\end{problem}

%%%%%%%%%%%%%%%%%%%%%%%%%%%%%%%%%%%%%%%%%%%%%%%%%%%%%%%%%%%%%%%%%%%%%
% Problem ends here
%%%%%%%%%%%%%%%%%%%%%%%%%%%%%%%%%%%%%%%%%%%%%%%%%%%%%%%%%%%%%%%%%%%%%

\endinput
