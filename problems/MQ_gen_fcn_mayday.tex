\documentclass[problem]{mcs}

\begin{pcomments}
  \pcomment{MQ_gen_fcn_for_mayday}
  \pcomment{from: S10}
  \pcomment{edited ARM 4/28/12}
\end{pcomments}

\pkeywords{
  generating functions
  counting
}

%%%%%%%%%%%%%%%%%%%%%%%%%%%%%%%%%%%%%%%%%%%%%%%%%%%%%%%%%%%%%%%%%%%%%
% Problem starts here
%%%%%%%%%%%%%%%%%%%%%%%%%%%%%%%%%%%%%%%%%%%%%%%%%%%%%%%%%%%%%%%%%%%%%

\begin{problem} 
Tomorrow is May Day, and you want to send a bouquet to Cindy Lou
Who.  You find exactly one online service that will make bouquets of
\textbf{lilies}, \textbf{roses} and \textbf{tulips} and deliver them
next-day to Whoville, subject to the following constraints:
\begin{itemize}
\item there must be at most 3 lilies,
\item there can be any number of roses, 
\item there must be a multiple of four tulips.
\end{itemize}

Example: A bouquet of 4 tulips, 5 roses and no lilies satisfies 
the constraints.

Let $b_n$ be the number of possible bouquets with $n$ flowers
that fit the service's constraints, and let
\[
B(x) = b_0 + b_1 x + b_2 x^2 + \cdots = b_nx^n + \cdots
\]
be the generating function.

\bparts

\ppart Carefully explain why
\begin{equation}\label{Bgenbouqet}
B(x)  = \frac{1}{(1-x)^2}\ .
\end{equation}

\begin{solution}
Generating function for the number of ways to choose
lilies is:
\[
B_L (x) = 1+ x+ x^2 + x^3 = \frac{1-x^4}{1-x}
\]

Generating function for the number of ways to choose roses is:
\[
B_R (x) = 1+x+ x^2+x^3 + x^4 + \cdots = \frac{1}{1-x}
\]

Generating function for the number of ways to choose tulips is:
\[
B_W (x) = 1 + x^3 + x^7 + \cdots = \frac{1}{1-x^4}
\]

By the Convolution Property, the generating function for $b_n$ is
the product of these functions, namely,
\begin{align*}
B(x) & = B_L(x) B_R(x) B_W(x) \\ 
     & = \frac{(1+ x+ x^2 + x^3)}{(1-x)(1-x^2)} \\
     & = \frac{(1-x^4)}{(1-x)^2(1-x^4)} \\
     & = \frac{1}{(1-x)^2} .
\end{align*}
\end{solution}

\ppart Write a simple formula for $b_n$.

\begin{solution}
\[n+1.\]

$B(x)$ is the generating function for the number of ways to select $n$
  donuts when there are two flavors of donuts, as explained in
  Section~\bref{sec:gf_counting}.  So by~(\bref{cor:donut_binom}),
\[
b_n  = \binom{n + (2 - 1)}{n} = n+1.
\]

\end{solution}
\end{problem}

%%%%%%%%%%%%%%%%%%%%%%%%%%%%%%%%%%%%%%%%%%%%%%%%%%%%%%%%%%%%%%%%%%%%%
% Problem ends here
%%%%%%%%%%%%%%%%%%%%%%%%%%%%%%%%%%%%%%%%%%%%%%%%%%%%%%%%%%%%%%%%%%%%%

\endinput
