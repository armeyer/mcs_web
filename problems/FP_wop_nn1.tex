\documentclass[problem]{mcs}

\begin{pcomments}
\pcomment{FP_wop_nn1}
\pcomment{rewrite of FP_induction_nn1}
\pcomment{from F04rec2}
\pcomment{edited by ARM 9/20/15}
\end{pcomments}

\pkeywords{
  induction
  series
  sum
}

%%%%%%%%%%%%%%%%%%%%%%%%%%%%%%%%%%%%%%%%%%%%%%%%%%%%%%%%%%%%%%%%%%%%%
% Problem starts here
%%%%%%%%%%%%%%%%%%%%%%%%%%%%%%%%%%%%%%%%%%%%%%%%%%%%%%%%%%%%%%%%%%%%%

\begin{problem}
Use the Well Ordering Principle to prove that
\begin{equation}\label{nn1n2}
1 \cdot 2 + 2 \cdot 3 + 3 \cdot 4 + \dots + n (n+1)
    = \frac{n (n+1) (n+2)}{3}
\end{equation}
for all integers, $n\geq 1$.

\begin{solution}

\begin{proof}
Suppose~\eqref{nn1n2} fails for some integer $n \geq 1$.  By the well
ordering principle, there is minimum integer $m \geq 1$ for
which~\eqref{nn1n2} fails, that is,
\begin{equation}\label{mm1m22}
1 \cdot 2 + 2 \cdot 3 + 3 \cdot 4 + \dots + m (m+1)
    \neq \frac{m (m+1) (m+2)}{3},
\end{equation}
but~\eqref{nn1n2} holds for all $n$ such that $1 \leq n < m$.

Note that~\eqref{nn1n2} holds for $n=1$, since the left hand side
of~\eqref{nn1n2} is $1 \cdot 2 = 2$, and the right hand side is
$(1\cdot 2 \cdot 3)/ 3 = 2$.  It follows that $m>1$, and so $1 \leq
(m-1) < m$.  Since $m$ is a minimum counter-example to~\eqref{nn1n2},
the equation therefore must hold for $m-1$, that is
\[
1 \cdot 2 + 2 \cdot 3 + 3 \cdot 4 + \dots + (m-1) ((m-1)+1) =
     \frac{(m-1) ((m-1)+1) ((m-1)+2)}{3}
\]
Simplifying, we have
\begin{equation}\label{m-1mm1}
1 \cdot 2 + 2 \cdot 3 + 3 \cdot 4 + \dots + (m-1) m =
     \frac{(m-1) m (m+1)}{3}
\end{equation}
Now adding $m(m+1)$ to both sides of~\eqref{m-1mm1}, we obtain

\begin{align*}
\lefteqn{[1 \cdot 2 + 2 \cdot 3 + \dots + (m-1)m] + m (m+1)}\\
    & = \frac{(m-1) m (m+1)}{3} + m(m+1) & \text{by~\eqref{m-1mm1}}\\
    & = \frac{(m-1) m (m+1)}{3} + \frac{3m(m+1)}{3}\\
    & = \frac{((m-1)+3) [m (m+1)]}{3}\\
    & = \frac{m(m+1)(m+2)}{3},
\end{align*}
contradicting~\eqref{mm1m22}.\footnote{We spelled out the last three
  algebraic simplification steps here.  In general, such routine
  algebra steps would, and should, be skipped.}

Since the assumption that~\eqref{nn1n2} failed for some $n\geq1$ led to a contradiction,
we conclude that~\eqref{nn1n2} holds for all $n\geq1$.
\end{proof}

A good question is how someone came up with equation~\eqref{nn1n2} in the first
place.  The proof above gives no hint about this, but it should be absolutely
convincing anyway.

\begin{editingnotes}
ref gen func chapter.  add problem about this.
\end{editingnotes}

\end{solution}

\end{problem}

%%%%%%%%%%%%%%%%%%%%%%%%%%%%%%%%%%%%%%%%%%%%%%%%%%%%%%%%%%%%%%%%%%%%%
% Problem ends here
%%%%%%%%%%%%%%%%%%%%%%%%%%%%%%%%%%%%%%%%%%%%%%%%%%%%%%%%%%%%%%%%%%%%%

\endinput
