\documentclass[problem]{mcs}

\begin{pcomments}
  \pcomment{PS_koch_snowflake}
  \pcomment{from: S06.ps3, S18.ps3}
  \pcomment{ edited zabel 2/3/18}
\end{pcomments}
        
\pkeywords{
  recursion
  structural_induction
  fractal     
}

%%%%%%%%%%%%%%%%%%%%%%%%%%%%%%%%%%%%%%%%%%%%%%%%%%%%%%%%%%%%%%%%%%%%%
% Problem starts here
%%%%%%%%%%%%%%%%%%%%%%%%%%%%%%%%%%%%%%%%%%%%%%%%%%%%%%%%%%%%%%%%%%%%%

\begin{problem}
  Fractals are an example of mathematical objects that can
  be defined recursively.  In this problem, we consider the Koch
  snowflake.  Any Koch snowflake can be constructed by the following
  recursive definition.
  \begin{itemize}

  \item \textbf{Base case}: An equilateral triangle with a positive
    integer side length is a Koch snowflake.

  \item \textbf{Constructor case}: Let $K$ be a Koch snowflake, and let $l$ be a
    single edge of the snowflake.  Remove the middle third of $l$, and replace
    it with two line segments of the same length as the middle third, as shown
    in Figure~\ref{kochline}

    The resulting figure is also a Koch snowflake.
  \end{itemize}

  \bparts

  \ppart Find a single Koch snowflake that has exactly 9 edges and includes at
  least $3$ different edge lengths.

  \begin{solution}
    Starting with an equilateral triangle of side length $1$ (as in the Base
    case), apply the Constructor case to a single edge of length $1$ and then
    again to a single edge of length $1/3$.

    This part emphasizes that the Constructor case applies to one edge at a
    time, not all edges at once.
  \end{solution}

  
  \ppart Prove using structural induction that the area inside any Koch
  snowflake is of the form $q\sqrt{3}$, where $q$ is a rational number. Be sure
  to clearly label your induction hypothesis and other necessary assumptions
  during your proof.

  \hint If you require other facts about Koch snowflakes, be sure to prove those
  by structural induction too.

\begin{solution}
    We first show that the side length of any Koch snowflake
    is rational, and prove it in the lemma below.

    \begin{lemma*}
      Every side length of a Koch snowflake is rational.
    \end{lemma*}

    \begin{proof}
      For the base case, every side length is the same positive
      integer.  For the inductive case, let $K$ be a Koch snowflake.
      Then $K$ was constructed by modifying a Koch snowflake $K'$ as
      in the recursive case.  By the induction hypothesis, each side
      length of $K'$ is rational.  Let $l$ be the line segment of $K'$
      modified via the recursive case.  The two new line segments are
      of length $\frac{l}{3}$, which is rational since $l$ is
      rational.
    \end{proof}

    Now, we prove the main theorem.
    \begin{proof}
      We prove the claim by structural induction.

      For the base case, the area of an equilateral triangle with side
      length $l$ is $q\sqrt{3}$, where $q = \frac{l}{2}$.

      For the inductive case, let $K$ be a Koch snowflake.  $K$ was
      constructed by modifying a Koch snowflake $K'$ as in the recursive
      case above.  By the induction hypothesis, the area of $K'$ is
      $q\sqrt{3}$ for some rational $q$.  Let $l'$ be the length of the
      line segment of $K'$ that was modified according to the recursive
      case.  The area added by the modification is $q' \sqrt{3}$, where
      $q' = \frac{l'}{6}$.  By the above lemma, $l'$ is rational, so
      $q'$ is rational.  Thus, the area of $K$ is $(q' + q) \cdot
      \sqrt{3}$, and $q' + q$ is rational, so we have proved our claim.
    \end{proof}
\end{solution}
\eparts

    \begin{figure}
      \includegraphics[width=2.5in]{koch}
     \caption{Constructing the Koch Snowflake.}
     \label{kochline}
    \end{figure}

\end{problem}

%%%%%%%%%%%%%%%%%%%%%%%%%%%%%%%%%%%%%%%%%%%%%%%%%%%%%%%%%%%%%%%%%%%%%
% Problem ends here
%%%%%%%%%%%%%%%%%%%%%%%%%%%%%%%%%%%%%%%%%%%%%%%%%%%%%%%%%%%%%%%%%%%%%

\endinput

