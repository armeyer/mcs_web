\documentclass[problem]{mcs}

\begin{pcomments}
  \pcomment{from: S06.ps3}
\end{pcomments}

\pkeywords{
   %I don't know the format for these keywords, how do I look them up?
	recursion
	structural_induction
}

%%%%%%%%%%%%%%%%%%%%%%%%%%%%%%%%%%%%%%%%%%%%%%%%%%%%%%%%%%%%%%%%%%%%%
% Problem starts here
%%%%%%%%%%%%%%%%%%%%%%%%%%%%%%%%%%%%%%%%%%%%%%%%%%%%%%%%%%%%%%%%%%%%%

\begin{problem}
  Fractals are example of a mathematical object that can
  be defined recursively.  In this problem, we consider the Koch
  snowflake.  Any Koch snowflake can be constructed by the following
  recursive definition.
  \begin{itemize}
  \item Base Case: An equilateral triangle with a positive integer
    side length is a Koch snowflake.
  \item Recursive case: Let $K$ be a Koch snowflake, and let $l$ be a
    line segment on the snowflake.  Remove the middle third of $l$,
    and replace it with two line segments of the same length as is
    done below:

    \begin{figure}[h]
      \graphic[width=2.5in]{koch}
    \end{figure}

    The resulting figure is also a Koch snowflake.
  \end{itemize}

  Prove by structural induction that the area inside any Koch
  snowflake is of the form $q\sqrt{3}$, where $q$ is a rational
  number.

\begin{solution}
	We first show that the side length of any Koch snowflake
    is rational, and prove it in the lemma below.

    \begin{lemma}
      Every side length of a Koch snowflake is rational.
    \end{lemma}

    \begin{proof}
      For the base case, every side length is the same positive
      integer.  For the inductive case, let $K$ be a Koch snowflake.
      Then $K$ was constructed by modifying a Koch snowflake $K'$ as
      in the recursive case.  By the induction hypothesis, each side
      length of $K'$ is rational.  Let $l$ be the line segment of $K'$
      modified via the recursive case.  The two new line segments are
      of length $\frac{l}{3}$, which is rational since $l$ is
      rational.
    \end{proof}

    Now, we prove the main theorem.
    \begin{proof}
      We prove the claim by structural induction.

      For the base case, the area of an equilateral triangle with side
      length $l$ is $q\sqrt{3}$, where $q = \frac{l}{2}$.

      For the inductive case, let $K$ be a Koch snowflake.  $K$ was
      constructed by modifying a Koch snowflake $K'$ as in the recursive
      case above.  By the induction hypothesis, the area of $K'$ is
      $q\sqrt{3}$ for some rational $q$.  Let $l'$ be the length of the
      line segment of $K'$ that was modified according to the recursive
      case.  The area added by the modification is $q' \sqrt{3}$, where
      $q' = \frac{l'}{6}$.  By the above lemma, $l'$ is rational, so
      $q'$ is rational.  Thus, the area of $K$ is $(q' + q) \cdot
      \sqrt{3}$, and $q' + q$ is rational, so we have proved our claim.
    \end{proof}
\end{solution}

\end{problem}

%%%%%%%%%%%%%%%%%%%%%%%%%%%%%%%%%%%%%%%%%%%%%%%%%%%%%%%%%%%%%%%%%%%%%
% Problem ends here
%%%%%%%%%%%%%%%%%%%%%%%%%%%%%%%%%%%%%%%%%%%%%%%%%%%%%%%%%%%%%%%%%%%%%

\endinput

