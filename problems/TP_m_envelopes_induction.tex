\documentclass[problem]{mcs}

\begin{pcomments}
  \pcomment{TP_m_envelopes_induction}
  \pcomment{CH, S14. Adapted from a question on stackexchange.}
  \pcomment{closely related to TP_m_envelopes_WOP}
\end{pcomments}

\pkeywords{
  induction
  envelopes
  dollars
  binary representations of integers
  base-two representations
}

%%%%%%%%%%%%%%%%%%%%%%%%%%%%%%%%%%%%%%%%%%%%%%%%%%%%%%%%%%%%%%%%%%%%%
% Problem starts here
%%%%%%%%%%%%%%%%%%%%%%%%%%%%%%%%%%%%%%%%%%%%%%%%%%%%%%%%%%%%%%%%%%%%%

\begin{problem}

You are given $m$ envelopes, numbered $0, 1, \ldots, m-1$. 
Envelope 0 contains one (1) dollar, Envelope 1 contains two (2)
dollars, and so on (i.e., Envelope $i$ contains $2^i$ dollars.) Using induction, prove the following 

\begin{quote}
\textbf{Property}:  Given a (nonnegative) dollar amount $n < 2^m$, you
can always choose a subset of envelopes whose contents add up to {\em exactly} $n$ dollars. 
\end{quote}

% \hint Construct the set $C$ consisting of positive integers $m$ for which the
% stated property does {\em not} hold for some $n < 2^m$. Prove via the
% WOP that $C$ is empty.

\hint Use induction on the number of envelopes $m$.

\begin{solution}

Denote $P(m)$ be the proposition that any dollar amount $n < 2^m$ can
be constructed using a subset of envelopes containing the stated
number of dollars. We will prove that $P(m)$
is true for all integers $m > 0$ using induction.

\inductioncase{Base case} ($m = 1$): We are given one (1) envelope that
contains one (1) dollar. Clearly, any dollar amount $n$ smaller than 2
can be achieved by either choosing the envelope itself, or nothing at all.

\inductioncase{Inductive step}
Assume that $P(m)$ is true for some $m > 0$. We need to show that
$P(m+1)$ is also true, i.e., any dollar amount smaller than $2^{m+1}$ can be
exactly obtained using a subset of the $m+1$ envelopes.

Consider $n < 2^{m+1}$. There are two mutually exclusive cases:
\begin{enumerate}

\item $n < 2^m .$ By the Induction Hypothesis, we can
  construct an amount of exactly $n$ dollars using a subset of the first $m$ envelopes. 

\item $2^m \leq n < 2^{m+1} .$ Then, $n = n' + 2^m$, where $0 \leq n' <
  2^m$. Since $n'$ can exactly be constructed using a subset of the
  first $m$ envelopes (via the Induction Hypothesis), we can 
  combine that subset with the contents of the $(m+1)^{\textrm{th}}$ envelope to
  construct an amount of exactly $n$ dollars.
 
\end{enumerate}

In either case, we are able to construct an amount of $n$ dollars using
a subset of the envelopes. Therefore, $P(m+1)$ is true. By induction,
we conclude that $P(m)$ is true for all $m > 0$.

\end{solution}

\end{problem}

%%%%%%%%%%%%%%%%%%%%%%%%%%%%%%%%%%%%%%%%%%%%%%%%%%%%%%%%%%%%%%%%%%%%%
% Problem ends here
%%%%%%%%%%%%%%%%%%%%%%%%%%%%%%%%%%%%%%%%%%%%%%%%%%%%%%%%%%%%%%%%%%%%%
