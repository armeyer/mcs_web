\documentclass[problem]{mcs}

\begin{pcomments}
  \pcomment{FP_asymptotics_proof}
  \pcomment{CH, Spring '14}
  \pcomment{Forked from MQ_asymptotics_proof}
\end{pcomments}

\pkeywords{
	asymptotics
        big Oh
        proofs
}

%%%%%%%%%%%%%%%%%%%%%%%%%%%%%%%%%%%%%%%%%%%%%%%%%%%%%%%%%%%%%%%%%%%%%
% Problem starts here
%%%%%%%%%%%%%%%%%%%%%%%%%%%%%%%%%%%%%%%%%%%%%%%%%%%%%%%%%%%%%%%%%%%%%

\begin{problem}

\bparts

Let $f, g : \mathbb{R} \rightarrow \mathbb{R}$ be nonnegative
  functions. 

\ppart If $f = O(g)$, prove that $f^2 = O(g^2)$.
\begin{solution}

We know that $f = O(g)$ iff there exists a constant $c \geq 0$ and an
$x_0$ such that for all $x \geq x_0$, $f(x) \leq c g(x)$. 
Therefore, there exists a constant $c' = c^2$ such that $f^2(x) \leq c^2 g^2(x)$ for all $x \geq
x_0$. Therefore, $f^2 = O(g^2)$. 

\end{solution}

\examspace[2in]

\ppart\label{fsimg} Let $f(x) = x^2 + x$ and $g(x) = x^2$. Show that $f \sim g$.
\begin{solution}
We have that 
\[
\lim_{x \to \infty} \frac{f(x)}{g(x)} = \lim_{x \to \infty} \frac{x^2
  + x}{x} = \lim_{x\to\infty} 1 + \frac{1}{x} \to 1,
\]
thereby showing that $f \sim g$

\end{solution}

\examspace[2in]

\ppart Construct a counter-example to prove that the following statement:
\[
 f \sim g \QIMPLIES 2^f = O( 2^g )
\]
is \emph{false}. 
\begin{solution}
Let $f(x) = x^2 + x$ and $g(x) = x^2$. In the previous part, we showed
that $f \sim g$. But $2^f = 2^{x^2 + x}$ and $2^g =
2^{x^2}$. Therefore, 
\[
\lim_{x \to infty} \frac{2^f}{2^f} = \lim_{x\to\infty} \frac{2^{x^2 +
    x}}{2^{x^2}} = \lim_{x\to\infty} 2^x \to \infty, 
\]
which shows that $2^f = O( 2^g )$ is false.
\end{solution}


\eparts
\end{problem}

\endinput
