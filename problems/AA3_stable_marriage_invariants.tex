\documentclass[problem]{mcs}

\begin{pcomments}
  \pcomment{CP_2_layer_array_network}
  \pcomment{from: S09.final}
\end{pcomments}

\pkeywords{
  networks
  congestion
  routing
}

%%%%%%%%%%%%%%%%%%%%%%%%%%%%%%%%%%%%%%%%%%%%%%%%%%%%%%%%%%%%%%%%%%%%%
% Problem starts here
%%%%%%%%%%%%%%%%%%%%%%%%%%%%%%%%%%%%%%%%%%%%%%%%%%%%%%%%%%%%%%%%%%%%%

\begin{problem}
Recall the stable marriage problem from class.

\bparts

\ppart Specify the preference rankings for $21$ boys and $21$ girls such that
there is only one possible stable matching.  Justify your answer.

\begin{solution}
For each boy $b$, let $b$'s favorite girl $g$ prefer $b$ over
all other boys (all other rankings can be assigned arbitrarily). The
stable matching would be to match each boy with his favorite girl. For
any matching in which some boy $b$ is NOT matched with his favorite girl
$g$, we know $g$ prefers $b$ over her husband and $b$ prefers $g$ over his wife,
so the matching is unstable because $b$ and $g$ form a rogue couple.

\end{solution}


\ppart Which of the following predicates are preserved invariants for the mating ritual? 
\begin{enumerate}

\item Alice’s optimal spouse is serenading her
\item Every girl appears on Bob’s list
\item Bob has fewer than 5 girls on his list
\item Alice is Bob’s optimal spouse
\item Every boy is serenading Alice
\item Bob is serenading his favorite girl
\item Bob prefers the girl he serenaded in the previous round over the girl he is serenading now
\item If not everyone is married today, then Alice is not married
\end{enumerate}
\begin{solution}
The preserved invariants are 1,3,4,7,8.
\end{solution}

  \eparts
\end{problem}

\endinput
