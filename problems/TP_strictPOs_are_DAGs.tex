\documentclass[problem]{mcs}

\begin{pcomments}
  \pcomment{TP_strictPOs_are_DAGs}
  \pcomment{was a Quick Question in the partial order chapter}
\end{pcomments}

\pkeywords{
  partial_orders
  strict
  DAG
}

%%%%%%%%%%%%%%%%%%%%%%%%%%%%%%%%%%%%%%%%%%%%%%%%%%%%%%%%%%%%%%%%%%%%%
% Problem starts here
%%%%%%%%%%%%%%%%%%%%%%%%%%%%%%%%%%%%%%%%%%%%%%%%%%%%%%%%%%%%%%%%%%%%%

\begin{problem}

\bparts

\ppart Why is every strict partial order a DAG?

\begin{solution}
If the strict partial was not a DAG, then it has a vertex $v$ that is
on a cycle.  So there is a positive length walk from $v$ to $v$, which
implies that $v$ is related to itself in the partial order.  This
contradicts assymetry.
\end{solution}

\ppart Give an example of a DAG that is not a strict partial order.

\begin{solution}
$\diredge{1}{2}, \diredge{2}{3}$ but not $\diredge{1}{3}$.
\end{solution}

\ppart Why is the positive walk relation of a DAG a strict partial
order?

\begin{solution}
In a DAG, there is no positive length walk from a vertex to itself, so
its positive walk relation is irreflexive.  If there a a positive
length walk from $u$ to $v$ and another from $v$ to $w$, then the
merge of the walks goes from $u$ to $w$, so the positive walk relation
is transitive.  These two properties make it a strict partial order.
\end{solution}

\eparts

\end{problem}


%%%%%%%%%%%%%%%%%%%%%%%%%%%%%%%%%%%%%%%%%%%%%%%%%%%%%%%%%%%%%%%%%%%%%
% Problem ends here
%%%%%%%%%%%%%%%%%%%%%%%%%%%%%%%%%%%%%%%%%%%%%%%%%%%%%%%%%%%%%%%%%%%%%

\endinput
