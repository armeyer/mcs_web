\documentclass[problem]{mcs}

\begin{pcomments}
  \pcomment{TP_strictPOs_are_DAGs}
  \pcomment{subsumed by TP_strictPOs_are_DAGs_modifiedpartb}
  \pcomment{was a Quick Question in the partial order chapter}
  \pcomment{minor edit ARM 10/27/15}
\end{pcomments}

\pkeywords{
  partial_orders
  strict
  DAG
}

%%%%%%%%%%%%%%%%%%%%%%%%%%%%%%%%%%%%%%%%%%%%%%%%%%%%%%%%%%%%%%%%%%%%%
% Problem starts here
%%%%%%%%%%%%%%%%%%%%%%%%%%%%%%%%%%%%%%%%%%%%%%%%%%%%%%%%%%%%%%%%%%%%%

\begin{problem}

\bparts

\ppart Prove that every strict partial order is a DAG.

\begin{solution}
If the strict partial was not a DAG, then it has a vertex $v$ that is
on a cycle.  So there is a positive length walk from $v$ to $v$, which
implies that $v$ is related to itself in the partial order.  This
contradicts assymetry, proving that the strict partial order must be a
DAG.
\end{solution}

\ppart Give an example of a DAG that is not a strict partial order.

\begin{solution}
$\diredge{1}{2}, \diredge{2}{3}$ but not $\diredge{1}{3}$.

While the path relation of a DAG is a partial order, the example above
illustrates the fact that the DAG may not have an edge between two
vertices that are connected by a path, which means it fails to define
a transitive relation.

\end{solution}

\ppart Prove that the positive walk relation of a DAG a strict partial
order.

\begin{solution}
We know that if there is positive length walk from a vertex to itself,
then the vertex is on a cycle\inbook{
  (Lemma~\bref{shortestclosedwalk_lem})}.  So in a DAG, there will be
no positive length walk from a vertex to itself---that is, the
positive-length walk relation is \textbf{irreflexive}.

If there is a positive length walk from $u$ to $v$ and another from
$v$ to $w$, then the merge of the walks is a positive-length walk from
$u$ to $w$---that is, the positive-length walk relation is
\textbf{transitive}.

These two properties imply the positive-length walk relation is a
strict partial order.
\end{solution}

\eparts

\end{problem}


%%%%%%%%%%%%%%%%%%%%%%%%%%%%%%%%%%%%%%%%%%%%%%%%%%%%%%%%%%%%%%%%%%%%%
% Problem ends here
%%%%%%%%%%%%%%%%%%%%%%%%%%%%%%%%%%%%%%%%%%%%%%%%%%%%%%%%%%%%%%%%%%%%%

\endinput
