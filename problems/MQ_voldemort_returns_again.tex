\documentclass[problem]{mcs}

\begin{pcomments}
  \pcomment{MQ_voldemort_returns_again}
  \pcomment{better version of MQ_voldemort_returns}
  \pcomment{slightly generalizes CP_conditional_prob_says_so_bug}
  \pcomment{ARM 11/24/13}
\end{pcomments}

\pkeywords{
  conditional_probability
  tree_diagram
  four-step_method
}

%%%%%%%%%%%%%%%%%%%%%%%%%%%%%%%%%%%%%%%%%%%%%%%%%%%%%%%%%%%%%%%%%%%%%
% Problem starts here
%%%%%%%%%%%%%%%%%%%%%%%%%%%%%%%%%%%%%%%%%%%%%%%%%%%%%%%%%%%%%%%%%%%%%

\begin{problem}
A guard is going to release exactly two of the three prisoners,
Sauron, Voldemort, and Bunny Foo Foo, and he's equally likely to
release any set of two prisoners.  The guard will truthfully tell
Sauron the name of one of the prisoners to be released.  We're
interested in the following events:

\begin{description}

\item[\ \ $S$:] \emph{S}auron is released.

\item[\textbf{``$F$''}:] The guard says that \emph{F}oo Foo will be released.

\item[\textbf{``$V$''}:] The guard says that \emph{V}oldemort will be released.

\end{description}
So
\begin{equation}\label{Sreleaseprob}
\prob{S} = \frac{2}{3}.
\end{equation}

The guard has two rules for choosing whom he names:
\begin{itemize}

\item never say that Sauron will be released,

\item if both Foo Foo and Voldemort are getting released, say ``Foo Foo''
  with probability 1/3 and ``Voldemort'' with probability 2/3.

\end{itemize}

\bparts

\ppart Use a tree diagram to derive a probability space suitable for
representing the events above.  Indicate which outcomes are in each of
the above events.

\examspace[3.5in]

\begin{solution}

\begin{center}
\begin{picture}(360,175)(0,-40)
%\put(0,-40){\dashbox(360,175){}} % bounding box
\put(0,60){\line(1,1){60}}
\put(0,60){\line(1,0){60}}
\put(0,60){\line(1,-1){60}}
\put(30,-10){\makebox(0,0){released}}
\put(60,120){\line(1,0){60}}
\put(60,60){\line(2,1){60}}
\put(60,60){\line(2,-1){60}}
\put(60,0){\line(1,0){60}}
\put(90,-25){\makebox(0,0){guard says}}
\put(11,90){\makebox(0,0){$F,S$}}
\put(40,68){\makebox(0,0){$F,V$}}
\put(11,30){\makebox(0,0){$V,S$}}
\put(52,96){\makebox(0,0){$1/3$}}
\put(40,50){\makebox(0,0){$1/3$}}
\put(52,24){\makebox(0,0){$1/3$}}
\put(90,128){\makebox(0,0){``$F$''}}
\put(90,90){\makebox(0,0){``$F$''}}
\put(90,30){\makebox(0,0){``$V$''}}
\put(90,-10){\makebox(0,0){``$V$''}}
\put(90,110){\makebox(0,0){$1$}}
\put(102,70){\makebox(0,0){$1/3$}}
\put(102,50){\makebox(0,0){$2/3$}}
\put(90,8){\makebox(0,0){$1$}}
\put(150,120){\makebox(0,0){$1/3$}}
\put(150,90){\makebox(0,0){$1/9$}}
\put(150,30){\makebox(0,0){$2/9$}}
\put(150,0){\makebox(0,0){$1/3$}}
\put(150,-20){\makebox(0,0){prob.}}
\put(210,120){\makebox(0,0){$\times$}}
\put(210,90){\makebox(0,0){$\times$}}
\put(210,30){\makebox(0,0){}}
\put(210,0){\makebox(0,0){}}
\put(210,-25){\makebox(0,0){\shortstack{guard says\\``Foo-foo''}}}
%\put(270,120){\makebox(0,0){$\times$}}
%\put(270,90){\makebox(0,0){$\times$}}
\put(270,30){\makebox(0,0){$\times$}}
\put(270,0){\makebox(0,0){$\times$}}
\put(270,-25){\makebox(0,0){\shortstack{guard says\\"Voldemort''}}}
\put(330,120){\makebox(0,0){$\times$}}
\put(330,90){\makebox(0,0){}}
\put(330,30){\makebox(0,0){}}
\put(330,0){\makebox(0,0){$\times$}}
\put(330,-25){\makebox(0,0){\shortstack{Sauron\\released}}}
\end{picture}
\end{center}


\end{solution}

\problempart\label{saysFF} Sauron worries that if the guard tells him
that Voldemort will be released, then his own probability of release
will be reduced.  Verify that Sauron \emph{is correct} by calculating
$\prcond{S}{\text{``$V$''}}$:

\examspace[0.5in]
\begin{center}
\exambox{0.5in}{0.5in}{0in}
\end{center}

\begin{solution}
\[
\frac{3}{5}\ ,
\]
because
\[
\prcond{S}{\textbf{``$V$''}} = \frac{ \pr{S \intersect \textbf{``$V$''} }}{ \pr{\textbf{``$V$''}} }
       = \frac{ 1/3 }{ 1/3 + 2/9 } = \frac{3}{5} < \frac{2}{3}\, .
\]
\end{solution}

\problempart\label{saysVo} On the other hand, if the guards tells
Sauron that Foo Foo will be released, then Sauron's probability of
release will be increased.  Verify that Sauron is correct by calculating
$\prcond{S}{\text{``$F$''}}$:

\examspace[0.5in]
\begin{center}
\exambox{0.5in}{0.5in}{0in}
\end{center}

\begin{solution}
\[
\frac{3}{4}\, ,
\]
because
\[
\prcond{S}{\text{``$F$''}}
   = \frac{ \pr{S \QAND \text{``$F$''}} }{\pr{\text{``$F$''}} }
  =\frac{ 1/3 }{ 1/3 + 1/9} = \frac{3}{4} > \frac{2}{3}\, .
\]
\end{solution}

\problempart Show how to use the Law of Total Probability to combine
your answers to parts~\eqref{saysFF} and~\eqref{saysVo} to verify
that the result matches~\eqref{Sreleaseprob}.

\examspace[2in]

\begin{solution}
\begin{align*}
\frac{2}{3}
 & = \prob{S}
      & \text{(by~\eqref{Sreleaseprob})}\\
 & = \prcond{S}{\text{``$F$''}}
              \cdot \prob{\text{``$F$''}} + \prcond{S}{\text{``$V$''}} \cdot \prob{\text{``$V$''}}
      & \text{(Total Probability)}\\
 & = \frac{3}{4} \cdot \frac{4}{9} + \frac{3}{5} \cdot \frac{5}{9}
       & \text{(parts~\eqref{saysFF} and~\eqref{saysVo})}\\
 & = \frac{2}{3}\, .
\end{align*}

\end{solution}

\eparts

\end{problem}


%%%%%%%%%%%%%%%%%%%%%%%%%%%%%%%%%%%%%%%%%%%%%%%%%%%%%%%%%%%%%%%%%%%%%
% Problem ends here
%%%%%%%%%%%%%%%%%%%%%%%%%%%%%%%%%%%%%%%%%%%%%%%%%%%%%%%%%%%%%%%%%%%%%

\endinput
