\documentclass[problem]{mcs}

\begin{pcomments}
  \pcomment{MQ_coloring_short}
  \pcomment{excerpt from CP_coloring}
\end{pcomments}

\pkeywords{
  coloring
  cycle
  chromatic_number
}

%%%%%%%%%%%%%%%%%%%%%%%%%%%%%%%%%%%%%%%%%%%%%%%%%%%%%%%%%%%%%%%%%%%%%
% Problem starts here
%%%%%%%%%%%%%%%%%%%%%%%%%%%%%%%%%%%%%%%%%%%%%%%%%%%%%%%%%%%%%%%%%%%%%

\begin{problem}
Determine a valid coloring of the graph shown in
Figure~\ref{fig:to_color} using as few colors as possible.  (Simply
write your proposed color for each vertex next to that vertex.  You
may use $R$ for red, $G$ for green, etc.)

\begin{figure}[h]
\includegraphics[width=0.4\linewidth]{MQ_coloring}
\caption{\label{fig:to_color}}
\end{figure}

\examspace[1in]

\begin{solution}
There are $K_3$'s---triangles---in the graph, so at least three colors
will be needed.  Three colors are all that's needed, as illustrated in
Figure~\ref{fig:colored}.

This 3-coloring was derived by starting with the triangle $abda$.  All
of its vertices must be colored differently, so assign red to $a$,
blue to $b$, and green to $d$.  The length three cycle $bdhb$ now
forces $h$ to be colored red.  $f$ must now be colored green and $g$
must be colored blue.  The coloring is valid so far.  $c$ is adjacent
to a blue vertex and a green vertex, and no others, it must be colored
red.  Finally, $e$ is not adjacent to any other vertices, so it can be
assigned any of the three colors.  There is no pair of like-colored
adjacent vertices, so this coloring is valid.

\begin{figure}[h]
\includegraphics[width=0.4\linewidth]{MQ_coloring_sol}
\caption{A valid coloring.}
\label{fig:colored}
\end{figure}
\end{solution}

\end{problem}

\endinput
