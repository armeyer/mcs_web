\documentclass[problem]{mcs}

\begin{pcomments}
  \pcomment{PS_multinomial_theorem}
  \pcomment{essentially the same as CP_multinomial_theorem, but
    mentions combinatorial proof}
  \pcomment{from: S09.cp11r (perturbed)}
\end{pcomments}

\pkeywords{
  multinomial
  combinatorial_proof
}

%%%%%%%%%%%%%%%%%%%%%%%%%%%%%%%%%%%%%%%%%%%%%%%%%%%%%%%%%%%%%%%%%%%%%
% Problem starts here
%%%%%%%%%%%%%%%%%%%%%%%%%%%%%%%%%%%%%%%%%%%%%%%%%%%%%%%%%%%%%%%%%%%%%

\begin{problem}
According to the \idx{Multinomial Theorem}~\bref{multinom-thm}, $(x_1
+ x_2 + \cdots + x_k)^n$ can be expressed as a sum of terms of the form
\[
\binom{n}{r_1,r_2,\dots,r_k}x_1^{r_1}x_2^{r_2}\dots x_k^{r_k}.
\]

\bparts

\ppart How many terms are there in the sum?

\begin{solution}
The sum is over all $k$-tuples of nonnegative integers
$(r_1,r_2,\dots,r_k)$ such that
\[
r_1+r_2+ \cdots +r_k = n.
\]
We know this is the same as the number of binary words with $n$ zeroes and
$k-1$ ones, namely
\[
\binom{n+k-1}{n}.
\]
\end{solution}

\ppart The sum of these multinomial coefficients has an easily expressed
value:
\begin{equation}\label{multi-coeff-sumk}
\sum_{\substack{r_1+r_2+\cdots+r_k = n,\\
      r_i \in \naturals}}         \binom{n}{r_1,\, r_2,\, \dots,\, r_k} = k^n
\end{equation}

Give a \idx{combinatorial proof} of this identity.

\hint How many terms are there when $(x_1+x_2+\cdots+x_k)^n$ is expressed
as a sum of monomials in $x_i$ \emph{before} terms with like powers of
these variables are collected together under a single coefficient?

\begin{solution}
  This is a nice example of a combinatorial proof: there are $k^n$ ways to
  choose one of the $k$ variables from each of the $n$ expressions
  $(x_1 + x_2 + \cdots + x_k)$, and each choice corresponds to a degree $n$ product of
  these variables.  Each of the $\binom{n+k-1}{n}$ coefficients in the
  sum~\eqref{multi-coeff-sumk} is a count of the number of these monomial
  products with a given number of occurrences of the variables, so the sum
  of these coefficients is simply the total number of products.
\end{solution}

\eparts
\end{problem}


%%%%%%%%%%%%%%%%%%%%%%%%%%%%%%%%%%%%%%%%%%%%%%%%%%%%%%%%%%%%%%%%%%%%%
% Problem ends here
%%%%%%%%%%%%%%%%%%%%%%%%%%%%%%%%%%%%%%%%%%%%%%%%%%%%%%%%%%%%%%%%%%%%%
\endinput
