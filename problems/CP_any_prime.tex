\documentclass[problem]{mcs}

\begin{pcomments}
  \pcomment{CP_any_prime}
  \pcomment{from: S07.ps7}
  \pcomment{appears in number_theory Notes}
\end{pcomments}

\pkeywords{
  prime
  divibility
  Euler
  phi
  Eulers_function
  prime_power
}

%%%%%%%%%%%%%%%%%%%%%%%%%%%%%%%%%%%%%%%%%%%%%%%%%%%%%%%%%%%%%%%%%%%%%
% Problem starts here
%%%%%%%%%%%%%%%%%%%%%%%%%%%%%%%%%%%%%%%%%%%%%%%%%%%%%%%%%%%%%%%%%%%%%

\begin{problem}
Prove that for any prime, $p$, and integer, $k\geq 1$,
\[
\phi(p^k) = p^k-p^{k-1},
\]
where $\phi$ is Euler's function.  \hint Which numbers between 0 and
$p^{k}-1$ \emph{are} divisible by $p$?  How many are there?

\textbf{Note:} This is proved in the text.  Don't look up that proof.

\begin{solution}
  The numbers in the interval from 0 to $p^{k}-1$ that are divisible by
  $p$ are all those of the form $mp$.  For $mp$ to be in the interval, $m$
  can take any value from 0 to $p^{k-1}-1$ and no others, so there are
  exactly $p^{k-1}$ numbers in the interval that are divisible by $p$.
  Now $\phi(p^{k})$ equals the number of remaining elements in the
  interval, namely, $p^k -p^{k-1}$.
\end{solution}

\end{problem}

%%%%%%%%%%%%%%%%%%%%%%%%%%%%%%%%%%%%%%%%%%%%%%%%%%%%%%%%%%%%%%%%%%%%%
% Problem ends here
%%%%%%%%%%%%%%%%%%%%%%%%%%%%%%%%%%%%%%%%%%%%%%%%%%%%%%%%%%%%%%%%%%%%%

\endinput
