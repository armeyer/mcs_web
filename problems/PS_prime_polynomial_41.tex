\documentclass[problem]{mcs}

\begin{pcomments}
  \pcomment{PS_prime_polynomial_41}
  \pcomment{simplified (no more cases) by ARM 1/4/11}
  \pcomment{by ARM 9/8/09, revised hint 9/15/09}
\end{pcomments}

\pkeywords{
  primes
  polynomials
  unbounded
  nonconstant
}


%%%%%%%%%%%%%%%%%%%%%%%%%%%%%%%%%%%%%%%%%%%%%%%%%%%%%%%%%%%%%%%%%%%%%
% Problem starts here
%%%%%%%%%%%%%%%%%%%%%%%%%%%%%%%%%%%%%%%%%%%%%%%%%%%%%%%%%%%%%%%%%%%%%


\begin{problem}
  For $n=40$, the value of polynomial $p(n) \eqdef n^2+n+41$ is not prime,
  as noted in Section~\bref{pred_sec}\inhandout{ of the Course Text}.  But
  we could have predicted based on general principles that no
  \idx{nonconstant polynomial} can generate only prime numbers.

  In particular, let $q(n)$ be a polynomial with integer coefficients, and
  let $c \eqdef q(0)$ be the constant term of $q$.

\begin{problemparts}
\ppart\label{qcnmulc} Verify that $q(cm)$ is a multiple of $c$ for all $m \in
\integers$.

\begin{solution}
  Say $q(n) = c+ \sum_{i=1}^k a_in^i$ where $a_i \in \integers$.  Then
 
\[
q(cm) = c+ \sum_{i=1}^k a_i\paren{c^i m^i}
      = c\paren{1+ \sum_{i=1}^{k} a_i m^{i} c^{i-1}}.
\]
\end{solution}

\ppart\label{consc>1} Show that if $q$ is nonconstant and $c > 1$, then
there are infinitely many $n \in \naturals$ such that $q(n)$ is \idx{not
prime}.

\hint You may assume the familiar fact that the magnitude
\iffalse (absolute value)\fi
of any nonconstant polynomial, $q(n)$, \idx{grows unboundedly} as $n$
grows.

\begin{solution}
  If $\abs{q(cm)} > c >1$, then $q(cm)$ won't be prime because by
  part~\eqref{qcnmulc}, it has $c$ as a factor.  Since $\abs{q(n)}$ grows
  unboundedly with $n$, there will be infinitely many different such
  values of $q(cm)$ as $m$ grows.
\end{solution}

\ppart Conclude immediately that for every nonconstant polynomial, $q$,
there must be an $n \in \naturals$ such that $q(n)$ is not prime.

\begin{solution}
By part~\eqref{consc>1}, the only remaining case is when $c \leq 1$.
But in that case $q(n)$ is not prime for $n=0$.
\end{solution}

\end{problemparts}

\end{problem}

%%%%%%%%%%%%%%%%%%%%%%%%%%%%%%%%%%%%%%%%%%%%%%%%%%%%%%%%%%%%%%%%%%%%%
% Problem ends here
%%%%%%%%%%%%%%%%%%%%%%%%%%%%%%%%%%%%%%%%%%%%%%%%%%%%%%%%%%%%%%%%%%%%%

\endinput
