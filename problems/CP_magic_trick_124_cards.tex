\documentclass[problem]{mcs}

\begin{pcomments}
  \pcomment{CP_magic_trick_124_cards}
  \pcomment{from: S09.cp10t, S06.cp9w with first prob shortened, 2nd problem
    omitted; F02.cp9W which was taken from F99 tut16 & S99 tut9}
  \pcomment{follow up to inclass Magic Trick demo}
\end{pcomments}

\pkeywords{
 magic_trick
 degree-constrained
 Halls_theorem
}

%%%%%%%%%%%%%%%%%%%%%%%%%%%%%%%%%%%%%%%%%%%%%%%%%%%%%%%%%%%%%%%%%%%%%
% Problem starts here
%%%%%%%%%%%%%%%%%%%%%%%%%%%%%%%%%%%%%%%%%%%%%%%%%%%%%%%%%%%%%%%%%%%%%

\begin{problem}
\bparts

\ppart Show that the Magician could not pull off the trick with a deck larger
than 124 cards.

\hint Compare the number of 5-card hands in an $n$-card deck with the
number of 4-card sequences.

\begin{solution}
For a match to be possible with a $n$-card deck, the number,
$\binom{n}{5}$, of 5-card hands must be at most as large as the number,
$(n)_4$, of 4-card sequences. So
\[
(n)_4 (n-4)/5! = \binom{n}{5} \leq (n)_4,
\]
which implies
\[
n-4 \leq 5!
\]
and hence $n \leq 124$.

\end{solution}

\ppart Show that, in principle, the Magician could pull off the Card Trick
with a deck of 124 cards.

\hint \idx{Hall's Theorem} and
\idx{degree-constrained}~\bref{degree-constrained_def} graphs.

\begin{solution}
In principle the trick is possible iff the bipartite graph
between 5-card hands and 4-card sequences has a matching for the hands.
In this graph, the degree of each hand is $5! = 120$, whatever the size of
deck.  The degree of each sequence of 4 will be the number of cards
remaining in the deck.  With a deck of 124, there will be 120 cards
remaining, so the degree of each sequence of 4 will also be 120.  Hence,
the graph is degree-constrained, and so satisfies Hall's condition for a
matching.
\end{solution}

\eparts

\end{problem}
 
\endinput
