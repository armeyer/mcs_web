\documentclass[problem]{mcs}

\begin{pcomments}
  \pcomment{CP_3color_OR_gate}
  \pcomment{forked and shortened from PS_3color_SAT}
  \pcomment{ARM Fall '11, edited by CH Spring '14 }
\end{pcomments}

\pkeywords{
SAT
logical_formula
propositional_logic
logic
proposition
negation
coloring
}

%%%%%%%%%%%%%%%%%%%%%%%%%%%%%%%%%%%%%%%%%%%%%%%%%%%%%%%%%%%%%%%%%%%%%
% Problem starts here
%%%%%%%%%%%%%%%%%%%%%%%%%%%%%%%%%%%%%%%%%%%%%%%%%%%%%%%%%%%%%%%%%%%%%

\begin{problem}

In this problem, we examine an interesting connection between
propositional logic and 3-colorings of certain special graphs.
Consider the graph in Figure~\ref{fig:3color-OR}. 
We designate the vertices connected in the triangle on the left as
\emph{color-vertices}; since they form a triangle, they are forced
to have different colors in any coloring of the
graph. The colors assigned to the color-vertices will be called $T,F$ and $N$.
The dotted lines indicate edges to the color-vertex $N$. 

\begin{figure}\inbook{[h]}
\includegraphics[width=4in]{3color-OR}
\caption{A 3-color $\QOR$-gate}
\label{fig:3color-OR}
\end{figure}

\bparts

\ppart Prove that there exists a 3-coloring of the graph iff neither $P$
nor $Q$ are colored $N$.

\begin{solution}

%\begin{proof}

\textbf{(left-to-right case)}: If there is a valid 3-coloring (or more generally, any
valid coloring) then the dotted edges ensure that $P$ and $Q$ are not
colored as $N$ in that coloring.

\textbf{(right-to-left case)}: If neither $P$ nor $Q$ are colored $N$,
then both $P$ and $Q$ have to colored $T$ or $F$.  

The diagram is symmetric in $P$ and $Q$, so there are really only three cases to
consider: $P$ and $Q$ are both colored $T$, both colored $F$, or $P$
and $Q$ are colored differently.  If $P$ and $Q$ are colored differently, we can verify that this leads
to only one possible 3-coloring where the vertex labelled $(P \QOR Q)$ is colored $T$.

If $P$ and $Q$ have the same color, then one of the vertices directly
above must be colored with $N$ and the other with the opposite color
as $P$ and $Q$.  This forces $(P \QOR Q)$ to be colored
with the same color as $P$ and $Q$.  There is then a unique coloring
of the bottom vertex, and the middle vertex on the arc on the left that
can complete a 3-coloring.

Therefore, in each case where neither $P$ nor $Q$ are colored $N$,
there exists a valid 3-coloring.

%\end{proof}

\end{solution}

\ppart 

Argue that a valid 3-coloring of the graph in Figure~\ref{fig:3color-OR} simulates the
functioning of a two-input $\QOR$-gate. 

\hint Examine the relation between the colors of $P$, $Q$, and $(P \QOR
Q)$ in your solution to the first part. 

\begin{solution}

We can think of $P$ and $Q$ as ``input'' vertices and $(P \QOR Q)$ as
the ``output'' vertex.  In the argument above, we concluded that when
$P$ and $Q$ have different colors, $(P \QOR Q)$ is colored
$T$. On the other hand, when $P$ and $Q$ have the same color, then $(P
\QOR Q)$ also shares this color. Therefore, the color of $P
\QOR Q$ is always the Boolean $\QOR$ of the colors assigned to $P$ and
$Q$. 

\end{solution}

\eparts

\end{problem}

%%%%%%%%%%%%%%%%%%%%%%%%%%%%%%%%%%%%%%%%%%%%%%%%%%%%%%%%%%%%%%%%%%%%%
% Problem ends here
%%%%%%%%%%%%%%%%%%%%%%%%%%%%%%%%%%%%%%%%%%%%%%%%%%%%%%%%%%%%%%%%%%%%%

\endinput
