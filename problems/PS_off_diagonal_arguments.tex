\documentclass[problem]{mcs}

\begin{pcomments}
  \pcomment{PS_off_diagonal_arguments}
  \pcomment{ARM 3.14.17}
\end{pcomments}

\pkeywords{
  diagonal_argument
  diagonal
  countable
  uncountable
  subset
}

%%%%%%%%%%%%%%%%%%%%%%%%%%%%%%%%%%%%%%%%%%%%%%%%%%%%%%%%%%%%%%%%%%%%%
% Problem starts here
%%%%%%%%%%%%%%%%%%%%%%%%%%%%%%%%%%%%%%%%%%%%%%%%%%%%%%%%%%%%%%%%%%%%%

\begin{problem}
You don't really have to go down the diagonal in a ``diagonal'' argument.

Let's review the basic diagonal argument using one-way infinite
\emph{streams}---also called \emph{lists} or \emph{sequences}---of
elements
\[
\ang{e_0,e_1,e_2, \dots, e_k, \dots}
\]
that keep going to the right forever.  The angle brackets appear
above as a reminder that the stream is not a set: its elements appear
in order, and the same element may appear multiple times.  (The right
angle-bracket is not really there, since the list does not have a
right end.)\footnote{If streams seem worrisome or too
  Computer-Sciencey, you can replace them with total functions on
  $\nngint$.  So the stream above can be replaced by a function $e$ on
  $\nngint$ where $e(n) \eqdef e_n$.}

The general setup for a diagonal argument is that we have some stream
$S$ whose elements are themselves one-way infinite streams.  We
picture the stream $S = \ang{s_0,s_1,s_2,\dots}$ running vertically
downward, and each stream $s_k \in S$ running horizontally to the right:
\[
s_k =  \ang{s_{k,0}, s_{k,1}, s_{k,2}, \dots}.
\]
So we have a 2-D matrix that is infinite down and to the right:

\[\begin{array}{|rccccccl}
    &     0  &  1     & 2      & \hspace{0.5in}& \dots & \hspace{0.5in} &  k \hspace{0.5in} \dots\\
\hline
s_0 &  s_{0,0} & s_{0,1} & s_{0,2} &  \hspace{0.5in}& \dots & \hspace{0.5in} &  s_{0,k} \hspace{0.5in} \dots\\
s_0 &  s_{1,0} & s_{1,1} & s_{1,2} &  \hspace{0.5in}& \dots & \hspace{0.5in} &  s_{1,k} \hspace{0.5in} \dots\\
s_0 &  s_{2,0} & s_{2,1} & s_{2,2} &  \hspace{0.5in}& \dots & \hspace{0.5in} &  s_{2, k} \hspace{0.5in} \dots\\
                                            &&&&&\vdots
\end{array}\]

The diagonal argument explains how to find a ``new'' stream, that is,
a stream that is not in $S$.  We do this simply by using the diagonal
of the matrix as a stream
\[
\ang{s_{0,0}, s_{1,1}, s_{2,2}, \dots, s_{k,k}, \dots}
\]
then defining the ``new'' stream to be
\[
D_S \eqdef \ang{\bar{s_{0,0}}, \bar{s_{1,1}}, \bar{s_{2,2}}, \dots, \bar{s_{k,k}}, \dots}
\]
where $\bar{x}$ indicates some element that is not equal to $x$.  So
$D_S$ is a new stream not in $S$ because it differs from every stream
in $S$, namely, it differs from the $k$th stream at position $k$.

For definiteness, let's say
\[
\bar{x} \eqdef \begin{cases}
               1 &\text{ if }x \neq 1,\\
               2 &\text{otherwise}.\\
               \end{cases}
\]
With this contrivance, we have made $D_S$ be a stream of 1's and 2's.

But as we said at the beginning, you don't have to go down the
diagonal.  You could, for example, follow a line with a slope of
$-1/4$ to get a new stream
\[
T_S \eqdef \langle \bar{s_{0,0}}, 2,2,2,\bar{s_{1,4}},2,2,2,\bar{s_{2,8}}, \dots, 2, \bar{s_{k,4k}}, 2 \dots.
\]
$T_s$ will be a new stream because it differs from every row of the
matrix, but this time it differs from the $k$th row at position $4k$.

\bparts

\ppart Notice that in the limit, at least 3/4 of the elements in $T_s$
above are 2's.  Explain why there are an \emph{uncountable} number of
new streams of 1's and 2's whose elements in the limit are at least 1/2
2s.

\hint In each group of three consecutive 2's indicated in $T_S$,
whatever you do, leave the last two of three 2's unchanged.  So at
least $1/2$ of the positions will still contain 2's.

\begin{solution}
Replacing any subset of the elements of $T_S$ that appear in positions
with remainder 1 on division by 4 will yield a new stream that still
differs from the $k$th row at position $4k$.  Moreover, half the
stream will still be 2's.

Since there are a countable infinity of positions that leave remainder
1, there are an uncountable number of \emph{subsets} of these
positions.
\end{solution}

\ppart Notice that in the limit, at most 1/3 of the elements in $T_s$
above are not 2's.  Let's say a stream has a \emph{neglible fraction
  of non-2 elements} if, in the limit, it has a fraction of at most
$\epsilon$ non-2 elements for every $\epsilon>0$.  Describe how to
define a new stream that has a \emph{neglible number of non-2
  elements}.

\begin{solution}
One way is to define
\[
N_S \eqdef  \ang{\bar{s_{0,0}},
2,\bar{s_{1,1}},2,2,\bar{s_{2,9}},2,2,2,2, \dots, 2, \bar{s_{k,k^2}},
\underbrace{2,2, \dots,2}_{= k^2}, \bar{s_{k,(k+1)^2}},\dots}
\]
In fact, we could replace $k^2$ by any function of $k$ that grows more
than linearly.
\end{solution}

\ppart Describe a new stream that differs infinitely many times from every stream in $S$.

\hint Divide \nngint\ into an infinite number of non-overlapping
infinite pieces.

\begin{solution}
Divide \nngint\ into an infinite number of non-overlapping infinite
pieces, and let the ``diagonal'' set differ from the $k$th row at all
the positions in the $k$th piece.  For example, let the first piece be
half of \nngint, the second piece be half of the remaining half, the
third piece be half of the remaining quarter, and so forth.  More
formally, let the ``diagonal'' set differ from the $k$th row at
positions
\[
2^k \cdot 1 + k, 2^k \cdot 2 + k, 2^k \cdot 3 + k, \dots, 2^k \dot n + n, \dots.
\]
\end{solution}

\eparts

\end{problem}
\endinput
