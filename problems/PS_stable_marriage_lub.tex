\documentclass[problem]{mcs}

\begin{pcomments}
  \pcomment{PS_stable_marriage_lub}
  \pcomment{by ARM 4/11/15}
\end{pcomments}

\pkeywords{
 stable_matching
 lub
 preferred
}


%%%%%%%%%%%%%%%%%%%%%%%%%%%%%%%%%%%%%%%%%%%%%%%%%%%%%%%%%%%%%%%%%%%%%
% Problem starts here
%%%%%%%%%%%%%%%%%%%%%%%%%%%%%%%%%%%%%%%%%%%%%%%%%%%%%%%%%%%%%%%%%%%%%

\begin{problem}
Suppose there are two stable sets of marriages, a first set and a
second set.  So each man has a first wife and a second wife (they may
be the same), and likewise each woman has a first husband and a second
husband.  We can form a third set of marriages by matching each man
with the wife he prefers among his first and second wives.

\bparts

\ppart Prove that this third set of marriages is an exact matching: no
woman is married to two men.

%\hint The definition of rogue couple.

\begin{solution}
Suppose two men, Bob and Ted, were both married to Alice in the third
marriage set.  So both men must prefer Alice to their other wives.  We
may assume that Alice prefers Bob to Ted.  This implies that Bob and
Alice are a rogue couple in the set of marriages where Alice is
married to Ted, contradicting stability.
\end{solution}

\ppart Prove that this third marriage set is stable.

\hint You may assume the following fact from
Problem~\bref{PS_stable_marriage_better_worse_count}.
\begin{equation}\label{swsl}
\text{In every marriage, someone is a winner iff their spouse is a loser,}
\end{equation}

\begin{solution}
Since the first and second marriage sets are stable, there will be no
rogue couple among third marriages that come from the same marriage
set.

Now suppose Bob and Alice are married in the first and third sets of
marriages, and Ted and Carol are married in the second and third sets
of marriages.  If we show that Bob and Carol are not a rogue couple in
third set of marriages, then we can conclude that there are no rogue
couples in the third set of marriages.


 \TBA{Needs completion.}
\end{solution}

\eparts
\end{problem}

\endinput
