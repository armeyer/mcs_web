\documentclass[problem]{mcs}

\begin{pcomments}
  \pcomment{MQ_counting_paths_with_obstacles}
\end{pcomments}

\pkeywords{
  inclusion-exclusion
}

%%%%%%%%%%%%%%%%%%%%%%%%%%%%%%%%%%%%%%%%%%%%%%%%%%%%%%%%%%%%%%%%%%%%%
% Problem starts here
%%%%%%%%%%%%%%%%%%%%%%%%%%%%%%%%%%%%%%%%%%%%%%%%%%%%%%%%%%%%%%%%%%%%%

\begin{problem}
A spacecraft is traveling through otherwise-empty three-dimensional
space.  It can move along only one dimension at a time, stepping
precisely one unit in the positive direction along that dimension with
each movement.  For any two points, $P$ and $Q$, in space, let
$n(P,Q)$ denote the number of distinct paths the spacecraft can
follow to go from $P$ to $Q$.

\begin{problemparts}

\problempart Let $P$ and $Q$ have coordinates $(0, 10, 20)$ and
$(10, 40, 90)$, respectively.  Assuming that $n(P,Q)$ is positive,
express $n(P,Q)$ \textbf{as a single multinomial coefficient}.

\examspace[2in]

\begin{solution}
Because each of the spacecraft's permissible atomic movements involves
incrementing precisely one of its three position coordinates,
$n(P,Q) > 0$ implies that all coordinates of $Q-P$ are nonnegative integers.
(The converse is also true.) To go from $P$ to $Q$,
the spacecraft must increment its first position coordinate
$10-0$ times, its second $40-10$ times, and its third $90-20$
times.  So it must undergo precisely $(10-0)+(40-10)+(90-20)$
atomic movements, $10-0$ of them along the first dimension,
$40-10$ of them along the second, and $90-20$ of them along the
third.

So, number the spacecraft's atomic movements:
$1,2,\ldots,(10-0)+(40-10)+(90-20)$.  Partition the set
$T=\set{1,2,\ldots,(10-0)+(40-10)+(90-20)}$ into three sets,
$T_x$, $T_y$, and $T_z$, such that $\abs{T_x}=10-0$,
$\abs{T_y}=40-10$, and $\abs{T_z}=90-20$.  $T_x$ then specifies
which atomic movements are along the first dimension, $T_y$ does the
same for the second dimension, and $T_z$ for the third.  Each distinct
partition corresponds to a single permissible path from $P$ to $Q$,
and each permissible path from $P$ to $Q$ corresponds to a single
partition.  So the number of permissible paths from $P$ to $Q$ is just
the number of distinct partitions -- that is, the number of
$(10-0,40-10,90-20)$-splits of the
$\paren{(10-0)+(40-10)+(90-20)}$-element set $T$.  And of
course this number is just:

\[
n(P,Q)=\binom{(10-0)+(40-10)+(90-20)}{10-0,40-10,90-20}
\]

Alternatively, consider a bijection between the set of possible paths
from $P$ to $Q$ and the set of sequences of length
$(10-0)+(40-10)+(90-20)$ that contain $(10-0)$ $1$s,
$(40-10)$ $2$s, and $(90-20)$ $3$s.  The $k$th term of each
sequence specifies the dimension associated with the $k$th atomic
movement in the corresponding path.  The Bookkeeper Rule then leads
directly to the expression for $n(P,Q)$.
\end{solution}

\problempart Suppose there exist five points in space,
$A=(0,0,0)$, $B=(1,1,1)$, $C=(3,3,3)$, and $D=(6,6,6)$,
such that it is possible for the spacecraft to travel
from $A$ to $B$, from $B$ to $C$, and from $C$ to $D$.
Write an expression for the number of distinct paths the
spacecraft can follow to go from $A$ to $D$
while \textbf{avoiding} $B$ and $C$.
%Your expression \textbf{must} be written entirely
%in terms of symbols of the form $n(P,Q}$, where
%$P,Q\in\set{A,B,C,D,E}$.

\hint Inclusion-Exclusion.

\examspace[3in]

\begin{solution}
$n(A,D) - n(A,B)n(B,D) - n(A,C)n(C,D) + n(A,B)n(B,C)n(C,D)
= \binom{18}{6,6,6}
  - \binom{3}{1,1,1}\binom{15}{5,5,5}
  - \binom{9}{3,3,3}\binom{9}{3,3,3}
  + \binom{3}{1,1,1}\binom{6}{2,2,2}\binom{9}{3,3,3}$
\end{solution}

\end{problemparts}

\end{problem}

%%%%%%%%%%%%%%%%%%%%%%%%%%%%%%%%%%%%%%%%%%%%%%%%%%%%%%%%%%%%%%%%%%%%%
% Problem ends here
%%%%%%%%%%%%%%%%%%%%%%%%%%%%%%%%%%%%%%%%%%%%%%%%%%%%%%%%%%%%%%%%%%%%%

\endinput
