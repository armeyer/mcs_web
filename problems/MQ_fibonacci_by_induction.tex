\documentclass[problem]{mcs}

\begin{pcomments}
  \pcomment{MQ_fibonacci_by_induction}
  \pcomment{new by Ali Kazerani, proof of Binet's formula using strong induction}
\end{pcomments}

\pkeywords{
  induction
  fibonacci
  recurrence
}

%%%%%%%%%%%%%%%%%%%%%%%%%%%%%%%%%%%%%%%%%%%%%%%%%%%%%%%%%%%%%%%%%%%%%
% Problem starts here
%%%%%%%%%%%%%%%%%%%%%%%%%%%%%%%%%%%%%%%%%%%%%%%%%%%%%%%%%%%%%%%%%%%%%

\begin{problem} Recall that the Fibonacci sequence is defined recursively by
\[ F_n = \left\{ \begin{array}{ll}
         0 & \mbox{if $n = 0$}\\
         1 & \mbox{if $n = 1$}\\
         F_{n-1}+F_{n-2} & \mbox{if $n = 2,3,4,\ldots$}\end{array} \right. \] 
Prove, using strong induction, the closed-form formula for the $n$th Fibonacci number, $n=0,1,2,\ldots$:
\[ F_n = \frac{p^n-q^n}{\sqrt{5}}\]
where $p=\frac{1+\sqrt{5}}{2}$ and $q=\frac{1-\sqrt{5}}{2}$.  Note that $p$ and $q$ are the roots of $x^2-x-1=0$, so they satisfy $p^2=p+1$ and $q^2=q+1$.
 
\begin{solution}
\begin{proof}
 We will proceed by induction on $n$.  Let the induction hypothesis, $P\left(n\right)$, be that the given closed-form formula holds at $n$:
\[ F_n = \frac{p^n-q^n}{\sqrt{5}}\]
\textbf{Base case:}  $P\left(0\right)$ is true, since
\[ \frac{p^n-q^n}{\sqrt{5}}=\frac{p^0-q^0}{\sqrt{5}}=\frac{1-1}{\sqrt{5}}=0=F_0\]
\textbf{Inductive step:}  Assume $P\left(k\right)$ holds for all $0\leq k\leq n$.  We will now prove that $P\left(n+1\right)$ holds.\\\\
\textbf{Case} ($n+1=1$):  $P\left(1\right)$ is true, since
\[ \frac{p^n-q^n}{\sqrt{5}}=\frac{p^1-q^1}{\sqrt{5}}=\frac{p-q}{\sqrt{5}}=\frac{\sqrt{5}}{\sqrt{5}}=1=F_1\]
\textbf{Case} ($n+1\geq 2$):\\\\
Since $0\leq n-1,n\leq n$, therefore by the strong induction hypothesis, $P\left(n-1\right)$ and $P\left(n\right)$ are both true.\\
That is, $F_{n-1} = \frac{p^{n-1}-q^{n-1}}{\sqrt{5}}$ and $F_n = \frac{p^n-q^n}{\sqrt{5}}$.\\
Now,
\[ p^2=p+1\Rightarrow p^2p^{n-1}=\left(p+1\right)p^{n-1}\Rightarrow p^{n+1}=p^n+p^{n-1}\]
Similarly,
\[ q^2=q+1\Rightarrow q^2q^{n-1}=\left(q+1\right)q^{n-1}\Rightarrow q^{n+1}=q^n+q^{n-1}\]
Subtracting these equations gives
\[ p^{n+1}-q^{n+1}=p^n-q^n+p^{n-1}-q^{n-1}\]
So
\begin{eqnarray*}
\frac{p^{n+1}-q^{n+1}}{\sqrt{5}} & = & \frac{p^n-q^n}{\sqrt{5}}+\frac{p^{n-1}-q^{n-1}}{\sqrt{5}}\\
\frac{p^{n+1}-q^{n+1}}{\sqrt{5}} & = & F_n+F_{n-1}\\
\end{eqnarray*}
But by the definition of the Fibonacci sequence, $F_{n+1} = F_n+F_{n-1}$ for $n+1\geq 2$.  Thus,
\[ F_{n+1} = \frac{p^{n+1}-q^{n+1}}{\sqrt{5}}\]
Therefore, $P\left(n+1\right)$ is true in this case as well.\\\\
Since $n\geq 0$, therefore $n+1\geq 1$, so these two cases cover every possibility.  Since $P\left(n+1\right)$ holds in both cases, we conclude that by strong induction, $P\left(n\right)$ holds for all $n=0,1,2,\ldots$
\end{proof}
\end{solution}
\end{problem}

%%%%%%%%%%%%%%%%%%%%%%%%%%%%%%%%%%%%%%%%%%%%%%%%%%%%%%%%%%%%%%%%%%%%%
% Problem ends here
%%%%%%%%%%%%%%%%%%%%%%%%%%%%%%%%%%%%%%%%%%%%%%%%%%%%%%%%%%%%%%%%%%%%%

\endinput
