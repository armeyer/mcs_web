\documentclass[problem]{mcs}

\begin{pcomments}
  \pcomment{CP_partially_ordered_by_divisibility}
  \pcomment{assumes CP_divisibility_partial_order}
  \pcomment{from: F03.ps2}

\end{pcomments}

\pkeywords{
  partial_order
  divisibility
  minimal
  maximal
  chain
  infinite_antichain
  antichain
}

%%%%%%%%%%%%%%%%%%%%%%%%%%%%%%%%%%%%%%%%%%%%%%%%%%%%%%%%%%%%%%%%%%%%%
% Problem starts here
%%%%%%%%%%%%%%%%%%%%%%%%%%%%%%%%%%%%%%%%%%%%%%%%%%%%%%%%%%%%%%%%%%%%%

\begin{problem}
Consider the nonnegative numbers partially ordered by divisibility.

\begin{problemparts}

\problempart 
Show that this partial order has a unique minimal element.

\begin{solution}
1 is minimal as there is no other natural number that divides 1.
It is unique because all other numbers are divisible by 1 and therefore
are not minimal.
\end{solution}

\problempart 
Show that this partial order has a unique maximal element.

\begin{solution}
0 is maximal: all nonnegative integer divide zero.  It is the only
maximal element, because for every positive natural number, $n$, we
have that $n$ is strictly ``smaller'' than $2n$ under divisibility.
\end{solution}

\problempart  
Describe an infinite chain in this partial order.

\begin{solution}
1 2 4 8 16 \dots is a chain with infinite length.
\end{solution}

\problempart \iffalse An \term{antichain} in a partial order is a set
of elements such that any two elements in the set are incomparable.\fi
Describe an infinite antichain in this partial order.
\begin{staffnotes}

\hint The primes.

\end{staffnotes}

\begin{solution}
The set of prime numbers is infinite.  Since no prime divides another,
any two primes are incomparable.  So the set of prime numbers is an
antichain.
\end{solution}

\ppart  What are the minimal elements of divisibility on the integers
greater than 1?  What are the maximal elements?
\begin{solution}
The primes are the minimal elements.  There are no maximal elements.
\end{solution}

\end{problemparts}

\end{problem}

%%%%%%%%%%%%%%%%%%%%%%%%%%%%%%%%%%%%%%%%%%%%%%%%%%%%%%%%%%%%%%%%%%%%%
% Problem ends here
%%%%%%%%%%%%%%%%%%%%%%%%%%%%%%%%%%%%%%%%%%%%%%%%%%%%%%%%%%%%%%%%%%%%%

 \endinput
