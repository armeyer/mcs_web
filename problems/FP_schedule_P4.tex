\documentclass[problem]{mcs}

\begin{pcomments}
  \pcomment{FP_schedule_P4}
  \pcomment{isomorphic to FP_schedule_B4}
  \pcomment{ARM 4/4/16}
\end{pcomments}

\pkeywords{
  partial_orders
  scheduling
  constraints
  parallel_time
}

%%%%%%%%%%%%%%%%%%%%%%%%%%%%%%%%%%%%%%%%%%%%%%%%%%%%%%%%%%%%%%%%%%%%%
% Problem starts here
%%%%%%%%%%%%%%%%%%%%%%%%%%%%%%%%%%%%%%%%%%%%%%%%%%%%%%%%%%%%%%%%%%%%%

\begin{problem} 
Answer the following questions about the powerset,
$\power(\set{1,2,3,4})$, partially ordered by the subset relation
$\subseteq$.

\bparts

\iffalse
\ppart Prove that $\sqsubseteq$ is a partial order on $B^4$.

\examspace[2.0in]
\fi

\ppart Give an example of a maximum length chain.

\examspace[1.0in]

\begin{solution}
\[
\emptyset, \set{1}, \set{1,2}, \set{1,2,3}, \set{1,2,3,4}.
\]

\end{solution}
\ppart Give an example of an antchain of size 6.

\examspace[1.0in]

\begin{solution}
\[
\set{1,2},\set{2,3},\set{3,4},\set{1,3},\set{2,4},\set{1,4}.
\]
\end{solution}

\ppart Describe an example of a topological sort of $\power(\set{1,2,3,4})$.

\examspace[1.0in]

\begin{solution}
The empty set, followed by the four 1-element sets in any order,
followed by the six 2-element sets in any order, followed by the
four 3-element sets in any order followed by $\set{1,2,3,4}$.
\end{solution}

\ppart Assuming that the partial order describes scheduling
constraints on 16 tasks\footnote{As usual, we assume each task
  requires one time unit to complete.} corresponding to the elements
of $\power(\set{1,2,3,4})$, what is the length of a minimum time
3-processor schedule?

\begin{center}
\exambox{0.5in}{0.5in}{0.0in}
\end{center}



\inbook{Prove it.}

\examspace[0.4in]

\begin{solution}
\textbf{7}.  For example, a length-7 3-processor schedule is:
\begin{align*}
\emptyset\\
o1,\ 2,\ 3\\
4,\ 12,\ 23\\
34,\ 14,\ 24\\
13,\ 124,\ 234\\
123,\ 134\\
1234.
\end{align*}
Moreover, no shorter schedule is possible: there is a unique minimum
task, $\emptyset$, which must come first and a unique maximum task,
$\set{1,2,3,4}$, which must come last; this leaves 14 tasks which
require at least $\ceil{14/3} = 5$ more parallel steps.
\end{solution}

\eparts

\end{problem} 


%%%%%%%%%%%%%%%%%%%%%%%%%%%%%%%%%%%%%%%%%%%%%%%%%%%%%%%%%%%%%%%%%%%%%
% Problem ends here
%%%%%%%%%%%%%%%%%%%%%%%%%%%%%%%%%%%%%%%%%%%%%%%%%%%%%%%%%%%%%%%%%%%%%

\endinput
