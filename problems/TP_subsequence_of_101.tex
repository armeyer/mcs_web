\documentclass[problem]{mcs}

\begin{pcomments}
  \pcomment{TP_subsequence_of_101}
  \pcomment{by ARM 9/24/09 from partial order notesproblem}
\end{pcomments}

\pkeywords{
  chain
  anti-chain
  increasing
  decreasing
}

%%%%%%%%%%%%%%%%%%%%%%%%%%%%%%%%%%%%%%%%%%%%%%%%%%%%%%%%%%%%%%%%%%%%%
% Problem starts here
%%%%%%%%%%%%%%%%%%%%%%%%%%%%%%%%%%%%%%%%%%%%%%%%%%%%%%%%%%%%%%%%%%%%%

\begin{problem}
Describe a sequence consisting of the integers from 1 to 10,000 in some
order so that there is no increasing or decreasing subsequence of size
101.

\begin{solution}
Cut the sequence of integers from 1 to 10,000 in contiguous
subsequences of 100 integers:
\begin{align*}
&1,2,\dots,100\\
&101, 102,\dots,200 \\
&\hspace{0.5in} \vdots \\ 
&9801,9802,\dots,9900\\
&9901,9902,\dots, 10000.
\end{align*}

Then list the 10,000 numbers by concatenating these 100 contiguous
subsequences in reverse order:
\begin{align*}
&9901,9902,\dots, 10000\\
&9801,9802,\dots,9900\\
&\hspace{0.5in} \vdots \\ 
&101, 102,\dots,200\\
&1,2,\dots,100.
\end{align*}


The longest increasing subsequence in this whole list of 10000 numbers
has length 100.  That's because for any given number $n$ in the
list, the only numbers to the right that are also greater than $n$ are
those in the same contiguous subsequence as $n$.  So the only way to
form an increasing subsequence of the whole list is to use all the
numbers in a contiguous subsequence.

The longest decreasing subsequence in the whole list also has length
100.  That's because no decreasing subsequence of the whole list can
use more than one number from each of the 100 contiguous subsequences.
Also, any sequence numbers containing exactly one one number from each
of the 100 contiguous subsequences will be a decreasing subsequence of
length 100.
\end{solution}
\end{problem}


%%%%%%%%%%%%%%%%%%%%%%%%%%%%%%%%%%%%%%%%%%%%%%%%%%%%%%%%%%%%%%%%%%%%%
% Problem ends here
%%%%%%%%%%%%%%%%%%%%%%%%%%%%%%%%%%%%%%%%%%%%%%%%%%%%%%%%%%%%%%%%%%%%%

\endinput
