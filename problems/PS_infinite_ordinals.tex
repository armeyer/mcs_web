\documentclass[problem]{mcs}

\begin{pcomments}
  \pcomment{PS_infinite_ordinals}
  \pcomment{generalizes CP_finite_ordinals}
  \pcomment{revised ARM 3/14/18 from original 2/20/14}
  \pcomment{extra parts in staff notes at end should be worked out and added}
\end{pcomments}

\pkeywords{
  sets
  set_theory
  member
  subset
  union
  ordinal
  Foundation
  member-minimal
}

\newcommand{\nextset}[1]{\text{next}(#1)}
\newcommand{\Ord}{\text{Ord}}

\begin{problem}

\begin{definition*}
A set $y$ is \emph{full} when every element of $y$ is also a
subset of $y$, that is
\[
\forall x \in y.\ \ x \subset y.
\]

A set $z$ is \emph{closed under membership} when
\[
\forall x,y.\ \ (x \in y \in z) \QIMPLIES x \in z.
\]
\end{definition*}

\bparts

\ppart 
Verify that $z$ is closed under membership iff $z$ is full.

\begin{solution}
($\Rightarrow$) Suppose $z$ is closed under membership and $y \in z$.
  We need to show $y \subset z$, that is, $\forall x.\ x \in y \QIMP x
  \in z$.  But this follows immediately since $z$ is closed under
  membership.

($\Leftarrow$) Suppose $z$ is full and $x \in y \in z$.  We need to show
  $x \in z$.  But $y \subset z$ since $z$ is full, and $x \in y
  \subset z$ implies $x \in z$ by definition of $\subset$.
\end{solution}

\eparts

For any set $y$, define $\nextset{y}$ to be the set consisting of all
the elements of $y$, along with $y$ itself:
\[
\nextset{y}  \eqdef y \union \set{y}.
\]
So by definition,
\begin{equation}\tag{*}
x \in \nextset{y} \QIFF\ (x = y \QOR x \in y).
\end{equation}

Now we give a recursive definition of a collection $\Ord$ of sets
called \term{ordinals} that provide a way to ``count'' infinite sets.
Namely,
\begin{definition*}
\begin{itemize}
\item $\emptyset \in \Ord$,
\item if $\nu  \in \Ord$, then $\nextset{\nu} \in \Ord$,
\item  if $S$ is a set of ordinals, the $(\lgunion S) \in \Ord$.\qquad\qquad  (\dag)
\end{itemize}\end{definition*}

The recursive definition of ordinals means we can prove things about them using
structural induction.  Namely, let $P(y)$ be some property of sets.  The
\emph{Ordinal Induction Rule} says that to prove that $P(\nu)$ is true for all
ordinals $\nu$, you need only show three things:
\begin{itemize}
\item $P(\emptyset)$,
\item if $P(y)$, then $P(\nextset{y})$, and
\item if $S$ is a set of ordinals and $P(\nu)$ holds for all $\nu \in S$, then $P(\lgunion S)$.
\end{itemize}

\bparts

\ppart Prove that every ordinal $\nu$ is full.

\begin{solution}
\begin{proof}
The proof is by structural induction on the definition of \Ord, with
hypothesis
\[
P(z) \eqdef z \text{ is full}.
\]

\inductioncase{Case: ($z = \emptyset$)}.  $P(z)$ is vacuously true.

\inductioncase{Case: ($z = \nextset{y}$)}: We need to prove that $z$
is full, assuming by induction that $y$ is full.  That is, if $x \in
z$, then we need to prove $x \subset z$.

If $x=y$, then $x \subset z$ since $y \subset \nextset{y}$ by
definition.  Otherwise $x \in y$ by (*).  Since $y$ is full, we have
$x \subset y \subset \nextset{y}$, which proves $x \subset z$.

\inductioncase{Case: ($z = \lgunion S$)}: We need
to show that if $x \in z$, then $x \subset z$, assuming each $\nu \in S$
is full.

Now $x \in z$ implies $x \in \nu$ for some $\nu \in S$ by
definition of $\lgunion$, and since $\nu$ is full, $x \subset
\nu$.  But $\nu \subset z$ by definition of $\lgunion$, which
proves $x \subset z$ as required.
\end{proof}

\end{solution}

\eparts


\begin{staffnotes}
\TBA{further parts}

Call a set \emph{double full} if it and all its members are full.

\begin{lemma*}
If $x$ is double full and $y \in x$, then $y$ is double full.
\end{lemma*}
\begin{proof}
\TBA{easy?}
\end{proof}

Prove conversely, that is $x$ is double full, then $x\in\Ord$.  \hint
use Foundation.

\begin{proof}
Suppose there is a double full set not in \Ord.  Let $x$ be a
member-minimal such set.  If $y \in x$, then $y$ is double full by the
Lemma, so $y \in \Ord$ by member-minimality of $x$.  Then
\TBA{consider $\lgunion x$}.
\end{proof}

Prove that for all $\nu \in \Ord$,
\[
x \in \nu \QAND \nextset{x} \neq \nu) \QIMP \nextset(x) \in \nu.
\]
Conclude that $\nu \notin \mu \QIMP \mu \in \nu$ for all $\mu,\nu \in \Ord$.

Use this to prove that every nonempty set of ordinals has a
\emph{unique} member-minimal element.

The Axiom of Choice implies that every set has a bijection to some
ordinal.  This leads of to The punchline: we can define the
\emph{cardinality} $\card{S}$ of any set $S$ to be the necessarily
unique member-minimal ordinal $\nu$ such that $S \bij \nu$.  In this
was we arrive at the fundamental property that $S \bij T$ iff
$\card{S} = \card{T}$.

\end{staffnotes}

\eparts

\end{problem}

\endinput


\iffalse
\ppart Prove that $\nu \neq \emptyset$ implies $\emptyset \in \nu$ for
all ordinals $\nu$.

\begin{solution}
\begin{proof}
By ordinal induction with hypothesis
\[
P(x) \eqdef\  x \neq \emptyset qimplies \emptyset \in x
\]

\inductioncase{base case}: ($x = \emptyset$):  $P(x)$ holds vacuously.

\inductioncase{Constructor case}: ($x = \nextset{y}$): If $y =
\emptyset$, then $\emptyset \in \set{\emptyset} = \nextset{y} = x$.
If $y \neq \emptyset$, then by induction hypothesis $\emptyset \in y
\subset \nextset{y} = x$.

\inductioncase{Constructor case}: ($x = \lgunion_{i \in \nngint}
\nu_i$): We are given $\nu_0 \in \nu_1$, so $\nu_1 \neq \emptyset$.
By induction hypothesis $P(\nu_1)$, we conclude $\emptyset \in \nu_1 \subset x$.

\end{proof}
\end{solution}

\examspace[2in]
\fi
