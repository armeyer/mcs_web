\documentclass[problem]{mcs}

\begin{pcomments}
  \pcomment{PS_infinite_ordinals}
  \pcomment{generalizes CP_finite_ordinals}
  \pcomment{revised ARM 3/14/18 from original 2/20/14}
\end{pcomments}

\pkeywords{
  sets
  set_theory
  subset
  union
  ordinal
  member
}

\newcommand{\nextset}[1]{\text{next}(#1)}
\newcommand{\Ord}{\text{Ord}}

\begin{problem}
For any set $y$, define $\nextset{y}$ to be the set consisting of all
the elements of $y$, along with $y$ itself:
\[
\nextset{y}  \eqdef y \union \set{y}.
\]
So by definition,
\begin{equation}\tag{*}
x \in \nextset{y} \QIFF\ (x = y \QOR x \in y).
\end{equation}

Now we give a recursive definition of a collection $\Ord$ of sets
called \term{ordinals} that provide a way to ``count'' infinite sets.
Namely,
\begin{definition*}
\begin{itemize}
\item $\emptyset \in \Ord$,
\item if $\nu  \in \Ord$, then $\nextset{\nu} \in \Ord$,
\item if
\begin{equation}\tag{\dag}
\nu_0 \in \nu_1  \in \nu_2 \in \cdots \nu_n \in \nu_{n+1} \in \cdots
\end{equation}
is a sequence of ordinals, then $\lgunion_{i \in \nngint} \nu_i\  \in \Ord$.
\end{itemize}\end{definition*}

The recursive definition of ordinals means we can prove things about them using
structural induction.  Namely, let $P(y)$ be some property of sets.  The
\emph{Ordinal Induction Rule} says that to prove that $P(\nu)$ is true for all
ordinals $\nu$, you need only show three things:
\begin{itemize}
\item $P(\emptyset)$,
\item If $P(y)$, then $P(\nextset{y})$, and
\item if $P(\nu_i)$ holds for all $\nu_i$ in~(\dag), then
  $P(\lgunion_{i \in \nngint} \nu_i)$.
\end{itemize}

\begin{definition*}
A set $y$ is \emph{full} if every element of $y$ is also a
subset of $y$, that is
\[
\forall x \in y.\ \ x \subset y.
\]
\end{definition*}

\begin{staffnotes}
Possible preliminary exercise: Say that a set $z$ is \emph{closed
  under membership} iff $\forall x,y.\ x \in y \in z \QIMPLIES x \in
z$].  Verify that $z$ is closed under membership iff $z$ is full.

\begin{solution}
($\Rightarrow$) Suppose $z$ is closed under membership and $y \in z$.
  We need to show $y \subset z$, that is, $\forall x.\ x \in y \QIMP x
  \in z$.  But this follows immediately since $z$ is closed under
  membership.

($\Leftarrow$) Suppose $z$ is full and $x \in y \in z$.  We need to show
  $x \in z$.  But $y \subset z$ since $z$ is full, and $x \in y
  \subset z$ implies $x \in z$ by definition of $\subset$.
\end{solution}
\end{staffnotes}

Prove that every ordinal $\nu$ is full.

\begin{solution}
\begin{proof}
The proof is by structural induction onthe definition of \Ord, with hypothesis
\[
P(z) \eqdef z \text{ is full}.
\]

\inductioncase{Case: ($z = \emptyset$)}.  $P(z)$ is vacuously true.

\inductioncase{Case: ($z = \nextset{y}$)}: We need to prove that $z$
is full, assuming by induction that $y$ is full.  That is, if $x \in
z$, then we need to prove $x \subset z$.

If $x=y$, then $x \subset z$ since $y \subset \nextset{y}$ by
definition.  Otherwise $x \in y$ by (*).  Since $y$ is full, we have
$x \subset y \subset \nextset{y}$, which proves $x \subset z$.

\inductioncase{Case: ($z = \lgunion_{i \in \nngint} \nu_i$)}: We need
to show that if $x \in z$, then $x \subset z$, assuming each $\nu_i$
is full.

Now $x \in z$ implies $x \in \nu_k$ for some $k \in \nngint$ by
definition of $\lgunion$, and since $\nu_k$ is full, $x \subset
\nu_k$.  But $\nu_k \subset z$ by definition of $\lgunion$, which
proves $x \subset z$ as required.
\end{proof}

\end{solution}

\begin{editingnotes}
Prove $forall S \in \text{Sets} \exists \nu \in \Ord.\ S \bij \nu$??
\end{editingnotes}
%\eparts

\end{problem}

\endinput
