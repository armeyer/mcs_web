\documentclass[problem]{mcs}

\begin{pcomments}
  \pcomment{FP_frog_jump_fibonacci}
  \pcomment{CH, Spring '14}
\end{pcomments}

\pkeywords{
	linear recurrences
        counting
        generating functions
        Fibonacci numbers
}

%%%%%%%%%%%%%%%%%%%%%%%%%%%%%%%%%%%%%%%%%%%%%%%%%%%%%%%%%%%%%%%%%%%%%
% Problem starts here
%%%%%%%%%%%%%%%%%%%%%%%%%%%%%%%%%%%%%%%%%%%%%%%%%%%%%%%%%%%%%%%%%%%%%

\begin{problem}

Kermit the Frog wants to climb up a stairwell with $n$ steps. He can
only make leaps in steps of 1 (called a \emph{hop}) or 2 (called a
\emph{skip}). 

Define $f_n$ to be the number of different ways in which Kermit
can reach the top of the stairwell. For example,  $f_2 = 2$ since
there are exactly two moves (a \emph{hop-hop}, or a \emph{skip}) available for Kermit to reach the
top. On the other hand, $f_3 = 3$, since Kermit can do a
\emph{hop-hop-hop}, or a \emph{hop-skip}, or a \emph{skip-hop} to
reach the top.

\bparts 

\ppart State a linear recurrence for $f_n$ for all $n \geq 3$. Reason why this is
correct (a fully rigorous proof is not required here).

\hint Related to a famous sequence of numbers.

\begin{solution}
\[
f_n = f_{n-1} + f_{n-2} .
\]
The recurrence is the same as the one used to define the Fibonacci
sequence (although the base cases are different).

The argument is as follows. Number the stairs from $1$ to $n$.  Every valid sequence of jumps terminating
at $n$ either ends with a \emph{hop} (and therefore passes through $n-1$), or
ends with a \emph{skip} (and therefore passes
through $n-2$). By definition, there are $f_{n-1}$ ways to arrive at
$n-1$ and $f_{n-2}$ ways to arrive at $n-2$. The recurrence follows. 

\end{solution}

\examspace[2.5in]

\ppart Let $F(x)$ be the generating function for the sequence $f_1,
f_2 \ldots$, i.e,
\[
F(x) = f_1 x + f_2 x^2 + \ldots + f_n x^n + \ldots .
\]
Derive a closed form expression for $F(x)$.

\begin{solution}
We know that $f_1 = 1$ (since there is only one move --- a
\emph{hop} --- permissible) and $f_2 = 2$ (given). 

As with the derivation for Fibonacci sequences, 
\begin{align*}
F(x) &= f_1 x + f_2 x^2 + f_3 x^3 + \ldots + f_n x^n + \ldots \\
-x F(x) &= -f_1 x^2 - f_2 x^3 - \ldots - f_{n-1} x^n -
\ldots \\
-x^2 F(x) &= -f_1 x^3 - \ldots - f_{n-2} x^n - \ldots .
\end{align*}
Adding, we get:
\begin{align*}
F(x) (1-x-x^2) &= f_1 x + (f_2 - f_1) x^2 + 0 x^3  + \ldots + 0 x^n +
\ldots \\
&= x + x^2 . 
\end{align*}
Therefore,
\[
F(x) = \frac{x + x^2}{1 - x - x^2} . 
\]
 
\end{solution}

\eparts
\end{problem}

%%%%%%%%%%%%%%%%%%%%%%%%%%%%%%%%%%%%%%%%%%%%%%%%%%%%%%%%%%%%%%%%%%%%%
% Problem ends here
%%%%%%%%%%%%%%%%%%%%%%%%%%%%%%%%%%%%%%%%%%%%%%%%%%%%%%%%%%%%%%%%%%%%%

\endinput
