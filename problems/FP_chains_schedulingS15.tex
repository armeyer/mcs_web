\documentclass[problem]{mcs}

\begin{pcomments}
  \pcomment{CP_chains_scheduling}
  \pcomment{CP version of FP_chains_scheduling}
  \pcomment{overlaps PS_Brents_theorem}
  \pcomment{from S02.final, F03.final}
  \pcomment{FP-converted from CP by AC 5/17/15}
\end{pcomments}

\pkeywords{
 chain
 antichain
 topological
 schedule
}

%%%%%%%%%%%%%%%%%%%%%%%%%%%%%%%%%%%%%%%%%%%%%%%%%%%%%%%%%%%%%%%%%%%%%
% Problem starts here
%%%%%%%%%%%%%%%%%%%%%%%%%%%%%%%%%%%%%%%%%%%%%%%%%%%%%%%%%%%%%%%%%%%%%

\begin{problem}

Sauron finds that conquering Middle Earth breaks down into a bunch of
tasks.  Each task can be completed by a horrible creature called a
\emph{Ringwraith} in exactly one week.  Sauron realizes the
prerequisite structure among the tasks defines a DAG.  He has $n$
tasks in his DAG, with a maximum length chain of $t$ tasks.

If Sauron is lucky, he will be able to get away with a small
crew of Ringwraiths.  Write a simple formula involving $n$ and $t$ for
the smallest number of Ringwraiths that could possibly be able to
 complete all $n$ tasks in $t$ weeks.  (Do not make any additional
assumptions about the relative sizes of $n$ and $t$ besides $t \leq
n$.)  Given any $n$ and $t$, describe a DAG that can be completed in $t$
weeks using this number of Ringwraiths.

%\begin{center}
%\exambox{1.0in}{0.3in}{0.3in}
%\end{center}
\examspace[0.5in]

\begin{solution}
\[
\ceil{\frac{n}{t}}
\]

Each of the $n$ tasks must be completed in one of the $t$ weeks.
Thus, by the pigeonhole principle,
at least $\ceil{n/t}$ must be
completed in some week, requiring at least that many Ringwraiths.

This bound can't be made larger, since a partial order with a single
length $t$ chain and the other $n-t$ tasks with no prerequisites
can be completed in this little time.
\end{solution}

\end{problem}

%%%%%%%%%%%%%%%%%%%%%%%%%%%%%%%%%%%%%%%%%%%%%%%%%%%%%%%%%%%%%%%%%%%%%
% Problem ends here
%%%%%%%%%%%%%%%%%%%%%%%%%%%%%%%%%%%%%%%%%%%%%%%%%%%%%%%%%%%%%%%%%%%%%

\endinput
 
