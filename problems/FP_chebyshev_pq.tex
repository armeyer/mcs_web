\documentclass[problem]{mcs}

\begin{pcomments}
  \pcomment{FP_chebyshev_pq}
  \pcomment{generalizes TP_Flipping_coins}
  \pcomment{Converted from prob2.scm
              by scmtotex and dmj
              on Sun 13 Jun 2010 05:11:14 PM EDT}
  \pcomment{edited by drewe 14 july 2011}
\end{pcomments}

\begin{problem}

You have a biased coin which flips Heads with probability $p$.  You
flip the coin $n$ times.  The coin flips are all mutually independent.

\bparts

\ppart\label{expected} Write a simple expression in terms of $p$ and
$n$ for the expected number of heads.

\begin{center}
\exambox{0.7in}{0.4in}{0in}
\end{center}

\examspace[0.2in]

\begin{solution}
$\mathbf{np}$.

Let $X$ denote the number of heads.  Let $X_{i}$ be the indicator
variable that is 1 if and only if the $i$th coin flip comes out heads (and 0 otherwise). Then
\begin{equation*}
    X = X_{1} + X_{2} + \dots + X_{n}.
\end{equation*}

Hence, by linearity of expectation, 
\[
\expect{X}
    = \expect{X_{1} + X_{2} + \cdots + X_{n}}
    = \expect{X_{1}} + \cdots + \expect{X_{n}}.
\]
The expectation of an indicator variable is the probability it
equals~1\inbook{ by Lemma~\bref{expindic}}.  Hence, $\expect{X_{i}} =
p$.  We conclude that $\expect{X} = n \cdot p$.
\end{solution}

\ppart What upper bound can we derive directly from Markov's Theorem
for the probability that the number of heads is at least 25\% larger
than expected?

\begin{center}
\exambox{0.7in}{0.4in}{0in}
\end{center}

\examspace[0.5in]

\begin{solution}
\textbf{1/1.25 = 0.8}.

The Markov for being $a$ times the mean is $1/a$.  Note that this does
not depend on $n$ or $p$.
\end{solution}

\ppart Write a simple expression in terms of $p$ and $n$ for the
variance of the number of heads.

\begin{center}
\exambox{0.7in}{0.4in}{0in}
\end{center}

\examspace[0.75in]

\begin{solution}
$\mathbf{np(1-p)}$.

By the independence of the $X_{i}$, we know
\[
\Var[X] = 
\Var[X_{1} + \cdots  + X_{n}] = 
\Var[X_{1}]+ \cdots  + \Var[X_{n}].
\]
Finally, we know \inbook{by Corollary~\bref{bernoulli-variance}} the
variance of an indicator with expectation $p$ is $p(1-p)$.
\end{solution}

\ppart\label{C25bnd} What upper bound does Chebyshev's Theorem give us on the
probability that the number of heads is at least 25\% larger than
expected?

\begin{center}
\exambox{0.7in}{0.4in}{0in}
\end{center}

\begin{solution}
\[
\paren{\frac{4}{5}}^2\frac{1-p}{np}
\]

Now, Chebyshev's Theorem says that the probability that $X$ deviates
from its expected value by 1.25 times the expected value $\mu$, is at most
\[
\frac{\variance {X}}{(1.25\mu)^2} = \frac{np(1-p)}{(1.25np)^2} =
\paren{\frac{4}{5}}^2\frac{(1-p)}{np}.
\]
\end{solution}

\iffalse \ppart According to the bound in part~\eqref{C25bnd}, how big
must $n$ be to ensure an at most 1\% chance of exceeding the expected
value by 25\% when the odds are four to one in favor of Tails??

\begin{solution}
Odd of four to one in favor of Tails means $p = 1/5$.  So
\[
\paren{\frac{4}{5}}^2\frac{1- 1/5}{n(1/5)}\leq \frac{1}{100}.
\]
So
\[
n \geq 4^3\frac{1-p}{p}.
\]
\end{solution}
\fi

\eparts

\end{problem}

\endinput
