\documentclass[problem]{mcs}

\begin{pcomments}
  \pcomment{new problem for final exam by YingZ, May 2014}
\end{pcomments}

\pkeywords{
  expectation
  conditional expectation
  linearity of expectation
  deviation
  sampling
  confidence
  tree diagram (4-step method)
  sample_space
  outcome
  probability space
}

%%%%%%%%%%%%%%%%%%%%%%%%%%%%%%%%%%%%%%%%%%%%%%%%%%%%%%%%%%%%%%%%%%%%%
% Problem starts here
%%%%%%%%%%%%%%%%%%%%%%%%%%%%%%%%%%%%%%%%%%%%%%%%%%%%%%%%%%%%%%%%%%%%%


\begin{problem}

%\begin{staffnotes}
% Point distribution goes here!
%\end{staffnotes}

Members of MIT�s Computer Science and Artificial Intelligence Laboratory (CSAIL) are known to thrive on caffeine.  Sixty percent are devoted coffee drinkers, who drink coffee and water exclusively and with equal probability in the lab.  Ten percent are devoted tea drinkers, who drink tea 100% of the time.  Ten percent of lab members are devoted soda drinkers, who drink soda 30\% of the time and water or juice with equal probability the rest of the time.  The remaining twenty percent drink all three types of beverages randomly half of the time, and water and juice with equal probability the rest of the time.

\bparts

\ppart As a 6.042 student, you go to CSAIL for office hours.  As soon as you get off the elevator, you see two people each holding a cup in front of the white board and discussing something mod something.  What is the probability that both of them are drinking coffee?  

\hint Draw a tree diagram first.

\examspace[2in]

\begin{solution}
\end{solution}

\ppart Assuming the two filled their cups independently.  What is the probability that their cups contain the same kind of beverage? 

\begin{solution}
\end{solution}


\eparts

\end{problem}

%%%%%%%%%%%%%%%%%%%%%%%%%%%%%%%%%%%%%%%%%%%%%%%%%%%%%%%%%%%%%%%%%%%%%
% Problem ends here
%%%%%%%%%%%%%%%%%%%%%%%%%%%%%%%%%%%%%%%%%%%%%%%%%%%%%%%%%%%%%%%%%%%%%

\endinput
