\documentclass[problem]{mcs}

\begin{pcomments}
  \pcomment{FP_infinite_repeat}
  \pcomment{compare to TP_infinite_repeat}
  \pcomment{by ARM 5/13/11}
\end{pcomments}

\pkeywords{
  infinite
  expectation
}

%%%%%%%%%%%%%%%%%%%%%%%%%%%%%%%%%%%%%%%%%%%%%%%%%%%%%%%%%%%%%%%%%%%%%
% Problem starts here
%%%%%%%%%%%%%%%%%%%%%%%%%%%%%%%%%%%%%%%%%%%%%%%%%%%%%%%%%%%%%%%%%%%%%

\begin{problem}

You have a process for generating a positive integer, $K$.  The
behavior of your process each time you use it is (mutually)
independent of all its other uses.  You use your process to generate a
random integer, and then use your procedure repeatedly until you
generate an integer as big as your first one.  Let $R$ be the number
of additional integers you have to generate.

\bparts

\ppart\label{probKgeqk} State and briefly explain a simple formula for
$\expcond{R}{K=k}$ in terms of $\prob{K \geq k}$.

\examspace[1in]

\begin{solution}
\[
\expcond{R}{K=k} = \frac{1}{\prob{K \geq k}}
\]

If $K=k$, then we can think of generating a number $\geq k$ as a
failure, so the expected number of repeats is the mean time to
failure, which is the reciprocal of probability of failure on a given
try.
\end{solution}

\eparts
\vspace{0.2in}
Suppose $\prob{K = k} = \Theta(k^{-4})$.

\bparts

\ppart \label{probKgeqkTheta}
Show that $\prob{K \geq k} = \Theta(k^{-3})$.

\examspace[2.5in]

\begin{solution}
It follows from the definition of $\Theta()$ that
\begin{equation}\label{Thetasum}
\text{If $f = \Theta(g)$, then $\sum_{n \in S} f(n) = \Theta\paren{\sum_{n \in S} g(n)}$}
\end{equation}
for any countable set $S$.  Therefore, with a slight abuse of
$\Theta()$ notation, we have
\begin{align*}
\prob{K \geq k}
   & = \sum_{m \geq k} \prob{K = m}\\
   & = \Theta\paren{\sum_{m \geq k} k^{-4}} & \text{by~\eqref{Thetasum}}\\
   & = \Theta\paren{\int_k^{\infty} x^{-4}\, dx} & \text{(the Integral Method)}\\
   & = \Theta\paren{k^{-3}}
\end{align*}

\end{solution}

\ppart  Show that $\expect{R}$ is infinite.

\examspace[3in]

\begin{solution}
It follows from the definition of $\Theta()$ that
\begin{equation}\label{Thetamult}
\text{If $f = \Theta(g)$ and $f^\prime = \Theta(g^\prime)$, then $f \cdot
  f^\prime = \Theta(g \cdot g^\prime)$.}
\end{equation}
Now we have By Total Expectation:
\begin{align*}
\expect{R}
   & = \sum_{k \in \integers^+} \expcond{R}{K=k} \cdot \prob{K=k}\\
   & = \sum_{k \in \integers^+} \frac{1}{\prob{K \geq k}} \cdot \prob{K=k}
           & \text{by part~\eqref{probKgeqk}}\\
   & = \Theta\paren{\sum_{k \in \integers^+} k^3 \cdot k^{-4}}
            & \text{(by part~\eqref{probKgeqkTheta}, equations~\eqref{Thetamult} and~\eqref{Thetasum})}\\
   & = \Theta\paren{\sum_{k \in \integers^+} k^{-1}} = \infty.
\end{align*}
\end{solution}

\eparts

\end{problem}

%%%%%%%%%%%%%%%%%%%%%%%%%%%%%%%%%%%%%%%%%%%%%%%%%%%%%%%%%%%%%%%%%%%%%
% Problem ends here
%%%%%%%%%%%%%%%%%%%%%%%%%%%%%%%%%%%%%%%%%%%%%%%%%%%%%%%%%%%%%%%%%%%%%

\endinput
 
