\documentclass[problem]{mcs}

\begin{pcomments}
  \pcomment{MQ_cycle_from_closed_walk}
  \pcomment{candidate problem introduced by ORM, S10}
\end{pcomments}

\pkeywords{
  digraphs
  cycle
  walk
  closed_walk
}

\begin{problem}
Prove the following claim:

Suppose there is a positive length walk in digraph $G$ that starts and
ends at node $v$.  Then $G$ has a cycle.

\inhandout{Recall that a \emph{cycle} is a positive length walk whose
  only repeat vertex is the start and end vertex.}

\begin{solution}
The set $L$ of \emph{lengths} of all positive-length walks starting
and ending in $v$ is a set of non-negative integers and is non-empty,
since from the hypothesis we know of at least one cycle.  By the Well
Ordering Principle, there exists a smallest $n_0 \in L$. Consider any
walk in $G$ with length $n_0$.  We prove it is a cycle.

Suppose for the purpose of contradiction that some walk of size
$n_0$ is not a cycle.  Then there is a repeated node $x$ in
the walk, in addition to $v$ (the start and end vertex).  The walk
looks like
\[
v,\dots ,x,\dots ,y,x,\dots ,v
\]
where $x \neq v$.  If we remove the segment $x, \cdots ,y$ we again
have a valid walk of length $n'$, still positive, such that $0 < n' <
n_0$, a contradiction.
\end{solution}

\end{problem}
