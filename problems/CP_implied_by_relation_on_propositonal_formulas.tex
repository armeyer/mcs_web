%CP_images_and-propositional_formulas

\documentclass[problem]{mcs}

\begin{pcomments}
  \pcomment{from: Megumi Ando, 9/19/09, reworked by ARM}
\end{pcomments}

\pkeywords{
  propositional formula
  implies
  valid
  relation
  codomain
  image
  inverse
  graph_of_relation
}

%%%%%%%%%%%%%%%%%%%%%%%%%%%%%%%%%%%%%%%%%%%%%%%%%%%%%%%%%%%%%%%%%%%%%
% Problem starts here
%%%%%%%%%%%%%%%%%%%%%%%%%%%%%%%%%%%%%%%%%%%%%%%%%%%%%%%%%%%%%%%%%%%%%

\begin{problem}
  Let $A$ be the following set of five propositional formulas shown below
  on the left, and let $C$ be the set of three propositional formulas on
  the right.  The ``is implied by'' binary relation, $J$, from $A$ to $C$
  is defined by the rule
\[
F \mrel{J} G \qiff [\text{the formula } (G \QIMPLIES F) \text{ is valid}].
\]
For example, $P \mrel{J} (P \QAND Q)$, because $P$ is implied by $(P \QAND
Q)$.  Also, it is not true that $P \mrel{J} (P \QOR Q)$ since $P$ is not
implied by $(P \QOR Q)$.

  \bparts
  \ppart Fill in the arrows so the following figure describes the graph of
  the relation, $J$:

\[\begin{array}{lcr}
A & \hspace{1in} \text{arrows} \hspace{1in} & C\\
\hline
&\\
M\\
&\\
                                  && M \QAND (P \QIMPLIES M)\\
&\\
P \QAND Q\\
&\\
                                  && Q\\
&\\
P \QOR Q\\
&\\
                                  && \bar{P} \QOR \bar{Q}\\
&\\
\QNOT(P \QAND Q)\\
&\\
&\\
P \QXOR Q\\
&\\
\end{array}\]

 \begin{solution}
Three arrows for the ``is implied by'' relation, $J$, 
\begin{align*}
M & \text{is implied by} & M \QAND (P \QIMPLIES M)\\
P \QXOR Q & \text{is implied by} & \bar{P} \QOR \bar{Q}\\
P \QOR Q & \text{is implied by} & Q
\end{align*}
So the ``is implied by'' relation, $J$, is a surjective, injective function.
This implies that its inverse, $\inv{J}$ is a total, injective
function.

 \end{solution}

 \ppart\label{Iprops} Circle the properties below possessed by the
 relation $J$:
  \[
  \begin{array}{ccccc}
  \mbox{FUNCTION~~} & 
  \mbox{~~TOTAL~~} &
  \mbox{~~INJECTIVE~~} &
  \mbox{~~SURJECTIVE~~} &
  \mbox{~~BIJECTIVE} 
  \end{array}
  \]

  \ppart  Circle the properties below possessed by the relation $\inv{J}$:
  \[
  \begin{array}{ccccc}
  \mbox{FUNCTION~~} & 
  \mbox{~~TOTAL~~} &
  \mbox{~~INJECTIVE~~} &
  \mbox{~~SURJECTIVE~~} &
  \mbox{~~BIJECTIVE~}
  \end{array}
  \]

\iffalse
%THERE is a good problem idea here, but what's written is garbled
%and needs work.

Here are some two groups of expressions involving images and inverse
images under the relation $I$:
\begin{enumerate}
\item
${P,Q}I$

\begin{solution}
$\set{(R \QOR \bar{R})}$.

${P,Q}I$ is by definition equal to the expressions in $C$ that are either
implied by $P$ or implied by $Q$.  The only expression in $C$ implied by
$P$ is the valid expression $(R \QOR \bar{R})$; this is also the only
expression in $C$ implied by $Q$.
\end{solution}

\item $\set{(P \QOR Q)}I \intersect \set{\QNOT(P\ \QAND\ Q)}I$

\begin{solution}
\[\begin{array}{l}
(P \XOR Q)I\\
\set{(\bar{P} \QOR \bar{Q}), (R \QOR \bar{R})}
\end{array}\]

These are the expressions in $C$ implied by both $(P \QOR Q)$ and also by
$\QNOT(P\ \QAND\ Q)$, which is the same as being implied by $P \XOR Q$.

\end{solution}

\item \set{(P \QAND Q),\QNOT(P\ \QAND\ Q)}I

\begin{solution}
These are the expressions in $C$ implied by the formula $(P \QAND Q)$ or by
$\QNOT(P \QAND Q)$

\end{solution}
\end{enumerate}
\fi

\iffalse

\ppart If we change the codomain $C$ to include $\bar{P}$, does your
answer to part~eqref{Iprops} change?  Explain your answer.

\fi

  \eparts

\end{problem}

%%%%%%%%%%%%%%%%%%%%%%%%%%%%%%%%%%%%%%%%%%%%%%%%%%%%%%%%%%%%%%%%%%%%%
% Problem ends here
%%%%%%%%%%%%%%%%%%%%%%%%%%%%%%%%%%%%%%%%%%%%%%%%%%%%%%%%%%%%%%%%%%%%%

\endinput
