\documentclass[problem]{mcs}

\begin{pcomments}
  \pcomment{FP_graph_logic_probability}
  \pcomment{from: S08.final}
\end{pcomments}

\pkeywords{
  graph
  logic
  probability
  independence
}

%%%%%%%%%%%%%%%%%%%%%%%%%%%%%%%%%%%%%%%%%%%%%%%%%%%%%%%%%%%%%%%%%%%%%
% Problem starts here
%%%%%%%%%%%%%%%%%%%%%%%%%%%%%%%%%%%%%%%%%%%%%%%%%%%%%%%%%%%%%%%%%%%%%

\begin{problem} \textbf{Graphs, Logic \& Probability}

  Let $G$ be an undirected simple graph with $n>3$ vertices.  \iffalse
  Suppose we want to define define the predicate $E(x,y)$ to be true when
  $G$ has an edge between $x$ and $y$ and the predicate \fi Let $E(x,y)$
  mean that $G$ has an edge between vertices $x$ and $y$, and let $P(x,y)$
  mean that there is a length 2 walk in $G$ between $x$ and $y$.

\bparts

% This part seems inconsistent with later elements, re: paths vs. walks.
\iffalse
\ppart Explain why $E(x,y)$ implies $P(x,x)$.

\begin{solution}
Going back and forth on edge $\edge{x}{y}$ is a
  length two walk from $x$ to $x$.  
\end{solution}
\fi

% Exam version commented out for use as a class problem.
\iffalse
\ppart Circle the mathematical formula that best expresses the
definition of $P(x,y)$.

\begin{itemize}
\item $P(x,y) \eqdef \exists z.\ E(x,z) \QAND\ E(y,z)$
\vspace{0.15in}
\item $P(x,y) \eqdef x \neq y \QAND\ \exists z.\ E(x,z) \QAND\ E(y,z)$
\vspace{0.15in}
\item $P(x,y) \eqdef \forall z.\ E(x,z) \QOR\ E(y,z)$
\vspace{0.15in}
\item $P(x,y) \eqdef \forall z.\ x \neq y \QIMPLIES\ [E(x,z) \QOR\ E(y,z)]$
\end{itemize}
\fi

\ppart Write a predicate-logic formula defining $P(x,y)$ in terms of $E(x,y)$.

\begin{solution}
$P(x,y) \eqdef \exists z.\ z \neq x \QAND\ z \neq y \QAND\ E(x,z) \QAND\ E(z,y)$.
\end{solution}

For the following parts~\eqref{EE}--\eqref{3cycle}, let $V$ be a fixed set
of $n>3$ vertices, and let $G$ be a graph with these vertices constructed
randomly as follows: for all distinct vertices $x,y \in V$, independently
include edge $\edge{x}{y}$ as an edge of $G$ with probability $p$.  In
particular, $\pr{E(x,y)} = p$ for all $x \neq y$.

\ppart\label{EE} For distinct vertices $w$, $x$, $y$ and $z$ in $V$, 
circle the event pairs that are independent.

\begin{staffnotes}
In class, ask for explanations.
\end{staffnotes}

\insolutions{Here our sample space consists of sets of edges.  In
  answering each individual question below, we rely on a key property:
  since edges are selected independently, two events are independent
  if \textbf{they depend on disjoint sets of edges}.  Formally, we say
  that an event $E$ depends on edge $\edge{u}{v}$ iff there exists
  $\omega$ such that exactly one of $\omega$ and $\omega \cup
  \{\edge{u}{v}\}$ is in $E$.}

\begin{enumerate}

\item $E(w,x)$ versus $E(x,y)$\label{EE1}

\insolutions{\True, as obviously each side depends on a single, different edge.}

\vspace{0.15in}

\item\label{wz-indep} $[E(w,x) \QAND\ E(w,y)]$ versus
      $[E(z,x) \QAND\ E(z,y)]$

\insolutions{\True, as obviously each side depends on two edges, and
  no edge could appear on both sides.}

\vspace{0.15in}

\item $E(x,y)$ versus $P(x,y)$

\insolutions{\True, as $\edge{x}{y}$ can't possibly be
  involved in a length-2 walk from $x$ to $y$, since it could only be
  connected to a self-loop for $x$ or $y$ to itself, which we disallow
  in simple graphs.}

\vspace{0.15in}

\item $P(w,x)$ versus $P(x,y)$ 

\insolutions{\False, as demonstrated by the counterexample of
  $\card{V} = 4$ and $p = \frac{1}{2}$.

\vspace{0.05in}

Original Solution: In this counterexample,
\[
P(w,x) \equiv (E(w,y) \QAND{} E(y,x)) \QOR{} (E(w,z) \QAND{} E(z,x))
\]
and
\[
P(x,y) \equiv (E(x,w) \QAND{} E(w,y)) \QOR{} (E(x,z) \QAND{} E(z,y)).
\]

By symmetry, we apply inclusion-exclusion to calculate the probability
for either of these events:
\[
\paren{\frac{1}{2}}^2 + \paren{\frac{1}{2}}^2 - \paren{ \frac{1}{2}}^4 = \frac{7}{16}.
\]

Now consider $\prcond{P(w,x)}{P(x,y)}$, the fraction of outcomes
satisfying $P(x, y)$ that also satisfy $P(w, x)$.  Partition the
outcomes satisfying $P(x, y)$ by whether they also satisfy $E(w, y)$.
Both sides of the partition are independent of $E(y, x)$ in the sense
formalized above, since $E(y, x)$ doesn't appear in the definition of
$P(x, y)$.  That means that the outcomes in the subcase for $E(w, y)$
can be partitioned into equally sized sets, one with $E(y, x)$ and the
other with $\bar{E(y, x)}$.  Clearly every element of the first set
satisfies $P(w, x)$, so
\[
\prcond{P(w, x)}{P(x, y) \QAND{} E(y, x)} \geq \frac{1}{2}.
\]

The outcomes in the subcase for $\bar{E(w, y)}$ must all have $E(x,
z)$, so, like above partitioning them based on $E(w, z)$, we get two
equal-size sets, where the set with $E(w, z)$ all satisfy $P(w, x)$,
and
\[
\prcond{P(w, x)}{P(x, y) \QAND\ \bar{E(y, x)}} \geq \frac{1}{2}.
\]
The true value of $\prcond{P(w, x)}{P(x, y)}$ must lie somewhere
between these two values, so it also must be no less than
$\frac{1}{2}$, and thus it must be greater than $\pr{P(w, x)} =
\frac{7}{16}$.

\vspace{0.05in} Less complicated explanation: We want to show that
$\prcond{P(w,x)}{P(x,y)} \neq \Pr{P(w,x)}$.  If $P(x,y)$, this
increases the probability of $E(z,x)$ and $E(w,y)$, which can be used
for $P(w,x)$.}

\vspace{0.15in}

\item $P(w,x)$ versus $P(y,z)$

\insolutions{\False, by similar reasoning to in the last part.  For
  $|V| = 4$ and $p = \frac{1}{2}$, we have
\[
P(w,x) \equiv (E(w,y) \QAND\ E(y,x)) \QOR\ (E(w,z) \QAND\ E(z,x))
\]
 and
\[
P(y,z) \equiv (E(y,x) \QAND\ E(x,z)) \QOR\ (E(y,w) \QAND\ E(w,z)).
\]

\[
\prcond{P(w, x)}{P(y, z) \QAND{} (E(y,x) \QAND{} E(x,z))} = \frac{3}{4},
\]
since under these conditions the formula for $P(w,x)$ simplifies to
$E(w,y) \QOR E(w,z)$.
\[
\prcond{P(w, x)}{P(y, z) \QAND\ \QNOT (E(y,x) \QAND\ E(x,z))} = \frac{1}{2}
\]
since under these conditions we know $E(y,w) \QAND\ E(w,z)$, and the
formula for $P(w,x)$ simplifies to 
\[
(E(y,x) \QOR\ E(z,x)) \QAND\ \QNOT(E(y,x) \QAND\ E(z,x)).
\]
Both sides of the partition have probabilities no less than
$\frac{1}{2}$, so $\prcond{P(w, x)}{P(y, z)} \geq \frac{1}{2}$, which
again is above $\pr{P(w, x)}$, which can be computed as $\frac{7}{16}$
as in the prior case.}

\end{enumerate}

%\examspace

\ppart \label{nottwopath} \; Write a simple formula in terms of $n$ and
$p$ for $\prob{\QNOT P(x,y)}$, for distinct vertices $x$ and $y$ in
$V$.

\hint Use part~\eqref{EE}, item~\ref{wz-indep}.
 
\begin{solution}
Let $Z \eqdef V - \set{x,y}$ be the set of all
  the vertices other than $x$ and $y$.

\begin{align*}
\pr{\QNOT(P(x,y))}
  & = \pr{\QAND_{z \in Z} \bar{E(x,z) \QAND\ E(y,z)}}\\
  & = \prod_{z \in Z} \pr{\bar{E(x,z) \QAND\ E(y,z)}} &
    \text{(indep. from item~\eqref{EE}~\ref{wz-indep})}  \\
  & = \prod_{z \in Z} (1-\pr{E(x,z)}\cdot\pr{E(y,z)}) & 
    \text{(indep. from item~\eqref{EE}~\ref{EE1})}\\
  & = \prod_{z \in Z} (1-p^2) & \\
  & = (1-p^2)^{n-2}
\end{align*}
\end{solution}

\ppart \label{3cycle} \; What is the probability that two distinct
vertices $x$ and $y$ lie on a three-cycle in $G$?  Answer with a
simple expression in terms of $p$ and $r$, where $r \eqdef
\pr{\QNOT(P(x,y))}$ is the correct answer to part~\eqref{nottwopath}.

\hint Express $x$ and $y$ being on a three-cycle as a simple formula
involving $E(x,y)$ and $P(x,y)$.

\begin{solution}
$x$ and $y$ lie on a three-cycle iff $E(x,y) \QAND\
P(x,y)$.

Since $E(x,y)$ and $P(x,y)$ are independent,
\begin{align*} 
\pr{E(x,y) \QAND\ P(x,y)} & = \pr{E(x,y)} \cdot \pr{P(x,y)} \\
                        & = p (1-r) .
\end{align*}
Substituting in for $r$ (not asked), we get
\[
\pr{E(x,y)\QAND\ P(x,y)} = p (1-(1-p^2)^{n-2}).
\]
\end{solution}

\eparts
\end{problem}

%%%%%%%%%%%%%%%%%%%%%%%%%%%%%%%%%%%%%%%%%%%%%%%%%%%%%%%%%%%%%%%%%%%%%
% Problem ends here
%%%%%%%%%%%%%%%%%%%%%%%%%%%%%%%%%%%%%%%%%%%%%%%%%%%%%%%%%%%%%%%%%%%%%

\endinput
