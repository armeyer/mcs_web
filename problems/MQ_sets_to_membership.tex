\documentclass[problem]{mcs}

\begin{pcomments}
\pcomment{MQ_sets_to_membership}
\pcomment{Created by Joliat and Kazerani; variant of CP_proving_basic_set_id, 2/27/11}
\end{pcomments}

\pkeywords{
  logic
  set_theory
  identity
  propositional
  chain_of_iff
  difference
}

%%%%%%%%%%%%%%%%%%%%%%%%%%%%%%%%%%%%%%%%%%%%%%%%%%%%%%%%%%%%%%%%%%%%%
% Problem starts here
%%%%%%%%%%%%%%%%%%%%%%%%%%%%%%%%%%%%%%%%%%%%%%%%%%%%%%%%%%%%%%%%%%%%%

\begin{problem}

Suppose we wish to prove the following identity:
\[ \bar{A-B} = \left(\bar{A}-\bar{C}\right)\union\left(B\intersect C\right)\union\left[\left(\bar{A}\union B\right)\intersect\bar{C}\right] \]

\begin{solution}

The identity holds iff membership in $\bar{A-B}$ implies and is implied by membership in $\left(\bar{A}-\bar{C}\right)\union\left(B\intersect C\right)\union\left[\left(\bar{A}\union B\right)\intersect\bar{C}\right]$.  That is, the given identity holds iff, for all $x$,
\[ x \in \bar{A-B} \qiff x \in \left\{\left(\bar{A}-\bar{C}\right)\union\left(B\intersect C\right)\union\left[\left(\bar{A}\union B\right)\intersect\bar{C}\right]\right\}\]
Define three propositions describing the membership of $x$ in each of the sets $A$, $B$, and $C$:
\begin{eqnarray*}
P & \eqdef & x\in A\\
Q & \eqdef & x\in B\\
R & \eqdef & x\in C\\
\end{eqnarray*}
Now, express membership in $\bar{A-B}$ in terms of $P$, $Q$, and $R$:
\begin{align*}
\lefteqn{x \in \bar{A-B}}\\
 & \qiff \bar{x \in \left(A\intersect\bar{B}\right)}\\
 & \qiff \bar{x \in A \QAND x \in \bar{B}}\\
 & \qiff \bar{x \in A \QAND \bar{x \in B}}\\
 & \qiff \QNOT\left(P \QAND \bar{Q}\right)
\end{align*}
Then express membership in $\left(\bar{A}-\bar{C}\right)\union\left(B\intersect C\right)\union\left[\left(\bar{A}\union B\right)\intersect\bar{C}\right]$ in terms of $P$, $Q$, and $R$:
\begin{align*}
\lefteqn{x \in \left\{\left(\bar{A}-\bar{C}\right)\union\left(B\intersect C\right)\union\left[\left(\bar{A}\union B\right)\intersect\bar{C}\right]\right\}}\\
 & \qiff x \in \left(\bar{A}-\bar{C}\right) \QOR x \in \left(B\intersect C\right) \QOR x \in \left[\left(\bar{A}\union B\right)\intersect\bar{C}\right]\\
 & \qiff x \in \left(\bar{A}\intersect \bar{\bar{C}}\right) \QOR x \in \left(B\intersect C\right) \QOR \left[x \in \left(\bar{A}\union B\right)\QAND x\in \bar{C}\right]\\
 & \qiff x \in \left(\bar{A}\intersect C\right) \QOR x \in \left(B\intersect C\right) \QOR \left[x \in \left(\bar{A}\union B\right)\QAND x\in \bar{C}\right]\\
 & \qiff \left(x \in\bar{A}\QAND x\in C\right) \QOR \left(x \in B\QAND x\in C\right) \QOR \left[\left(x \in \bar{A} \QOR x \in B\right)\QAND x\in \bar{C}\right]\\
 & \qiff \left(\bar{x \in A}\QAND x\in C\right) \QOR \left(x \in B\QAND x\in C\right) \QOR \left[\left(\bar{x \in A} \QOR x \in B\right)\QAND \bar{x\in C}\right]\\
 & \qiff \left(\bar{P}\QAND R\right) \QOR \left(Q\QAND R\right) \QOR \left[\left(\bar{P} \QOR Q\right)\QAND \bar{R}\right]
\end{align*}
So the given identity holds iff, for all $x$:
\[ \QNOT\left(P \QAND \bar{Q}\right) \qiff \left(\bar{P}\QAND R\right) \QOR \left(Q\QAND R\right) \QOR \left[\left(\bar{P} \QOR Q\right)\QAND \bar{R}\right] \]

\end{solution}

\end{problem}

%%%%%%%%%%%%%%%%%%%%%%%%%%%%%%%%%%%%%%%%%%%%%%%%%%%%%%%%%%%%%%%%%%%%%
% Problem ends here
%%%%%%%%%%%%%%%%%%%%%%%%%%%%%%%%%%%%%%%%%%%%%%%%%%%%%%%%%%%%%%%%%%%%%

\endinput
