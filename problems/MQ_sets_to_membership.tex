\documentclass[problem]{mcs}

\begin{pcomments}
\pcomment{MQ_sets_to_mapping}
\pcomment{from cp3m CP_proving_basic_set_id, 2/27/11}
\end{pcomments}

\pkeywords{
}

%%%%%%%%%%%%%%%%%%%%%%%%%%%%%%%%%%%%%%%%%%%%%%%%%%%%%%%%%%%%%%%%%%%%%
% Problem starts here
%%%%%%%%%%%%%%%%%%%%%%%%%%%%%%%%%%%%%%%%%%%%%%%%%%%%%%%%%%%%%%%%%%%%%



\begin{problem}

Prove that\footnote{The \term{set difference}, $A-B$, of sets $A$ and $B$ is
\[
A-B \eqdef \set{a \in A \suchthat a \notin B}.
\]}
\[
A = (A-B) \union (A \intersect B)
\]
for all sets, $A, B$, by using a chain of iff's to show that
\[
x \in A \QIFF x \in (A-B) \union (A \intersect B)
\]
for all elements, $x$. \\

\emph{Hint: reframe this problem in terms of set membership}

\begin{solution}
Two sets are equal iff they have the same elements, that is, $x$ is in
one set iff $x$ is in the other set, for any $x$.  We'll now prove this for $A$
and $(A-B) \union (A \intersect B)$.

\begin{align*}
\lefteqn{x \in (A-B) \union (A \intersect B)}\\
 & \qiff x \in (A-B) \QOR x \in (A
 \intersect B) & \text{(by def of $\union$)}\\
 & \qiff (x \in A \QAND \bar{x \in B})\\
 & \qquad\qquad \QOR (x \in A \QAND  x \in B)
   & \text{(by def of $\intersect$ and $-$)}\\
	 & \qiff (P \QAND \bar{Q}) \QOR (P \QAND Q) & \text{(where $P \eqdef [x \in A]$ and $Q \eqdef [x \in B]$)}\\
	 & \qiff P & \text{(by part~\eqref{verprop})}\\
	 & \qiff x \in A & \text{(by def of $P$)}.
	 \end{align*}

	 \end{solution}

\end{problem}

%%%%%%%%%%%%%%%%%%%%%%%%%%%%%%%%%%%%%%%%%%%%%%%%%%%%%%%%%%%%%%%%%%%%%
% Problem ends here
%%%%%%%%%%%%%%%%%%%%%%%%%%%%%%%%%%%%%%%%%%%%%%%%%%%%%%%%%%%%%%%%%%%%%

\endinput
