\documentclass[problem]{mcs}

\begin{pcomments}
\pcomment{MQ_sets_to_membership}
\pcomment{Created by Joliat and Kazerani; variant of CP_proving_basic_set_id, 2/27/11}
\pcomment{revised by ARM, 3/1/11}
\end{pcomments}

\pkeywords{
  logic
  set_theory
  identity
  propositional
  chain_of_iff
  difference
}

%%%%%%%%%%%%%%%%%%%%%%%%%%%%%%%%%%%%%%%%%%%%%%%%%%%%%%%%%%%%%%%%%%%%%
% Problem starts here
%%%%%%%%%%%%%%%%%%%%%%%%%%%%%%%%%%%%%%%%%%%%%%%%%%%%%%%%%%%%%%%%%%%%%

%\newcommand{\paren}[1]{\paren{#1}}
%\newcommand{\brack}[1]{\paren{#1}}

\begin{problem}

Set equalities such as the one below can be proved with a chain of
\textit{iff}s starting with ``$x \in $ left-hand-set'' and ending with ``$x
\in$ right-hand-set,'' as done in class and the text.  A key step in
such a proof involves invoking a propositional equivalence.  State a
propositional equivalence that would do the job for this set equality:

\[
\bar{A-B} =
\paren{\bar{A} - \bar{C}} \union
\paren{B \intersect C}    \union
\paren{\paren{\bar{A} \union B} \intersect \bar{C}}
\]
Do not simplify or prove the propositional equivalence you obtain.\\\\

%%%%%%%%%%%%%%

For example, to prove $A \union (B \intersect A) = A$, we have the ``iff'' chain''
\begin{align*}
x \in A \union (B \intersect A)
& \qiff x \in A \QOR (x \in (B \intersect A))\\
& \qiff x \in A \QOR (x \in B \QAND x \in A)\\
& \qiff x \in A & \text{(since $P \QOR (Q \QAND P)$ equiv to $P$)}
\end{align*}
%%%%%%%%%%%%%%%%%%%%%%

\iffalse

\textit{\textbf{Example.}}  Proving the set equality
\[ \bar{A\union B} = \bar{A}\intersect \bar{B}\]
would involve invoking the propositional equivalence
\[ \QNOT\paren{P \QOR Q} \QIFF \paren{\bar{P} \QAND \bar{Q}} \]


One approach involves recasting this set equality as a propositional
formula that must be true for all possible values of a test element,
$x$.  Write this formula, but do not simplify it.\\\\ Your formula
must contain only propositional variables and their negations,
parentheses, and operators such as $\QNOT$, $\QOR$, $\QAND$, and
$\QIFF$.\\\\ \textit{Hint:} You may find it useful to define three
propositions describing the membership of $x$ in each of the sets $A$,
$B$, and $C$:
\begin{eqnarray*}
P & \eqdef & x\in A\\
Q & \eqdef & x\in B\\
R & \eqdef & x\in C\\
\end{eqnarray*}
\fi

\begin{solution}

The stated set equality holds iff membership in $\bar{A-B}$ implies and is
implied by membership in
$\paren{\bar{A}-\bar{C}}\union\paren{B\intersect C}\union\paren{\paren{\bar{A}\union B}\intersect\bar{C}}$.  That is, the set equality
holds iff, for all $x$,
\[ x \in \bar{A-B} \QIFF x \in \paren{\bar{A}-\bar{C}}\union\paren{B\intersect C}\union\paren{\paren{\bar{A}\union B}\intersect\bar{C}}\]
Define three propositions describing the membership of $x$ in each of
the sets $A$, $B$, and $C$:
\begin{eqnarray*}
P & \eqdef & x\in A\\
Q & \eqdef & x\in B\\
R & \eqdef & x\in C\\
\end{eqnarray*}
Now, express membership in $\bar{A-B}$ in terms of $P$, $Q$, and $R$:
\begin{align*}
\lefteqn{x \in \bar{A-B}}\\
 & \qiff \bar{x \in \paren{A\intersect\bar{B}}}\\
 & \qiff \bar{x \in A \QAND x \in \bar{B}}\\
 & \qiff \bar{x \in A \QAND \bar{x \in B}}\\
 & \qiff \QNOT\paren{P \QAND \bar{Q}}
\end{align*}
Then express membership in $\paren{\bar{A}-\bar{C}}\union\paren{B\intersect C}\union\paren{\paren{\bar{A}\union B}\intersect\bar{C}}$ in terms of $P$, $Q$, and $R$:
\begin{align*}
\lefteqn{x \in \paren{\bar{A}-\bar{C}}\union\paren{B\intersect C}\union\paren{\paren{\bar{A}\union B}\intersect\bar{C}}}\\
 & \qiff x \in \paren{\bar{A}-\bar{C}} \QOR x \in \paren{B\intersect C} \QOR x \in \paren{\paren{\bar{A}\union B}\intersect\bar{C}}\\
 & \qiff x \in \paren{\bar{A}\intersect \bar{\bar{C}}} \QOR x \in \paren{B\intersect C} \QOR \paren{x \in \paren{\bar{A}\union B}\QAND x\in \bar{C}}\\
 & \qiff x \in \paren{\bar{A}\intersect C} \QOR x \in \paren{B\intersect C} \QOR \paren{x \in \paren{\bar{A}\union B}\QAND x\in \bar{C}}\\
 & \qiff \paren{x \in\bar{A}\QAND x\in C} \QOR \paren{x \in B\QAND x\in C} \QOR \paren{\paren{x \in \bar{A} \QOR x \in B}\QAND x\in \bar{C}}\\
 & \qiff \paren{\bar{x \in A}\QAND x\in C} \QOR \paren{x \in B\QAND x\in C} \QOR \paren{\paren{\bar{x \in A} \QOR x \in B}\QAND \bar{x\in C}}\\
 & \qiff \paren{\bar{P}\QAND R} \QOR \paren{Q\QAND R} \QOR \paren{\paren{\bar{P} \QOR Q}\QAND \bar{R}}
\end{align*}
So the stated set equality holds if and only if the following propositional equivalence is valid:
\[ \QNOT\paren{P \QAND \bar{Q}} \QIFF \paren{\paren{\bar{P}\QAND R} \QOR \paren{Q\QAND R} \QOR \paren{\paren{\bar{P} \QOR Q}\QAND \bar{R}}} \]

\end{solution}

\end{problem}

%%%%%%%%%%%%%%%%%%%%%%%%%%%%%%%%%%%%%%%%%%%%%%%%%%%%%%%%%%%%%%%%%%%%%
% Problem ends here
%%%%%%%%%%%%%%%%%%%%%%%%%%%%%%%%%%%%%%%%%%%%%%%%%%%%%%%%%%%%%%%%%%%%%

\endinput
