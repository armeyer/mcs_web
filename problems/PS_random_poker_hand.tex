\documentclass[problem]{mcs}

\begin{pcomments}
  \pcomment{PS_random_poker_hand}
  \pcomment{from: F06.ps11 (ported by Rich)}
\end{pcomments}

\pkeywords{
  probability
  playing cards
}

%%%%%%%%%%%%%%%%%%%%%%%%%%%%%%%%%%%%%%%%%%%%%%%%%%%%%%%%%%%%%%%%%%%%%
% Problem starts here
%%%%%%%%%%%%%%%%%%%%%%%%%%%%%%%%%%%%%%%%%%%%%%%%%%%%%%%%%%%%%%%%%%%%%

\newcommand{\cE}{\mathcal{E}}

\begin{problem}

We're interested in the probability that a randomly chosen poker hand (5
cards from a standard 52-card deck) contains cards from at most two suits.

%We use a few steps to attack the problem.

\bparts

\ppart What is an appropriate sample space to use for this problem?  What
are the outcomes in the event, $\cE$, we are interested in?  What are the
probabilities of the individual outcomes in this sample space?

\begin{solution}
  The natural sample space to use consists of the $\binom{52}{5}$
  possible poker hands.  The sample space is \emph{uniform}: Each hand is
  equally likely and comes up with probability $1 / \binom{52}{5}$.

  We define $\cE$ to be the subset of outcomes in which
  the 5 cards on the outcome come from at most two suits.
\end{solution}

\ppart What is $\prob{\cE}$?

\begin{solution}
  Since the sample space is uniform,
  \[
  \prob{\cE} = \frac{\card{\cE}}{\dbinom{52}{5}}.
  \]
  So we just need to determine the number of outcomes in $\cE$.  For this,
  we resort to our usual counting techniques.  Doing it by cases works well.
  There are three cases: 5 cards of one suit, 4 cards of one suit and 1 of
  another suit, 3 cards of one suit and 2 of another suit.

  For 5 of one suit, there are 4 ways to choose the suit and then
  $\binom{13}{5}$ ways to choose 5 cards of that suit.

  For 4 of one suit and 1 of another, there are 4 ways to choose the suit of
  the 4 and $\binom{13}{4}$ ways to choose 4 cards of that suit, and there
  are 3 remaining suits to choose for the 1, and 13 choices for the 1 card
  of that suit.

  Finally, for 3 of one suit and 2 of another, there are 4 ways to choose
  the suit of the 3 and $\binom{13}{3}$ ways to choose 3 cards of that suit,
  and there are 3 remaining suits to choose for the 2 cards, and
  $\binom{13}{2}$ choices for the 2 cards of that suit.   So the total is
  \[
  4\cdot \binom{13}{5} + 4 \cdot \binom{13}{4} \cdot 3 \cdot 13 + 4 \cdot
  \binom{13}{3} \cdot 3 \cdot \binom{13}{2},
  \]
  and the probability of at most two suits 
  is
  \[
  \frac{4\cdot \dbinom{13}{5} + 4 \cdot \dbinom{13}{4} \cdot 3 \cdot 13 + 4 \cdot
  \dbinom{13}{3} \cdot 3 \cdot \dbinom{13}{2}}{\dfrac{1}{\dbinom{52}{5}}}
   = 88/595 \approx 0.15.
  \]

  \iffalse
  There are $\binom{4}{2}$ ways to choose two suits and $\binom{26}{5}$ ways
  to choose five cards from these two suits.  But we need to be careful: some
  of those $\binom{26}{5}$ choices might all come from the same suit.  These
  hands will be triple-counted.  (For example, a hand containing \emph{only}
  spades is counted once as a spades-hearts hand, a second time as a
  spades-clubs hand, and a third time as a spades-diamonds hand.)  We can
  correct for this triple-counting by subtracting off $2$ times the number of
  hands that are triple-counted.  Since the number of hands with a single
  suit is $4 \cdot \binom{13}{5}$, the number of hands with at most two suits
  is
  \[
  \binom{4}{2} \cdot \binom{26}{5} - 2 \cdot 4 \cdot \binom{13}{5}.
  \]
  The probability that a random poker hand contains cards from at most
  two suits is:
  \begin{align*}
  \prob{\cE}
          & =  \frac{\dbinom{4}{2} \cdot \dbinom{26}{5} -
                          2 \cdot 4 \cdot \dbinom{13}{5}}
                  {\dbinom{52}{5}} \\
          & =  \frac{88}{595} \approx 0.15
  \end{align*}
  \fi

\end{solution}

\eparts

\end{problem}

%%%%%%%%%%%%%%%%%%%%%%%%%%%%%%%%%%%%%%%%%%%%%%%%%%%%%%%%%%%%%%%%%%%%%
% Problem ends here
%%%%%%%%%%%%%%%%%%%%%%%%%%%%%%%%%%%%%%%%%%%%%%%%%%%%%%%%%%%%%%%%%%%%%

\endinput
