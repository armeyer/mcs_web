\documentclass[problem]{mcs}

\begin{pcomments}
  \pcomment{PS_triangle_free_planar_graphs}
  \pcomment{formerly PS_simple_triangle_free_planar_graphs}
  \pcomment{subsumes PS_planar_triangle_free_graph_coloring}
  \pcomment{from: S09.ps6}
\end{pcomments}

\pkeywords{
  planar_graphs
  handshaking
  graph_coloring
  strong_induction
  induction_on_graphs
  colorable
}

%%%%%%%%%%%%%%%%%%%%%%%%%%%%%%%%%%%%%%%%%%%%%%%%%%%%%%%%%%%%%%%%%%%%%
% Problem starts here
%%%%%%%%%%%%%%%%%%%%%%%%%%%%%%%%%%%%%%%%%%%%%%%%%%%%%%%%%%%%%%%%%%%%%

\begin{problem} A simple graph is \emph{triangle-free} when it has no
cycle of length three.

\bparts

\ppart\label{e2v4} Prove for any connected triangle-free planar graph
with $v>2$ vertices and $e$ edges,
\begin{equation}\label{eq:e2v-4}
e \leq 2v - 4.
\end{equation}
\hint Similar to the proof that $e \leq 3v-6$.
Use Problem~\ref{planar-structural}.

\begin{solution}
  The proof that $e \leq 2v - 4$ for any connected triangle-free
  planar graph $G$ with more than two vertices is identical to the proof
  of the same inequality for bipartite graph planar graphs:

\begin{proof}
  By Problem~\ref{planar-structural}.\ref{structind-face-length},
%\bref{CP_planar_structural_induction}.\bref{structind-face-length},
  every face is of length at least 3.  But in a triangle-free graph
  there are no faces of size 3, so all must be of length at least 4.

  Each edge is occurs exactly twice in the faces, so
  \begin{equation}\label{4f}
    2e = \sum_{f \in\text{ faces}} \text{length}(f) \geq
    \sum_{f \in\text{ faces}} 4 = 4f.
  \end{equation}
  By Euler's formula, $f = e-v+2$, so
  substituting for $f$ in~\eqref{4f}, yields
\[
2e \geq 4(e-v+2),
\]
which simplifies to~\eqref{eq:e2v-4}.

\end{proof}
\end{solution}

\ppart\label{triangle-free} Show that any connected triangle-free planar
  graph has at least one vertex of degree three or less.

\begin{solution}
If $v \leq 4$, \emph{all} vertices have degree at most three,
  so the claim is immediate for $v\leq 4$.

  Also, by the Handshaking Lemma, the sum of degrees is $2e$ so the average
  degree is $2e/v$.  By part~\eqref{e2v4}, $2e/v \leq (4v-8)/v < 4$ for
  $v>2$.  But the average degree can be less than 4 only if at least one
  vertex has degree less than 4.

  It follows that for all $v > 0$, there is a vertex of degree three or
  less.
\end{solution}

\ppart Prove by induction on the number of vertices that any connected
  triangle-free planar graph is 4-colorable.

  \hint use part \eqref{triangle-free}.

\begin{solution}
\mbox{}

  \begin{proof} By strong induction on the number of vertices with the
  induction hypothesis that if a graph is connected, planar and
  triangle-free then it is 4-colorable.

  \inductioncase{base case:} A planar graph with a single vertex is trivially
  connected, triangle-free and 1-colorable.

  \inductioncase{inductive step:} Any connected triangle-free planar graph
  $G$ with 2 or more vertices has a vertex of degree 3 or less.
  Removing this vertex and any incident edges results in a graph
  $H$ whose connected components are subgraphs of a planar graph
  and therefore planar. They are also triangle-free since removing
  vertices/edges from a graph with no triangles cannot create
  triangles. Since the components have strictly fewer vertices than $G$,
  the induction hypothesis implies each connected component is
  4-colorable and thus $H$ is 4-colorable.

  A 4-coloring of $G$ is then given by a 4-coloring of $H$ where the
  removed vertex is colored with a color not used for the (at most 3)
  adjacent vertices. \end{proof}
\end{solution}

\eparts

\end{problem}

%%%%%%%%%%%%%%%%%%%%%%%%%%%%%%%%%%%%%%%%%%%%%%%%%%%%%%%%%%%%%%%%%%%%%
% Problem ends here
%%%%%%%%%%%%%%%%%%%%%%%%%%%%%%%%%%%%%%%%%%%%%%%%%%%%%%%%%%%%%%%%%%%%%
\endinput
