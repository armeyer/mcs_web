


\pcomment{FP_tetris_wop}

As in the pset, a winning configuration in the game of Mini-Tetris is a complete tiling of a 2 x n board. This time only these five shapes can be used:
[shapes]
Let T(n) denote the number of different winning configurations on a 2 x n board. Rotationally symmetric boards count as different configurations, i.e.
[image showing rotationally symmetric boards]

(a)
T(0) = 1 because there is exactly one way to tile a 2 x 0 board, which is to use no tiles at all. Determine the values of T(1) and T(2). A couple example 2 x 2 boards are shown for you.

ANSWER:
T(1) = 2
T(2) = 7

(b) 

(c)
Use the Well Ordering Principle to prove that the number of winning configurations on a 2 x n Mini-Tetris board is:

T(n) = [(-1)^n + 3^(n+1)]/4 for n>=0

ANSWER:
Let P(n) be the predicate
P(n) ::= [T(n) = [(-1)^n + 3^(n+1)]/4]
and let C be the set of counterexamples to P:
C::={n>=1 such that NOT(P(n))}
Assume for the sake of contradiction the C is not empty. Then by the Well Ordering Principle, there is some minimum element m in C. But P(n) is true for n=0,1:

T(0) = 1 = [(-1)^0 + 3^(0+1)]/4
T(1) = 2 = [(-1)^1 + 3^(1+1)]/4

This means that m must be greater than 1. So both m-1 and m-2 are >= 0, and since m is the smallest counterexample >= 0, neither m-1 nor m-2 will be counterexamples. Thus, we have

T(m) = 2*T(m-1) + 3*T(m-2)
= 2*[(-1)^(m-1) + 3^m]/4 + 3*[(-1)^(m-2) + 3^(m-1)]/4
= [2*(-1)^(m-1) + 3*(-1)^(m-2) + 2*3^m + 3*3^(m-1)]/4
= [2*(-1)^(m-1) + 3*(-1)(-1)^(m-1) + 2*3^m + 3^m]/4
= [(2-3)(-1)^(m-1) + 3*3^m]/4
= [(-1)^(m) + 3^(m+1)]/4

This shows that m satisfies (*), so is not a counterexample, contradicting the definition of m. So C must be empty, which proves that (*) holds for all n>=0, as desired.


