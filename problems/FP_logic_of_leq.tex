\documentclass[problem]{mcs}

\begin{pcomments}
  \pcomment{FP_logic_of_leq}
  \pcomment{INCOMPLETE by ARM F09}
\end{pcomments}

\pkeywords{
 predicate_calculus
 domain_of_discourse
}

%%%%%%%%%%%%%%%%%%%%%%%%%%%%%%%%%%%%%%%%%%%%%%%%%%%%%%%%%%%%%%%%%%%%%
% Problem starts here
%%%%%%%%%%%%%%%%%%%%%%%%%%%%%%%%%%%%%%%%%%%%%%%%%%%%%%%%%%%%%%%%%%%%%

\begin{problem}

  Translate each of the following statements about the nonnegative
  integers, $\naturals$, into predicate formulas in which the \emph{only}
  predicate that may appear is $\leq$.  The domain of discourse is
  $\naturals$.

  For example, equality is definable by the formula
  \[
  \text{equal}(x,y) \eqdef [x \leq y] \QAND [y \leq x].
  \]
  Now it is OK to use equality in formulas, since it can be replaced by the
  formula $\text{equal}(x,y)$ in which $\leq$ is the only predicate.
  
\end{itemize}

\bparts

\ppart[1] $\text{zero}(x) \equiv [x = 0]$.

\examspace[0.3in]
\begin{solution}
\[
\forall y.\ x \leq y
\]
\end{solution}

\ppart[3] $\text{sucessor}(x,y) \equiv [x = y+1]$

\examspace[0.5in]
\begin{solution}
  \[ y \leq x \QAND
    \forall z \neq x.\, x \leq z \QIMPLIES y \leq z.
  \]
\end{solution}

\ppart $x = 2$

\examspace[0.5in]

\iffalse

\ppart
How many possible lineups are there that satisfy all three of these rules?

\examspace[1in]

\begin{solution}
   We may consider the number of ways to place $A, B, C, D$ by cases:
   
   Suppose that $B$ or $C$ is last. There are still $2$ ways to place $B$ and $C$.
   A may then take any of the remaining 7 spots. So the total is $2 \cdot 7$.
   
   Suppose that neither $B$ or $C$ is last.  There are $6$ (they may be anywhere
   between 3rd to 9th) places to place them together and $2$ ways to order them.
   $A$ may be placed in any of the remaining $6$ places (discounting the 
   last place and the places B and C occupy).  The total is therefore 
   $6 \cdot 2 \cdot 6$.
   
   We know that, in both cases, there are $6!$ ways to place the remaining 6 
   people into the remaining spots, so the total number of possible lineups is:
   \[
     6! \cdot (6 \cdot 2 \cdot 6 + 2 \cdot 7) =
      6! \cdot 86 =
      61920
   \]
\end{solution}


\ppart
How many possible lineups are there that satisfy at least one of these rules?

% seems like too much work for an inclusion-exclusion problem?

\examspace[3in]
\begin{solution}
   
   \# lineups satisfying rule I: 
   There are 9 places to place A, and $9!$ ways to place the remaining people.
   
   \[ 9 \cdot 9! \]
   
   \# lineups satisfying rule II:
   We can consider $B$ and $C$ to be one person, so there are $9!$ ways to place them, and
   2 ways to order $B$ and $C$.
   
   \[ 2 \cdot 9! \]
   
   \# lineups satisfying rule III:
   There are $9!$ ways to order the remaining 9 people:
   
   \[ 9! \]
   
   \# lineups satisfying rule I and II:
   If B or C is last, there are $8!$ ways to order the remaining people.
   Otherwise, there are 8 places to place B and C; 7 places to place A; and
   $7!$ ways to order the remaining numbers.
   
   \[ 2 \cdot 8! + 2 \cdot 8 \cdot 7 \cdot 7! \]
   
   \# lineups satisfying rule I and III:
   There are 8 places to put $A$ and $8!$ ways to order the remaining people.
   
   \[ 8 \cdot 8! \]
   
   \# lineups satisfying rule II and III:
   There are 7 places to place $B$ and $C$ together, and $7!$ ways to order
   the remaining people.
   
   \[ 2 \cdot 7 \cdot 7! \]
   
   \# lineups satisfying rule I, II and III:
   From the previous part.
   
   \[ 6! \cdot 86 \]
   
   So the total is, using inclusion-exclusion:
   \[ 
   9 \cdot 9! + 2 \cdot 9! + 9! - (2 \cdot 8! + 2 \cdot 8 \cdot 7 \cdot 7! + 8 \cdot 8! + 2 \cdot 7 \cdot 7!) + 6! \cdot 86 =
   4692 \cdot 6! =
   3378240
   \]
\end{solution}
\fi

\eparts
\end{problem}

\endinput
