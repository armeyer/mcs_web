\documentclass[problem]{mcs}

\begin{pcomments}
  \pcomment{FP_chains_scheduling}
  \pcomment{from S02.final, F03.final }
\end{pcomments}

\pkeywords{
 chain
 antichain
 topological
 schedule
}

%%%%%%%%%%%%%%%%%%%%%%%%%%%%%%%%%%%%%%%%%%%%%%%%%%%%%%%%%%%%%%%%%%%%%
% Problem starts here
%%%%%%%%%%%%%%%%%%%%%%%%%%%%%%%%%%%%%%%%%%%%%%%%%%%%%%%%%%%%%%%%%%%%%

\begin{problem}

Sauron wants to conquer Middle Earth.  This project involves $n$
tasks.  Some of these tasks must be completed before others are begun.
For example, the task of {\em locating} the One Ring must precede the
task of {\em seizing} the ring.  Each task can be completed by a
horrible creature called a {\em Ringwraith} in exactly one week.  A
Ringwraith can complete multiple tasks, but can only work on one at a
time.  Sauron uses a DAG whose vertices are tasks and whose edges
indicate the prerequisite tasks of each task.

\bparts
\iffalse

\ppart  What should vertices represent?  When should
there be a directed edge between two vertices?

\examspace[1.0in]

\begin{solution}
Each vertex corresponds to a task.  There is a
directed edge from vertex $a$ to vertex $b$ if task $a$ must precede
task $b$.
\end{solution}
\fi

\ppart Sauron is trying to describe various features of his
scheduling problem using standard terminology.  Next to each feature
below, write the number of the corresponding term.

\begin{center}
\textbf{Standard Terminology}\\
\begin{tabular}{rlrl}
1. & Transitive closure &
2. & Topological sort \\
3. & Chain &
4. & Antichain \\
5. & Size of the largest antichain &
6. & Size of the smallest antichain \\
7. & Length of the longest chain &
8. & Length of the shortest chain
\end{tabular}
\end{center}

%\examspace[0.2in]

\begin{tabular}{rll}
1. & \parbox[t]{5in}{
A set of tasks that can be worked on simultaneously.
} & \brule{0.5in} \\
\\
2. & \parbox[t]{5in}{
A possible order in which all the tasks could be completed, if only one
Ringwraith were available.
} & \brule{0.5in} \\
\\
3. & \parbox[t]{5in}{
The minimum number of weeks required to complete all tasks, if an
unlimited number of Ringwraiths were available.
} & \brule{0.5in}
\end{tabular}

\begin{solution}
4. Antichain; 2. Topological Sort; 7.  Length of longest chain
\end{solution}

\ppart Sauron discovers that conquering Middle Earth would take $t$
weeks, if he had an unlimited supply of Ringwraiths, due to the
dependencies between tasks.

If Sauron is lucky, he will be able to get by with a smallish crew of
Ringwraiths and still be able to conquer the world in $t$ weeks.  Write a
simple formula involving $n$ and $t$ for the smallest number of
Ringwraiths that could possibly be able to complete all $n$ tasks in $t$
weeks.

\begin{center}
\exambox{1.0in}{0.3in}{0.3in}
\end{center}
\examspace[0.5in]


\begin{solution}
At least $\ceil{n/t}$ Ringwraiths are
necessary.  Each of the $n$ tasks must be completed in one of the $t$
weeks.  Thus, by the pigeonhole principle, at least $\ceil{n/t}$ must
be completed in some week.  But that would require at least
$\ceil{n/t}$ Ringwraiths.

This bound is the best possible.  If there is a chain of $t$ tasks and
the remaining tasks have no dependencies, then $\ceil{n/t}$
Ringwraiths are sufficient.
\end{solution}

\ppart On the other hand, if Sauron is unlucky, he may need a large
crew of Ringwraiths in order to conquer Middle Earth in $t$ weeks.  Write
a simple formula involving $n$ and $t$ for the largest number of
Ringwraiths that Suaron would need in order to be sure of completing all
$n$ tasks in $t$ weeks -- no matter how unlucky he was.

\begin{center}
\exambox{1.0in}{0.3in}{0.3in}
\end{center}
\examspace[0.5in]

\begin{solution}
Up to $n - t + 1$ Ringwraiths may be necessary.  This is the
case is, for example, if there is an antichain of $n - t + 1$ tasks
that can not be started until a chain of $t - 1$ tasks is completed.

Again, this bound is the best possible.  If more than $n - t + 1$
tasks are completed in one week, then there must be a week when no
tasks are completed.  If this is the case, then all the tasks can be
completed in fewer than $t$ weeks, contradicting the assumption.
\end{solution}

\eparts

\end{problem}

%%%%%%%%%%%%%%%%%%%%%%%%%%%%%%%%%%%%%%%%%%%%%%%%%%%%%%%%%%%%%%%%%%%%%
% Problem ends here
%%%%%%%%%%%%%%%%%%%%%%%%%%%%%%%%%%%%%%%%%%%%%%%%%%%%%%%%%%%%%%%%%%%%%

\endinput
 
