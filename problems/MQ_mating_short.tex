\documentclass[problem]{mcs}

\begin{pcomments}
  \pcomment{MQ_mating}
\end{pcomments}

\pkeywords{
}

%%%%%%%%%%%%%%%%%%%%%%%%%%%%%%%%%%%%%%%%%%%%%%%%%%%%%%%%%%%%%%%%%%%%%
% Problem starts here
%%%%%%%%%%%%%%%%%%%%%%%%%%%%%%%%%%%%%%%%%%%%%%%%%%%%%%%%%%%%%%%%%%%%%

\begin{problem}

In the Mating Ritual, suppose Tiger is one of the boys and Elin is one of the girls.
Which of the following are preserved invariants in general?
\begin{enumerate}
\item Tiger is Elin's only suitor.
\item Tiger's optimal wife is the most preferred girl on his list.
\item Elin has crossed off everyone above and including Tiger on her list.
\end{enumerate}
\examspace[2in]
\begin{solution}
The statements that are preserved invariants in general appear in boldface below:
\begin{enumerate}
\item Tiger is Elin's only suitor.
\item \textbf{Tiger's optimal wife is the most preferred girl on his list.}
\item \textbf{Elin has crossed off everyone above and including Tiger on her list.} (Note that this is a preserved invariant because it cannot ever be true; if this were ever true at any step, Tiger and Elin would end up as a rogue couple.)
\end{enumerate}
\end{solution}

\end{problem}

\endinput
