\documentclass[problem]{mcs}

\begin{pcomments}
  \pcomment{CP_perfect_numbers}
  \pcomment{from: S10.cp8m, S09.cp8m, S06.cp6w}
\end{pcomments}

\pkeywords{
  geometric_series
  geometric_sum
  divides
  number_theory
  perfect_number
  prime
  power_of_2
  Euclid
  Euler
}

%%%%%%%%%%%%%%%%%%%%%%%%%%%%%%%%%%%%%%%%%%%%%%%%%%%%%%%%%%%%%%%%%%%%%
% Problem starts here
%%%%%%%%%%%%%%%%%%%%%%%%%%%%%%%%%%%%%%%%%%%%%%%%%%%%%%%%%%%%%%%%%%%%%

% S09, S06

\begin{problem}
A number is \emph{perfect}\index{perfect number} if it is equal 
to the sum of its positive
divisors, other than itself.  For example, 6 is perfect, because $6 =
1 + 2 + 3$.  Similarly, 28 is perfect, because $28 = 1 + 2 + 4 + 7 +
14$.  Explain why $2^{k-1} (2^k - 1)$ is perfect when $2^k - 1$ is
prime.\footnote{Euclid proved this 2300 years ago.  About 250
  years ago, Euler proved the converse: \emph{every} even
  perfect number is of this form (for a simple proof see
  \href{http://primes.utm.edu/notes/proofs/EvenPerfect.html}
       {\texttt{http://primes.utm.edu/notes/proofs/EvenPerfect.html}}).
       As is typical in number theory, apparently simple results lie
       at the brink of the unknown.  For example, it is not known if
       there are an infinite number of even perfect numbers or any odd
       perfect numbers at all.}

\begin{solution}
If $2^k - 1$ is prime, then the only divisors of $2^{k-1} (2^k - 1)$
are:
\begin{equation}\label{pow2hatk-1}
1,\quad 2,\quad 4,\quad \ldots,\quad 2^{k-1},
\end{equation}
and
\begin{equation}\label{pow2pow2kless1}
1 \cdot (2^k - 1),\quad 2 \cdot (2^k - 1),\quad 4 \cdot (2^k - 1),\quad
   \dots, \quad 2^{k-2} \cdot (2^k - 1).
\end{equation}
The sequence~\eqref{pow2hatk-1} sums to $2^k-1$ (using the formula for
a geometric series),\footnote{It's fun to notice the ``computer
  science'' proof that~\eqref{pow2hatk-1} sums to $2^k-1$.  The
  binary representation of $2^j$ is a \texttt{1} followed by
  $j$ \texttt{0}'s, so expressed in binary, the sum is
\[
\mathtt{1} + \mathtt{10} + \mathtt{100} +\cdots + \mathtt{1}\underbrace{\mathtt{0}\cdots \mathtt{0}}_{k-1} = \underbrace{\mathtt{1} \cdots \mathtt{1}}_k,
\]
and the right hand expression is what you get by subtracting
\texttt{1} from the binary representation of $2^k$.} and likewise the
sequence~\eqref{pow2pow2kless1} sums to $(2^{k-1} - 1) \cdot (2^k -
1)$.  Adding these two sums gives $2^{k-1} (2^k - 1)$, so the number
is perfect.
\end{solution}

\end{problem}

%%%%%%%%%%%%%%%%%%%%%%%%%%%%%%%%%%%%%%%%%%%%%%%%%%%%%%%%%%%%%%%%%%%%%
% Problem ends here
%%%%%%%%%%%%%%%%%%%%%%%%%%%%%%%%%%%%%%%%%%%%%%%%%%%%%%%%%%%%%%%%%%%%%

\endinput
