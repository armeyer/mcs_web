\documentclass[problem]{mcs}

\begin{pcomments}
  \pcomment{Source: f01-ps9-3 and s01-ps7-5}
\end{pcomments}

\pkeywords{
counting
binomial
combinatorics
}

%%%%%%%%%%%%%%%%%%%%%%%%%%%%%%%%%%%%%%%%%%%%%%%%%%%%%%%%%%%%%%%%%%%%%
% Problem starts here
%%%%%%%%%%%%%%%%%%%%%%%%%%%%%%%%%%%%%%%%%%%%%%%%%%%%%%%%%%%%%%%%%%%%%

\begin{problem}
\bparts
\ppart How many ways can $2n$ people be divided into $n$ pairs?

\begin{solution}
\[
\frac{(2n)!}{n! 2^n}.
\]

Consider all the ways of ordering the $2n$ people in a line. There are
$(2n)!$ ways of doing this. Now group them into pairs based on the
linear order, i.e. group the first and second persons, group the third
and fourth persons,..., group the $(2n-1)$th and $2n$th persons. Thus
each linear order gives us a unique grouping into pairs. But observe
that two different linear orders could give rise to the same grouping
into pairs. So let us now count the number of linear orders that give
rise to the same grouping into pairs. Given a grouping into pairs, to
generate a linear order we need to order each pair (there are $2^n$
ways to do this) and then order the $n$ pairs (there are $n!$ ways to
do this). Hence $n! 2^n$ linear orders give the same grouping into
pairs. Thus the total number of ways $2n$ people can be divided into
$n$ pairs is $ (2n)!/ n! 2^n$.

\iffalse
First we pick $n$ leaders from the $2n$ people. 
There are $\binom{2n}{n}$ ways to do this. 
Then we pair the remaining $n$ people with the $n$ leaders. That gives
us $n!$ different ways for each choice of leaders. However,
$\binom{2n}{n}n!$ will be overcounting, since for any fixed pairing,
we count each pair twice depending on who is the leader.
 So for any valid pairing, we counted it $2^n$ times.
Therefore, the answer should be
 $\displaystyle \binom{2n}{n} n! /2^n$.

\fi
\end{solution}

\ppart How many ways can you choose $n$ out of $2n$ objects, given that
$n$ of the $2n $ objects are identical?

\begin{solution}
The answer is $2^n$, since you can pick any subset from the $n$
nonidentical objects, and make up the rest with the identical ones. And the
number of subsets of $n$ different objects is $2^n$. 
\end{solution}

\ppart
Six women and nine men are on the faculty of a school's
EECS department.  The individuals are distinguishable.
How many ways are there to select a committee of 5
members if at least 1 woman must be on the committee?

\begin{solution}
        
\textbf{First method: (by brute force)}We need to count all possible combinations of people such that there
is at least one woman in every combination, but we must remember not
to count any combinations multiple times.  

We can have committees with 

1 woman, 4 men: $ { 6 \choose 1} {9 \choose 4}$\\
2 women, 3 men: $ { 6 \choose 2} {9 \choose 3}$\\
3 women, 2 men: $ { 6 \choose 3} {9 \choose 2}$\\
4 women, 1 man: $ { 6 \choose 4} {9 \choose 1}$\\
5 women, 0 men: $ { 6 \choose 5} {9 \choose 0}$\\

So there are 
$ { 6 \choose 1} {9 \choose 4} 
+ { 6 \choose 2} {9 \choose 3}
+ { 6 \choose 3} {9 \choose 2}
+ { 6 \choose 4} {9 \choose 1}
+ { 6 \choose 5} {9 \choose 0} = 2877$ different possibilities for
committees.

\medskip

\textbf{Second method:}
Another way to solve this problem is to say that there are ${15 \choose 5}$
different committees, and ${9 \choose 5}$ committees of just men.  So there
are ${15 \choose 5} - {9 \choose 5}= 2877$ different possibilities for committees. 


Note that $6 \cdot { 14 \choose 4} =6006$ is not a correct answer.  
The product rule
only applies when you are choosing two separate sets of items which are
independent of each other.  
In this case, a
woman is chosen and then the remainder of the committee is chosen, but since
there may be women in the last set of $4$ chosen, we will be double counting.
For example, if $A$ and $B$ are women, and $C, D, $ and $E$ are men, then
we will count $A$, \{$BCDE$\} as well as $B$, \{$ACDE$\}.  
\end{solution}

\eparts
\end{problem}

%%%%%%%%%%%%%%%%%%%%%%%%%%%%%%%%%%%%%%%%%%%%%%%%%%%%%%%%%%%%%%%%%%%%%
% Problem ends here
%%%%%%%%%%%%%%%%%%%%%%%%%%%%%%%%%%%%%%%%%%%%%%%%%%%%%%%%%%%%%%%%%%%%%
