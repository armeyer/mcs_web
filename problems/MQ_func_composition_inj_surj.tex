\documentclass[problem]{mcs}

\begin{pcomments}
\pcomment{MQ_func_composition_inj_surj}
\pcomment{solution overlaps TP_injR_surjinvR}
\pcomment{S13, miniq2}
\pcomment{ARM 3/21/13, revised 9/18/13}
\end{pcomments}

\pkeywords{
  relation
  composition
  inverse
  identity
  arrows
}

%%%%%%%%%%%%%%%%%%%%%%%%%%%%%%%%%%%%%%%%%%%%%%%%%%%%%%%%%%%%%%%%%%%%%
% Problem starts here
% %%%%%%%%%%%%%%%%%%%%%%%%%%%%%%%%%%%%%%%%%%%%%%%%%%%%%%%%%%%%%%%%%%%%

\begin{problem}
Prove that if relation $R:A \to B$ is a total injection, $[\ge
  1\ \text{out}], [\le 1\ \text{in}]$, then
\[
R^{-1} \compose R = \ident{A},
\]
where $\ident{A}$ is the identity function on $A$.

(A simple argument in terms of "arrows" will do the job.)


\begin{solution}
There is an $R$-arrow from any element of $A$ (by $[\ge
  1\ \text{out}]$) to a unique element of $B$ (by $[\le
  1\ \text{in}]$).  Composing with the inverse function corresponds to
following the $R$-arrows backwards.  Since each arrow's endpoint in
$B$ is unique, going backwards implies finishing where the arrow
began.  That is, starting at any $a \in A$ and following the forward
and backward arrows in the composition leads back to $a$.  So the
composition defines the identity function on $A$.
\end{solution}
\end{problem}
%%%%%%%%%%%%%%%%%%%%%%%%%%%%%%%%%%%%%%%%%%%%%%%%%%%%%%%%%%%%%%%%%%%%%
% Problem ends here
%%%%%%%%%%%%%%%%%%%%%%%%%%%%%%%%%%%%%%%%%%%%%%%%%%%%%%%%%%%%%%%%%%%%%

\endinput


