\documentclass[problem]{mcs}

\begin{pcomments}
  \pcomment{from: S09.ps1}
  \pcomment{from: F07.ps1}
  \pcomment{from: F05}
  \pcomment{commented out in S09}
\end{pcomments}

\pkeywords{
  set_theory
  false_proof
}

%%%%%%%%%%%%%%%%%%%%%%%%%%%%%%%%%%%%%%%%%%%%%%%%%%%%%%%%%%%%%%%%%%%%%
% Problem starts here
%%%%%%%%%%%%%%%%%%%%%%%%%%%%%%%%%%%%%%%%%%%%%%%%%%%%%%%%%%%%%%%%%%%%%

\begin{problem}

\bparts
\ppart Give an example where the following result fails:

\begin{falsethm*}
For sets $A$, $B$, $C$, and $D$, let
\begin{align*}
L \eqdef (A \union C) \times (B \union D),\\
R \eqdef (A \times B) \union (C \times D).
\end{align*}
Then $L=R$.
\end{falsethm*}

\solution{
If $A=D=\emptyset$ and $B$ and $C$ are both nonempty, then $L = C \times B
\neq \emptyset$, but $R = \emptyset$.
}

\ppart Identify the mistake in the following proof of the False Theorem.

\begin{bogusproof}
Since $L$ and $R$ are both sets of pairs, it's sufficient to prove that
$(x,y) \in L \iff (x,y) \in R$ for all $x,y$.

The proof will be a chain of iff implications:
\begin{center}
\begin{tabular}{ll}
    & $(x, y) \in L$  \\
iff & $x \in A \union C$ and $y \in B \union D$ \\
iff & either $x \in A$ or $x \in C$, and either $y \in B$ or $y \in D$ \\
iff & ($x \in A$ and $y \in B$) or else ($x \in C$ and $y \in D$)  \\
iff & $(x,y) \in A \times B$, or $(x,y) \in C \times D$ \\
iff & $(x,y) \in (A \times B) \union (C \times D) $ \\
iff & $(x,y)\in R$.
\end{tabular}
\end{center}

\end{bogusproof}

\solution{The mistake is in the third ``iff.''  If [$x \in A$ or $x \in
C$, and either $y \in B$ or $y \in D$], it does not necessarily follow
that $(x,y) \in (A \times B) \union (C \times D)$.  It might be that
$(x,y)$ is in $A \times D$ instead.  This happens, for example, if $A =
\set{1}, B = \set{2}, C = \set{3}, D = \set{4}$, and $(x,y) = (1, 4)$.}

\ppart Fix the proof to show that $R \subseteq L$.

\solution{
Replacing the third ``iff'' by ``which will be true when,'' yields a
correct proof that $(x,y) \in L$ will be true when $(x,y) \in R$, which
implies that $R \subseteq L$.
}

\eparts

\end{problem}

%%%%%%%%%%%%%%%%%%%%%%%%%%%%%%%%%%%%%%%%%%%%%%%%%%%%%%%%%%%%%%%%%%%%%
% Problem ends here
%%%%%%%%%%%%%%%%%%%%%%%%%%%%%%%%%%%%%%%%%%%%%%%%%%%%%%%%%%%%%%%%%%%%%
