\documentclass[problem]{mcs}

\begin{pcomments}
  \pcomment{MQ_pigeonhole_cards_flush}
\end{pcomments}

\pkeywords{
  counting
  pigeonhole principle
}

%%%%%%%%%%%%%%%%%%%%%%%%%%%%%%%%%%%%%%%%%%%%%%%%%%%%%%%%%%%%%%%%%%%%%
% Problem starts here
%%%%%%%%%%%%%%%%%%%%%%%%%%%%%%%%%%%%%%%%%%%%%%%%%%%%%%%%%%%%%%%%%%%%%

\begin{problem}
In a standard 52-card deck, what is the smallest $k$ such that
every size $k$ subset of the 52 cards contains a flush?
(flush is defined as 5 cards of the same suit).
Briefly explain your answer.

\begin{solution}
$17$.

There are 4 possible card suits, so by the Generalized Pigeonhole
Principle, we want the smallest integer $k$ such that
\[
\ceil{\frac{k}{4}} = 5,
\]
which is equivalent to
\[
\frac{k}{4} > 4.
\]
\end{solution}

\end{problem}

%%%%%%%%%%%%%%%%%%%%%%%%%%%%%%%%%%%%%%%%%%%%%%%%%%%%%%%%%%%%%%%%%%%%%
% Problem ends here
%%%%%%%%%%%%%%%%%%%%%%%%%%%%%%%%%%%%%%%%%%%%%%%%%%%%%%%%%%%%%%%%%%%%%

\endinput
