\documentclass[problem]{mcs}

\begin{pcomments}
  \pcomment{FP_353_factor}
  \pcomment{ARM 5/6/16}
\end{pcomments}

\pkeywords{
  number_theory
  Pulverizer
  modular_arithmetic
  inverses
  Fermat_theorem
  remainder
}

%%%%%%%%%%%%%%%%%%%%%%%%%%%%%%%%%%%%%%%%%%%%%%%%%%%%%%%%%%%%%%%%%%%%%
% Problem starts here
%%%%%%%%%%%%%%%%%%%%%%%%%%%%%%%%%%%%%%%%%%%%%%%%%%%%%%%%%%%%%%%%%%%%%

\begin{problem}
Show that 1412 does not have an inverse modulo 1059.

\begin{solution}
1412 and 1059 are not relatively prime, so neither has an inverse
modulo the other.  In particular, they are both divisible by 353:
\[
\gcd(1059,1412) = \gcd(1059,1412-1059) = \gcd(1059, 353) = 353.
\]
\end{solution}

\end{problem}


%%%%%%%%%%%%%%%%%%%%%%%%%%%%%%%%%%%%%%%%%%%%%%%%%%%%%%%%%%%%%%%%%%%%%
% Problem ends here
%%%%%%%%%%%%%%%%%%%%%%%%%%%%%%%%%%%%%%%%%%%%%%%%%%%%%%%%%%%%%%%%%%%%%

\endinput

