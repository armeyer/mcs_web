\documentclass[problem]{mcs}

\begin{pcomments}
  \pcomment{from: S09.cp3r}
%  \pcomment{}
%  \pcomment{}
\end{pcomments}

\pkeywords{
  relational_properties
  partial_orders
}

%%%%%%%%%%%%%%%%%%%%%%%%%%%%%%%%%%%%%%%%%%%%%%%%%%%%%%%%%%%%%%%%%%%%%
% Problem starts here
%%%%%%%%%%%%%%%%%%%%%%%%%%%%%%%%%%%%%%%%%%%%%%%%%%%%%%%%%%%%%%%%%%%%%

\begin{problem}
  Prove that if $R$ is transitive and irreflexive, then it is a strict
  partial order.

\begin{solution}
  Suppose $R$ is transitive and irreflexive.  To show that $R$ is a strict
  partial order, we need to show that it is transitive and asymmetric.  That
  $R$ is transitive we know already. To prove that it is asymmetric, suppose
  $a\mrel{R}b$ holds for some $a,b\in A$.  We need to prove 
  $\QNOT(b \mrel{R} a)$.

  So assume to the contrary that $bRa$ holds.  Now $aRb$ and $bRa$, imply
  $aRa$ since $R$ is transitive. This contradicts the fact that $R$ is
  irreflexive.  So $bRa$ cannot hold, that is, $not(bRa)$ holds.

\end{solution}
\end{problem}

%%%%%%%%%%%%%%%%%%%%%%%%%%%%%%%%%%%%%%%%%%%%%%%%%%%%%%%%%%%%%%%%%%%%%
% Problem ends here
%%%%%%%%%%%%%%%%%%%%%%%%%%%%%%%%%%%%%%%%%%%%%%%%%%%%%%%%%%%%%%%%%%%%%

\endinput
