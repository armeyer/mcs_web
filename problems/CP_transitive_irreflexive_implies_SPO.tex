%/CP_transitive_irreflexive_implies_SPO

\documentclass[problem]{mcs}

\begin{pcomments}
  \pcomment{from: S09.cp3r}
%  \pcomment{}
%  \pcomment{}
\end{pcomments}

\pkeywords{
  relational_properties
  partial_orders
}

%%%%%%%%%%%%%%%%%%%%%%%%%%%%%%%%%%%%%%%%%%%%%%%%%%%%%%%%%%%%%%%%%%%%%
% Problem starts here
%%%%%%%%%%%%%%%%%%%%%%%%%%%%%%%%%%%%%%%%%%%%%%%%%%%%%%%%%%%%%%%%%%%%%

\begin{problem}
  A relation, $R$, on a set, $A$, is \emph{irreflexive} iff $\QNOT(a
  \mrel{R} a)$ for all $a \in A$.

  Prove that if $R$ is transitive and irreflexive, then it is a strict
  partial order.

\begin{solution}
  Suppose $R$ is transitive and irreflexive.  To show that $R$ is a strict
  partial order, we need to show that it is transitive and asymmetric.  That
  $R$ is transitive we know already. To prove that it is asymmetric, suppose
  $a\mrel{R}b$ holds for some $a,b\in A$.  We need to prove 
  $\QNOT(b \mrel{R} a)$.

  So assume to the contrary that $b\mrel{R}a$ holds.  Now $a\mrel{R}b$ and
  $b\mrel{R}a$, imply $a\mrel{R}a$ since $R$ is transitive. This
  contradicts the fact that $R$ is irreflexive.  So $b\mrel{R}a$ cannot
  hold, that is, $\QNOT(b\mrel{R}a)$ holds.
\end{solution}

\end{problem}

%%%%%%%%%%%%%%%%%%%%%%%%%%%%%%%%%%%%%%%%%%%%%%%%%%%%%%%%%%%%%%%%%%%%%
% Problem ends here
%%%%%%%%%%%%%%%%%%%%%%%%%%%%%%%%%%%%%%%%%%%%%%%%%%%%%%%%%%%%%%%%%%%%%

\endinput
