\documentclass[problem]{mcs}

\begin{pcomments}
  \pcomment{CP_Schroeder_Bernstein_theorem}
  \pcomment{from: S09.cp3t}
  \pcomment{edited by ARM 2/12/11, revised 3/8/16}
\end{pcomments}

\pkeywords{
  relation
  mapping_lemma
  function
  injection
  surjection
  bijection
  Schroeder
  Bernstein
  Cantor
  total
}

%%%%%%%%%%%%%%%%%%%%%%%%%%%%%%%%%%%%%%%%%%%%%%%%%%%%%%%%%%%%%%%%%%%%%
% Problem starts here
%%%%%%%%%%%%%%%%%%%%%%%%%%%%%%%%%%%%%%%%%%%%%%%%%%%%%%%%%%%%%%%%%%%%%

\begin{problem}
This problem provides a proof of the [\idx{Schr\"oder-Bernstein}] Theorem:
\begin{equation}\label{SB_thm_surj}
\text{If $A \inj B$ and $B \inj A$, then $A \bij B$.}
\end{equation}

Assume for simplicity that $A$ and $B$ have no elements in common.
Let's picture the elements of $A$ arranged in a column and likewise
$B$ arranged in a second column to the right.  Add left-to-right
arrows connecting $a$ to $f(a)$ for each $a \in A$ and likewise
right-to-left arrows for $g$.  So $A \inj B$ means there is
\emph{exactly one} arrow \emph{out} of each element of since $f$ and
$g$ are total functions.  Also, there is \emph{at most one} arrow
\emph{into} any element, because $f$ and $g$ are injections.

So starting at any element, there is a unique and unending path of
arrows going forwards (it might repeat).  There is also a unique path
of arrows going backwards, which might be unending, or might end at an
element that has no arrow into it.  These paths are completely
separate: if two ran into each other, there would be two arrows into
the element where they ran together.

This divides all the elements into separate paths of four kinds:
%\renewcommand\theenumi {\roman{enumi}}
\begin{enumerate}[(i)]
\item paths that are infinite in both directions,
\item paths that are infinite going forwards starting from some element of
  $A$.
\item paths that are infinite going forwards starting from some element of
  $B$.
\item\label{cycle} paths that are unending but finite.
\end{enumerate}
%\renewcommand\theenumi {enumi}

\bparts

\ppart What do the paths of the last type~\eqref{cycle} look like?

\begin{solution}
An even-length cycle of alternating $f$- and $g$-arrows.
\end{solution}

\ppart\label{pathbijs} Show that for each type of path, either
\begin{enumerate}[(i)]

\item the $f$-arrows define a bijection between the $A$ and $B$ elements
  on the path, or

\item the $g$-arrows define a bijection between $B$ and $A$ elements on
  the path, or

\item both sets of arrows define bijections.
\end{enumerate}
For which kinds of paths do both sets of arrows define bijections?

\begin{solution}
For paths that start at a point in $A$, there will be an $f$-arrow out
of every point on the path, so the $f$-arrows will define a bijection
from the $A$ elements to the $B$ elements on the path.  The $g$-arrows
don't define a bijection the other way, because they don't hit the
starting point.

For paths that start at a point in $B$, the $g$-arrows will define a
bijection from the $B$ elements to the $A$ elements, by the same
reasoning.

For the other two types of path---cycles and two-way infinite---every
point in $B$ has exactly one $f$-arrow coming in, so these arrows
define a bijection from the $A$ elements to the $B$ elements.
Likewise, the $g$-arrows define a bijection from $B$ to $A$.

\end{solution}

\ppart Explain how to piece these bijections together to form a
bijection between $A$ and $B$.
\begin{solution}
Define a bijection $h:A \to B$ as follows:
\[
h(a) \eqdef
\begin{cases}
f(a) & \text{if $a$'s path does not start at a point in $B$},\\
\inv{g}(a) & \text{otherwise}.
\end{cases}
\]
By part~\eqref{pathbijs}, $h$ is a combination of bijections defined by
  different non-overlapping paths, and since every point is on a
  unique path used by $h$, the function $h$ must be a bijection.
\end{solution}

\ppart Justify the assumption that $A$ and $B$ are disjoint.

\begin{solution}
  We can always find sets $A' \bij A$ and $B' \bij B$ such that $A'$ and
  $B'$ are disjoint.  For example, let $A' = A \cross \set{0}$ and $B' = B
  \cross \set{1}$.  Then if we prove~\eqref{SB_thm_surj} for $A'$ and
  $B'$, we could conclude it held for $A$ and $B$ because
\[
A \bij A' \bij B' \bij B.
\]
\end{solution}

\eparts

\end{problem}

%%%%%%%%%%%%%%%%%%%%%%%%%%%%%%%%%%%%%%%%%%%%%%%%%%%%%%%%%%%%%%%%%%%%%
% Problem ends here
%%%%%%%%%%%%%%%%%%%%%%%%%%%%%%%%%%%%%%%%%%%%%%%%%%%%%%%%%%%%%%%%%%%%%

\endinput
