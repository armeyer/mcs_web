\documentclass[problem]{mcs}

\begin{pcomments}
  \pcomment{FP_chinese_remainder}
  \pcomment{adapted from: CP_chinese_remainder, which used to appear as first part of PS_Euler_function_multiplicativity}
  \pcomment{original by: ARM 2/27/11; revised 3/2/11}
  \pcomment{adapted by: Kazerani 5/11}
\end{pcomments}

\pkeywords{
  prime
  relatively_prime
  number_theory
  modular_arithmetic
  chinese_remainder
  remainder
}

%%%%%%%%%%%%%%%%%%%%%%%%%%%%%%%%%%%%%%%%%%%%%%%%%%%%%%%%%%%%%%%%%%%%%
% Problem starts here
%%%%%%%%%%%%%%%%%%%%%%%%%%%%%%%%%%%%%%%%%%%%%%%%%%%%%%%%%%%%%%%%%%%%%

\begin{problem}
If $a$ and $b$ are relatively prime and greater than $1$, then for any $x$,
\begin{equation}
\brac{x \equiv 0 \bmod a \ \QAND\  x \equiv 0 \bmod b} \qimplies x \equiv 0 \bmod{ab}.\label{0cong}
\end{equation}
Using this, prove that
\begin{equation}
\brac{x \equiv x^\prime \bmod{a}\ \QAND\ x \equiv x^\prime \bmod b} \qimplies
x \equiv x^\prime \bmod{ab}.\label{primecong}
\end{equation}

\begin{staffnotes}
If needed suggest ``Look at $x^\prime - x$.''
\end{staffnotes}

\begin{solution}
$x \equiv x^\prime \bmod{a}$ implies $(x - x^\prime)\equiv 0 \bmod a$, and
$x \equiv x^\prime \bmod{b}$ implies $(x - x^\prime)\equiv 0 \bmod b$.
So by ~\eqref{0cong}, $(x-x^\prime) \equiv 0 \bmod{ab}$, and hence
\[
x \equiv x^\prime \bmod{ab}.
\]
\end{solution}

\end{problem}

%%%%%%%%%%%%%%%%%%%%%%%%%%%%%%%%%%%%%%%%%%%%%%%%%%%%%%%%%%%%%%%%%%%%%
% Problem ends here
%%%%%%%%%%%%%%%%%%%%%%%%%%%%%%%%%%%%%%%%%%%%%%%%%%%%%%%%%%%%%%%%%%%%%

\endinput
