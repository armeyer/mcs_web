\documentclass[problem]{mcs}

\begin{pcomments}
  \pcomment{TP_sort_cyclic_shift_three}
  \pcomment{related to CP_fifteen_puzzle}
  \pcomment{ARM and CH, S14}
\end{pcomments}

\pkeywords{
  state_machines
  sorting
  fifteen_puzzle
  parity
}

%%%%%%%%%%%%%%%%%%%%%%%%%%%%%%%%%%%%%%%%%%%%%%%%%%%%%%%%%%%%%%%%%%%%%
% Problem starts here
%%%%%%%%%%%%%%%%%%%%%%%%%%%%%%%%%%%%%%%%%%%%%%%%%%%%%%%%%%%%%%%%%%%%%

\begin{problem}
The following problem is a twist on the {Fifteen-Puzzle} problem that
we did in class.

Let $A$ be a sequence consisting of the numbers $1,\dots,n$ in some
order.  A pair of integers in $A$ is called an \emph{out-of-order
  pair} when the first element of the pair both comes \emph{earlier}
in the array, and \emph{is larger}, than the second element of the
pair.  For example, the array $A = (1,2,4,5,3)$ has two out-of-order
pairs: $(4,3)$ and $(5,3)$.  We let $t(A)$ equal the number of
out-of-order pairs in $A$, so $t((1,2,4,5,3)) = 2$.

The elements in $A$ can rearranged using the \textsc{Rotate-Triple}
operation, in which three consecutive elements of $A$ are rotated to
move the smallest of them to be first.

For example, if $A = (2,4,1,5,3)$ the \textsc{Rotate-Triple} operation
might rotate the consecutive numbers $4,1,5$, into $1, 5, 4$ so that
\[
A = (2,4,1,5,3) \movesto (2, 1, 5, 4, 3).
\]

Alternatively, the \textsc{Rotate-Triple} might rotate the consecutive
numbers $2,4,1$ into $1,2,4$ so that
\[
(2,4,1,5,3) \movesto (1, 2, 4, 5, 3).
\]

We can think of a sequence $A$ as a state of a state machine whose
transitions correspond to applications of the \textsc{Rotate-Triple}
operation.

\begin{problemparts}
\iffalse

\problempart Write out the set of transitions that describe the
actions of $\textsc{Rotate-Triple}$ at any time step.

\hint You will need to consider six different cases.

\begin{solution}

Consider any consecutive triple in $A$ containing the elements $\{ a,b,c
\}$ in some order. Without loss of generality, assume that $a < b < c$. The
six different cases correspond to the different permutations of
$\{a,b,c\}$. Explicitly, the set of transitions can be described as follows:
\begin{align*}
\textsc{Rotate-Triple}(a,b,c) &\movesto (a,b,c) \\
\textsc{Rotate-Triple}(a,c,b) &\movesto (a,c,b) \\
\textsc{Rotate-Triple}(b,c,a) &\movesto (a,b,c) \\
\textsc{Rotate-Triple}(b,a,c) &\movesto (a,c,b) \\
\textsc{Rotate-Triple}(c,b,a) &\movesto (a,c,b) \\
\textsc{Rotate-Triple}(c,a,b) &\movesto (a,b,c).
\end{align*}

\end{solution}

\examspace[1in]
\fi

\problempart
Argue that for the state machine model described above, the derived
variable $t$ is \emph{weakly decreasing}.

\begin{solution}
Suppose $a,b,c$ are three consecutive elements in $A$.  If $b$ is the
smallest element, then these elements get rearranged into $b,c,a$.n  If
$c$ is the smallest, they get rearranged into $c,a,b$.


$(b,a,c) \movesto (a,c,b)$ removes one out-of-order pair $(b,a)$, but
introduces a {\em new} out-of-order pair $(c,b)$.  therefore resulting
in no change in $f$.

The transitions $(b,c,a) \movesto (a,b,c)$ and $(c,a,b) \movesto (a,b,c)$
both \emph{decrease} the number of out-of-order pairs from 2 to 0. 

Finally, the transition $(c,b,a) \movesto (a,c,b)$ decreases the
number of out-of-order pairs from 3 to 1.

In each of the cases, $f$ is either constant or reduces in value. In other words, $f$ is weakly decreasing.

\end{solution}

\examspace[1in]

\problempart  Define the \emph{parity} of a list $L$ to be 0 or 1, depending on whether the number of
out-of-order pairs in $L$ is even or odd. For example, the parity of the list $L = (1,2,4,5,3)$ is 0
(since $f(L)=2$). 

Show that the parity of a state is preserved under transitions.  

\begin{solution}

This part follows directly from the argument that $f$ is weakly
decreasing. We have shown that in each transition, $f$ changes  by a
value of 0 or 2. Therefore, the parity of $f$ remains unchanged.

\end{solution}

\examspace[1.5in]

\problempart Argue that if the procedure terminates, the all
of the numbers will eventually end up in their ``correct'' place, {except possibly
  the last two numbers $n-1$ and $n$}.

\begin{solution}
\TBA{\textbf{FILL IN.}}
\end{solution}

\end{problemparts}

\end{problem}

%%%%%%%%%%%%%%%%%%%%%%%%%%%%%%%%%%%%%%%%%%%%%%%%%%%%%%%%%%%%%%%%%%%%%
% Problem ends here
%%%%%%%%%%%%%%%%%%%%%%%%%%%%%%%%%%%%%%%%%%%%%%%%%%%%%%%%%%%%%%%%%%%%%

\endinput
