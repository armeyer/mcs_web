\documentclass[problem]{mcs}

\begin{pcomments}
  \pcomment{TP_sort_cyclic_shift_three}
  \pcomment{related to CP_fifteen_puzzle}
  \pcomment{ARM and CH, S14}
\end{pcomments}

\pkeywords{
  state_machines
  sorting
  fifteen_puzzle
  parity
}

%%%%%%%%%%%%%%%%%%%%%%%%%%%%%%%%%%%%%%%%%%%%%%%%%%%%%%%%%%%%%%%%%%%%%
% Problem starts here
%%%%%%%%%%%%%%%%%%%%%%%%%%%%%%%%%%%%%%%%%%%%%%%%%%%%%%%%%%%%%%%%%%%%%

\begin{problem}
The following problem is a variation on the \emph{Fifteen Puzzle}
problem encountered in class. It describes a (somewhat nonstandard)
approach to sort an array of integers.

Let $L$ be a sequence of the numbers $1,\dots,n$ in some order.  A
subsequence of 3 integers is called an \emph{out-of-order triple} in $L$ when the first element of
the triple pair both comes \emph{earlier} in the list and \emph{is larger} than
the second element of the pair.  For example, the sequence $1,2,4,5,3,6$ has two
out-of-order triples: $(4,5,3)$ and $(5,3,6)$.  The increasing list
$1,2 \ldots n$ has no out-of-order triples. 


Let a state, $S$, be a pair $(L, (i,j))$ described above.  
% We define the
% \emph{parity} of $S$ to be 0 or 1 depending on whether the sum of the number of
% out-of-order pairs in $L$ and the row-number of the empty square is
% even or odd. that is
% \[
% \text{parity}(S) \eqdef\begin{cases}
% 0 & (\text{if } p(L) + i \text{ is even},\\
% 1 & \text{otherwise}.
% \end{cases}
% \]

\begin{problemparts}

\problempart Verify that the parity of the start state and the target
state are different.

\begin{solution}
The parity of the start state is 0 since $(0+4)$ is even.  The parity
of the target is 1 because $((15 \cdot 14/ 2) + 4)$ is odd.

\end{solution}

\problempart Show that the parity of a state is preserved under
transitions.  

\begin{solution}

\TBA{\textbf{FILL IN.}}

\end{solution}

\problempart Argue that the procedure ensures that all
of the numbers will eventually end up in their ``correct'' place, {except possibly
  the last two numbers $n-1$ and $n$}.

\begin{solution}
\TBA{\textbf{FILL IN.}}
\end{solution}

\end{problemparts}

\end{problem}

%%%%%%%%%%%%%%%%%%%%%%%%%%%%%%%%%%%%%%%%%%%%%%%%%%%%%%%%%%%%%%%%%%%%%
% Problem ends here
%%%%%%%%%%%%%%%%%%%%%%%%%%%%%%%%%%%%%%%%%%%%%%%%%%%%%%%%%%%%%%%%%%%%%

\endinput
