\documentclass[problem]{mcs}

\begin{pcomments}
  \pcomment{FP_multiple_choice_unhidden_fall13}
  \pcomment{revised from FP_multiple_choice_unhidden by ARM 12/13/13}
  \pcomment{overlaps FP_graphs_short_answer}  
\end{pcomments}

\pkeywords{
  isomorphism
  partial_order  
  total_order
  linear
  asymptotic
  vertices
  gcd
  modular
  mod_n
  stable_distribution
  countable
}

%%%%%%%%%%%%%%%%%%%%%%%%%%%%%%%%%%%%%%%%%%%%%%%%%%%%%%%%%%%%%%%%%%%%%
% Problem starts here
%%%%%%%%%%%%%%%%%%%%%%%%%%%%%%%%%%%%%%%%%%%%%%%%%%%%%%%%%%%%%%%%%%%%%

\begin{problem} \mbox{}

% FROM: Spring07 MQ-4/6-2a
%
% COMMENTS: Reduced number, changed statements, now asking for
% counterexample. Solution incomplete. Should we replace "Lemma 3.4"
% with an \eqref?
%
% Need to make boxes (make sure they know that counterexamples req'd): 
%     true        false__________________ <-- room for counterexample
%

\textbf{\large Circle \textbf{true} or \textbf{false} for the statements below,
and \emph{provide counterexamples} for those that are \textbf{false}.}

\bparts

\ppart The following statements about the greatest common divisor:

\begin{itemize}

\item If $\gcd(a, b) \neq 1$ and $\gcd(b, c) \neq 1$, then $\gcd(a, c) \neq 1$. \hfill 
\textbf{true} \qquad  \textbf{false} \examspace[0.4in]

\begin{solution}
\textbf{false} $a=2\cdot 3, b=3\cdot 5, c=5\cdot 7$
\end{solution}

\item If $a \divides b c$ and $\gcd(a, b) = 1$, then $a \divides c$.  \hfill 
\textbf{true} \qquad \textbf{false} \examspace[0.4in]

\begin{solution}
\textbf{true}
\end{solution}

\item $\gcd(a^n,b^n) =  (\gcd(a,b))^n$  \hfill 
\textbf{true} \qquad \textbf{false} \examspace[0.4in]

\begin{solution}
\textbf{true}.
\end{solution}

\item $\gcd(ab, ac) = a \gcd(b, c)$.  \hfill 
\textbf{true} \qquad  \textbf{false} \examspace[0.4in]

\begin{solution}
\textbf{true}
\end{solution}

\end{itemize}

\ppart The following statements about congruence modulo $n$, where $n > 1$.

\begin{itemize}

\item If $a c \equiv b c \pmod{n}$ and $n$ does not divide $c$, then $a \equiv b \pmod{n}$.
\hfill \textbf{true} \qquad \textbf{false} \examspace[0.4in]

\begin{solution}
\textbf{false}.  Need $c$ relatively prime to $n$.  Counterexample:
$n=2 \cdot 3, a=0, b=2, c=3$
\end{solution}

\item If $a \equiv b \pmod{\phi(n)}$ for $a, b > 0$, then $c^a \equiv c^b
\pmod{n}$.  \hfill \textbf{true} \qquad \textbf{false} \examspace[0.4in]

\begin{solution}
  \textbf{false}.  Need $c$ relatively prime to $n$.  Counterexample:
  $n=4$, so $\phi(n) = 2$; $a=1, b=3$, so $a \equiv b
  \pmod \phi(n)$, $c = 2$, so $c^a = 2 \not\equiv 0 = c^b \pmod 4$.
\end{solution}

\iffalse
\item If $a \equiv b \pmod{n}$, then $P(a) \equiv P(b) \pmod{n}$ for any
polynomial $P(x)$ with integer coefficients.
 \hfill \textbf{true} \qquad \textbf{false} \examspace[0.4in]
\begin{solution}
true
\end{solution}
\fi

\item If $a \equiv b \pmod{nm}$, then $a \equiv b \pmod{n}$, for $m,mn > 1$.
 \hfill \textbf{true} \qquad \textbf{false} \examspace[0.4in]

\begin{solution}
\textbf{true}
\end{solution}

\item For relatively prime $m,n >1$,\\
$[a \equiv b \pmod{m} \QAND a \equiv b \pmod{n}] \QIFF [a \equiv b
  \pmod{mn}]$
\hfill \textbf{true} \qquad \textbf{false} \examspace[0.4in]

\begin{solution}
\textbf{true}
\end{solution}

\end{itemize}

\ppart The following statements about trees:

\begin{itemize}

\item Any connected subgraph is a tree.  \hfill \textbf{true} \qquad \textbf{false} \examspace[0.4in]

\begin{solution}
\textbf{true}
\end{solution}

\item Adding an edge between two nonadjacent vertices creates a cycle. \hfill 
\textbf{true} \qquad \textbf{false}  \examspace[0.4in]

\begin{solution}
\textbf{true}
\end{solution}

\item The number of vertices is one less than twice the number of
  leaves.  \hfill 
\textbf{true} \qquad \textbf{false}  \examspace[0.4in]

\begin{solution}
  \textbf{false}.  This property holds for full binary trees, but not in
  general.  A tree with two vertices is a counterexample.
\end{solution}

\end{itemize}

\eparts

\examspace[0.2in]
Now answer the following:

\bparts

\ppart List the numbers of the properties below that are preserved under graph
isomorphism. \hfill\examrule[1in]

\begin{enumerate}

\item There is a cycle that includes all the vertices.

\item Two edges are of equal length.

\item The graph remains connected if any two edges are removed.

\item There exists an edge that is an edge of every spanning tree.

\item The negation of a property that is preserved under isomorphism.

\end{enumerate}

\begin{solution}
All but the second one are preserved.
\end{solution}

\ppart A \term{sink} in a random walk digraph is a vertex with no
edges leaving it.  Circle whichever of the following assertions are
true of \idx{stable distributions} on random walk graphs with exactly
two sinks:

\begin{itemize}

\item there may not be any

\item there may be a unique one

\item there are exacty two

\item there may be a countably infinite number

\item there may be a uncountable number

\item there always is an uncountable number

\end{itemize}

\begin{solution}
The first three choices are false, and the last three are true.
That's because a distribution in which one sink has probability $r \in
[0,1] \subseteq \reals$ and the other sink has probability $1-r$ is
stable, and there are an is uncountable number of real numbers in $[0,1]$.
\end{solution}

\eparts

\end{problem}

\endinput
