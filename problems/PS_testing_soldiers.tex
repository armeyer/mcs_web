 \documentclass[problem]{mcs}

\begin{pcomments}
  \pcomment{PS_testing_soldiers}
  \pcomment{F95.ps11}
  \pcomment{from Feller, exercises on expectation}
  \pcomment{revised by ARM 5/22/12}
\end{pcomments}

\pkeywords{
 probability
 random_variable
 expectationa
}

%%%%%%%%%%%%%%%%%%%%%%%%%%%%%%%%%%%%%%%%%%%%%%%%%%%%%%%%%%%%%%%%%%%%%
% Problem starts here
%%%%%%%%%%%%%%%%%%%%%%%%%%%%%%%%%%%%%%%%%%%%%%%%%%%%%%%%%%%%%%%%%%%%%

\begin{problem} (A true story from world war two).

The army needs to test each of its soldiers for a disease.  There is a
blood test that accurately determines when a blood sample contains
blood from a diseased soldier.  Assume that $p$ is the fraction of
diseased soldiers and that there are $n$ soldiers.

Approach 1.\ is to test blood from each soldier individually; this
requires $n$ tests.  Approach 2./ is to randomly group the soldiers
into $g$ groups of $k$ soldiers, where $n = gk$.  Then blend the blood
samples of each group, and apply the test once to each of the $g$
blended samples.  If the group-blend is free of the disease, we are
done with that group after one test.  If the group-blend fails the
test, then someone in the group has the disease, and we then test all
$k$ people for a total of $k+1$ tests on that group.

\bparts

\ppart What is the expected number of tests in Approach 2.\ as a
function of the number of soldiers $n$, the disease fraction $p$, and
the group size $k$?  (Assume that the probability that a soldier who
is chosen to be in a group is diseased remains equal to $p$,
independently of which other soldiers are chosen to be in the group.
This approximation is justified if $k$ is small relative to $pn$.)

\begin{solution}
Let the random variable $X_i$ be the number of tests performed on
group $i$.  Assuming mutual independence of soldiers in a group being
diseased, $X_i$ takes value $1$ with probability $(1-p)^k$ and value
$k+1$ with probability $1-(1-p)^k$.  Hence the expected number of
tests is
\begin{align}
\Expect{\sum_{i=1}^g X_i}
   & = \sum_{i=1}^g \expect{X_i}\notag\\
   & = g\paren{(1-p)^k + (k+1)(1 - (1-p)^k)}\notag\\
   & = n\paren{\frac{k+1}{k} - (1-p)^k}.\label{EsXi}
\end{align}

\end{solution}

\ppart Assuming $p$ is reasonably small, show how to choose $k$ so
that the expected number of tests using Approach 2.\ is approximately
$n\sqrt{p}$.

\begin{solution}
Choose $k$ to minimize~\eqref{EsXi}.  To do this, choose $k$ so that
the derivative wrt $k$ of~\eqref{EsXi}, namely,
\[
\frac{1}{k^2} + (1-p)^k \ln(1-p),
\]
equals 0.  Assuming that $p$ is small, we can approximate $(1-p)^k$
by $1$ and $\ln(1-p)$ by $-p$, which implies that
\[
k \approx \frac{1}{\sqrt{p}}.
\]
Plugging this approximation for $k$ into~\eqref{EsXi} yields
\[
\expect{\text{\#tests with Approach 2}} \approx n\sqrt{p}.
\]
\end{solution}

\ppart What fraction of the work does Approach 2.\ expect to save over
Approach 1.\ in a million-strong army with disease incidence $1\%$?

\begin{solution}
If $p = 0.01$ then the fraction of work saved is
\[
\frac{1M - 1M(\sqrt{0.01})}{1M} = 0.9
\]
That is, Approach 2.\ saves $90\%$ of the testing that Approach
1.\ would require.

\end{solution}

\ppart Can you come up with a better scheme by using multiple levels
of grouping, that is, groups of groups?

\begin{solution}
There are many possible improvements.  Here is an Approach 3: we blend
all the soldiers' blood and perform the test.  If it comes up negative
we stop, else we split the soldiers into two groups and repeat on each
group.  If $E_n$ is the expected number of test with $n$ soldiers.
Under a similar independence assumption about soldiers in the same
group being diseased, we get
\[
E_n = (1-p)^n + (1- (1-p)^n)2E_{n/2}.
\]
This implies that $E_n = \Theta(pn)$ (see
Chapter~\bref{chap:recurrences} for an explanation of such
recurrences).  Since, in an expected sense, there will $pn$ soldiers
with the disease, and the number of soldiers with the disease is a
lower bound on the number of tests, this approach is optimal up to
constant factors.
\end{solution}

\eparts

\end{problem}


%%%%%%%%%%%%%%%%%%%%%%%%%%%%%%%%%%%%%%%%%%%%%%%%%%%%%%%%%%%%%%%%%%%%%
% Problem ends here
%%%%%%%%%%%%%%%%%%%%%%%%%%%%%%%%%%%%%%%%%%%%%%%%%%%%%%%%%%%%%%%%%%%%%

\endinput
