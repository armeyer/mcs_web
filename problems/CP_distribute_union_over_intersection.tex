\documentclass[problem]{mcs}

\begin{pcomments}
  \pcomment{CP_distribute_union_over_intersection}
  \pcomment{ARM 2/17/13}
\end{pcomments}

\pkeywords{
  logic
  set_theory
  identity
  propositional
  chain_of_iff
  distributivity
  distribute
}

%%%%%%%%%%%%%%%%%%%%%%%%%%%%%%%%%%%%%%%%%%%%%%%%%%%%%%%%%%%%%%%%%%%%%
% Problem starts here
%%%%%%%%%%%%%%%%%%%%%%%%%%%%%%%%%%%%%%%%%%%%%%%%%%%%%%%%%%%%%%%%%%%%%

\begin{problem}
Prove
\begin{theorem*}[Distributivity of union over intersection]
\begin{equation}\label{AUBIC=}
A \union (B \intersect C) = (A \union B) \intersect (A union C)
\end{equation}
\end{theorem*}
for all sets, $A, B, C$, by using a chain of iff's to show that
\[
x \in A \union (B \intersect C) \QIFF x \in (A \union B) \intersect (A union C)
\]
for all elements, $x$.  You may assume the corresponding propositional
equivalence Theorem~\bref{thm:distribute-or-and}.

\begin{solution}
Two sets are equal iff they have the same elements, that is, $x$ is in
one set iff $x$ is in the other set, for any $x$.  We'll now prove this
for the two sides of~\eqref{AUBIC=}:

\begin{align*}
\lefteqn{x \in A \union (B \intersect C)}\\
 & \qiff x \in A\ \QOR\ x \in (B \intersect C)
     & \text{(by def of $\union$)}\\
 & \qiff (x \in A \QOR\ (x \in B\ QAND\ x \in C) & \text{(by def of $\intersect$)}\\
 & = (P \QOR (Q \QAND R)
   & \text{(where $P \eqdef [x \in A]$, $Q \eqdef [x \in B]$, $R \eqdef [x \in C]$ )}\\
 & \qiff (P \QOR Q) \QAND (P \QOR R)
     & \text{($\QOR$-$\QAND$ distributivity, Theorem~bref{thm:distribute-or-and})}
 & = (x \in A \QOR\ x \in B) QAND\ (x \in A \QOR\ x \in C)\\
 & \qiff (x \in A \union B) \QAND\ (x \in A \union C)
     & \text{(by def of $\union$)}\\
 & \qiff x \in (A \union B) \intersect (A \union C)
     & \text{(by def of $\intersect$)}.
\end{align*}

\end{solution}

\begin{staffnotes}
Ask your students if they can now see how a computer could
automatically check such equalities between set formulas involving the
basic set operators like $\union, \intersect, -, \dots$?  The answer
is that proving such equalities reduces to verifying equivalence of
corresponding propositional formulas as above.
\end{staffnotes}

\end{problem}

%%%%%%%%%%%%%%%%%%%%%%%%%%%%%%%%%%%%%%%%%%%%%%%%%%%%%%%%%%%%%%%%%%%%%
% Problem ends here
%%%%%%%%%%%%%%%%%%%%%%%%%%%%%%%%%%%%%%%%%%%%%%%%%%%%%%%%%%%%%%%%%%%%%

\endinput

