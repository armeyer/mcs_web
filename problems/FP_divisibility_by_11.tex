\documentclass[problem]{mcs}

\begin{pcomments}
  \pcomment{FP_divisibility_by_11}
  \pcomment{CH Spring '14; edited ARM 4/1/14 9:15PM}
  \pcomment{based on CP_multiples_of_9_and_11}
\end{pcomments}

\pkeywords{
  divisibility
  modular arithmetic
}


%%%%%%%%%%%%%%%%%%%%%%%%%%%%%%%%%%%%%%%%%%%%%%%%%%%%%%%%%%%%%%%%%%%%%
% Problem starts here
%%%%%%%%%%%%%%%%%%%%%%%%%%%%%%%%%%%%%%%%%%%%%%%%%%%%%%%%%%%%%%%%%%%%%

\begin{problem}
\iffalse Divisibility tests for small numbers such as 3, 4, 5, and 9 are
well-known.
\begin{staffnotes}
This will be demoralizing for the many students who don't know about
divisibility tests and is unnecessary for those that do know.
\end{staffnotes}
\fi
In this problem, we will construct a simple test for divisibility by 11.

\bparts

\ppart\label{rem10n11} Explain why \iffalse
$10^n \equiv (-1)^n \pmod{11}$\fi
\[
\rem{10^n}{11} = \begin{cases} 1 & \text{if $n$ is even,}\\
  \rem{-1}{11}   & \text{if $n$ is odd,}
\end{cases}
\]
for all nonnegative integers $n$.

\iffalse
\begin{staffnotes}
ARM: I've replaced the call for proof by induction by ``explain why.''

It would have been clearer to ask for a proof that
\[
a \equiv b \pmod{c}\ \QIMPLIES\ a^n \equiv b^n \pmod{c},
\]
but this of course is so obvious that proving it by induction is
perverse.
\end{staffnotes}
\fi

\begin{solution}
$10 \equiv -1 \pmod{11}$ and since congruence is preserved under
  product, $10^n \equiv (-1)^n \pmod{11}$.  Hence $\rem{10^n}{11} =
  \rem{(-1)^n}{11}$.
\end{solution}

\iffalse

\begin{solution}
Let $P(n)$ be the assertion that:
\begin{quote}
 $10^n \equiv (-1)^n \pmod{11}$ for all nonnegative integers $n$.
\end{quote}

\inductioncase{Base case} ($n = 0$) This follows easily from the fact
that $10^0 = (-1)^0 = 1$, which is congruent to $1 \pmod{11}$. 

\inductioncase{Inductive step} Assume that $P(n)$ is true, i.e., $10^n \equiv (-1)^n
\pmod{11}$ for some nonnegative integer $n$. Then, since $10 \equiv -1
\pmod{11}$, we know that
\begin{align*}
10^{n+1} &\equiv 10 \cdot 10^{n} \pmod{11} \\
 & \equiv (-1) \cdot (-1)^n \pmod{11}~~~\text{by Induction Hypothesis} \\
 & \equiv (-1)^{n+1} \pmod{11} .
\end{align*}
In other words, $P(n+1)$ is true. Therefore, by induction $P(n)$ is
true for all nonnegative integers $n$.
\end{solution}
\fi

\examspace[1.25in]

\ppart Take a big number, such as 47262938151.  Sum the digits, where
every other digit is negated:
\[
4 + (-7) + 2 + (-6) + 2 + (-9) + 3 + (-8) + 1 + (-5) + 1  =  -22.
\]
Explain why the original number is a multiple of 11 if and only if
this sum is a multiple of 11.  For example, this number 47262938151 is
divisible by 11 since -22 is divisible by 11.

\begin{solution}
A number in decimal has the form:
\[
d_k \cdot 10^k + d_{k-1} \cdot 10^{k-1} + \ldots + d_1 \cdot 10 + d_0
\]

By part~\eqref{rem10n11},  %Since $10 \equiv -1 \pmod{11}$, we know that:
\begin{align*}
\lefteqn{d_k \cdot 10^k + d_{k-1} \cdot 10^{k-1} + \cdots + d_1 \cdot 10 + d_0} \\
    & \equiv d_k \cdot 1 + d_{k-1} \cdot (-1) + \cdots  + d_1 \cdot (-1) + d_0 \cdot 1 \pmod{11}\\
    & = d_k - d_{k-1} + \cdots - d_1 + d_0
\end{align*}
when $k$ is even.  The case where $k$ is odd is the same with the
signs reversed.

So the procedure given in the problem computes $\pm$ this alternating
sum of digits.  In particular, the original number is congruent to
zero mod 11 iff the alternating sum is congruent to zero mod 11, which
is the same as being divisible by 11.
\end{solution}

\eparts

\end{problem}

%%%%%%%%%%%%%%%%%%%%%%%%%%%%%%%%%%%%%%%%%%%%%%%%%%%%%%%%%%%%%%%%%%%%%
% Problem ends here
%%%%%%%%%%%%%%%%%%%%%%%%%%%%%%%%%%%%%%%%%%%%%%%%%%%%%%%%%%%%%%%%%%%%%

\endinput
