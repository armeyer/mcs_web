\documentclass[problem]{mcs}

\begin{pcomments}
  \pcomment{TP_basic_partial_orders}
  \pcomment{from: S09.ps3, F10.cp4w}
  \pcomment{changed from to TP from CP, 3/16/10 by ARM}
\end{pcomments}


\pkeywords{
  partial_orders
  weak_partial_order  
  strict_partial_order
  transitive
  reflexive
  antisymmetric
  maximal
  minimal
}

%%%%%%%%%%%%%%%%%%%%%%%%%%%%%%%%%%%%%%%%%%%%%%%%%%%%%%%%%%%%%%%%%%%%%
% Problem starts here
%%%%%%%%%%%%%%%%%%%%%%%%%%%%%%%%%%%%%%%%%%%%%%%%%%%%%%%%%%%%%%%%%%%%%

\begin{problem}
For each of the binary relations below, state whether it is a strict
partial order, a weak partial order, or neither.  If it is not a partial
order, indicate which of the axioms for partial order it violates.
\iffalse
If it is a partial order, state whether it is a path-total order and
identify its maximal and minimal elements, if any.
\fi

\bparts

\ppart The superset relation, $\supseteq$ on the power set
$\power{\set{1, 2, 3, 4, 5}}$.

\begin{solution}
This is a weak partial order, but not a path-total one.  For example, 
the sets of size 3 form an antichain.
\end{solution}

\ppart The relation between any two nonegative integers, $a$, $b$ that 
$a \equiv b \pmod{8}$.

%The remainder of $a$ divided by 8 equals the remainder of $b$ divided by 8.

\begin{solution}
Violates antisymmetry: $8 \mrel{R} 16$ and $16 \mrel{R} 8$ but $8 \neq
16$.  It is transitive, though.
\end{solution}

\ppart The relation between propositional formulas, $G$, $H$, that $G
\QIMPLIES H$ is valid.

\begin{solution}
  Violates antisymmetry: $P$ and $\QNOT(\QNOT(P))$ imply each other but
  are different expressions.  It is transitive, though.
\end{solution}

\ppart The relation 'beats' on Rock, Paper and Scissor (for those who don't
know the game Rock, Paper, Scissors, Rock beats Scissors, Scissors beats
Paper and Paper beats Rock).

\begin{solution}
  Violates transitivity: obviously.  Also violates antisymmetry.
\end{solution}

\ppart The \idx{empty relation} on the set of real numbers.
\begin{solution}
  It's vacuously asymmetric and transitive, so it's a strict partial
  order.  It's irreflexive.  It's not path-total.  Every element is
  vacuously both minimal and maximal.
\end{solution}

\ppart The \idx{identity relation} on the set of integers.

\begin{solution}

  It's obviously reflexive, antisymmetric and transitive, so it's a
  weak partial order.  It's not path-total.  Every element is vacuously
  both minimal and maximal.

\end{solution}

\iffalse

\ppart The divisibility relation on the integers, $\integers$.

\begin{solution}
Not antisymmetric: 3 and -3 divide each other.  It is transitive and reflexive.
\end{solution}
\fi

\eparts

\end{problem}

%%%%%%%%%%%%%%%%%%%%%%%%%%%%%%%%%%%%%%%%%%%%%%%%%%%%%%%%%%%%%%%%%%%%%
% Problem ends here
%%%%%%%%%%%%%%%%%%%%%%%%%%%%%%%%%%%%%%%%%%%%%%%%%%%%%%%%%%%%%%%%%%%%%

\endinput
