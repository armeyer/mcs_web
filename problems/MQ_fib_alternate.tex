\documentclass[problem]{mcs}

\begin{pcomments}
  \pcomment{MQ_fibonacci_by_induction}
  \pcomment{taken from book/recursive_data.tex staff notes.}
\end{pcomments}

\pkeywords{
  induction
  fibonacci
  recurrence
}

%%%%%%%%%%%%%%%%%%%%%%%%%%%%%%%%%%%%%%%%%%%%%%%%%%%%%%%%%%%%%%%%%%%%%
% Problem starts here
%%%%%%%%%%%%%%%%%%%%%%%%%%%%%%%%%%%%%%%%%%%%%%%%%%%%%%%%%%%%%%%%%%%%%

\begin{problem} 

We can use the recursive definition of a function to establish its
properties by structural induction.

As an illustration, we'll prove a cute identity involving Fibonacci
numbers.  Fibonacci numbers provide lots of fun for mathematicians because
they satisfy many such identities.
\begin{proposition}
  $\forall n \geq 0 (\Sigma_{i=0}^n F_i^2 = F_n F_{n+1})$.
	\end{proposition}

	Example: $n = 4$:
	\[
	0^2 + 1^2 + 1^2 + 2^2 + 3^2 = 15 = 3 \cdot 5.
	\]
	Let's try a proof by (standard, not strong) induction.  The theorem
	statement suggests trying it with $P(n)$ defined as:
	\[
	\sum_{i=0}^n F_i^2 = F_n F_{n+1}.
	\]

	\textbf{Base case} ($n=0$). 
	$\Sigma_{i=0}^0 F_i^2 \eqdef (F_0)^2 = 0 = F_0 F_1$ because
	$F_0 \eqdef 0$.

	\textbf{Inductive step} ($n\geq 0$).  Now we stare at the gap between
	$P(n)$ and $P(n+1)$.  $P(n+1)$ is given by a summation that's obtained
	from that for $P(n)$ by adding one term; this suggests that, once again,
	we subtract.  The difference is just the term $F_{n+1}^2$.  Now, we are
	assuming that the original $P(n)$ summation totals $F_n F_{n+1}$ and want
	to show that the new $P(n+1)$ summation totals $F_{n+1} F_{n+2}$.  So we
	would {\em like\/} the difference to be
	\[
	F_{n+1} F_{n+2} - F_n F_{n+1}.
	\]

	So, the actual difference is $F_{n+1}^2$ and the difference we want is
	$F_{n+1} F_{n+2} - F_n F_{n+1}$.  Are these the same?  We want to check
	that:
	\[
	F_{n+1}^2 = F_{n+1} F_{n+2} - F_n F_{n+1}.
	\]
	But this is true, because it is really the Fibonacci definition in
	disguise: to see this, divide by $F_{n+1}$.

\end{problem}

%%%%%%%%%%%%%%%%%%%%%%%%%%%%%%%%%%%%%%%%%%%%%%%%%%%%%%%%%%%%%%%%%%%%%
% Problem ends here
%%%%%%%%%%%%%%%%%%%%%%%%%%%%%%%%%%%%%%%%%%%%%%%%%%%%%%%%%%%%%%%%%%%%%

\endinput
