\documentclass[problem]{mcs}

\begin{problem}
Prove the following theorem.
\begin{theorem*}\label{thm:weightmatrix-min+}
  If $W$ is the minimum weight matrix for length $k$ walks in a weighted
  graph $G$, and $M$ is the minimum weight matrix for length $m$ walks,
  then then $W\minplusop M$ is the minimum weight matrix for length $k+m$
  walks.
\end{theorem*}

\begin{solution}
\begin{proof}
  The proof is virtually the same as the proof of Theorem~\ref{thm:CkDm}
  with multiplication of elements replaced by addition, and the sum of the
  multiplications by the minimum of the additions:

  Any length $k+m$ path between vertices $u$ and $v$ begins with a length
  $k$ path starting at $u$ and ending at some vertex, $x$, followed by a
  length $m$ path starting at $x$ and ending at $v$.  So the minimum
  weight of a length $k+m$ path from $u$ to $v$ that goes through $x$ at
  the $k$th step equals the minimum weight $W_{ux}$ of length $k$ walks
  from $u$ to $x$, plus the minimum weight $M_{xv}$ of length $m$ walks
  from $w$ to $v$.  So we can get the minimum weight of length $k+m$ walks
  from $u$ to $v$ by taking the minimum over all possible vertices $x$ of
  the minimum weight of such walks that go through $w$ at the $k$th step.
  In other words,
\begin{equation}\label{ln-min+nuv}
\text{min weight of a length $n+m$ path from $u$ to $v$} =
              \min_{x \in \vertices{G}} W_{ux}+M_{xv}\, .
\end{equation}
But the right hand side of~\eqref{ln-min+nuv} is precisely the definition of
$(W\minplusop M)_{uv}$.  Thus, $W\minplusop M$ is indeed the minimum weight
matrix for walks of length $k+m$.
\end{proof}
\end{solution}

\end{problem}

\endinput
