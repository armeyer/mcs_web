\documentclass[problem]{mcs}

\begin{pcomments}
\pcomment{CP_sum_bound_by_integral}
\pcomment{Theorem~\bref{weak_increasing_sum_bounds}of the text}
\end{pcomments}

\pkeywords{
integral
sum
bound
weakly_decreasing
}

%%%%%%%%%%%%%%%%%%%%%%%%%%%%%%%%%%%%%%%%%%%%%%%%%%%%%%%%%%%%%%%%%%%%%
% Problem starts here
%%%%%%%%%%%%%%%%%%%%%%%%%%%%%%%%%%%%%%%%%%%%%%%%%%%%%%%%%%%%%%%%%%%%%

\begin{problem}
Let $f: \reals^+ \to \reals^+$ be a weakly decreasing function.  Define
\[
    S \eqdef \sum_{i = 1}^n f(i)
\]
and
\[
    I \eqdef \int_1^n f(x)\, dx.
\]
Prove that
\[
    I + f(n) \le S \le I + f(1).
\]

\begin{solution}

The argument for the case when $f$is weakly decreasing is very
similar.  The analogous graphs to those shown in
Figures~\bref{fig:9G4}--\bref{fig:9G6} are provided in
Figure~\ref{fig:9G9}.  As you can see by comparing the shaded regions
in Figures \ref{fig:9G9}(a) and~\ref{fig:9G9}(b),
\begin{equation*}
    S \le I + f(1).
\end{equation*}
Similarly, comparing the shaded regions in Figures \ref{fig:9G9}(a)
and~\ref{fig:9G9}(c) reveals that
\begin{equation*}
    S \ge I + f(n).
\end{equation*}
Hence, if $f$~is nonincreasing,
\begin{equation*}
    I + f(n) \le S \le I + f(1).
\end{equation*}
as claimed.

\begin{figure}

\subfloat[]{\graphic{Fig_G9-a}}

\subfloat[]{\graphic{Fig_G9-b}}

\subfloat[]{\graphic{Fig_G9-c}}

\caption{The area of the shaded region in~(a) is $S = \sum_{i = 1}^n
  f(i)$.  The area in the shaded regions in (b) and~(c) is $I =
  \int_1^n f(x)\,dx$.}

\label{fig:9G9}

\end{figure}

\end{solution}
\end{problem}

%%%%%%%%%%%%%%%%%%%%%%%%%%%%%%%%%%%%%%%%%%%%%%%%%%%%%%%%%%%%%%%%%%%%%
% Problem ends here
%%%%%%%%%%%%%%%%%%%%%%%%%%%%%%%%%%%%%%%%%%%%%%%%%%%%%%%%%%%%%%%%%%%%%

\endinput
