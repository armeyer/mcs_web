\documentclass[problem]{mcs}

\begin{pcomments}
  \pcomment{PS_RSA_key_implies_factoring}
  \pcomment{from: S09.ps8; F06.ps6}
  \pcomment{commented out in S09 - has typos and probably needs substantial revision}
\end{pcomments}

\pkeywords{
  RSA
  number_theory
  modular_arithmetic
}

%%%%%%%%%%%%%%%%%%%%%%%%%%%%%%%%%%%%%%%%%%%%%%%%%%%%%%%%%%%%%%%%%%%%%
% Problem starts here
%%%%%%%%%%%%%%%%%%%%%%%%%%%%%%%%%%%%%%%%%%%%%%%%%%%%%%%%%%%%%%%%%%%%%

\begin{problem}
You've seen how the RSA encryption scheme works, but why is it hard to
break?  In this problem, you will see that finding secret keys is as hard
as finding the prime factorizations of integers.  Since there is a general
consensus in the crypto community (enough to persuade many large financial
institutions, for example) that factoring numbers with a few hundred
digits requires astronomical computing resources, we can therefore be sure
it will take the same kind of overwhelming effort to find RSA secret keys
of a few hundred digits.  This means we can be confident the private RSA
keys are not somehow revealed by the public keys
\footnote{This is a very weak kind of ``security'' property, because it
doesn't even rule out the possibility of deciphering RSA encoded messages
by some method that did not require knowing the secret key.  Nevertheless,
over twenty years experience supports the security of RSA in practice.}


% It would calm a lot of people down if we could show that the ability to
% compute any nontrivial piece of information about a message sent via RSA
% implies the ability to factor, but no one has been able to do that.  (To
% put this in perspective, some people would not even be calmed by such a
% statement, since we are not sure that factoring is computationally
% difficult.  Still, if someone figures out how to factoring efficiently,
% then there would be much bigger problems in the world than whether or not
% you can send your message securely.)  Instead, what .


For this problem, assume that $n = p \cdot q$ where $p,q$ are both
\emph{odd} primes \iffalse (this only rules out the prime 2, which would
not be a very secure prime to choose)\fi and that $e$ is the public key
and $d$ the secret key of the RSA protocol..
Let $x \eqdef e \cdot d - 1$.

\bparts

\ppart  Show that $\phi(n)$ divides $x$.

\begin{solution}
$ed \equiv 1 \pmod \phi(n)$ by definition of $d$, so $\phi(n)$ divides $x$
by definition of $\equiv$ mod $\phi(n)$.

\end{solution}

\ppart Conclude that 4 divides $x$.

\begin{solution}
Since $p,q$ are odd, both $p-1$ and $q-1$ are even.  Thus $4$
divides $(p-1)(q-1) = \phi(n)$, so by part~(a), 4 also divides $x$.
\end{solution}

\ppart
\label{squareequals1modn}
Show that if $\gcd(r,n)=1$, then $r^x \equiv 1 \pmod{n}.$

\begin{solution}
By Euler's Theorem, $r^{\phi(n)} \equiv 1 \pmod{n}$.  By part~(a), $x =
k\phi(n)$ for some integer, $k$, so
\[
r^x = r^{k\phi(n)} = \paren{r^{\phi(n)}}^k \equiv 1^k \equiv 1 \pmod n.
\]

\end{solution}


% \ppart\label{k} Show that if you are lucky and find a positive integer $k<n$ such
% that $\gcd(k,n) \neq 1$, then you can factor $n$.

% \begin{solution}
%Since $0< k<n$, we know $0 < \gcd(k,n) < n$.  Also $\gcd(k,n)$
% divides $n$.  So if $\gcd(k,n) \neq 1$, it must be $p$ or $q$.
%\end{solution}

\eparts

A \emph{square root} of $m$ modulo $n$ is a nonnegative integer $s
< n$ such that $s^2 \equiv m \pmod n$.  Here is a nice fact to know: when
$n$ is a product of two odd primes, then every number $m$ such that
$\gcd(m,n)=1$ has 4 square roots modulo $n$.

In particular, the number 1 has four square roots modulo $n$.  The two
trivial ones are 1 and $n-1$ (which is $\equiv - 1 \pmod n$).  The other
two are called the \emph{nontrivial} square roots of 1.

\bparts
\ppart\label{y2=1}
Since you know $x$, then for any integer, $r$, you can also compute the
remainder, $y$, of $r^{x/2}$ divided by $n$.  So $y^2 \equiv r^x \pmod n$.
Now if $r$ is relatively prime to $n$, then $y$ will be a square root of 1
modulo $n$ by part~\eqref{squareequals1modn}.

Show that if $y$ turns out to be a \emph{nontrivial} root of 1 modulo $n$,
then you can factor $n$.  \hint From the fact that $y^2 - 1 = (y+1)(y-1)$,
show that $y+1$ must be divisible by exactly one of $q$ and $p$.

\begin{solution}
Since $y$ is a square root of 1 modulo $n$, we know that $n$
divides $y^2 -1 = (y+1)(y-1)$.  So $p$ must divide either $y+1$ or $y-1$,
and likewise $q$ must divide either $y+1$ or $y-1$.

But if $y$ is nontrivial, then $y+1$ and $y-1$ are positive and smaller
than $n$, so if $y+1$ is divisible by $p$ it can't also be divisible by
$q$, and likewise, if it is divisible by $q$ it can't also be divisible by
$p$.  So $y+1$ must be divisible by exactly one of $p$ and $q$.  So
$\gcd(y+1,n)$ must equal $p$ or $q$.
\end{solution}

\ppart It turns out that at least half the positive integers $r< n$ that
are relatively prime to $n$ will yield $y$'s in part~\eqref{y2=1} that are
nontrivial roots of 1.  Conclude that if, in addition to $n$ and the
public key, $e$, you also knew the secret key $d$, then you can be sure of
being able to factor $n$.

\begin{solution}
Keep choosing $r$'s at random.  Most $r$'s wil be relatively
prime to $n$ and at least half of these will yield nontrivial $y$'s in
part~\eqref{y2=1}, so you can be sure to turn up the needed nontrivial $y$
in not very many tries.  
\end{solution}

\eparts
\end{problem}

%%%%%%%%%%%%%%%%%%%%%%%%%%%%%%%%%%%%%%%%%%%%%%%%%%%%%%%%%%%%%%%%%%%%%
% Problem ends here
%%%%%%%%%%%%%%%%%%%%%%%%%%%%%%%%%%%%%%%%%%%%%%%%%%%%%%%%%%%%%%%%%%%%%
