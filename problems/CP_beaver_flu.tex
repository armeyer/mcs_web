\documentclass[problem]{mcs}

\begin{pcomments}
  \pcomment{CP_beaver_flu}
  \pcomment{from: S09.cp5r; F06.ps2}
  \pcomment{revised S09 by ARM}
\end{pcomments}

\pkeywords{
  state_machines
  unreachable_states
  increasing_decreasing_variables
}

%%%%%%%%%%%%%%%%%%%%%%%%%%%%%%%%%%%%%%%%%%%%%%%%%%%%%%%%%%%%%%%%%%%%%
% Problem starts here
%%%%%%%%%%%%%%%%%%%%%%%%%%%%%%%%%%%%%%%%%%%%%%%%%%%%%%%%%%%%%%%%%%%%%

\begin{problem}
  In some terms when 6.042 is not taught in a TEAL room, students sit in a
  square arrangement during recitations.  An outbreak of beaver flu
  sometimes infects students in recitation; beaver flu is a rare variant
  of bird flu that lasts forever, with symptoms including a yearning for
  more quizzes and the thrill of late night problem set sessions.

  Here is an example of a $6 \times 6$ recitation arrangement with the
  locations of infected students marked with an asterisk.

\[
\begin{array}{|c|c|c|c|c|c|}
\hline
\ast& & & &\ast& \\ \hline
 &\ast& & & & \\ \hline
& &\ast&\ast& & \\ \hline
& & & & & \\ \hline
& &\ast& & & \\ \hline
& & &\ast& &\ast \\ \hline
\end{array}
\]

Outbreaks of infection spread rapidly step by step.  A student is infected
after a step if either

\begin{itemize}
\item the student was infected at the previous step (since beaver flu
  lasts forever), or

\item the student was adjacent to \textit{at least two} already-infected
  students at the previous step.

\end{itemize}
Here \textit{adjacent} means the students' individual squares share an
edge (front, back, left or right, but \emph{not} diagonal).  Thus, each
student is adjacent to 2, 3 or 4 others.

In the example, the infection spreads as shown below.
%
\[
\begin{array}{|c|c|c|c|c|c|}
\hline
\ast& & & &\ast& \\ \hline
 &\ast& & & & \\ \hline
& &\ast&\ast& & \\ \hline
& & & & & \\ \hline
& &\ast& & & \\ \hline
& & &\ast& &\ast \\ \hline
\end{array}
\Rightarrow
\begin{array}{|c|c|c|c|c|c|}
\hline
\ast&\ast& & &\ast& \\ \hline
\ast&\ast&\ast& & & \\ \hline
&\ast&\ast&\ast& & \\ \hline
& &\ast& & & \\ \hline
& &\ast&\ast& & \\ \hline
& &\ast&\ast&\ast&\ast \\ \hline
\end{array}
\Rightarrow
\begin{array}{|c|c|c|c|c|c|}
\hline
\ast&\ast&\ast& &\ast& \\ \hline
\ast&\ast&\ast&\ast& & \\ \hline
\ast&\ast&\ast&\ast& & \\ \hline
&\ast&\ast&\ast& & \\ \hline
& &\ast&\ast&\ast& \\ \hline
& &\ast&\ast&\ast&\ast \\ \hline
\end{array}
\]
%
In this example, over the next few time-steps, all the students in class
become infected.

\begin{theorem*}
  If fewer than $n$ students among those in an $n \times n$ arrangment are
  initially infected in a flu outbreak, then there will be at least one
  student who never gets infected in this outbreak, even if students
  attend all the lectures.
\end{theorem*}

Prove this theorem.

\hint Think of the state of an outbreak as an $n \times n$ square above,
with asterisks indicating infection.  The rules for the spread of
infection then define the transitions of a state machine.  Try to derive a
weakly decreasing state variable that leads to a proof of this theorem.

\begin{solution}
\begin{proof}
  Define the \term{perimeter} of an infected set of students to be the
  number of edges with infection on exactly one side.  Let $\nu$ be size
  (number of edges) in the perimeter.

  We claim that $\nu$ is a weakly decreasing variable.  This follows
  because the perimeter changes after a transition only because some
  squares became newly infected.  By the rules above, each newly-infected
  square is adjacent to at least two previously-infected squares.  Thus,
  for each newly-infected square, at least two edges are removed from the
  perimeter of the infected region, and at most two edges are added to
  the perimeter.  Therefore, the perimeter of the infected region cannot
  increase.

  Now if an $n \times n$ grid is completely infected, then the perimeter
  of the infected region is $4n$.  Thus, the whole grid can become
  infected only if the perimeter is initially at least $4n$.  Since each
  square has perimeter 4, at least $n$ squares must be infected initially
  for the whole grid to become infected.
\end{proof}

\end{solution}

\end{problem}

%%%%%%%%%%%%%%%%%%%%%%%%%%%%%%%%%%%%%%%%%%%%%%%%%%%%%%%%%%%%%%%%%%%%%
% Problem ends here
%%%%%%%%%%%%%%%%%%%%%%%%%%%%%%%%%%%%%%%%%%%%%%%%%%%%%%%%%%%%%%%%%%%%%

\endinput
