\documentclass[problem]{mcs}

\begin{pcomments}
  \pcomment{FP_MST_min_out_S16}
  \pcomment{subsumed by MQ_local_min_gray_edge}
  \pcomment{subsumes FP_MST_min_out}
  \pcomment{ARM 4/24/16, soln revised 11/2/17}
\end{pcomments}

\pkeywords{
  tree
  spanning_tree
  MST
  weight
  minimum
  cycle
}

%%%%%%%%%%%%%%%%%%%%%%%%%%%%%%%%%%%%%%%%%%%%%%%%%%%%%%%%%%%%%%%%%%%%%
% Problem starts here
%%%%%%%%%%%%%%%%%%%%%%%%%%%%%%%%%%%%%%%%%%%%%%%%%%%%%%%%%%%%%%%%%%%%%

\begin{problem}
Let $G$ be a connected weighted simple graph and let $v$ be a vertex
of $G$.  Suppose $e \eqdef \edge{v}{w}$ is an edge of $G$ that is
strictly smaller than the weight of every other edge incident to $v$.
Let $T$ be a minimum weight spanning tree of $G$.

Give a direct proof that $e$ is an edge of $T$ (without appeal to any
``gray edge'' argument).

\begin{solution}
Suppose to the contrary that $e$ is not in $T$.  There is a unique
path $\vec{p}$ in $T$ from $v$ to the other endpoint $w$ of $e$.  This
path must start with some edge $f \eqdef \edge{v}{u}$.

Now $\vec{p} + e$ is a cycle, so when we remove $f$, the endpoints of
$e$ remain connected by the rest of the cycle.  This means that $T - f
+ e$ remains connected.  Therefore $T-f+e$ is also a spanning tree of
$G$ since it has the same number of edges as $T$.

But the weight of $T - f + e$ is smaller than the weight than $T$,
since the weight of $e$ is smaller than the weight of $f$,
contradicting that fact that $T$ is a minimum weight spanning tree.
\end{solution}

\end{problem}

%%%%%%%%%%%%%%%%%%%%%%%%%%%%%%%%%%%%%%%%%%%%%%%%%%%%%%%%%%%%%%%%%%%%%
% Problem ends here
%%%%%%%%%%%%%%%%%%%%%%%%%%%%%%%%%%%%%%%%%%%%%%%%%%%%%%%%%%%%%%%%%%%%%
\endinput
