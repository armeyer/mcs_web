\documentclass[problem]{mcs}

\begin{pcomments}
  \pcomment{TP_gcd_linear_combination_wop}
  \pcomment{ARM 3/17/13}
\end{pcomments}

\pkeywords{
  gcd
  induction
  linear_combination
}

%%%%%%%%%%%%%%%%%%%%%%%%%%%%%%%%%%%%%%%%%%%%%%%%%%%%%%%%%%%%%%%%%%%%%
% Problem starts here
%%%%%%%%%%%%%%%%%%%%%%%%%%%%%%%%%%%%%%%%%%%%%%%%%%%%%%%%%%%%%%%%%%%%%

\begin{problem}
Use the Well Ordering Principle to prove that the gcd of a finite set
of integers is an integer linear combination of the numbers in the
set.  You may assume that the gcd of two integers is an integer linear
combination of them, which was proved \inhandout{in the
  text}\inbook{Theorem~\bref{gcd_is_lin_thm}}.  You may also assume
the easily verified fact that
\begin{equation}\label{gcdAua}
\gcd(A \union \set{a}) = \gcd(\gcd(A),a),
\end{equation}  
for any nonempty set $A$ of integers.

Be sure to define and clearly label the set of counterexamples you are
assuming is nonempty.

\examspace


\begin{solution}
Let
\begin{align*}
C \eqdef & \{n \geq 1 \suchthat \exists A \subset \integers.\, \card{A} = n\ \QAND\\
         & \QNOT(\gcd(A)\text{ is a linear combination of } a \in A\},
\end{align*}
and assume for the sake of contradiction that $C$ is not empty.

By the WOP, there is a least integer $m \in C$.  So there must be
integers $ a_1, a_2,\dots, a_{m}$ such that $\gcd(a_1,a_2,\dots, a_m)$ is
not a linear combination of $a_1,a_2,\dots,a_m$.

Since $\gcd(\set{a_1}) = 1 \cdot a_1$, we know that $m-1 \geq 1$.
Since $m$ is the smallest element of $C$, it follows that
\[
\gcd(a_1,\dots,a_{m-1}) = s_1a_1 + s_2a_2 + \cdots + s_{m-1}a_{m-1}.
\]
Now
\begin{align*}
\gcd(a_1,a_2,\dots,a_{m})
  & = \gcd(\gcd(a_1,a_2,\dots,a_{m-1}), a_{m})
       & \text{(by~\eqref{gcdAua})}\\
  & = s \cdot \gcd(a_1,a_2,\dots,a_{m-1}) + t \cdot a_{m} \text{ for some } s,t \in \integers
        &  \text{(the two element case)}\\
  & = s(s_1a_1 + s_2a_2 + \cdots + s_na_{m=1}) + t a_{m}\\
  & = (ss_1)a_1 + (ss_2)a_2 + \cdots + (ss_{m-1})a_{m-1} + t a_{m}.
\end{align*}
This shows that $\gcd(a_1,a_2,\dots,a_{m})$ is also a linear
combination of $a_1,a_2,\dots,a_{m}$, contradicting the choice of $m$.

The contradiction implies that $C$ must be empty, proving that the
claim holds for all $n \geq 1$.
\end{solution}
  
\end{problem}

%%%%%%%%%%%%%%%%%%%%%%%%%%%%%%%%%%%%%%%%%%%%%%%%%%%%%%%%%%%%%%%%%%%%%
% Problem ends here
%%%%%%%%%%%%%%%%%%%%%%%%%%%%%%%%%%%%%%%%%%%%%%%%%%%%%%%%%%%%%%%%%%%%%

\endinput
