\documentclass[problem]{mcs}

\begin{pcomments}
    \pcomment{TP_Flipping_coins}
    \pcomment{Converted from prob2.scm
              by scmtotex and dmj
              on Sun 13 Jun 2010 05:11:14 PM EDT}
    \pcomment{edited by drewe 14 july 2011}
\end{pcomments}

\begin{problem}

%% type: short-answer
%% title: Flipping coins

Suppose you flip a fair coin 100 times.  The coin flips are all
mutually independent.

\bparts

\ppart\label{expected}
What is the expected number of heads?

\exambox{0.7in}{0.4in}{0in}

\examspace[0.2in]

\begin{solution}
\textbf{50}.

Let $X$ denote the number of heads.  Let $X_{i}$ be the indicator
variable that is 1 if and only if the $i$th coin flip comes out heads (and 0 otherwise). Then
\begin{equation*}
    X = X_{1} + X_{2} + \dots + X_{100}.
\end{equation*}

Hence, by linearity of expectation, 
\[
\expect{X}
    = \expect{X_{1} + X_{2} + \cdots + X_{100}}
    = \expect{X_{1}} + \cdots + \expect{X_{100}}.
\]
The expectation of an indicator variable is the probability it
equals~1\inbook{ by Lemma~\bref{expindic}}.  Hence, $\expect{X_{i}} =
1/2$.

Putting everything together, we conclude that $\expect{X} = 100 \cdot
1/2 = 50$.
\end{solution}

\ppart What upper bound does Markov's Theorem give for the probability
that the number of heads is at least 70?

\exambox{0.7in}{0.4in}{0in}

\examspace[0.5in]

\begin{solution}
\textbf{5/7}.

The expected number of heads is 50.  So the probability that the
number of heads is at least 70 is at most $50/70 = 5/7$.
\end{solution}

\ppart
What is the variance of the number of heads?

\exambox{0.7in}{0.4in}{0in}

\examspace[0.75in]

\begin{solution}
\textbf{25}.

Let $X$ and $X_{i}$ be as in part ~\ref{expected}.  Then, by the independence of
the $X_{i}$, we know
\[
\Var[X] = 
\Var[X_{1} + \dots  + X_{100}] = 
\Var[X_{1}]+ \dots  + \Var[X_{100}].
\]
Since the variance of an indicator with expectation $p$ is
$p(1-p)$\inbook{ by Corollary~\bref{bernoulli-variance}}, we have
$\Var[X_{i}] = (1/2)(1 - 1/2) = 1/4$.  Therefore, 
\[
\Var[X] = 100 \cdot (1/4) = 25.
\]
\end{solution}

\ppart

What upper bound does Chebyshev's Theorem give for the probability
that the number of heads is either less than 30 or greater than 70?

\exambox{0.7in}{0.4in}{0in}

\begin{solution}
\textbf{1/16 = 0.0625}

The mean is 50.  So $X$ is less than 30 and more than 70 iff it
deviates from its mean by at least $20=\abs{30-50}=\abs{70-50}$.  How
big is this deviation in comparison to the \emph{standard deviation}?
Since the variance of the number of heads is 25, its standard
deviation is 5.  So 20 is 4 times the standard deviation.

Now, Chebyshev's Theorem says that the probability of the number of
heads deviating from its expected value by 4 times the standard
deviation is at most $1/(4^{2}) = 1/16$.
\end{solution}

\eparts

\end{problem}

\endinput
