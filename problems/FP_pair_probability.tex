\documentclass[problem]{mcs}

\begin{pcomments}
  \pcomment{FP_pair_probability}
  \pcomment{revised by ARM 5/4/16 from F14 final}
\end{pcomments}

\pkeywords{
  counting
  cards
  poker
  pair
  rank
  suit
  probability
}

%%%%%%%%%%%%%%%%%%%%%%%%%%%%%%%%%%%%%%%%%%%%%%%%%%%%%%%%%%%%%%%%%%%%%
% Problem starts here
%%%%%%%%%%%%%%%%%%%%%%%%%%%%%%%%%%%%%%%%%%%%%%%%%%%%%%%%%%%%%%%%%%%%%

\begin{problem}
Write a closed formula, possibly including factorials and binomial
coefficients, for the probability that a five card poker hand has just
a pair.  This means that exactly two cards share the same rank, so two
pairs or three of a kind would not count.  (The cards are drawn from a
standard 52-card deck where each card has one of thirteen \emph{ranks}
and one of four \emph{suits}.

\begin{solution}
The pair can have one of 13 ranks and any two out of four suits.  The
remaining three cards must have different ranks from among the
remaining 12 ranks.  Of these three, the smallest rank card may have
any one of four suits, and likewise for the middle and largest rank cards.
SO there are 
\[
13 \binom{4}{2} \binom{12}{3} 4^3
\]
hands with just a pair.  Therefore the probability of drawing one of these hands is
\[
\frac{13 \binom{4}{2} \binom{12}{3} 4^3}{\binom{52}{5}}
\]
\end{solution}
\end{problem}

%%%%%%%%%%%%%%%%%%%%%%%%%%%%%%%%%%%%%%%%%%%%%%%%%%%%%%%%%%%%%%%%%%%%%
% Problem ends here
%%%%%%%%%%%%%%%%%%%%%%%%%%%%%%%%%%%%%%%%%%%%%%%%%%%%%%%%%%%%%%%%%%%%%

\endinput
