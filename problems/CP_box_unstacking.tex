\documentclass[problem]{mcs}

\begin{pcomments}
  \pcomment{CP_box_unstacking}
  \pcomment{from: S09 notes}
\end{pcomments}

\pkeywords{
  induction
  ordinary_induction
  potential
  invariance
}

%%%%%%%%%%%%%%%%%%%%%%%%%%%%%%%%%%%%%%%%%%%%%%%%%%%%%%%%%%%%%%%%%%%%%
% Problem starts here
%%%%%%%%%%%%%%%%%%%%%%%%%%%%%%%%%%%%%%%%%%%%%%%%%%%%%%%%%%%%%%%%%%%%%

\begin{problem}

\inhandout{

The Block Stacking Game\footnote{Excerpted from~\cite{Lehman2015},
  Section~\bref{block-stack-subsection}.} goes
as follows: You begin with a stack of $n$ boxes.  Then you make a
sequence of moves.  In each move, you divide one stack of boxes into
two nonempty stacks.  The game ends when you have $n$ stacks, each
containing a single box.  You earn points for each move; in
particular, if you divide one stack of height $a + b$ into two stacks
with heights $a$ and $b$, then you score $ab$ points for that move.
Your overall score is the sum of the points that you earn for each
move.  What strategy should you use to maximize your total score?

As an example, suppose that we begin with a stack of $n = 10$ boxes.
Then the game might proceed as shown in Figure~\ref{fig:stacking-10}.
%Can you find a better strategy?
%
\begin{figure}\redrawntrue
\[
\begin{array}{cccccccccccl}
\multicolumn{10}{c}{\textbf{Stack Heights}} & \quad & \textbf{Score} \\
\underline{10}&&&&&&&&& && \\
5&\underline{5}&&&&&&&& && 25 \text{ points} \\
\underline{5}&3&2&&&&&&& && 6 \\
\underline{4}&3&2&1&&&&&& && 4 \\
2&\underline{3}&2&1&2&&&&& && 4 \\
\underline{2}&2&2&1&2&1&&&& && 2 \\
1&\underline{2}&2&1&2&1&1&&& && 1 \\
1&1&\underline{2}&1&2&1&1&1&& && 1 \\
1&1&1&1&\underline{2}&1&1&1&1& && 1 \\
1&1&1&1&1&1&1&1&1&1 && 1 \\ \hline
\multicolumn{10}{r}{\textbf{Total Score}} & = & 45 \text{ points}
\end{array}
\]
\caption{An example of the stacking game with $n = 10$ boxes.  On each
line, the underlined stack is divided in the next step.}
\label{fig:stacking-10}
\end{figure}
}

Define the \emph{potential}, $p(S)$, of a stack of blocks, $S$, to be
$k(k-1)/2$ where $k$ is the number of blocks in $S$.  Define the
potential, $p(A)$, of a set of stacks, $A$, to be the sum of the
potentials of the stacks in $A$.

\inbook{Generalize Theorem~\bref{stacking} about scores in the stacking game to
s}\inhandout{S}how that for any set of stacks, $A$, if a sequence of moves starting
with $A$ leads to another set of stacks, $B$, then $p(A) \geq p(B)$, and
the score for this sequence of moves is $p(A)-p(B)$.

\hint Try induction on the number of moves to get from $A$ to $B$.

\begin{solution}
\begin{proof}
The proof is by ordinary induction on the number of moves, $n$.  The
induction hypothesis will be
\begin{quote}
$P(n) \eqdef$ If $n$ moves from a set of stacks, $A$, leads to a set
  $B$ of stacks, then $p(A) \geq p(B)$ and the score for these $n$
  moves is $p(A)-p(B)$.
\end{quote}

\inductioncase{Base case:} ($n$ = 0) This means no moves have been made and
$B=A$, so it's obvious that $P(0)$ holds.

\inductioncase{Inductive step:} Assume that $P(n)$ is true for some $n \in
\naturals$, and suppose $A$ leads to $B$ in $n+1$ moves.  This means
that $A$ leads to some set of stacks, $A_1$, and $A_1$ leads to $B$ in
$n$ steps.  So the inductive hypothesis $P(n)$ implies that $p(A_1)
\geq p(B)$ and the score for going from $A_1$ to $B$ is $p(A_1)-p(B)$.

So all we have to do is show that the score for the single move from
$A$ to $A_1$ is $p(A)-p(A_1) > 0$.  The only difference between $A$
and $A_1$ is that some stack $S \in A$ of size $k>1$ splits into two
stacks of sizes $k_1,k_2 \geq 1$ where $k = k_1+k_2$. The score for
such a move is $k_1k_2$.  Also,
\[
p(S) = \frac{(k_1+k_2)((k_1+k_2)-1)}{2} =
\frac{(k_1^2+ 2k_1k_2+k_2^2)-(k_1 + k_2)}{2},
\]
and the potential of the two stack sets is the sum of their potentials, namely,
\[
\frac{k_1(k_1-1) + k_2(k_2-1)}{2} = \frac{k_1^2 + k_2^2 -(k_1 + k_2)}{2},
\]
So the difference between these potentials equals $k_1k_2>0$, and this is
indeed equal to the score of the move.

\end{proof}

\end{solution}

\end{problem}


%%%%%%%%%%%%%%%%%%%%%%%%%%%%%%%%%%%%%%%%%%%%%%%%%%%%%%%%%%%%%%%%%%%%%
% Problem ends here
%%%%%%%%%%%%%%%%%%%%%%%%%%%%%%%%%%%%%%%%%%%%%%%%%%%%%%%%%%%%%%%%%%%%%

\endinput
