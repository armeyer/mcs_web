\documentclass[problem]{mcs}

\begin{pcomments}
  \pcomment{PS_strongly_connected}
  \pcomment{ARM 4/4/17}
  \pcomment{abstract path properties; needs connection ot Computer Science}
\end{pcomments}

\pkeywords{
  digraph
  connected
  dag
  strongly_connected
  }

\newcommand{\mutcon}{\ensuremath{\mrel{\overset{*}{\longleftrightarrow}}}}
\newcommand{\reach}{\ensuremath{\mrel{\rightsquigarrow}}}

%%%%%%%%%%%%%%%%%%%%%%%%%%%%%%%%%%%%%%%%%%%%%%%%%%%%%%%%%%%%%%%%%%%%%
% Problem starts here
%%%%%%%%%%%%%%%%%%%%%%%%%%%%%%%%%%%%%%%%%%%%%%%%%%%%%%%%%%%%%%%%%%%%%
\begin{problem}
Say that vertices $u,v$ in a digraph $G$ are \emph{mutually connected}
and write
\[
u \mutcon v,
\]
when there is a path from $u$ to $v$ and also a path from $v$ to $u$.

\bparts

\ppart Prove that \mutcon\ is an equivalence relation on
$\vertices{G}$.

\examspace[3.0in]

\begin{solution}
To show that it is an equivalence relation, we must show that $u \mutcon v$ is reflexive, symmetric, and transitive. \\

Reflexive: There is a path from $u$ to itself, trivially, so $u \mutcon u$. \\
Symmetric: $u \mutcon v$ means there is a path from $u$ to $v$, and a path from $v$ to $u$, meaning $v \mutcon u$ also holds. \\
Transitive: We need to show if $u \mutcon v$ and $v \mutcon w$, then $u \mutcon w$. So we need to show there is a path from $u$ to $w$ and a path from $w$ to $u$. There is a path from $u$ to $w$ by taking the given path from $u$ to $v$, then from $v$ to $w$. Similarly, the path from $w$ to $u$ is from taking the given path from $w$ to $v$ and $v$ to $u$. Of note, since paths should not take have duplicate nodes, we if any node is duplicated when concatenating these paths, we can just delete the cycle to remove the duplicate.
\end{solution}

\ppart The blocks of the equivalence relation \mutcon\ are called the
\emph{strongly connected components} of $G$.  Define a relation
\reach\ on the strongly connected components of $G$ by the rule
\[
C \reach D\quad \QIFF \text{ there is a path from some vertex in $C$
  to some vertex in $D$}.
\]

Prove that \reach\ is a weak partial order on the strongly connected
components.

\begin{solution}
To show that is a weak partial order, we must show that \reach\ is reflexive, antisymmetric, and transitive. \\

Reflexive: There is a trivial path from $C$ to itself, so $C \reach C$.\\
Antisymmetric: Since $C$ and $D$ are two different connected components and $C \reach D$ means there is a path from some vertex in $C$ to some vertex in $D$, here cannot be a path from some vertex in $D$ to some vertex in $C$, otherwise $C$ and $D$ would actually be the same connected components. Thus $D \reach C$ cannot hold, and \reach\ is antisymmetric.\\
Transitive: If $C \reach D$ and $D \reach E$, then to get from some vertex in $C$ to some vertex in $E$, we take the given path from some vertex in $C$ to a vertex in $D$, then travel through $D$ to the vertex that can then connect $E$. This travel through $D$ is always possible because $D$ is strongly connected.    
\end{solution}

\eparts
\end{problem}


%%%%%%%%%%%%%%%%%%%%%%%%%%%%%%%%%%%%%%%%%%%%%%%%%%%%%%%%%%%%%%%%%%%%%
% Problem ends here
%%%%%%%%%%%%%%%%%%%%%%%%%%%%%%%%%%%%%%%%%%%%%%%%%%%%%%%%%%%%%%%%%%%%%

\endinput
