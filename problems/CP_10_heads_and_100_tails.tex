\documentclass[problem]{mcs}

\begin{pcomments}
  \pcomment{CP_10_heads_and_100_tails}
  \pcomment{from S10}
  \pcomment{require n>0 ARM 10/4/13}
\end{pcomments}

\pkeywords{
  state_machines
  derived_variable
  increasing
  decreasing
}

%%%%%%%%%%%%%%%%%%%%%%%%%%%%%%%%%%%%%%%%%%%%%%%%%%%%%%%%%%%%%%%%%%%%%
% Problem starts here
%%%%%%%%%%%%%%%%%%%%%%%%%%%%%%%%%%%%%%%%%%%%%%%%%%%%%%%%%%%%%%%%%%%%%

\begin{problem}
Start with 110 coins on a table, 10 showing heads and 100 showing tails.

There are two ways to change the coins:

\begin{enumerate}
\item[(i)] Remove $20$ coins from the table, $10$ of which must be 
  \emph{heads} and the other $10$ must be \emph{tails}, or
\item[(ii)] Let $n>0$ be the number of heads showing.  If there are 
  \emph{more tails than heads} on the table, place $n$ additional coins,
  all showing heads, on the table.
\end{enumerate}

\bparts

\ppart
Model this situation as a state machine, carefully defining the set of
states, the start state and the possible state transitions.
\hint Be sure to state the conditions of the state transitions.
\begin{solution}
States are tuples of the form $(H, T)$ where $H,T \in \naturals$.  The
start state is $(10, 100)$.  The transitions are of the form $(H, T)
\rightarrow (2H, T)$ with the restriction of $T > H > 0$, and $(H, T)
\rightarrow (H-10, T-10)$ with the restriction of $H,T \ge 10$.
\end{solution}

\examspace[2.0in]

\ppart Let $H \eqdef \text{the number of heads}$ and $T \eqdef
\text{the number of tails}$.  For each of the derived variables below,
indicate the \emph{strongest} of the following properties that it
satisfies: \emph{constant} \textbf{C}, \emph{strictly increasing}
\textbf{Sinc}, \emph{strictly decreasing} \textbf{Sdec}, \emph{weakly
  increasing} \textbf{Winc}, \emph{weakly decreasing} \textbf{Wdec},
\emph{none of these} \textbf{N}.

% \renewcommand{\labelenumi}{(\roman{enumi})}
\begin{enumerate}
\item $T$ \hfill\examrule[0.5in]
\item $H+T$ \hfill\examrule[0.5in]
\item $T-H$ \hfill\examrule[0.5in]
\item $2T-H$ \hfill\examrule[0.5in]
\end{enumerate}

\begin{solution}
\begin{enumerate}
\item $T$: weakly decreasing
\item $H+T$: none
\item $T-H$: weakly decreasing
\item $2T-H$: strictly decreasing
\end{enumerate}
\end{solution}


% 
% \ppart
% Prove that any long enough sequence of transitions will arrive at a 
% state in which no transition is possible: by providing a strictly decreasing
% derived variable and proving that it has a minimum value.
% 
% 
% \begin{solution}
% TBA
% \end{solution}
% 
\eparts
\end{problem}

%%%%%%%%%%%%%%%%%%%%%%%%%%%%%%%%%%%%%%%%%%%%%%%%%%%%%%%%%%%%%%%%%%%%%
% Problem ends here
%%%%%%%%%%%%%%%%%%%%%%%%%%%%%%%%%%%%%%%%%%%%%%%%%%%%%%%%%%%%%%%%%%%%%

\endinput
