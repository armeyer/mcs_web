\documentclass[problem]{mcs}

\begin{pcomments}
  \pcomment{TP_pigeonhole_setsum}
  \pcomment{Stimulated by suggested of R.C.Spence, May, 2016}
\end{pcomments}

\pkeywords{
  pigeonhole
  subset
}

%%%%%%%%%%%%%%%%%%%%%%%%%%%%%%%%%%%%%%%%%%%%%%%%%%%%%%%%%%%%%%%%%%%%%
% Problem starts here
%%%%%%%%%%%%%%%%%%%%%%%%%%%%%%%%%%%%%%%%%%%%%%%%%%%%%%%%%%%%%%%%%%%%%
\begin{problem}
Let
\[
k_1, k_2, \dots, k_{101}
\]
be a sequence of 101 integers.  A sequence
\[
k_{m+1}, k_{m+2}, \dots, k_{n}
\]
where $0 \leq m < n\leq 101$ is called a \emph{subsequence}.  Prove
that there is a subsequence whose elements sum to a number divisible
by 100.

\begin{solution}
A length-$m$ \emph{prefix} is an initial subsequence of the form
\[
k_1, k_2, \dots, k_{m}
\]
for $m \in \Zintv{1}{101}$.  Map each prefix to the remainder on
division by 100 of the sum of its elements.  The 101 prefixes are the
``pigeons,'' and the 100 possible remainders are the ``pigeonholes.''
There must be a length-$m$ and a length-$n$ prefix in the same
pigeonhole, where $m<n$.  This implies that 100 divides the difference
of their sums, which is equal to the sum of the elements
\[
k_{m+1}, k_{m+2}, \dots, k_{n}.
\]
\end{solution}
\end{problem}

\endinput
