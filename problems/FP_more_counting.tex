\documentclass[problem]{mcs}

\begin{pcomments}
  \pcomment{FP_more_counting}
  \pcomment{from: F03.final, prob5; F01.final}
  \pcomment{Adapted Steven F09}
\end{pcomments}

\pkeywords{
  counting
  combinatorial_proof
  inclusion-exclusion
  binomial_coefficient
  factorial
  bijection
}

%%%%%%%%%%%%%%%%%%%%%%%%%%%%%%%%%%%%%%%%%%%%%%%%%%%%%%%%%%%%%%%%%%%%%
% Problem starts here
%%%%%%%%%%%%%%%%%%%%%%%%%%%%%%%%%%%%%%%%%%%%%%%%%%%%%%%%%%%%%%%%%%%%%

\begin{problem}
Solve the following counting problems.  You may leave factorials and
binomial coefficients in your solutions.  No explanation is required.

\bparts
\iffalse

\ppart
Suppose that two identical 52-card decks of cards are mixed
together.  In how many ways can the cards in this double-size deck be
arranged?
[\vspace{3in}]

\begin{solution}
\[
\frac{(52 \cdot 2)!}{2^{52}}
\]
\end{solution}

\ppart
How many different solutions are there to the following
equation, if all variables must be \textbf{even}, nonnegative integers?
\[
y_1 + y_2 + y_3 + y_4 + y_5  =  1000.
\]
[\vspace{3in}]

\begin{solution}
There is a bijection between solutions to this
equation and ways of assigning 500 identical balls to 5
distinguishable bins.  (If $k$ balls are assigned to bin $i$, then
$y_i = 2k$.)  Therefore, the number of solutions is:

\[
\binom{500 + 4}{4} = \binom{500 + 4}{500} = \frac{(500 + 4)!}{4!\ 500!}
\]
\end{solution}

%\instatements{\newpage}

\ppart
Theory Hippo wants to travel from point (0, 0) to point (20,
20) using only steps that either increment the first coordinate by 1
or else increment the second coordinate by 1.  How many different
paths can he choose from?
[\vspace{3in}]

\begin{solution}He must choose 20 of the 40 steps of the first
type and the remainder are of the second type.  Therefore, the number
of different paths is:

\[
\binom{40}{20} = \frac{40!}{20!\ 20!}
\]
\end{solution}
\fi

\ppart Theory Hippo wants to travel from point (0, 0) to point (20, 20)
using only steps that add one to the $x$ or the $y$ coordinate.  There are
Bottomless Pits of Utter Annihilation at points (5, 5) and (10, 10)
through which paths cannot pass.  How many paths are there for Theory
Hippo to follow?  \hint Ignoring the pits, let $N_5$ [\dots $N_{10}$] be
the set of paths from (0, 0) to (20, 20) that go through (5,5) [\dots
(10,10)].

\examspace[0.7in]

\begin{solution}
\[
\binom{40}{20} - 
\binom{10}{5} \cdot \binom{30}{15} - \binom{20}{10} \cdot \binom{20}{10}
+ \binom{10}{5} \cdot \binom{10}{5} \cdot \binom{20}{10}.
\]

The paths that are blocked by some pit is $N_5 \union N_{10}$.  By
Inclusion-Exclusion,
\begin{align}
\card{N_5 \union N_{10}}
   & = \card{N_5} + \card{N_{10}} - \card{N_5 \intersect N_{10}}\notag\\
   & = \binom{10}{5} \cdot \binom{30}{15} + \binom{20}{10} \cdot \binom{20}{10}
- \binom{10}{5} \cdot \binom{10}{5} \cdot \binom{20}{10}.\label{bin10-5}
\end{align}

Consequently, the number of paths that do not cross a Bottomless Pit of
Utter Annihilation is the number, $\binom{40}{20}$, of possible paths
minus~\eqref{bin10-5}.

\end{solution}

%\instatements{\newpage}

\ppart
Below is a combinatorial proof of an equation.  Fill in the
empty boxes in the Theorem statement with the proper expressions.

\begin{theorem*}
\begin{eqnarray*}
\fbox{\begin{minipage}{2.5in}
\vspace{1in}
\hspace{1in}
\end{minipage}}
& = &
\fbox{\begin{minipage}{2.5in}
\vspace{1in}
\hspace{1in}
\end{minipage}}
\end{eqnarray*}
\end{theorem*}

\begin{proof}
Stinky Peterson owns $n$ newts, $t$ toads, and $s$ slugs.
Conveniently, he lives in a dorm with $n + t + s$ other students.
(The students are distinguishable, but creatures of the same variety
are not distinguishable.)  Stinky wants to put one creature in each
neighbor's bed.  Let $W$ be the set of all ways in which this can be
done.

On one hand, he could first determine who gets the slugs.  Then, he
could decide who among his remaining neighbors has earned a toad.
Therefore, $\card{W}$ is equal to the expression on the left.

On the other hand, Stinky could first decide which people deserve
newts and slugs and then, from among those, determine who truly merits
a newt.  This shows that $\card{W}$ is equal to the expression on the
right.

Since both expressions are equal to $\card{W}$, they must be equal to each
other.
\end{proof}

\begin{solution}
\begin{eqnarray*}
\binom{n + t + s}{s} \cdot \binom{n + t}{t}
& = &
\binom{n + t + s}{n + s} \cdot \binom{n + s}{n}
\end{eqnarray*}
\end{solution}

\eparts
\end{problem}

%%%%%%%%%%%%%%%%%%%%%%%%%%%%%%%%%%%%%%%%%%%%%%%%%%%%%%%%%%%%%%%%%%%%%
% Problem ends here
%%%%%%%%%%%%%%%%%%%%%%%%%%%%%%%%%%%%%%%%%%%%%%%%%%%%%%%%%%%%%%%%%%%%%

\endinput
