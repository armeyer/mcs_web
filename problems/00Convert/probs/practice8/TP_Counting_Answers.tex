\documentclass[problem]{mcs}

\begin{pcomments}
    \pcomment{TP_Counting_Answers}
    \pcomment{Converted from prob5.scm
              by scmtotex and dmj
              on Sun 13 Jun 2010 03:52:26 PM EDT}
\end{pcomments}

\begin{problem}

%% type: short-answer
%% title: Counting Answers

In how many different ways is it possible to answer next week's
practice problems if:
     
\begin{itemize}
     
\item
the first problem has four \emph{true}/\emph{false} questions,
     
\item
the second problem requires choosing one of four alternatives,
     
\item
the answer to the third problem was an integer $\ge 15$ and $\le 20$?
     
\end{itemize}

%% You may answer with a formula such as \textbf{(3*7)^2/(3 + 7)!}

\begin{solution}
384

Each ``way'' to answer the practice problems can be represented as a
6-long sequence in which: the first four elements are true/false
answers, the 5th element is in $\{1,2,3,4\}$ indicating which
alternative was chosen, and the 6th element is in
$\{15,16,17,18,19,20\}$.  By the Product Rule, there are $2^{4} * 4 *
6$ such sequences, for a total of 384 ways to answer.
\end{solution}

\end{problem}

\endinput
