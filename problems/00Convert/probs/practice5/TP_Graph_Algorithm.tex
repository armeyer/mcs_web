\documentclass[problem]{mcs}

\begin{pcomments}
    \pcomment{Converted from prob5-short.scm
              by scmtotex and dmj
              on Sun 13 Jun 2010 11:22:25 AM EDT}
\end{pcomments}

\begin{problem}

%% type: multi-part
%% title: Graph Algorithm

The algorithm \textbf{Mark} starts with a simple undirected graph,
$G$, with a finite set of vertices, $V$, and a set of \emph{one or
  more} edges, $E$.  Initially, all edges are unmarked.  Then it
proceeds to mark some of the edges, by repeatedly performing the
following steps until no further step is possible:

\begin{enumerate}

\item
Choose any unmarked edge $e \in E$ such that there is currently no
path of marked edges between the endpoints of $e$.

\item
Mark edge $e$.

\end{enumerate}

\bparts

\ppart
%% type: short-answer
%% title: Preserved Invariants

Which of the following predicates are \textbf{both} \emph{preserved
  invariants} \textbf{and} also hold for the start state?

\begin{enumerate}
  
\item There is always an edge that has not been marked.
  
\item The marked edges form an acyclic graph.
  
\item The marked edges form a tree.
  
\item There is always a vertex not touching a marked edge.

\end{enumerate}

\begin{solution}
2

\begin{enumerate}

\item
This is true for the start state (since, by definition, there is at
least one edge), but is \textbf{not} preserved.
  
\item
This is a preserved invariant. It also holds vacuously for the start
state, since it has no marked edges.
  
\item
This is true vacuously for the start state, but is \textbf{not}
preserved.
  
\item
This is true for the start state, but is \textbf{not} preserved.

\end{enumerate}

\end{solution}

\ppart
%% type: short-answer
%% title: Derived Variables

Please choose the property that best describes each of the following
derived variables.


Answer with the strongest applicable property, that is, for a variable
that is constant the answer should be ``constant'', even though it is
also necessarily both weakly increasing and weakly decreasing.

\begin{enumerate}

\item
\#unmarked edges

\begin{solution}
strictly decreasing

In every iteration, the number of unmarked edges
decreases by 1.
\end{solution}

\item
\#marked edges

\begin{solution}
strictly increasing

In every iteration, the number of marked edges increases by~1.
\end{solution}

\item
\#unmarked edges + \#marked edges

\begin{solution}
constant

In every iteration, the number of marked edges increases by~1 and the
number of unmarked edges decreases by~1. So, their sum remains
constant, equal to the original number of edges in the graph.
\end{solution}

\item
\#marked edges - \#unmarked edges

\begin{solution}
strictly increasing

In every iteration, the number of marked edges increases by~1 and the
number of unmarked edges decreases by~1. So, their difference
increases by~2.
\end{solution}

\item
\#connected components in the graph $G'$ with vertices $V$ and edges
$M$, where $M$ is the set of marked edges

\begin{solution}
strictly decreasing

Each new marked edge connects two vertices that previously had no path
between them in~$G'$. So, the vertices belonged to different
components of~$G'$, and the new marked edge merges these components
into one. As a result, the number of components in~$G'$ decreases by
1.
\end{solution}

\end{enumerate}

\eparts

\end{problem}

\endinput
