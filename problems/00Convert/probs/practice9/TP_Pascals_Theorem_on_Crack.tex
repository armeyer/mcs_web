\documentclass[problem]{mcs}

\begin{pcomments}
    \pcomment{TP_Pascals_Theorem_on_Crack}
    \pcomment{Converted from repeated-pascal.scm
              by scmtotex and dmj
              on Sun 13 Jun 2010 04:04:28 PM EDT}
\end{pcomments}

\begin{problem}

%% type: short-answer
%% title: Pascal's Theorem on Crack

Pascal's Theorem gives us that ${{n-1} \choose {k}} + {{n-1} \choose
  {k-1}} = {{n} \choose {k}}$. We will try and apply this result
repeatedly to get a cool identity. For each of the parts, your answer
should be a single number corresponding to the appropriate expression
among the list of choices for that part.  

\bparts

\ppart

As a start, what is ${{n} \choose {n}}$? How about ${{n+1} \choose {n+1}}$?

\begin{solution}

1.

There is only one way to choose \emph{all} the options.
\end{solution}

\ppart
Show that ${{n} \choose {n}} +
{{n+1} \choose n} = {{n+2} \choose {n+1}}$.
      
\hint{Make a substitution and then apply Pascal's Theorem.}
      
\begin{solution}
${{n} \choose {n}} + {{n+1} \choose {n}} = {{n+1}
  \choose {n+1}} + {{n+1} \choose {n}} = {{n+2} \choose {n+1}}$.
\end{solution}

\ppart
Show that ${{n} \choose {n}}
+ {{n+1} \choose {n}} + {{n+2} \choose {n}} = {{n+3} \choose {n+1}}$.
      
\hint{Same hint as the previous part.}

\begin{solution}
${{n} \choose {n}} + {{n+1} \choose {n}} + {{n+2}
  \choose {n}} = {{n+2} \choose {n+1}} + {{n+2} \choose {n}} = {{n+3}
  \choose {n+1}}$.
\end{solution}

\ppart
Now, show that ${{n} \choose {n}} + {{n+1} \choose {n}} + {{n+2}
  \choose {n}} + \cdots + {{n+k} \choose {n}} = {{n+k+1} \choose {n+1}}$

\begin{solution}

We can do this by following the same steps as in the previous part:

${{n} \choose {n}} + {{n+1} \choose {n}} + {{n+2}
  \choose {n}} + \cdots + {{n+k} \choose {n}} = {{n+k+1} \choose {n+1}} + \cdots + {{n+2} \choose {n+1}} + {{n+2} \choose {n}} = {{n+k+1}
  \choose {n+1}}$ .

If you would like to be rigorous, try and prove
this result using induction on $k$ or come up with an elegant
combinatorial argument. 

Here is a not-so-elegant combinatorial
argument but it is still important to understand: The number of
solutions to the equation $x_1 + x_2 + \cdots + x_k + x_{k+1} = n+1$
is ${{n+k+1} \choose {n+1}}$ as there is a bijection between solutions
to this equation and bit sequenes of length $(n+k+1)$ with exactly
$(n+1)$ zeros.

Now, let's count these solutions in another way:

The number of solutions to $x_1 = n+1$ is ${{n+1} \choose {n+1}} =
{{n} \choose {n}}$.

The number of solutions to $x_1 + x_2 = n+1$ with $x_2 \ge 1$ is
${{n+1} \choose {n}}$. At least 1 zero must follow the last one to
ensure that $x_{2} >= 1$. We can group the last one with a zero to form
10. Thus, we need to pick the positions of $n$ remaining zeros (out
of $(n+1)$ possible positions) and insert the 10 in the remaining
position.

The number of solutions to $x_1 + x_2 \ldots x_k + x_{k+1} = n+1$ with
$x_{k+1} \ge 1$ is ${{n+k} \choose {n}}$. This is because at least 1
zero must follow the last one and so, we can group them
together. Thus, we need to pick the positions of $n$ remaining zeros
(out of $(n+k)$ possible positions). We then insert a one in the empty
positions except for the last empty position where we insert a 10.

Note that we have not double counted anywhere and have accounted for
all the solutions to the original equation by requiring that the last
of the $x_{i}$'s be non-zero in each of the equations. Counting the
solutions in these two ways yields the identity that we proved using
Pascal's Theorem.
\end{solution}

\eparts

\end{problem}

\endinput
