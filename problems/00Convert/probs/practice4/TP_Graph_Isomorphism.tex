\documentclass[problem]{mcs}

\begin{pcomments}
    \pcomment{Converted from prob3.scm
              by scmtotex and dmj
              on Sun 13 Jun 2010 10:52:29 AM EDT}
\end{pcomments}

\begin{problem}

%% type: multi-part
%% title: Graph Isomorphism

\bparts

\ppart
%% type: short-answer
%% title: Isomorphic Graphs

\includegraphics{prob3a1}

The two graphs above are isomorphic.  This means that there exists an
edge-preserving bijection from the set of vertices of the first graph
to the set of vertices of the second graph.

How many such bijections are there?

\begin{solution}
10

We may map $u_{1}$ to any of the 5 vertices of the second graph, and
then $u_{2}$ to either of its 2 neighbors.  The remaining vertices are
then uniquely determined.
\end{solution}

\ppart

%% type: short-answer
%% title: Non-isomorphic Graphs

\includegraphics{prob3b1}

The two graphs above (call them $G$ and $H$) are not isomorphic.

Indicate which of the following assertions can prove this fact.

\begin{enumerate}

\item
$H$ has a vertex of degree~4, but $G$ does not.

\item
$u_{1}$ has degree~3, but $v_{1}$ has degree~2.

\item
$G$ has four vertices of degree~3, but $H$ only has two vertices of
degree~3.

\item
$G$ has a vertex named $u_{1}$, but $H$ does not.

\item
$G$ has three vertices that form a right triangle, but no three
  vertices of $H$ form a right triangle.

\item
$G$ has an edge that is incident to two vertices of degree~3, but $H$
  does not.

\end{enumerate}

\begin{solution}

1
3


Note that assertion 6 mentions properties that \emph{are} preserved
under isomorphism, and so if it were true, it could prove that $G$
and~$H$ are not isomorphic.  Unfortunately, the assertion isn't true;
both graphs actually have an edge that is incident to two vertices of
degree~3.
\end{solution}

%% Answer with a sequence of integers separated by some
%% spaces, for example, 
%% \begin{equation*}
%% 4 3 2 
%% \end{equation*}
%%  Don't use commas or
%% periods.
%% 

\eparts

\end{problem}

\endinput
