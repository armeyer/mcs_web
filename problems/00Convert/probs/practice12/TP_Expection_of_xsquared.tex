\documentclass[problem]{mcs}

\begin{pcomments}
    \pcomment{TP_Expection_of_xsquared}
    \pcomment{Converted from uniform-mean-var.scm
              by scmtotex and dmj
              on Sun 13 Jun 2010 05:11:14 PM EDT}
\end{pcomments}

\begin{problem}

%% type: short-answer
%% title: Expection of <em>X</em><sup>2</sup>

Let $X$ be a random variable with uniform distribution over the
integers from $-n$ to $n$.  Let $Y ::= X^{2}$.  Which of the following
are true?

\begin{enumerate}
 
\item $\Ex[X] = 0$
 
\item $\Ex[Y] = 0$
 
\item $\Ex[Y] > \Ex[X]$
 
\item $\Ex[X+Y] = \Ex[X] + \Ex[Y]$
 
\item $\Ex[XY] = \Ex[X] \Ex[Y]$
 
\item $X$, $Y$ are independent variables

\item $\Var[X] = 0$
 
\item $\Var[Y] = 0$

\end{enumerate}

% Give you answer as a sequence of numbers separated by spaces (e.g.,
% ``6 2 4 ``).

\begin{solution}

1

3

4

5

1. Since the \pdf\ of $X$ is symmetric around 0, we know the mean has
to be 0.

2. All values of $Y$ are nonnegative and some of them are actually
positive with non-zero probability. So, there is no way for the mean
to be~0. It has to be some positive value.

3. Obvious, from our answers to 1 and 2. 

4. True by linearity of expectation, no matter what $X$ and $Y$  
we are actually talking about. 

5. This is tricky.  The equation does not hold in general for
non-independent $X$ and $Y$. However, in this particular case, it
happens to hold.  To see this, note that the right hand side is~0, by
our answer to~1. At the same time, the variable $XY$ is simply
$X^{3}$. So, its \pdf\ is symmetric around~0 (the same way the
\pdf\ of~$X$ is).  Therefore, its mean has to be~0, as well.

6. This is false. For example, without any knowledge about $X$, there
is some non-zero probability for $Y$ to be $n^{2}$.  However, if we
know that $X=0$, then the probability that $Y=n^{2}$ becomes~0.

7.  Variance is zero only for for random variables that are constant.
$Y$ is not constant.
\end{solution}

\end{problem}

\endinput
