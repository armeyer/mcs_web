\documentclass[problem]{mcs}

\begin{pcomments}
  \pcomment{FP_recitation_coloring}
  \pcomment{variant of PS_TA_recitation_graph_coloring}
  \pcomment{from: S09.ps6, S06.ps4, S04.ps4}
\end{pcomments}

\pkeywords{
  graph_coloring
  scheduling
}

%%%%%%%%%%%%%%%%%%%%%%%%%%%%%%%%%%%%%%%%%%%%%%%%%%%%%%%%%%%%%%%%%%%%%
% Problem starts here
%%%%%%%%%%%%%%%%%%%%%%%%%%%%%%%%%%%%%%%%%%%%%%%%%%%%%%%%%%%%%%%%%%%%%

\begin{problem}
Math for Computer Science is often taught using recitations.  Suppose
it happens that six recitations are needed.  The assignment of
available staff to recitation sections is as follows:

%\begin{itemize}
%\item R1:  Eli, Rong, Eric \\
%\item R2:  Eli, Megumi, Albert\\
%\item R3:  Rong, Jay\\
%\item R4:  Chuck, Megumi, Eric\\
%\item R5:  Chuck, William, Albert\\
%\item R6:  William, Jay\\
%\item R7:  William, Megumi\\
%\item R8:  Rong, Jay, Albert
%\end{itemize}
\begin{itemize}
\item R1:  Suzy, Peppa, Rebecca \\
\item R2:  Peppa, George \\
\item R3:  George, Richard \\
\item R4:  Richard, Rebecca, George \\
\item R5:  Rebecca, Pedro, Peppa \\
\item R6:  Pedro, Suzy
\end{itemize}

Two recitations cannot be held in the same one-hour time slot if some
staff member is assigned to both recitations.  The problem is to
determine the minimum number of time slots required to complete all
the recitations.

\bparts

\ppart Recast this problem as a question about coloring the
vertices of a particular graph.  Draw the graph and explain what the
vertices, edges, and colors represent.

\examspace[2.5in]

\begin{solution}
Each vertex in the graph below represents a recitation
section.  An edge connects two vertices if the corresponding
recitation sections share a staff member and thus can not be scheduled
at the same time.  The color of a vertex indicates the time slot of
the corresponding recitation.

\begin{figure}[h]
\graphic[height=2in]{rec_sched_Peppa}
\end{figure}

\end{solution}

\ppart Give a coloring of this graph using the fewest possible colors,
and explain why no fewer colors are possible.  What schedule of
recitations does this imply?

\begin{solution}
Four colors are necessary and sufficient.   Four
are \emph{sufficient} as shown by the coloring:

\iffalse
\begin{figure}[h]
\graphic[height=2in]{ps3-schedule-colored}
\end{figure}
\fi

\begin{itemize}
\item red: R1, R3

\item white: R2, R6

\item blue: R4

\item green: R5
\end{itemize}
This corresponds having four recitation times, where recitations
assigned the same color run at the same time.  Other schedules are
also possible.

Four colors are \emph{necessary} because each of the four vertices R1,
R2, R4, and R5 are adjacent to other three, so they must all have
different colors.

Another way to say this is that the subgraph with these vertices is
isomorphic to $K_4$, which we know requires four colors.
\end{solution}

\eparts

\end{problem}

%%%%%%%%%%%%%%%%%%%%%%%%%%%%%%%%%%%%%%%%%%%%%%%%%%%%%%%%%%%%%%%%%%%%%
% Problem ends here
%%%%%%%%%%%%%%%%%%%%%%%%%%%%%%%%%%%%%%%%%%%%%%%%%%%%%%%%%%%%%%%%%%%%%

\endinput
