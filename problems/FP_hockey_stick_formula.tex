\documentclass[problem]{mcs}

\begin{pcomments}
  \pcomment{FP_hockey_stick_formula}
  \pcomment{CH, Spring '14}
\end{pcomments}

\pkeywords{
	binomial coefficients
        Pascal's triangle
        algebra
        induction
}

%%%%%%%%%%%%%%%%%%%%%%%%%%%%%%%%%%%%%%%%%%%%%%%%%%%%%%%%%%%%%%%%%%%%%
% Problem starts here
%%%%%%%%%%%%%%%%%%%%%%%%%%%%%%%%%%%%%%%%%%%%%%%%%%%%%%%%%%%%%%%%%%%%%

\begin{problem}

In this problem, we prove a couple of interesting identities involving
binomial coefficients.

\bparts 

\ppart  \textbf{Using algebra}, prove the following statement (known as \emph{Pascal's Triangle
  Identity}): for all positive integers $n , k$ such
that $k \leq n$,
\begin{equation}
\binom{n-1}{k} + \binom{n-1}{k-1} = \binom{n}{k} .
\label{eq:pascalid}
\end{equation}

\begin{solution}
\begin{align*}
\binom{n-1}{k} + \binom{n-1}{k-1} &= \frac{(n-1)!}{(n-1-k)! k!} +
\frac{(n-1)!}{(n-k)! (k-1)!} \\
&= \frac{n-k}{n} \cdot \frac{n!}{(n-k)! k!} + \frac{k}{n} \cdot
\frac{n!}{(n-k)! k!} \\
&= \frac{n-k}{n} \binom{n}{k} + \frac{k}{n} \binom{n}{k} \\ 
&= \binom{n}{k} .
\end{align*}
\end{solution}

\examspace[2.5in]

\ppart \textbf{Using induction}, prove the following statement (known as the \emph{Hockey-Stick
  Identity}): for all positive integers $n \geq 2$, 
\begin{equation}
\binom{2}{2} + \binom{3}{2} + \ldots + \binom{n}{2} = \binom{n+1}{3}.
\label{eq:hockeystick}
\end{equation}

\hint Invoke Equation \eqref{eq:pascalid} .

\begin{solution}

Let $P(n)$ be the statement \eqref{eq:hockeystick}. 

\inductioncase{Base case}: ($n=2$). It is easy to see that
$\binom{2}{2} = 1 = \binom{3}{3}$. Therefore, $P(2)$ is true.

\inductioncase{Induction step}: Suppose that $P(n)$ is true for some
positive integer $m \geq 2$, i.e.,
\[
\binom{2}{2} + \binom{3}{2} + \ldots + \binom{m}{2} = \binom{m+1}{3}.
\]
Adding $\binom{m+1}{2}$ to both sides, we get:
\[
\binom{2}{2} + \binom{3}{2} + \ldots + \binom{m}{2} + \binom{m+1}{2} =
\binom{m+1}{3} + \binom{m+1}{2}.
\]
But if we substitute $k=3$ and $n=m+2$ in \eqref{eq:pascalid}, we have:
\[
\binom{m+1}{3} + \binom{m+1}{2} = \binom{m+2}{3}.
\]
Combining, we get
\[
\binom{2}{2} + \binom{3}{2} + \ldots + \binom{m}{2} + \binom{m+1}{2} =
\binom{m+2}{3},
\]
and therefore $P(m+1)$ is true. By induction, $P(n)$ is true for all
$n \geq 2$.

\end{solution}

\eparts

\end{problem}

%%%%%%%%%%%%%%%%%%%%%%%%%%%%%%%%%%%%%%%%%%%%%%%%%%%%%%%%%%%%%%%%%%%%%
% Problem ends here
%%%%%%%%%%%%%%%%%%%%%%%%%%%%%%%%%%%%%%%%%%%%%%%%%%%%%%%%%%%%%%%%%%%%%

\endinput
