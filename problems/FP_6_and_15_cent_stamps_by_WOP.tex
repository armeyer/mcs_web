\documentclass[problem]{mcs}

\begin{pcomments}
  \pcomment{FP_6_and_15_cent_stamps_by WOP}
  \pcomment{from: S09.cp4t, S08.cp4m}
  \pcomment{full version from TP_10_and_15_cent_stamps_by_WOP}
\end{pcomments}

\pkeywords{
  well_ordering
  WOP
  postage_stamps
  divides
}

%%%%%%%%%%%%%%%%%%%%%%%%%%%%%%%%%%%%%%%%%%%%%%%%%%%%%%%%%%%%%%%%%%%%%
% Problems start here
%%%%%%%%%%%%%%%%%%%%%%%%%%%%%%%%%%%%%%%%%%%%%%%%%%%%%%%%%%%%%%%%%%%%%

\begin{problem}
Say a number of cents is \emph{makeable} if it is the value of some
set of 6 cent and 15 cent stamps.  Use the Well Ordering Principle to
show that every integer $\geq 12$ that is a multiple of 3 is makeable.

\begin{solution}

\begin{proof}
  Let 
\[
C \eqdef \set{ n\geq 12 \suchthat n \text{ is a multiple of 3 and is
    not makeable}}
\]
 be the set of counter examples to the makeability claim.  Assume
 for the sake of contradiction that $C$ is not empty.  Then by the
 Well Ordering Principle, $C$ must have some minimum element $m\in C$.

  First, observe that 12 is makeable using two 6\textcent\ stamps.
  The next multiple of 3 is 15, which is makeable using a single
  stamp.  So $m$ is not 12 or 15, and since it is a multiple of 3, it
  must be as large as next multiple of 3 after 15, namely, $m \geq
  18$.  This implies that $m-6$ is multiple of 3 that is $\geq 12$ and
  less than $m$.  Since $m$ is a minimum counterexample, $m-6$ must be
  makeable.  But then by adding a 6\textcent\ stamp to the stamps that
  make up $m-6$, we get a set of stamps that make $m$\textcent.  This
  contradicts the fact that $m$ is not makeable.

  This contradictin implies that $C$ must be empty.  That is, all
  $n\geq 12$ are multiples of 3 are makeable.
\end{proof}

\end{solution}

\end{problem}

%%%%%%%%%%%%%%%%%%%%%%%%%%%%%%%%%%%%%%%%%%%%%%%%%%%%%%%%%%%%%%%%%%%%%
% Problems end here
%%%%%%%%%%%%%%%%%%%%%%%%%%%%%%%%%%%%%%%%%%%%%%%%%%%%%%%%%%%%%%%%%%%%%

\endinput
