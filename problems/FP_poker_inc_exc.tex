\documentclass[problem]{mcs}

\begin{pcomments}
  \pcomment{FP_poker_inc_exc}
  \pcomment{by Emanuele Ceccarelli 4/20/16}
\end{pcomments}

\pkeywords{
 inclusion-exclusion
 counting
 poker
}

%%%%%%%%%%%%%%%%%%%%%%%%%%%%%%%%%%%%%%%%%%%%%%%%%%%%%%%%%%%%%%%%%%%%%
% Problem starts here
%%%%%%%%%%%%%%%%%%%%%%%%%%%%%%%%%%%%%%%%%%%%%%%%%%%%%%%%%%%%%%%%%%%%%

\begin{problem}
A poker hand is a set of five cards drawn from a 52 card deck.  Each
card has a suit and a rank; there are four suits: spades, hearts,
diamonds, clubs, and there are 13 ranks, A,2,3,10,J,Q,K.

\bparts

\ppart How many poker hands are there with exactly 2 aces? \hfill \examrule

\examspace[1.0in]

\begin{solution}
\[
\binom{4}{2} \cdot \binom{48}{3}.
\]

There are $\binom{4}{2}$ ways to choose the two aces, and $\binom{48}{3}$ way to choose the remaining 3 cards.
\end{solution}

\ppart How many poker hands are there with at least 2 aces? \hfill \examrule

\examspace[1.0in]

\begin{solution}
\[
\binom{52}{5} - \binom{48}{5} - 4 \binom{48}{4}.
\]

The total number of hands is $\binom{52}{5} $.  There are
$\binom{48}{5}$ hands with no aces and $4 \binom{48}{4}$ hands with
exactly one ace.
\end{solution}

\ppart How many poker hands contain the ace of spades, or the ace of
clubs, or both? \hfill \examrule

\begin{solution}
\[
2\binom{51}{4} - \binom{50}{3},
\]
by inclusion-exclusion, because there are $\binom{51}{4}$ hands with
the ace of spades, likewise $\binom{51}{4}$ hands with the ace of
clubs, and $\binom{50}{3}$ hands with both aces.

Alternatively,
\[
\binom{52}{5} - \binom{50}{5},
\]
because there are $\binom{50}{5}$ hands that contain neither of the
aces.
\end{solution}
\eparts

\end{problem}

%%%%%%%%%%%%%%%%%%%%%%%%%%%%%%%%%%%%%%%%%%%%%%%%%%%%%%%%%%%%%%%%%%%%%
% Problem ends here
%%%%%%%%%%%%%%%%%%%%%%%%%%%%%%%%%%%%%%%%%%%%%%%%%%%%%%%%%%%%%%%%%%%%%

\endinput
