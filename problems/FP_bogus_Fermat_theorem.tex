\documentclass[problem]{mcs}

\begin{pcomments}
  \pcomment{FP_bogus_Fermat_theorem}
  \pcomment{renamed from FP_Fermat_theorem_bogus}
  \pcomment{from S09.final.prob6 by ARM 1/31/12}
  \pcomment{formatted ARM 3/13/12}
\end{pcomments}

\pkeywords{
  modular_arithmetic
  Euler_function
  Euler_theorem
  Fermat_theorem
  remainder
  bogus_proof
}

%%%%%%%%%%%%%%%%%%%%%%%%%%%%%%%%%%%%%%%%%%%%%%%%%%%%%%%%%%%%%%%%%%%%%
% Problem starts here
%%%%%%%%%%%%%%%%%%%%%%%%%%%%%%%%%%%%%%%%%%%%%%%%%%%%%%%%%%%%%%%%%%%%%

\begin{problem}

\bparts

\ppart\label{findrem} Calculate the remainder of $35^{86}$ divided by 29.

\examspace[3.0in]

\begin{solution}
Since 29 is prime and 35 is not a multiple of 29, Fermat's Little
Theorem implies that
\[
35^{28} \equiv 1 \pmod{29}.
\]
But $35 \equiv 6 \pmod{29}$ and $86 = 3 \cdot 28 + 2$, so
\[
35^{86} = 35^{3 \cdot 28  + 2} = \paren{35^{28}}^3 \cdot 35^2
\equiv 1^{3} \cdot 6^2 \equiv 7 \pmod{29}.
\]
  Therefore, $\rem{35^{86}}{29} = 7$.

\end{solution}

 \ppart Part~\eqref{findrem} implies that the remainder of $35^{86}$
 divided by 29 is not equal to 1.  So there there must be a mistake in
 the following proof, where all the congruences are taken with modulus
 29:
  \begin{align}
   1 & \not \equiv 35^{86}
       & \text{(by part~\eqref{findrem})} \\ 
    & \equiv 6^{86} 
       & \text{(since $35 \equiv 6 \pmod {29} $)} \\
    & \equiv 6^{28} 
       & \text{(since $86 \equiv 28 \pmod{29} $)}\label{bug} \\
    & \equiv 1 
       & \text{(by Fermat's Little Theorem)}
  \end{align}

  Identify the exact line containing the mistake and explain the 
  logical error.

 \examspace[2in]

\begin{solution}
The mistake occurs at line~\eqref{bug}.

Exponents can be replaced by their remainders on division by $\phi(29)
= 28$, not on division by 29.  So the ``explanation'' that $86 \equiv
28 \pmod {29}$ on the third line is true, but does not justify that
mistaken step.
\end{solution}

\eparts
\end{problem}

\endinput

