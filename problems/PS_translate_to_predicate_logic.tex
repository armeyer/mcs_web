\documentclass[problem]{mcs}

\begin{pcomments}
  \pcomment{PS_translate_to_predicate_logic}
  \pcomment{from: F09.ps2}
  \pcomment{from: S03.ps4}
\end{pcomments}

\pkeywords{
  logic
  predicate_calculus
  domain_of_discourse
  translating_english_statements
}

%%%%%%%%%%%%%%%%%%%%%%%%%%%%%%%%%%%%%%%%%%%%%%%%%%%%%%%%%%%%%%%%%%%%%
% Problem starts here
%%%%%%%%%%%%%%%%%%%%%%%%%%%%%%%%%%%%%%%%%%%%%%%%%%%%%%%%%%%%%%%%%%%%%

\begin{problem}
Translate the following statements into predicate logic.  For
each, specify the domain of discourse.  In addition to logic symbols,
you may build predicates using arithmetic, relational symbols and
constants.  For example, the statement ``$n$ is an odd number'' could
be translated as $\exists m (2m+1 = n)$ where the domain of discourse
is $\mathbb{Z}$, the set of integers.

\bparts
\ppart 
(Lagrange's Four-Square Theorem) Every nonnegative integer is
expressible as the sum of four perfect squares.

\begin{solution}
The domain of discourse is $\mathbb{N}$.

\[
\forall n \exists w \exists x \exists y \exists z (n = w^2 + x^2 + y^2 + z^2)
\]

\end{solution}

\ppart
(Goldbach Conjecture) Every even integer greater than two
is the sum of two primes.

\begin{solution}
The domain of discourse is $\mathbb{N}$.

Denote $prime(p)$ as 
\[
(p > 1)
\QAND\
\QNOT \left( \exists m \exists n (m > 1 \QAND\ n > 1 \QAND\ mn = p) \right)
\]

the statement could be translated as 

\[
\forall n 
\left(
((n > 2) \QAND\ \exists m (n = 2m)) \QIMP\ \exists p \exists q (prime(p) \QAND\ prime(q) \QAND\ (n = p + q))
\right)
\]
\end{solution}

\ppart
The function $f : \mathbb{R} \mapsto \mathbb{R}$ is
continuous.

\begin{solution}
The domain of discourse is $\mathbb{R}$

\[
\forall a \forall x \exists b \forall y
\left(
(a > 0 \QAND\ b > 0 \QAND\ \abs{x-y} < b) \QIMP\ \abs{f(x) - f(y)} < a
\right)
\]
\end{solution}

\ppart
(Fermat's Last Theorem) There are no nontrivial solutions
to the equation:

\[
x^n + y^n = z^n
\]

over the nonnegative integers  when $n > 2$.

\begin{solution}
The domain of discourse is $\mathbb{N}$.

\[
\forall x \forall y \forall z \forall n
\left(
(x > 0 \QAND\ y > 0 \QAND\ z > 0 \QAND\ n > 2)
\QIMP\
\QNOT (x^n + y^n = z^n)
\right)
\]
\end{solution}

\ppart
There is no largest prime number.

\begin{solution}
The domain of discourse is $\mathbb{Z}$.

\[
\QNOT \left(\exists p (Prime(p) \QAND\ (\forall q (Prime(q) \QIMP\ p \geq q)))
\right)
\]
\end{solution}

\ppart
(Bertrand's Postulate) If $n > 1$, then there is always
at least one prime $p$ such that $n < p < 2n$.

\begin{solution}
The domain of discourse is $\mathbb{Z}$.
\[
\forall n
\left( (n > 1) \QIMP\ (\exists p ( Prime(p)  \QAND\ (n < p) \QAND\ (p < 2n))) 
\right)
\]
\end{solution}

\eparts
\end{problem}

%%%%%%%%%%%%%%%%%%%%%%%%%%%%%%%%%%%%%%%%%%%%%%%%%%%%%%%%%%%%%%%%%%%%%
% Problem ends here
%%%%%%%%%%%%%%%%%%%%%%%%%%%%%%%%%%%%%%%%%%%%%%%%%%%%%%%%%%%%%%%%%%%%%

\endinput
