\documentclass[problem]{mcs}

\begin{pcomments}
  \pcomment{PS_gen_fcns_pennies_nickels_etc}
  \pcomment{from: S08.ps8}
\end{pcomments}

\pkeywords{
  generating functions
}

%%%%%%%%%%%%%%%%%%%%%%%%%%%%%%%%%%%%%%%%%%%%%%%%%%%%%%%%%%%%%%%%%%%%%
% Problem starts here
%%%%%%%%%%%%%%%%%%%%%%%%%%%%%%%%%%%%%%%%%%%%%%%%%%%%%%%%%%%%%%%%%%%%%

\begin{problem}
  We will use generating functions to determine how many ways there
  are to use pennies, nickels, dimes, quarters, and half-dollars to
  give $n$ cents change.

  \bparts
  \ppart Write the sequence $P_n$ for the number of ways to use
  only pennies to change $n$ cents.  Write the generating function for
  that sequence.

  \begin{solution} Since there is only one way to change any given amount
    with only pennies the sequence is $P_n = <1, 1, 1, 1, \ldots>$.
    The generating function for the sequence is
\[
P(x) \eqdef 1 + x + x^2 + x^3 + \cdots = \frac{1}{1-x}.
\]
\end{solution}

  \ppart Write the sequence $N_n$ for the number of ways to use only
  nickels to change $n$ cents.  Write the generating function for that
  sequence.

  \begin{solution} There is no way to change amounts that are not multiples
    of five with only nickels.  There is exactly one way to change
    amounts that are multiples of five with only nickels.  So the
    sequence is $N_n = <1, 0, 0, 0, 0, 1, 0, 0, 0, 0, 1, \ldots>$.
    The generating function for the sequence i
\[
N(x) \eqdef 1 + x^5 + x^{10} + x^{15} + \cdots = \frac{1}{1-x^5}.
\]
\end{solution}

  \ppart Write the generating function for the number of ways to
  use only nickels and pennies to change $n$ cents.

  \begin{solution} Since P and N are generating functions for the number of ways
    to choose pennies and nickels separately, we can apply the Convolution
    Rule from Notes 10 and simply multiply the two generating functions
    from the previous parts to find a generating function for the number
    of way to choose pennies and nickels together, namely, the generating
    function is
\[
N(x) \cdot P(x) = \frac{1}{(1-x)(1-x^5)}.
\]
\end{solution}

  \ppart Write the generating function for the number of ways to use
  pennies, nickels, dimes, quarters, and half-dollars to give $n$
  cents change.

  \begin{solution} Generalizing our method gives;
\[
C(x) \eqdef \frac{1}{(1-x)(1-x^5)(1-x^{10})(1-x^{25})(1-x^{50})}.
\]

%=(1-x) (1-x^5)^2 (1+x^5) (1-x^{25})^2 (1+x^{25})
\end{solution}

  \ppart Explain how to use this function to find out how many ways are
  there to change 50 cents; you do \emph{not} have to provide the answer
  or actually carry out the process.

  \begin{solution}
    The answer is the coefficient to $x^{50}$ of the power series for $C$.
    (It happens to be 50.)  This could be extracted by taking the 50th
    derivative of $C$ or using the partial fraction method to obtain a
    system of 50+25+10+1 linear equations in as many variables and then
    solving for the variables.  Neither of these approaches would be
    appropriate for hand calculation.
  \end{solution}

  \eparts
\end{problem}
%%%%%%%%%%%%%%%%%%%%%%%%%%%%%%%%%%%%%%%%%%%%%%%%%%%%%%%%%%%%%%%%%%%%%
% Problem ends here
%%%%%%%%%%%%%%%%%%%%%%%%%%%%%%%%%%%%%%%%%%%%%%%%%%%%%%%%%%%%%%%%%%%%%

\endinput
