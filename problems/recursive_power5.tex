\documentclass[problem]{mcs}

\begin{pcomments}
  \pcomment{recursive_power5}
  \pcomment{similar to MQ_RM_subs_M, PS_RM_equal_AM, MQ_ambiguous_recursive_def}
  \pcomment{ARM 9/27/13}
\end{pcomments}

\pkeywords{
  structural_induction
  induction
  ambiguous
}

%%%%%%%%%%%%%%%%%%%%%%%%%%%%%%%%%%%%%%%%%%%%%%%%%%%%%%%%%%%%%%%%%%%%%
% Problem starts here
%%%%%%%%%%%%%%%%%%%%%%%%%%%%%%%%%%%%%%%%%%%%%%%%%%%%%%%%%%%%%%%%%%%%%

\begin{problem}
Define the sets $F_1$ and $F_2$ recursively:
\begin{itemize}

\item $F_1$:
\begin{itemize}
\item $5 \in F_1$,
\item if $n \in F_1$, then $5n \in F_1$.
\end{itemize}

\item $F_2$:
\begin{itemize}
\item $5 \in F_2$,
\item if $n,m \in F_1$, then $nm \in F_2$.
\end{itemize}

\end{itemize}

\bparts

\ppart Show that one of these definitions is technically
\emph{ambiguous}.  (Remember that ``ambiguous recursive definition''
has a technical mathematical meaning which does not imply that the
ambiguous definition is unclear.)

\ppart Briefly explain what advantage unambiguous recursive
definitions have over ambiguous ones.

\examspace[1in]

\begin{solution}
  If a definition is ambiguous, functions defined recursively on it
  may not be well-defined.
\end{solution}

\bpart A way to prove that $F_1=F_2$, is to show firat that $F_1
\subseteq F_2$ and second that $F_2 \subseteq F_1$.  One of these
containments follows easily by structural induction.  Which one?  What
would be the induction hypothesis?  (You do not need to complete a
proof.)

\examspace[1in]

\eparts

\end{problem}

%%%%%%%%%%%%%%%%%%%%%%%%%%%%%%%%%%%%%%%%%%%%%%%%%%%%%%%%%%%%%%%%%%%%%
% Problem ends here
%%%%%%%%%%%%%%%%%%%%%%%%%%%%%%%%%%%%%%%%%%%%%%%%%%%%%%%%%%%%%%%%%%%%%

\endinput











