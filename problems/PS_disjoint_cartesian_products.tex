\documentclass[problem]{mcs}

\begin{pcomments}
  \pcomment{from: S00.ps3}
\end{pcomments}

\pkeywords{
  cartesian product
  disjoint
  sets
}

%%%%%%%%%%%%%%%%%%%%%%%%%%%%%%%%%%%%%%%%%%%%%%%%%%%%%%%%%%%%%%%%%%%%%
% Problem starts here
%%%%%%%%%%%%%%%%%%%%%%%%%%%%%%%%%%%%%%%%%%%%%%%%%%%%%%%%%%%%%%%%%%%%%

\begin{problem}
% topic: Cartesian products, disjoint, direct proof
% source: Velleman 4.1.8

Prove that for any sets $A$, $B$, $C$, and $D$, if 
$A \times B$ and $C \times D$ 
are disjoint, then either $A$ and $C$ are disjoint or $B$ and $D$ are disjoint.

\solution{
\begin{proof}
We will prove the contrapositive.  In other words, we will
assume
\begin{equation}\label{neitherempty}
[(A \intersect C) \neq \emptyset \QAND (B \intersect D) \neq \emptyset]
\end{equation}
and prove that
\begin{equation}\label{prodnonempty}
(A \times B) \intersect (C \times D) \neq \emptyset.
\end{equation}

Now by~\ref{neitherempty}, there must be an element $e \in A \QAND e
\in C$, as well as an element $f \in B \QAND f \in D$.  So, $(e, f)
\in A \times B$ by definition of Cartesian product, and likewise $(e,
f) \in C \times D$.  This means that
\[
(e,f) \in (A \times B) \intersect (C \times D),
\]
so 
$(A \times B) \intersect (C \times D) \neq \emptyset$

\end{proof}}
\end{problem}


%%%%%%%%%%%%%%%%%%%%%%%%%%%%%%%%%%%%%%%%%%%%%%%%%%%%%%%%%%%%%%%%%%%%%
% Problem ends here
%%%%%%%%%%%%%%%%%%%%%%%%%%%%%%%%%%%%%%%%%%%%%%%%%%%%%%%%%%%%%%%%%%%%%
