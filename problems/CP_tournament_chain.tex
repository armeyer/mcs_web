\documentclass[problem]{mcs}

\begin{pcomments}
  \pcomment{CP_tournament_chain}
  \pcomment{from ps4.S10}
  \pcomment{substantially revised by ARM 3/8/11}
  \pcomment{by ARM 10/17/09 but I saw it somewhere}
\end{pcomments}

\pkeywords{
  digraphs
  relations
  ranking
  tournament
  rationals
  path
}

%%%%%%%%%%%%%%%%%%%%%%%%%%%%%%%%%%%%%%%%%%%%%%%%%%%%%%%%%%%%%%%%%%%%%
% Problem starts here
%%%%%%%%%%%%%%%%%%%%%%%%%%%%%%%%%%%%%%%%%%%%%%%%%%%%%%%%%%%%%%%%%%%%%

\begin{problem} %\textbf{Tournament Graphs}

%  \newcommand{\beats}{\!\rightarrow\!}

  In a round-robin tournament, every two distinct players play against
  each other just once.  For a round-robin tournament with with no
  tied games, a record of who beat whom can be described with a
  \term{tournament digraph}, where the vertices correspond to players
  and there is an edge $\diredge{x}{y}$ iff $x$ beat $y$ in their
  game.

  A \term*{ranking} is a path that includes all the players.

\begin{staffnotes}
 Paths used to be called ``simple paths'' or ``\emph{directed} simple
 paths.''
\end{staffnotes}

\bparts

\ppart Give an example of a tournament digraph with more than one
ranking.

\begin{solution}
  Let $n=2$ with edges $\diredge{u}{v}$ and $\diredge{v}{u}$.  Then
  both $u,v$ and $v,u$ are rankings.
\end{solution}

\ppart Prove that if a tournament digraph is a DAG, then it has at
most one ranking.  \hint Prove that the elements below $u$ in any
ranking are uniquely determined.

\begin{solution}
If there is a (positive length) path from $u$ to $v$, there cannot be
one back in a DAG, so $v$ must be below $u$ in any ranking.
Conversely, it $v$ is (strictly) below $u$ in some ranking, then by
definition there is a positive length path from $u$ to $v$.  So the
elements below $u$ in any ranking are uniquely determined to be
\[
\set{v \suchthat \text{there is a positive length path from $u$ to
    $v$}} = \set{v \suchthat u \mrel{G^+} v} \eqdef G^+(\set{v}).
\]

\begin{staffnotes}
  Alternative explanation after \emph{total partial orders} are
  defined:

  If it's a DAG, then its path relation defines a partial order, and
  since every pair of elements are comparable, this is in fact a total
  order.  So there's no alternative but to rank the elements from
  smallest to largest.
\end{staffnotes}

\end{solution}

\ppart\label{hasrank} Prove that every finite tournament digraph has a
ranking.

\begin{staffnotes}
\hint Induction on the size of the tournament.  Could also rephrase
the proof below by considereing a maximum length ranking (there is one
by WOP applied to $\#\text{players} - \text{ranking length}$).
\end{staffnotes}

\begin{solution}
By induction on $n$ with induction hypothesis
\[
P(n) \eqdef \text{every tournament digraph with $n$ vertices has a
  ranking.}
\]
 
\textbf{base case} $n=1$:  Trivial.

\textbf{inductive step}: Let $G$ be a tournament digraph with $n+1$
vertices.  Remove one vertex, $v$, to obtain the subgraph, $H$, with the
$n$ remaining vertices.  Clearly, $H$ is also a tournament digraph, so by
induction hypothesis it has a ranking.  Now if the last player in this
$H$-ranking beat player $v$, then $v$ can be added at the end to form a
ranking in $G$.  On the other hand, if $v$ beat the last player in the
$H$-ranking, then there will (by WOP) be a first player in the $H$-ranking
that $v$ beats.  Inserting $v$ just before that first player gives a
ranking for $G$.  Since $G$ was an arbitrary $n+1$ vertex tournament
graph, we conclude that $P(n+1)$ holds, which completes the proof.
\end{solution}

\ppart Give an example of a tournament with a countably infinite number of
players,$p_0,p_1,\dots$ that has no ranking.

\hint $\rationals$.

\begin{staffnotes}
Intervene if teams start to spend much time on this part.  Tell them
we mostly will focus on finite graphs, and in general, infinite
counter-examples like this are not particularly important.

On the other hand, to really understand a proof, you need to be aware
where the proof depends on finiteness (whether it needs to or not).
\end{staffnotes}

\begin{solution}
The rationals, $\rationals$, are a countable set, and specifying that
$r$ beats $s$ precisely when $r > s$ defines a tournament graph with
$\rationals$ as the set of players.

Now in any tournament graph, vertex $u$ can come before vertex $u$ in
some ranking only if there is a path from $u$ to $v$.  This implies
that if $r > s$, then $r$ must come before $s$ in \emph{any} ranking
of $\rationals$.

So suppose there was a ranking of $\rationals$ and $\diredge{r}{s}$
was an edge on the path.  This implies that $r >s$.  Now let $t$ be
any rational such that $r>t>s$.  Now in a ranking, $t$ must come
before $r$ or after $s$, which implies $t >r$ or $s>t$, a
contradicting the choice of $t$.  SO there cannot be a ranking of the 
$\rationals$ tournament.
\end{solution}

 \eparts
\end{problem}

%%%%%%%%%%%%%%%%%%%%%%%%%%%%%%%%%%%%%%%%%%%%%%%%%%%%%%%%%%%%%%%%%%%%%
% Problem ends here
%%%%%%%%%%%%%%%%%%%%%%%%%%%%%%%%%%%%%%%%%%%%%%%%%%%%%%%%%%%%%%%%%%%%%

\endinput
