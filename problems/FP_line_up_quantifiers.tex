\documentclass[problem]{mcs}

\begin{pcomments}
  \pcomment{FP_line_up_quantifiers}
  \pcomment{by ARM 12/23/11 from FP_}
\end{pcomments}

\pkeywords{
 predicate_calculus
 quantifier
}

%%%%%%%%%%%%%%%%%%%%%%%%%%%%%%%%%%%%%%%%%%%%%%%%%%%%%%%%%%%%%%%%%%%%%
% Problem starts here
%%%%%%%%%%%%%%%%%%%%%%%%%%%%%%%%%%%%%%%%%%%%%%%%%%%%%%%%%%%%%%%%%%%%%

\begin{problem}
Some (but not necessarily all) students from a large class will be lined up left to right.
There will be at least two students in the line.  Translate each of
the following assertions into predicate formulas, using quantifiers $\exists$ and $\forall$ and any logical operators like $\QAND$, $\QOR$, $\QNOT$, etc., with the set of
students in the class as the domain of discourse.  The only predicates
you may use about students are
\begin{itemize}

\item equality, and

\item $F(x,y)$, meaning that ``$x$ is somewhere to the left of $y$ in
  the line.''  For example, in the line ``cda'', both $F(c,a)$ and
  $F(c,d)$ are true. (A student is not considered to be ``to the left of'' themself, so $F(x,x)$ is always false.)

\end{itemize}
%
As a worked example, the predicate ``there are at least two students after $x$'' may be written as
\begin{equation*}
  \exists y\, \exists z.\, \QNOT(y=z) \QAND (F(x,y) \QAND F(x,z)).
\end{equation*}

In your answers you may use earlier predicates in later parts, even if you did not
solve the earlier parts.  So your answer to
part~\eqref{imright} might refer to $\text{inline}(x)$ and/or
$\text{first}(x)$, for example.


\bparts

\ppart Student $x$ is in the line.  Call this predicate
$\text{inline}(x)$.

\hint Since there are at least $2$ students in the line, $\text{inline}(x)$ means $x$ is to the left or right of anyone else.

\inhandout{
\begin{center}
\exambox{3in}{0.5in}{0.3in}
\end{center}
}

\begin{solution}
  \[\text{inline}(x) \eqdef \exists y.\, F(y,x) \QOR F(x,y) \]
\end{solution}

\ppart Student $x$ is first in line.  Call this predicate
$\text{first}(x)$.


\inhandout{\begin{center}
\exambox{3in}{0.5in}{0.3in}
\end{center}
}

\begin{solution}
\[\text{first}(x) \eqdef \text{inline}(x) \QAND \QNOT(\exists y.\, F(y,x)) \]
% \[\text{first}(x) \eqdef \text{inline}(x) \QAND \forall y.\, \QNOT F(y,x) \]
\end{solution}


\ppart\label{imright} Student $x$ is immediately to the right of student $y$.  Call
this predicate $\text{isnext}(x,y)$.

\hint No other student can be between $x$ and $y$.

\inhandout{
\begin{center}
\exambox{3in}{0.5in}{0.3in}
\end{center}
}

\begin{solution}
  \[\text{isnext}(x,y) \eqdef F(y,x) \QAND \QNOT(\exists z.\, F(y,z) \QAND F(z,x))
  \]

  % \[\text{isnext}(x,y) \eqdef F(y,x) \QAND \forall z.\, \QNOT(F(y,z) \QAND F(z,x))
  % \]
\end{solution}

\ppart
Student $x$ is second in line.

\inhandout{
\begin{center}
\exambox{3in}{0.5in}{0.3in}
\end{center}
}

\begin{solution}
  \[
 \exists y. \text{first}(y) \QAND \text{isnext}(x,y)
 \]
\end{solution}

\eparts
\end{problem}

\endinput
