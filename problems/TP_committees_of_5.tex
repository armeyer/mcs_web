\documentclass[problem]{mcs}

\begin{pcomments}
  \pcomment{TP_committees_of_5}
  \pcomment{Source: f01-ps9-3 and s01-ps7-5}
  \pcomment{from subsumed PS_counting_problems}
\end{pcomments}

\pkeywords{
counting
binomial_coefficient
}

%%%%%%%%%%%%%%%%%%%%%%%%%%%%%%%%%%%%%%%%%%%%%%%%%%%%%%%%%%%%%%%%%%%%%
% Problem starts here
%%%%%%%%%%%%%%%%%%%%%%%%%%%%%%%%%%%%%%%%%%%%%%%%%%%%%%%%%%%%%%%%%%%%%

\begin{problem}
Six women and nine men are on the faculty of a school's
EECS department.  The individuals are distinguishable.
How many ways are there to select a committee of 5
members if at least 1 woman must be on the committee?

\begin{solution}
There are $\binom{15}{5}$ different committees, and $\binom{9}{5}$
committees of just men.  So there are $\binom{15}{5} - \binom{9}{5}=
2877$ different possibilities for committees.

\end{solution}

\end{problem}

%%%%%%%%%%%%%%%%%%%%%%%%%%%%%%%%%%%%%%%%%%%%%%%%%%%%%%%%%%%%%%%%%%%%%
% Problem ends here
%%%%%%%%%%%%%%%%%%%%%%%%%%%%%%%%%%%%%%%%%%%%%%%%%%%%%%%%%%%%%%%%%%%%%

\endinput
