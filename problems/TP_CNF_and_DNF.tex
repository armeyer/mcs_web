\documentclass[problem]{mcs}

\begin{pcomments}
    \pcomment{TP_CNF_and_DNF}
    \pcomment{by zabel 2/17/18}
\end{pcomments}

\pkeywords{
  Conjunctive Normal Form
  CNF
  Disjunctive Normal Form
  DNF
  Full CNF
}

\begin{problem}

\bparts

\ppart The five-variable propositional formula
\[
    P\eqdef (A\QAND B \QAND \bar{C} \QAND D \QAND \bar{E}) \QOR
    (\bar{A} \QAND B \QAND \bar{C} \QAND \bar{E})
\]
is in Disjunctive Normal Form with two ``$\QAND$-of-literal''
clauses.  Find a \textbf{Full}~Disjunctive Normal Form that is
equivalent to $P$, and explain your reasoning.

\hint You shouldn't need a truth table.
  
\begin{solution}
The formula $P$ is not a Full DNF because the second clause does not
mention variable $D$.  So let's put in the possibilities for $D$.
Using our equivalence axioms, $P$'s second clause can be equivalently
written as
\begin{align*}
\lefteqn{(\bar{A} \QAND B \QAND \bar{C} \QAND \bar{E})}\\
& \leftrightarrow (\bar{A} \QAND B \QAND \bar{C} \QAND \bar{E}) \QAND \True \\
& \leftrightarrow (\bar{A} \QAND B \QAND \bar{C} \QAND \bar{E}) \QAND (D \QOR \bar{D})\\
& \leftrightarrow (\bar{A} \QAND B \QAND \bar{C} \QAND \mathbf{D} \QAND \bar{E}) \QOR
                  (\bar{A} \QAND B \QAND \bar{C} \QAND \mathbf{\bar{D}} \QAND \bar{E}),
\end{align*}
where the last line uses the Distributive rule.\footnote{It also uses
  the Associative and Commutative rules for \QAND, but we take these
  ``formatting'' rules for granted.}  So a full DNF is
\begin{align*}
P &\leftrightarrow (A\QAND B \QAND \bar{C} \QAND D \QAND \bar{E}) \QOR {}\\
  &\mathbin{\phantom{\leftrightarrow}}
                   (\bar{A} \QAND B \QAND \bar{C} \QAND D \QAND \bar{E})\QOR {}\\
  &\mathbin{\phantom{\leftrightarrow}}
                   (\bar{A} \QAND B \QAND \bar{C} \QAND \bar{D} \QAND \bar{E}).
  \end{align*}
Note that writing ``$P = \cdots$'' instead of ``$P \leftrightarrow
\cdots$'' would not quite be correct: $P$ and the Full DNF are
\emph{different} propositional formulas, even though they are
logically equivalent.
\end{solution}

\begin{staffnotes}
The problem asks for and provides \textbf{\emph{a}} Full DNF.  Full
DNF is not unique (canonical) until each of the $\QAND$-of-literal
clauses is alphabetized, and the clauses themselves are sorted in some
standard order without duplicates.  The Full DNF given above is
canonical because its three clauses are distinct, alphabetized, and
appear in dictionary order determined by having each variable
alphabetically precede its complemented form, that is, $V$ precedes
$\bar{V}$ for each variable $V$ so the dictionary order of the
literals above is $A, \bar{A}, B, \bar{B}, C, \bar{C},\dots$.
\end{staffnotes}

\ppart
How many ``$\QOR$-of-literal'' clauses are there in a
\textbf{Full}~\textbf{Con}junctive Normal Form that is equivalent to
$P$?  Briefly explain your answer. (Please \textbf{don't} write out
this Full CNF.)

\begin{solution}
  Answer: \textbf{$29$ clauses}.
    
As $P$ has five variables, $P$'s truth table has $2^5 = 32$ rows.  We've
shown that precisely $3$ rows evaluate to True, since these correspond
to the clauses of $P$'s Full DNF.  The remaining $2^5 - 3 = 29$ rows
correspond to the clauses of $P$'s Full CNF.
\end{solution}

\eparts

\end{problem}

\endinput
