\documentclass[problem]{mcs}

\begin{pcomments}
    \pcomment{Converted from ./00Convert/probs/practice8/prob4.scm
              by scmtotex and drewe
              on Tue 02 Aug 2011 06:32:56 PM EDT}
\end{pcomments}

\begin{problem}

%% type: short-answer
%% title: Counting Questions

Alice is thinking of a number between 1 and 1000. 

 What is
the least number of yes/no questions you could ask her and be guaranteed to discover what it is?
\bparts


\ppart


\begin{solution}

10


The best you can do with a yes/no question is to just halve the
space of remaining possibilities about what the number might be.  So,
starting with 1000 possibilities, the first question ( ``Is the number
greater than 500? ``) can lead you to only 500 possibilities, the second
question can lead you to only 250, etc.  The smallest \emph{n} such that $2^n \geq 1000$ is $n=10$, so you will need 10 questions
(in the worst case).
\end{solution}

\eparts


\end{problem}

\endinput
