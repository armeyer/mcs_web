\documentclass[problem]{mcs}

\begin{pcomments}
  \pcomment{TP_stable_wedding}
  \pcomment{excerpted from FP_stable_matching_unlucky}
\end{pcomments}

\pkeywords{
  stable_matching
  state_machines
  invariant
}

\providecommand{\boys}{\text{Boys}}
\providecommand{\girls}{\text{Girls}}

%%%%%%%%%%%%%%%%%%%%%%%%%%%%%%%%%%%%%%%%%%%%%%%%%%%%%%%%%%%%%%%%%%%%%
% Problem starts here
%%%%%%%%%%%%%%%%%%%%%%%%%%%%%%%%%%%%%%%%%%%%%%%%%%%%%%%%%%%%%%%%%%%%%

\begin{problem}
  In the Mating Ritual for stable marriages between an equal number of
  boys and girls, explain why there must be a girl to whom no boy
  proposes (serenades) until the last day.

\begin{solution}
Since there are an equal number of boys and girls, the wedding day is
when every girl is being serenaded by some (necessarily only one) boy.

So the day before the wedding day there must be some girl who is not
being serenaded.  But ``being serenaded'' is a preserved invariant, so
this girl must never have been serenaded on any earlier day than the
wedding day.
\end{solution}

\end{problem}


%%%%%%%%%%%%%%%%%%%%%%%%%%%%%%%%%%%%%%%%%%%%%%%%%%%%%%%%%%%%%%%%%%%%%
% Problem ends here
%%%%%%%%%%%%%%%%%%%%%%%%%%%%%%%%%%%%%%%%%%%%%%%%%%%%%%%%%%%%%%%%%%%%%

\endinput

