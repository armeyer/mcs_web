\documentclass[problem]{mcs}

\begin{pcomments}
  \pcomment{PS_walk_relation_composition}
  \pcomment{revised for walks by ARM, 3/10/11}
  \pcomment{from: digraph notes}
\end{pcomments}

\pkeywords{
  binary_relation
  composition
  walk
  walk_relation }

%%%%%%%%%%%%%%%%%%%%%%%%%%%%%%%%%%%%%%%%%%%%%%%%%%%%%%%%%%%%%%%%%%%%%
% Problem starts here
% %%%%%%%%%%%%%%%%%%%%%%%%%%%%%%%%%%%%%%%%%%%%%%%%%%%%%%%%%%%%%%%%%%%%

\begin{problem}
%Prove Lemma~\bref{lem:Rn-paths}, Namely,

Let $R$ be a binary relation on a set $A$ and $C^n$ be the composition
of $R$ with itself $n$ times for $n \ge 0$.  So $C^0 \eqdef
\ident{A}$, and $C^{n+1} \eqdef R \compose C^n $.  Regarding $R$ as a
digraph, let $R^n$ denote the length-$n$ walk relation in the digraph
$R$, that is,
\[
a \mrel{R^n} b \eqdef \mbox{there is a length-$n$ walk from $a$ to
    $b$ in $R$}.
\]
Prove that
\begin{equation}\label{RnRpnp}
R^n = C^n
\end{equation}
for all $n \in \naturals$.

\begin{solution}

\begin{proof}
By induction on $n$ with equation~\eqref{RnRpnp} as induction
hypothesis.

\textbf{Base case} $n=0$: $C^0 = \ident{A}$ by definition, and $R^0$
is the length-0 walk relation which also equals $\ident{A}$ by
definition.

\textbf{Inductive step:} Suppose~\eqref{RnRpnp} holds for some $n\ge 0$.
We want to prove it holds with ``$n$'' replaced by ``$n+1$.''

We begin by showing that
\begin{equation}\label{arn1bimp}
a \mrel{R^{n+1}} b \qimplies a \mrel{C^{n+1}} b.
\end{equation}
So suppose $a \mrel{R^{n+1}} b$, that is, there is a length-$(n+1)$
walk
\[
a=a_0\ \diredge{a_0}{a_1}\ a_1\ \diredge{a_1}{a_2} \dots \diredge{a_n}{a_{n+1}}\ a_{n+1} =b
\]
in $R$.  In particular, there is a length-$n$ walk from $a$ to some
vertex $a_n$ such that $a_n \mrel{R} b$.  By induction hypothesis, we
have that $a\mrel{C^n} a_n$.  Therefore,
\[
a \mrel{(C^n \compose R)} b
\]
by the definition\inbook{~(\bref{rel_compose_def})} of composition.
This proves~\eqref{arn1bimp}.

Conversely, we show that
\begin{equation}\label{acn1imp}
a \mrel{C^{n+1}} b \qimplies a \mrel{R^{n+1}} b.
\end{equation}
So suppose $a \mrel{C^{n+1}} b$, that is $a \mrel{(C^n \compose R)}
b$.  By definition of composition, there must be an $a_n \in A$ such
that
\[
a \mrel{C^n} a_n \QAND a_n \mrel{R} b.
\]
So there is a length-$n$ walk from $a$ to $a_n$ by induction
hypothesis, and $\diredge{a_n}{b} \in \edges{R}$.  Merging the walk
with this edge yields a length-$(n+1)$ walk from $a$ to $b$.  That is,
$a \mrel{R^{n+1}} b$.  This proves~\eqref{acn1imp}

Combining~\eqref{arn1bimp} and~\eqref{acn1imp} we conclude that
$R^{n+1} = C^{n+1}$, which is the exactly~\eqref{RnRpnp} with ``$n$''
replaced by ``$n+1$,'' completing the proof by induction.
\end{proof}

\end{solution}
\end{problem}

%%%%%%%%%%%%%%%%%%%%%%%%%%%%%%%%%%%%%%%%%%%%%%%%%%%%%%%%%%%%%%%%%%%%%
% Problem ends here
%%%%%%%%%%%%%%%%%%%%%%%%%%%%%%%%%%%%%%%%%%%%%%%%%%%%%%%%%%%%%%%%%%%%%

\endinput
