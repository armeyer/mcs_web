\documentclass[problem]{mcs}

\begin{pcomments}
  \pcomment{FP_asymptotics_define_functions}
  \pcomment{from: S09final.prob7}
  \pcomment{Adapted by Jodyann F09}
  \pcomment{partc from FP_multiple_choice adapted by Tom Brown}
\end{pcomments}

\pkeywords{
  asymptotics
  little_oh
  big_oh
  Theta
  asymptotically_equal
}

%%%%%%%%%%%%%%%%%%%%%%%%%%%%%%%%%%%%%%%%%%%%%%%%%%%%%%%%%%%%%%%%%%%%%
% Problem starts here
%%%%%%%%%%%%%%%%%%%%%%%%%%%%%%%%%%%%%%%%%%%%%%%%%%%%%%%%%%%%%%%%%%%%%


\begin{problem} %\textbf{Asymptotics}

\begin{problemparts} 

\problempart
Define a function $f(n)$ such that $f = \Theta(n^4)$ and $\QNOT (f \sim n^4)$.

$f(n) = $\examrule[1in]

\begin{solution}
$2n^4$
\end{solution}

\problempart
Define a function $g(n)$ such that $g = O(n^4)$, $g \neq \Theta(n^4)$ and $g \neq o(n^4)$.

$g(n) = $\examrule[3in]

\begin{solution}
  $n^4\sin\paren{\frac{n\pi}{2}} + n^4\cos\paren{\frac{n\pi}{2}}$ or any
  nonzero linear combination of the preceding terms.
\end{solution}

% FROM: Spring-07 MQ-2/28-1
%
% COMMENTS: Change to ask if they are weak/strict partial/total orders?
%           f=O(g), f=o(g), f~g, and subsets 


\ppart For each of the asymptotic relations below on the set of
nonnegative real-valued functions, indicate whether it is
\emph{transitive} but not a partial order (\textbf{Tr}), a \emph{total
order} (\textbf{Tot}), a \emph{strict partial order} that is not total
(\textbf{Str}), a \emph{weak partial order} that is not total (\textbf{Wk}),
or \emph{none} of the above (\textbf{Non}).

\begin{itemize}

\item the ``Asymptotically Equal'' relation, $f \sim g$. \hfill \examrule

\item  the ``Little Oh'' relation, $f=o(g)$, \hfill \examrule

\item  the ``Big Oh'' relation, $f=O(g)$, \hfill \examrule

\item The ``Theta'' relation, $f=\Theta(g)$, \hfill \examrule

\end{itemize}

\begin{solution}

\begin{itemize}
\item Asymptotically Equal is \textbf{Tr}.
\item Lttle Oh is \textbf{Str},
\item Big Oh is \textbf{Tr} (as it is not antisymmetric),
\item Theta is \textbf{Tr}.

\end{itemize}

\end{solution}

\end{problemparts}
\end{problem}


%%%%%%%%%%%%%%%%%%%%%%%%%%%%%%%%%%%%%%%%%%%%%%%%%%%%%%%%%%%%%%%%%%%%%
% Problem ends here
%%%%%%%%%%%%%%%%%%%%%%%%%%%%%%%%%%%%%%%%%%%%%%%%%%%%%%%%%%%%%%%%%%%%%

\endinput
